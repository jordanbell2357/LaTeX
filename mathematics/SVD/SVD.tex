\documentclass{article}
\usepackage{amsmath,amssymb,graphicx,subfig,mathrsfs,amsthm}
%\usepackage{tikz-cd}
%\usepackage{hyperref}
\newcommand{\innerL}[2]{\langle #1, #2 \rangle_{L^2}}
\newcommand{\inner}[2]{\left\langle #1, #2 \right\rangle}
\newcommand{\HSinner}[2]{\left\langle #1, #2 \right\rangle_{\ensuremath\mathrm{HS}}}
\newcommand{\tr}{\ensuremath\mathrm{tr}\,} 
\newcommand{\Span}{\ensuremath\mathrm{span}} 
\def\Re{\ensuremath{\mathrm{Re}}\,}
\def\Im{\ensuremath{\mathrm{Im}}\,}
\newcommand{\id}{\ensuremath\mathrm{id}} 
\newcommand{\rank}{\ensuremath\mathrm{rank\,}} 
\newcommand{\point}{\ensuremath\sigma_{\mathrm{point}}} 
\newcommand{\Hom}{\ensuremath\mathrm{Hom}}
\newcommand{\norm}[1]{\left\Vert #1 \right\Vert}
\newtheorem{theorem}{Theorem}
\newtheorem{lemma}[theorem]{Lemma}
\newtheorem{proposition}[theorem]{Proposition}
\newtheorem{corollary}[theorem]{Corollary}
\theoremstyle{definition}
\newtheorem{definition}[theorem]{Definition}
\begin{document}
\title{The singular value decomposition of compact operators on Hilbert spaces}
\author{Jordan Bell}
\date{April 3, 2014}

\maketitle

\section{Preliminaries}
The purpose of these notes is to present material about compact operators on Hilbert spaces that is special  to Hilbert spaces, rather than what applies to all Banach spaces. 
We use statements about compact operators on Banach spaces without proof.  For instance, any compact  operator
from one Banach space to another has separable image; the set of compact operators from one Banach space to another is a closed subspace of the set of all bounded
linear operators; a Banach space is reflexive if and only if the closed unit ball is weakly compact (Kakutani's theorem); etc. We do however state precisely each result that we are using for Banach spaces and show that  its hypotheses are satisfied.

Let
$\mathbb{N}$ be the set of positive integers. We say that a set is countable if it is bijective with a subset of
$\mathbb{N}$.
In this note I do not presume unless I say so that any set is countable or that any Hilbert space is separable. A neighborhood of a point in a topological space is a set that contains an open set that
contains the point; one reason why it can be handy to speak about neighborhoods of a point rather than just open sets that contain the point is that
the set of all neighborhoods of a point is a filter, whereas it is unlikely that the set of all open sets that contain a point is a filter. If $z \in \mathbb{C}$, we denote $z^* = \overline{z}$.

\section{Bounded linear operators}
An advantage of working with normed spaces rather than merely topological vector spaces is that continuous linear maps between normed spaces have a
simple characterization. If $X$ and $Y$ are normed spaces and $T:X \to Y$ is linear, the {\em operator norm} of $T$ is
\[
\norm{T} = \sup_{\norm{x} \leq 1} \norm{Tx}.
\]
If $\norm{T} < \infty$, then we say that $T$ is {\em bounded}. 

\begin{theorem}
If $X$ and $Y$ are normed spaces, a linear map $T:X \to Y$ is continuous if and only if it is bounded.
\end{theorem}
\begin{proof}
Suppose that $T$ is continuous. In particular $T$ is continuous at $0$, so there is some $\delta>0$ such that if $\norm{x} \leq \delta$ then
$\norm{Tx}=\norm{Tx-T0} \leq 1$. If $x \neq 0$ then, as $T$ is linear,
\[
\norm{Tx} = \frac{1}{\delta}\norm{x} \norm{T\left( \frac{\delta}{\norm{x}} x \right)}
\leq  \frac{1}{\delta} \norm{x}.
\]
Thus $\norm{T} \leq \frac{1}{\delta}<\infty$, so $T$ is bounded.

Suppose that $T$ is bounded. Let $x_0 \in X$, and let $\epsilon>0$. If $\norm{x-x_0} \leq \frac{\epsilon}{\norm{T}}$,
then
\[
\norm{Tx-Tx_0} = \norm{T(x-x_0)} \leq \norm{T} \norm{x-x_0} \leq \norm{T} \cdot  \frac{\epsilon}{\norm{T}} = \epsilon.
\]
Hence $T$ is continuous at $x_0$, and so $T$ is continuous.
\end{proof}

If $X$ and $Y$ are normed spaces,
we denote by $\mathscr{B}(X,Y)$ the set of bounded linear maps
$X \to Y$. It is straightforward to check that $\mathscr{B}(X,Y)$ is a normed space with the operator norm.
One proves that
if $Y$ is a Banach space, then $\mathscr{B}(X,Y)$ is a Banach space,\footnote{Walter Rudin, {\em Functional Analysis}, second ed., p.~92, Theorem 4.1.}
and if $X$ is a Banach space one then checks that $\mathscr{B}(X)=\mathscr{B}(X,X)$  is a Banach algebra. 
If $X$ is a normed space, we define $X^*=\mathscr{B}(X,\mathbb{C})$, which is a Banach space,  called the {\em dual space of $X$}.


If $X$ and $Y$ are normed spaces and $T:X \to Y$ is linear, we say that $T$ has {\em finite rank} if
\[
\rank T = \dim T(X)
\]
 is finite. If $X$ is infinite dimensional and $Y \neq \{0\}$,
let $\mathscr{E}$ be a Hamel basis for $X$, let $\{e_n: n \in \mathbb{N}\}$ be a countable subset of $\mathscr{E}$, and let $y \in Y$ be nonzero.
If we define
$T:X \to Y$ by $Te_n=n\norm{e_n}y$ and $Te=0$ if $e \in \mathscr{E} \setminus \{e_n:n \in \mathbb{N}\}$, then $T$ is a linear map with finite rank yet $T$ is unbounded.
Thus a finite rank linear map is not necessarily bounded. We denote by $\mathscr{B}_{00}(X,Y)$ the set of bounded finite rank linear maps $X \to Y$, and check that
$\mathscr{B}_{00}(X,Y)$ is a vector space.
If $X$ is a Banach space, one checks that $\mathscr{B}_{00}(X)$ is an ideal of the algebra $\mathscr{B}(X)$ (if we either pre- or postcompose a linear map with a finite rank
linear map, the image will be finite dimensional).

If $X$ and $Y$ are Banach spaces, we say that $T:X \to Y$ is {\em compact} if the image of any bounded set under $T$ is precompact (has compact
closure). One
checks that if a linear map is compact then it is bounded (unlike a finite rank linear map, which is not necessarily bounded). There are several ways to state that a linear map is compact that one proves are equivalent:
$T$ is compact if and only if the image of the closed unit ball is precompact;
$T$ is compact if and only if the image of the open unit ball is precompact;
$T$ is compact
if and only if  the image under it of any bounded sequence has a convergent subsequence.
In a complete metric space, the {\em Heine-Borel theorem} asserts that a set is precompact   if and only if it is {\em totally
bounded} (for any $\epsilon>0$, the set can be covered by a finite number of balls of radius $\epsilon$).
We denote by $\mathscr{B}_0(X,Y)$ the set of compact
linear maps $X \to Y$, and it is straightforward to check that this is a vector space. One proves that if an operator is in the closure of the compact operators then the image of the closed unit
ball under it is totally bounded, and from this it follows that $\mathscr{B}_0(X,Y)$ is a closed
subspace of $\mathscr{B}(X,Y)$.
$\mathscr{B}_0(X)$ is an ideal of the algebra $\mathscr{B}(X)$: if $K \in \mathscr{B}_0(X)$ and $T \in \mathscr{B}(X)$, one checks that $TK \in \mathscr{B}_0(X)$ and 
$KT \in \mathscr{B}_0(X)$. 

Let $X$ and $Y$ be Banach spaces.
Using the fact that a bounded set in a finite dimensional normed vector space is precompact,
we can prove that a bounded finite rank operator is compact: $\mathscr{B}_{00}(X,Y) \subseteq \mathscr{B}_0(X,Y)$.
Also, it doesn't take long to prove that the image of a compact operator is {\em separable}: if $T \in \mathscr{B}_0(X,Y)$ then $T(X)$ has a countable dense subset. (We can
prove this
using the fact that a compact metric space is separable.) 


If $H$ is a Hilbert space and $S_i, i \in I$ are  subsets of $H$, we define $\bigvee_{i \in I} S_i$ to be the closure of the span of
$\bigcup_{i \in I} S_i$. We say that $\mathscr{E}$ is an {\em orthonormal basis} for $H$ if
$\inner{e}{f}=\delta_{e,f}$ and  $H=\bigvee \mathscr{E}$.

A {\em sesquilinear form} on $H$ is a function
$f:H \times H \to \mathbb{C}$ that is linear
in its first argument and  that satisfies $f(x,y)=f(y,x)^*$. If $f$ is a sesquilinear form on $H$, we say that
$f$ is {\em bounded} if
\[
\sup\{|f(x,y)|:\norm{x},\norm{y} \leq 1\} < \infty.
\]
The {\em Riesz representation theorem}\footnote{Walter Rudin, {\em Functional Analysis}, second ed., p.~310, Theorem 12.8.}
 states that if $f$ is a bounded sesquilinear form on $H$, then there is a unique $B 
\in \mathscr{B}(H)$ such that
\[
f(x,y)=\inner{x}{By}, \qquad x,y \in H,
\]
and $\norm{B}=\sup\{|f(x,y)|:\norm{x},\norm{y} \leq 1\}$.
It follows from the Riesz representation that if $A \in \mathscr{B}(H)$, then there is a unique $A^* \in \mathscr{B}(H)$ such that
\[
\inner{Ax}{y}=\inner{x}{A^*y}, \qquad x,y \in H,
\]
and  $\norm{A^*}=\norm{A}$.
$\mathscr{B}(H)$ is a {\em $C^*$-algebra}:
if $A,B \in \mathscr{B}(H)$ and $\lambda \in \mathbb{C}$ then $A^{**}=A$, $(A+B)^*=A^*+B^*$, $(AB)^*=B^*A^*$, $(\lambda A)^*=\lambda^* A^*$,
and $\norm{A^*A}=\norm{A}^2$.

We say that $A \in \mathscr{B}(H)$ is {\em normal} if $A^*A=AA^*$, and 
 {\em self-adjoint} if $A^*=A$. One proves using the parallelogram law that $A \in \mathscr{B}(H)$ is self-adjoint if and only if
$\inner{Ax}{x} \in \mathbb{R}$ for all $x \in H$. If $A \in \mathscr{B}(H)$ is self-adjoint, we say that $A$
is {\em positive} if $\inner{Ax}{x} \geq 0$ for all $x \in H$.




\section{Spectrum in Banach spaces}
If $X$ and $Y$ are Banach spaces and
$T \in \mathscr{B}(X,Y)$ is a bijection, 
then its inverse function $T^{-1}:Y \to X$ is linear, since the inverse of a linear bijection is itself linear. Because $T$ is a surjective bounded linear map,
by the {\em open mapping theorem} it is an open map:
if $U$ is an open subset of $X$ then $T(U)$ is an open subset of $Y$, and it follows that $T^{-1} \in \mathscr{B}(Y,X)$. That is, if a bounded linear operator from one Banach space to another is bijective then its inverse function is also a bounded linear operator. 

If $X$ is a Banach space and $T \in \mathscr{B}(X)$, the {\em spectrum} $\sigma(T)$ of $T$ is the set of those $\lambda \in \mathbb{C}$ such that the map $T-\lambda \id_X:X \to X$ is not a bijection. One proves that $\sigma(T)$ is nonempty (the proof uses Liouville's theorem, which states that a bounded entire function is constant). One also
proves that
if $\lambda \in \sigma(T)$ then $|\lambda| \leq \norm{T}$. We define the {\em spectral radius} of $T$ to be
\[
r(T) = \sup_{\lambda \in \sigma(T)} |\lambda|,
\]
and so
$r(T) \leq \norm{T}$.
Because $\mathscr{B}_0(X)$ is an ideal in the algebra $\mathscr{B}(X)$, if $T \in \mathscr{B}_0(X)$ is invertible then $\id_X$ is compact. One checks that
if $\id_X$ is compact then $X$ is finite dimensional (a locally compact topological vector space is finite dimensional), and 
therefore, if $X$ is an infinite dimensional Banach space and $T \in \mathscr{B}_0(X)$, then $0 \in \sigma(T)$. 

The {\em resolvent set} of $T$ is $\rho(T) = \mathbb{C} \setminus \sigma(T)$. One proves that
$\rho(T)$ is open,\footnote{Gert K. Pedersen, {\em Analysis Now}, revised printing, p.~131, Theorem 4.1.13.} from which it then follows that $\sigma(T)$ is a compact set.
For $\lambda \in \rho(T)$, we define
\[
R(\lambda,T) = (T-\lambda \id_X)^{-1} \in \mathscr{B}(X),
\]
called the {\em resolvent} of $T$.


If $X$ is a Banach space and  $T \in \mathscr{B}_0(X)$, then
the {\em point spectrum} $\point(T)$ of $T$ is the set of those $\lambda \in \mathbb{C}$ such that $T-\lambda \id_X$ is not injective. In other words, to say that 
$\lambda \in \point(T)$ is to say that 
\[
\dim \ker(T-\lambda \id_X) >0.
\]
If $\lambda \in \point(T)$, we say that $\lambda$ is an {\em eigenvalue} of $T$, and call $\dim \ker(T-\lambda \id_X)$ its {\em geometric multiplicity}.\footnote{There is
also a notion of {\em algebraic multiplicity} of an eigenvalue: the algebraic multiplicity of $\lambda$ is defined to be
\[
\sup_{n \in \mathbb{N}} \dim \ker((A-\lambda \id_H)^n).
\]
For self-adjoint operators this is equal to the geometric multiplicity of $\lambda$, while for a operator that is not
self-adjoint the algebraic multiplicity of an eingevalue may be greater than its geometric multiplicity.
Nonzero elements of $\ker((A-\lambda \id_H)^n)$ are called {\em generalized eigenvectors} or {\em root vectors}.
See
I. C. Gohberg and M. G. Krein, {\em Introduction to the Theory
of Linear Nonselfadjoint Operators in Hilbert Space}.}
It is a fact that each nonzero eigenvalue of $T$ has finite geometric multiplicity, and  it is also a fact that if $T \in \mathscr{B}_0(X)$,
then $\point(T)$ is a bounded countable set and that if $\point(T)$ has a limit point that limit point is $0$.
The {\em Fredholm alternative} tells us that
\[
\sigma(T) \subseteq \point(T) \cup \{0\}.
\]
If $X$ is infinite dimensional then $\sigma(T) = \point(T) \cup \{0\}$, and $\point(T)$ might or might not include $0$.
If $T \in \mathscr{B}_{00}(X)$, check that $\point(T)$ is a finite set. 


If $H$ is a Hilbert space, 
using the fact that a bounded linear
operator $T \in \mathscr{B}(H)$ is invertible $T$ if and only if both $TT^*$ and $T^*T$ are  bounded below ($S$ is {\em bounded below} if there is some $c>0$ such that
$\norm{Sx} \geq c\norm{x}$ for all $x \in X$), one can prove that the spectrum of a bounded self-adjoint operator is a set
of real numbers, and the spectrum
of a bounded positive operator is a set of nonnegative real numbers.


\section{Numerical radius}
If $H$ is a Hilbert space and
 $A \in \mathscr{B}(H)$, the {\em numerical range} of $A$ is the set $\{\inner{Ax}{x}:\norm{x}=1\}$.\footnote{The {\em Toeplitz-Hausdorff theorem} states that
the numerical range of any bounded linear operator is a convex set. See Paul R. Halmos, {\em A Hilbert Space Problem Book}, Problem 166.} 
The closure of the numerical range contains the spectrum of $A$.\footnote{Paul R. Halmos, {\em A Hilbert Space Problem Book}, Problem 169. If $A$
is normal, then the closure of its numerical range is the convex hull of the spectrum: Problem 171.}
The {\em numerical radius} $w(A)$ of $A$ is the supremum of the numerical range of $A$: $w(A)=\sup_{\norm{x}=1} |\inner{Ax}{x}|$.
If $A \in \mathscr{B}(H)$ is self-adjoint, one can prove that\footnote{John B. Conway,
{\em A Course in Functional Analysis}, second ed., p.~34, Proposition 2.13.}
\[
w(A) = \norm{A}.
\]

The following theorem asserts that a compact self-adjoint operator $A$ has an eigenvalue whose absolute value is equal to the norm of the operator.
Thus in particular, the spectral radius of a compact self-adjoint operator is equal to its numerical radius.
Since a self-adjoint operator has real spectrum, to say that $|\lambda|=\norm{A}$ is to say that either $\lambda=\norm{A}$ or $\lambda=-\norm{A}$.
A compact operator on a Hilbert space can have empty point spectrum (e.g.  the Volterra operator on $L^2([0,1])$) and a bounded self-adjoint
operator can have empty point spectrum (e.g. the multiplication operator $T\phi(t)=t\phi(t)$ on
$L^2([0,1])$), but this theorem shows that if an operator is compact and self-adjoint then its point spectrum is nonempty.

\begin{theorem}
If $A \in \mathscr{B}(H)$ is compact and self-adjoint then at least one of $-\norm{A},\norm{A}$ is an eigenvalue of $A$.
\end{theorem}
\begin{proof}
Because $A$ is self-adjoint, $\norm{A}=w(A)=\sup_{\norm{x}=1} |\inner{Ax}{x}|$. Also, as $A$ is self-adjoint, $\inner{Ax}{x}$ is a real
number, and thus either $\norm{A}=\sup_{\norm{x}=1} \inner{Ax}{x}$ or $\norm{A}=-\inf_{\norm{x}} \inner{Ax}{x}$. 
In the first case, due to $\norm{A}$ being a supremum there is a sequence $x_n$, all with norm $1$, such that
$\inner{Ax_n}{x_n} \to \norm{A}$.
Using that $A$ is compact, there is a subsequence $x_{a(n)}$ such that $Ax_{a(n)}$ converges to some $x$, and $\norm{x}=1$
because each $x_n$ has norm $1$.
Using $A=A^*$,
\begin{eqnarray*}
\inner{Ax_n-\norm{A}x_n}{Ax_n-\norm{A}x_n}&=&\inner{Ax_n}{Ax_n}-\inner{Ax_n}{\norm{A}x_n}\\
&&-\inner{\norm{A}x_n}{Ax_n}+\inner{\norm{A}x_n}{\norm{A}x_n}\\
&=&\norm{Ax_n}^2-2\norm{A}\inner{Ax_n}{x_n}+\norm{A}^2 \norm{x_n}^2\\
&\leq&\norm{A}^2 \norm{x_n}^2 -2\norm{A}\inner{Ax_n}{x_n}+\norm{A}^2 \norm{x_n}^2\\
&=&2\norm{A}^2 \norm{x_n}^2 -2\norm{A}\inner{Ax_n}{x_n}\\
&\to&2\norm{A}^2 \norm{x}^2 -2\norm{A}\norm{A}\\
&=&0.
\end{eqnarray*}
Therefore, as $n \to \infty$, the sequence $Ax_n - \norm{A}x_n$ tends to $0$, so $Ax-\norm{A}x=0$, i.e. $x$ is an eigenvector for the eigenvalue
$\norm{A}$. If $\norm{A}=-\inf_{\norm{x}=1} \inner{Ax}{x}$ the argument goes the same.
\end{proof}




\section{Polar decomposition}
If $H$ is a Hilbert space and $P \in \mathscr{B}(H)$ is positive, there is a unique positive element of $\mathscr{B}(H)$, denoted $P^{1/2}$, satisfying
$(P^{1/2})^2=P$, which we call the {\em positive square root} of $P$.\footnote{Gert K. Pedersen,
{\em Analysis Now}, revised printing, p.~92, Proposition 3.2.11.} If $A \in \mathscr{B}(H)$ one checks that $A^*A$ is positive, and hence 
$A^*A$ has a positive square root, which we denote by $|A|$ and call the {\em absolute value} of $A$. One proves that
$|A|$ is the unique positive operator in $\mathscr{B}(H)$ satisfying
\[
\norm{Ax} = \norm{|A|x}, \qquad x \in H.
\]
An element $U$ of $\mathscr{B}(H)$ is said to be a {\em partial isometry} if there is a closed subspace $X$ of $H$ such that the restriction of $U$ to $X$ is an isometry
$X \to U(X)$ and $\ker U = X^\perp$. One proves that $U^*U$ is the orthogonal projection of $H$ onto $X$. It can be proved that if $A \in \mathscr{B}(H)$ then there is a unique partial isometry
$U$ satisfying both $\ker U= \ker A$ and $A=U|A|$.\footnote{Gert K. Pedersen, {\em Analysis Now}, revised printing, p.~96, Theorem 3.2.17.} This is called the {\em polar decomposition} of $A$. The
polar decomposition satisfies
\[
U^*U|A|=|A|, \quad U^*A=|A|, \quad UU^*A=A.
\]



\section{Spectral theorem}
If $e,f \in H$, we define $e \otimes f:H \to H$ by $e \otimes f (h) = \inner{h}{f}e$. $e \otimes f$ is linear, and
\[
\norm{e \otimes f (h)} = \norm{\inner{h}{f}e}= |\inner{h}{f}| \norm{e} \leq \norm{h}\norm{f}\norm{e},
\]
so $\norm{e \otimes f} \leq \norm{e} \norm{f}$. Depending on whether $f=0$ the image of $e \otimes f$ is $\{0\}$ or the span
of $e$,  and in either case $e \otimes f \in \mathscr{B}_{00}(H)$. If either of $e$ or $f$ is $0$ then $e \otimes f$ has rank $0$,
and otherwise $e \otimes f$ has rank $1$, and it is an orthogonal projection precisely when $f$ is a multiple of $e$. 

If $\mathscr{E}$ is an orthonormal set in a Hilbert space $H$, then $\mathscr{E}$ is an orthonormal basis for $H$ if and only if the unordered sum
\[
\sum_{e \in \mathscr{E}} e \otimes e
\]
converges strongly to $\id_H$.\footnote{John B. Conway, {\em A Course in Functional Analysis}, second ed., p.~16, Theorem 4.13.}

Let's summarize what we have stated so far about the spectrum and point spectrum of a compact self-adjoint operator on a Hilbert space.

\begin{theorem}[Spectrum of compact self-adjoint operators]
If $H$ is a Hilbert space and $A \in \mathscr{B}_0(H)$ is self-adjoint, then:
\begin{itemize}
\item $\sigma(A)$ is a nonempty compact subset of $\mathbb{R}$.
\item If $H$ is infinite dimensional,  then $0 \in \sigma(A)$.
\item $\sigma(A) \subseteq \point(A) \cup \{0\}$.
\item $\point(A)$ is countable.
\item If $\lambda \in \mathbb{R}$ is a limit point of $\point(A)$, then $\lambda=0$.
\item At least one of $\norm{A},-\norm{A}$ is an element of $\point(A)$.
\item Each nonzero eigenvalue of $A$ has finite geometric multiplicity: If $\lambda \in \point(A)$ and $\lambda \neq 0$, then $\dim \ker(A-\lambda \id_H)<\infty$.
\item If $A \in \mathscr{B}_{00}(H)$, then $\point(A)$ is a finite set.
\end{itemize}
\label{sactheorem}
\end{theorem}

We say that $A \in \mathscr{B}(H)$ is {\em diagonalizable} if there is an orthonormal basis $\mathscr{E}$ for $H$ and a bounded set $\{\lambda_e \in \mathbb{C}: e \in \mathscr{E}\}$
such that the unordered sum
\[
\sum_{e \in \mathscr{E}} \lambda_e e \otimes e
\]
converges  strongly to $A$.

The following is the {\em spectral theorem} for normal compact operators.\footnote{Gert K. Pedersen, {\em Analysis Now}, revised printing,
p.~108, Theorem 3.3.8.} 

\begin{theorem}[Spectral theorem]
If $A \in \mathscr{B}_0(H)$ is normal, then $A$ is diagonalizable.
\end{theorem}

The last assertion of Theorem \ref{sactheorem} is that a bounded  self-adjoint finite rank operator on a Hilbert space has finitely many elements in
its point spectrum. Using the spectral theorem, we get that if $A \in \mathscr{B}_0(H)$ is self-adjoint and $\point(A)$ is finite, then $A$ is finite rank.
In the notation we introduce in the following definition, $\nu(A)<\infty$ precisely when $A$ has finite rank.

\begin{definition}
If $A \in \mathscr{B}_0(H)$ is self-adjoint, define $0 \leq \nu(A) \leq \infty$ to be the sum of the geometric multiplicities of the nonzero eigenvalues of $A$:
\[
\nu(A) = \sum_{\lambda \in \point(A) \setminus \{0\}} \dim \ker(A-\lambda \id_H).
\]
Define 
\[
(\lambda_n(A):n \in \mathbb{N}) \in \mathbb{R}^{\mathbb{N}}
\]
to be the sequence whose first term is the element of
$\point(A) \setminus \{0\}$ with
largest absolute value repeated as many times as its geometric multiplicity. If $\lambda,-\lambda$ are both  nonzero elements of $\point(A)$, we put the  positive one  first. We
repeat this for the remaining elements of $\point(A) \setminus \{0\}$.
If $\nu(A) < \infty$, we define $\lambda_n(A) = 0$ for $n>\nu(A)$.
\label{nudef}
\end{definition}

Using the spectral theorem and the notation in the above definition we get the following.

\begin{theorem}
If $A \in \mathscr{B}_0(H)$ is self-adjoint, then there is an orthonormal set $\{e_n: n \in \mathbb{N}\}$ in $H$ such that
\[
\sum_{n \in \mathbb{N}} \lambda_n(A) e_n \otimes e_n
\]
converges strongly to $A$.
\label{spectralsac}
\end{theorem}

If $A \in \mathscr{B}_0(H)$, then its absolute value $|A|$ is a positive compact operator, and $\lambda_n(|A|) \geq 0$ for all $n \in \mathbb{N}$. 

\begin{definition}
If $A \in \mathscr{B}_0(H)$ and
$\lambda$ is an eigenvalue of $|A|$, we call $\lambda$ a {\em singular
value} of $A$, and we define
\[
\sigma_n(A) = \lambda_n(|A|), \qquad n \in \mathbb{N}.
\]
\end{definition}

Because the absolute value of the absolute value of an operator is the absolute value of the operator, if $A \in \mathscr{B}_0(H)$ and $n \in \mathbb{N}$ then
$\sigma_n(|A|)=\sigma_n(A)$.
If $A \in \mathscr{B}_0(H)$ and
 $\lambda \neq 0$, one proves that $\lambda$ is an eigenvalue of $AA^*$ if and only if $\lambda$ is an eigenvalue of $A^*A$ and that they
have the same geometric multiplicity.
From this we get that $\sigma_n(A) = \sigma_n(A^*)$ for all $n \in \mathbb{N}$.



\section{Finite rank operators}
\begin{theorem}[Singular value decomposition]
If $H$ is a Hilbert space and $A \in \mathscr{B}_{00}(H)$ has $\rank A=N$, then there is an orthonormal set $\{e_n:1 \leq n \leq N\}$ and an orthonormal set 
$\{f_n:1 \leq n \leq N\}$ such that
\[
A = \sum_{n=1}^N \sigma_n(A) e_n \otimes f_n, \qquad Ah=\sum_{n=1}^N \sigma_n(A) \inner{h}{f_n}e_n.
\]
\label{finiterankSVD}
\end{theorem}
\begin{proof}
$|A|$ is a positive operator with $\rank |A|=N$, and according to Theorem \ref{spectralsac}, there is an orthonormal set $\{f_n: n \in \mathbb{N}\}$ in $H$
such that 
\[
|A| = \sum_{n \in \mathbb{N}} \lambda_n(|A|) f_n \otimes f_n= \sum_{n \in \mathbb{N}} \sigma_n(A) f_n \otimes f_n.
\]
Using the polar decomposition
$A=U|A|$,
\[
A = U|A| = \sum_{n=1}^N \sigma_n(A) (Uf_n) \otimes f_n.
\]
Define $e_n=Uf_n$. As $U^*U|A|=|A|$ and as $|A| \frac{f_m}{\sigma_m(A)}=f_m$,
\begin{eqnarray*}
\inner{e_n}{e_m}&=&\inner{Uf_n}{Uf_m}\\
&=&\inner{f_n}{U^*Uf_m}\\
&=&\inner{f_n}{U^*U |A| \frac{f_m}{\sigma_m(A)}}\\
&=&\inner{f_n}{|A| \frac{f_m}{\sigma_m(A)}}\\
&=&\inner{f_n}{f_m}\\
&=&\delta_{n,m},
\end{eqnarray*}
showing that $\{e_n:1 \leq n \leq N\}$ is an orthonormal set.
\end{proof}

If $e,f,x,y \in H$, then
\[
\inner{e \otimes f (x)}{y}
=\inner{\inner{x}{f}e}{y}
=\inner{x}{f}\inner{e}{y}
=\inner{y}{e}^* \inner{x}{f}
=\inner{x}{\inner{y}{e}f}
=\inner{x}{f \otimes e (y)},
\]
so $(e \otimes f)^*=f \otimes e$.

\begin{theorem}
If $A \in \mathscr{B}_{00}(H)$ then
$A^* \in \mathscr{B}_{00}(H)$.
\end{theorem}
\begin{proof}
Let $A$ have the singular value decomposition
\[
A=\sum_{n=1}^N \sigma_n(A) e_n \otimes f_n.
\]
Taking the adjoint, and because $\sigma_n \in \mathbb{R}$,
\[
A^* = \sum_{n=1}^N \sigma_n(A) f_n \otimes e_n.
\]
 $A^*$ is a sum of finite rank operators and is therefore itself a finite rank operator.
\end{proof}



\section{Compact operators}
If $X$ and $Y$ are Banach spaces, $\mathscr{B}_{00}(X,Y) \subseteq \mathscr{B}_0(X,Y)$. But if $H$ is a Hilbert space we can say much more:
$\mathscr{B}_{00}(H)$ is a dense subset of  $\mathscr{B}_0(H)$. In other words, any compact operator on a Hilbert space can be approximated by
a sequence of bounded finite rank operators.\footnote{John B. Conway, {\em A Course in Functional Analysis}, second ed., p.~41, 
Theorem 4.4.} As the adjoint $A_n^*$ of each of these finite rank operators $A_n$  is itself a bounded finite rank operator,
\[
\norm{A^*-A_n^*} =  \norm{(A-A_n)^*} = \norm{A-A_n} \to 0,
\]
so $A_n^* \to A^*$. Because each bounded finite rank operator is compact and $\mathscr{B}_0(H)$ is closed,
this establishes that  $A^* \in \mathscr{B}_0(H)$. (In fact, it is true that the adjoint of a compact linear operator
between Banach spaces is itself compact, but there we don't have the tool of showing that the adjoint is the limit of the adjoints  of finite rank operators.)

If $H$ is a Hilbert space, the {\em weak topology} is the topology on $H$ such that a net $x_\alpha$ converges to $x$ weakly if for all $h \in H$ the net $\inner{x_\alpha}{h}$
converges to $\inner{x}{h}$ in $\mathbb{C}$. Let $\mathfrak{B}$ be the closed unit ball in $H$, and let it be a topological space with the subspace topology inherited from $H$ with the weak
topology. Thus, a net $x_\alpha \in \mathfrak{B}$ converges to $x \in \mathfrak{B}$ if and only if for all $h \in H$ the net $\inner{x_\alpha}{h}$ converges to $\inner{x}{h}$.

\begin{theorem}
If $H$ is a Hilbert space,
$A \in \mathscr{B}(H)$, and $\mathfrak{B}$ is the closed unit ball in $H$ with the subspace topology inherited from $H$ with the weak topology,
then $A$ is compact if and only if $A|\mathfrak{B}:\mathfrak{B} \to H$ is continuous.
\end{theorem}
\begin{proof}
Suppose that $A$ is compact and let $x_\alpha$ be a net in $\mathfrak{B}$ that converges weakly to some $x \in \mathfrak{B}$. 
If $\epsilon>0$, then there is some $B \in \mathscr{B}_{00}(H)$ with $\norm{A-B}<\epsilon$.
Let $B$ have the singular value decomposition
\[
B=\sum_{n=1}^N \sigma_n(B) e_n \otimes f_n.
\]
We have, using that the $e_n$ are orthonormal,
\begin{eqnarray*}
\norm{Bx_\alpha - Bx}^2&=&\norm{ \sum_{n=1}^N \sigma_n(B) \inner{x_\alpha}{f_n}e_n - \sum_{n=1}^N \sigma_n(B) \inner{x}{f_n}e_n}^2\\
&=&\norm{ \sum_{n=1}^N \sigma_n(B) \inner{x_\alpha - x}{f_n}e_n}^2\\
&=&\sum_{n=1}^N  \sigma_n(B)^2 |\inner{x_\alpha-x}{f_n}|^2.
\end{eqnarray*}
Eventually this is $<\frac{\epsilon}{3}$, and for such $\alpha$,
\begin{eqnarray*}
\norm{Ax_\alpha-Ax}&\leq&\norm{Ax_\alpha-Bx_\alpha}+\norm{Bx_\alpha-Bx}+\norm{Bx-Ax}\\
&\leq&\norm{A-B}\norm{x_\alpha}+\norm{Bx_\alpha-Bx}+\norm{B-A}\norm{x}\\
&\leq&\norm{A-B}+\norm{Bx_\alpha-Bx}+\norm{B-A}\\
&<&\frac{\epsilon}{3}+\frac{\epsilon}{3}+\frac{\epsilon}{3}\\
&=&\epsilon.
\end{eqnarray*}
We have shown that $Ax_\alpha \to Ax$ in the no rm of $H$, and this shows that $A|\mathfrak{B}:\mathfrak{B} \to H$ is continuous.

Suppose that $A|\mathfrak{B}:\mathfrak{B} \to H$ is continuous. Kakutani's theorem states that a Banach space is reflexive if and only if the closed unit ball is weakly
compact. A Hilbert space is reflexive, hence $\mathfrak{B}$, the closed unit ball with the weak topology, is a compact topological
space.\footnote{cf. Paul R. Halmos, {\em A Hilbert Space Problem Book}, Problem 17.} 
Since $A|\mathfrak{B}:\mathfrak{B} \to H$
is continuous and $\mathfrak{B}$ is compact, the image $A(\mathfrak{B})$ is compact (the image of a compact set under a continuous map is a compact set). We have shown that the image of the closed unit ball is a compact subset of $H$, and this
shows that $A$ is compact; in fact, to have shown that $A$ is compact we merely needed to show that the image of the closed unit ball is precompact, and $H$ is a Hausdorff
space so a compact set is precompact.
\end{proof}


A compact linear operator on an infinite dimensional Hilbert space $H$ is not invertible, lest $\id_H$ 
be compact. However, operators of the form $A-\lambda \id_H$ may indeed be invertible.\footnote{Ward
Cheney, {\em Analysis for Applied Mathematics}, p.~94, Theorem 2.}

\begin{theorem}
If $A \in \mathscr{B}_0(H)$ is a normal operator with diagonalization
\[
A=\sum_{n=1}^\infty \lambda_n e_n \otimes e_n
\]
and $0 \neq \lambda \not \in \point(A)$, 
 then $A-\lambda \id_H$ is invertible and
\[
(A-\lambda \id_H)^{-1} = -\frac{1}{\lambda}+\frac{1}{\lambda} \sum_{n=1}^\infty \frac{\lambda_n}{\lambda_n-\lambda} e_n \otimes e_n,
\]
where the series converges in the strong operator topology.
\end{theorem}
\begin{proof}
As $\lambda_n \to 0$ we have $\alpha = \sup_n |\lambda_n|<\infty$, and as $\lambda \neq 0$ we have $\beta=\inf_n |\lambda_n-\lambda|>0$.
Define
\[
T_N =-\frac{1}{\lambda}+\frac{1}{\lambda} \sum_{n=1}^N \frac{\lambda_n}{\lambda_n-\lambda} e_n \otimes e_n \in \mathscr{B}_{00}(H),
\]
and if $N>M$, then, for any $h \in H$,
\begin{eqnarray*}
\norm{T_Nh-T_Mh}^2 &=&\frac{1}{|\lambda|^2} \norm{\sum_{n=M+1}^N  \frac{\lambda_n}{\lambda_n-\lambda} \inner{h}{e_n}e_n}^2\\
&=&\frac{1}{|\lambda|^2} \sum_{n=M+1}^N  \frac{|\lambda_n|^2}{|\lambda_n-\lambda|^2} |\inner{h}{e_n}|^2\\
&\leq&\frac{1}{|\lambda|^2} \sum_{n=M+1}^N \frac{\alpha^2 |\inner{h}{e_n}|^2}{\beta^2}\\
&=&\frac{\alpha^2}{|\lambda|^2 \beta^2} \sum_{n=M+1}^N |\inner{h}{e_n}|^2.
\end{eqnarray*}
By Bessel's inequality, $\sum_{n=1}^\infty |\inner{h}{e_n}|^2 \leq \norm{h}^2$, hence $\sum_{n=N}^\infty |\inner{h}{e_n}|^2 \to 0$ as $N \to \infty$; this $N$ depends on $h$,
and this is why the claim is stated merely for the strong operator topology and not the norm topology. We have shown that $T_N h$ is a Cauchy sequence in $H$ and hence
$T_N h$ converges. We define $Bh$ to be this limit. 
For $h \in H$,
\begin{eqnarray*}
\norm{\frac{1}{\lambda} \sum_{n=1}^\infty \frac{\lambda_n}{\lambda_n-\lambda} \inner{h}{e_n}e_n}^2&=&\frac{1}{|\lambda|^2} \sum_{n=1}^\infty \frac{|\lambda_n|^2}{|\lambda_n-\lambda|^2} |\inner{h}{e_n}|^2\\
&\leq&\frac{\alpha^2}{|\lambda|^2 \beta^2} \sum_{n=1}^\infty |\inner{h}{e_n}|^2\\
&\leq&\frac{\alpha^2}{|\lambda|^2 \beta^2} \norm{h}^2,
\end{eqnarray*}
whence
\begin{eqnarray*}
\norm{Bh} &\leq& \norm{-\frac{1}{\lambda}h} + \norm{\frac{1}{\lambda} \sum_{n=1}^\infty \frac{\lambda_n}{\lambda_n-\lambda} \inner{h}{e_n}e_n}\\
&\leq&\frac{1}{|\lambda|} \norm{h} + \frac{\alpha}{|\lambda|\beta} \norm{h},
\end{eqnarray*}
showing that $\norm{B} \leq \frac{1}{|\lambda|} + \frac{\alpha}{|\lambda| \beta}$. It is straightforward to check that $B$ is linear, thus
$B \in \mathscr{B}(H)$. (Thus $B$ is a strong limit of finite rank operators. But if $H$ is infinite dimensional then $B$ is in fact not the norm limit of the sequence: for
if it were it would be compact, and we will show that $B$ is invertible, which would tell us that $\id_H$ is compact, contradicting $H$ being infinite dimensional.)

For $h \in H$, 
\begin{eqnarray*}
(A-\lambda \id_H)Bh&=&-\frac{1}{\lambda}(Ah-\lambda h) + \frac{1}{\lambda} \sum_{n=1}^\infty \frac{\lambda_n}{\lambda_n-\lambda}
\inner{h}{e_n}(Ae_n-\lambda e_n)\\
&=&h-\frac{1}{\lambda}Ah + \frac{1}{\lambda} \sum_{n=1}^\infty \frac{\lambda_n}{\lambda_n-\lambda}
\inner{h}{e_n}(\lambda_n e_n-\lambda e_n)\\
&=&h-\frac{1}{\lambda} Ah + \frac{1}{\lambda} \sum_{n=1}^\infty \lambda_n \inner{h}{e_n}e_n\\
&=&h,
\end{eqnarray*}
where the final equality is because the series is the diagonalization of $A$. On the other hand,
\begin{eqnarray*}
B(A-\lambda \id_H)h&=&-\frac{1}{\lambda}(A-\lambda \id_H)h + \frac{1}{\lambda} \sum_{n=1}^\infty
 \frac{\lambda_n}{\lambda_n-\lambda} \inner{(A-\lambda \id_H)h}{e_n}e_n\\
 &=&h-\frac{1}{\lambda}Ah + \frac{1}{\lambda} \sum_{n=1}^\infty
 \frac{\lambda_n}{\lambda_n-\lambda} \inner{Ah-\lambda h}{e_n}e_n\\
 &=&h-\frac{1}{\lambda}\sum_{n=1}^\infty \lambda_n \inner{h}{e_n}e_n+\frac{1}{\lambda}\sum_{n=1}^\infty \frac{\lambda_n}{\lambda_n-\lambda}
 \inner{Ah}{e_n}e_n\\
 &&-\sum_{n=1}^\infty \frac{\lambda_n}{\lambda_n-\lambda} \inner{h}{e_n}e_n\\
 &=&h-\frac{1}{\lambda}\sum_{n=1}^\infty \lambda_n \inner{h}{e_n}e_n+\frac{1}{\lambda}\sum_{n=1}^\infty \frac{\lambda_n}{\lambda_n-\lambda}
\lambda_n \inner{h}{e_n}e_n\\
 &&-\sum_{n=1}^\infty \frac{\lambda_n}{\lambda_n-\lambda} \inner{h}{e_n}e_n\\
 &=&h+\frac{1}{\lambda} \sum_{n=1}^\infty \frac{-\lambda_n(\lambda_n-\lambda) +\lambda_n^2-\lambda_n \lambda}{\lambda_n-\lambda}
 \inner{h}{e_n}e_n\\
 &=&h,
\end{eqnarray*}
showing that $B=(A-\lambda \id_H)^{-1}$.
\end{proof}





We can start with a function and ask what kind of series it can be expanded into, or we can start with a series and ask what kind of function it defines. The following theorem does the latter.
It shows that if $e_n$ and $f_n$ are each orthonormal sequences and $\lambda_n$ is a sequence of complex numbers whose
limit of $0$, then the series
\[
\sum_{n=1}^\infty \lambda_n e_n \otimes f_n
\]
converges and  is an element of $\mathscr{B}_0(H)$.

\begin{theorem}
If $H$ is a Hilbert space,
 $\{e_n:n \in \mathbb{N}\}$ is an orthonormal set, $\{f_n: n\in \mathbb{N}\}$ is an orthonormal set, 
and $\lambda_n \in \mathbb{C}$ is a sequence tending to $0$, then the sequence
\[
A_N = \sum_{n=1}^N \lambda_n e_n \otimes f_n \in \mathscr{B}_{00}(H).
\]
converges to an element of $\mathscr{B}_0(H)$.
\label{compactdefine}
\end{theorem}
\begin{proof}
Let $\epsilon>0$ and let $N_0$ be such that if $n \geq N_0$ then $|\lambda_n|<\epsilon$. If $N > M \geq N_0$ and $h \in H$, then, as the $e_n$ are orthonormal,
\begin{eqnarray*}
\norm{(A_N-A_M)h}^2=&&\norm{\sum_{n=M+1}^N \lambda_n e_n \otimes f_n (h)}^2\\
&=&\norm{\sum_{M+1}^N \lambda_n \inner{h}{f_n}e_n}^2\\
&=&\sum_{n=M+1}^N \norm{\lambda_n \inner{h}{f_n}e_n}^2\\
&=&\sum_{n=M+1}^N |\lambda_n|^2 |\inner{h}{f_n}|^2\\
&<&\epsilon^2 \sum_{n=M+1}^N  |\inner{h}{f_n}|^2.
\end{eqnarray*}
By Bessel's inequality, $\sum_{n=M+1}^N  |\inner{h}{f_n}|^2 \leq \norm{h}^2$, and hence
\[
\norm{(A_N-A_M)h} < \epsilon \norm{h}.
\]
As this holds for all $h \in H$, 
\[
\norm{A_N-A_M} \leq \epsilon,
\]
showing that $A_N$ is a Cauchy sequence, which  therefore converges in $\mathscr{B}(H)$.  As each term in the sequence is finite rank and so compact, the limit is a
compact operator.
\end{proof}

Continuing the analogy we used with the above theorem, 
now we start with a function and ask what kind of series it can be expanded into. This is called the {\em singular value decomposition} of a compact operator.
Helemskii calls the series in the following theorem the {\em Schmidt series} of 
the operator.\footnote{A. Ya. Helemskii, {\em Lectures and Exercises on Functional Analysis}, p.~215, Theorem 1.} We have already presented the singular value decomposition
for finite rank operators in Theorem \ref{finiterankSVD}.


\begin{theorem}[Singular value decomposition]
If $H$ is a Hilbert space and
\[
A \in \mathscr{B}_0(H) \setminus \mathscr{B}_{00}(H),
\]
then there is an orthonormal set $\{e_n: n \in \mathbb{N}\}$ and
an orthonormal set $\{f_n: n\in \mathbb{N}\}$ such that
$A_N \to A$, where
\[
A_N = \sum_{n=1}^N \sigma_n(A) e_n \otimes f_n.
\]
\end{theorem}
\begin{proof}
As $|A|$ is self-adjoint and compact, 
by Theorem \ref{spectralsac} there is an orthonormal set $\{f_n: n \in \mathbb{N}\}$ such that
\[
|A|= \sum_{n=1}^\infty \lambda_n(|A|)  f_n \otimes f_n=\sum_{n=1}^\infty \sigma_n(A) f_n \otimes f_n.
\]
That is, with $|A|_N \in \mathscr{B}_{00}(H)$ defined by
\[
|A|_N = \sum_{n=1}^N \sigma_n(A) f_n \otimes f_n,
\]
we have $|A|_N \to |A|$.

Let $A=U|A|$ be the polar decomposition of $A$, and
define $e_n=Uf_n$.
As $U^*U|A|=|A|$ and as $\sigma_m(A) >0$ (because $A$ is not finite rank), we have $|A|\frac{f_m}{\sigma_m(A)}=f_m$, and hence
\begin{eqnarray*}
\inner{e_n}{e_m}&=&\inner{Uf_n}{Uf_m}\\
&=&\inner{f_n}{U^*Uf_m}\\
&=&\inner{f_n}{U^*U|A|\frac{f_m}{\sigma_m(A)}}\\
&=&\inner{f_n}{|A|\frac{f_m}{\sigma_m(A)}}\\
&=&\inner{f_n}{f_m}\\
&=&\delta_{n,m},
\end{eqnarray*}
showing that $\{e_n: n \in \mathbb{N}\}$ is an orthonormal set.
Define
\[
A_N = \sum_{n=1}^N \sigma_n(A) e_n \otimes f_n,
\]
and we have $A_N = U|A|_N$. As $A=U|A|$ and $A_N=U|A|_N$, we get
\[
\norm{A-A_N} = \norm{U|A|-U|A|_N} \leq \norm{U} \norm{|A|-|A|_N} = \norm{|A|-|A|_N} \to 0,
\]
showing that $A_N \to A$.
\end{proof}





\section{Courant min-max theorem}
\begin{theorem}[Courant min-max theorem]
Let $H$ be an infinite dimensional Hilbert space and let $A \in \mathscr{B}_0(H)$ be a positive operator.
If $k \in \mathbb{N}$ then
\[
\max_{\dim S = k} \min_{x \in S, \norm{x}=1} \inner{Ax}{x}=\lambda_k(A)=\sigma_k(A)
\]
and
\[
\min_{\dim S=k-1} \max_{x \in S^\perp, \norm{x}=1} \inner{Ax}{x}=\lambda_k(A)=\sigma_k(A).
\]
\end{theorem}
\begin{proof}
 $|A|$ is compact and positive, so
according to Theorem \ref{spectralsac} there is an orthonormal set $\{e_n: n \in \mathbb{N}\}$ such that
\[
A=\sum_{n=1}^\infty \sigma_n(A) e_n \otimes e_n.
\]
For $k \in \mathbb{N}$,
 let $S_k=\bigvee_{n=k}^\infty \{e_n\}$. $S_k^\perp=\bigvee_{k=1}^{n-1} \{e_n\}$, so $S_k$ has codimension $k-1$. (The {\em codimension} of a closed subspace
of a Hilbert space is the dimension of its orthogonal complement.)
If $S$ is a
$k$ dimensional subspace of $H$,  then there is some $x \in S_k \cap S$ with $\norm{x}=1$. This is because if $V$ is a closed
subspace with codimension $k-1$ of a Hilbert space and $W$ is a  $k$ dimensional subspace of the Hilbert space, then their intersection is a subspace of nonzero dimension. 
As $x \in S_k$, there are $\alpha_n \in \mathbb{C}$, $n \geq k$, with
\[
x=\sum_{n=k}^\infty \alpha_n e_n, \qquad \norm{x}^2 =\sum_{n=k}^\infty |\alpha_n|^2.
\]
As the sequence $\sigma_n(A)$ is nonincreasing,
\begin{eqnarray*}
\inner{Ax}{x}&=&\inner{\sum_{n=1}^\infty \sigma_n(A) \inner{x}{e_n} e_n}{x}\\
&=&\inner{\sum_{n=1}^\infty \sigma_n(A) \inner{\sum_{m=k}^\infty \alpha_m e_m}{e_n} e_n}{\sum_{m=k}^\infty \alpha_m e_m}\\
&=&\inner{\sum_{n=1}^\infty \sigma_n(A) \sum_{m=k}^\infty \alpha_m \delta_{m,n} e_n}{\sum_{m=k}^\infty \alpha_m e_m}\\
&=&\inner{\sum_{n=1}^\infty \sigma_n(A) \alpha_n \chi_{\geq k}(n) e_n}{\sum_{m=k}^\infty \alpha_m e_m}\\
&=&\inner{\sum_{n=k}^\infty \sigma_n(A) \alpha_n  e_n}{\sum_{m=k}^\infty \alpha_m e_m}\\
&=&\sum_{n=k}^\infty \sigma_n(A) |\alpha_n|^2\\
&\leq&\sigma_k(A) \sum_{n=k}^\infty |\alpha_n|^2\\
&=&\sigma_k(A),
\end{eqnarray*}
where we write
\[
\chi_{\geq k}(n)=\begin{cases}
1&n \geq k\\
0&n<k.
\end{cases}
\]
This shows that if $\dim S=k$ then
\[
\inf_{x \in S, \norm{x}=1} \inner{Ax}{x} \leq \sigma_k(A).
\] 
Let $M = \inf_{x \in S, \norm{x}=1} \inner{Ax}{x}$, and
let $x_n \in S$, $\norm{x_n}=1$, with $\inner{Ax_n}{x_n} \to M$.
As $S$ is a finite dimensional Hilbert space, the unit sphere in it is compact, so there a 
a subsequence $x_{a(n)}$ that converges to some $z \in S$, $\norm{z}=1$. 
We have
\begin{eqnarray*}
|\inner{Az}{z}-\inner{Ax_n}{x_n}|&\leq&|\inner{Az}{z}-\inner{Ax_n}{z}|+\inner{Ax_n}{z}-\inner{Ax_n}{x_n}|\\
&=&|\inner{A(z-x_n}{z}| + |\inner{Ax_n}{z-x_n}|\\
&\leq&\norm{A}\norm{z-x_n} \norm{z} + \norm{A}\norm{x_n}\norm{z-x_n}\\
&=&2\norm{A} \norm{z-x_n}.
\end{eqnarray*}
As $x_{a(n)} \to z$, we get $\inner{Ax_{a(n)}}{x_{a(n)}} \to \inner{Az}{z}$. As $\inner{Ax_n}{Ax_n} \to M$, we get
\[
\inner{Az}{z}=M=\inf_{x \in S, \norm{x}=1} \inner{Ax}{x}.
\]
As $z \in S$ and $\norm{z}=1$, we have in fact
\[
\min_{x \in S, \norm{x}=1}  \inner{Ax}{x}=\inf_{x \in S, \norm{x}=1} \inner{Ax}{x} \leq \sigma_k(A).
\]
This is true for any $k$ dimensional subspace of $H$, so
\[
\sup_{\dim S = k} \min_{x \in S, \norm{x}=1} \inner{Ax}{x} \leq \sigma_k(A).
\]
If $S=S_{k+1}^\perp$ then $e_k \in S$, $\norm{e_k}=1$,  and
\[
\inner{Ae_k}{e_k} = \inner{\sigma_k(A) e_k}{e_k} = \sigma_k(A),
\]
so in fact
\[
\max_{\dim S = k} \min_{x \in S, \norm{x}=1} \inner{Ax}{x} = \sigma_k(A),
\]
which is the first of the two formulas that we want to prove.

For $k \geq 1$, let $S_k = \bigvee_{n=1}^k \{e_n\}$. 
If $S$ is a $k-1$ dimensional subspace of $H$, then $S^\perp$ is a closed subspace with codimension $k-1$, so
the intersection of $S_k$ and $S^\perp$ has nonzero dimension, and so there is some $x \in S_k \cap S^\perp$ with $\norm{x}=1$.
As $x \in S_k$ there are $\alpha_1,\ldots,\alpha_k$ with $x=\sum_{n=1}^k \alpha_n e_n$, giving
\begin{eqnarray*}
\inner{Ax}{x}&=&\inner{\sum_{n=1}^\infty \sigma_n(A) \inner{x}{e_n} e_n}{x}\\
&=&\inner{\sum_{n=1}^\infty \sigma_n(A) \inner{\sum_{m=1}^k \alpha_m e_m}{e_n} e_n}{\sum_{m=1}^k \alpha_m e_m}\\
&=&\inner{\sum_{n=1}^\infty \sigma_n(A) \sum_{m=1}^k \alpha_m \delta_{m,n} e_n}{\sum_{m=1}^k \alpha_m e_m}\\
&=&\inner{\sum_{n=1}^\infty \sigma_n(A)  \alpha_n \chi_{\leq k}(n)  e_n}{\sum_{m=1}^k \alpha_m e_m}\\
&=&\inner{\sum_{n=1}^k \sigma_n(A)  \alpha_n  e_n}{\sum_{m=1}^k \alpha_m e_m}\\
&=&\sum_{n=1}^k \sigma_n(A) |\alpha_n|^2\\
&\geq&\sigma_k(A) \sum_{n=1}^k  |\alpha_n|^2\\
&=&\sigma_k(A).
\end{eqnarray*}
This shows that
\[
\sup_{x \in S^\perp, \norm{x}=1} \inner{Ax}{x} \geq \sigma_k(A).
\]
Define $M=\sup_{x \in S^\perp, \norm{x}=1}$. Because $M$ is a supremum,
there is a sequence $x_n$ on the unit sphere in $S_{k-1}$ such that
$\inner{Ax_n}{x_n} \to M$.
The unit sphere in $S_{k-1}$ is compact because $S_{k-1}$ is finite dimensional, so this sequence has a convergent
subsequence $x_{a(n)} \to z$.
As
\[
|\inner{Az}{z}-\inner{Ax_n}{x_n}| \leq 2\norm{A} \norm{z-x_n}
\]
and $x_{a(n)} \to z$, we get
\[
\inner{Az}{z}=M=\sup_{x \in S^\perp, \norm{x}=1} \inner{Ax}{x},
\]
whence
\[
\max_{x \in S^\perp, \norm{x}=1} \inner{Ax}{x}=\sup_{x \in S^\perp, \norm{x}=1} \inner{Ax}{x} \geq \sigma_k(A).
\]
As this is true for any $k-1$ dimensional subspace $S$,
\[
\inf_{\dim S=k-1} \max_{x \in S^\perp, \norm{x}=1} \inner{Ax}{x} \geq \sigma_k(A).
\]
But for $S=S_{k-1}$ we have $e_k \in S^\perp$, $\norm{e_k}=1$, and
\[
\inner{Ae_k}{e_k}=\inner{\sigma_k(A) e_k}{e_k} =\sigma_k(A),
\]
which implies that
\[
\min_{\dim S=k-1} \max_{x \in S^\perp, \norm{x}=1} \inner{Ax}{x} = \sigma_k(A).
\]
\end{proof}

If $A \in \mathscr{B}(H)$ is compact, then the eigenvalues of $|A|$ are equal to the singular values of $|A|$. Therefore
the Courant min-max theorem gives expressions for the singular values of a compact linear operator on a Hilbert space, whether or not the operator
is itself self-adjoint.


Allahverdiev's theorem\footnote{I. C. Gohberg and M. G. Krein, {\em Introduction to the Theory
of Linear Nonselfadjoint Operators in Hilbert Space}, p.~28, Theorem
2.1; cf. J. R. Retherford, {\em Hilbert Space: Compact Operators and the Trace Theorem}, p.~75 and
p.~106.} gives an expression for the singular values of a compact operator that does not involve orthonormal
sets, unlike Courant's min-max theorem. Thus this formula makes sense for a compact operator from one Banach space to another. 


\begin{theorem}[Allahverdiev's theorem]
Let $H$ be a Hilbert space and let $\mathscr{F}_n(H)$ be the set of bounded finite rank operators of rank $\leq n$. If $A \in \mathscr{B}_0(H)$
and $n \in \mathbb{N}$, then
\[
\sigma_n(A) = \inf_{T \in \mathscr{F}_{n-1}} \norm{A-T}. 
\]
\end{theorem}




\section{Schatten class operators}
If $1 \leq p < \infty$ and
$A \in \mathscr{B}_0(H)$, we define
\[
\norm{A}_p=\left( \sum_{n \in \mathbb{N}} \sigma_n(A)^p \right)^{1/p},
\]
and define $\mathscr{B}_p(H)$ to be those $A \in \mathscr{B}_0(H)$ with $\norm{A}_p < \infty$. In other words, an element of $\mathscr{B}_p(H)$ is a 
compact operator whose sequence of singular values is an element of $\ell^p$. We call an element of $\mathscr{B}_p(H)$ a {\em Schatten class operator}.
We call elements of $\mathscr{B}_1(H)$ {\em trace class operators} and elements of $\mathscr{B}_2(H)$ {\em Hilbert-Schmidt operators}.

If $A \in \mathscr{B}_0(H)$ is positive, then, according to Theorem \ref{spectralsac}, there is an orthonormal
set $\{e_n: n \in \mathbb{N}\}$ such that
\[
A = \sum_{n \in \mathbb{N}} \lambda_n(A)  e_n \otimes e_n,
\]
where the series converges in the strong operator topology. As the $e_n$ are orthonormal, we have
\[
A^p = \sum_{n \in \mathbb{N}} \lambda_n(A)^p e_n \otimes e_n,
\]
which is itself a positive compact operator, and thus
$\sigma_n(A^p)=\sigma_n(A)^p$ for $n \in \mathbb{N}$. Therefore, if $A$ is a positive compact operator, then
$\norm{A}_p = \norm{A^p}_1^{1/p}$.

If $A \in \mathscr{B}_0(H)$ and $n \in \mathbb{N}$, then 
$\sigma_n(|A|)=\sigma_n(A)$ and $\sigma_n(A^*)=\sigma_n(A)$. Hence, if $1 \leq p < \infty$ then
\[
\norm{|A|}_p = \norm{A}_p, \qquad \norm{A^*}_p=\norm{A}_p.
\]

As $|A|$ is compact and self-adjoint, it has an eigenvalue with absolute value $\norm{A}$, from which it follows that if $1 \leq p < \infty$ then
$\norm{A} \leq \norm{A}_p$.



\begin{theorem}
If $A \in \mathscr{B}_1(H)$, $B \in \mathscr{B}(H)$, and  $k \in \mathbb{N}$, then
\[
\sigma_k(BA) \leq \norm{B} \sigma_k(A).
\]
\label{singularproduct}
\end{theorem}
\begin{proof}
For all $x \in H$,
\begin{eqnarray*}
\inner{(BA)^*BAx}{x}&=&\inner{BAx}{BAx}\\
&=&\norm{BAx}^2\\
&\leq&\norm{B}^2 \norm{Ax}^2\\
&=&\norm{B}^2 \inner{Ax}{Ax}\\
&=&\norm{B}^2 \inner{A^*Ax}{x}.
\end{eqnarray*}
Applying the Courant min-max theorem to the positive operators $(BA)^*BA$ and $A^*A$, if $k \in \mathbb{N}$ then
\begin{eqnarray*}
\sigma_k((BA)^*BA)&=&\max_{\dim S=k} \min_{x \in S, \norm{x}=1} \inner{(BA)^*BAx}{x}\\
&\leq&\norm{B}^2 \max_{\dim S=k} \min_{x \in S, \norm{x}=1} \inner{A^*Ax}{x}\\
&=&\norm{B}^2 \sigma_k(A^*A).
\end{eqnarray*}
But
\[
\sigma_k((BA)^*BA)=\sigma_k(|BA|^2)=\sigma_k(|BA|)^2=\sigma_k(BA)^2
\]
and
\[
\sigma_k(A^*A)=\sigma_k(|A|^2)=\sigma_k(|A|)^2=\sigma_k(A)^2,
\]
so taking the square root,
\[
\sigma_k(BA) \leq \norm{B} \sigma_k(A).
\]
\end{proof}

Using Theorem \ref{singularproduct}, if $1 \leq p < \infty$ then
\[
\norm{BA}_p=\left( \sum_{n \in \mathbb{N}} \sigma_n(BA)^p \right)^{1/p}
\leq \left( \sum_{n \in \mathbb{N}} \norm{B}^p \sigma_k(A)^p \right)^{1/p}
=\norm{B} \norm{A}_p.
\]

The following theorem states that the Schatten class operators are Banach spaces.\footnote{Gert K. Pedersen,
{\em Analysis Now}, revised printing, p.~124, E 3.4.4} 

\begin{theorem}
If $1 \leq p < \infty$, then $\mathscr{B}_p(H)$ is a Banach space with the norm $\norm{\cdot}_p$.
\end{theorem}





\section{Weyl's inequality}
Weyl's inequality relates the eigenvalues of a self-adjoint compact operator with its singular values.\footnote{Peter D. Lax,
{\em Functional Analysis}, p.~336, chapter 30, Lemma 7.} We use the notation from Definition \ref{nudef}. For $N>\nu(A)$ the left hand side is equal to $0$ so the inequality
is certainly true then.


\begin{theorem}[Weyl's inequality]
If $A \in \mathscr{B}_0(H)$ is self-adjoint and $N \leq \nu(A)$, then
\[
\prod_{n=1}^N |\lambda_n(A)| \leq \prod_{n=1}^N \sigma_n(A).
\]
\label{weyl}
\end{theorem}
\begin{proof}
Let
\[
E_N = \bigvee_{n=1}^N \ker(A-\lambda_n(A) \id_H),
\]
which is finite dimensional.
Check that $E_N$ is an invariant subspace of $A$, and  let
$A_N:E_N \to E_N$ be the restriction of $A$ to $E_N$. $A_N$ is a positive operator.
As $E_N$ is spanned by eigenvectors for nonzero eigenvalues of $A$ it follows that $\ker A_N = \{0\}$,
and as $E_N$ is finite dimensional, we get that $A_N$ is invertible. 
If $A_N$ has polar decomposition $A_N=U_N|A_N|$, then $U_N$ is invertible; if a partial isometry is invertible then it is
unitary, so $U_N$ is unitary, and therefore the eigenvalues of $U_N$ all have absolute value $1$. 
As the determinant of a linear operator on a finite dimensional vector space is the product of its eigenvalues counting algebraic
multiplicity,
\begin{equation}
 \det |A_N|=\frac{1}{|\det U_N|} |\det A_N| = |\det A_N| = \prod_{n=1}^N |\lambda_n(A)|.
 \label{determinants}
  \end{equation}

Let $P_N$ be the orthogonal projection onto $E_N$.
 If $v \in E_N$, then $AP_N v = A_N v$, and if $v \in E_N^\perp$
then $AP_N v = A(0)=0$. We get that
\[
|AP_N|v = \begin{cases}
|A_N|v&v \in E_N\\
0&v \in E_N^\perp,
\end{cases}
\]
and it follows that if $1 \leq n \leq N$ then $\sigma_n(A_N) = \sigma_n(AP_N)$. Using Theorem \ref{singularproduct} we get
\[
\sigma_n(AP_N) \leq \norm{P} \sigma_n(A) \leq \sigma_n(A);
\]
the second inequality is an equality unless $P_N=0$. We have shown that if $1 \leq n \leq N$ then
$\sigma_n(A_N) \leq \sigma_n(A)$, and  combining this with \eqref{determinants} gives us
\[
\prod_{n=1}^N |\lambda_n(A)| = \det |A_N| = \prod_{n=1}^N \sigma_n(A_N) \leq \prod_{n=1}^N \sigma_n(A).
\]
\end{proof}


\begin{theorem}
If $0<p<\infty$, $A \in \mathscr{B}_0(H)$ is self-adjoint, and $N \in \mathbb{N}$, then
\[
\sum_{n=1}^N |\lambda_n(A)|^p \leq \sum_{n=1}^N \sigma_n(A)^p.
\]
\end{theorem}
\begin{proof}
{\em Schur's majorization inequality}\footnote{Peter D. Lax,
{\em Functional Analysis}, p.~337, chapter 30, Lemma 8; cf. J. Michael Steele, {\em The Cauchy-Schwarz Master
Class}, p.~201, Problem 13.4.}
 states that if
 $a_1 \geq a_2 \geq \cdots$ and
$b_1 \geq b_2 \geq \cdots$ are nonincreasing sequences of real numbers satisfying, for each $N \in \mathbb{N}$,
\[
\sum_{n=1}^N a_n \leq \sum_{n=1}^N b_n,
\]
and $\phi:\mathbb{R} \to \mathbb{R}$ is a convex function with $\lim_{x \to -\infty} \phi(x)=0$, then for every $N \in \mathbb{N}$,
\[
\sum_{n=1}^N \phi(a_n) \leq \sum_{n=1}^N \phi(b_n).
\]
With the hypotheses of Theorem \ref{weyl}, for $1 \leq n \leq \nu(A)$, define
 $a_n = \log |\lambda_n(A)|$ and
$b_n = \log \sigma_n(A)$ and let $\phi(x)=e^{px}$. By Theorem \ref{weyl} these satisfy the conditions of Schur's majorization inequality,
which then gives us for $1 \leq N \leq \nu(A)$ that
\[
\sum_{n=1}^N |\lambda_n(A)|^p \leq \sum_{n=1}^N \sigma_n(A)^p.
\]
If $n>\nu(A)$ then $\lambda_n(A)=0$.
\end{proof}



\section{Rayleigh quotients for self-adjoint operators}
If $A \in \mathscr{B}(H)$ is self-adjoint, we define the {\em Rayleigh quotient} of $A$ by
\[
f(x)=\frac{\inner{Ax}{x}}{\inner{x}{x}}, \quad x \in H, x \neq 0, \qquad f:H \setminus \{0\} \to \mathbb{R}.
\]

Let $X$ and $Y$ be normed spaces, $U$ an open subset of $X$, and $f:U \to Y$ a function.
If $x \in U$ and there is some $T \in \mathscr{B}(X,Y)$ such that 
\begin{equation}
\lim_{h \to 0} \frac{\norm{f(x+h)-f(x)-Th}}{\norm{h}} = 0,
\label{frechet}
\end{equation}
then $f$ is said to be {\em Fr\'echet differentiable at $x$}, and $T$ is called the {\em Fr\'echet derivative of $f$ at $x$};\footnote{Ward
Cheney, {\em Analysis for Applied Mathematics}, p.~149.} it does not take long to prove that if $T_1,T_2 \in \mathscr{B}(X,Y)$ both satisfy
\eqref{frechet} then $T_1=T_2$. We denote the Fr\'echet derivative of $f$ at $x$ by $(Df)x$. $Df$ is a map from the set of all points at which
$f$ is Fr\'echet differentiable to $\mathscr{B}(X,Y)$.

To say that $x$ is a {\em stationary point} of $f$ is to say that
$f$ is Fr\'echet differentiable at $x$ and that the Fr\'echet derivative of $f$ at $x$ is the zero map. One proves that
if $T_1,T_2$ are Fr\'echet derivatives of $f$ at $x$ then $T_1=T_2$, and thus speak about {\em the} Fr\'echet derivative
of $f$ at $x$

\begin{theorem}
If $A \in \mathscr{B}(H)$ is self-adjoint, then each eigenvector of $A$ is a stationary point
of the Rayleigh quotient of $A$.
\end{theorem}
\begin{proof}
If $\lambda$ is an eigenvalue of $A$ then, as $A$ is self-adjoint, $\lambda \in \mathbb{R}$. Let $v \neq 0$ satisfy
$Av=\lambda v$. We have
\[
f(v)=\frac{\inner{Av}{v}}{\inner{v}{v}}= \frac{\inner{\lambda v}{v}}{\inner{v}{v}}=\lambda.
\]
For $h \neq 0$, using that $A$ is self-adjoint and that $\lambda \in \mathbb{R}$,
\begin{eqnarray*}
\frac{|f(v+h)-f(v)-0v|}{\norm{h}}&=&\frac{1}{\norm{h}} \cdot \left| \frac{\inner{A(v+h)}{v+h}}{\inner{v+h}{v+h}} - \lambda \right|\\
&=&\frac{1}{\norm{h} \norm{v+h}^2} \left| \inner{A(v+h)}{v+h} - \lambda \inner{v+h}{v+h} \right|\\
&=&\frac{1}{\norm{h} \norm{v+h}^2} \big| \inner{Av}{v}+\inner{Av}{h}+\inner{Ah}{v}+\inner{Ah}{h}\\
&&-\lambda \inner{v}{v} - \lambda \inner{v}{h}-\lambda \inner{h}{v} - \lambda \inner{h}{h}  \big|\\
&=&\frac{1}{\norm{h} \norm{v+h}^2} \big| \inner{\lambda v}{v}+\inner{\lambda v}{h}+\inner{ h}{\lambda v}+\inner{A h}{h}\\
&&-\lambda \inner{v}{v} - \lambda \inner{v}{h}-\lambda \inner{h}{v} - \lambda \inner{h}{h}  \big|\\
&=&\frac{1}{\norm{h} \norm{v+h}^2} \left| \inner{A h}{h} - \lambda \inner{h}{h} \right|\\
&=&\frac{1}{\norm{h} \norm{v+h}^2} |\inner{Ah-\lambda h}{h}|.
\end{eqnarray*}
Therefore 
\begin{eqnarray*}
\frac{|f(v+h)-f(v)-0v|}{\norm{h}}&\leq&\frac{\norm{Ah-\lambda h} \norm{h}}{\norm{h} \norm{v+h}^2}\\
&=&\frac{\norm{(A-\lambda \id_H)h}}{\norm{v+h}^2}\\
&=&\frac{\norm{A-\lambda \id_H} \norm{h}}{\norm{v+h}^2}.
\end{eqnarray*}
As $h \to 0$ the right-hand side tends to $0$ (one of the terms tends to $0$, one doesn't depend on $h$, and the denominator
is bounded below in terms just of $v$ for sufficiently small $h$), showing that $0$ is the Fr\'echet derivative of $f$ at $v$.
\end{proof}




\end{document}