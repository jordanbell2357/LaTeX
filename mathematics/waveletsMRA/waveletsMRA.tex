\documentclass{article}
\usepackage{amsmath,amssymb,graphicx,subfig,mathrsfs,amsthm}
%\usepackage{tikz-cd}
%\usepackage{hyperref}
\newcommand{\innerL}[2]{\langle #1, #2 \rangle_{L^2}}
\newcommand{\inner}[2]{\left\langle #1, #2 \right\rangle}
\newcommand{\HSinner}[2]{\left\langle #1, #2 \right\rangle_{\ensuremath\mathrm{HS}}}
\newcommand{\tr}{\ensuremath\mathrm{tr}\,} 
\newcommand{\Span}{\ensuremath\mathrm{span}\,} 
\def\Re{\ensuremath{\mathrm{Re}}\,}
\def\Im{\ensuremath{\mathrm{Im}}\,}
\newcommand{\id}{\ensuremath\mathrm{id}} 
\newcommand{\GL}{\ensuremath\mathrm{GL}} 
\newcommand{\rank}{\ensuremath\mathrm{rank\,}} 
\newcommand{\co}{\ensuremath\mathrm{co}}
\newcommand{\ext}{\ensuremath\mathrm{ext}\,}
\newcommand{\Ind}{\ensuremath\mathrm{Ind}}
\newcommand{\cco}{\overline{\ensuremath\mathrm{co}}}
\newcommand{\point}{\ensuremath\sigma_{\mathrm{point}}} 
\newcommand{\supp}{\ensuremath\mathrm{supp}}
\newcommand{\Hom}{\ensuremath\mathrm{Hom}}
\newcommand{\norm}[1]{\left\Vert #1 \right\Vert}
\theoremstyle{definition}
\newtheorem{theorem}{Theorem}
\newtheorem{lemma}[theorem]{Lemma}
\newtheorem{proposition}[theorem]{Proposition}
\newtheorem{corollary}[theorem]{Corollary}
\theoremstyle{definition}
\newtheorem{definition}[theorem]{Definition}
\begin{document}
\title{Haar wavelets and multiresolution analysis}
\author{Jordan Bell}
\date{April 3, 2014}

\maketitle

\section{Introduction}
Let
\[
\psi(x)=\begin{cases}
0&x<0,\\
1&0 \leq x < \frac{1}{2},\\
-1&\frac{1}{2} \leq x < 1,\\
0&x \geq 1.
\end{cases}
\]
For $n,k \in \mathbb{Z}$, we define
\[
\psi_{n,k}(x)=2^{n/2} \psi(2^n x-k), \qquad x \in \mathbb{R}.
\]



$L^2(\mathbb{R})$  is a complex Hilbert space with the inner product
\[
\inner{f}{g} = \int_{\mathbb{R}} f(x) \overline{g(x)} dx.
\]

We will prove that $\psi$ satisfies   the following definition of an {\em orthonormal wavelet}.\footnote{Mark A. Pinsky, {\em Introduction to
Fourier Analysis and Wavelets}, p.~303, Definition 6.4.1.}

\begin{definition}[Orthonormal wavelet]
If $\Psi \in L^2(\mathbb{R})$, $\Psi_{n,k}(x)=2^{n/2}\Psi(2^nx-k)$, and the set $\{\Psi_{n,k}: n,k \in \mathbb{Z}\}$ is an orthonormal basis for $L^2(\mathbb{R})$,
then $\Psi$ is called an {\em orthonormal wavelet}.
\end{definition}

\begin{lemma}
$\{\psi_{n,k}: n,k \in \mathbb{Z}\}$ is an  orthonormal set in $L^2(\mathbb{R})$.
\label{orthonormal}
\end{lemma}
\begin{proof}
If $n,n',k,k' \in \mathbb{Z}$, then
\begin{eqnarray*}
\int_{\mathbb{R}} \psi_{n,k}(x) \overline{\psi_{n',k'}(x)} dx&=&\int_{\mathbb{R}} 2^{n/2} \psi(2^nx-k)2^{n'/2} \psi(2^{n'}x-k') dx\\
&=&\int_{\mathbb{R}} 2^{(n'-n)/2} \psi(x-k) \psi(2^{n'-n}x-k') dx\\
&=& 2^{(n'-n)/2} \delta_{k,k'}  \int_0^1 \psi(x) \psi(2^{(n'-n)/2}x) dx\\
&=&\delta_{k,k'} \cdot \delta_{n,n'},
\end{eqnarray*}
hence  $\{\psi_{n,k}: n,k \in \mathbb{Z}\}$ is an orthonormal set.
\end{proof}

Bessel's inequality states that if $\mathscr{E}$ is an orthonormal set in a Hilbert space $H$, then for any $f \in H$ we have
  $\sum_{e \in \mathscr{E}} |\inner{f}{e}|^2 \leq \norm{f}_2^2$, from which it follows that $\sum_{e \in \mathscr{E}} \inner{f}{e}e \in H$.
To say that a subset $\mathscr{E}$ of a Hilbert space $H$ is  an orthonormal basis is equivalent to saying that $\mathscr{E}$ is
an orthonormal set and that
\[
\id_H = \sum_{e \in \mathscr{E}} e \otimes e
\]
in the strong operator topology. In other words, for $\mathscr{E}$ to be an orthonormal basis of $H$ means that $\mathscr{E}$
is an orthonormal set and that for every
$f \in H$ we have
\[
f = \sum_{e \in \mathscr{E}} \inner{f}{e}e.
\]
From Lemma \ref{orthonormal}  and Bessel's inequality, we know that for each $f \in L^2(\mathbb{R})$,
\[
\sum_{n,k \in \mathbb{Z}} |\inner{f}{\psi_{n,k}}|^2 \leq \norm{f}_2^2, \qquad
\sum_{n,k \in \mathbb{Z}} \inner{f}{\psi_{n,k}} \psi_{n,k} \in L^2(\mathbb{R}).
\]
We have not yet  proved that $f$ is equal to the series $\sum_{n,k \in \mathbb{Z}} \inner{f}{\psi_{n,k}} \psi_{n,k}$, and this will not be accomplished until later
in this note.




\section{Coarser sigma-algebras}
For $n,k \in \mathbb{Z}$, let
\[
I_{n,k} = \left[ \frac{k}{2^n}, \frac{k+1}{2^n} \right),
\]
and let $\mathscr{F}_n$ be the $\sigma$-algebra generated by $\{I_{k,n}: k \in \mathbb{Z}\}$. 
$\mathbb{R} = \bigcup_{k \in \mathbb{Z}} I_{n,k}$, and if $k \neq k'$ then $I_{n,k} \cap I_{n,k'} = \emptyset$.
 If $n < n'$ then
\[
\mathscr{F}_n \subset \mathscr{F}_{n'} \subset \mathscr{F},
\]
where $\mathscr{F}$ is the $\sigma$-algebra of Lebesgue measurable subsets of $\mathbb{R}$.
An element  of $L^2(\mathbb{R},\mathscr{F}_n)$ is an element of $L^2(\mathbb{R},\mathscr{F})$ that is constant on each set $I_{n,k}$, $k \in \mathbb{Z}$. In other words, an element
of $L^2(\mathbb{R},\mathscr{F}_n)$ is a function $f:\mathbb{R} \to \mathbb{C}$
such that if $k \in \mathbb{Z}$ then the image $f(I_{n,k})$ is a single element of $\mathbb{R}$ and such that
\[
\norm{f}_2^2=\int_{\mathbb{R}} |f(x)|^2 dx =\sum_{k \in \mathbb{Z}} \int_{I_{n,k}} |f(x)|^2 dx= \sum_{k \in \mathbb{Z}}  \frac{1}{2^n} \cdot |f(I_{n,k})|^2 < \infty.
\]
If $n<n'$, then
\[
L^2(\mathbb{R},\mathscr{F}_n) \subset L^2(\mathbb{R},\mathscr{F}_{n'}) \subset L^2(\mathbb{R},\mathscr{F}).
\]

\section{Integral kernels}
We define
\[
\phi(x)= \begin{cases}
0&x<0,\\
1&0 \leq x < 1,\\
0&x \geq 1.
\end{cases}
\] 
For $n \in \mathbb{Z}$ we define
\[
K_n(x,y) = 2^n \sum_{k \in \mathbb{Z}} \phi(2^nx-k)\phi(2^n y-k), \qquad x, y \in \mathbb{R}.
\]
We have
\[
K_n(x,y) \in \{0,2^n\}.
\]
$K_n(x,y)=2^n$ if and only if there is some $k \in \mathbb{Z}$ such that $2^n x-k, 2^ny-k \in [0,1)$, equivalently
 there is some $k \in \mathbb{Z}$ with $2^nx,2^ny \in [k,k+1)$, which is equivalent to there being some
$k \in \mathbb{Z}$ such that
\[
x,y \in \left[\frac{k}{2^n},\frac{k+1}{2^n}\right)=I_{n,k}.
\] 

We define
\[
P_n f(x)= \int_{\mathbb{R}} K_n(x,y) f(y) dy.
\]
If $x \in \mathbb{R}$ then there is a unique $k_x \in \mathbb{Z}$ with $x \in I_{n,k_x}$, and 
\begin{equation}
P_n f(x)=2^n \int_{I_{n,k_x}}  f(y)dy.
\label{boxprojection}
\end{equation}
It is straightforward to check
that $L^2(\mathbb{R},\mathscr{F}_n)$ is a closed subspace of $L^2(\mathbb{R},\mathscr{F})$, and in the following theorem
we prove that $P_n$ is the orthogonal projection onto $L^2(\mathbb{R},\mathscr{F}_n)$.

\begin{lemma}
If $n \in \mathbb{Z}$, then $P_n$ is the orthogonal projection of $L^2(\mathbb{R},\mathscr{F})$ onto $L^2(\mathbb{R},\mathscr{F}_n)$.
\end{lemma}
\begin{proof}
For each $k \in \mathbb{Z}$, the function
$P_n f$ is constant on the interval $I_{n,k}$, and using \eqref{boxprojection} and the Cauchy-Schwarz inequality,
\begin{align*}
\norm{P_nf}_2^2&=\sum_{k \in \mathbb{Z}} \int_{I_{n,k}} |P_nf(x)|^2 dx\\
&=\sum_{k \in \mathbb{Z}} \int_{I_{n,k}} \left| 2^n \int_{I_{n,k}} f(y) dy \right|^2 dx\\
&= 2^n \sum_{k \in \mathbb{Z}} \left| \int_{I_{n,k}} f(y) dy \right|^2\\
&\leq 2^n \sum_{k \in \mathbb{Z}} \left( \int_{I_{n,k}} |f(y)|^2 dy \right) \left( \int_{I_{n,k}} dy\right)\\
&=\sum_{k \in \mathbb{Z}} \int_{I_{n,k}} |f(y)|^2 dy\\
&=\int_{\mathbb{R}} |f(y)|^2 dy.
\end{align*}
Therefore, $P_n:L^2(\mathbb{R},\mathscr{F}) \to L^2(\mathbb{R},\mathscr{F}_n)$. 
Moreover, the left-hand side of the above inequality is equal to $\norm{P_n f}_2^2$ and the right-hand side is
equal to $\norm{f}_2^2$, hence we have
$\norm{P_n f}_2 \leq \norm{f}_2$, giving $\norm{P_n} \leq 1$.


If $f \in L^2(\mathbb{R},\mathscr{F}_n)$, then
\begin{align*}
P_n f(x)&= \int_{\mathbb{R}} K_n(x,y) f(y) dy\\
&=2^n \int_{I_{n,k_x}} f(y) dy\\
&=f(I_{n,k_x})\\
&=f(x),
\end{align*}
hence if $f \in L^2(\mathbb{R},\mathscr{F}_n)$ then $P_n f=f$.
\end{proof}

For $n \in \mathbb{Z}$, we define
\[
L_n = K_{n+1}-K_n,
\]
and the following lemma gives a different expression for $L_n$.\footnote{Mark A. Pinsky, {\em Introduction to
Fourier Analysis and Wavelets}, p.~293, \S 6.3.2.}

\begin{lemma}
If $n \in \mathbb{Z}$, then
\[
L_n(x,y) = \sum_{k \in \mathbb{Z}} \psi_{n,k}(x) \psi_{n,k}(y), \qquad x, y \in \mathbb{R}.
\]
\label{Lnformula}
\end{lemma}
\begin{proof}
$\psi(2^nx-k)=1$ means that $0\leq 2^nx-k< \frac{1}{2}$, which is equivalent to 
$\frac{k}{2^n} \leq  x < \frac{k+\frac{1}{2}}{2^n}$, which is equivalent to $\frac{2k}{2^{n+1}} \leq x < \frac{2k+1}{2^{n+1}}$,
which is equivalent to $x \in I_{n+1,2k}$.
$\psi(2^nx-k)=-1$ means that $\frac{1}{2} \leq 2^nx-k < 1$, which
is equivalent to $\frac{k+\frac{1}{2}}{2^n} \leq x < \frac{k+1}{2^n}$, and this is equivalent to $x \in I_{n+1,2k+1}$.
$\psi(2^nx-k)=0$ if and only if $x \not \in  I_{n+1,2k} \cup I_{n+1,2k+1}$.
Therefore, 
\[
\psi_{n,k}(x)\psi_{n,k}(y)=
\begin{cases}
2^n&(x,y) \in I_{n+1,2k} \times I_{n+1,2k} \cup I_{n+1,2k+1} \times I_{n+1,2k+1},\\
-2^n&(x,y) \in I_{n+1,2k} \times  I_{n+1,2k+1} \cup  I_{n+1,2k+1} \times  I_{n+1,2k},\\
0&\mathrm{otherwise.}
\end{cases}
\]


If there is no $k \in \mathbb{Z}$ such that $(x,y) \in I_{n,k} \times I_{n,k}$, then $L_n(x,y)=0$. Otherwise, suppose that $k \in \mathbb{Z}$ and that
$(x,y) \in I_{n,k} \times I_{n,k}$. We have
\[
I_{n,k} = I_{n+1,2k} \cup I_{n+1,2k+1}.
\]
If $(x,y) \in I_{n+1,2k} \times I_{n+1,2k}$, then
\[
L_n(x,y)=K_{n+1}(x,y)-K_n(x,y)=2^{n+1}-2^n=2^n;
\]
if $(x,y) \in I_{n+1,2k+1} \times I_{n+1,2k+1}$, then
\[
L_n(x,y)=K_{n+1}(x,y)-K_n(x,y)=2^{n+1}-2^n=2^n;
\]
if $(x,y) \in I_{n+1,2k} \times I_{n+1,2k+1}$, then
\[
L_n(x,y)=K_{n+1}(x,y)-K_n(x,y)=0-2^n=-2^n;
\]
and if $(x,y) \in  I_{n+1,2k+1}  \times I_{n+1,2k}$, then
\[
L_n(x,y)=K_{n+1}(x,y)-K_n(x,y)=0-2^n=-2^n.
\]
It follows that 
\[
L_n(x,y)=\sum_{k \in \mathbb{Z}} \psi_{n,k}(x)\psi_{n,k}(y).
\]
\end{proof}




\section{Continuous functions}
Let $C_0(\mathbb{R})$ denote those continuous functions $f:\mathbb{R} \to \mathbb{C}$ such that if $\epsilon>0$ then there is some compact
subset $K$ of $\mathbb{R}$ such that $x \not \in K$ implies that $|f(x)|<\epsilon$.
We say that an element of $C_0(\mathbb{R})$ is a continuous function that {\em vanishes at infinity}.
Let $C_c(\mathbb{R})$ denote the set of continuous functions
$f:\mathbb{R} \to \mathbb{C}$ such that
\[
\supp(f) = \overline{\{x \in \mathbb{R}:f(x) \neq 0\}}
\]
is a compact set.

In the following lemma, we prove that the larger the intervals over which we average a continuous function  vanishing at infinity,
the smaller the supremum of the averaged function.\footnote{Mark A. Pinsky, {\em Introduction to
Fourier Analysis and Wavelets}, p.~295, Lemma 6.3.2.}

\begin{lemma} If $f \in C_0(\mathbb{R})$, then $\norm{P_n f}_\infty \to 0$ as $n \to -\infty$.
\label{negativeuniform}
\end{lemma}
\begin{proof}
If $g \in C_c(\mathbb{R})$ and
$x \in \mathbb{R}$, then
\begin{align*}
|P_n g(x)| &= \left| \int_{\mathbb{R}} K_n(x,y)g(y)dy \right|\\
&=\left| \int_{\supp(g)} K_n(x,y) g(y) dy \right|\\
&\leq \int_{\supp(g)} K_n(x,y) |g(y)| dy\\
&\leq \int_{\supp(g)} 2^n |g(y)| dy\\
&\leq 2^n\cdot  \mu(\supp(g)) \cdot \norm{g}_\infty,
\end{align*}
hence
\begin{equation}
\norm{P_n g}_\infty \leq 2^n\cdot  \mu(\supp(g)) \cdot \norm{g}_\infty.
\label{compactinequality}
\end{equation}
If $f \in C_0(\mathbb{R})$ and $\epsilon>0$ then there is some $g \in C_c(\mathbb{R})$ with $\norm{f-g}_\infty < \epsilon$.
Hence,
\[
\norm{P_nf}_\infty \leq \norm{P_n(f-g)}_\infty + \norm{P_n g}_\infty.
\]
If $x \in \mathbb{R}$,  then
\[
|P_n (f-g)(x)| =2^n \left| \int_{I_{n,k_x}} (f-g)(y) dy \right| \leq 2^n \int_{I_{n,k_x}} |(f-g)(y)| dy \leq \norm{f-g}_\infty,
\]
hence $\norm{P_n(f-g)}_\infty \leq \norm{f-g}_\infty$. Using this and \eqref{compactinequality} we obtain
\[
\norm{P_nf}_\infty \leq \norm{f-g}_\infty + 2^n\cdot  \mu(\supp(g)) \cdot \norm{g}_\infty < \epsilon+2^n\cdot  \mu(\supp(g)) \cdot \norm{g}_\infty.
\]
Hence,
\[
\limsup_{n \to -\infty} \norm{P_nf}_\infty \leq  \limsup_{n \to -\infty} \big(\epsilon+2^n\cdot  \mu(\supp(g)) \cdot \norm{g}_\infty \big)
=\epsilon.
\]
This is true for every $\epsilon>0$, so
\[
\lim_{n \to -\infty} \norm{P_n f}_\infty = 0.
\]
\end{proof}



\begin{lemma}
If $f \in L^2(\mathbb{R})$, then $\norm{P_n f}_2 \to 0$ as $n \to -\infty$.
\label{negativelimit}
\end{lemma}
\begin{proof}
If $\epsilon>0$ then there is some $g \in C_c(\mathbb{R})$ such that $\norm{f-g}_2<\epsilon$. Say $\supp(g) \subseteq [-K,K]$. If $2^m > K$, then
we have by \eqref{boxprojection} and because $\supp(g) \subseteq I_{-m,-1} \cup I_{-m,0}$,
\begin{align*}
\norm{P_{-m}g}_2^2&=\int_{\mathbb{R}} \left| 2^{-m} \int_{I_{-m,k_x}} g(y) dy \right|^2 dx\\
&=2^m \left|2^{-m} \int_{I_{-m,-1}} g(y) dy \right|^2+2^m\left|2^{-m} \int_{I_{-m,0}} g(y) dy \right|^2\\
&=2^{-m} \left| \int_{-K}^0 g(y) dy \right|^2+2^{-m} \left| \int_0^K g(y) dy \right|^2\\
&\leq 2^{-m} \mu([-K,0]) \norm{g}_2^2 +2^{-m} \mu([0,K]) \norm{g}_2^2 \\
&=2K \cdot 2^{-m} \norm{g}_2^2.
\end{align*}
Therefore,  when $2^m > K$ we have $\norm{P_{-m}g}_2 \leq 2^{-\frac{m}{2}} \sqrt{2K} \norm{g}_2$, and so, as the operator norm of $P_{-m}$ on $L^2(\mathbb{R})$ is 1,
\begin{align*}
\norm{P_{-m}f}_2& \leq \norm{P_{-m}(f-g)}_2 + \norm{P_{-m}g}_2\\
& \leq \norm{f-g}_2 + \norm{P_{-m}g}_2\\
& < \epsilon+  2^{-\frac{m}{2}} \sqrt{2K} \norm{g}_2.
\end{align*}
Thus, if $\epsilon>0$ then
\[
\limsup_{m \to \infty} \norm{P_{-m} f}_2 \leq \epsilon.
\]
This is true for all $\epsilon>0$, so we obtain
\[
\lim_{m \to \infty} \norm{P_{-m}f}_2 = 0.
\]
\end{proof}



The following lemma shows that if $f \in C_c(\mathbb{R})$, then $P_nf$ converges to $f$ in
the $L^2$ norm and in the $L^\infty$ norm as $n \to \infty$.\footnote{Mark A. Pinsky, {\em Introduction to Fourier Analysis and Wavelets}, p.~296, Lemma 6.3.3.}

\begin{lemma}
If $f \in C_c(\mathbb{R})$, then $P_nf \to f$ in the $L^2$ norm and in the $L^\infty$ norm as $n \to \infty$.
\label{positivelimit}
\end{lemma}
\begin{proof}
Suppose that $\supp(f) \subseteq [-2^M,2^M]$ for $M \geq 0$. $f$ is uniformly continuous on the compact set $[-2^M,2^M]$, thus, if $\epsilon>0$ then there is some
$\delta>0$ such that $x, y \in [-2^M,2^M]$ and $|x-y|<\delta$ imply that $|f(x)-f(y)|<\frac{\epsilon}{2^M}$.
Let $2^{-n} \leq \delta$. For each $x \in \mathbb{R}$,  there is some $k_x \in \mathbb{Z}$ such that $x \in I_{n,k_x}$ and we have
\begin{align*}
|P_n f(x)-f(x)| &= \left|  2^n \int_{I_{n,k_x}} f(y) dy - f(x) \right|\\
&=2^n \left| \int_{I_{n,k_x}} f(y)-f(x) dy \right|\\
&\leq 2^n \int_{I_{n,k_x}} |f(y)-f(x)| dy\\
&<2^n \int_{I_{n,k_x}} \frac{\epsilon}{2^M} dy\\
&=\frac{\epsilon}{2^M}.
\end{align*}
This tells us that if $2^{-n} \leq \delta$ then $\norm{P_nf -f}_\infty \leq \frac{\epsilon}{2^M}$. Therefore, if $\epsilon>0$ then for sufficiently large
$n$ we have $\norm{P_nf-f}_\infty \leq \frac{\epsilon}{2^M}$, showing that
\[
\lim_{n \to \infty} \norm{P_nf-f}_\infty=0.
\]
Furthermore, if $n \geq 0$ then
\[
\norm{P_nf-f}_2^2=\int_{\mathbb{R}} |P_nf(x)-f(x)|^2 dx
=\int_{-2^M}^{2^M} |P_nf(x)-f(x)|^2 dx
\leq 2\cdot 2^M \cdot \norm{P_nf-f}_\infty^2,
 \]
 and because $\norm{P_nf-f}_\infty \to 0$ as $n \to \infty$ we get $\norm{P_nf-f}_2 \to 0$ as $n \to \infty$.
\end{proof}

From Lemma \ref{Lnformula}, we get
\begin{align*}
(P_{n+1}-P_n)f(x)&=\int_{\mathbb{R}} K_{n+1}(x,y)f(y)dy-\int_{\mathbb{R}} K_n(x,y)f(y)dy\\
&=\int_{\mathbb{R}} L_n(x,y)f(y) dy\\
&=\int_{\mathbb{R}} \sum_{k \in \mathbb{Z}} \psi_{n,k}(x)\psi_{n,k}(y) f(y) dy\\
&=\sum_{k \in \mathbb{Z}} \inner{f}{\psi_{n,k}} \psi_{n,k}(x),
\end{align*}
thus
\begin{equation}
P_{n+1}-P_n=\sum_{k \in \mathbb{Z}} \psi_{n,k} \otimes \psi_{n,k}
\label{Pdifference}
\end{equation}
in the strong operator topology. Using \eqref{Pdifference}, we obtain for $n \geq 0$ that
\begin{align*}
P_{n+1}&=P_0 + \sum_{j=0}^n P_{j+1}-P_j\\
&=P_0+\sum_{j=0}^n \sum_{k \in \mathbb{Z}} \psi_{j,k} \otimes \psi_{j,k}
\end{align*}
in the strong operator topology. For $n<0$,
\begin{align*}
P_n&=P_0-\sum_{j=-n}^{-1} P_{j+1}-P_j\\
&=P_0 - \sum_{j=-n}^{-1} \sum_{k \in \mathbb{Z}} \psi_{j,k} \otimes \psi_{j,k}
\end{align*}
in the strong operator topology.

We have already shown in Lemma \ref{orthonormal} that $\{\psi_{n,k}:n,k \in \mathbb{Z}\}$ is an orthonormal set in $L^2(\mathbb{R})$, and we now prove
that it  is an orthonormal basis for
$L^2(\mathbb{R})$. 

\begin{theorem}
In the strong operator topology,
\[
\id_{L^2(\mathbb{R})} = \sum_{n,k \in \mathbb{Z}} \psi_{n,k} \otimes \psi_{n,k}. 
\]
\end{theorem}
\begin{proof}
Let $f \in L^2(\mathbb{R})$ and suppose $\epsilon>0$. 
By Lemma \ref{negativelimit}, there is some $M$ such that $m \geq M$ implies that 
$\norm{P_{-m} f}_2 < \frac{\epsilon}{2}$.
There is some $g \in C_c(\mathbb{R})$ satisfying $\norm{f-g}_2 < \frac{\epsilon}{6}$, and 
 by Lemma \ref{positivelimit} there is some $N$ such that $n \geq N$ implies that $\norm{P_n g-g}_2<\frac{\epsilon}{6}$.
Hence, if $n \geq N$ then
\begin{align*}
\norm{P_nf-f}_2 &\leq \norm{P_nf-P_ng}_2+\norm{P_ng-g}_2+\norm{g-f}_2\\
&\leq2\norm{f-g}_2 + \norm{P_ng-g}_2\\
&<\frac{2\epsilon}{6}+\frac{\epsilon}{6}\\
&=\frac{\epsilon}{2}.
\end{align*}
Therefore, if $m \geq M$ and $n \geq N$, then
\[
\norm{(P_n-P_{-m}-\id_{L^2(\mathbb{R}})f}_2\leq  \norm{P_nf-f}_2 + \norm{P_{-m}f}_2 < \frac{\epsilon}{2}+\frac{\epsilon}{2} = \epsilon.
\]
For $m,n>0$, we have
\begin{align*}
P_{n+1}-P_{-m}&= \sum_{j=0}^n \sum_{k \in \mathbb{Z}} \psi_{j,k} \otimes \psi_{j,k} + \sum_{j=-m}^{-1} \sum_{k \in \mathbb{Z}}
\psi_{j,k} \otimes \psi_{j,k}\\
&=\sum_{j=-m}^n \sum_{k \in \mathbb{Z}} \psi_{j,k} \otimes \psi_{j,k}
\end{align*}
in the strong operator topology.
\end{proof}


\section{Other function spaces}
Let $C_b(\mathbb{R})$ denote those continuous functions $\mathbb{R} \to \mathbb{C}$ that are bounded. We
have
\[
C_c(\mathbb{R}) \subset C_0(\mathbb{R}) \subset C_b(\mathbb{R}) \subset C(\mathbb{R}).
\]

\begin{lemma}
If $n \in \mathbb{Z}$ and $f \in C_b(\mathbb{R})$, then $\norm{P_nf}_\infty \leq \norm{f}_\infty$.
\end{lemma}
\begin{proof}
If $x \in \mathbb{R}$, then
there is a unique $k_x \in \mathbb{Z}$ with $x \in I_{n,k_x}$, and
\[
|P_nf(x)| = \left| 2^n \int_{I_{n,k_x}} f(y) dy \right| \leq 2^n \int_{I_{n,k_x}} |f(y)| dy \leq
 \norm{f}_\infty.
\]
\end{proof}



\begin{theorem}
If $f \in C_0(\mathbb{R})$, then the series $\sum_{n,k \in \mathbb{Z}} \inner{f}{\psi_{n,k}} \psi_{n,k}$ 
converges to $f$ uniformly on $\mathbb{R}$.
\end{theorem}
\begin{proof}
If $\epsilon>0$ then there is some $g \in C_c(\mathbb{R})$ with $\norm{f-g}_\infty < \frac{\epsilon}{6}$. 
By Lemma \ref{negativeuniform}, there is some $M$ such that $m \geq M$ implies that
$\norm{P_{-m} g}_\infty < \frac{\epsilon}{3}$, hence
\begin{align*}
\norm{P_{-m}f}_\infty &\leq \norm{P_{-m}f-P_{-m}g}_\infty + \norm{P_{-m}g}_\infty\\
&\leq \norm{f-g}_\infty + \norm{P_{-m}g}_\infty\\
&<\frac{\epsilon}{6}+\frac{\epsilon}{3}\\
&=\frac{\epsilon}{2}.
\end{align*}
By Lemma \ref{positivelimit}, there is some $N$ such that $n \geq N$ implies that
$\norm{P_ng-g}_\infty <\frac{\epsilon}{6}$, hence
\begin{align*}
\norm{P_nf-f}_\infty& \leq \norm{P_nf-P_ng}_\infty + \norm{P_ng-g}_\infty + \norm{g-f}_\infty\\
&\leq 2\norm{f-g}_\infty + \norm{P_ng-g}_\infty\\
&<\frac{\epsilon}{2}.
\end{align*}
Therefore, if $n \geq N$ and $m \geq M$, then
\[
\norm{P_nf-P_{-m}f-f}_\infty \leq \norm{P_nf-f}_\infty + \norm{P_{-m}f}_\infty < 
\frac{\epsilon}{2}+\frac{\epsilon}{2}=\epsilon.
\]
\end{proof}


The following theorem states that $P_n$ is an operator on $L^p(\mathbb{R})$ with operator norm $\leq 1$.\footnote{Mark A. Pinsky, {\em Introduction to Fourier Analysis and Wavelets}, p.~297, Lemma 6.3.9.} In particular, it asserts that if $f \in L^p(\mathbb{R})$ then the averaged function $P_n f$ is also an element of $L^p(\mathbb{R})$.

\begin{theorem}
If $1 \leq p < \infty$, $n \in \mathbb{Z}$, and $f \in L^p(\mathbb{R})$, then $\norm{P_nf}_p
\leq \norm{f}_p$.
\end{theorem}
\begin{proof}
Let $\frac{1}{p}+\frac{1}{q}=1$, so $q=\frac{p}{p-1}$.  (If $p=1$ then $q=\infty$.)
If $x \in \mathbb{R}$, then there is a unique  $k_x \in \mathbb{Z}$ with $x \in I_{n,k_x}$, and  using H\"older's inequality we get
\begin{align*}
|P_n f(x)|& = \left| 2^n \int_{I_{n,k_x}} f(y) dy \right| \\
&\leq 2^n \left(  \int_{I_{n,k_x}} |f(y)|^p dy \right)^{1/p} (\mu(I_{n,k_x}))^{1/q}\\
&=2^n \left(  \int_{I_{n,k_x}} |f(y)|^p dy \right)^{1/p} 2^{-n/q}.
\end{align*}
Therefore, if $k \in \mathbb{Z}$ then
\begin{align*}
\int_{I_{n,k}} |P_nf(x)|^p dx&\leq \int_{I_{n,k}}  2^{np} 2^{-np/q}  \int_{I_{n,k_x}} |f(y)|^p dydx\\
&= \int_{I_{n,k}}  2^{np} 2^{-np/q}  \int_{I_{n,k}} |f(y)|^p dydx\\
&=2^{-n}  2^{np} 2^{-np/q}\int_{I_{n,k}} |f(y)|^p dy\\
&=\int_{I_{n,k}} |f(y)|^p dy.
\end{align*}
We obtain
\begin{align*}
\norm{P_nf}_p^p&=\sum_{k \in \mathbb{Z}} \int_{I_{n,k}} |P_nf(x)|^p dx\\
&\leq\sum_{k \in \mathbb{Z}} \int_{I_{n,k}} |f(y)|^p dy\\
&=\int_{\mathbb{R}}  |f(y)|^p dy\\
&=\norm{f}_p^p,
\end{align*}
giving $\norm{P_nf}_p \leq \norm{f}_p$.
\end{proof}



\section{Multiresolution analysis}
For $a \in \mathbb{R}$, we define $m_a:\mathbb{R} \to \mathbb{R}$ by $m_a(x)=ax$,
and we define $\tau_a:\mathbb{R} \to \mathbb{R}$ by $\tau_a(x)=x-a$.


\begin{definition}[Multiresolution analysis]
A {\em multiresolution analysis of $L^2(\mathbb{R})$} is a set $\{V_n:n \in \mathbb{Z}\}$ of
closed subspaces of the Hilbert space $L^2(\mathbb{R})$ and a function $\Phi \in L^2(\mathbb{R})$ satisfying
\begin{enumerate}
\item If $n \in \mathbb{Z}$, then $f \in V_n$ if and only if $f \circ m_2 \in V_{n+1}$.
\item $V_n \subseteq V_{n+1}$.
\item $\overline{\bigcup_{n \in \mathbb{Z}} V_n}=  L^2(\mathbb{R})$.
\item $\bigcap_{n \in \mathbb{Z}} V_n=\{0\}$.
\item $\{\Phi \circ \tau_k : k \in \mathbb{Z}\}$ is an orthonormal basis for $V_0$.
\end{enumerate}
\label{MRAdef}
\end{definition}

It is straightforward to prove the following theorem using what we have established so far.


\begin{theorem}
The closed subspaces $\{L^2(\mathbb{R},\mathscr{F}_n):n \in \mathbb{Z}\}$ of $L^2(\mathbb{R})$ and the function $\phi=\chi_{[0,1)}$ is a multiresolution analysis
of $L^2(\mathbb{R})$.
\end{theorem}



The following lemma shows that if $P_n$ is the projection onto $V_n$, where $V_n$ is a closed subspace of a multiresolution analysis of $L^2(\mathbb{R})$, then
$P_n \to 0$ in the strong operator topology as $n \to -\infty$.\footnote{Mark A. Pinsky, 
 {\em Introduction to Fourier Analysis and Wavelets}, p.~313, Lemma 6.4.28.}

\begin{lemma}
If $\{V_n:n \in \mathbb{Z}\}$ and $\Phi \in L^2(\mathbb{R})$ is a multiresolution analysis of $L^2(\mathbb{R})$,
$P_n:L^2(\mathbb{R}) \to V_n$ is the orthogonal projection onto $V_n$, and $f \in L^2(\mathbb{R})$, then
\[
\lim_{n \to -\infty} P_nf=0.
\]
\label{MRAprojections}
\end{lemma}
\begin{proof}
Define $\Phi_{n,k}(x)=2^{n/2} \Phi(2^nx-k)$.
The set $\{\Phi_{0,k}: k \in \mathbb{Z}\}$ is an orthonormal basis for $V_0$, and one checks that the set $\{\Phi_{n,k}: k \in \mathbb{Z}\}$ is an orthonormal
basis for $V_n$. Therefore
\[
P_n = \sum_{k \in \mathbb{Z}} \Phi_{n,k} \otimes \Phi_{n,k}
\]
in the strong operator topology.

For $R>0$, let $f_R= f\chi_{[-R,R]}$. If $2^nR<\frac{1}{2}$, then, using the Cauchy-Schwarz inequality,
\begin{align*}
\norm{P_nf_R}_2^2&=\sum_{k \in \mathbb{Z}} |\inner{P_nf_R}{\Phi_{n,k}}|^2\\
&=\sum_{k \in \mathbb{Z}} |\inner{f_R}{\Phi_{n,k}}|^2\\
&=\sum_{k \in \mathbb{Z}} |\inner{f_R}{\chi_{[-R,R]} \Phi_{n,k}}|^2\\
&\leq \sum_{k \in \mathbb{Z}} \left( \int_{-R}^R |f_R(x)|^2 dx \right) \left( \int_{-R}^R |\Phi_{n,k}(x)|^2 dx \right)\\
&=\norm{f_R}_2^2 \sum_{k \in \mathbb{Z}}  \int_{-R}^R |\Phi_{n,k}(x)|^2 dx\\
&=\norm{f_R}_2^2 \sum_{k \in \mathbb{Z}} 2^n \int_{-R}^R |\Phi(2^nx-k)|^2 dx\\
&=\norm{f_R}_2^2 \sum_{k \in \mathbb{Z}}  \int_{-2^n R-k}^{2^nR-k} |\Phi(x)|^2 dx\\
&=\norm{f_R}_2^2 \int_{U_n} |\Phi(x)|^2 dx,
\end{align*}
where
\[
U_n = \bigcup_{k \in \mathbb{Z}} (-k-2^nR,-k+2^nR);
\]
the intervals are disjoint because $2^nR<\frac{1}{2}$. Define $F_n(x)=|\Phi(x)|^2 \chi_{U_n}(x)$. For all $x \in \mathbb{R}$
we have $|F_n(x)| \leq |\Phi(x)|^2$, and if $x \in \mathbb{R}$ then
\[
\lim_{n \to -\infty} F_n(x) \to |\Phi(x)|^2 \chi_{\mathbb{Z}}(x),
\]
where $\mathbb{Z}=\bigcap_{n \in \mathbb{Z}} U_n$. Thus by the dominated convergence theorem we get
\[
\lim_{n \to -\infty} \int_\mathbb{R} F_n(x) dx = \int_\mathbb{R} |\Phi(x)|^2 \chi_{\mathbb{Z}}(x) dx = 0,
\]
because $\mu(\mathbb{Z})=0$. Therefore,
\[
\lim_{n \to -\infty} \norm{P_n f_R}_2 =0.
\]

If $\epsilon>0$ then there is some $R$ such that $\norm{f-f_R}_2<\epsilon$. We have, because $P_n$ is an orthogonal projection,
\begin{align*}
\limsup_{n \to -\infty} \norm{P_n f}_2 &\leq \limsup_{n \to -\infty} \norm{P_nf -P_nf_R}_2 + \limsup_{n \to -\infty} \norm{P_nf_R}_2\\
&= \limsup_{n \to -\infty} \norm{P_n f- P_nf_R}_2\\
&\leq \limsup_{n \to -\infty} \norm{f-f_R}_2\\
&<\epsilon.
\end{align*}
This is true for all $\epsilon>0$, so we obtain
\[
\lim_{n \to -\infty} \norm{P_nf}_2 =0.
\]

\end{proof}

If $S_\alpha, \alpha \in I$, are subsets of a Hilbert space $H$, we denote by 
$\bigvee_{\alpha \in I} S_\alpha$ the closure of the span of $\bigcup_{\alpha \in I} S_\alpha$. If $S$ is a subset of $H$, let $S^\perp$ be the set
of all $x \in H$ such that $y \in S$ implies that $\inner{x}{y}=0$. If $S_n, n \in \mathbb{Z}$, are mutually orthogonal closed subspaces of a Hilbert space, we write
\[
\bigoplus_{n \in \mathbb{Z}} S_n = \bigvee_{n \in \mathbb{Z}} S_n.
\]

The following theorem shows a consequence of Definition \ref{MRAdef}.

\begin{theorem}
If $\{V_n: n \in \mathbb{Z}\}$ are the closed subspaces of a multiresolution analysis of $L^2(\mathbb{R})$ and $W_n=V_{n+1} \cap V_n^\perp$, then
\[
L^2(\mathbb{R}) = \bigoplus_{n \in \mathbb{Z}} W_n.
\]
\end{theorem}
\begin{proof}
Because $W_n=V_{n+1} \cap V_n^\perp$ is the intersection of two closed subspaces, it is itself a closed subspace. 
Suppose that $n < n'$, $f \in W_n, g \in W_{n'}$.  $n+1 \leq n'$, and hence $V_{n+1} \subseteq V_{n'}$. Therefore
\[
W_{n'} = V_{n'+1} \cap V_{n'}^\perp \subset V_{n'}^\perp \subseteq V_{n+1}^\perp.
\]
But $f \in W_n \subset V_{n+1}$ and $g \in W_{n'} \subset V_{n+1}^\perp$, so $\inner{f}{g}=0$. Therefore $W_n \perp W_{n'}$.


If $f \in V_n$ and $f \neq 0$, then there is a minimal $N$ such that $f \in V_N$; this minimal $N$ exists because $V_n \subseteq V_{n+1}$ and $\bigcap_{n \in \mathbb{Z}} V_n=
\{0\}$. We have
\[
V_N = V_{N-1} \oplus W_{N-1},
\]
hence $f=f_{N-1} + g_{N-1}$, with $f_{N-1} \in V_{N-1}$ and $g_{N-1} \in W_{N-1}$. Likewise,
\[
V_{N-1} = V_{N-2} \oplus W_{N-2},
\]
hence $f_{N-1} = f_{N-2}+g_{N-2}$, with $f_{N-2} \in V_{N-2}$ and $g_{N-2} \in W_{N-2}$. In this way, for any $M \geq 0$ we obtain
\[
f =f_{N-M}+ \sum_{m=1}^M g_{N-m},
\]
where $f_{N-M} \in V_{N-M}$ and $g_{N-m} \in W_{N-m}$. Check that $f_{N-M}$ is the orthogonal projection of $f$ onto $V_{N-M}$. It thus
 follows from Lemma \ref{MRAprojections} that $f_{N-M} \to 0$ as $M \to \infty$. Thus, for any $\epsilon>0$ there is some $M$ with
$\norm{f_{N-M}}_2<\epsilon$ and $f \in f_{N-M} + \bigoplus_{m=1}^M W_{N-m}$. Therefore, if $f \in \bigcup_{n \in \mathbb{Z}} V_n$ then
there is some $g \in \bigoplus_{n \in \mathbb{Z}} W_n$
satisfying $\norm{f-g}_2 < \infty$. Thus
\[
\overline{\bigcup_{n \in \mathbb{Z}} V_n} \subseteq \bigoplus_{n \in \mathbb{Z}} W_n,
\]
and so
\[
L^2(\mathbb{R}) = \bigoplus_{n \in \mathbb{Z}} W_n.
\]
\end{proof}


\section{The unit interval}
$L^2([0,1))$ is a Hilbert space with the inner product
\[
\inner{f}{g} = \int_0^1 f(x) \overline{g(x)} dx.
\]
If $n \geq 0$, then $I_{n,0}=\left[0,\frac{1}{2^n}\right)$ and $I_{n,2^n-1}=\left[1-\frac{1}{2^n},1\right)$, and we have
\[
[0,1)=\bigcup_{k=0}^{2^n-1} I_{n,k}.
\]


Let $n \geq 0$,
let $\mathscr{G}_n$ be the $\sigma$-algebra generated by $\{I_{n,k}: 0 \leq k \leq 2^n-1\}$, and let
$\mathscr{G}$ be the $\sigma$-algebra of Lebesgue measurable subsets of $[0,1)$. If $n<n'$, then
\[
\mathscr{G}_n \subset \mathscr{G}_{n'} \subset \mathscr{G}.
\]
An element of $L^2([0,1),\mathscr{G}_n)$ is an element of $L^2([0,1),\mathscr{G})$ that is constant on each
set $I_{n,k}, 0 \leq k \leq 2^n-1$. Equivalently, an element of $L^2([0,1),\mathscr{G}_n)$ is a function
$f:[0,1) \to \mathbb{C}$ that is constant on each set $I_{n,k}, 0 \leq k \leq 2^n-1$; because $[0,1)$ is a union
of finitely many $I_{n,k}$, any such function will be an element of $L^2([0,1),\mathscr{G})$.
It is apparent that
\[
L^2([0,1),\mathscr{G}_n) \subset L^2([0,1),\mathscr{G}_{n'}) \subset L^2([0,1),\mathscr{G}).
\]
We check that
$L^2([0,1),\mathscr{G}_n)$ is a complex vector space of dimension $2^n$.

$I_{n,k}=I_{n+1,2k} \cup I_{n+1,2k+1}$. If $x \in I_{n+1,2k}$, then $\frac{2k}{2^{n+1}} \leq x < \frac{2k+1}{2^{n+1}}$, so
$\frac{k}{2^n} \leq x< \frac{k}{2^n}+\frac{1}{2^{n+1}}$, hence
$0 \leq 2^nx-k< \frac{1}{2}$. If $x \in I_{n+1,2k+1}$, then $\frac{2k+1}{2^{n+1}} \leq x < \frac{2k+2}{2^{n+1}}$, hence
$\frac{k}{2^n}+\frac{1}{2^{n+1}} \leq x < \frac{k+1}{2^n}$, and so $\frac{1}{2} \leq 2^nx-k < 1$.
Thus, if $x \in I_{n+1,2k}$ then
\[
\psi_{n,k}(x)=2^{n/2}\psi(2^nx-k) = 2^{n/2}
\]
and  if $x \in I_{n+1,2k+1}$ then
\[
\psi_{n,k}(x)=2^{n/2}\psi(2^nx-k)=-2^{n/2}.
\]
Otherwise $x \not \in I_{n,k}$, for which $\psi_{n,k}(x)=0$.
It follows that $\psi_{n,k} \in L^2([0,1),\mathscr{G}_{n+1})$.


\begin{theorem}
If
\[
\mathscr{B}_0=\{\chi_{[0,1)}\}
\]
and, for $n \geq 0$,
\[
\mathscr{B}_{n+1}=\{\psi_{n,k}: 0 \leq k \leq 2^n-1\},
\]
then
\[
\bigcup_{n=0}^N \mathscr{B}_n
\]
is an orthonormal basis of $L^2([0,1),\mathscr{G}_N)$.
\label{orthonormalGN}
\end{theorem}
\begin{proof}
It follows from Lemma \ref{orthonormal} that $\bigcup_{n=1}^N \mathscr{B}_n$ is orthonormal
in  $L^2([0,1))$, as it is a subset of an orthonormal set.
If $0 \leq n \leq N$ then $\mathscr{B}_n \subset L^2([0,1),\mathscr{G}_N)$, hence
$\bigcup_{n=1}^N \mathscr{B}_n$ is orthonormal  in $L^2([0,1),\mathscr{G}_N)$.
If $0 <n \leq N$ and $0 \leq k \leq 2^{n-1}-1$, then $\psi_{n-1,k} \in \mathscr{B}_n$ and
\begin{align*}
\inner{\psi_{n-1,k}}{\chi_{[0,1)}}&=\int_0^1 \psi_{n-1,k}(x) \overline{\chi_{[0,1)}(x)}dx\\
&=\int_0^1 \psi_{n-1,k}(x)dx\\
&=\int_{I_{n,2k}} \psi_{n-1,k}(x) dx + \int_{I_{n,2k+1}} \psi_{n-1,k}(x) dx\\
&=\int_{I_{n,2k}} 2^{(n-1)/2} dx + \int_{I_{n,2k+1}} -2^{(n-1)/2} dx\\
&=0.
\end{align*} 
Therefore, $\bigcup_{n=0}^N \mathscr{B}_n$ is orthonormal in $L^2([0,1),\mathscr{G}_N)$.


$|\mathscr{B}_0|=1$, and if $n \geq 1$ then $|\mathscr{B}_n|=2^{n-1}$. Therefore
the number of elements of $\bigcup_{n=0}^N \mathscr{B}_n$ is
\[
1+\sum_{n=1}^N 2^{n-1} = 1+\sum_{n=0}^{N-1} 2^n = 2^N.
\]
As $\dim L^2([0,1),\mathscr{G}_N)=2^N$, the orthonormal set $\bigcup_{n=0}^N \mathscr{B}_n$ is an orthonormal basis for $L^2([0,1),\mathscr{G}_N)$.
\end{proof}


By Theorem \ref{orthonormalGN}, if $N \geq 0$ then $\bigcup_{n=0}^N \mathscr{B}_n$ is an orthonormal set in $L^2([0,1))$.
Hence
\[
\mathscr{B}=\bigcup_{n=0}^\infty \mathscr{B}_n
\]
 is an orthonormal set in $L^2([0,1))$: if $f,g \in \mathscr{B}$ then there is some $N$ with $f,g \in
 \bigcup_{n=0}^N \mathscr{B}_n$, which is an orthonormal set.
The following theorem shows that $\mathscr{B}$ is an orthonormal basis for the Hilbert space $L^2([0,1))$.\footnote{John K. Hunter and Bruno
Nachtergaele, {\em Applied Analysis}, p.~177, Lemma 7.13.}


\begin{theorem}
$\mathscr{B}$ is an orthonormal basis for $L^2([0,1))$.
\end{theorem}
\begin{proof}
If $f \in L^2([0,1))$ and $\epsilon>0$ then there is some $g \in C([0,1])$ with
$\norm{f-g}_2<\frac{\epsilon}{2}$. $g$ is uniformly continuous on the compact set $[0,1]$, so there is some $\delta>0$ such that
$|x-y|<\delta$ implies that $|g(x)-g(y)|<\frac{\epsilon}{2}$. Let $2^{-n} \leq \delta$, and define $h:[0,1) \to \mathbb{C}$ by
\[
h(x)=\sum_{k=0}^{2^n-1} g\left(\frac{k}{2^n}\right) \chi_{I_{n,k}}(x).
\]
If $x \in [0,1)$ then there is a unique $k_x, 0 \leq k_x \leq 2^n-1$, with $x \in I_{n,k_x}$, and for this $k_x$ we have $\left| x - \frac{k_x}{2^n} \right| < 2^{-n} \leq
\delta$, and hence 
\[
|g(x)-h(x)| = \left| g(x)-g\left(\frac{k_x}{2^n}\right) \right|<\frac{\epsilon}{2}.
\]
Therefore $\norm{g-h}_\infty \leq \frac{\epsilon}{2}$.

We have $h \in L^2([0,1),\mathscr{G}_n)$, and 
\[
\norm{f-h}_2 \leq  \norm{f-g}_2 + \norm{g-h}_2 <\frac{\epsilon}{2} + \norm{g-h}_\infty \leq \epsilon.
\]
We have shown that if $f \in L^2([0,1))$ and $\epsilon>0$ then there is some $n$ and some $h \in L^2([0,1),\mathscr{G}_n)$ with
$\norm{f-h}_2 < \epsilon$. This tells us that $\bigcup_{n=0}^\infty L^2([0,1),\mathscr{G}_n)$ is a dense subset of $L^2([0,1))$.
Since $\mathscr{B}$ is orthonormal 
and $\Span \mathscr{B} = \bigcup_{n=0}^\infty L^2([0,1),\mathscr{G}_n)$, $\mathscr{B}$ is an orthonormal basis for $L^2([0,1))$. 
\end{proof}






\section{References}
Useful references on wavelets and multiresolution analysis are Mark A. Pinsky, {\em Introduction to Fourier Analysis and Wavelets}; P. Wojtaszczyk, {\em A Mathematical Introduction to Wavelets}; 
Yves Meyer, {\em Wavelets and Operators};  Eugenio Hern\'andez and Guido Weiss, {\em A First Course on Wavelets}.
 
\end{document}