\documentclass{article}
\usepackage{amsmath,amssymb,mathrsfs,amsthm}
%\usepackage{tikz-cd}
%\usepackage{hyperref}
\newcommand{\inner}[2]{\left\langle #1, #2 \right\rangle}
\newcommand{\tr}{\ensuremath\mathrm{tr}\,} 
\newcommand{\Span}{\ensuremath\mathrm{span}} 
\def\Re{\ensuremath{\mathrm{Re}}\,}
\def\Im{\ensuremath{\mathrm{Im}}\,}
\newcommand{\Lip}{\ensuremath\mathrm{Lip}} 
\newcommand{\id}{\ensuremath\mathrm{id}} 
\newcommand{\var}{\ensuremath\mathrm{var}} 
\newcommand{\Hilb}{\ensuremath\mathrm{Hilb}} 
\newcommand{\GL}{\ensuremath\mathrm{GL}} 
\newcommand{\Ran}{\ensuremath\mathrm{Ran}} 
\newcommand{\diam}{\ensuremath\mathrm{diam}} 
\newcommand{\sgn}{\ensuremath\mathrm{sgn}\,} 
\newcommand{\lcm}{\ensuremath\mathrm{lcm}} 
\newcommand{\supp}{\ensuremath\mathrm{supp}\,}
\newcommand{\spr}{\ensuremath\mathrm{spr}}
\newcommand{\dom}{\ensuremath\mathrm{dom}\,}
\newcommand{\upto}{\nearrow}
\newcommand{\downto}{\searrow}
\newcommand{\norm}[1]{\left\Vert #1 \right\Vert}
\theoremstyle{definition}
\newtheorem{theorem}{Theorem}
\newtheorem{lemma}[theorem]{Lemma}
\newtheorem{proposition}[theorem]{Proposition}
\newtheorem{corollary}[theorem]{Corollary}
\theoremstyle{definition}
\newtheorem{definition}[theorem]{Definition}
\newtheorem{example}[theorem]{Example}
\begin{document}
\title{Chebyshev polynomials}
\author{Jordan Bell}
\date{December 7, 2016}

\maketitle

\section{Chebyshev polynomials of first kind}
On the one hand, 
\begin{align*}
(\cos \theta + i \sin \theta)^n&=\sum_{0 \leq \nu \leq n} i^\nu \binom{n}{\nu} \cos^{n-\nu} \theta \sin^\nu \theta\\
&=\sum_{0 \leq 2k \leq n} (-1)^k \binom{n}{2k}\cos^{n-2k}(\theta) \sin^{2k}(\theta)\\
&+i\sum_{0 \leq 2k+1 \leq n}  (-1)^k \binom{n}{2k+1}\cos^{n-2k-1}(\theta) \sin^{2k+1}(\theta).
\end{align*}
On the other hand,
\[
(\cos \theta+i\sin \theta)^n = (e^{i\theta})^n = e^{in\theta} = \cos n\theta + i\sin n\theta.
\]
Therefore
\begin{align*}
\cos n\theta&=\sum_{0 \leq k \leq n/2} (-1)^k \binom{n}{2k}\cos^{n-2k}(\theta) \sin^{2k}(\theta)\\
&=\sum_{0 \leq k \leq n/2} (-1)^k \binom{n}{2k}\cos^{n-2k}(\theta) (1-\cos^2 \theta)^k\\
&=\sum_{0 \leq k \leq n/2} \binom{n}{2k}\cos^{n-2k}(\theta) (\cos^2 \theta-1)^k\\
&=\sum_{0 \leq k \leq n/2} \binom{n}{2k}\cos^{n-2k}(\theta) \sum_{0 \leq j \leq k} \binom{k}{j} \cos^{2k-2j}(\theta)  (-1)^j\\
&=\sum_{0 \leq j \leq n/2} (-1)^j \cos^{n-2j}(\theta) \sum_{j \leq k \leq n/2} \binom{n}{2k} \binom{k}{j}.
\end{align*}
Now,
\begin{align*}
\sum_{j \leq k \leq n/2} \binom{n}{2k} \binom{k}{j}&=
2^{n-2j-1} \binom{n-j}{j} \frac{n}{n-j}.
\end{align*}
Hence
\[
\cos n\theta = \sum_{0 \leq j \leq n/2} (-1)^j \cos^{n-2j}(\theta)2^{n-2j-1} \binom{n-j}{j} \frac{n}{n-j}.
\]

For $z \in \mathbb{C}$ let 
\begin{equation}
T_n(z) =  \sum_{0 \leq j \leq n/2} (-1)^j 2^{n-2j-1} \binom{n-j}{j} \frac{n}{n-j} z^{n-2j}.
\label{Tn}
\end{equation}
Note
\[
T_n(z)[z^n] = 2^{n-1} z^n.
\]

\begin{theorem}
\[
T_n(\cos \theta) = \cos(n\theta)
\]
and
\[
T_m \circ T_n = T_{mn}.
\]
\label{Tncos}
\end{theorem}
\begin{proof}
For $\theta \in \mathbb{R}$,
\[
T_n(\cos \theta) = \cos(n\theta).
\]
Then
\[
T_m(T_n(\cos \theta)) = T_m( \cos (n\theta))
=\cos (mn\theta)
=T_{mn}(\theta).
\]
That is, for $z \in [-1,1]$ we have
$T_m(T_n(z)) = T_{mn}(z)$. Then by analytic continuation it follows that
this is true for all $z$.
\end{proof}

\begin{theorem}
\[
T_n(z)+T_{n-2}(z)=2zT_{n-1}(z).
\]
\end{theorem}
\begin{proof}
Using $\cos(\alpha+\beta)=\cos \alpha \cos \beta - \sin \alpha \sin \beta$, 
\[
\cos(n\theta) = \cos(\theta+(n-1)\theta)
=\cos \theta \cos((n-1)\theta)-\sin\theta\sin((n-1)\theta)
\]
and
\[
\cos((n-2)\theta) = \cos(-\theta+(n-1)\theta)
=\cos \theta \cos((n-1)\theta)+\sin\theta\sin((n-1)\theta).
\]
Then
\[
\cos(n\theta)+\cos((n-2)\theta)=2 \cos \theta \cos((n-1)\theta).
\]
Therefore
\begin{align*}
T_n(\cos \theta)+T_{n-2}(\cos\theta)&=\cos(n\theta)
+\cos((n-2)\theta)\\
&=2 \cos \theta \cos((n-1)\theta)\\
&=2 \cos \theta \cdot T_{n-1}(\cos\theta).
\end{align*}
That is, for $z \in [-1,1]$,
\[
T_n(z)+T_{n-2}(z)=2zT_{n-1}(z),
\]
and by analytic continuation this is true for all $z \in \mathbb{C}$.
\end{proof}



\section{Chebyshev polynomials of second kind}
Define
\begin{equation}
nU_{n-1}(z) = T_n'(z).
\label{Un}
\end{equation}


\begin{theorem}
\[
U_{n-1}(\cos \theta) = \frac{\sin(n\theta)}{\sin \theta}
\]
and
\[
(1-z^2)T_n''(z) = n U_n(z) - n(n+1) T_n(z).
\]
\label{D2T}
\end{theorem}
\begin{proof}
On the one hand,
\begin{align*}
(T_n(\cos \theta))'&=-\sin\theta \cdot T_n'(\cos \theta).
\end{align*}
On the other hand,
\[
(T_n(\cos \theta))'
=(\cos (n\theta))'
=-n\sin(n\theta).
\]
Hence
\[
T_n'(\cos \theta) = n \frac{\sin (n\theta)}{\sin \theta},
\qquad U_{n-1}(\cos \theta) = \frac{\sin(n\theta)}{\sin \theta}.
\]
Now,
\[
(T_n'(\cos \theta))' =
-\sin \theta\cdot T_n''(\cos \theta)
\]
and
\begin{align*}
(T_n'(\cos \theta))' &= n \frac{n\cos(n\theta)\sin\theta - \sin(n\theta)\cos \theta}{\sin^2 \theta}\\
&=-n\frac{\cos(n\theta)\sin \theta+\sin(n\theta)\cos \theta}{\sin^2\theta}
+n(n+1) \frac{\cos(n\theta)\sin\theta}{\sin^2\theta}\\
&=-n\frac{\sin((n+1)\theta)}{\sin^2\theta}+n(n+1) \frac{\cos(n\theta)}{\sin\theta}\\
&=-n \frac{U_n(\cos \theta)}{\sin \theta} + n(n+1) \frac{T_n(\cos \theta)}{\sin \theta}.
\end{align*}
Hence
\[
T_n''(\cos \theta) = n \frac{U_n(\cos \theta)}{\sin^2 \theta} - n(n+1) \frac{T_n(\cos \theta)}{\sin^2 \theta}
\]
and then
\[
T_n''(\cos \theta) 
=n \frac{U_n(\cos \theta)}{1-\cos^2\theta} - n(n+1) \frac{T_n(\cos \theta)}{1-\cos^2\theta}.
\]
By analytic continuation,
\[
(1-z^2)T_n''(z) = n U_n(z) - n(n+1) T_n(z).
\]
\end{proof}

\begin{theorem}
\[
T_{n+1}(z) = z T_n(z) - (1-z^2)U_{n-1}(z).
\]
\label{Tnformula}
\end{theorem}
\begin{proof}
\begin{align*}
T_{n+1}(\cos \theta)&=\cos(n\theta+\theta)\\
&=\cos(n\theta)\cos\theta - \sin(n\theta)\sin\theta\\
&=T_n(\cos \theta) \cos \theta - U_{n-1}(\cos \theta)\sin^2\theta\\
&=T_n(\cos \theta) \cos \theta - U_{n-1}(\cos \theta)(1-\cos^2\theta).
\end{align*}
Therefore by analytic continuation,
\[
T_{n+1}(z) = z T_n(z) - (1-z^2)U_{n-1}(z).
\]
\end{proof}

\begin{theorem}
\[
U_n(z)=T_n(z)+zU_{n-1}(z).
\]
\label{Unformula}
\end{theorem}
\begin{proof}
\begin{align*}
U_n(\cos\theta)&=\frac{\sin (n\theta+\theta)}{\sin \theta}\\
&=\frac{\cos(n\theta)\sin\theta + \cos \theta \sin(n\theta)}{\sin \theta}\\
&=T_n(\cos \theta)+\cos \theta \cdot U_{n-1}(\cos \theta).
\end{align*}
Therefore by analytic continuation,
\[
U_n(z)=T_n(z)+zU_{n-1}(z).
\]
\end{proof}

\begin{theorem}
\[
U_n(z)=2zU_{n-1}(z)+U_{n-2}(z).
\]
\end{theorem}
\begin{proof}
Using Theorem \ref{Tnformula} and Theorem \ref{Unformula},
\begin{align*}
U_n(z) &= T_n(z)+zU_{n-1}(z)\\
&=zT_{n-1}(z) - (1-z^2)U_{n-2}(z)+zU_{n-1}(z)\\
&=z \bigg[U_{n-1}(z)-zU_{n-2}(z)\bigg] - (1-z^2)U_{n-2}(z) + zU_{n-1}(z)\\
&=2zU_{n-1}(z)+U_{n-2}(z).
\end{align*}
\end{proof}

\begin{theorem}
\[
(1-z^2)T_n''(z)-zT_n'(z)+n^2T_n(z)=0.
\]
\end{theorem}
\begin{proof}
Using Theorem \ref{D2T},
and Theorem \ref{Unformula},
\[
\begin{split}
&(1-z^2)T_n''(z)-zT_n'(z)+n^2T_n(z)\\
=&n U_n(z) - n(n+1)T_n(z)
-nzU_{n-1}(z)+n^2T_n(z)\\
=&n(T_n(z)+zU_{n-1}(z))-n(n+1)T_n(z)-nzU_{n-1}(z)+n^2T_n(z)\\
=&0.
\end{split}
\]
\end{proof}

From Theorem \ref{Tncos}
\[
T_n(1)=T_n(\cos 0) = \cos(n \cdot 0)=1.
\]
From Theorem \ref{D2T},
\[
T_n'(1) = nU_{n-1}(1) = n^2.
\]
Thus, $T_n$ is the unique solution of the initial value problem
\[
(1-x^2)y''(x)-xy'(x)+n^2y(x)=0,\qquad
y(1)=1,y'(1)=n^2.
\]


\begin{theorem}
\[
T_n(z)^2- (z^2-1)U_{n-1}(z)^2=1.
\]
\label{pell}
\end{theorem}
\begin{proof}
Using Theorem \ref{Tncos} and Theorem \ref{D2T}, for $z=\cos \theta$,
\begin{align*}
T_n(z)^2-(z^2-1)U_{n-1}(z)^2&=
T_n(\cos \theta)^2 + (\sin^2\theta) U_{n-1}(\cos \theta)^2\\
&=\cos^2(n\theta)+(\sin^2 \theta) \frac{\sin^2(n\theta)}{\sin^2\theta}\\
&=\cos^2(n\theta)+\sin^2(n\theta)\\
&=1.
\end{align*}
By analytic continuation, this is true for all $z$. 
\end{proof}







\section{Inner products}
For $0 \leq \theta \leq \pi$ let $y_n(\theta) = \cos(n\theta)$. 
\[
y_n''+n^2y_n=0,\qquad y_n'(0)=0,y_n'(\pi)=0.
\]


\begin{theorem}
\[
\int_{-1}^1 \frac{T_m(x)T_n(x)}{\sqrt{1-x^2}} dx=
\int_0^\pi y_my_nd\theta = \frac{\pi}{2} \cdot \delta_{m,n}.
\]
\end{theorem}
\begin{proof}
Let $W=y_my_n'-y_ny_m'$. 
We calculate
\begin{align*}
W'&=y_m'y_n'+y_my_n''-y_n'y_m'-y_ny_m''
y_my_n''-y_ny_m''\\
&=y_my_n''-y_ny_m''\\
&=y_m(-n^2y_n) - y_n(-m^2y_m)\\
&=(m^2-n^2)y_my_n.
\end{align*}
Using $W(0)=0$ and $W(\pi)=0$,
\[
\int_0^\pi W'(\theta) d\theta = W(\pi)-W(0) = 0.
\]
Then
\[
\int_0^\pi (m^2-n^2)y_my_n d\theta = 0.
\]
Doing the substitution $\phi=n\theta$,
\begin{align*}
\int_0^\pi y_n^2 d\theta &= \int_0^\pi \cos^2(n\theta) d\theta\\
&=\int_0^\pi \frac{1+\cos(2n\theta)}{2} d\theta\\
&=\frac{\pi}{2}.
\end{align*}
Therefore
\[
\int_0^\pi y_my_nd\theta = \frac{\pi}{2} \cdot \delta_{m,n}.
\]

For $0 \leq \theta \leq \pi$, 
$\sqrt{1-\cos^2 \theta}=\sin \theta$. Then
doing the substitution $x=\cos \theta$, $dx=-\sin \theta d\theta$,
\begin{align*}
\int_0^\pi y_my_nd\theta&=\int_0^\pi \cos(m\theta) \cos(n\theta) d\theta\\
&=\int_0^\pi \cos(m\theta) \cos(n\theta) \frac{-\sin \theta d\theta}{-\sin \theta}\\
&=\int_0^\pi \frac{\cos(m\theta) \cos(n\theta)}{-\sqrt{1-\cos^2\theta}} (-\sin \theta) d\theta\\
&=\int_1^{-1} \frac{T_m(x)T_n(x)}{-\sqrt{1-x^2}} dx\\
&=\int_{-1}^1 \frac{T_m(x)T_n(x)}{\sqrt{1-x^2}} dx.
\end{align*}
\end{proof}







\end{document}