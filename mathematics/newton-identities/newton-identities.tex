\documentclass{article}
\usepackage{amssymb,latexsym,amsmath,amsthm,mathrsfs,amsthm}
\usepackage{graphicx}
%\usepackage{setspace}
\theoremstyle{definition}
\newtheorem{theorem}{Theorem}
\newtheorem{remark}[theorem]{Remark}
\newtheorem{conjecture}[theorem]{Conjecture}
\newtheorem{construction}[theorem]{Construction}
\DeclareMathOperator{\Sin}{sin.}
%\doublespacing
\begin{document}
\title{Newton's identities and the pentagonal number theorem}
\author{Jordan Bell}
\date{April 3, 2014}
\maketitle


\section{Introduction}
The pentagonal number theorem is the identity
\[
\prod_{n=1}^\infty (1-x^n)=\sum_{n=-\infty}^\infty (-1)^n x^{\frac{n(3n-1)}{2}}.
\]
Euler discovered the pentagonal number theorem in 1740, and finally proved
the theorem in 1750 \cite{bell}. Euler's proof is the following.
Let
\[
S_N=\sum_{n=1}^\infty x^{N(n-1)}(1-x^N)\cdots(1-x^{n+N-1}).
\]
Euler uses the identity $\prod_{n=1}^\infty (1-a_n)=
1-a_1-\sum_{n=2}^\infty a_n(1-a_1)\cdots(1-a_{n-1})$ to get
$\prod_{n=1}^\infty (1-x^n)=1-x-\sum_{n=2}^\infty x^n(1-x)\cdots(1-x^{n-1})$
and so $\prod_{n=1}^\infty (1-x^n)=1-x-x^2S_1$.
Euler proves that $S_N=1-x^{2N+1}-x^{3N+2}S_{N+1}$, and then iterates this
to obtain
the pentagonal number theorem.

In \cite{E542},
Euler uses Newton's identities together with the pentagonal number theorem
to prove identities for divergent series.
Euler says that all the roots of unity of each power will be roots
of the equation
\[
0=1-x^1-x^2+x^5+x^7-x^{12}-x^{15}+x^{22}+x^{26}-\textrm{etc.},
\]
and so if we denote the roots by $\alpha,\beta,\gamma,\delta$, etc. then
\[
\frac{1}{\alpha}+\frac{1}{\beta}+\frac{1}{\gamma}+\frac{1}{\delta}+\textrm{etc.}=1.
\]
By applying Newton's identities, Euler gets
\[
\frac{1}{\alpha^2}+\frac{1}{\beta^2}+\frac{1}{\gamma^2}+
\frac{1}{\delta^2}+\textrm{etc.}=3,
\]
and likewise formulas for the sums of the higher powers of the roots.
Rademacher \cite{MR0262045}
clarifies these statements.

Euler uses Newton's identities
in his discovery of the values of $\zeta(2n)$.
Generally, Euler uses Newton's identities to relate expressions
for the same thing as sums and products,
such as the representations of Bessel functions as infinite
series and infinite products (cf. \cite[pp. 497--503]{watson}).

We denote by $\sigma(n)$ the sum of the positive divisors of $n$.
In 1747, Euler discovered the following recurrence relation
for $\sigma(n)$ \cite[pp. 345--356]{bell}.
Let $\omega_j=\frac{j(3j-1)}{2}$, the $j$th pentagonal number.
For $n \neq \frac{m(3m-1)}{2}$,
\[
\sigma(n)=\sum_{j=1}^\infty (-1)^{j-1}(\sigma(n-\omega_j)+\sigma(n-\omega_{-j})),
\]
and for $n=\frac{m(3m-1)}{2}$,
\[
\sigma(n)=(-1)^{m-1}n+\sum_{j=1}^\infty (-1)^{j-1}(\sigma(n-\omega_j)+
\sigma(n-\omega_{-j})).
\]

In this note we give a proof of Euler's pentagonal number theorem using
Newton's identities and Euler's recurrence relation for the sum of divisors function
$\sigma$. This proof draws attention to the connection between the product
$\prod (1-x^n)$ and roots of unity.

\section{Results}
Let $\prod_{k=1}^n (1+X_k T)=\sum_{k=0}^n s_k T^k$.
For $d \geq 1$, let $p_d=\sum_{k=1}^n X_k^d$.
Newton's identities \cite[IV.65]{MR643362} are
\begin{equation}
\label{newton}
p_d=\sum_{k=1}^{d-1} (-1)^{k-1} s_k p_{d-k} +(-1)^{d+1} ds_d,
\quad d \geq 1.
\end{equation}

Now we shall present a proof of the pentagonal number theorem
using Newton's identities and Euler's recurrence relation for $\sigma(n)$.
Let $e_k(j)=e^{\frac{2\pi ij}{k}}$. Then
\[
1-t^k=\prod_{j=1}^k (1-e_k(j)t).
\]
Let $N(n)=\frac{n(n+1)}{2}$.

Define $s_{k,n}$ by
\[
\prod_{k=1}^n \prod_{j=1}^k (1-e_k(j)t)
=\sum_{k=0}^{N(n)} (-1)^k s_{k,n} t^k.
\]

On the one hand,
\[
\prod_{k=1}^{n+1}(1-t^k)=(1-t^{n+1})\prod_{k=1}^n (1-t^k)
=(1-t^{n+1})\sum_{k=0}^{N(n)} (-1)^k s_{k,n} t^k.
\]
On the other hand,
\[
\prod_{k=1}^{n+1}(1-t^k)=\sum_{k=0}^{N(n+1)} (-1)^k s_{k,n+1} t^k.
\]
Hence for $k \leq n$, $s_{k,n}=s_{k,n+1}$.

Define $s_k$ by
\[
(-1)^k s_k=\begin{cases}0,&k \neq \frac{j(3j-1)}{2},\\
(-1)^j,&k=\frac{j(3j-1)}{2}.
\end{cases}
\]
We want to show that $s_{n,n}=s_n$ for all $n \geq 1$.

First, $1-t=s_{0,1}-s_{1,1}t$ and so $s_{1,1}=1$. By its definition,
$s_1=1$. Assume now tha
$s_{n,n}=s_n$, and thus that
$s_{k,n}=s_n$ for all $k \leq n$. 

By Newton's identities \eqref{newton},
\[
p_{n+1,n+1}=(-1)^{n+2}(n+1)s_{n+1,n+1}
+\sum_{k=1}^n (-1)^{k-1} s_{k,n+1}p_{n+1-k,n+1}.
\]

We note the following.
\begin{eqnarray*}
p_{d,n}&=&\sum_{k=1}^n \sum_{j=1}^k e_k(j)^d\\
&=&\sum_{k=1}^n \begin{cases}
k,&k|d,\\
0,&\textrm{otherwise}.
\end{cases}
\end{eqnarray*}
Hence $p_{d,n}=\sigma(d)$ if $d \leq n$.
Euler studies exponential sums in \cite{E447}.

We use the fact that $p_{k,n}=\sigma(k)$ if $k \leq n$ to get
\[
\sigma(n+1)=
(-1)^{n+2}(n+1)s_{n+1,n+1}
+\sum_{k=1}^n (-1)^{k-1} s_{k,n+1}\sigma(n+1-k).
\]

But furthermore, by assumption $s_{k,n+1}=s_k$ for $k \leq n$, so
\[
\sigma(n+1)=(-1)^{n+2}(n+1)s_{n+1,n+1}
+\sum_{k=1}^n (-1)^{k-1} s_k\sigma(n+1-k).
\]

Either $n+1 \neq \frac{m(3m-1)}{2}$ or $n+1=\frac{m(3m-1)}{2}$.
In the first case, the recurrence relation for $\sigma(n+1)$ is
\[
\sigma(n+1)=\sum_{j=1}^\infty (-1)^{j-1}(\sigma(n+1-\omega_j)
+\sigma(n+1-\omega_{-j})).
\]
This implies that $s_{n+1,n+1}=0$. But $s_{n+1}=0$ also, so $s_{n+1,n+1}=s_{n+1}$.

In the second case, the recurrence relation for $\sigma(n+1)$ is
\[
\sigma(n+1)=(-1)^{m-1}(n+1)
+\sum_{j=1}^\infty (-1)^{j-1}(\sigma(n+1-\omega_j)+\sigma(n+1-\omega_{-j})).
\]
This implies that $(-1)^{m-1}(n+1)=(-1)^{n+2}(n+1)s_{n+1,n+1}$, and hence
$(-1)^{n+1}s_{n+1,n+1}=(-1)^m$. But $(-1)^{n+1}s_{n+1}=(-1)^m$, so
$s_{n+1,n+1}=s_{n+1}$.

This completes the proof by induction that $s_{n,n}=s_n$ for all $n \geq 1$.


\bibliographystyle{plain}
\bibliography{newton-identities}
\end{document}
