\documentclass{article}
\usepackage{amsmath,amssymb,mathrsfs,amsthm}
%\usepackage{tikz-cd}
%\usepackage{hyperref}
\newcommand{\inner}[2]{\left\langle #1, #2 \right\rangle}
\newcommand{\tr}{\ensuremath\mathrm{tr}\,} 
\newcommand{\Span}{\ensuremath\mathrm{span}} 
\def\Re{\ensuremath{\mathrm{Re}}\,}
\def\Im{\ensuremath{\mathrm{Im}}\,}
\newcommand{\id}{\ensuremath\mathrm{id}} 
\newcommand{\var}{\ensuremath\mathrm{var}} 
\newcommand{\Lip}{\ensuremath\mathrm{Lip}} 
\newcommand{\GL}{\ensuremath\mathrm{GL}} 
\newcommand{\diam}{\ensuremath\mathrm{diam}} 
\newcommand{\sgn}{\ensuremath\mathrm{sgn}\,} 
\newcommand{\lcm}{\ensuremath\mathrm{lcm}} 
\newcommand{\supp}{\ensuremath\mathrm{supp}\,}
\newcommand{\ran}{\ensuremath\mathrm{ran}\,}
\newcommand{\dom}{\ensuremath\mathrm{dom}\,}
\newcommand{\upto}{\nearrow}
\newcommand{\downto}{\searrow}
\newcommand{\norm}[1]{\left\Vert #1 \right\Vert}
\newcommand{\HS}[1]{\left\Vert #1 \right\Vert_\ensuremath\mathrm{HS}}
\newtheorem{theorem}{Theorem}
\newtheorem{lemma}[theorem]{Lemma}
\newtheorem{proposition}[theorem]{Proposition}
\newtheorem{corollary}[theorem]{Corollary}
\theoremstyle{definition}
\newtheorem{definition}[theorem]{Definition}
\newtheorem{example}[theorem]{Example}
\begin{document}
\title{Orthonormal bases for product measures}
\author{Jordan Bell}
\date{October 22, 2015}

\maketitle

\section{Measure and integration theory}
Let $\mathscr{B}$ be the Borel $\sigma$-algebra of $\mathbb{R}$, and let
$\overline{\mathscr{B}}$ be the Borel $\sigma$-algebra of $[-\infty,\infty]=\mathbb{R} \cup \{-\infty,\infty\}$:
the elements of $\overline{\mathscr{B}}$ are those subsets of $\overline{\mathbb{R}}$ of
the form
$B,B \cup \{-\infty\},B \cup \{\infty\}, B \cup \{-\infty,\infty\}$, with $B \in \mathscr{B}$. 

Let $(X,\mathscr{A},\mu)$ be a measure space.
It is a fact that if $f_n$ is a sequence of $\mathscr{A} \to \overline{\mathscr{B}}$ measurable functions then
$\sup_n f_n$ and $\inf_n f_n$ are $\mathscr{A} \to \overline{\mathscr{B}}$ measurable, and thus  if
$f_n$ is a sequence of $\mathscr{A} \to \overline{\mathscr{B}}$ measurable functions
that converge pointwise to a function $f:X \to \overline{\mathbb{R}}$, then
$f$ is $\mathscr{A}\to \overline{\mathscr{B}}$ measurable.\footnote{Heinz Bauer,
{\em Measure and Integration Theory}, p.~52, Corollary 9.7.} If $f_1,\ldots,f_n$ are $\mathscr{A} \to \overline{\mathscr{B}}$ measurable,
then so are $f_1 \vee \cdots \vee  f_n$ and $f_1 \wedge \cdots \wedge f_n$, and
a function $f:X \to \overline{\mathbb{R}}$ is $\mathscr{A} \to \overline{\mathscr{B}}$ measurable
if and only if both
$f^+=f \vee 0$ and $f^-=-(f \wedge 0)$ are $\mathscr{A} \to \overline{\mathscr{B}}$ measurable. In particular,
if $f$ is $\mathscr{A} \to \overline{\mathscr{B}}$ measurable then so is $|f|=f^++f^-$. 

A \textbf{simple function} is a function $f:X \to \mathbb{R}$ that is 
$\mathscr{A} \to \mathscr{B}$ measurable and whose range is finite. 
Let $E=E(\mathscr{A})$ be the collection of nonnegative simple functions. It is straightforward to prove that
\[
u,v \in E,\; \alpha \geq 0\quad \Rightarrow \quad \alpha u, \; u+v, \; u\cdot v, \; u\vee v, \; u\wedge v \in E.
\]
Define $I_\mu:E \to [0,\infty]$ by
\[
I_\mu u =  \sum_{i=1}^n a_i \mu(A_i),
\]
where $u$ has
 range $\{a_1,\ldots,a_n\}$ and 
$A_i = u^{-1}(a_i)$.
One proves that $I_\mu:E \to [0,\infty]$ is positive homogeneous, additive, and order preserving.\footnote{Heinz Bauer,
{\em Measure and Integration Theory}, pp.~55--56, \S 10.} 

It is a fact\footnote{Heinz Bauer,
{\em Measure and Integration Theory}, p.~57, Theorem 11.1.} that if $u_n$ is a  nondecreasing sequence in $E$ and $u \in E$ then
\[
u \leq \sup_n u_n \quad \Rightarrow  \quad I_\mu u \leq \sup_n I_\mu u_n.
\]
It follows that if $u_n$ and $v_n$ are sequences 
in $E$ then
\begin{equation}
\sup_n u_n = \sup_n v_n \quad \Rightarrow \quad
\sup_n I_\mu u_n = \sup_n I_\mu v_n.
\label{sups}
\end{equation}

Define $E^*=E^*(\mathscr{A})$ to be the set of all functions $f:X \to [0,\infty]$ for which there is 
a nondecreasing sequence $u_n$ in $E$ satisfying $\sup_n u_n = f$, in other words, there is a sequence
$u_n$ in $E$ satisfying $u_n \uparrow f$. 
From \eqref{sups}, for $f \in E^*$ and sequences $u_n,v_n \in E$ with $\sup_n u_n = f$ and
$\sup_n v_n=f$, it holds that $\sup_n I_\mu u_n = \sup_n I_\mu v_n$. Also,
if $u \in E$ then $u_n=u$ is a nondecreasing sequence in $E$ with
$u=\sup_n u_n$, so $u \in E^*$.
Then it makes sense to extend $I_\mu$ from $E \to [0,\infty]$ to $E^* \to [0,\infty]$ by defining
$I_\mu f = \sup_n I_\mu u_n$. 
One proves\footnote{Heinz Bauer,
{\em Measure and Integration Theory}, pp.~58--59, \S 11.} that
\[
f,g \in E^*,\; \alpha \geq 0 \quad \Rightarrow\quad \alpha f, \; f+g,\; f\cdot g,\; f \vee g, \; f\wedge g \in E^*
\]
and that $I_\mu:E^* \to [0,\infty]$ is positive homogeneous, additive, and order preserving. 





The \textbf{monotone convergence theorem}\footnote{Heinz Bauer,
{\em Measure and Integration Theory}, p.~59, Theorem 11.4.}
states that if $f_n$ is a sequence in $E^*$ then $\sup_n f_n \in E^*$ and
\[
I_\mu \left( \sup_n f_n \right) = \sup_n I_\mu f_n.
\]
We now prove a characterization of $E^*$.\footnote{Heinz Bauer,
{\em Measure and Integration Theory}, p.~61, Theorem 11.6.}

\begin{theorem}
$E^*$ is equal to the set of   functions $X \to [0,\infty]$ that are $\mathscr{A} \to \overline{\mathscr{B}}$ measurable.
\label{Estar}
\end{theorem}
\begin{proof}
If $f \in E^*$, then  there is a sequence $u_n$ in $E$ with $u_n \uparrow f$. Because each $u_n$ is measurable $\mathscr{A} \to \overline{\mathscr{B}}$,
so is $f$.

Now suppose that $f:X \to [0,\infty]$ is  $\mathscr{A} \to \overline{\mathscr{B}}$ measurable.
For $n \geq 1$ and $0 \leq i \leq n2^n-1$ let
\[
A_{i,n} = \{f \geq i2^{-n}\} \cap \{f<(i+1)2^{-n}\} = \{i2^{-n} \leq f < (i+1)2^{-n}\},
\]
and for $i=n2^n$ let
\[
A_{i,n} = \{f \geq n\}.
\]
Because $f$ is $\mathscr{A} \to \overline{\mathscr{B}}$ measurable, the sets $A_{i,n}$ belong to $\mathscr{A}$. For each
$n$, the sets $A_{0,n},\ldots A_{n2^n-1,n},A_{n2^n,n}$ are pairwise disjoint and their union is equal to $X$. 
It is apparent that
\begin{equation}
A_{i,n} = A_{2i,n+1} \cup A_{2i+1,n+1},\qquad 0 \leq i \leq n2^n-1.
\label{A2i}
\end{equation}
Define
\[
u_n = \sum_{i=0}^{n2^n} i2^{-n} 1_{A_{i,n}},
\]
which belongs to $E$. For $x \in X$, either $f(x)=\infty$ or $0 \leq f(x) < \infty$. In the first case,
$u_n(x)=n$ for all $n \geq 1$. In the second case,
$u_n(x) \leq f(x) < u_n(x)+2^{-n}$ for all $n>f(x)$. Therefore
$u_n(x) \uparrow f(x)$ as $n \to \infty$, and because this is true for each $x \in X$, this means
$u_n \uparrow f$ and so $f \in E^*$.
\end{proof}


So far we have defined $I_\mu:E^* \to [0,\infty]$. 
Suppose that $f:X \to \overline{\mathbb{R}}$ is $\mathscr{A} \to \overline{\mathscr{B}}$
measurable.
Then $f^+,f^-:X \to [0,\infty]$ are  $\mathscr{A} \to \overline{\mathscr{B}}$
measurable so by Theorem \ref{Estar}, 
$f^+,f^- \in E^*$. Then  
$I_\mu f^+, I_\mu f^- \in [0,\infty]$.
We say that a function $f:X \to \overline{\mathbb{R}}$ is \textbf{$\mu$-integrable} if it is $\mathscr{A} \to \overline{\mathscr{B}}$
measurable and 
$I_\mu f^+<\infty$ and $I_\mu f^-<\infty$. 
One checks that a function $f:X \to \overline{\mathbb{R}}$ is $\mu$-integrable if and only if it
is $\mathscr{A} \to \overline{\mathscr{B}}$ measurable and $I_\mu |f| < \infty$.
If $f:X \to \overline{\mathbb{R}}$ is $\mu$-integrable, we now define $I_\mu f \in \mathbb{R}$ by
\[
I_\mu f = I_\mu f^+ - I_\mu f^-.
\]

For example, if  $\mu(X)<\infty$ and   $S$ is a subset of $X$ that does not belong to $\mathscr{A}$, define $f:X \to \mathbb{R}$
by $f = 1_S - 1_{X \setminus S}$. Then  $f^+=1_S$ and $f^-=1_{X \setminus S}$, and thus $f$ is not $\mathscr{A} \to \overline{\mathscr{B}}$
measurable, so it is not $\mu$-integrable. But $|f|=1$ belongs to $E$, and $I_\mu |f| = \mu(X)<\infty$ by hypothesis, showing
that $|f|$ is $\mu$-integrable while $f$ is not. 

One proves that if $f,g:X \to \overline{\mathbb{R}}$ are $\mu$-integrable and $\alpha \in \mathbb{R}$ then 
$\alpha f$ is $\mu$-integrable and
\[
I_\mu (\alpha f) = \alpha I_\mu f,
\]
if $f+g$ is defined on all $X$ then $f+g$ is $\mu$-integrable and 
\[
I_\mu(f+g) = I_\mu f+ I_\mu g,
\]
and $f \vee g, f \wedge g$ are $\mu$-integrable.\footnote{Heinz Bauer,
{\em Measure and Integration Theory}, p.~65, Theorem 12.3.}
Furthermore, $I_\mu$ is order preserving.

Let $f:X \to \mathbb{C}$ be a function and write $f=u+iv$. One proves that $f$ is Borel measurable (i.e.
$\mathscr{A} \to \mathscr{B}_\mathbb{C}$ measurable), if and only if $u$ and $v$ are measurable $\mathscr{A} \to \mathscr{B}$. 
We define $f$ to be $\mu$-integrable if both $u$ and $v$ are $\mu$-integrable, and define 
\[
I_\mu f = I_\mu u + i I_\mu v.
\]







\section{ℒ²}
Let $(X,\mathscr{A},\mu)$ be a measure space and
for $1 \leq p < \infty$ let $\mathscr{L}^p(\mu)$ be the collection
of Borel measurable functions $f:X \to \mathbb{C}$ such that $|f|^p$ is $\mu$-integrable. 
For  complex $a,b$, because
$x \mapsto x^p$ is convex we have by Jensen's inequality
\[
\left| \frac{a+b}{2} \right|^p \leq \left( \frac{1}{2}|a| + \frac{1}{2}|b| \right)^p
\leq \frac{1}{2} |a|^p + \frac{1}{2}|b|^p=\frac{1}{2}(|a|^p+|b|^p),
\]
so $|a+b|^p \leq 2^{p-1} (|a|^p+|b|^p)$. Thus if $f,g \in \mathscr{L}^p(\mu)$ then
\[
|f+g|^p \leq 2^{p-1}(|f|^p+|g|^p),
\]
 which implies that $\mathscr{L}^p(\mu)$ is a linear space. 

For Borel measurable $f:X \to \mathbb{C}$ define
\[
\norm{f}_{L^p} = \left( \int_X |f|^p d\mu \right)^{1/p}.
\]
For $f,g \in \mathscr{L}^p(\mu)$, by H\"older's inequality, with $\frac{1}{p}+\frac{1}{p'}=1$ (for which $p'=\frac{p}{p-1}$),
\begin{align*}
\norm{f+g}_{L^p}^p& \leq \int_X |f| |f+g|^{p-1} d\mu
+\int_X |g| |f+g|^{p-1} d\mu\\
&\leq \norm{f}_{L^p} \norm{|f+g|^{p-1}}_{L^{p'}} + \norm{g}_{L^p} \norm{|f+g|^{p-1}}_{L^{p'}}\\
&= \norm{f}_{L^p} \norm{f+g}_{L^p}^{p-1} + \norm{g}_{L^p} \norm{f+g}_{L^p}^{p-1},
\end{align*}
which implies that $\norm{f+g}_{L^p} \leq \norm{f}_{L^p} + \norm{g}_{L^p}$, and hence $\norm{\cdot}_{L^p}$ is
a seminorm on $\mathscr{L}^p(\mu)$. 

Let $\mathscr{N}^p(\mu)$ be the set of those $f \in \mathscr{L}^p(\mu)$ such that $\norm{f}_{L^p}=0$. 
$\mathscr{N}^p(\mu)$ is a linear subspace of $\mathscr{L}^p(\mu)$, and we define
\[
L^p(\mu)=\mathscr{L}^p(\mu)/\mathscr{N}^p(\mu) = \{f+\mathscr{N}^p(\mu): f \in \mathscr{L}^p(\mu)\}.
\]
$L^p(\mu)$ is a normed linear space with the norm $\norm{\cdot}_{L^p}$.



It is a fact that if $V$ is a normed linear space then $V$ is complete if and only if
each absolutely convergent series in $V$ converges in $V$.
Suppose that $f_k$ is a sequence in $\mathscr{L}^p(\mu)$ with $\sum_{k=1}^\infty \norm{f}_{L^p} < \infty$.
For $n \geq 1$ let $g_n(x) = \left( \sum_{k=1}^n |f_k(x)| \right)^p$ and 
define $g:X \to [0,\infty]$ by
\[
g(x) = \left( \sum_{k=1}^\infty |f_k(x)| \right)^p = \lim_{n \to \infty} g_n(x),
\]
which is $\mathscr{A} \to \overline{\mathscr{B}}$ measurable, being the pointwise limit of a sequence of functions each
of which is $\mathscr{A} \to \overline{\mathscr{B}}$ measurable.
Because $g_1 \leq g_2 \leq \cdots$, 
by the monotone convergence theorem,
\[
\int_X g d\mu = \lim_{n \to \infty} \int_X g_n d\mu.
\]
But
\[
\left( \int_X g_n d\mu \right)^{1/p} = \norm{\sum_{k=1}^n |f_k|}_{L^p} \leq 
\sum_{k=1}^n \norm{f_k}_{L^p} \leq \sum_{k=1}^\infty \norm{f_k}_{L^p},
\]
which implies that 
$\int_X g d\mu < \infty$, meaning that
$g:X \to [0,\infty]$ is integrable. The fact that $g$ is integrable implies $\mu(E)=0$, where
$E=\{x \in X: g(x)=\infty\} \in \mathscr{A}$. For $x \in X \setminus E$, $\sum_{k=1}^\infty |f_k(x)| < \infty$ and because
$\mathbb{C}$ is complete this implies that $\sum_{k=1}^\infty f_k(x) \in \mathbb{C}$, and so it makes sense to define 
 $f:X \to \mathbb{C}$ by
\[
f(x) = 1_{X \setminus E}(x) \sum_{k=1}^\infty f_k(x),
\]
which is Borel measurable. Furthermore, $|f|^p \leq g$, and because $g$ is integrable this implies that
$f \in \mathscr{L}^p(\mu)$. 
For $x \in X \setminus E$,
\[
\lim_{n \to \infty} \left| \sum_{k=1}^n f_k(x) - f(x) \right|^p = 0
\]
and
\[
\left| \sum_{k=1}^n f_k(x)-f(x) \right|^p \leq g(x),
\]
so by the dominated convergence theorem,\footnote{Heinz Bauer, {\em Measure and Integration Theory},
p.~83, Theorem 15.6.}
\[
\lim_{n \to \infty} \int_X \left| \sum_{k=1}^n f_k(x)-f(x) \right|^p d\mu = 0.
\]
Because $x \mapsto x^{1/p}$ is continuous this implies
\[
\lim_{n \to \infty} \norm{ \sum_{k=1}^n f_k - f}_{L^p} = 0.
\]
Hence, if $f_k$ is a sequence in $L^p(\mu)$ such that $\sum_{k=1}^\infty \norm{f_k}_{L^p}<\infty$ then there is
some $f \in L^p(\mu)$ such that $\sum_{k=1}^n f_k \to f$ in the norm $\norm{\cdot}_{L^p}$. This implies that $L^p(\mu)$ is a Banach space.


We say that the $\sigma$-algebra $\mathscr{A}$ is \textbf{countably generated} if there is a countable subset $\mathscr{C}$ of $\mathscr{A}$
such that $\mathscr{A}=\sigma(\mathscr{C})$ and we say that a topological space is \textbf{separable} if there exists a countable dense subset of it.
It can be proved that
if $\mathscr{A}$ is countably generated and $\mu$ is $\sigma$-finite, 
then for $1 \leq p < \infty$ there is a countable collection of simple functions that is dense in 
$L^p(\mu)$, showing that $L^p(\mu)$ is separable.\footnote{Donald L. Cohn,
{\em Measure Theory}, second ed., p.~102, Proposition 3.4.5.}

\begin{theorem}
Let $(X,\mathscr{A},\mu)$ be a measure space and let $1 \leq p < \infty$. $L^p(\mu)$ with the norm $\norm{\cdot}_{L^p}$ is a Banach space, and if $\mathscr{A}$ is countably generated
and $\mu$ is $\sigma$-finite then $L^p(\mu)$ is separable. 
\end{theorem}

For $f,g \in \mathscr{L}^2(\mu)$, let
\[
\inner{f}{g}_{L^2(\mu)} = \int_X f \cdot \overline{g} d\mu.
\]
This is an inner product on $L^2(\mu)$, and thus $L^2(\mu)$ is a Hilbert space. 


\section{Product measures}
Let $(X_1,\mathscr{A}_1,\mu_1)$ and $(X_1,\mathscr{A}_1,\mu_1)$ be measure spaces and let
$\mathscr{A}_1 \otimes \mathscr{A}_2$ be the product $\sigma$-algebra.
For $Q \subset X_1 \times X_2$, write
\[
Q_{x_1} = \{x_2 \in X_2: (x_1,x_2) \in Q\},
\qquad Q_{x_2} = \{x_1 \in X_1: (x_1,x_2) \in Q\}.
\]
One proves that if $\mu_1$ and $\mu_2$ are $\sigma$-finite, then
for each $Q \in \mathscr{A}_1 \otimes \mathscr{A}_2$ the function
$x_1 \mapsto \mu_2(Q_{x_1})$ is $\mathscr{A}_1 \to \overline{\mathscr{B}}$ measurable
and the function $x_2 \mapsto \mu_1(Q_{x_2})$ is $\mathscr{A}_2 \to \overline{\mathscr{B}}$ measurable.\footnote{Heinz Bauer,
{\em Measure and Integration Theory}, p.~135, Lemma 23.2.}
If $\mu_1$ and $\mu_2$ are $\sigma$-finite, one proves\footnote{Heinz Bauer,
{\em Measure and Integration Theory}, p.~136, Theorem 23.3.}  that
there is a unique measure 
$\mu:\mathscr{A}_1 \otimes \mathscr{A}_2 \to [0,\infty]$ that satisfies
\[
\mu(A_1 \times A_2) = \mu_1(A_1) \mu_2(A_2),\qquad A_1 \in \mathscr{A}_1, A_2 \in \mathscr{A}_2.
\]
The measure $\mu$ satisfies
\[
\mu(Q) = \int_{X_1} \mu_2(Q_{x_1}) d\mu_1(x_1) = \int_{X_2} \mu_1(Q_{x_2}) d\mu_2(x_2)
\]
for $Q \in \mathscr{A}_1 \otimes \mathscr{A}_2$,
and is itself $\sigma$-finite.
We write $\mu=\mu_1 \otimes \mu_2$, and call $\mu$ the \textbf{product measure of $\mu_1$ and $\mu_2$}.

Let $X'$ be a set and let $f:X_1 \times X_2 \to X'$ be a function. For $x_1 \in X_1$, define
$f_{x_1}:X_2 \to X'$ by
\[
f_{x_1}(x_2) = f(x_1,x_2),\qquad x_2 \in X_2
\]
and for $x_2 \in X_2$, define $f_{x_2}:X_1 \to X'$ by
\[
f_{x_2}(x_1) = f(x_1,x_2),\qquad x_1 \in X_1.
\]
For $Q \subset X_1 \times X_2$,
\[
(1_Q)_{x_1} = 1_{Q_{x_1}},\qquad (1_Q)_{x_2}=1_{Q_{x_2}}.
\]
It is straightforward to prove that if $(X',\mathscr{A}')$ is a measurable space and
$f:(X_1 \times X_2,\mathscr{A}_1 \otimes \mathscr{A}_2) \to (X',\mathscr{A}')$ is measurable,
then for each $x_1 \in X_1$ the function $f_{x_1}:X_2 \to X'$ is measurable
$\mathscr{A}_2 \to \mathscr{A}'$ and for each $x_2 \in X_2$ the function
$f_{x_2}:X_1 \to X'$ is measurable
$\mathscr{A}_1 \to \mathscr{A}'$.\footnote{Heinz Bauer,
{\em Measure and Integration Theory}, p.~138, Lemma 23.5.} 

\textbf{Tonelli's theorem}\footnote{Heinz Bauer,
{\em Measure and Integration Theory}, p.~138, Theorem 23.6.}  states that
if $(X_1,\mathscr{A}_1,\mu_1)$ and $(X_1,\mathscr{A}_1,\mu_1)$ are $\sigma$-finite measure spaces and
$f:X_1 \times X_2 \to [0,\infty]$ is $\mathscr{A}_1 \otimes \mathscr{A}_2 \to \overline{\mathscr{B}}$ measurable, then
the functions
\[
x_2 \mapsto \int_{X_1} f_{x_2} d\mu_1,\qquad x_1 \mapsto \int_{X_2} f_{x_1} d\mu_2
\]
are $\mathscr{A}_2 \to \overline{\mathscr{B}}$ measurable
and $\mathscr{A}_1 \to \overline{\mathscr{B}}$ measurable respectively, and
\begin{equation}
\begin{split}
&\int_{X_1 \times X_2} f d(\mu_1 \otimes \mu_2)\\
=& \int_{X_2} \left( \int_{X_1} f_{x_2} d\mu_1 \right) d\mu_2(x_2)\\
=&\int_{X_1} \left( \int_{X_2} f_{x_1} d\mu_2 \right) d\mu_1(x_1).
\end{split}
\label{tonelli}
\end{equation}

\textbf{Fubini's theorem}\footnote{Heinz Bauer,
{\em Measure and Integration Theory}, p.~139, Corollary 23.7.} states that if
$(X_1,\mathscr{A}_1,\mu_1)$ and $(X_2,\mathscr{A}_2,\mu_2)$ are $\sigma$-finite measure spaces
and $f:X_1 \times X_2 \to \overline{\mathbb{R}}$ is $\mu_1 \otimes \mu_2$-integrable then
there is some $A_1 \in \mathscr{A}_1$ with $\mu_1(A_1)=0$ such that
for $x_1 \in X_1 \setminus A_1$ the function $f_{x_1}:X_2 \to \overline{\mathbb{R}}$ is $\mu_2$-integrable,
and there is some $A_2 \in \mathscr{A}_2$ with $\mu_2(A_2)=0$ such that for $x_2 \in X_2 \setminus A_2$ the function
$f_{x_2}:X_1 \to \overline{\mathbb{R}}$ is $\mu_1$-integrable. Furthermore, define
$F_1:X_1 \to \mathbb{R}$ by $F_1(x_1) = \int_{X_2} f_{x_1} d\mu_2$ for $x_1 \in X_1 \setminus A_1$ and $F_1(x_1)=0$ for $x_1 \in A_1$,
and define $F_2:X_2 \to \mathbb{R}$ by $F_2(x_2) = \int_{X_1} f_{x_2} d\mu_1$ for $x_2 \in X_2 \setminus A_2$ and $F_2(x_2)=0$ for $x_2 \in A_2$.
The functions $F_1$ and $F_2$ are $\mu_1$-integrable and $\mu_2$-integrable respectively, and 
\[
\int_{X_1 \times X_2} f d(\mu_1 \otimes \mu_2) = \int_{X_1} F_1 d\mu_1 = \int_{X_2} F_2 d\mu_2.
\]




Suppose that $(X_1,\mathscr{A}_1,\mu_1)$ and $(X_2,\mathscr{A}_2,\mu_2)$ are 
$\sigma$-finite measure spaces.
For $e:X_1 \to \mathbb{C}$ and $f:X_2 \to \mathbb{C}$, define $e \otimes f:X_1 \times X_2 \to \mathbb{C}$ by
\[
(e \otimes f)(x_1,x_2) = e(x_1) f(x_2),
\]
 which is Borel measurable $X_1 \times X_2 \to \mathbb{C}$ if $e$
and $f$ are Borel measurable.
If $e \in \mathscr{L}^2(\mu_1)$ and $f \in \mathscr{L}^2(\mu_2)$, then by
Tonelli's theorem $e \otimes f:X_1 \times X_2 \to \mathbb{C}$ belongs to
$\mathscr{L}^2(\mu_1 \otimes \mu_2)$. 
For $e,e' \in \mathscr{L}^2(\mu_1)$ and $f,f' \in \mathscr{L}^2(\mu_2)$, by Fubini's theorem,
\[
\begin{split}
&\inner{e \otimes f}{e' \otimes f'}_{L^2(\mu_1 \otimes \mu_2)}\\
=&\int_{X_1 \times X_2} e(x_1) f(x_2) \overline{e'(x_1) f'(x_2)} d(\mu_1 \otimes \mu_2)(x_1,x_2)\\
=&\int_{X_2} \left( \int_{X_1} e(x_1) \overline{e'(x_1)} d\mu_1(x_1) \right)  f(x_2) \overline{f'(x_2)}   d\mu_2(x_2)\\
=&\inner{e}{e'}_{L^2(\mu_1)} \cdot \inner{f}{f'}_{L^2(\mu_2)}.
\end{split}
\]
Therefore, if 
$E \subset \mathscr{L}^2(\mu_1)$ is an orthonormal set in $L^2(\mu_1)$
and $F \subset \mathscr{L}^2(\mu_2)$ is an orthonormal set in $L^2(\mu_2)$, then
$\{e \otimes f: e \in E, f \in F\} \subset \mathscr{L}^2(\mu_1 \otimes \mu_2)$ is an orthonormal set
in $L^2(\mu_1 \otimes \mu_2)$. 

\begin{theorem}
Let  $(X_1,\mathscr{A}_1,\mu_1)$ and $(X_2,\mathscr{A}_2,\mu_2)$ be
$\sigma$-finite measure spaces and suppose that
$L^2(\mu_1)$ and $L^2(\mu_2)$ are separable. If
$E \subset \mathscr{L}^2(\mu_1)$ is an orthonormal basis for $L^2(\mu_1)$ and $F \subset \mathscr{L}^2(\mu_2)$ is an orthonormal basis for $L^2(\mu_2)$,
then  $\Phi=\{e \otimes f: e \in E, f \in F\} \subset \mathscr{L}^2(\mu_1 \otimes \mu_2)$ is an orthonormal basis for $L^2(\mu_1 \otimes \mu_2)$. 
\end{theorem}
\begin{proof}
To show that $\Phi$ is an orthonormal basis for $L^2(\mu_1 \otimes \mu_2)$ it suffices to prove that if
$h \in \mathscr{L}^2(\mu_1 \otimes \mu_2)$ belongs to the orthogonal  complement  of $\Phi$ then
$h \in \mathscr{N}^2(\mu_1 \otimes \mu_2)$. 
Thus, suppose that $h \in \mathscr{L}^2(\mu_1 \otimes \mu_2)$ and that 
$\inner{h}{e \otimes f}_{L^2(\mu_1 \otimes \mu_2)}=0$ for all $e \in E, f \in F$. Using Fubini's theorem,
\[
\int_{X_1} e(x_1) \left( \int_{X_2} h_{x_1}(x_2)  f(x_2) d\mu_2(x_2) \right) d\mu_1(x_1)=0.
\]
Because this is true for all $e \in E$ and $E$ is dense in $L^2(\mu_1)$, it follows that there is some
$A_f \in \mathscr{A}_1$ with $\mu_1(A_f)=0$ such that 
$\int_{X_2} h_{x_1} f d\mu_2=0$ for $x_1 \not \in A_f$.  
Let $A_1=\bigcup_{f \in F} A_f$, for which $\mu_1(A_1)=0$. If $x_1 \not \in A_1$ then 
$\int_{X_2} h_{x_1} f d\mu_2=0$ for all $f \in F$, and because $F$ is dense in $L^2(\mu_2)$ this implies that
$h_{x_1}=0$ $\mu_2$-almost everywhere. Then 
\begin{align*}
\int_{X_1 \times X_2} |h|^2 d(\mu_1 \otimes \mu_2) &= 
\int_{X_1} \left( \int_{X_2} |h_{x_1}|^2 d\mu_2 \right) d\mu_1(x_1)\\
&=\int_{X_1 \setminus A_1} \left( \int_{X_2} |h_{x_1}|^2 d\mu_2 \right) d\mu_1(x_1)\\
&=0,
\end{align*}
which implies that $h=0$ $\mu_1 \otimes \mu_2$-almost everywhere.
\end{proof}





\end{document}