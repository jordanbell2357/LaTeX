\documentclass{article}
\usepackage{amsmath,amssymb,graphicx,subfig,mathrsfs,amsthm}
%\usepackage{tikz-cd}
%\usepackage{hyperref}
\newcommand{\inner}[2]{\left\langle #1, #2 \right\rangle}
\newcommand{\tr}{\ensuremath\mathrm{tr}\,} 
\newcommand{\Span}{\ensuremath\mathrm{span}} 
\def\Re{\ensuremath{\mathrm{Re}}\,}
\def\Im{\ensuremath{\mathrm{Im}}\,}
\newcommand{\id}{\ensuremath\mathrm{id}} 
\newcommand{\rank}{\ensuremath\mathrm{rank\,}} 
\newcommand{\diam}{\ensuremath\mathrm{diam}} 
\newcommand{\osc}{\ensuremath\mathrm{osc}} 
\newcommand{\co}{\ensuremath\mathrm{co}\,} 
\newcommand{\cco}{\ensuremath\overline{\mathrm{co}}\,}
\newcommand{\supp}{\ensuremath\mathrm{supp}\,}
\newcommand{\ext}{\ensuremath\mathrm{ext}\,}
\newcommand{\ba}{\ensuremath\mathrm{ba}\,}
\newcommand{\cl}{\ensuremath\mathrm{cl}\,}
\newcommand{\dom}{\ensuremath\mathrm{dom}\,}
\newcommand{\Cyl}{\ensuremath\mathrm{Cyl}\,}
\newcommand{\extreals}{\overline{\mathbb{R}}}
\newcommand{\upto}{\nearrow}
\newcommand{\downto}{\searrow}
\newcommand{\norm}[1]{\left\Vert #1 \right\Vert}
\newtheorem{theorem}{Theorem}
\newtheorem{lemma}[theorem]{Lemma}
\newtheorem{proposition}[theorem]{Proposition}
\newtheorem{corollary}[theorem]{Corollary}
\theoremstyle{definition}
\newtheorem{definition}[theorem]{Definition}
\newtheorem{example}[theorem]{Example}
\begin{document}
\title{Oscillatory integrals}
\author{Jordan Bell}
\date{August 4, 2014}

\maketitle

\section{Oscillatory integrals}
Suppose that $\Phi \in C^\infty(\mathbb{R}^d)$, $\psi \in \mathscr{D}(\mathbb{R}^d)$, and that $\Phi$ is real-valued. Define
$I:(0,\infty) \to \mathbb{C}$ 
by
\[
I(\lambda) = \int_{\mathbb{R}^d} e^{i\lambda \Phi(x)} \psi(x) dx, \qquad \lambda>0.
\]
We call $\Phi$ a \textbf{phase} and $\psi$ an \textbf{amplitude}, and $I(\lambda)$ an \textbf{oscillatory integral}.

The following proof follows Stein and Shakarchi.\footnote{Elias
M. Stein and Rami Shakarchi, {\em Functional Analysis}, p.~325, Proposition 2.1.}

\begin{theorem}
If there is some $c>0$ such that $|(\nabla \Phi)(x)| \geq c$ for all $x \in \supp \psi$, then for each nonnegative integer $N$ there is some
$c_N \geq 0$ such that
\[
|I(\lambda)| \leq c_N \lambda^{-N}, \qquad \lambda>0.
\]
\end{theorem}
\begin{proof}
There is some $h \in \mathscr{D}(\mathbb{R}^d)$, $h \geq 0$, such that $h(x)=1$ for $x \in \supp \psi$.\footnote{Walter Rudin, {\em Functional Analysis},
second ed., p.~162, Theorem 6.20.}
Define $a:\mathbb{R}^d \to \mathbb{R}^d$ by
\[
a = h \frac{\nabla \Phi}{|\nabla \Phi|^2},
\]
whose entries each belong to $\mathscr{D}(\mathbb{R}^d)$,
and define $L: C^\infty(\mathbb{R}^d) \to \mathscr{D}(\mathbb{R}^d)$ by
\[ 
Lf = \frac{1}{i\lambda} \sum_{k=1}^d a_k \partial_k f = \frac{1}{i\lambda} (a\cdot \nabla) f.
\]
$L$ satisfies, doing integration by parts and using the fact that $a$ has compact support,
\[
\int_{\mathbb{R}^d} (Lf)g dx = \frac{1}{i\lambda}  \sum_{k=1}^d \int_{\mathbb{R}^d} a_k (\partial_k f) g dx
=\frac{1}{i\lambda}  \sum_{k=1}^d -\int_{\mathbb{R}^d} f \partial_k(ga) dx.
\]
Thus the \textbf{transpose} of $L$ is
\[
L^t g = -\frac{1}{i\lambda} \sum_{k=1}^d \partial_k(ga) = -\frac{1}{i\lambda} \nabla \cdot (ga).
\]

Furthermore,  in $\supp \psi$,
\begin{eqnarray*}
L(e^{i\lambda \Phi}) &=& e^{i\lambda \Phi} \sum_{k=1}^d a_k (\partial_k \Phi)  \\
&=&  e^{i\lambda \Phi} \sum_{k=1}^d \frac{\partial_k \Phi}{|\nabla \Phi|^2} \partial_k \Phi\\
&=& e^{i\lambda \Phi}.
\end{eqnarray*}
Thus   for any positive integer $N$ and for $x \in \supp \psi$, $L(e^{i\lambda \Phi})(x) = e^{i\lambda \Phi(x)}$, hence
\[
I(\lambda) = \int_{\mathbb{R}^d} L^N(e^{i\lambda \Phi}) \psi dx = \int_{\mathbb{R}^d} e^{i\lambda \Phi} (L^t)^N \psi dx.
\]
But
\[
\int_{\mathbb{R}^d} |(L^t)^N \psi| dx = \int_{\mathbb{R}^d} | \lambda^{-N} A_N| dx,
\]
where  $A_1 = \nabla \cdot (\psi a)$ and $A_n=\nabla \cdot (A_{n-1} a)$. With
\[
c_N = \int_{\mathbb{R}^d} |A_N| dx < \infty,
\]
we obtain
\[
|I(\lambda)|
=\left|  \int_{\mathbb{R}^d} e^{i\lambda \Phi} (L^t)^N \psi dx \right|
 \leq \int_{\mathbb{R}^d} |(L^t)^N \psi| dx = c_N \lambda^{-N},
\]
completing the proof.
\end{proof}

The following is an estimate for a one-dimensional oscillatory integral without an amplitude term.\footnote{Elias
M. Stein and Rami Shakarchi, {\em Functional Analysis}, p.~326, Proposition 2.2.}


\begin{lemma}
Let $a<b$,
 and suppose that $\Phi \in C^2(\mathbb{R})$ is real-valued, that either $\Phi''(x) \geq 0$ for all $x \in [a,b]$ or $\Phi''(x) \leq 0$ for all $x \in [a,b]$,
 and   that $\Phi'(x) \geq 1$ for  all $x \in [a,b]$. 
 Then 
 \[
\left| \int_a^b e^{i\lambda \Phi(x)} dx \right| \leq 3 \lambda^{-1}, \qquad \lambda >0.
 \]
 \label{22}
\end{lemma}
\begin{proof}
Write
\[
L = \frac{1}{i \lambda \Phi'} \frac{d}{dx},
\]
which satisfies
\[
\int_a^b (Lf)g dx = \int_a^b  \frac{1}{i \lambda \Phi'} f' g dx =  \frac{1}{i \lambda \Phi'} f g \bigg|_a^b 
-\int_a^b f \left( \frac{g}{i \lambda \Phi'} \right)' dx.
\]
With $f=e^{i\lambda \Phi}$ and $g=1$, we have $Lf = e^{i\lambda \Phi}$ and hence
\begin{eqnarray*}
\int_a^b e^{i\lambda \Phi} dx &=& \frac{e^{i\lambda \Phi}}{i \lambda \Phi'} \bigg|_a^b - \int_a^b e^{i\lambda \Phi} \left( \frac{1}{i \lambda \Phi'} \right)' dx\\
&=& \frac{e^{i\lambda \Phi}}{i \lambda \Phi'} \bigg|_a^b  + \frac{1}{i\lambda} \int_a^b e^{i\lambda \Phi} (\Phi')^{-2} \Phi'' dx.
\end{eqnarray*}
For $\lambda>0$, using that $\Phi'(x) \geq 1$ for all $x \in [a,b]$ 
 the boundary terms have absolute value
 \[
\left|  \frac{e^{i\lambda \Phi(b)}}{i \lambda \Phi'(b)} -  \frac{e^{i\lambda \Phi(a)}}{i \lambda \Phi'(a)} \right|
\leq \frac{1}{\lambda |\Phi'(b)|} + \frac{1}{\lambda |\Phi'(a)|}
\leq \frac{2}{\lambda}.
\]
Because $\Phi'' \geq 0$ or $\Phi'' \leq 0$ on $[a,b]$,
\begin{eqnarray*}
\frac{1}{\lambda} \left| \int_a^b e^{i\lambda \Phi} (\Phi')^{-2} \Phi'' dx \right| 
&\leq& \frac{1}{\lambda} \int_a^b |  (\Phi')^{-2} \Phi''| dx\\
&=&\frac{1}{\lambda} \left| \int_a^b (\Phi')^{-2} \Phi'' dx \right| \\
&=&\frac{1}{\lambda} \left| \frac{1}{\Phi'(a)}- \frac{1}{\Phi'(b)} \right|\\
&\leq&\frac{1}{\lambda};
\end{eqnarray*}
the final inequality uses the fact that the two terms inside the absolute value are both $\geq 1$, and thus the absolute value can be bounded by the larger
of them.
Putting together the two inequalities,
\[
\left|\int_a^b e^{i\lambda \Phi} dx\right| \leq \frac{2}{\lambda}+\frac{3}{\lambda}= 3 \lambda^{-1}, \qquad \lambda>0,
\]
proving the claim.
\end{proof}


\begin{lemma}
Let $a<b$,
 and suppose that $\Phi \in C^2(\mathbb{R})$ is real-valued,  that either $\Phi''(x) \geq 0$ for all $x \in [a,b]$ or $\Phi''(x) \leq 0$ for all $x \in [a,b]$,
 and that  there is some $\mu > 0$ such that $|\Phi'(x)| \geq \mu$ for  all $x \in [a,b]$. 
 Then 
 \[
\left| \int_a^b e^{i\lambda \Phi(x)} dx\right|  \leq 3 \mu^{-1} \lambda^{-1}, \qquad \lambda >0.
 \]
 \label{muinequality}
\end{lemma}
\begin{proof}
$\Phi'$ is continuous on  $[a,b]$, so, by the intermediate value theorem, either
$\Phi'(x) \geq \mu$ for all $x \in [a,b]$ or $\Phi'(x) \leq -\mu$ for all $x \in [a,b]$. Let $\epsilon=1$ in the first
case and $\epsilon=-1$ in the second case, and define $\Phi_0=\epsilon \frac{\Phi}{\mu}$.
Then applying Lemma \ref{22}, for $\lambda>0$ we have, writing $\lambda_0=\mu \lambda$,
\[
\left| \int_a^b e^{i \lambda_0 \Phi_0(x)} dx \right| \leq 3\lambda_0^{-1},
\]
i.e.
\[
\left| \int_a^b e^{i \epsilon \lambda \Phi(x)} dx \right| \leq 3(\mu \lambda)^{-1}.
\]
If $\epsilon=1$ this is the claim. If $\epsilon=-1$, then the above integral is the complex conjugate of the integral in the claim, and these have the same
absolute values.
\end{proof}


\begin{theorem}
Let $a<b$,
 and suppose that $\Phi \in C^2(\mathbb{R})$ is real-valued, that either $\Phi''(x) \geq 0$ for all $x \in [a,b]$ or $\Phi''(x) \leq 0$ for all $x \in [a,b]$,
 and  there is some $\mu > 0$ such that $|\Phi'(x)| \geq \mu$ for  all $x \in [a,b]$.  Suppose also
 that $\psi \in C^1(\mathbb{R})$.
 Then with
 \[
 c_\psi = 3\left( |\psi(b)| + \int_a^b |\psi'(x)| dx \right),
 \]
 we have
 \[
 \left| \int_a^b e^{i\lambda \Phi(x)} \psi(x) dx \right| \leq c_\psi \mu^{-1} \lambda^{-1}.
 \]
 \label{lambdaamplitude}
\end{theorem}
\begin{proof}
Define $J:[a,b] \to \mathbb{C}$ by
\[
J(x) = \int_a^x e^{i\lambda \Phi(u)} du,
\]
which satisfies $J'(x)=e^{i\lambda \Phi(x)}$. Integrating by parts, 
\[
\int_a^b e^{i\lambda \Phi(x)} \psi(x) dx = \int_a^b J'(x) \psi(x) dx =
J(x)\psi(x) \bigg|_a^b -\int_a^b J(x) \psi'(x) dx,
\]
and as $J(a)=0$ this is equal to
\[
J(b)\psi(b)-\int_a^b J(x) \psi'(x) dx.
\]
Lemma \ref{muinequality} tells us that $|J(x)| \leq 3\mu^{-1} \lambda^{-1}$ for all $x \in [a,b]$, so
\[
\left|J(b)\psi(b)-\int_a^b J(x) \psi'(x) dx\right| \leq 3\mu^{-1} \lambda^{-1} |\psi(b)|
+ 3\mu^{-1} \lambda^{-1} \int_a^b |\psi'(x)| dx,
\]
proving the claim.
\end{proof}


The following is \textbf{van der Corput's lemma}.\footnote{Elias M. Stein and Rami Shakarchi, {\em Functional Analysis}, p.~328, Proposition 2.3.}

\begin{lemma}[van der Corput's lemma]
Let $a<b$ and suppose that $\Phi \in C^2(\mathbb{R})$ is real-valued and satisfies $\Phi''(x) \geq 1$ for all $x \in [a,b]$.  Then
\[
\left| \int_a^b e^{i\lambda \Phi(x)} dx\right|  \leq 8 \lambda^{-1/2}, \qquad \lambda>0.
\]
\label{vandercorput}
\end{lemma}
\begin{proof}
Because $\Phi'$ is strictly increasing on $[a,b]$, $\Phi'$ has at most one zero in this interval.
If $\Phi'(x_0)=0$, then for $x \geq x_0+ \lambda^{-1/2}$ we have
$\Phi'(x) \geq \lambda^{-1/2}$, and  applying Lemma \ref{muinequality} with $\mu=\lambda^{-1/2}$,
\[
\left| \int_{[x_0+\lambda^{-1/2},b]} e^{i\lambda \Phi(x)} dx \right| \leq 3\mu^{-1} \lambda^{-1} = 3\lambda^{-1/2}.
\]
For $x \leq x_0-\lambda^{-1/2}$ we have $\Phi'(x) \leq - \lambda^{-1/2}$, and  applying Lemma \ref{muinequality} with $\mu=\lambda^{-1/2}$,
\[
\left| \int_{[a,x_0-\lambda^{-1/2}]} e^{i\lambda \Phi(x)} dx \right| \leq 3\mu^{-1} \lambda^{-1} = 3\lambda^{-1/2}.
\]
But
\[
\left| \int_{[x_0-\lambda^{-1/2},x_0+\lambda^{-1/2}] \cap [a,b]} e^{i\lambda \Phi(x)} dx \right| \leq  \int_{[x_0-\lambda^{-1/2},x_0+\lambda^{-1/2}] \cap [a,b]}  dx \leq
2\lambda^{-1/2},
\]
and
\[
\int_a^b = \int_{[a,x_0-\lambda^{-1/2}]} +  \int_{[x_0-\lambda^{-1/2},x_0+\lambda^{-1/2}] \cap [a,b]} +  \int_{[x_0+\lambda^{-1/2},b]},
\]
so 
\[
\left| \int_a^b e^{i\lambda \Phi(x)} dx\right| \leq 3\lambda^{-1/2}+2\lambda^{-1/2}+3\lambda^{-1/2}= 8\lambda^{-1/2}.
\]

If there is no $x_0 \in [a,b]$ such that $\Phi'(x_0)=0$, then either $\Phi'>0$ on $[a,b]$ or $\Phi'<0$ on $[a,b]$. In the first case, 
because $\Phi'$ is strictly increasing on $[a,b]$, $\Phi'(x) > \lambda^{-1/2}$ for 
$x \in [a+\lambda^{-1/2},b]$, and applying Lemma \ref{muinequality} with $\mu=\lambda^{-1/2}$ gives
\begin{eqnarray*}
\left| \int_a^b e^{i\lambda \Phi(x)} dx \right| 
&\leq& \left| \int_{[a,a+\lambda^{-1/2}] \cap [a,b]} e^{i\lambda \Phi(x)} dx \right|
+\left| \int_{[a+\lambda^{-1/2},b]} e^{i\lambda \Phi(x)} dx \right|\\
&\leq&\lambda^{-1/2} + 3\mu^{-1} \lambda^{-1}\\
&=&4\lambda^{-1/2}.
\end{eqnarray*}
In the second case, $\Phi'(x) < -\lambda^{-1/2}$ for $x \in [a,b-\lambda^{-1/2}]$, and applying Lemma
\ref{muinequality} with $\mu=\lambda^{-1/2}$ also gives
\[
\left| \int_a^b e^{i\lambda \Phi(x)} dx \right|  \leq 4 \lambda^{-1/2}.
\]
Therefore, if $\Phi'$ does not have a zero on $[a,b]$ then
\[
\left| \int_a^b e^{i\lambda \Phi(x)} dx \right|  \leq 4 \lambda^{-1/2} < 8\lambda^{-1/2}.
\]
\end{proof}


\begin{lemma}
Let $a<b$ and suppose that $\Phi \in C^2(\mathbb{R})$ is real-valued and that there is some $\mu>0$ such that
$|\Phi''(x)| \geq \mu$ for all $x \in [a,b]$.  Then
\[
\left| \int_a^b e^{i\lambda \Phi(x)} dx\right|  \leq 8\mu^{-1/2}   \lambda^{-1/2}, \qquad \lambda>0.
\]
\label{vandercorputmu}
\end{lemma}
\begin{proof}
$\Phi''$ is continuous on $[a,b]$, so by the intermediate value theorem  either $\Phi''(x) \geq \mu$ for all  $x \in [a,b]$ or $\Phi''(x) \leq -\mu$ for all  $x\in [a,b]$. 
Let $\epsilon=1$ in the first case and $\epsilon=-1$ in the second case, and define $\Phi_0=\epsilon \frac{\Phi}{\mu}$. 
Then $\Phi_0''(x) \geq 1$ for all $x \in [a,b]$, and
applying Lemma \ref{vandercorput},
\[
\left| \int_a^b e^{i \mu \lambda \Phi_0(x)} dx \right| \leq 8 (\mu \lambda)^{-1/2}, \qquad \lambda>0,
\]
i.e.
\[
\left| \int_a^b e^{i\epsilon \lambda \Phi(x)} dx \right| \leq 8(\mu \lambda)^{-1/2}, \qquad \lambda>0.
\]
If $\epsilon=1$ this is the inequality in the claim. If $\epsilon=-1$, then the above integral is the complex conjugate of the integral in the claim, and these have the same
absolute values.
\end{proof}


We use the above  to prove the following estimate which involves an amplitude.\footnote{Elias M. Stein and Rami Shakarchi, {\em Functional Analysis}, p.~328, Corollary 2.4.}

\begin{theorem}
Let $a<b$ and suppose that $\Phi \in C^2(\mathbb{R})$ is real-valued and that there is some $\mu>0$ such that
 $|\Phi''(x)| \geq \mu$ for all $x \in [a,b]$.
Suppose also that $\psi \in C^1(\mathbb{R})$. 
  Then
  with
\[
c_\psi = 8\left(|\psi(b)|+\int_a^b |\psi'(x)| dx\right),
\]
we have
\[
\left| \int_a^b e^{i\lambda \Phi(x)} \psi(x) dx \right| \leq c_\psi \mu^{-1/2} \lambda^{-1/2}, \qquad \lambda>0.
\]
\label{squarerootamplitude}
\end{theorem}
\begin{proof}
Define $J:[a,b] \to \mathbb{C}$ by 
\[
J(x) = \int_a^x e^{i\lambda \Phi(u)} du,
\]
which satisfies $J'(x)=e^{i\lambda \Phi(x)}$.
Integrating by parts,
\[
\int_a^b e^{i\lambda \Phi(x)} \psi(x) dx = \int_a^b J'(x) \psi(x) dx =
J(x)\psi(x) \bigg|_a^b -\int_a^b J(x) \psi'(x) dx.
\]
and as $J(a)=0$ this is equal to
\[
J(b)\psi(b)-\int_a^b J(x) \psi'(x) dx.
\]
But for each $x \in [a,b]$ we have by Lemma \ref{vandercorputmu} that $|J(x)| \leq 8\mu^{-1/2} \lambda^{-1/2}$, so 
\[
\left| J(b)\psi(b)-\int_a^b J(x) \psi'(x) dx \right|
\leq 8\mu^{-1/2}\lambda^{-1/2} |\psi(b)| + 8\mu^{-1/2}\lambda^{-1/2} \int_a^b |\psi'(x)| dx,
\]
completing the proof.
\end{proof}


\section{Bessel functions}
For $n \in \mathbb{Z}$, the \textbf{$n$th Bessel function of the first kind} $J_n:\mathbb{R} \to \mathbb{R}$ is
\[
J_n(\lambda) = \frac{1}{2\pi} \int_0^{2\pi} e^{i\lambda \sin x} e^{-inx} dx, \qquad \lambda\in \mathbb{R}.
\]
Let
\[
I_1 = \left[0,\frac{\pi}{4}\right], \quad I_2 = \left[ \frac{3\pi}{4},\pi \right], \quad  I_3 = \left[\pi,\frac{5\pi}{4}\right], \quad I_4= \left[\frac{7\pi}{4},2\pi\right],
\]
on which $|\cos x| \geq \frac{1}{\sqrt{2}}$, and
\[
I_5 = \left[\frac{\pi}{4},\frac{3\pi}{4}\right], \quad I_6=\left[\frac{5\pi}{4},\frac{7\pi}{4}\right],
\]
on which $|\sin x| \geq \frac{1}{\sqrt{2}}$. 
Write $\Phi(x)=\sin x$ and $\psi(x)=e^{-inx}$. $\Phi'(x)=\cos(x)$ and $\Phi''(x)=-\sin(x)$, 
and for $I_1,I_2,I_3,I_4$ we apply Theorem \ref{lambdaamplitude} with $\mu=\frac{1}{\sqrt{2}}$.
For each of $I_1,I_2,I_3,I_4$ we compute $c_\psi = 3\left(1+\frac{\pi n}{4}\right)$, 
which gives us
\[
\left| \int_{I_k} e^{i\lambda \Phi(x)} \psi(x) dx \right| \leq c_\psi \mu^{-1} \lambda^{-1}
=3\left(1+\frac{\pi n}{4}\right) \cdot \sqrt{2} \cdot \lambda^{-1}.
\]


For $I_5$ and $I_6$, we apply Theorem \ref{squarerootamplitude} with $\mu=\frac{1}{\sqrt{2}}$. For each of $I_5$ and $I_6$ we compute
$c_\psi = 8\left(1+\frac{\pi n}{2} \right)$, which gives us
\[
\left| \int_{I_k} e^{i\lambda \Phi(x)} \psi(x) dx \right| \leq c_\psi \mu^{-1/2} \lambda^{-1/2} = 
 8\left(1+\frac{\pi n}{2} \right) \cdot 2^{1/4} \cdot \lambda^{-1/2}.
\]

Therefore
\[
|J_n(\lambda)| \leq 4\cdot\frac{1}{2\pi} \cdot 3\left(1+\frac{\pi n}{4}\right) \cdot \sqrt{2} \cdot \lambda^{-1}
+ 2 \cdot \frac{1}{2\pi} \cdot 8\left(1+\frac{\pi n}{2} \right) \cdot 2^{1/4} \cdot \lambda^{-1/2},
\]
which shows that for each $n \in \mathbb{Z}$,
\[
J_n(\lambda) = O_n(\lambda^{-1/2})
\]
as $\lambda \to \infty$.

\end{document}