\documentclass{article}
\usepackage{amsmath,amssymb,mathrsfs,amsthm}
%\usepackage{tikz-cd}
%\usepackage{hyperref}
\newcommand{\inner}[2]{\left\langle #1, #2 \right\rangle}
\newcommand{\tr}{\ensuremath\mathrm{tr}\,} 
\newcommand{\Span}{\ensuremath\mathrm{span}} 
\def\Re{\ensuremath{\mathrm{Re}}\,}
\def\Im{\ensuremath{\mathrm{Im}}\,}
\newcommand{\id}{\ensuremath\mathrm{id}} 
\newcommand{\var}{\ensuremath\mathrm{var}} 
\newcommand{\Lip}{\ensuremath\mathrm{Lip}} 
\newcommand{\GL}{\ensuremath\mathrm{GL}} 
\newcommand{\diam}{\ensuremath\mathrm{diam}} 
\newcommand{\sgn}{\ensuremath\mathrm{sgn}\,} 
\newcommand{\lcm}{\ensuremath\mathrm{lcm}} 
\newcommand{\ess}{\ensuremath\mathrm{ess}\,}
\newcommand{\supp}{\ensuremath\mathrm{supp}\,}
\newcommand{\dom}{\ensuremath\mathrm{dom}\,}
\newcommand{\upto}{\nearrow}
\newcommand{\downto}{\searrow}
\newcommand{\norm}[1]{\left\Vert #1 \right\Vert}
\newtheorem{theorem}{Theorem}
\newtheorem{lemma}[theorem]{Lemma}
\newtheorem{proposition}[theorem]{Proposition}
\newtheorem{corollary}[theorem]{Corollary}
\theoremstyle{definition}
\newtheorem{definition}[theorem]{Definition}
\newtheorem{example}[theorem]{Example}
\begin{document}
\title{Bernstein's inequality and Nikolsky's inequality for $\mathbb{R}^d$}
\author{Jordan Bell}
\date{February 16, 2015}

\maketitle

\section{Complex Borel measures and the Fourier transform}
Let $\mathcal{M}(\mathbb{R}^d)=rca(\mathbb{R}^d)$ be the set of complex Borel measures on $\mathbb{R}^d$.  This is a Banach
algebra with the total variation norm, with convolution as multiplication; for $\mu \in \mathcal{M}(\mathbb{R}^d)$, we denote by $|\mu|$ the \textbf{total variation of $\mu$}, which
itself belongs to $\mathcal{M}(\mathbb{R}^d)$, and the \textbf{total variation norm of $\mu$} is $\norm{\mu}=|\mu|(\mathbb{R}^d)$.


For $\mu \in \mathcal{M}(\mathbb{R}^d)$, it is a fact that the union $O$ of all open sets $U \subset \mathbb{R}^d$ such that
$|\mu|(U)=0$ itself satisfies $|\mu|(O)=0$. We define $\supp \mu = \mathbb{R}^d \setminus O$, called the \textbf{support of $\mu$}.

For $\mu \in \mathcal{M}(\mathbb{R}^d)$, we define $\hat{\mu}:\mathbb{R}^d \to \mathbb{C}$ by
\[
\hat{\mu}(\xi) = \int_{\mathbb{R}^d} e^{-2\pi i\xi\cdot x} d\mu(x), \qquad
\xi \in \mathbb{R}^d.
\]
It is a fact that  $\hat{\mu}$ belongs to $C_u(\mathbb{R})$, the collection of bounded uniformly continuous functions $\mathbb{R}^d \to \mathbb{C}$.
For $\xi \in \mathbb{R}^d$,
\begin{equation}
|\hat{\mu}(\xi)| \leq \int_{\mathbb{R}^d} |e^{-2\pi i\xi\cdot x}| d|\mu|(x)
=|\mu|(\mathbb{R}^d) = \norm{\mu}.
\label{TVnorm}
\end{equation}
Let $m_d$ be Lebesgue measure on $\mathbb{R}^d$.
For $f \in L^1(\mathbb{R}^d)$, let
\[
\Lambda_f = f m_d,
\]
which belongs to $\mathcal{M}(\mathbb{R}^d)$. We define $\hat{f}:\mathbb{R}^d \to \mathbb{C}$ by
\[
\hat{f}(\xi)  = \widehat{\Lambda_f}(\xi) = \int_{\mathbb{R}^d} e^{-2\pi i\xi\cdot x} d\Lambda_f(x)=
\int_{\mathbb{R}^d} f(x) e^{-2\pi i\xi\cdot x} dm_d(x),
\quad \xi \in \mathbb{R}^d.
\]

The following theorem establishes properties of the Fourier transform of a complex Borel measure with compact support.\footnote{Thomas H. Wolff,
{\em Lectures on Harmonic Analysis}, p.~3, Proposition 1.3.}

\begin{theorem}
If $\mu \in \mathcal{M}(\mathbb{R}^d)$ and $\supp \mu$ is compact, then $\hat{\mu} \in C^\infty(\mathbb{R}^d)$
and for any multi-index $\alpha$,
\[
D^\alpha \hat{\mu} = \mathscr{F}((-2\pi ix)^\alpha \mu).
\]
For $R>0$, if $\supp \mu \subset \overline{B(0,R)}$, then 
\[
\norm{D^\alpha \hat{\mu}}_\infty \leq (2\pi R)^{|\alpha|_1} \norm{\mu}.
\]
\label{mucompact}
\end{theorem}
\begin{proof}
For $j=1,\ldots,d$, let $e_j$ be the $j$th coordinate vector in $\mathbb{R}^d$, with length 1. 
Let $\xi \in \mathbb{R}^d$, and define
\[
\Delta(h) = \frac{\hat{\mu}(\xi+he_j)-\hat{\mu}(\xi)}{h}, \qquad h \neq 0.
\]
We can write this as
\[
\Delta(h) = \int_{\mathbb{R}^d} \frac{e^{-2\pi ihx_j}-1}{h} e^{-2\pi i\xi\cdot x} d\mu(x).
\]

For any $x \in \mathbb{R}^d$,
\[
\left|  \frac{e^{-2\pi ihx_j}-1}{h} \right| = \frac{|e^{-2\pi ihx_j}-1|}{|h|} 
\leq \frac{|-2\pi i hx_j|}{|h|} = 2\pi |x_j|.
\]
Because $\mu$ has compact support, $2\pi |x_j| \in L^1(\mu)$. Furthermore,
for each $x \in \mathbb{R}^d$ we have
\[
 \frac{e^{-2\pi ihx_j}-1}{h} \to -2\pi i x_j, \qquad h \to 0.
\]
Therefore, the dominated convergence theorem tells us that
\[
\lim_{h \to 0} \Delta(h) = \int_{\mathbb{R}^d} -2\pi i x_j e^{-2\pi i\xi \cdot x} d\mu(x).
\]
On the other hand, for $\alpha_k=1$ for $k=j$ and $\alpha_k=0$ otherwise,
\[
(D^\alpha \hat{\mu}) (\xi) = \lim_{h \to 0} \Delta(h),
\]
so
\[
(D^\alpha \hat{\mu}) (\xi) =  \int_{\mathbb{R}^d} (-2\pi i x)^\alpha e^{-2\pi i\xi\cdot x} d\mu(x)
= \mathscr{F}((-2\pi ix)^\alpha \mu)(\xi),
\]
and in particular, $\hat{\mu} \in C^1(\mathbb{R}^d)$. (The Fourier transform of a regular complex Borel measure on a locally compact abelian group
is bounded and uniformly continuous.\footnote{Walter Rudin, {\em Fourier Analysis on Groups}, p.~15, Theorem 1.3.3.})
Because $\mu$ has compact support so does $(-2\pi ix)^\alpha \mu$, hence we can play the above game
with $(-2\pi ix)^\alpha \mu$, and by induction it follows that for any $\alpha$,
\[
D^\alpha \hat{\mu} = \mathscr{F}((-2\pi ix)^\alpha \mu),
\]
and in particular, $\hat{\mu} \in C^\infty(\mathbb{R}^d)$. 

Suppose that $\supp \mu \subset \overline{B(0,R)}$. 
The total variation  of the complex measure $(-2\pi ix)^\alpha \mu$ is the positive measure
\[
(2\pi)^{|\alpha|_1} |x_1|^{\alpha_1} \cdots |x_d|^{\alpha_d} |\mu|,
\]
hence
\begin{align*}
\norm{(-2\pi ix)^\alpha \mu} &=(2\pi)^{|\alpha|_1} \int_{\mathbb{R}^d} |x_1|^{\alpha_1} \cdots |x_d|^{\alpha_d} d|\mu|(x)\\
&=(2\pi)^{|\alpha|_1} \int_{\overline{B(0,R)}} |x_1|^{\alpha_1} \cdots |x_d|^{\alpha_d} d|\mu|(x)\\
&\leq (2\pi)^{|\alpha|_1} \int_{\overline{B(0,R)}} R^{\alpha_1} \cdots R^{\alpha_d} d|\mu|(x)\\
&=(2\pi R)^{|\alpha|_1} \int_{\overline{B(0,R)}} d|\mu|(x)\\
&=(2\pi R)^{|\alpha|_1} \int_{\mathbb{R}^d} d|\mu|(x)\\
&=(2\pi R)^{|\alpha|_1} \norm{\mu}.
\end{align*}
Then using \eqref{TVnorm}, 
\[
\norm{\mathscr{F}((-2\pi ix)^\alpha \mu)}_\infty  \leq \norm{(-2\pi ix)^\alpha \mu} \leq (2\pi R)^{|\alpha|_1} \norm{\mu}.
\]
But we have already established that  $D^\alpha \hat{\mu} = \mathscr{F}((-2\pi ix)^\alpha \mu)$, which with the above inequality completes the proof.
\end{proof}


\section{Test functions}
For an open subset $\Omega$ of $\mathbb{R}^d$,  we denote by $\mathscr{D}(\Omega)$ the set of those
$\phi \in C^\infty(\Omega)$ such that $\supp \phi$ is a compact set. Elements of $\mathscr{D}(\Omega)$ are called
\textbf{test functions}.


It is a fact that there is a test function $\phi$ satisfying: (i) $\phi(x)=1$ for $|x| \leq 1$, (ii) $\phi(x)=0$ for $|x| \geq 2$,
(iii) $0 \leq \phi \leq 1$, and (iv) $\phi$ is radial. We write, for $k=1,2,\ldots$,
\[
\phi_k(x) = \phi(k^{-1} x), \qquad x \in \mathbb{R}^d.
\]
For any multi-index $\alpha$,
\[
(D^\alpha \phi_k)(x) = k^{-|\alpha|_1} (D^\alpha \phi)(k^{-1}x), \qquad x \in \mathbb{R}^d,
\]
hence 
\begin{equation}
\norm{D^\alpha \phi_k}_\infty = k^{-|\alpha|_1} \norm{D^\alpha \phi}_\infty.
\label{Dalpha}
\end{equation}
We use the following lemma to prove the theorem that comes after it.\footnote{Thomas H. Wolff, {\em Lectures on Harmonic Analysis}, p.~4, Lemma 1.5.}


\begin{lemma}
Suppose that $f \in C^N(\mathbb{R}^d)$ and $D^\alpha f \in L^1(\mathbb{R}^d)$ for each $|\alpha| \leq N$.
Then
for each $|\alpha| \leq N$, $D^\alpha  (\phi_k f) \to D^\alpha f$ in $L^1(\mathbb{R}^d)$ as $k \to \infty$.
\label{testfunction}
\end{lemma}
\begin{proof}
Let $|\alpha| \leq N$.
In the case $\alpha=0$,
\begin{align*}
\norm{\phi_k f-f}_1 &= \int_{\mathbb{R}^d} |\phi_k(x) f(x)-f(x)| dx\\
&=\int_{|x| \geq k} |\phi_k(x)f(x)-f(x)| dx\\
&\leq \int_{|x| \geq k} |f(x)| dx.
\end{align*}
Because $f \in L^1(\mathbb{R}^d)$, this tends to $0$ as $k \to \infty$.

Suppose that $\alpha>0$.
The Leibniz rule tells us that with $c_\beta =\binom{\alpha}{\beta}$, we have, for each $k$,
\[
D^\alpha(\phi_k f) = \phi_k D^\alpha f + \sum_{0<\beta \leq \alpha} c_\beta D^{\alpha-\beta} f D^\beta \phi_k.
\]
For $C_1=\max_\beta |c_\beta|$,
\begin{align*}
\norm{D^\alpha(\phi_k f)-\phi_k D^\alpha f}_1&\leq \sum_{0<\beta \leq \alpha} \norm{c_\beta D^{\alpha-\beta} f D^\beta \phi_k}_1\\
&\leq C_1 \sum_{0<\beta \leq \alpha} \norm{D^\beta \phi_k}_\infty \norm{D^{\alpha-\beta} f}_1.
\end{align*}
Let $C_2 = \max_{0<\beta \leq \alpha} \norm{D^\beta \phi}_\infty$. By \eqref{Dalpha}, for $0<\beta \leq \alpha$ we have
\[
\norm{D^\beta \phi_k}_\infty = k^{-|\beta|_1} \norm{D^\beta \phi}_\infty \leq C_2 k^{-|\beta|_1} \leq C_2 k^{-1}.
\]
Thus
\[
\norm{D^\alpha(\phi_k f)-\phi_k D^\alpha f}_1 \leq C_1 C_2 k^{-1} \sum_{0<\beta \leq \alpha} \norm{D^{\alpha-\beta}f}_1,
\]
which tends to $0$ as $k \to \infty$.
For any $k$,
\begin{align*}
\norm{\phi_k D^\alpha f - D^\alpha f}_1 &=\int_{\mathbb{R}^d} |\phi_k(x) (D^\alpha f)(x)-(D^\alpha f)(x)| dx\\
&=\int_{|x| \geq k} |\phi_k(x) (D^\alpha f)(x)-(D^\alpha f)(x)| dx\\
&\leq \int_{|x| \geq k} |(D^\alpha f)(x)| dx,
\end{align*}
and because $D^\alpha f \in L^1(\mathbb{R}^d)$, this tends to $0$ as $k \to \infty$.
But
\[
\norm{D^\alpha (\phi_k f)-D^\alpha f}_1 \leq
\norm{D^\alpha (\phi_k f)-\phi_k D^\alpha f}_1
+\norm{\phi_k D^\alpha f-D^\alpha f}_1,
\]
which completes the proof.
\end{proof}

Now we calculate the Fourier transform of the derivative of a function, and show that
the smoother a function is the faster its Fourier transform decays.\footnote{Thomas H. Wolff, {\em Lectures on Harmonic Analysis}, p.~4, Proposition 1.4.}

\begin{theorem}
If $f \in C^N(\mathbb{R}^d)$ and $D^\alpha f \in L^1(\mathbb{R}^d)$ for each
$|\alpha| \leq N$, then for each $|\alpha| \leq N$,
\begin{equation}
\widehat{D^\alpha f}(\xi) = (2\pi i\xi)^\alpha \hat{f}(\xi), \qquad \xi \in \mathbb{R}^d.
\label{fourierD}
\end{equation}
There is a constant $C=C(f,N)$ such that
\[
|\hat{f}(\xi)| \leq C(1+|\xi|)^{-N}, \qquad \xi \in \mathbb{R}^d.
\]
\end{theorem}
\begin{proof}
If $g \in C^1_c(\mathbb{R}^d)$, then for any $1 \leq j \leq d$, integrating by parts,
\[
\int_{\mathbb{R}^d} (\partial_j g)(x) e^{-2\pi i\xi \cdot x} dx = 2\pi i \xi_j \int_{\mathbb{R}^d} g(x) e^{-2\pi i\xi\cdot x} dx.
\]
It follows by induction that if $g \in C^N_c(\mathbb{R}^d)$, then for each $|\alpha| \leq N$,
\[
\widehat{D^\alpha g}(\xi) = (2\pi i\xi)^{\alpha} \hat{g}(\xi), \qquad \xi \in \mathbb{R}^d.
\]

Let $|\alpha| \leq N$. 
For $k=1,2,\ldots$, let $f_k = \phi_k f$. For each $k$ we have
 $f_k \in C^N(\mathbb{R}^d)$, hence
\[
\widehat{D^\alpha f_k}(\xi) = (2\pi i\xi)^\alpha \widehat{f_k}(\xi), \qquad \xi \in \mathbb{R}^d.
\]
On the one hand,
\[
\norm{\widehat{D^\alpha f_k}-\widehat{D^\alpha f}}_\infty 
=\norm{\mathscr{F}(D^\alpha f_k - D^\alpha f)}_\infty
\leq \norm{D^\alpha f_k-D^\alpha f}_1,
\]
and Lemma \ref{testfunction} tells us that this tends to $0$ as $k \to \infty$. 
On the other hand, 
for $\xi \in \mathbb{R}^d$,  
\begin{align*}
|\widehat{D^\alpha f_k}(\xi)  - (2\pi i\xi)^\alpha \hat{f}(\xi)|&=
|(2\pi i\xi)^\alpha \widehat{f_k}(\xi)-(2\pi i\xi)^\alpha \hat{f}(\xi)|\\
&=|(2\pi i \xi)^\alpha|  |\mathscr{F}(f_k-f)(\xi)|\\
&\leq |(2\pi i \xi)^\alpha| \norm{f_k-f}_1,
\end{align*}
which by Lemma \ref{testfunction} tends to $0$ as $k \to \infty$.
Therefore, for $\xi \in \mathbb{R}^d$,
\[
|\widehat{D^\alpha f}(\xi)-(2\pi i\xi)^\alpha \hat{f}(\xi)| \leq \norm{\widehat{D^\alpha f_k}-\widehat{D^\alpha f}}_\infty 
+|\widehat{D^\alpha f_k}(\xi)  - (2\pi i\xi)^\alpha \hat{f}(\xi)|,
\]
and because the right-hand side tends to $0$ as $k \to \infty$, we get
\[
\widehat{D^\alpha f}(\xi) = (2\pi i\xi)^\alpha \hat{f}(\xi).
\]

If $y \in S^{d-1}$ then there is at least one $1 \leq j \leq d$ with $y_j \neq 0$, from which we get
\[
\sum_{|\beta|_1 = N} |y^\beta| > 0.
\]
The function $y \mapsto \sum_{|\beta|_1 = N} |y^\beta|$ is continuous $S^{d-1} \to \mathbb{R}$, so
there is some $C_N>0$ such that
\[
\frac{1}{C_N} \leq \sum_{|\beta|_1 = N} |y^\beta|, \qquad y \in S^{d-1}.
\]
For  nonzero $x \in \mathbb{R}^d$, write $x=|x|y$, with which
$\sum_{|\beta|_1 = N} |x^\beta| = |x|^N \sum_{|\beta|_1=N} |y^\beta|$.  
Therefore
\[
|x|^N \leq C_N \sum_{|\beta|_1 = N} |x^\beta|, \qquad x \in \mathbb{R}^d.
\]

For $|\alpha| \leq N$, because the Fourier transform of an element of $L^1$ belongs
to $C_0$, we have by  \eqref{fourierD} that  $\xi \mapsto \xi^\alpha \hat{f}(\xi)$ belongs to $C_0(\mathbb{R}^d)$, and in 
particular is bounded. Then for $\xi \in \mathbb{R}^d$,
\begin{align*}
|\xi|^N |\hat{f}(\xi)|& \leq C_N \sum_{|\beta|_1 = N} |\xi^\beta| |\hat{f}(\xi)|\\
&= C_N \sum_{|\beta|_1 = N} |\xi^\beta \hat{f}(\xi)|\\
&\leq C_N \sum_{|\beta|_1 = N} \norm{\xi^\beta \hat{f}(\xi)}_\infty\\
&=C'.
\end{align*}
On the one hand,
for $|\xi| \geq 1$ we have
\[
1+|\xi| \leq 2|\xi|,
\]
hence
\[
|\xi|^{-N} \leq \left( \frac{1+|\xi|}{2} \right)^{-N}  = 2^N (1+|\xi|)^{-N},
\]
giving
\[
|\hat{f}(\xi)| \leq C' |\xi|^{-N} \leq C' 2^N (1+|\xi|)^{-N}.
\]
On the other hand, for $|\xi| \leq 1$ we have
\[
1+|\xi| \leq 2,
\]
and so
\[
|\hat{f}(\xi)| \leq \norm{\hat{f}}_\infty 2^N 2^{-N}   \leq 
  \norm{\hat{f}}_\infty 2^N (1+|\xi|)^{-N} .
\]
Thus, for
\[
C=\max\left\{2^N C', 2^N \norm{\hat{f}}_\infty\right\}
\]
we have 
\[
|\hat{f}(\xi)| \leq C(1+|\xi|)^{-N}, \qquad \xi \in \mathbb{R}^d,
\]
completing the proof.
\end{proof}




\section{Bernstein's inequality for {\em L\textsuperscript{2}}}
For a Borel measurable function $f:\mathbb{R}^d \to \mathbb{C}$, 
let $O$ be the union of those open subsets $U$ of $\mathbb{R}^d$ such that $f(x)=0$ for almost all
$x \in U$. In other words, $O$ is the largest open set on which $f=0$ almost everywhere.
The 
\textbf{essential support of $f$} is the set
\[
\ess \supp f = \mathbb{R}^d \setminus O.
\]


The following is \textbf{Bernstein's inequality for $L^2(\mathbb{R}^d)$}.\footnote{Thomas H. Wolff,
{\em Lectures on Harmonic Analysis}, p.~31, Proposition 5.1.}

\begin{theorem}
If $f \in L^2(\mathbb{R}^d)$, $R>0$, and 
\begin{equation}
\ess \supp \hat{f} \subset \overline{B(0,R)},
\label{esssup}
\end{equation}
then there is some $f_0 \in C^\infty(\mathbb{R}^d)$ such that $f(x)=f_0(x)$ for almost all $x \in \mathbb{R}^d$,
and for any multi-index $\alpha$,
\[
\norm{D^\alpha f_0}_2 \leq (2\pi R)^{|\alpha|_1} \norm{f}_2.
\]
\end{theorem}
\begin{proof}
Let $\chi_R$ be the indicator function for  $\overline{B(0,R)}$. By \eqref{esssup}, the Cauchy-Schwarz inequality, and the Parseval identity,
\[
\norm{\hat{f}}_1 =  \norm{\chi_R \hat{f}}_1 \leq \norm{\chi_R}_2 \norm{\hat{f}}_2
=m_d(\overline{B(0,R)})^{1/2} \norm{f}_2
< \infty,
\]
so $\hat{f} \in L^1(\mathbb{R}^d)$.
The Plancherel theorem\footnote{Walter Rudin, {\em Real and Complex Analysis}, third ed., p.~187, Theorem 9.14.}
 tells us that if
$g \in L^2(\mathbb{R}^d)$ and $\hat{g} \in L^1(\mathbb{R}^d)$, then
\[
g(x) = \int_{\mathbb{R}^d} \hat{g}(\xi) e^{2\pi ix\cdot \xi} d\xi
\]
for almost all $x \in \mathbb{R}^d$. 
Thus, for
 $f_0:\mathbb{R}^d \to \mathbb{C}$ defined by
\[
f_0(x) = \int_{\mathbb{R}^d} \hat{f}(\xi) e^{2\pi ix \cdot \xi} d\xi=\mathscr{F}(\hat{f})(-x), \qquad
x \in \mathbb{R}^d,
\]
we have $f(x)=f_0(x)$ for almost all $x \in \mathbb{R}^d$. Because $f=f_0$ almost everywhere,
\[
\widehat{f_0}=\hat{f}.
\]

Applying Theorem \ref{mucompact} to $d\mu(\xi)=\widehat{f_0}(-\xi) d\xi$, 
we have $f_0 \in C^\infty(\mathbb{R}^d)$ and for any multi-index $\alpha$,
\[
D^\alpha f_0 = \mathscr{F}((-2\pi i\xi)^\alpha \hat{f}(-\xi)).
\]
By Parseval's identity,
\begin{align*}
\norm{D^\alpha f_0}_2&=\norm{(-2\pi i\xi)^\alpha \hat{f}(-\xi)}_2\\
&=\norm{(2\pi i\xi)^\alpha \chi_R(\xi) \hat{f}(\xi)}_2\\
&\leq \norm{(2\pi i\xi)^\alpha \chi_R(\xi)}_\infty \norm{\hat{f}}_2\\
&\leq (2\pi R)^{|\alpha|_1} \norm{\hat{f}}_2\\
&= (2\pi R)^{|\alpha|_1} \norm{f}_2,
\end{align*}
proving the claim.
\end{proof}



\section{Nikolsky's inequality}
\textbf{Nikolsky's inequality} tells us that if the Fourier transform of a function is supported on a ball centered at the origin, then
for $1 \leq p \leq q \leq \infty$, the $L^q$ norm of the function is bounded above in terms of its $L^p$ norm.\footnote{Camil Muscalu and Wilhelm Schlag, {\em Classical and Multilinear
Harmonic Analysis}, volume I, p.~83, Lemma 4.13.}

\begin{theorem}
There is a constant $C_d$ such that
if $f \in \mathscr{S}(\mathbb{R}^d)$, $R>0$, 
\[
\supp \hat{f} \subset \overline{B(0,R)},
\]
and $1 \leq p \leq q \leq \infty$, then
\[
\norm{f}_q \leq C_d R^{d\left(\frac{1}{p}-\frac{1}{q}\right)} \norm{f}_p.
\]
\end{theorem}
\begin{proof}
Let $g=f_{R}$, i.e. 
\[
g(x) = R^{-d} f(R^{-1}x), \qquad x \in \mathbb{R}^d.
\]
Then  for $\xi \in \mathbb{R}^d$,
\[
\hat{g}(\xi) = \int_{\mathbb{R}^d} g(x) e^{-2\pi i\xi\cdot x} dx
=\int_{\mathbb{R}^d} R^{-d} f(R^{-1}x) e^{-2\pi i\xi \cdot x} dx
=\hat{f}(R\xi),
\]
showing that $\supp \hat{g} = R^{-1} \supp \hat{f} \subset \overline{B(0,1)}$.
Let $\chi \in \mathscr{D}(\mathbb{R}^d)$ with $\chi(\xi)=1$ for $|\xi| \leq 1$, with which
\[
\hat{g}= \chi \hat{g}.
\]
Then $g = (\mathscr{F}^{-1} \chi)*g$, and using Young's inequality, with $1+\frac{1}{q}=\frac{1}{r}+\frac{1}{p}$,
\begin{equation}
\norm{g}_q\leq \norm{\mathscr{F}^{-1} \chi}_{r} \norm{g}_q=\norm{\hat{\chi}}_r \norm{g}_q.
\label{ginequality}
\end{equation}
Moreover,
\begin{align*}
\norm{g}_a &= \left( \int_{\mathbb{R}^d} |R^{-d} f(R^{-1}x)|^a dx \right)^{1/a}\\
&=\left( \int_{\mathbb{R}^d} R^{-da+d} |f(y)|^a dy \right)^{1/a}\\
&=R^{d\left(\frac{1}{a}-1\right)} \norm{f}_a,
\end{align*}
so \eqref{ginequality} tells us
\[
R^{d\left(\frac{1}{q}-1\right)} \norm{f}_q \leq \norm{\hat{\chi}}_r 
R^{d\left(\frac{1}{p}-1\right)} \norm{f}_p,
\]
i.e.
\[
\norm{f}_q \leq \norm{\hat{\chi}}_r R^{d\left(\frac{1}{p}-\frac{1}{q}\right)} \norm{f}_p.
\]
Now, $\frac{1}{r}=1+\frac{1}{q}-\frac{1}{p}$, so $0 \leq \frac{1}{r} \leq 1$ because $1 \leq p \leq q \leq \infty$, namely,
$1 \leq r \leq \infty$. 
By the log-convexity of $L^r$ norms, for $\frac{1}{r}=1-\theta$ we have
\[
\norm{\hat{\chi}}_r \leq \norm{\hat{\chi}}_1^{1-\theta} \norm{\hat{\chi}}_\infty^{\theta}.
\]
Thus with
\[
C_d = \max\{\norm{\hat{\chi}}_1,\norm{\hat{\chi}}_\infty\}
\]
we have proved the claim.
\end{proof}


\section{The Dirichlet kernel and Fej\'er kernel for \textbf{R}}
The function $D_M \in C_0(\mathbb{R})$ defined by
\[
D_M(x)=\frac{\sin 2\pi Mx}{\pi x}, \qquad x \neq 0
\]
and $D_M(0)=2M$, is called the \textbf{Dirichlet kernel}.
Let $\chi_M$ be the indicator function for the set $[-M,M]$. We have, for $x \neq 0$,
\begin{align*}
\widehat{\chi_R}(x)& = \int_{\mathbb{R}} \chi_R(\xi) e^{-2\pi ix\xi} d\xi\\
&=\int_{-M}^M e^{-2\pi ix\xi} d\xi\\
&=\frac{e^{-2\pi ix\xi}}{-2\pi ix} \bigg|_{-M}^M\\
&=\frac{e^{-2\pi iMx}}{-2\pi ix} + \frac{e^{2\pi iMx}}{2\pi ix}\\
&=\frac{1}{\pi x} \frac{e^{2\pi iMx}-e^{-2\pi iMx}}{2i}\\
&=\frac{\sin 2\pi Mx}{\pi x}.
\end{align*}
For $x=0$, $\widehat{\chi_R}(0)=2M=D_M(0)$. Thus,
\[
D_M=\widehat{\chi_R}.
\]

For $f \in L^1(\mathbb{R})$ and $M>0$, we define
\[
(S_M f)(x) = \int_{-M}^M \hat{f}(\xi) e^{2\pi i\xi x} d\xi, \qquad x \in \mathbb{R}.
\]
It is straightforward to check that
\[
(S_M f)(x) = \int_{\mathbb{R}} \frac{\sin 2\pi Mt}{\pi t} f(x-t) dt
=(D_M*f)(x), \qquad x \in \mathbb{R}.
\]


For $f \in L^1(\mathbb{R})$, $M>0$, and $x \in \mathbb{R}$,
\begin{align*}
\frac{1}{M} \int_0^M (S_mf)(x) dm&=\frac{1}{M} \int_0^M \left( \int_{-m}^m \hat{f}(\xi) e^{2\pi i\xi x} d\xi \right) dm\\
&=\frac{1}{M}  \int_0^M \left( \int_{-m}^m \left( \int_{\mathbb{R}} f(y) e^{-2\pi i\xi y} dy \right) e^{2\pi i\xi x} d\xi \right) dm\\
&=\frac{1}{M} \int_{\mathbb{R}} f(y) \left( \int_0^M \left( \int_{-m}^m e^{-2\pi i\xi(y-x)} d\xi\right) dm \right) dy\\
&=\frac{1}{M} \int_{\mathbb{R}} f(y) \left( \int_0^M D_m(y-x) dm \right) dy\\
&=\frac{1}{M} \int_{\mathbb{R}} f(y)  \left( \int_0^M \frac{\sin 2\pi m(y-x)}{\pi(y-x)} dm \right) dy\\
&=\frac{1}{M} \int_{\mathbb{R}} f(y) \left( -\frac{\cos 2\pi m(y-x)}{2\pi^2 (y-x)^2} \bigg|_{0}^M \right) dy\\
&=\frac{1}{M} \int_{\mathbb{R}} f(y) \left(\frac{1}{2\pi^2(y-x)^2} - \frac{\cos 2\pi M(y-x)}{2\pi^2(y-x)^2} \right) dy.
\end{align*}
We define the \textbf{Fej\'er kernel} $K_M \in C_0(\mathbb{R})$ by
\[
K_M(x) = \frac{1-\cos 2\pi M x}{2M \pi^2 x^2}, \qquad x \neq 0,
\]
and $K_M(0)=M$. Thus, because $K_M$ is an even function,
\[
\frac{1}{M} \int_0^M (S_m f)(x) dm = (K_M*f)(x).
\]

One proves that $K_M$ is an \textbf{approximate identity}:  $K_M \geq 0$, 
\[
\int_{\mathbb{R}} K_M(x) dx = 1,
\]
and for any $\delta>0$,
\[
\lim_{M \to \infty} \int_{|x|>\delta} K_M(x) dx =0.
\]
The fact that $K_M$ is an approximate identity implies that for any $f \in L^1(\mathbb{R})$,
$K_M*f \to f$ in $L^1(\mathbb{R})$ as $M \to \infty$. 


We shall use the Fej\'er kernel to prove Bernstein's inequality for $\mathbb{R}$.\footnote{Mark A. Pinsky,
{\em Introduction to Fourier Analysis and Wavelets}, p.~122, Theorem 2.3.17.}

\begin{theorem}
If $\mu \in \mathcal{M}(\mathbb{R})$, $M>0$, and
\[
\supp \mu \subset [-M,M],
\]
then
\[
\norm{\hat{\mu}'}_\infty \leq 4\pi M \norm{\hat{\mu}}_\infty.
\]
\end{theorem}
\begin{proof}
For $x_0 \in \mathbb{R}$, let $d\mu_{x_0}(t)=e^{-2\pi i x_0 t} d\mu(t)$. $\mu_{x_0}$ has the same support has $\mu$, and
\[
\widehat{\mu_{x_0}}(x) = \int_{\mathbb{R}} e^{-2\pi ix t} d\mu_{x_0}(t)
=\int_{\mathbb{R}} e^{-2\pi ix t} e^{-2\pi ix_0 t} d\mu(t)
=\hat{\mu}(x+x_0).
\]
It follows that to prove the claim it suffices to prove that $|\hat{\mu}'(0)| \leq 4\pi M \norm{\hat{\mu}}_\infty$. 

Write $f=\hat{\mu} \in C_u(\mathbb{R})$. Define $\Delta_M \in C_c(\mathbb{R})$ by
\[
\Delta_M(t)=\begin{cases}
M-|t|&|t|<M\\
0&|t| \geq M,
\end{cases}
\qquad t \in \mathbb{R}.
\]
We calculate, for $x \neq 0$,
\begin{align*}
\int_{\mathbb{R}} \Delta_M(t) e^{-2\pi ix t} dt&=-\frac{e^{-2\pi iMx}(-1+e^{2\pi iMx})^2}{4\pi^2 x^2}\\
&=\frac{(\sin \pi Mx)^2}{\pi^2 x^2}\\
&=\frac{1-\cos 2\pi Mx}{2\pi^2 x^2}.
\end{align*}
so
\[
\widehat{\Delta_M}(x) = MK_M(x).
\]
Then for $t \in [-M,M]$,
\begin{align*}
\int_{\mathbb{R}} (e^{2\pi iM\xi}-e^{-2\pi iM\xi}) K_M(\xi) e^{-2\pi i\xi t} d\xi&=
\widehat{K_M}(t-M)-\widehat{K_M}(t+M)\\
&=\frac{\Delta_M(-t+M)-\Delta_M(-t-M)}{M}\\
&=\frac{t}{M}.
\end{align*}
On the one hand, the integral of the left-hand side with respect to $\mu$ is
\[
\begin{split}
&\int_{\mathbb{R}} \int_{\mathbb{R}} (e^{2\pi iM\xi}-e^{-2\pi iM\xi}) K_M(\xi) e^{-2\pi i\xi t} d\xi d\mu(t)\\
=&\int_{\mathbb{R}}  (e^{2\pi iM\xi}-e^{-2\pi iM\xi}) K_M(\xi) f(\xi) d\xi.
\end{split}
\]
On the other hand, the integral of the right-hand side with respect to $\mu$ is
\begin{align*}
\int_{\mathbb{R}} \frac{t}{M} d\mu(t)&=\frac{1}{-2\pi i M} \int_{\mathbb{R}} -2\pi i t d\mu(t)\\
&=\frac{1}{-2\pi iM} \mathscr{F}((-2\pi it) \mu)(0)\\
&=\frac{1}{-2\pi iM} f'(0).
\end{align*}
Hence
\[
\frac{1}{-2\pi iM} f'(0) = \int_{\mathbb{R}}  (e^{2\pi iM\xi}-e^{-2\pi iM\xi}) K_M(\xi) f(\xi) d\xi,
\]
giving
\[
|f'(0)| \leq 4 \pi M \norm{f}_\infty  \norm{K_M}_1 = 4\pi M \norm{f}_\infty,
\]
proving the claim.
\end{proof}

\end{document}

