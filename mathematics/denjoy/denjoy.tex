\documentclass{article}
\usepackage{amsmath,amssymb,graphicx,subfig,amsthm,mathrsfs}
\newtheorem{theorem}{Theorem}
\newtheorem{lemma}[theorem]{Lemma}
\newcommand{\norm}[1]{\Vert #1 \Vert}
\newcommand{\Var}[0]{\operatorname{Var}}
\theoremstyle{definition}
\newtheorem{remark}[theorem]{Remark}
\begin{document}
\title{Denjoy's theorem on circle diffeomorphisms}
\author{Jordan Bell}
\date{April 3, 2014}
\maketitle

In this note I'm just presenting the proof of Denjoy's theorem in Michael Brin and Garrett Stuck's {\em Introduction
to dynamical systems}, Cambridge University Press, 2002. 

Let $S^1=\mathbb{R}/\mathbb{Z}$. For $\alpha \in \mathbb{R}$, define $R_\alpha:S^1 \to S^1$ by $R_\alpha(x)=x+\alpha
+\mathbb{Z}$.

We say that a homeomorphism $f:S^1 \to S^1$ is {\em orientation preserving} if it lifts to an increasing homeomorphism
$F:\mathbb{R} \to \mathbb{R}$: $\pi \circ F=f \circ \pi$.

The {\em rotation number} of an orientation preserving homeomorphism $f$ is defined by
\[
\rho(f)=\lim_{n \to \infty} \frac{F^n(x)-x}{n}.
\]
One proves that this is independent both of the lift $F$ of $f$ and the point $x \in \mathbb{R}$.
Some facts about the rotation number: it is an invariant of topological conjugacy, and $\rho(f)$ is rational if and only
if $f$ has a periodic point. A periodic point is $x \in S^1$ such that $f^n(x)=x$ for some $n \geq 1$.

There are some lemmas in Chapter 7 that I don't want to write out. The important theorem that
we're going to use without proof
is that if $f:S^1 \to S^1$ is an orientation preserving homeomorphism that is topologically
transitive with irrational rotation
number $\rho(f)$, then $f$ is topologically conjugate to $R_{\rho(f)}$. 
This reduces our problem to showing that a map is topologically transitive.

We will use the following lemma in the proof of Denjoy's theorem.

\begin{lemma}
Let $f:S^1 \to S^1$ be a $C^1$ diffeomorphism and
let $J$ be an interval in $S^1$. Let $g=\log f'$. If the interiors of $J,f(J),\ldots,f^{n-1}(J)$ are pairwise
disjoint, then for any $x, y \in J$ and any $n \in \mathbb{Z}$ we have
\[
\Var(g) \geq |\log((f^n)'(x))-\log((f^n)'(y))|.
\]
\end{lemma}
\begin{proof}
The intervals $[x,y], [f(x),f(y)], \ldots, [f^{n-1}(x),f^{n-1}(y)]$ are pairwise disjoint, so
they are part of a partition of $[0,1]$. The total variation of $g$ is defined as a supremum
over all partitions, so in particular it will be $\geq$ the sum coming from any particular
partition or a subset of that partition.

\begin{eqnarray*}
\Var(g)&\geq&\sum_{k=0}^{n-1} |g(f^k(y))-g(f^k(x))|\\
&\geq&\Big| \sum_{k=0}^{n-1} g(f^k(y))-g(f^k(x)) \Big|\\
&=&\Big| \log \prod_{k=0}^{n-1} f'(f^k(y))-\log \prod_{k=0}^{n-1} f'(f^k(x)) \Big|\\
&=&|\log((f^n)'(x))-\log((f^n)'(y))|.
\end{eqnarray*}
\end{proof}

Now we can prove Denjoy's theorem.

\begin{theorem}
If $f:S^1 \to S^1$ is a $C^1$ diffeomorphism that is orientation preserving, that has irrational rotation
number $\rho(f)$, and whose derivative $f':S^1 \to \mathbb{R}$ has bounded variation, then $f$ is topologically
conjugate to $R_{\rho(f)}$.
\end{theorem}
\begin{proof}
Suppose by contradiction that $f$ is not topologically transitive. It's a fact proved in Chapter 7
of Brin and Stuck that this implies that $\omega(x)$ is perfect and nowhere dense, and is independent
of the point $x$. (Recall that $\omega(x)=\bigcap_{n \geq 1} \overline{\bigcup_{i \geq n} f^i(x)}$.)
It follows that
 there is an interval $I=(a,b)$
in its complement.

The intervals $f^n(I)$, $n \in \mathbb{Z}$, are pairwise disjoint, for otherwise
$f$ would have a periodic point. Let $\mu$ be Haar measure on $S^1$. Then
\[
\sum_{n \in \mathbb{Z}} \mu(f^n(I)) \leq 1.
\]

Let $x \in S^1$. Suppose for the moment that there are infinitely $n \geq 1$ such that
the intervals $(x,f^{-n}(x)), (f(x),f^{1-n}(x)),\ldots, (f^n(x),x)$ are pairwise disjoint;
we shall prove that this is true later.
By applying the lemma we proved with
$y=f^{-n}(x)$ we get 
\[
\Var(g) \geq \Big| \log \frac{(f^n)'(x)}{(f^n)'(y)} \Big|=
|\log((f^n)'(x) (f^{-n})'(x)|. 
\]
To see the equality in the above line it helps to write out what $(f^{-n})'(x)$ is.

Then for infinitely many $n$ we have
\begin{eqnarray*}
\mu(f^n(I))+\mu(f^{-n}(I))&=&\int_I (f^n)'(x) dx +\int_I (f^{-n})'(x) dx\\
&=&\int_I ((f^n)'(x)+(f^{-n})'(x)) dx\\
&\geq&\int_I \sqrt{(f^n)'(x) (f^{-n})'(x)}dx\\
&=&\int_I \sqrt{\exp \log((f^n)'(x) (f^{-n})'(x))} dx\\
&\geq&\int_I \sqrt{\exp(-| \log((f^n)'(x) (f^{-n})'(x)) |)} dx\\
&\geq&\int_I \sqrt{\exp(-\Var(g))} dx\\
&=&\exp\Big( -\frac{1}{2}\Var(g) \Big) \mu(I).
\end{eqnarray*}
Since $\mu(I)>0$ this implies that $\sum_{n \in \mathbb{Z}} \mu(f^n(I))=\infty$, a contradiction. Therefore
$f$ is topologically transitive, and so it is topologically conjugate
to the $R_{\rho(f)}$.
\end{proof}

It is indeed necessary that $f'$ has bounded variation. Brin and Stuck give an example on p. 161 that they attribute to Denjoy: for any irrational number $\rho \in (0,1)$, there is a nontransitive orientation preserving $C^1$
diffeomorphism of $S^1$ with rotation number $\rho$. The only condition of Denjoy's theorem that isn't
satisfied here is that $f'$ have bounded variation.

\end{document}
