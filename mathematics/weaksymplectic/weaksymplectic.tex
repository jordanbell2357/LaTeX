\documentclass{article}
\usepackage{amsmath,amssymb,mathrsfs,amsthm}
%\usepackage{tikz-cd}
%\usepackage{hyperref}
\newcommand{\inner}[2]{\left\langle #1, #2 \right\rangle}
\newcommand{\tr}{\ensuremath\mathrm{tr}\,} 
\newcommand{\Span}{\ensuremath\mathrm{span}} 
\def\Re{\ensuremath{\mathrm{Re}}\,}
\def\Im{\ensuremath{\mathrm{Im}}\,}
\newcommand{\id}{\ensuremath\mathrm{id}} 
\newcommand{\var}{\ensuremath\mathrm{var}} 
\newcommand{\Lip}{\ensuremath\mathrm{Lip}} 
\newcommand{\Sh}{\ensuremath\mathrm{Sh}} 
\newcommand{\GL}{\ensuremath\mathrm{GL}} 
\newcommand{\diam}{\ensuremath\mathrm{diam}} 
\newcommand{\sgn}{\ensuremath\mathrm{sgn}} 
\newcommand{\lcm}{\ensuremath\mathrm{lcm}} 
\newcommand{\supp}{\ensuremath\mathrm{supp}\,}
\newcommand{\dom}{\ensuremath\mathrm{dom}\,}
\newcommand{\upto}{\nearrow}
\newcommand{\downto}{\searrow}
\newcommand{\norm}[1]{\left\Vert #1 \right\Vert}
\theoremstyle{definition}
\newtheorem{theorem}{Theorem}
\newtheorem{lemma}[theorem]{Lemma}
\newtheorem{proposition}[theorem]{Proposition}
\newtheorem{corollary}[theorem]{Corollary}
\theoremstyle{definition}
\newtheorem{definition}[theorem]{Definition}
\newtheorem{example}[theorem]{Example}
\begin{document}
\title{Weak symplectic forms and differential calculus in Banach spaces}
\author{Jordan Bell}
\date{May 16, 2015}

\maketitle

\section{Introduction}
There are scarcely any decent expositions of infinite dimensional symplectic vector spaces. One good basic exposition
is by Marsden and Ratiu.\footnote{Jerrold E. Marsden and Tudor S. Ratiu, {\em Introduction to Mechanics and Symmetry}, second ed.,
Chapter 2.} The Darboux theorem for a real reflexive Banach space is proved in Lang and probably in fewer other places than one might
guess.\footnote{Serge Lang,
{\em Differential and Riemannian Manifolds}, p.~150, Theorem 8.1;
Mircea Puta, {\em Hamiltonian Mechanical Systems and Geometric Quantization}, p.~12, Theorem 1.3.1.}
(Other references.\footnote{Andreas Kriegl and Peter W. Michor, {\em The Convenient Setting of Global Analysis}, p.~522, \S 48;
Peter W. Michor,
{\em Some geometric evolution equations arising as geodesic equations on groups of diffeomorphisms including the Hamiltonian approach},
pp.~133--215, in
Antonio Bove, Ferruccio Colombini, and Daniele Del Santo (eds.), {\em Phase Space Analysis of Partial Differential Equations};
K.-H. Need, H. Sahlmann, and T. Thiemann, {\em Weak Poisson Structures on Infinite Dimensional Manifolds
and Hamiltonian Actions}, pp. 105--135, in Vladimir Dobrev (ed.), {\em Lie Theory and Its Applications in Physics};
Tudor S. Ratiu, {\em Coadjoint Orbits and the Beginnings of a Geometric Representation Theory}, pp.~417--457, in 
Karl-Hermann Neeb and Arturo Pianzola (eds.), {\em Developments and Trends in Infinite-Dimensional Lie Theory}.})


\section{Bilinear forms}
Let $E$ be a real Banach space. For a bilinear form $B:E \times E \to \mathbb{R}$, define
\[
\norm{B} = \sup_{\norm{e} \leq 1, \norm{f} \leq 1} |B(e,f)|.
\]
One proves that $B$ is continuous if and only if $\norm{B}<\infty$. Namely, a bilinear form is continuous if and only if it is bounded.

If $B:E \times E \to \mathbb{R}$ is a continuous bilinear form,
we define 
$B^\flat:E \to E^*$ by
\[
B^\flat(e)(f) = B(e,f), \qquad e \in E, f \in E;
\]
indeed, for $e \in E$, $\norm{B^\flat(e)f} = \norm{B(e,f)} \leq \norm{B} \norm{e} \norm{f}$, showing that
$\norm{B^\flat(e)} \leq \norm{B} \norm{e}$, so that
$B^\flat(e)$ is continuous $E \to \mathbb{R}$. Moreover, it is apparent that $B^\flat$ is linear, and
\begin{align*}
\norm{B^\flat}&=\sup_{\norm{e} \leq 1} \norm{B^\flat(e)}\\
&=\sup_{\norm{e} \leq 1} \sup_{\norm{f} \leq 1} |B^\flat(e)(f)|\\
&=M,
\end{align*}
so $B^\flat:E \to E^*$ is continuous. 

We call a continuous bilinear form $B:E \times E \to F$ \textbf{weakly nondegenerate} if $B^\flat:E \to E^*$
is one-to-one. Since $B^\flat$ is linear, this is equivalent to the statement that
$B^\flat(e)=0$ implies that $e=0$, which is equivalent to the statement that
 if $B(e,f)=0$ for all $f$ then $e=0$.
 
An \textbf{isomorphism of Banach spaces} is a linear isomorphism $T:E \to F$ that is continuous such that
$T^{-1}F \to E$ is continuous. Equivalently, to say that $T:E \to F$ is an isomorphism of Banach spaces
means that $T:E \to F$ is a bijective bounded linear map such that $T^{-1}:F \to E$ is a bounded linear map.
It follows from the open mapping theorem that if $T:E \to F$ is an onto bounded linear isomorphism, hence is an isomorphism of Banach spaces.

 We say that a continuous bilinear form $B:E \times E \to \mathbb{R}$ is \textbf{strongly nondegenerate} if $B^\flat:E \to E^*$ is an isomorphism of Banach spaces. 

For a real vector space $V$ and a bilinear form $B:V \times V \to \mathbb{R}$, we say that
$B$ is \textbf{alternating} if $B(v,v)=0$ for all $v \in V$. We say that $B$ is \textbf{skew-symmetric} if 
$B(u,v)=-B(v,u)$ for all $u,v \in V$. It is straightforward to check that $B$ is alternating if and only if $B$ is skew-symmetric.


For Banach spaces $E_1,\ldots,E_p$ and $F$, let
$\mathscr{L}(E_1,\ldots,E_p;F)$ denote the set of continuous multilinear maps
$E_1 \times \cdots E_p \to F$. For a multilinear map $T:E_1 \times \cdots \times E_p \to F$ to be continuous it is equivalent that
\[
\norm{T} = \sup_{\norm{e_1} \leq 1, \ldots \norm{e_p} \leq 1} \norm{T(e_1,\ldots,e_n)}
< \infty,
\]
namely that it is bounded with the operator norm. 
With this norm, $\mathscr{L}(E_1,\ldots,E_p;F)$ is a Banach space.\footnote{Henri Cartan, {\em Differential Calculus}, p.~22, Theorem 1.8.1.} 
We write
\[
\mathscr{L}_p(E;F) = \mathscr{L}(E_1,\ldots,E_p;F).
\]


For Banach spaces $E$ and $F$,
we denote by $\GL(E;F)$ the set of isomorphisms $E \to F$. One proves that $\GL(E;F)$ is an open set in the Banach space
$\mathscr{L}(E;F)$ and that with the subspace topology, $u \mapsto u^{-1}$ is continuous $\GL(E;F) \to \GL(F;E)$.\footnote{Henri Cartan, {\em Differential Calculus}, p.~20, Theorem 1.7.3.}


For Banach spaces $E,F,G$, define
\[
\phi:\mathscr{L}(E,F;G) \to \mathscr{L}(E;\mathscr{L}(F,G))
\]
by $\phi(f)(x)(y) = f(x,y)$ for $f \in \mathscr{L}(E,F;G)$, $x \in E$, and $y \in F$. One proves that
$\phi$ is an isometric isomorphism.\footnote{Henri Cartan, {\em Differential Calculus}, p.~23, \S 1.9.}


\section{Differentiable functions}
Let $E$ and $F$ be Banach spaces and  let $U$ be a nonempty open subset of $E$.
For $a \in U$, a function $f:U \to F$  is said to be \textbf{differentiable at $a$} if (i) $f$ is continuous at $a$ and (ii)
there is a linear mapping $g:E \to F$ such that
\[
\norm{f(x)-f(a)-(g(x)-g(a))}_F = o(\norm{x-a}_E),
\]
as $x \to a$ in $E$. 
We prove that there is at most one such linear mapping $g$ and write $f'(a)=g$, and call $f'(a)$ the \textbf{derivative of $f$ at $a$}.
 We also prove that
if $f$ is differentiable at $a$ then $f'(a):E \to F$ is continuous at $a$ and therefore, being linear, is continuous on $E$, namely
$f'(a) \in \mathscr{L}(E;F)$.\footnote{Henri Cartan, {\em Differential Calculus}, p.~25.}

If $f:U \to F$ is differentiable at each $a \in U$, we say that \textbf{$f$ is differentiable on $U$}.
We call $f':U \to \mathscr{L}(E;F)$ the \textbf{derivative of $f$}. We also write $Df=f'$.

We say that $f:U \to F$ is $C^1$, also called \textbf{continuously differentiable}, if (i) $f$ is differentiable on $U$ and (ii) $f':U \to \mathscr{L}(E;F)$ is continuous.

Let $E,F,G$ be Banach spaces, let $U$ be an open subset of $E$,  let $V$ be an open subset of $F$, and let
$f:U \to F$ and $g:V \to G$ be continuous. Suppose that $a \in U$ and that $f(a) \in V$. We define
$g \circ f:f^{-1}(V) \to G$ on $f^{-1}(V)$. One proves that if $f$ is differentiable at $a$ and $g$ is differentiable at $f(a)$, then
$h=g \circ f:f^{-1}(V) \to F$ is differentiable at $a$ and satisfies\footnote{Henri Cartan, {\em Differential Calculus}, p.~27, Theorem 2.2.1.}
\[
h'(a) = g'(f(a)) \circ f'(a).
\]

For Banach spaces $E$ and $F$, 
let $\phi:\GL(E;F) \to \mathscr{L}(F;E)$ be defined by $\phi(u)=u^{-1}$. $\GL(E;F)$ is an open subset of
the Banach space $\mathscr{L}(E;F)$ and
 $\phi$ is continuous. It is proved that $\phi$ continuously differentiable, and that for $u \in \GL(E;F)$,
the derivative of $\phi$ at $u$,
\[
\phi'(u) \in \mathscr{L}(\mathscr{L}(E;F);  \mathscr{L}(F;E)),
\]
 satisfies\footnote{Henri Cartan, {\em Differential Calculus}, p.~31, Theorem 2.4.4.}
 \[
 \phi'(u)(h) = -u^{-1} \circ h \circ u^{-1}, \qquad h \in \mathscr{L}(E;F).
 \]



\section{Symplectic forms}
A \textbf{weak symplectic form} on a Banach space $E$ is a continuous bilinear form $\Omega:E \times E \to \mathbb{R}$ that is weakly nondegenerate and 
and alternating. 

A \textbf{strong symplectic form} on a Banach space $E$ is a continuous bilinear form
$\Omega:E \times E \to \mathbb{R}$ that is
strongly nondegenerate and  alternating. If $\Omega$ is a strong symplectic form on a Banach space $E$, we define
$\Omega^\sharp:E^* \to E$ by $\Omega^\sharp = (\Omega^\flat)^{-1}$, which is an isomorphism of Banach spaces.




\section{Hamiltonian functions}
Let $E$ be a  real Banach space $E$, let $\mathscr{D}(A)$ be a linear subspace of $E$, and let
$A:\mathscr{D}(A) \to E$ be a linear map, called an \textbf{operator in $E$}. Write
$\mathscr{R}(A)=A \mathscr{D}(A)$.
For a weak symplectic form $\omega$ on $E$, we say that $A$ is \textbf{$\omega$-skew} if
\[
\omega(Ae,f) = -\omega(e,Af), \qquad e,f \in \mathscr{D}(A).
\]
If $\mathscr{R}(A) \subset \mathscr{D}(A)$ and $A^2=-I$, then for $e,f \in \mathscr{D}(A)$ we have
$\omega(Ae,Af)=-\omega(e,A^2f)=-\omega(e,-f)=\omega(e,f)$. 

For an $\omega$-skew operator $A$ in $E$, we define $H:\mathscr{D}(A) \to \mathbb{R}$,
called the \textbf{Hamiltonian function of $A$},\footnote{See Jerrold E. Marsden and Thomas J. R. Hughes, {\em Mathematical Foundations of Elasticity}, p.~253, \S 5.1.}
 by
\[
H(u) = \frac{1}{2}\omega(Au,u), \qquad u \in \mathscr{D}(A).
\]

For a linear operator $A$ in $E$, we define 
\[
\mathscr{G}(A) = \{(u,Au): u \in \mathscr{D}(A)\}.
\]
$\mathscr{G}(A)$ is a linear subspace of $E \times E$. We say that
$A$ is \textbf{closed} if $\mathscr{G}(A)$ is a closed subset of $E \times E$. 
One proves that a linear operator $A$ in $E$ is closed if and only if 
the linear space $\mathscr{D}(A)$ with the norm
\[
\norm{e}_A = \norm{e}+\norm{Ae}, \qquad e \in \mathscr{D}(A)
\]
is a Banach space. 

For $T \in \mathscr{L}(E)$, we define $T^*\omega:E \times E \to \mathbb{R}$ by
\[
(T^*\omega)(e,f) = \omega(Te,Tf), \qquad (e,f) \in E \times E;
\]
 $T^*\omega$ is called the \textbf{pullback of $\omega$ by $T$}. It is apparent that $T^*\omega$ is  bilinear.
 We have
 \begin{align*}
 \norm{T^*\omega}&=\sup_{\norm{e} \leq 1, \norm{f} \leq 1}
 |\omega(Te,Tf)|\\
 &\leq \sup_{\norm{e} \leq 1, \norm{f} \leq 1} \norm{\omega} \norm{Te} \norm{Tf}\\
 &\leq \sup_{\norm{e} \leq 1, \norm{f} \leq 1} \norm{\omega} \norm{T} \norm{e}
 \norm{T} \norm{f}\\
 &=\norm{\omega} \norm{T}^2,
 \end{align*}
 showing that $T^*\omega$ is continuous. For $e \in E$, because $\omega$ is alternating we have
 \[
 (T^*\omega)(e,e) = \omega(Te,Te) = 0,
 \]
 i.e. $T^*\omega$ is alternating. 
 For $e \in E$, suppose that $(T^*\omega)(e,f)=0$ for all $f \in E$. That is,
 $\omega(Te,Tf)=0$ for all $f \in E$, and thus, to establish that $T^*\omega$ is weakly nondegenerate it suffices that
 $T$ be onto. In the case that $T^*\omega = \omega$, we say that $T \in \mathscr{L}(E)$ is a \textbf{canonical transformation}.

Suppose that $A$ is a closed $\omega$-skew operator in $E$, with Hamiltonian function
$H:\mathscr{D}(A) \to \mathbb{R}$. $\mathscr{D}(A)$ is a Banach space with the norm
$\norm{e}_A = \norm{e}+\norm{Ae}$.
For $u \in \mathscr{D}(A)$
and $v \in \mathscr{D}(A)$, using the fact that $A$ is $\omega$-skew we check that
\[
H(v)-H(u)-\omega(Au,v-u) = \frac{1}{2}\omega(A(v-u),v-u),
\]
hence
\[
|H(v)-H(u) - \omega(Au,v-u)| \leq \frac{1}{2}\norm{\omega} \norm{A(v-u)} \norm{v-u}
\leq \frac{1}{2}\norm{\omega} \norm{v-u}_A^2.
\]
This shows that $H$ is differentiable on the Banach space $\mathscr{D}(A)$, with derivative
$H':\mathscr{D}(A) \to \mathscr{D}(A)^*$ defined by\footnote{cf.
Jerrold E. Marsden and Thomas J. R. Hughes, {\em Mathematical Foundations of Elasticity}, p.~254, Proposition 2.2.}
\[
H'(u)(e) = \omega(Au,e), \qquad u \in \mathscr{D}(A), \quad e \in \mathscr{D}(A).
\]
Moreover, for $u, v \in \mathscr{D}(A)$ we have
\begin{align*}
\norm{H'(v)-H'(u)} &= \sup_{\norm{e}_A \leq 1}
|H'(v)(e)-H'(u)(e)|\\
&= \sup_{\norm{e}_A \leq 1} |\omega(Av,e)-\omega(Au,e)|\\
&= \sup_{\norm{e}_A \leq 1} |\omega(A(v-u),e)|\\
&\leq \sup_{\norm{e}_A \leq 1} \norm{\omega} \norm{A(v-u)} \norm{e}\\
&\leq \norm{\omega} \norm{A(v-u)}\\
&\leq \norm{\omega} \norm{v-u}_A,
\end{align*}
showing that $H':\mathscr{D}(A) \to \mathscr{D}(A)^*$ is continuous, namely
that $H$ is $C^1$.
(We also write $DH=H'$.)

Suppose that $A$ is a closed  operator in $E$ and that $H:\mathscr{D}(A) \to \mathbb{R}$ is some function
such that
$H'(u)e=\omega(Au,e)$ for all $u \in \mathscr{D}(A)$ and $e \in \mathscr{D}(A)$.
On the one hand, because $H'$ is continuous and linear, the second derivative $D^2 H:\mathscr{D}(A) \to \mathscr{L}(\mathscr{D}(A),\mathscr{D}(A)^*)$  is
\[
(D^2H)(u)(e)(f) = H'(e)(f) =\omega(Ae,f), \qquad u, e, f \in \mathscr{D}(A).
\]
On the other hand, because $D^2 H$ is continuous, for each $u \in \mathscr{D}(A)$, the bilinear form $(D^2H)(u):\mathscr{D}(A) \times
\mathscr{D}(A) \to \mathbb{R}$ is symmetric.\footnote{Serge Lang, {\em Real and Functional Analysis}, third ed., p.~344, Theorem 5.3.}
That is, $(D^2H)(u)(e)(f)=(D^2H)(u)(f)(e)$, which by the above means
\[
\omega(Ae,f)=\omega(Af,e), \qquad e,f \in \mathscr{D}(A),
\]
showing that $A$ is $\omega$-skew. Let $G:\mathscr{D}(A) \to \mathbb{R}$ be the Hamiltonian function of $A$, i..e
\[
G(u) = \frac{1}{2}\omega(Au,u), \qquad u \in \mathscr{D}(A).
\]
What we established earlier tells us that
\[
G'(u)(e)=\omega(Au,e), \qquad u \in \mathscr{D}(A), \quad e \in \mathscr{D}(A).
\]
Then we have that for $G'=H'$. Let $K=G-H$, which is $C^1$ with  $K'=0$. The
\textbf{mean value theorem}\footnote{Serge Lang, {\em Real and Functional Analysis}, third ed., p.~341, Theorem 4.2.} tells us that
for any $x,y \in \mathscr{D}(A)$,
\[
K(x+y)-K(x) = \int_0^1 K'(x+ty)(y) dt = 0,
\]
and thus $K(u)=K(0)=C$ for all $u \in \mathscr{D}(A)$. Therefore, 
$G=H+C$. 


\section{Semigroups}
Let $E$ be a real Banach space, let $\omega$ be a weak symplectic form on $E$, and let
$A$ be a closed densely defined $\omega$-skew linear operator in $E$.
Suppose that $A$ is the infinitesimal generator of a strongly continuous one-parameter semigroup $\{U_t: t \geq 0\}$, where
$U_t \in \mathscr{L}(E)$ for each $t$, and let
$H$ be the Hamiltonian function of $A$.\footnote{Jerrold E. Marsden and Thomas J. R. Hughes, {\em Mathematical Foundations of Elasticity}, p.~256, Proposition 2.6.}

\begin{theorem}
For each $t \geq 0$, $U_t$ is a canonical transformation.

For each $t \geq 0$ and for each $x \in \mathscr{D}(A)$,
\[
H(U_t x) = H(x).
\]
\label{flow}
\end{theorem}
\begin{proof}
For $u,v \in \mathscr{D}(A)$ and $t \geq 0$, using the chain rule and the fact that $\omega$ is a bilinear form,\footnote{Henri Cartan,
{\em Differential Calculus}, p.~30, Theorem 2.4.3.}
\[
\frac{d}{dt} \omega(U_t u,U_tv)=\omega\left(\frac{d}{dt} U_t u, U_tv \right)
+\omega\left(U_t u,\frac{d}{dt}U_tv\right).
\]
Because $A$ is the infinitesimal generator of $\{U_t: t \geq 0\}$, it follows
that $\frac{d}{dt}(U_t w) = AU_t w$ for each $w \in \mathscr{D}(A)$. Using this and the fact
that $A$ is $\omega$-skew,
\begin{align*}
\frac{d}{dt} \omega(U_t u,U_tv)&=\omega(AU_tu,U_tv)+\omega(U_tu,AU_tv)\\
&=-\omega(U_tu,AU_tv)+\omega(U_tu,AU_tv)\\
&=0.
\end{align*}
This implies that $\omega(U_t u,U_tv) = \omega(U_0 u, U_0 v)=\omega(u,v)$ for all $t \geq 0$, which means
that $U_t$ is a canonical transformation for each $t \geq 0$. 

For any $t \geq 0$ and $x \in \mathscr{D}(A)$, $AU_t x= U_t A x$. 
(The infinitesimal generator of a one-parameter semigroup commutes with each element of the semigroup.)
Then, using the fact that $U_t$ is a canonical transformation,
\begin{align*}
H(U_t x)&=\frac{1}{2}\omega(A(U_t x),U_t x)\\
&=\frac{1}{2}\omega(U_t Ax,U_tx)\\
&=\frac{1}{2}\omega(Ax,x)\\
&=H(x).
\end{align*}
\end{proof}


Suppose that there is some $c>0$ such that $H(u) \geq c \norm{u}_A^2$ for all $u \in \mathscr{D}(A)$, namely
that $H$ is \textbf{coercive} on the Banach space $\mathscr{D}(A)$.
Let $t \geq 0$ and 
let $u \in \mathscr{D}(A)$. 
Then $U_t u \in \mathscr{D}(A)$, so using the hypothesis and Theorem \ref{flow},
\[
\norm{U_t u}_A^2 \leq \frac{1}{c} H(U_t u)=\frac{1}{c}H(u)
=\frac{1}{2c}\omega(Au,u)
\leq \frac{1}{2c} \norm{\omega} \norm{Au} \norm{u}
\leq \frac{\norm{\omega}}{2c} \norm{u}_A^2.
\]
Therefore, for each $t \geq 0$ and $u \in \mathscr{D}(A)$,
\[
\norm{U_t u}_A \leq \sqrt{\frac{\norm{\omega}}{2c}} \norm{u}_A.
\]



\section{Hilbert spaces}
For a real vector space $V$, a \textbf{complex struture on $V$} is a linear map
$J:V \to V$ such that $J^2=-I$.
For $v \in V$, define $iv=Jv \in V$,
for which on the one hand,
\begin{align*}
(\alpha+i\beta)(\gamma+i\delta)v &= 
(\alpha+i\beta)(\gamma v + \delta Jv)\\
&=\alpha \gamma v + \alpha \delta Jv
+J(\beta \gamma v) + J(\beta \delta Jv)\\
&=\alpha \gamma v+(\alpha \delta+\beta \gamma)Jv + \beta \delta J^2 v\\
&=(\alpha \gamma - \beta \delta)v + (\alpha \delta+\beta \gamma)Jv,
\end{align*}
and on the other hand,
\[
(\alpha+i\beta)(\gamma+i\delta)v  = (\alpha \gamma-\beta \delta + (\alpha \delta + \beta \gamma)i)v.
\]
It follows that $V$ with $iv=Jv$ is a complex vector space. We emphasize that the complex vector space $V$ contains the same
elements as the real vector space $V$.
The following theorem connects symplectic forms, real inner products, and complex inner products.\footnote{Paul R. Chernoff and Jerrold E. Marsden, {\em Properties of Infinite Dimensional Hamiltonian Systems}, p.~6, Theorem 2.} By a complex inner product on a complex vector space
$W$, we mean a function $h:W \times W \to \mathbb{C}$ that is conjugate symmetric, complex linear in the first argument, $h(w,w) \geq 0$
for all $w \in W$, and $h(w,w)=0$ implies $w=0$.

\begin{theorem}
Let $H$ be a real Hilbert space with inner product $\inner{\cdot}{\cdot}:H \times H \to \mathbb{R}$
and let $\omega$ be a weak symplectic
form on $H$. Then there is a complex structure $J:H \to H$ and a real inner product
$s$ on $H$ such that
\[
s(x,y) = -\omega(Jx,y), \qquad x,y \in H
\]
is a real inner product on the real vector space $H$, and
\[
h(x,y) = s(x,y) - i\omega(x,y), \qquad x,y \in H
\]
is a complex inner product on $H$ with the complex structure $J$. 

Furthermore, the following are equivalent:
\begin{enumerate}
\item The norm induced by $h$ is equivalent with the norm induced by $\inner{\cdot}{\cdot}$.
\item The norm induced by $s$ is equivalent with the norm induced by $\inner{\cdot}{\cdot}$.
\item $\omega$ is a strong symplectic form on the real Hilbert space $H$.
\end{enumerate}
\end{theorem}
\begin{proof}
By the Riesz representation theorem,\footnote{Walter Rudin,
{\em Functional Analysis}, second ed., p.~310, Theorem 12.8.}
because $\omega$ is a bounded bilinear form there is a unique $A \in \mathscr{L}(H)$ such that
\begin{equation}
\omega(x,y) = \inner{Ax}{y}, \qquad x,y \in H.
\label{riesz}
\end{equation}
Because $\omega$ is skew-symmetric,
\[
\inner{Ax}{y} = \omega(x,y) = -\omega(y,x) = -\inner{Ay}{x} = \inner{(-A)y}{x}.
\]
On the other hand, because $\inner{\cdot}{\cdot}$ is a real inner product,
 $\inner{Ax}{y}=\inner{x}{A^*y}=\inner{A^*y}{x}$. Therefore $A^*=-A$. 

$A^*A = (-A)A=-A^2$ and $AA^*=A(-A)=-A^2$, so $A$ is normal. 
Therefore $A$ has a \textbf{polar decomposition}:\footnote{Walter Rudin,
{\em Functional Analysis}, second ed., p.~332, Theorem 12.35.}
there is a unitary $U \in \mathscr{L}(H)$ and some $P \in \mathscr{L}(H)$ with
$P \geq 0$,
 such that
\[
A=UP,
\]
and such that $A,U,P$ commute; a fortiori, $P$ is self-adjoint.
If $Ax=0$, then $\omega(x,y)=\inner{Ax}{y}=\inner{0}{y}=0$ for all $y \in H$, and because $\omega$ is weakly nondegenerate this implies
that $x=0$, hence $A$ is one-to-one, which implies that $P$ is one-to-one (this implication
does not  use that $U$ is unitary). 
We have
\[
A^*=(UP)^*=P^*U^*=PU^*, \qquad A^*=-A=-UP=-PU,
\]
hence
\[
PU^*=P(-U).
\]
Because $P$ is one-to-one, this yields $U^*=-U$.
But $U$ is unitary, i.e. $U^*U=I$ and $UU^*=I$. 
Therefore $(-U)U=I$, i.e. $-U^2=I$. 
This means that $U$ is a complex structure on the real Hilbert space $H$. We write $J=U$.

The complex structure $J$ satisfies, for $x,y \in H$,
\[
\omega(Jx,Jy) = \inner{AJx}{Jy} = \inner{JAx}{Jy} = \inner{Ax}{J^*Jy}
=\inner{Ax}{y}=\omega(x,y),
\]
showing that $J$ is a canonical transformation.

$s:H \times H \to \mathbb{R}$ is defined, for $x,y \in H$, by
\[
s(x,y) = -\omega(Jx,y) = -\inner{AJx}{y} = \inner{(-J)Ax}{y}=
\inner{J^{-1}Ax}{y}
=\inner{Px}{y}.
\]
It is apparent that $s$ is bilinear. Because $P$ is self-adjoint and $\inner{\cdot}{\cdot}$ is symmetric,
\[
s(x,y) = \inner{Px}{y} = \inner{x}{Py} = \inner{Py}{x} = s(y,x),
\]
showing that $s$ is symmetric.
Because $P \geq 0$, for any $x \in H$ we have $s(x,x)=\inner{Px}{x} \geq 0$, namely $s$ is positive.
Also because $P \geq 0$, there is a unique $S \in \mathscr{L}(H)$, $S \geq 0$, satisfying
$S^2=P$.\footnote{Walter Rudin, {\em Functional Analysis}, second ed.,
p.~331, Theorem 12.33.}
If $s(x,x)=0$, we get
\[
0=\inner{Px}{x}=\inner{S^2x}{x}=\inner{Sx}{Sx}=\norm{Sx}^2,
\]
hence $Sx=0$ and so $Px=0$, and because $P$ is one-to-one, $x=0$.
Therefore
$s$ is positive definite, and thus is a real inner product on $H$. 

$h:H \times H \to \mathbb{C}$ is defined, for $x,y \in H$, by
\[
h(x,y) = s(x,y)- i\omega(x,y) = \inner{Px}{y}- i\omega(x,y)
=\inner{Px}{y}-i\inner{Ax}{y}.
\]
For $x_1,x_2, y \in H$,
\[
h(x_1+x_2,y) = h(x_1,y)+h(x_2,y).
\]
For $\alpha+i\beta \in \mathbb{C}$,
\begin{align*}
h((\alpha+i\beta)x,y)&=h(\alpha x+\beta Jx,y)\\
&=h(\alpha x,y)+\beta h(Jx,y)\\
&=\alpha h(x,y) + \beta \inner{PJx}{y}- i\beta \inner{AJx}{y}\\
&=\alpha h(x,y) + \beta \inner{Ax}{y}-i\beta \inner{A(-J^{-1})x}{y}\\
&=\alpha h(x,y) + \beta \omega(x,y) + i\beta \inner{Px}{y}\\
&=\alpha h(x,y) + \beta \omega(x,y) + i\beta s(x,y)\\
&=\alpha h(x,y) + i\beta (s(x,y)-i\omega(x,y))\\
&=\alpha h(x,y) + i\beta h(x,y)\\
&=(\alpha+i\beta) h(x,y).
\end{align*}
Therefore $h$ is complex linear in its first argument. 
Because $s$ is symmetric and $\omega$ is skew-symmetric, $h(x,y) = s(x,y)-i\omega(x,y)$ satisfies
\[
h(y,x) = s(y,x)-i\omega(y,x) = s(x,y) + i \omega(x,y) = \overline{h(x,y)},
\]
showing that $h$ is conjugate symmetric. For $x \in H$,
\[
h(x,x) = s(x,x)-i\omega(x,x) = s(x,x) \geq 0.
\]
If $h(x,x)=0$, then
$s(x,x)=0$, which implies that $x=0$. Therefore $h$ is a complex inner product on $H$ with the complex structure $J$.

Suppose that $\omega$ is a strong symplectic form on the real Hilbert space $H$. 
That is, $\omega^\flat:H \to H^*$ is an isomorphism of Banach spaces. We shall show that
$A$, from \eqref{riesz}, is onto. For $y \in H$, define $\lambda:H \to \mathbb{R}$ by
$\lambda(x) = \inner{x}{y}$. Then $\lambda \in H^*$, so there is some
$v \in H$ for which $\omega^\flat(v)=\lambda$. That is,
$\omega(v,x)=\lambda(x)=\inner{x}{y}=\inner{y}{x}$ for all $x \in H$. 
But $\omega(v,x) = \inner{Av}{x}$, so 
$\inner{Av}{x}=\inner{y}{x}$ for all $x \in H$, which implies that $Av=y$, and thus shows that
$A$ is onto, and hence invertible in $\mathscr{L}(H)$. 
Because $A=UP$ and $A,U$ are invertible in $\mathscr{L}(H)$, $P$ is invertible
in $\mathscr{L}(H)$.
Therefore $S$, $P=S^2$, $S \geq 0$, is invertible in $\mathscr{L}(H)$, whence
\begin{align*}
\norm{x}^2 &= \norm{S^{-1} Sx}^2\\
&\leq \norm{S^{-1}}^2 \norm{Sx}^2\\
&=\norm{S^{-1}}^2 \inner{Sx}{Sx}\\
&=\norm{S^{-1}}^2 \inner{Px}{x}\\
&=\norm{S^{-1}}^2 s(x,x)\\
&=\norm{S^{-1}} \norm{x}_s^2,
\end{align*}
and on the other hand
\[
\norm{x}_s^2=s(x,x) = \inner{Px}{x} \leq \norm{Px} \norm{x} \leq \norm{P} \norm{x}^2=\norm{S}^2 \norm{x}^2.
\]
so
\[
\norm{x} \leq \norm{S^{-1}} \norm{x}_s, \qquad \norm{x}_s \leq 
\norm{S} \norm{x}.
\]
Namely this establishes that  the norms $\norm{x}^2 = \inner{x}{x}$ and
$\norm{x}_s^2 = s(x,x)$ are equivalent.
\end{proof}


\section{Hamiltonian vector fields}
Let $E$ be a real Banach space and let $k \geq 1$; if we do not specify $k$ we merely suppose that it is $\geq 1$.
A \textbf{$C^k$ vector field on $U$}, where $U$ an open subset of $E$, is a $C^k$ function $v:U \to E$.

Let $v$ be a $C^k$, $k \geq 1$, vector field on $E$.
For $x \in E$, an \textbf{integral curve of $v$ through $x$} is a differentiable function $\phi:J \to E$,
where $J$ is some open interval in $\mathbb{R}$ containing $0$, that satisfies
\[
\phi'(t) = (v \circ \phi)(t), \qquad t \in J, \qquad \phi(0)=x.
\]
If $\psi:I \to E$ and $\phi:J \to E$ are integral curves of $v$ through $x$, it is proved that 
for $t \in I \cap J$, $\psi(t)=\phi(t)$.\footnote{Rodney Coleman, {\em Calculus on Normed Vector Spaces}, p.~194, Proposition 9.3.}
An integral curve of $v$ through $x$, $\phi:J \to E$, is said to be \textbf{maximal} if
there is no integral curve of $v$ through $x$ whose domain strictly includes $J$. 
If $X:E \to E$ is a $C^1$ vector field,  for each $x \in E$ it is proved that there is a unique maximal integral curve of $v$ through $x$, denoted
$\phi_x:J_x \to E$.\footnote{Rodney Coleman, {\em Calculus on Normed Vector Spaces}, p.~194, Theorem 9.2.}
A vector field $v:E \to E$ is called \textbf{complete} when $J_x=\mathbb{R}$ for each $x \in E$. 
For a vector field $v:E \to E$, a $C^1$ function $f:E \to \mathbb{R}$ is called a \textbf{first integral of $v$} if
for any integral curve $\phi:J \to E$ of $v$, $f \circ \phi:J \to E$ is constant. 
It is proved that if a vector field has a first integral $f:E \to \mathbb{R}$ such that
$f^{-1}(c)$ is a compact subset of $E$ for each $c \in \mathbb{R}$, then
$v$ is a complete vector field.\footnote{Rodney Coleman, {\em Calculus on Normed Vector Spaces}, p.~207, Theorem 9.8.}

The \textbf{flow of $v$} is the function $\sigma:\Sigma_v \to E$, where
\[
\Sigma_v=\bigcup_{x \in E} J_x \times \{x\},
\]
such that for each $x \in E$, 
$\sigma(t,x)=\phi_x(t)$, $t \in J_x$.
It is proved that $\Sigma_v$ is an open subset of $\mathbb{R} \times E$, and that $\sigma:\Sigma_v \to E$ is continuous.\footnote{Rodney Coleman, {\em Calculus on Normed Vector Spaces}, p.~213, Theorem 10.1.}
It is also proved that for any $k \geq 1$, if $v$ is $C^k$ then
$\sigma:\Sigma_v \to E$ is $C^k$.\footnote{Rodney Coleman, {\em Calculus on Normed Vector Spaces}, p.~222, Theorem 10.3.}
If $(s,x),(t,\sigma(s,x)),(t+s,x) \in \Sigma_v$, then\footnote{Yvonne Choquet-Bruhat and Cecile DeWitt-Morette, {\em Analysis, Manifolds and Physics, Part I}, p.~551.}
\[
\sigma(t+s,x) = \sigma(t,\sigma(s,x)).
\]

When $v$ is a complete vector field, its flow is called a \textbf{global flow}. 
In this case, for $t \in \mathbb{R}$ we define $\sigma_t:E \to E$ by $\sigma_t(x)=\sigma(t,x)$. 
Then $\sigma_t^{-1}=\sigma_{-t}$, and thus each $\sigma_t$ is a $C^k$ diffeomorphism $E \to E$. 




\section{Differential forms}
For vector spaces $V$ and $W$ and for $p \geq 1$, a function $f:V^p \to W$ is called
\textbf{alternating} if 
$(v_1,\ldots,v_p) \in V^p$ and
$v_i=v_{i+1}$ for some $1 \leq i \leq p-1$ 
imply that $f(v_1,\ldots,v_p)=0$.

For Banach spaces $E$ and $F$ and for $p \geq 1$,
we denote by $\mathscr{A}_p(E;F)$ the set of alternating elements of
$\mathscr{L}_p(E;F)$. In particular, $\mathscr{A}_1(E;F)=\mathscr{L}_1(E;F)=
\mathscr{L}(E;F)$. $\mathscr{A}_p(E;F)$ is a closed linear subspace
of the Banach space $\mathscr{L}_p(E;F)$.\footnote{Henri Cartan,
{\em Differential Forms}, p.~9.}
We define
\[
\mathscr{A}_0(E;F)=\mathscr{L}_0(E;F)=F.
\] 

Let $\Sigma_n$ be the set of permutation $\{1,\ldots,n\}$, which has $n!$ elements.
Let $\Sh_{p,q}$ be the set of permutations $\sigma$ of $\{1,\ldots,p,p+1,\ldots,p+q\}$ for which
\[
\sigma(1)<\cdots<\sigma(p), \qquad \sigma(p+1)<\cdots<\sigma(p+q).
\]
The set $\Sh_{p,q}$ has $\binom{p+q}{p}=\binom{p+q}{q}$ elements. 

For $f \in \mathscr{A}_p(E;\mathbb{R})$ and $g \in \mathscr{A}_q(E;\mathbb{R})$,
we define $f \wedge g:E^p \times E^q \to \mathbb{R}$ by
\[
\begin{split}
&(f \wedge g)(x_1,\ldots,x_p,x_{p+1},\ldots,x_{p+q})\\
=&\sum_{\sigma \in \Sh_{p,q}} \sgn(\sigma) f(x_{\sigma(1)},\ldots,x_{\sigma(p)})
g(x_{\sigma(p+1)},\ldots,x_{\sigma(p+q)}).
\end{split}
\]
It is proved that $f \wedge g \in \mathscr{A}_{p+q}(E;\mathbb{R})$.\footnote{Henri Cartan,
{\em Differential Forms}, pp.~12--14.}

For $f \in \mathscr{A}_p(E;\mathbb{R})$ and
$g \in \mathscr{A}_q(E;\mathbb{R})$,
\begin{align*}
\norm{f \wedge g}&=\sup_{\norm{x_1} \leq 1,\ldots, \norm{x_{p+q}} \leq 1}
|(f \wedge g)(x_1,\ldots,x_p,x_{p+1},\ldots,x_{p+q})|\\
&\leq \sup_{\norm{x_1} \leq 1,\ldots, 
\norm{x_{p+q}} \leq 1}
\sum_{\sigma \in \Sh_{p,q}} |f(x_{\sigma(1)},\ldots,x_{\sigma(p)})
g(x_{\sigma(p+1)},\ldots,x_{\sigma(p+q)})|\\
&\leq  \sup_{\norm{x_1} \leq 1,\ldots, \norm{x_{p+q}} \leq 1}
\sum_{\sigma \in \Sh_{p,q}} \norm{f} \norm{g}\\
&=\binom{p+q}{p} \norm{f} \norm{g},
\end{align*}
showing that the operator norm of the bilinear map $(f,g) \mapsto f \wedge g$,
$\mathscr{A}_p(E;\mathbb{R}) \times \mathscr{A}_q(E;\mathbb{R})$ is
$\leq \binom{p+q}{p}$, and thus is continuous.

One proves that for $f \in \mathscr{A}_p(E;\mathbb{R})$ and $g \in \mathscr{A}_q(E;\mathbb{R})$,
then\footnote{Henri Cartan,
{\em Differential Forms}, p.~14, Proposition 1.5.1.}
\[
g \wedge f = (-1)^{pq} f \wedge g.
\]
It is also proved that for $f \in \mathscr{A}_p(E;\mathbb{R})$, 
$g \in \mathscr{A}_q(E;\mathbb{R})$, and
$h \in \mathscr{A}_r(E;\mathbb{R})$, then\footnote{Henri Cartan,
{\em Differential Forms}, p.~15, Proposition 1.5.2.}
\[
(f \wedge g) \wedge h = f \wedge (f \wedge h).
\]
It thus makes sense to speak about $f_1 \wedge  \cdots \wedge f_n$. 
We remind ourselves that $\mathscr{A}_1(E;\mathbb{R})=\mathscr{L}(E;\mathbb{R})=E^*$.
It is proved that if $f_1,\ldots,f_n \in E^*$, then $f_1 \wedge \cdots \wedge f_n
\in \mathscr{A}_n(E;\mathbb{R})$ satisfies
\[
f_1 \wedge \cdots \wedge f_n (x_1,\ldots,x_n) = \sum_{\sigma \in \Sigma_n}
\sgn(\sigma) f_1(x_{\sigma(1)}) \cdots f_n(x_{\sigma(n)}),
\quad (x_1,\ldots,x_n) \in E^n,
\]
and that $f_1,\ldots,f_n \in E^*$ are linearly independent if and only if
$f_1 \wedge \cdots \wedge f_n=0$.\footnote{Henri Cartan,
{\em Differential Forms}, p.~16, Proposition 1.6.1.}


Let $U$ be an open subset of the Banach space $E$. 
For $k \geq 0$ and $p \geq 0$, a \textbf{$C^k$ differential form of degree $p$ on $U$}
is a $C^k$ function
\[
\alpha: U \to \mathscr{A}_p(E;\mathbb{R}). 
\]
We abbreviate ``differential form of degree $p$'' as ``differential $p$-form''.
In particular, a $C^k$ differential $0$-form is a $C^k$ function $U \to \mathscr{A}_0(E;\mathbb{R})
=\mathbb{R}$.
We denote by $\Omega_p^{(k)}(U,\mathbb{R})$ the set of $C^k$ differential
$p$-forms on $U$. It is apparent that this is a real vector space. 

For a $C^k$ function $f:U \to \mathbb{R}$, with $k \geq 1$,
the derivative $f'$ is $C^{k-1}$ function
$U \to \mathscr{L}(E;\mathbb{R})=\mathscr{A}_1(E;\mathbb{R})$, hence
$f' \in \Omega_p^{(k-1)}(U)$. 

For $\alpha \in \Omega_p^{(k)}(U,\mathbb{R})$ and 
$\beta \in \Omega_q^{(k)}(U,\mathbb{R})$, we define
$\alpha \wedge:U \to \mathscr{A}_{p+q}(E;\mathbb{R})$ by
\[
(\alpha \wedge \beta)(x) = (\alpha(x)) \wedge (\beta(x)), \qquad x \in U.
\]
It is proved that $\alpha \wedge \beta \in \Omega_{p+q}^{(k)}(U,\mathbb{R})$.\footnote{Henri Cartan,
{\em Differential Forms}, p.~19, \S 2.2.}

Suppose that $k \geq 1$ and $\alpha \in \Omega_p^{(k)}(U,\mathbb{R})$, i.e.
$\alpha:U \to \mathscr{A}_p(U;\mathbb{R})$ is a $C^k$ function. Then
the derivative is the $C^{k-1}$ function
\[
\alpha':U \to \mathscr{L}(E;\mathscr{A}_p(E;\mathbb{R}).
\]
We define $d\alpha:U \to \mathscr{A}_{p+1}(E;\mathbb{R})$ by
\[
(d\alpha)(x)(\xi_0,\xi_1,\ldots,\xi_p)
=\sum_{i=0}^p (-1)^i \alpha'(x)(\xi_i)(\xi_0,\ldots,\hat{\xi}_i,\ldots,\xi_p)
\]
It is proved that $d\alpha \in \Omega_{p+1}^{(k-1)}(U,\mathbb{R})$.\footnote{Henri Cartan,
{\em Differential Forms}, pp.~20--21, \S 2.3.}

In particular, if $f:U \to \mathbb{R}$ is a $C^k$ function, then
$df \in \Omega_1^{(k-1)}(U,\mathbb{R})$ is the function
$df:U \to \mathscr{A}_1(E;\mathbb{R})=\mathscr{L}(E;\mathbb{R})$ defined by
\[
(df)(x)(\xi) = f'(x)(\xi), \qquad x \in U, \quad \xi \in E.
\]
Thus, $df=f'$. 

For $\alpha \in \Omega_p^{(k)}(U,\mathbb{R})$ and $\beta \in \Omega_q^{(k)}(U,\mathbb{R})$
with $k \geq 1$, it is a fact that\footnote{Henri Cartan,
{\em Differential Forms}, p.~22, Theorem 2.4.2.}
\[
d(\alpha \wedge \beta) = (d\alpha) \wedge \beta + (-1)^p \alpha \wedge(d\beta).
\]
In particular, an element $f$ of $\Omega_0^{(k)}(U,\mathbb{R})$ is a $C^k$ function
$U \to \mathbb{R}$, for which, because $f \wedge \beta = f \beta$,  
\[
d(f \beta) = (df) \wedge \beta + f (d\beta).
\]

For $\alpha \in \Omega_p^({k)}(U,\mathbb{R})$, with $k \geq 2$,\footnote{Henri Cartan,
{\em Differential Forms}, p.~23, Theorem 2.5.1.}
\[
d(d\alpha)=0.
\]

Let $\alpha \in \Omega_p^{(k)}(U,\mathbb{R})$, let
$V$ be an open subset of a Banach space $F$, and let 
$\phi:V \to U$ be a $C^{k+1}$ function. 
The \textbf{the pullback of $\alpha$ by $f$}, denoted
$\phi^* \alpha:V \to \mathscr{A}_p(F;\mathbb{R})$, is an element of
$\Omega_p^{(k)}(V,\mathbb{R})$ 
satisfying\footnote{Henri Cartan,
{\em Differential Forms}, p.~29, Proposition 2.8.1.}
\[
(\phi^* \alpha)(y)(\eta_1,\ldots,\eta_p) = \alpha(\phi(y))(\phi'(y)(\eta_1),\ldots,
\phi'(y)(\eta_p)),
\quad (\eta_1,\ldots,\eta_p) \in F^p.
\]
The pullback satisfies, for $\alpha \in \Omega_p^{(k)}(U,\mathbb{R})$
and $\beta \in \Omega_q^{(k)}(U,\mathbb{R})$,
\[
\phi^*(\alpha \wedge \beta)  = (\phi^* \alpha) \wedge (\phi^* \beta),
\]
which is an element of $\Omega_{p+q}^{(k)}(V,\mathbb{R})$. It also satisfies,
if $\phi:V \to U$ and $f:U \to \mathbb{R}$ are $C^1$, 
\[
\phi^*(df) = d(\phi^*f),
\]
where $(\phi^*f)(y) = f(\phi(y))$. 



\section{Contractions and Lie derivatives}
Let $U$ be an open subset of a Banach space $E$, let $k \geq 1$, $p \geq 1$, let
$v$ be a $C^k$ vector field on $U$, and let $\alpha \in \Omega_p^{(k)}(U,\mathbb{R})$. 
We define $\iota_v \alpha:U \to \mathscr{A}_{p-1}(E;\mathbb{R})$ by
\[
(\iota_v \alpha)(x)(v_1,\ldots,v_{p-1}) = \alpha(v(x),v_1,\ldots,v_{p-1}), \qquad (v_1,\ldots,v_{p-1}) \in E^{p-1}.
\]
(It is straightforward to check that indeed $(\iota_v \alpha)(x) \in \mathscr{A}_{p-1}(E;\mathbb{R})$.)
It is proved that $\iota_v \alpha:U \to \mathscr{A}_{p-1}(E;\mathbb{R})$ is $C^k$, and thus
$\iota_v \alpha \in \Omega_{p-1}^{(k)}(U,\mathbb{R})$.\footnote{cf. Serge Lang,
{\em Differential and Riemannian Manifolds}, p.~137, V, \S 5.}
For $p=0$, with $f \in \Omega_0^{(k)}(U,\mathbb{R})$, i.e. $f$ is a $C^k$ function
$U \to \mathbb{R}$,
we define $\iota_v f = 0$.
We call $\iota_v \alpha$ the \textbf{contraction of $\alpha$ by $v$}.

It can be proved that if $\alpha \in \Omega_p^{(k)}(U,\mathbb{R})$
and $\beta \in \Omega_q^{(k)}(U,\mathbb{R})$,
\[
\iota_v(\alpha \wedge \beta) = (\iota_v \alpha) \wedge \beta + (-1)^p \alpha \wedge \iota_v \beta.
\]
Also, for a $C^k$ vector field $w$ on $U$,
\[
\iota_v(\iota_w \alpha) = - \iota_w(\iota_v \alpha),
\]
and hence $\iota_v^2 \alpha=0$. And $(v,\alpha) \mapsto \iota_v \alpha$ is bilinear.

For a $C^k$ vector field $v$ on $U$ and $\alpha \in \Omega_p^{(k)}(U,\mathbb{R})$, the \textbf{Lie
derivative of $\alpha$ with respect to $v$} is\footnote{cf. Serge Lang,
{\em Differential and Riemannian Manifolds}, pp.~138--141, V, \S 5.}
\[
\mathscr{L}_v \alpha  = d(\iota_v \alpha)+\iota_v d\alpha \in \Omega_p^{(k)}(U,\mathbb{R}).
\]
The Lie derivative satisfies 
\[
\mathscr{L}_v(\alpha \wedge \beta) = (\mathscr{L}_v \alpha) \wedge \beta + \alpha \wedge \mathscr{L}_v
\beta.
\]

If $\omega$ is a weak symplectic form on a Banach space $E$ and 
$v$ is a $C^1$ vector field on $E$,
we say that $v$ is a \textbf{symplectic vector field} if
\[
\mathscr{L}_v \omega = 0.
\]
If there is some $C^1$ function $H:E \to E$ such that
\[
\iota_v \omega = - dH,
\]
we say that $v$ is a \textbf{Hamiltonian vector field with Hamiltonian function $H$}.
If $v$ is a Hamiltonian vector field with Hamiltonian function $H$, then
\[
\mathscr{L}_v \omega = d(\iota_v \omega) + \iota_v d\omega
=d(\iota_v \omega)
=d(-dH)
=-d^2H=0,
\]
showing that if a vector field is Hamiltonian then it is symplectic. (This is analogous to the statement that if a differential form is exact
then it is closed.)

\end{document}