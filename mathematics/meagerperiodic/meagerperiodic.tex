\documentclass{article}
\usepackage{amsmath,amssymb,mathrsfs,amsthm}
%\usepackage{tikz-cd}
%\usepackage{hyperref}
\newcommand{\inner}[2]{\left\langle #1, #2 \right\rangle}
\newcommand{\tr}{\ensuremath\mathrm{tr}\,} 
\newcommand{\Span}{\ensuremath\mathrm{span}} 
\def\Re{\ensuremath{\mathrm{Re}}\,}
\def\Im{\ensuremath{\mathrm{Im}}\,}
\newcommand{\id}{\ensuremath\mathrm{id}} 
\newcommand{\var}{\ensuremath\mathrm{var}} 
\newcommand{\Lip}{\ensuremath\mathrm{Lip}} 
\newcommand{\GL}{\ensuremath\mathrm{GL}} 
\newcommand{\diam}{\ensuremath\mathrm{diam}} 
\newcommand{\sgn}{\ensuremath\mathrm{sgn}\,} 
\newcommand{\lcm}{\ensuremath\mathrm{lcm}} 
\newcommand{\supp}{\ensuremath\mathrm{supp}\,}
\newcommand{\dom}{\ensuremath\mathrm{dom}\,}
\newcommand{\upto}{\nearrow}
\newcommand{\downto}{\searrow}
\newcommand{\norm}[1]{\left\Vert #1 \right\Vert}
\newtheorem{theorem}{Theorem}
\newtheorem{lemma}[theorem]{Lemma}
\newtheorem{proposition}[theorem]{Proposition}
\newtheorem{corollary}[theorem]{Corollary}
\theoremstyle{definition}
\newtheorem{definition}[theorem]{Definition}
\newtheorem{example}[theorem]{Example}
\begin{document}
\title{Meager sets of periodic functions}
\author{Jordan Bell}
\date{February 6, 2015}

\maketitle

The following is often useful.\footnote{Walter Rudin, {\em Real and Complex Analysis}, third ed.,
p.~68, Theorem 3.12.}

\begin{theorem}
If $(X,\mu)$ is a measure space, $1 \leq p \leq \infty$, and $f_n \in L^p(\mu)$ is a sequence
that converges in $L^p(\mu)$ to some $f \in L^p(\mu)$, then there is a subsequence
of $f_n$ that converges pointwise almost everywhere to $f$.
\end{theorem}
\begin{proof}
Assume that $1 \leq p < \infty$. For each $n$ there is some $a_n$ such that
\[
\norm{f_{a_n}-f}_p < 2^{-n}.
\]
Then
\[
\sum_{n=1}^\infty \norm{f_{a_n}-f}_p^p  < \sum_{n=1}^\infty 2^{-np}
= \frac{2^{-p}}{1-2^{-p}}<\infty.
\]

Let $\epsilon>0$. We have
\[
\left\{x \in X: \limsup_{n \to \infty} |f_{a_n}(x)-f(x)|>\epsilon\right\}
\subset \bigcap_{N=1}^\infty \bigcup_{n = N}^\infty \left\{x \in X: |f_{a_n}(x)-f(x)|>\epsilon\right\}.
\]
For any $N$, this gives, using Chebyshev's inequality, 
\[
\begin{split}
&\mu\left(\left\{x \in X: \limsup_{n \to \infty} |f_{a_n}(x)-f(x)|>\epsilon\right\}\right)\\
\leq&
\sum_{n=N}^\infty \mu\left( \left\{x \in X: |f_{a_n}(x)-f(x)|>\epsilon\right\}\right)\\
\leq& \epsilon^{-p}  \sum_{n=N}^\infty\norm{f_{a_n}-f}_p^p.
\end{split}
\]
Because $\sum_{n=1}^\infty \norm{f_{a_n}-f}_p^p<\infty$,  we have
$\sum_{n=N}^\infty\norm{f_{a_n}-f}_p^p \to 0$ as $N \to \infty$, which implies that
\[
\mu\left(\left\{x \in X: \limsup_{n \to \infty} |f_{a_n}(x)-f(x)|>\epsilon\right\}\right)
=0.
\]
This is true for each $\epsilon>0$, hence
\[
\mu\left(\left\{x \in X: \limsup_{n \to \infty} |f_{a_n}(x)-f(x)|>0\right\}\right)=0,
\]
which means that for almost all $x \in X$,
\[
\lim_{n \to \infty} |f_{a_n}(x)-f(x)| =0.
\]

Assume that $p=\infty$. Let
\[
E_k = \{x \in X: |f_k(x)|>\norm{f_k}_\infty\}.
\]
The measure of each of these sets is $0$, so for
\[
E = \bigcup_k E_k
\]
we have $\mu(E)=0$.
For $x \not \in E$,
\[
|f(x)-f_k(x)| \leq \norm{f-f_k}_\infty \to 0, \qquad k \to \infty,
\]
showing that for almost all $x \in X$, $f_k(x) \to f(x)$.
\end{proof}



The following results are in the pattern of $A$ being a strict subset of $X$ implying
that $A$ is meager in $X$.

We first work out two proofs of the following theorem.

\begin{theorem}
For $1 < p \leq \infty$, $L^p(\mathbb{T})$ is a meager subset of $L^1(\mathbb{T})$.
\end{theorem}
\begin{proof}
For $n \geq 1$, let
\[
C_n = \left\{f \in L^1(\mathbb{T}): \norm{f}_p \leq n \right\}.
\]
Let $n \geq 1$. If a sequence $f_k \in C_n$ converges in $L^1(\mathbb{T})$ to some
$f \in L^1(\mathbb{T})$, 
then there is a subsequence
$f_{a_k}$ of $f_k$
such that for almost all $x \in \mathbb{T}$,
$f_{a_k}(x) \to f(x)$, and so $f_{a_k}(x)^p \to f(x)^p$.
Applying the dominated convergence theorem gives
\[
\frac{1}{2\pi} \int_{\mathbb{T}} |f(x)|^p dx 
= \lim_{k \to \infty} \frac{1}{2\pi} \int_{\mathbb{T}} |f_{a_k}(x)|^p dx
=\lim_{k \to \infty} \norm{f_{a_k}}_p^p
\leq n^p,
\]
hence $\norm{f}_p \leq n$, showing that $f \in C_n$. Therefore, $C_n$ is a closed subset of $L^1(\mathbb{T})$
On the other hand,
let $f \in C_n$ and
 let $g \in L^1(\mathbb{T}) \setminus L^p(\mathbb{T})$. 
Then $f+\frac{1}{k}g \to f$ in $L^1(\mathbb{T})$, and for each $k$ we have
$f+\frac{1}{k}g \not \in C_n$, as that would imply 
$g \in L^p(\mathbb{T})$.
This shows that $f$ does not belong to the interior of $C_n$. 
Because $C_n$ is closed and has empty interior, it is nowhere dense. 
Therefore
\[
L^p(\mathbb{T}) = \bigcup_{n=1}^\infty  \left\{f \in L^1(\mathbb{T}): \norm{f}_p \leq n \right\}
\]
is meager in $L^1(\mathbb{T})$.
\end{proof}


\begin{proof}
The open mapping theorem
tells us that if $X$ is an $F$-space, $Y$ is a topological vector space,
$\Lambda:X \to Y$ is continuous and linear, and $\Lambda(X)$ is not meager
in $Y$, then $\Lambda(X)=Y$, $\Lambda$ is an open mapping, and $Y$ is an $F$-space.\footnote{Walter Rudin,
{\em Functional Analysis}, second ed., p.~48, Theorem 2.11.}

Let $j:L^p(\mathbb{T}) \to L^1(\mathbb{T})$ be the inclusion map.
For $f \in L^p(\mathbb{T})$,
\[
\norm{j(f)}_1  = \norm{f}_1 \leq \norm{f}_p,
\]
showing that the inclusion map is continuous. On the other hand, 
$j$ is not onto, so the open mapping theorem tells us that $j(L^p(\mathbb{T}))=L^p(\mathbb{T})$ is meager in $L^1(\mathbb{T})$.
\end{proof}


Suppose that $X$ is a topological vector space, that $Y$ is an $F$-space, and that $\Lambda_n$ is a sequence of continuous linear maps
$X \to Y$. Let $L$ be the set of those $x \in X$ such that
\[
\Lambda x = \lim_{n \to \infty} \Lambda_n x
\]
exists. It is a consequence of the uniform boundedness principle that if $L$ is not meager in $X$, then $L=X$ and $\Lambda:X \to Y$ is continuous.\footnote{Walter Rudin,
{\em Functional Analysis}, second ed., p.~45, Theorem 2.7.}

For $n \geq 1$, define $\Lambda_n:L^2(\mathbb{T}) \to \mathbb{C}$ by
\[
\Lambda_n f = \sum_{|k| \leq n} \hat{f}(k),
\qquad f \in L^1(\mathbb{T}).
\]
Define
\[
L=\left\{ f \in L^2(\mathbb{T}): \textrm{$\lim_{n \to \infty} \Lambda_n f$ exists}\right\}.
\]
The sequence $t \mapsto \sum_{k=1}^n \frac{e^{ikt}}{k}$ is a Cauchy sequence in $L^2(\mathbb{T})$, 
hence converges to some $f \in L^2(\mathbb{T})$, which satisfies
\[
\hat{f}(k) =\begin{cases}
 \frac{1}{k} & k \geq 1\\
 0&k \leq 0.
 \end{cases}
\]
Then
\[
\Lambda_n f = \sum_{k=1}^n \frac{1}{k} \to \infty, \qquad n \to \infty,
\]
meaning that $f \in L^2(\mathbb{T}) \setminus L$. This shows that
$L \neq L^2(\mathbb{T})$. Therefore,  the above consequence of the uniform boundedness principle
tells us that
$L$ is meager. 



\end{document}