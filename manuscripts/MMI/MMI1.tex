\documentclass{amsart}
\usepackage{amssymb,latexsym,amsmath,amsthm,mathrsfs}
%\usepackage{graphicx}
\newcommand{\Res}{\mathrm{Res}}
\newcommand{\Ei}{\mathrm{Ei}}
\def\Re{\ensuremath{\mathrm{Re}}\,}
\def\Im{\ensuremath{\mathrm{Im}}\,}
\newtheorem{theorem}{Theorem}
\newtheorem{remark}[theorem]{Remark}
\newtheorem{lemma}[theorem]{Lemma}
\begin{document}
\title{Lecture 1}
\author{Jordan Bell}
%\email{jordan.bell@gmail.com}
%\address{Toronto, Ontario, Canada}
\date{\today}
\maketitle

\textbf{page 20:} \textbf{(a)} Suppose it is now time $A$ years. At time $A+1$, 
Mary Beth pays 6500 $A+1$-dollars. At time $A+2$, Mary Beth pays
6500 $A+2$-dollars. At time $A+3$, Mary Beth pays 6500 $A+3$-dollars.

``What is the total present value of Mary‐Beth’s residence fees if the discount rate (interest rate) is 6 percent per year?''

$r=0.06$

The present value of 6500 $A+1$-dollars is in $A$-dollars (present dollars)
\[
PV_1 = \frac{6500}{1+0.06} = 6132.08 
\]
Saying this means that if at time $A$ (which we call the present)
we invest 6132.08 $A$-dollars (present dollars) then at time $A+1$ the value of this is
in $A+1$-dollars is
\[
6132.08(1+0.06) = 6500.
\]
The present value of 6500 $A+2$ dollars in $A$-dollars (present dollars) is
\[
PV_2 = \frac{6500}{(1+0.06)^2} = 5784.98.
\]
This means that if at time $A$ (the present) we invest 5784.98 $A$-dollars (present dollars) then at time $A+2$ the value of this in $A+2$-dollars is
\[
5784.98 \cdot (1+0.06)^2 = 6500.
\]
Likewise, the present value of 6500 $A+3$ dollars in $A$-dollars (present dollars) is
\[
PV_3 = \frac{6500}{(1+0.06)^3} = 5457.53.
\]

The 6500 $A$-dollars, 6500 $A+1$-dollars, and 6500 $A+2$-dollars are
the \textbf{cash flows}. Finding the present value (value in $A$-dollars) of these flows is called \textbf{discounting cash flows}.

The total present value (that is, value in $A$-dollars) of the cash flows (this could equivalently be called the
discounted value of the cash flows) 
\begin{align*}
PV &= PV_1+PV_2+PV_3\\
& = 6132.08 + 5784.98 + 5457.53 \\
&= 17374.59.
\end{align*}

\textbf{(b)} ``As an alternative, Mary-Beth is offered the option of paying a lower amount in her first year of \$6,250, and then increasing her payment by 5\% each year. Would she take this option?''

$C=6250$

$r=0.06$

$g=0.05$

The cash flows here are 6250 in $A+1$-dollars, then $(1+g) \cdot 6250 = 6562.50$ in
$A+2$-dollars, then $(1+g)^2 \cdot 6250=6890.63$ in $A+3$ dollars.

The present value (in $A$-dollars) of the cash flows is
\begin{align*}
PV&=\frac{C}{1+r} + \frac{(1+g)C}{(1+r)^2} + \frac{(1+g)^2 C}{(1+r)^3}\\
&= \frac{C}{1+r} \left( \frac{1+g}{1+r} + \left( \frac{1+g}{1+r} \right)^2 \right)\\
&= \frac{C}{1+r} \sum_{n=0}^2 \left(\frac{1+g}{1+r} \right)^n
\end{align*}

\textbf{Sum of geometric progression:}
\[
\sum_{n=0}^N x^n = \frac{1-x^{N+1}}{1-x}.
\]

Therefore
\begin{align*}
PV&= \frac{C}{1+r} \sum_{n=0}^2 \left(\frac{1+g}{1+r} \right)^n\\
&= \frac{C}{1+r} \cdot \frac{1- \left(\frac{1+g}{1+r} \right)^3}{1- \left(\frac{1+g}{1+r} \right)}\\
&= \frac{C}{1+r} \cdot \frac{1- \left(\frac{1+g}{1+r} \right)^3}{\frac{(1+r)-(1+g)}{1+r}}\\
&=\frac{C}{1+r} \cdot \frac{1- \left(\frac{1+g}{1+r} \right)^3}{\frac{r-g}{1+r}}\\
&= \frac{C}{1+r} \cdot \left(1- \left(\frac{1+g}{1+r} \right)^3 \right) \cdot
\frac{1+r}{r-g}\\
&=\frac{C}{r-g}\cdot \left(1- \left(\frac{1+g}{1+r} \right)^3 \right)
\end{align*}

I worked this out in general. For this problem we have the values of the
variables and we get
\begin{align*}
PV &= \frac{6250}{0.06-0.05} \cdot  \left(1- \left(\frac{1+0.05}{1+0.06} \right)^3 \right)\\
&=\frac{6250}{0.01} \cdot \left(1-\left(\frac{1.05}{1.06} \right)^3 \right)\\
&=17522.33.
\end{align*}

That is, the present value of the cash flows in scenario (b) is 
17522.33 $A$-dollars (present dollars).

The PV of the cash flows in (a) is 17374.59 and the PV of the cash flows
in (b) is 17522.33, so (b) has greater PV of cash flows. Since
Mary Beth is paying the cash flows, she chooses (a).

\textbf{In general:} For (b), we determined that for a growing annuity
with a payment of $C$ at time $A+1$, a payment of $C$ at time $A+2$,
etc., and a payment of $C$ at time $A+N$, the present value of the growing annuity
is 
\begin{equation}
PV=\frac{C}{r-g} \cdot \left(1-\left(\frac{1+g}{1+r} \right)^N \right).
\end{equation}

A simple annuity (in other words, an annuity without the adjective ``growing'') is indeed a growing annuity with $g=0$. Therefore the work we
did can be used to determine the present value of the simple
annuity in (a):
\begin{equation}
PV = \frac{C}{r} \cdot \left(1-\frac{1}{(1+r)^N} \right)
\end{equation}
Let us calculate the present value of the simple annuity (a) using this
formula:
\begin{align*}
PV &= \frac{6500}{0.06} \cdot \left(1-\frac{1}{1.06^3} \right)\\
&=17374.58.
\end{align*}
This is off by one cent from the present value we calculated directly already,
and this difference is from rounding during my calculations doing it
the first time.

\newpage

\textbf{page 23 (c)}``Using Q30 (Mary-Beth), calculate the following:
How would the PV of her costs change if she had to make her payments at the beginning of each year? For the annuity. For the growing annuity.''

For the annuity (a), the cash flows are now
6500 in $A$-dollars, then 6500 in $A+1$ dollars, then 6500 in $A+2$
dollars. The present value (in $A$-dollars) of these cash flows is
\begin{align*}
PV&=6500+\frac{6500}{1+0.06} + \frac{6500}{(1+0.06)^2}\\
&=6500 + 6132.08 + 5784.98\\
&=18417.06.
\end{align*}
(FYI: The present value of cash flows means the same thing as the discounted
value of the cash flows.)

For the growing annuity (b), the cash flows are
now 6250 in $A$-dollars, then $(1+0.05) \cdot 6250$ in $A+1$ dollars,
then $(1+0.05)^2 \cdot 6250$ in $A+2$ dollars. The present value of these
cash flows is, for $C=6250$,
\begin{align*}
PV&= C + \frac{(1+g)C}{1+r} + \frac{(1+g)^2C}{(1+r)^2}\\
&=C \sum_{n=0}^2 \left( \frac{1+g}{1+r} \right)^n
\end{align*}
Let $x=\frac{1+g}{1+r}$. So
\begin{align*}
PV&=C \sum_{n=0}^2 x^n\\
&= C \cdot \frac{1-x^3}{1-x}\\
&= C \cdot \frac{1-\left( \frac{1+g}{1+r} \right)^3}{1-\frac{1+g}{1+r}}\\
&=C \cdot \frac{1+r}{r-g} \cdot \left(1-\left( \frac{1+g}{1+r} \right)^3\right)
\end{align*}

That is, the present value of the cash flows for the growing annuity (b)
in $A$-dollars (present dollars) is
\begin{align*}
PV&=6250 \cdot \frac{1.06}{0.01} \cdot \left(1- \left( \frac{1.05}{1.06} \right)^3\right)\\
&=6250 \cdot 106 \cdot 0.0280357\ldots\\
&=18573.67.
\end{align*}

The PV of the annuity is 18417.06 in $A$-dollars (present dollars) and the PV of the growing
annuity is 18573.67 in $A$-dollars (present dollars). Therefore
Mary Beth will as before choose the annuity not the growing annuity, because she is paying and wants to pay
less so wants to pay the cash flows with the smaller present value.

\newpage

\textbf{page 26:} ``Find the PV and FV (in 10 years from now) of a growing annuity with payments of 100 in 1 year, growing at 8\% for 9 more years (i.e. 10 payments in total), if the relevant interest rate is 6\% per year. 
What if the payments were at the beginning of each year?''



\end{document}