\documentclass{article}
\usepackage{amsmath,amssymb,graphicx,subfig,mathrsfs,amsthm,enumitem,xfrac,flexisym}
\usepackage[polutonikogreek,english]{babel}
\newcommand{\Gk}[1]{\selectlanguage{polutonikogreek}#1\selectlanguage{english}}
\newcommand{\textoverline}[1]{$\overline{\mbox{#1}}$}
\newtheorem{theorem}{Theorem}
\newtheorem{lemma}[theorem]{Lemma}
\newtheorem{proposition}[theorem]{Proposition}
\newtheorem{corollary}[theorem]{Corollary}
\theoremstyle{definition}
\newtheorem{definition}[theorem]{Definition}
\newtheorem{example}[theorem]{Example}
\begin{document}
\title{Ancient metrology}
\author{Jordan Bell\\ \texttt{jordan.bell@gmail.com}\\Department of Mathematics, University of Toronto}
\date{\today}
\maketitle


The Oxford Handbook of the History of Physics
edited by Jed Z. Buchwald, Robert Fox

Medieval Latin: An Introduction and Bibliographical Guide


Minoan

Mycenaean

Cypriot

Cycladic

Egyptian

Ugarit

Hittite

Hurrian

Mittani

Armenian

Aramaic

Canaan

Phoenician

Akkadian

Assyrian

Ubaid

Sumerian

Elamite

Persian

Indus

Vedic

Indo-European

Semitic





Seleucid

\bibliographystyle{amsplain}
\bibliography{metrology}

\end{document}