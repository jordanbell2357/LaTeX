\documentclass{article}
\usepackage{amsmath,amssymb,mathrsfs,amsthm,xfrac}
\usepackage[LGR,T1]{fontenc}
\newcommand{\textgreek}[1]{\begingroup\fontencoding{LGR}\selectfont#1\endgroup}
%\usepackage{tikz-cd}
%\usepackage{hyperref}
\newcommand{\inner}[2]{\left\langle #1, #2 \right\rangle}
\newcommand{\tr}{\ensuremath\mathrm{tr}\,} 
\newcommand{\Span}{\ensuremath\mathrm{span}} 
\def\Re{\ensuremath{\mathrm{Re}}\,}
\def\Im{\ensuremath{\mathrm{Im}}\,}
\newcommand{\id}{\ensuremath\mathrm{id}} 
\newcommand{\gcm}{\ensuremath\mathrm{gcm}} 
\newcommand{\diam}{\ensuremath\mathrm{diam}} 
\newcommand{\sgn}{\ensuremath\mathrm{sgn}\,} 
\newcommand{\lcm}{\ensuremath\mathrm{lcm}} 
\newcommand{\supp}{\ensuremath\mathrm{supp}\,}
\newcommand{\dom}{\ensuremath\mathrm{dom}\,}
\newcommand{\norm}[1]{\left\Vert #1 \right\Vert}
\newcommand*\rfrac[2]{{}^{#1}\!/_{#2}}
\newtheorem{theorem}{Theorem}
\newtheorem{lemma}[theorem]{Lemma}
\newtheorem{proposition}[theorem]{Proposition}
\newtheorem{corollary}[theorem]{Corollary}
\begin{document}
\title{Denomination}
\author{Jordan Bell\\ \texttt{jordan.bell@gmail.com}\\Department of Mathematics, University of Toronto}
\date{\today}

\maketitle

Aristotle, {\em Categories} 1, 1a1 \cite[p.~3]{ackrill}, and Boethius:

\begin{quote}
When things have only a name in common and the definition of being which corresponds to the name is different, they are called {\em homonymous}. Thus, for example, both a man and a picture are animals. These have only a name in common and the definition of being which corresponds to the name is different ; for if one is to say what being an animal is for each of them, one will give
two distinct definitions.

{\em Aequivoca dicuntur quorum nomen solum commune est, secundum nomen vero substantiae ratio diversa, ut animal homo et quod pingitur. Horum enim solum nomen commune est, secundum nomen vero substantiae ratio diversa; si enim quis assignet quid est utrique eorum quo sint animalia, propriam assignabit utriusque rationem.}

When things have the name in common and the definition of being which corresponds to the name is the same, they are called {\em synonymous}. Thus, for example, both a man and an ox are animals. Each of these is called by a common name, `animal', and the definition of being is also the same; for if one is to give the definition of each?--what being an animal is for each of them--one will give
the same definition.

{\em Univoca vero dicuntur quorum et nomen commune est et secundum nomen eadem substantiae ratio, ut animal homo atque bos. Communi enim nomine utrique animalia nuncupantur, et est ratio substantiae eadem; si quis enim assignet utriusque rationem, quid utrique sit quo sint animalia, eandem assignabit rationem.}

When things get their name from something, with a difference of ending, they are called {\em paronymous}. Thus, for example, the grammarian gets his name from grammar, the brave get theirs from bravery.

{\em Denominativa vero dicuntur quaecumque ab aliquo, solo differentia casu, secundum nomen habent appellationem, ut a grammatica grammaticus et a fortitudine fortis.}
\end{quote}



Porphyry, {\em in Cat.}
\begin{quote}
A. He is saying that paronyms are those things that get their designation from a name by a change in its grammatical form. For example, 'bravery' is predicated of a certain virtue,\end{quote}



Boethius, {\em in Cat.}


Martianus Capella, {\em The Marriage of Philology and Mercury} V, 512 \cite[p.~191]{capellaII}:

\begin{quote}
Again, words are in a way transferred when
they express the whole by the part or the part by the whole or a 
plurality by one or an individual by the plural. The whole is expressed
by the part in `it struck at the helm'; or `that I can be
safe within the same walls as you,' when the meaning is `in the same
house.' The grammarians call this trope {\em metonymy}; the Greeks also
call it {\em catachresis}, or as we say, {\em abusio}; as when we use `the nature
of the gods' for `substance.' 
\end{quote}

paronymy, metonymy, {\em denominatio}

Duns Scotus, {\em Ordinatio} I, dist. 5, Peter Simpson:

\begin{quote}
23. About the other term of the major, namely that the predicate `is of necessity formally predicated about whatever it is predicated,' one must note that substantives can be doubly predicated in divine reality, sometimes formally and sometimes by identity; but adjectives, if they are predicated, are of necessity formally predicated, and this because they are adjectives, -- for, from the fact they are adjectives, they signify form by way of what informs; and so they are said denominatively of the subject, and consequently by way of what informs the subject, and thus they are said formally of it; of such sort are not only adjectival nouns but all participles and verbs.
\end{quote}

Anselm, {\em De Grammatico} 12: denominative terms are ``attributive adjectives in combination with a noun
they modify'', and ``stand-alone nouns'' \cite{anselm}.

``Denomination'' means assigning a name like ``white'' or ``running'' to a subject  \cite[p.~21]{wilson}. For example, 
in Sophisma 5, ``Omnis homo qui est albus currit'', of his {\em Sophismata}, William Heytesbury talks about when a man can be called ``white'':
``It is fist proposed that a thing should be called `white' if and only if every quantitative part of it is white. This proposal is rejected, however, since it would exclude all men from the class of white things: neither the flesh nor the blood is white. The same objection holds against the proposal that a thing be called `white' if more than one half of it is white. The correct rule, according to Heytesbury, is that a man is to be called `white' if and only if the external {\em surface} of the upper half of him is white.''  \cite[p.~22]{wilson}
(William Heytesbury was a fellow of Merton College, Oxford, in 1330, and was Chancellor of Oxford in 1371 \cite[p.~7]{wilson}.)
Wilson \cite[p.~23]{wilson} writes, ``Since the whiteness of an object may vary not only as to the area which it qualifies but also as to its intensity at any point on the surface
of the object, it is necessary to decide upon a further convention as to the {\em degree of intensity of whiteness} required for denominating an object {`white.'}''

To denominate the hotness of a ball means to assign a single label to the ball that names its hotness.
Supposing we are comfortable assigning a name to the hotness when each part of the ball has the same
hotness, if the hotness is difform how do we assign a single name? 

Peter of Spain, {\em Summaries of Logic} 3.1 \cite[p.~147]{copenhaver}:

\begin{quote}
Denominatives are said to be any that get their
designation from something else, differing only by termination in regard to that name,
as when `grammatical' comes from `grammar.' They differ only by a termination -- in
other words, only by an ending apart from the content -- and they get their designation
in regard to that name. At the beginning, then, a denominative name must coincide
with a univocal name, like `grammar' and `grammatical,' `white' and `whiteness.'
\end{quote}

Peter of Spain, {\em Summaries of Logic} 3.25 \cite[p.~163]{copenhaver}:

\begin{quote}
But those things are said to be {\em what-kind} that are said denominatively in regard to
this, like `grammatical' from `grammar' and `just' from `justice'--or else they are said
from some quality but not denominatively. And this happens in two ways: some are
said not denominatively from a quality in that a name has not been imposed for that
quality, as when a runner is not described denominatively because a name has not been
imposed for his quality; others are said to be {\em what-kind} yet not denominatively in that
they do not share the name of the quality from which they are called, even though a
name has been imposed, as when a person is called `diligent' from virtue. And so there
are three ways to get {\em what-kind} from quality.
\end{quote}

Nicomachus, {\em Introduction to Arithmetic} \cite{nicomachus}

{\em megethos}, \textgreek{m\'egejos}, greatness, magnitude

{\em mekos}, \textgreek{m\^{h}kos}, length

{\em pelikon}, \textgreek{phl\'ikon}, how large?

{\em poson}, \textgreek{p\'oson}, quantity, how many? This category in Aristotle covers both {\em plethos}, \textgreek{pl\~{h}jos}, plurality, 

{\em metron}, \textgreek{m\'etron}, measure, that by which something is measured

\textgreek{phl'ikos}: ``how great''

\textgreek{phlik'oths}, magnitude, size, {\em Elements} VI Def. 5 \cite[p.~189]{euclidII}.

\textgreek{pl'atos}: breadth, a dimension of a solid

\textgreek{poi\'os}, ``of what sort''

Cicero, {\em Orator} 92--94: metonymy {\em immutatio}

Quintilian, {\em Institutio Oratoria} 1.5.71 \cite{LCL124}:

\begin{quote}
metaphorical, when they are used in a sense different from their natural meaning. Current words are safest to use: there is a spice of danger in coining new. For if they are adopted, our style wins but small glory from them; while if they are rejected, they become a subject for jest.

propria sunt verba, cum id significant, in quod primo denominata sunt; translata, cum alium natura intellectum alium loco praebent. usitatis tutius utimur, nova non sine quodam periculo fingimus. nam si recepta sunt, modicam laudem adferunt orationi, repudiata etiam in iocos exeunt.
\end{quote}

Quintilian, {\em Institutio Oratoria} 8.6.23 \cite{LCL126}:

\begin{quote}
nec procul ab hoc genere discedit metonymia, quae est nominis pro nomine positio, sed, ut ait Cicero, hypallagen rhetores dicunt.
\end{quote}

{\em Rhetorica ad Herennium} 4.43 \cite[pp.~334--337]{LCL403}:

\begin{quote}
Metonymy is the figure which draws from an object closely akin or associated an expression suggesting the object meant, but not called by its own name. This is accomplished by substituting the name of the greater thing for that of the lesser, as if one speaking of the Tarpeian Rock should term it ``the Capitoline''; \dots; or by substituting the name of the thing invented for that of the inventor, as if one should say ``wine'' for ``Liber,'' ``wheat'' for ``Ceres''; ``\dots;'' or the instrument for the possessor, as if one should refer to the Macedonians
as follows: ``Not so quickly did the Lances get possession of Greece,'' and likewise, meaning the Gauls: ``nor was the Transalpine Pike so easily driven from Italy''; the cause for the effect, as if a speaker, wishing to show that some one has done something in war, should say: ``Mars forced you to do that''; or effect for cause, as when we call an art idle because it produces idleness in people, or speak of numb cold because cold produces numbness. Content will be designated by means of container as follows: ``Italy cannot be vanquished in warfare nor Greece in studies''; for here instead of Greeks and Italians the lands that comprise them are designated. Container will be designated by means of content: as if one wishing to give a name to wealth should call it gold or silver or ivory. It is harder to distinguish all these metonymies in teaching the principle than to find them when searching for them, for the use of metonymies of this kind is abundant not only amongst the poets and orators but also in everyday speech.

{\em Denominatio est quae ab rebus propinquis
et finitimis trahit orationem qua possit intellegi res quae non suo vocabulo sit appellata.}\dots
\end{quote}

Donatus, {\em Ars maior} III.5, ``De schematibus'' GLK 4,397 \cite{GLK}:

\begin{quote}
Schemata lexeos sunt et dianoeas, id est figurae verborum et sensuum.
sed schemata dianoeas ad oratores pertinent, ad grammaticos lexeos.
quae cum multa sint, ex omnibus necessaria fere sunt decem et septem, quorum haec sunt nomina,
 prolepsis zeugma hypozeuxis syllepsis anadiplosis anaphora epanalepsis epizeuxis paronomasia, schesis onomaton, parhomoeon homoeoptoton homoeoteleuton polyptoton hirmos, polysyndeton dialyton.
\end{quote}

Donatus, {\em Ars maior} III.5, ``De schematibus'' GLK 4,398 \cite{GLK}:

\begin{quote}
Paranomasia est veluti quaedam denominatio, ut
\begin{quote}
nam inceptio est amentium, haut amantium.
\end{quote}
\end{quote}

Donatus, {\em Ars maior} III.6, ``De tropis'' GLK 4,399 \cite{GLK}, \cite[p.~97]{MGR}:

\begin{quote}
A trope is a word transferred from its proper signification to a likeness that is not proper to
it for reasons of embellishment [{\em ornatus}] or necessity. There are thirteen tropes: metaphor,
catachresis, metalepsis, metonymy, antonomasia, synecdoche, epitheton, onomatopoeia,
periphrasis, hyperbaton, hyperbole, allegory, homoeosis.

{\em Tropus est dictio translata a propria significatione ad non propriam
similitudinem ornatus necessitatisve causa. sunt autem tropi tredecim,
metaphora catachresis metalepsis metonymia antonomasia epitheton synecdoche
onomatopoeia periphrasis hyperbaton hyperbole allegoria homoeosis.}
\end{quote}

Priscian, {\em Institutiones grammaticae} GLK 2,117 \cite{GLK}, \cite[pp.~86--87]{aelfric}:

\begin{quote}
Denominativum appellatur a voce primitivi sic nominatum, non ab
aliqua speciali significatione, sicut supra dictae species. habet igitur generalem
nominationem omnium formarum, quae a nomine derivantur.

Denominatives are so named from their root forms (rather than from
some special meaning, like the types discussed above). The denominative is a
general term for all the forms derived from nouns.
\end{quote}

Diomedes, {\em Ars grammatica} GLK 1,446 \cite{GLK}:

\begin{quote}
paronomasia est ueluti quaedam denominatio, cum
\end{quote}

Diomedes, {\em Ars grammatica} GLK 1,458 \cite{GLK}:

\begin{quote}
De metonymia. Metonymia dicitur transnominatio. est autem dictio
ab alia propria significatione ad aliam propriam translata.
\end{quote}

Bede, {\em De schematibus et tropis} \cite[p.~240]{tanenhaus}:

\begin{quote}
It is quite usual to find that, for the
sake of embellishment, word-order in
written compositions is frequently
fashioned in a figured manner different
from that of ordinary speech. The grammarians use the Greek term ``schema''
for this practice, whereas we correctly
label it a ``manner,'' ``form,'' or ``figure,''
because through it speech is in some way
clothed and adorned. Metaphorical language is also quite commonly found
when, either from need or for adornment, a word's specific meaning is replaced by one similar but not proper to
it.
\end{quote}

Bede, {\em De schematibus et tropis} \cite[p.~240]{tanenhaus}:

\begin{quote}
There are to be sure many varieties of figures, but the following are the more prominent: Prolepsis, Zeugma, Hypozeuxis, Syllepsis, Anadiplosis, Anaphora, Epanalepsis, Epizeuxis, Paronomasia, Schesis Onomaton, Paromoeon, Homoeoteleuton, Homoeoptoton, Polyptoton, Hirmos, Polysyndeton, and Dialyton.
\end{quote}

Bede, {\em De schematibus et tropis} \cite[p.~609]{RLM}, \cite[p.~242]{tanenhaus}:

\begin{quote}
{\em Paronamasia, id est, denominatio, dicitur, quotiens dictio paene similis
ponitur in significatione diversa, mutata videlicet littera vel syllaba, ut
in psalmo XXI iuxta hebraicum veritatem: In te confisi sunt, et non
sunt confusi.}

Paronomasia or word-play is the figure
in which the words used closely resemble
one another in sound but differ
in meaning; the letters or syllables have
obviously been changed, as in Psalm
XXII following the Hebrew version:
\begin{quote}
{\em In te confisi sunt, et non sunt confusi.}
\end{quote}
\end{quote}

Psalm 22:4--5:

\begin{quote}
in te confisi sunt patres nostri confisi sunt et salvasti eos

ad te clamaverunt et salvati sunt in te confisi sunt et non sunt confusi
\end{quote}

Bede, {\em De schematibus et tropis} \cite[p.~244]{tanenhaus}:

\begin{quote}
A trope is a figure in which a word,
either from need or for the purpose of embellishment, is shifted from its proper meaning to one similar but not proper to it. There are thirteen tropes which Latin custom and usage recognize: Metaphor, Catachresis, Metalepsis, Metonymy, Antonomasia, Epithet, Synecdoche, Onomatopoeia, Periphrasis, Hyperbaton, Hyperbole, Allegory, Homoeosis.
\end{quote}

Bede, {\em De schematibus et tropis} \cite[p.~246]{tanenhaus}:

\begin{quote}
Metonymy is a kind of substitution of names. There are many types of this trope; for example, when the name of a container is used to designate its contents:
\begin{quote}
Pouring the pitcher in the troughs.
\end{quote}
Or
\begin{quote}
Take thy letter.
\end{quote}
The pitcher is not poured, but rather that which it contains; and it is not the letter that is taken, but the paper upon which it is written. Again:
\begin{quote}
And send it away, that it may go\\
And see:
\end{quote}
Not the ark but only the cart in which the ark was contained, and the cattle which were leading the cart were able to move. Metonymy often reveals the effect of an action through its cause and, conversely, the cause of an action through its effect.
\end{quote}

Pompeius, {\em Commentum artis Donati} GLK 5,307,1 \cite{GLK}:

\begin{quote}
metonymia est quaedam denominatio.
\end{quote}

Isidore of Seville, {\em Etymologies} 1.37.8 \cite[p.~61]{isidore}:

\begin{quote}
Metonymy ({\em metonymia}) is a designation ({\em transnominatio}) that is transferred from one meaning to another similar meaning. It is made in many ways. For instance, it expresses what is contained by what contains, as ``the theater applauds,'' ``the meadows low,'' when in the first instance people applaud and in the second, cows low. In the opposite way, it also expresses that which contains by that which is contained, as (Vergil, {\em Aen.} 2.311):
\begin{quote}
Now the nearby Ucalegon burns,
\end{quote}
when it is not Ucalegon (i.e. a Trojan citizen), but his house, that burns. 
\end{quote}

Isidore of Seville, {\em Etymologies} 2.25.1--4 \cite[p.~81]{isidore}:

\begin{quote}
1. We come to the categories ({\em categoria}) of Aristotle, which in Latin are called `predications' ({\em praedicamentum}). With these every form of discourse is included in accordance with their various significations. 2. The instruments ({\em instrumentum}) of the categories are three: the first is equivocal ({\em aequivocus}), the second univocal ({\em univocus}), the third denominative ({\em denominativus}). They are equivocal when many things possess the same name, but not the same definition, as `lion' -- for with regard to the name, the actual, the painted, and the zodiacal lion are called `lion'; with regard to the definition, the actual is defined one way, the painted another, the zodiacal another. 3. The instruments are univocal when two or more things share a single name and a single definition, as `clothing.' Thus both a cloak and a tunic can take the name `clothing' along with its definition. Therefore this is understood to be univocal among the types of instruments, because it gives both a name and a definition to its forms. 4. We call denominative, that is `derivative' ({\em derivativus}), whichever instruments take their name from some single instance of differentiation with regard to a noun, as `good' from `goodness,' `wicked' from `wickedness.'
\end{quote}















Boethius, {\em Categories}:

\begin{quote}
Denominativa vero dicuntur quaecumque ab aliquo, solo differentia casu, secundum nomen habent appellationem, ut a grammatica grammaticus et a fortitudine fortis.
\end{quote}

Boethius, {\em in Cat.}:

\begin{quote}
Haec quoque definitio nihil habet obscurum. Casus enim antiqui nominabant aliquas nominum transfigurationes, ut a iustitia iustus, a fortitudine fortis, etc. Haec igitur nominis transfiguratio, casus ab antiquioribus vocabatur. Atque ideo quotiescumque aliqua res alia participat, ipsa participatione sicut rem, ita quoque nomen adipiscitur, ut quidam homo, quia iustitia participat et rem quoque inde trahit et nomen, dicitur enim iustus. Ergo denominativa vocantur quaecumque a principali nomine solo casu, id est sola transfiguratione discrepant. Nam cum sit nomen principale iustitia, ab hoc transfiguratum nomen iustus efficitur. Ergo illa sunt denominativa quaecumque a principali nomine solo casus id est sola nominis discrepantia, secundum principale nomen habent appellationem.
\end{quote}








Boethius, {\em Institutio arithmetica} \cite{friedlein}: the number 3 gives its name, i.e. ``denominates'', a third part.
A ratio is a relationship, not a quantity, but the denomination of a ratio is a quantity.

{\em Liber mahamaleth} A-IV \cite[pp.~66, 639]{mahameleth}:

\begin{quote}
{\em Quisquis dividit numerum per numerum unum duorum intendit. Aut enim intendit scire quid accidat uni, scilicet, cum dividit rem unam per aliam alterius generis; veluti cum dividit decem nummos per quinque homines non intendit nisi scire quid accidat uni illorum. Aut intendit scire que est comparatio unius ad alterum, scilicet dividendi ad dividentem, cum dividit unam rem per aliam eiusdem generis; veluti si vellet dividere viginti sextarios per decem sextarios non vult scire nisi quam comparationem habent viginti sextarii ad decem. In hiis autem duobus modis modus
agendi idem est.}

Anyone dividing a number by a number has one of two purposes. Either his purpose is to know what will be attributed to one, namely when he divides one thing by another of a different kind; for example, if he divides ten nummi by five men, his purpose is just to know what will be attributed to one of them. Or his purpose is to know what the ratio of the two numbers is, namely of the dividend to the divisor, when he divides one thing by another of the same kind; for example, if he has to divide twenty sextarii by ten sextarii he just wants to know the ratio of the twenty sextarii to the ten. But the way of proceeding is the same in both cases.
\end{quote}

{\em Liber mahameleth} A-IV \cite[pp.~66, 639]{mahameleth}:

\begin{quote}
{\em Sciendum autem quod in utraque divisione aut dividitur maius per minus, et hec dicitur proprie divisio; aut minus per maius, et dicitur denominatio; aut equale per equale, in qua non exit nisi unum.}

It must be known that in both (types of) division either larger is divided by smaller, and this is, properly speaking, division; or smaller by larger, and this is called denomination; or equal by equal, in which case the result is just one.
\end{quote}

{\em Liber mahameleth} A-IV \cite[pp.~66--67, 639--640]{mahameleth}:

\begin{quote}
{\em Si volueris dividere viginti per quatuor.

Quere numerum in quem multiplicati quatuor fiunt viginti; et hic est
quinque. Et hoc est quod de divisione exit.

Vel denomina unum de quatuor, scilicet quartam. Tanta igitur pars accepta de viginti, scilicet quarta, que est quinque, est id quod de divisione exit.
Cuius probatio manifesta est. Nam talis est comparatio unius ad dividentem qualis est comparatio quesiti ad dividendum. Cum igitur denominaveris unum de dividente, tunc talis pars dividendi est id quod de divisione exit.

Experientia autem talis est hic: Videlicet, multiplica quinque in quatuor, et fient viginti. Redit igitur dividendus. Cum enim multiplicatur id quod de divisione exit in dividentem exit dividendus, sicut predictum est.}

You want to divide twenty by four.

Look for the number which multiplying four produces twenty; this
is five. Such is the result of the division.

Or denominate one from four; this gives a fourth. Such a fraction, that is, a fourth, being taken of twenty, which gives five, is the result of the division.

The proof of this is clear. For the ratio of one to the divisor is the same as the ratio of the required quantity to the dividend. Thus denominating one from the divisor and taking such a fraction of the dividend gives the result of the division.
\end{quote}










Blasius of Parma, {\em Questiones circa tractatum proportionum magistri Thomas Braduardini} \cite[pp.~701--702]{rommevaux}:

\begin{quote}
First conclusion: every ratio is a certain quantity or has the nature of a quantity.  It is obvious because every ratio has a denomination according to which it is called a ratio of equality or inequality, and consequently according to which this ratio is said to be equal or unequal to another. And since this is a property of quantity, any ratio will be a certain quantity.

{\em Prima conclusio: omnis proportio est quedam quantitas vel habet rationem quantitatis. Patet quia omnis proportio habet denominationem secundum quam dicitur proportio equalitatis vel inequalitatis,
 et per consequens secundum quam ista proportio dicitur esse equalis vel inequalis alteri. Et quia hoc est proprium quantitati, ideo omnis proportio erit quedam quantitas.}
\end{quote}










Robert Grosseteste, {\em in An. Post.} I.5, pp.~119--120:

\begin{quote}
Quod autem hoc ipsum proportionale sit intentio ambigua sic patet. Proportionalitas est similitudo proportionum, proportio autem est quantitatum eiusdem generis quantecumque sint certa habitudo. Haec autem certitudo habitudinis principaliter et proprie dicta est diffinita denominatio ipsius proportionis ab aliquo numero, unde haec certitudo proprie et principaliter dicta cadit solum in proportionibus numeralibus. Haec autem certitudo communiter et minus proprie dicta est comparatio diffinita ad denominationem proportionis sumptam ab aliquo numero, sicut dicitur diameter ad costam habere proportionem. Proportio enim illa non est denominata ab aliquo numero, tamen ipsa est collatio certa per comparationem ad denominationem
a numero, ipsa enim est medietas duple proportionis.
\end{quote}











Campanus, {\em Elements} V Def. vi \cite[p.~696]{rommevaux}:

\begin{quote}
Quantities which are said to be in the same ratio, the first relative to the second and the
third to the fourth, are those for which the equimultiples of the first and the third are
similar, whether in excess or in deficit or in equality, to the equimultiples of the second
and of the fourth, if they are taken in the same order.

{\em Quantitates que dicuntur esse secundum proportionem unam, prima ad secundam et tertia ad quartam,
sunt, quarum prime et tertie multiplicationes equales multiplicationibus secunde et quarte equalibus
fuerint simul vel additione vel diminitione vel equalitate eodem ordine sumpte.}
\end{quote}

Jordanus Nemorarius, {\em De elementis arithmetice artis} II \cite[p.~697]{rommevaux}:

\begin{quote}
What we call the denomination of a ratio, at least of a smaller number to a greater, is the
part or parts that the smaller is of the greater; and of a greater number to a smaller, the
number by which it contains it and the part or parts of the smaller that remain in the
greater.

{\em Denominatio dicitur proportionis minoris quidem ad maiorem pars vel partes quote illius fuerit,
maioris vero ad minus numerus secundum quem eum continet et pars vel partes minoris que in maiore
superfluunt.}
\end{quote}

Jordanus Nemorarius, {\em De numeris datis} I Def. 3 \cite[pp.~57, 127]{hughes}:

\begin{quote}
Data est autem proportio cum ipsius denominatio est cognita.

{\em A ratio is given} whose denomination is known.
\end{quote}












Clagett \cite[p.~22]{archimedes2}:

\begin{quote}
When one of the two quantities of the same kind divides the other,
that which results is called the ``denomination'' of the ratio of the
\end{quote}










Johannes de Muris, {\em Musica speculativa}










Thomas Bradwardine, {\em Geometria speculativa} \cite[p.~121]{molland}:

\begin{quote}
An irrational ratio, however, is not in this way immediately denominated by a number or even by a numerical ratio, for it is not possible in that case that any part of the smaller quantity should number the greater according to some number. It may happen, however, that an irrational ratio be mediately denominated by number. For example the ratio of the diagonal of a square to its side is a half of the double ratio, and in this manner other species of such ratios receive denominations by number.

{\em Proportio autem irrationalis non sic immediate denominatur ab aliquo numero licet ab aliqua proportione numerali, quoniam non est ibi possibile ut secundum aliquem numerum pars aliqua minoris maiorem numeret. Contingit tamen mediate denominari proportionem irration- alem a numero, ut proportio dyametri ad costam est medietas duple proportionis, et ita capiunt alie species huius proportionis denominationes a numero.}
\end{quote}









Thomas Bradwardine, {\em Tractatus de proportionibus} 1.3 \cite[pp.~76--77]{bradwardine}:

\begin{quote}
{\em Iam superest tertia pars huius capituli, quasdam suppositiones
praemittens.}

{\em Quarum haec est prima: Omnes proportiones sunt aequales
quarum denominationes sunt eaedem vel aequales.}

{\em Secunda est ista: Quibuscumque duobus extremis, interposito
medio, cuius ad utrumque est aliqua proportio, erit proportio
primi ad tertium composita ex proportione primi ad secundum et
proportione secundi ad tertium.}

There now remains part three of the present chapter, commencing
with certain axioms.

The first is that all proportions are equal whose denominations are
the same, or equal.

The second is that, given two extreme terms, and interposing an
intermediate term possessing a given proportion to each, the proportion
of the first to the third will be the product of the proportions of
the first to the second and the second to the third.
\end{quote} 

Thomas Bradwardine, {\em Tractatus de proportionibus} 1.3 \cite[pp.~78--79]{bradwardine}:

\begin{quote}
{\em Prima conclusio: Si fuerit proportio maioris inaequalitatis
primi ad secundum ut secundi at tertium, erit proportio primi
ad tertium praecise dupla ad proportionem primi ad secundum
et secundi ad tertium.}

{\em Hanc probes ostensive hoc modo: Eaedem vel similes sunt denominationes
proportionum primi ad secundum et secundi ad tertium;
igitur, per primam suppositionem, istae sunt aequales et, per
secundum suppositionem, proportio primi ad tertium componitur
praecise ex illis. Igitur, per definitionem dupli, ista est praecise
dupla ad utramque illarum. Et hoc est quod ostendere volebamus.}

Theorem I: If a proportion of greater inequality between a first
and a second term is the same as that between the second and a third,
the proportion of the first to the third will be exactly the square of
the proportions between the first and the second, and the second and
the third.

This you may prove conclusively as follows: The denominations
of the proportions between the first and second and the second and
third are the same, or similar. Therefore (by Axiom 1) these are
equal, and (by Axiom 2) the proportion of the first to the third is
their exact product. Therefore (by the definition of ``square'') this
proportion is exactly the square of each of the others, and this is
what we wished to show.
\end{quote}









Murdoch \cite{murdoch}



\bibliographystyle{amsplain}
\bibliography{latitudo}

\end{document}