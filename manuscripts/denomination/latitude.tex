\documentclass{article}
\usepackage{amsmath,amssymb,mathrsfs,amsthm,xfrac}
\usepackage[LGR,T1]{fontenc}
\newcommand{\textgreek}[1]{\begingroup\fontencoding{LGR}\selectfont#1\endgroup}
%\usepackage{tikz-cd}
%\usepackage{hyperref}
\newcommand{\inner}[2]{\left\langle #1, #2 \right\rangle}
\newcommand{\tr}{\ensuremath\mathrm{tr}\,} 
\newcommand{\Span}{\ensuremath\mathrm{span}} 
\def\Re{\ensuremath{\mathrm{Re}}\,}
\def\Im{\ensuremath{\mathrm{Im}}\,}
\newcommand{\id}{\ensuremath\mathrm{id}} 
\newcommand{\gcm}{\ensuremath\mathrm{gcm}} 
\newcommand{\diam}{\ensuremath\mathrm{diam}} 
\newcommand{\sgn}{\ensuremath\mathrm{sgn}\,} 
\newcommand{\lcm}{\ensuremath\mathrm{lcm}} 
\newcommand{\supp}{\ensuremath\mathrm{supp}\,}
\newcommand{\dom}{\ensuremath\mathrm{dom}\,}
\newcommand{\norm}[1]{\left\Vert #1 \right\Vert}
\newcommand*\rfrac[2]{{}^{#1}\!/_{#2}}
\newtheorem{theorem}{Theorem}
\newtheorem{lemma}[theorem]{Lemma}
\newtheorem{proposition}[theorem]{Proposition}
\newtheorem{corollary}[theorem]{Corollary}
\begin{document}
\title{Latitude, intension, and remission}
\author{Jordan Bell\\ \texttt{jordan.bell@gmail.com}\\Department of Mathematics, University of Toronto}
\date{\today}

\maketitle


Aristotle, {\em Categories} 2, 1a20 \cite[p.~4]{ackrill}:
Things can be {\em said of} a subject and {\em said in} a subject. 

Aristotle, {\em Categories} 4, 1b25 \cite[p.~5]{ackrill}:

\begin{quote}
Of things said without any combination, each signifies
either substance or quantity or qualification or a relative
or where or when or being-in-a-position or having or
doing or being-affected. To give a rough idea, examples
of substance are man, horse; of quantity: four-foot, five-foot;
of qualification: white, grammatical; of a relative:
double, half, larger; of where: in the Lyceum, in the
market-place; of when: yesterday, last-year; of being-in-a-position:
is-lying, is-sitting; of having: has-shoes-on, has-armour-on;
of doing: cutting, burning; of being-affected:
being-cut, being-burned.
\end{quote}

substance: \textgreek{o>us'ia}, {\em substantia};
quantity: \textgreek{poson}, {\em quantitas}; 
qualification: \textgreek{poi'on}, {\em qualitas}.

Aristotle, {\em Categories} 5, 2a11 \cite[pp.~5--6]{ackrill}:

\begin{quote}
A {\em substance}--that which is called a substance most
strictly, primarily, and most of all--is that which is neither
said of a subject nor in a subject, e.g. the individual man
or the individual horse. The species in which the things
primarily called substances are, are called {\em secondary substances},
as also are the genera of these species. For
example, the individual man belongs to a species, man,
and animal is a genus of the species; so these--both man
and animal--are called secondary substances.
\end{quote}

Aristotle, {\em Categories} 5, 3b24 \cite[p.~10]{ackrill}:

\begin{quote}
Another characteristic of substances is that there
is nothing contrary to them. For what would be contrary
to a primary substance? For example, there is nothing
contrary to an individual man, nor yet is there anything
contrary to man or to animal. This, however is not peculiar
to substance but holds of many other things also, for
example, of quantity. For there is nothing contrary to four-foot
or to ten or to anything of this kind--unless someone
were to say that many is contrary to few or large to small;
but still there is nothing contrary to any {\em definite} quantity.
\end{quote}

Aristotle, {\em Categories} 5, 3b33 \cite[p.~10]{ackrill}:

\begin{quote}
Substance, it seems, does not admit of a more and
a less. I do not mean that one substance is not more
a substance than another (we have said that it is), but
that any given substance is not called more, or less, that
which it is. For example, if this substance is a man, it will
not be more a man or less a man either than itself or than
another man. For one man is not more a man than
another, as one pale thing is more pale than another and
one beautiful thing more beautiful than another. Again,
a thing is called more, or less, such-and-such than itself;
for example, the body that is pale is called more pale now
than before, and the one that is not is called more, or less
hot. Substance, however, is not spoken of thus. For a man
is not called more than a man now than before, nor is anything
else that is a substance. Thus substance does not admit of
a more and a less.
\end{quote}

Aristotle, {\em Categories} 5, 4a10 \cite[p.~11]{ackrill}:

\begin{quote}
It seems most distinctive of substance that what is
numerically one and the same is able to receive contraries.
In no other case could one bring forward anything,
numerically one, which is able to receive contraries. For
example, a colour which is numerically one and the same
will not be black and white, nor will numerically one and
the same action be bad and good; and similarly with
everything else that is not substance. A substance, however,
numerically one and the same, is able to receive
contraries. For example, an individual man--one and
the same--becomes pale at one time and dark at another,
and hot and cold, and bad and good. Nothing like this is
to be sen in any other case.
\end{quote}

Aristotle, {\em Categories} 5, 4a22 \cite[pp.~11--12]{ackrill}:

\begin{quote}
For in the case of substances it is by themselves
changing that they are able to receive contraries.
For what has become cold instead of hot, or dark instead
of pale, or good instead of bad, has changed (has altered);
similarly in other cases too it is by itself undergoing
change that each thing is able to receive contraries. Statements
and beliefs on the other hand themselves remain
completely unchangeable in every way; it is because the
{\em actual thing} changes that the contrary comes to belong to
them. For the statement that somebody is sitting remains
the same; it is because of a change in the actual thing that
it comes to be true at one time and false at another.
\end{quote}

Aristotle, {\em Categories} 6, 4b20 \cite[p.~12]{ackrill}:

\begin{quote}
Of quantities some are discrete, others continuous;
and some are composed of parts which have position in
relation to one another, others are not composed of parts
which have position.

Discrete are number and langue; continuous
are lines, surfaces, bodies, and also, besides these, time
and place.
\end{quote}

Aristotle, {\em Categories} 6, 5a38 \cite[p.~12]{ackrill}:

\begin{quote}
Only these we have mentioned are called quantities
strictly, all the others derivatively; for it is to these we
look when we call the others quantities. For example, we
speak of a large amount of white because the {\em surface} is
large, and an action or a change is called long because
the {\em time} is long. For it is not in its own right that each of
these others is called a quantity. For example, if one is to
say how long an action is, one will determine this by the
time, saying that it is a-year-long or something of that sort;
and in saying how white white one will determine it by
the surface--whatever the size of the surface one will say
that the white too is that size. 
\end{quote}

Aristotle, {\em Categories} 6, 6a19 \cite[p.~16]{ackrill}:

\begin{quote}
A quantity does not seem to admit of a more and
a less. Four-foot for example: one thing is not more four-foot
than another. Or take number: we do not speak of
a three as more three than a five, nor of one three as more
three than another three. Nor yet is one time called more
a time than another.
\end{quote}

Aristotle, {\em Categories} 6, 6a26 \cite[pp.~16--17]{ackrill}:

\begin{quote}
Most distinctive of a quantity is its being called 
both equal and unequal. For each of the quantities we
spoke of is called both equal and unequal. For example,
a body is called both equal and unequal, and a number
is called both equal and unequal, and so is a time; so also
with the others we spoke of, each is called both equal
and unequal. But anything else -- whatever is not a quantity -- is
certainly not, it would seem, called equal and
unequal. For example, a condition is certainly not called
equal and unequal, but, rather, similar; and white is
certainly not equal and unequal, but similar. Thus most
distinctive of a quantity would be its being called both
equal and unequal. 
\end{quote}

Aristotle, {\em Categories} 6, 6b19 \cite[pp.~16--17]{ackrill}:

\begin{quote}
Relatives seem also to admit of a more and a less.
For a thing is called more similar and less similar, and more
unequal and less unequal; and each of these is relative,
since what is similar is called similar {\em to} something and
what is unequal unequal {\em to} something. But not all admit
of a more and less; for what is double, or anything like
that, is not called more double or less double.
\end{quote}

Aristotle, {\em Categories} 8, 8b25 \cite[p.~24]{ackrill}:

\begin{quote}
By a {\em quality} I mean that in virtue of which things are
said to be qualified somehow. But quality is one of the
things spoken of in a number of ways.
\end{quote}

Aristotle, {\em Categories} 8, 8b26 \cite[p.~24]{ackrill}:

\begin{quote}
One kind of quality let us call {\em states} and {\em conditions}.
A state differs from a condition in being more stable and
lasting longer. Such are the branches of knowledge and
the virtues. \dots.
It is what are easily changed and quickly changing we call conditions, e.g. hotness and chill
and sickness and health and the like.
\end{quote}

Aristotle, {\em Categories} 8, 10b26 \cite[p.~24]{ackrill}:

\begin{quote}
Qualifications admit of a more and a less; for
one thing is called more pale or less pale than another, and
more just than another. Moreover, it itself sustains increase
(for what is pale can still become paler)--not in all cases
though, but in most.
\end{quote}

Aristotle, {\em Categories} 9, 11b1 \cite[p.~31]{ackrill}:

\begin{quote}
Doing and being-affected admit of contrariety and
of a more and a less. For heating is contrary to cooling,
and being heated to being cooled, and being pleased to
being pained; so they admit of contrariety. And of a more
and a less also. For it is possible to heat more and less, and
to be heated more and less, and to be pained more and less;
hence doing and being-affected admit of a more and a 
less.
\end{quote}

Aristotle, {\em Categories} 10, 11b38 \cite[pp.~32--33]{ackrill}:

\begin{quote}
But if it is not necessary
for one or the other to belong, there is something
intermediate between them. For example, black and white
naturally occur in bodies, but it is not necessary for one
or the other of them to belong to a body (for not every
body is either white or black); again, bad and good are
predicated both of men and of many other things, but it
is not necessary for one or the other of them to belong to
those things they are predicated of (for not all are either
bad or good). And between these there is certainly something
intermediate--between white and black are grey
yellow and all other colours, and between the bad and the
good the neither bad nor good. In some cases there exist
names for the intermediates, as with grey and yellow
between white and black; in some, however, it is not easy
to find a name for the intermediate, but it is by the negation
of each of the extremes that the intermediate is
marked off, as with the neither good nor bad and neither
just nor unjust.
\end{quote}

Aristotle, {\em Categories} 14, 15a13 \cite[p.~41]{ackrill}:

\begin{quote}
There are six kinds of change: generation, destruction,
increase, diminution, alteration, change of place.
\end{quote}

Porphyry, {\em Isagoge} 2--3 \cite[pp.~4--5]{isagoge}:

\begin{quote}
Genera differ from what is predicated of only one item in that they are predicated of several items. Again, they differ from what is predicated of several
items--from species because species, even if they are predicated of several items, are predicated of items which differ not in species but in number. Thus man, being a species,
is predicated of Socrates and of Plato, who differ from one another not in species but in number, whereas animal, being a genus, is predicated of man and of cow and of horse,
which differ from one another not only in number but also in species. Again, a genus differs from a property because a property is predicated of only one species--the
species of which it is a property--and of the individuals under the species (as laughing is predicated only of man, and of particular men), whereas a genus is predicated not of
one species but of several which differ. Again, a genus differs from a difference and from common accidents because differences and common accidents,
even if they are predicated of several items which differ in species, are not predicated of them in answer to `What is it?' but rather to `What sort of so-and-so is it?'.
Asked what sort of so-and-so a man
is, we say that he is rational; and asked what sort of so-and-so a raven is, we say that it is black--rational is a difference, black an accident. But when we are asked what a man is,
we answer an animal--and animal is a genus of man.
\end{quote}

Porphyry, {\em Isagoge} 12--13 \cite[p.~12]{isagoge}:

\begin{quote}
Accidents are items which come and go without the destruction of their subjects. They are divided into two: some are separable and some inseparable. Sleeping is a separable accident, whereas being black is an inseparable accident for ravens and Ethiopians--it is possible to think of a white raven and an Ethiopian losing his skin-colour without the destruction of the subjects.
\end{quote}

Porphyry, {\em Isagoge} 21 \cite[p.~19]{isagoge}:

\begin{quote}
Participating in species occurs equally, in accidents--even inseparable ones--not equally. For one Ethiopian compared to another may have a skin-colour either diminished or augmented in blackness.
\end{quote}

Porphyry, {\em in Cat.} 105 \cite[pp.~102--103]{strange}:

\begin{quote}
Q. What is quantity accidentally?

A. What is said to be a quantity in virtue of something else,
as white is said to be large in virtue of its surface and a man is
said to be tall insofar as his size is large. Again, we might say
that a fever is great if it lasts for a long time -- since if someone were to use `great' not of a protracted fever but of a intense
one, he would not be using this expression in the strict sense:
he would be speaking about a qualification rather than about
a quantity. For `intense' is a characteristic of quality. And if
we were to say that such-and-such a person has done a great
deal of running, we would be reckoning his motion ({\em kin\^{e}sis}) by 
the large amount of time that it had taken, and it would be derivatively from this time that we would say that he had
done a great deal of running.
\end{quote}

Porphyry, {\em in Cat.} 110--111 \cite[p.~111]{strange}:

\begin{quote}
Q. What then will the proprium of quantity be?

A. To be called equal and unequal. For a line will either be equal or unequal to another line, and a body will be either equal or unequal to another body,
and a surface will be either equal or unequal to another surface.

Q. When some one applies this expression to white, and says `this white is equal to that one', what does he mean?

A. He is not using `equal' in the strict sense, but improperly, in place of `similar'.

Q. How is it that a man is said to be equal to a man, or a tower to a tower, if these are not quantities?

A. It is said accidentally.

Q. Why is it said accidentally?

A. It is used not because the thing in question is a
substance, but because it has a size. For the primary sorts of quantities that pertain to the substance of our realm
 are the so-called dimensions ({\em diastaseis}), which are length, width, and breadth.
\end{quote}

similar: {\em homoios}, equal: {\em isos}.

Porphyry, {\em In Cat.} 137--138 \cite[pp.~152--153]{strange}:

\begin{quote}
Q. State more clearly and distinctly whether qualities and the qualified things that derive from them admit of more and less.

A. But how can one say anything clear about these matters, when there have been so many different schools of thought about them? For some have claimed that all states of matter and qualified entities becomemore and less intense, since matter itself admits of a more and less. Some Platonists have taken this view. Others have held that some states and the
{$\langle$}entities capable of having the state{$\rangle$} that are qualified by them do not admit of a more and less, as is the case with the virtues and persons who are qualified by them,
while other states and qualified entities do admit of intensification and relaxation, as is the case with all intermediate arts and intermediate qualities, and the persons who are qualified by them. The Stoics held this view. But there is a third view, which is the one Aristotle refers to, that holds that states cannot be more intense or more relaxed, but
that the persons that are qualified by them do admit of a more and less: not all of them, however.
\end{quote}

Porphyry, {\em In Cat.} 139 \cite[p.~154]{strange}:

\begin{quote}
Q. But if this is not the proprium of quality either, what will
its proprium be?

A. Similarity and dissimilarity, for it is only in respect of
qualities that something is said to be similar to something else, and being similar or dissimilar is predicated only of qualities, so that to be said to be similar and dissimilar will be a proprium of quality.
\end{quote}

Dexippus, {\em 	In Cat.} 2.30 \cite[pp.~98--99]{dexippus}:

\begin{quote}
It is a fact of nature that along with each
form there goes a quality, distinct from the form, but
dependent upon it; and the form, being a part of the individual
composite substance, is constitutive of it and never admits of
more and less, but the qualities which go along with it,
whether it be rationality, or heat, or dryness, do admit of
degrees, and so this apparent variation of degree is not a
function of the form but of the quality, so that when the good
man ({\em spoudaios}) is said to be more of a man, this increase in
degree is true of him not {\em qua} man, but {\em qua} man as being in a
certain condition. But this is a matter not of substance, but of
quality, so that we are right to accept the doctrine that one
substance is not more or less so than another.
\end{quote}

Ammonius, {\em in Cat.} 55 \cite[p.~66]{ammonius}, on 4b20:

\begin{quote}
Some people say that, properly speaking, there are three
species of quantity -- number, volume, and power, i.e. weight.
They argue that (1) statement and time are the same as
number; (2) line, surface, and body can be reduced to something common -- magnitude;
(3) place is the same as surface,
and therefore (4) one kind of quantity is number, another
magnitude, another power. (For under this last heading they
place heavy and light, which are weights of quantity, for they
fall under it.)

Why didn't Aristotle mention change ({\em kin\^esis})? Our response
is that change is not an actuality ({\em energeia}), but an indefinite
thing. Therefore he did not mention it, since he was addressing
his discussion to beginners. For change itself is nothing other
than the transition ({\em hodos}) from something in potentiality to
something in actuality.
\end{quote}

Ammonius, {\em in Cat.} 60--61 \cite[p.~71]{ammonius}, on 5b1:

\begin{quote}
The scientist's task is not only to investigate things he himself
proposes, but also to go through in detail and refute those that
seem to be so but in truth are not. Now white may seem to be a
quantity. For we speak of white as more and less, which is
characteristic of quantity; but then we also call an action long.
So Aristotle says that these are not quantities in the strict
sense, but only {\em per accidens}. For since white is in a surface,
and that can be more or less, we say that the white is more or
less. It is the same with an action; for example, a war is called
long {\em per accidens}. Thus, since a war goes on for a certain
period of time -- e.g. for ten years -- and we say that its time is
long, for this reason we say that the action, as well, is long, {\em per
accidens}. A change, too, is called large if its time is large. For
time is the measure of change. Thus, we call a revolution of
the moon a month, of the sun a year, and of the entire heaven
a day. So if someone were to ask how long an action is, the
answer is its time, e.g. ten years. It is the same with a surface;
one may say that the white is as large as the surface is. For if
we are asked how much white there is, we say two cubits if
that is how large the surface containing the white is.
Therefore, we mean that the surface, not the white itself, is
more or less. Indeed, the white in a one-cubit surface can be
whiter than that in a two-cubit surface, and then we do not
say that there is more white ({\em leukon pleon}) but that one white
is whiter than another ({\em leukon leukou mallon}).
\end{quote}

Ammonius, {\em in Cat.} 62 \cite[pp.~72--73]{ammonius}, on 5b18:

\begin{quote}
If a thing were said to be large or small just in itself, a
mountain would never be said to be small or a millet seed
large. But we call a mountain small obviously comparing it to
another mountain, or a millet seed large obviously because it
is larger than another. Therefore, a thing is not said to be
large or small just in itself, but in relation to something else.
It is the same, Aristotle shows, with many and few, which do
not belong to number, Le. to the discrete; rather, they are
relatives.
\end{quote}

Ammonius, {\em in Cat.} 63 \cite[pp.~73--74]{ammonius}, on 5b30:

\begin{quote}
What he means is this. Contraries must first exist in their
own right and have real and independent being. Only so do
they then meet in battle and declare war, that is, oppose each
other. This is not possible for relatives, for they do not fight
each other, but rather they bring each other in together. For
relatives differ from contraries in that contraries exist first in
their own right and thereafter join battle and fight, whereas
relatives produce each other reciprocally. In this way, then,
even if white didn't exist at all, black would remain, but if the
father were taken away, the son would be gone. Since large
and small {$\langle$}exist{$\rangle$} not in their own right but with respect to
something else, and similarly for many and few, and {$\langle$}since{$\rangle$}
neither large nor small exists independently in and of itself, it
is clear that they are not contraries.
\end{quote}

Ammonius, {\em in Cat.} 65 \cite[pp.~75--76]{ammonius}, on 6a19:

\begin{quote}
He says that not having a contrary is a {\em proprium} ({\em idion}) of
quantity, and he shows that it belongs to every {$\langle$}quantity{$\rangle$},
but he does not mention that it does not belong to {$\langle$}quantity{$\rangle$}
alone. (Indeed, this {$\langle$}last point{$\rangle$} should be clear from what
has been said about substance.) Rather, he passed over to
another {\em proprium} of quantity: not admitting more or less. And
this is reasonable. For where there is contrariety there is
more and less, but where there is not, more and less are not
found. For more and less arise from a mixture of contraries.
One must realize that Aristotle again rejects this, since it also
fits substance, and passes over to another {\em proprium},
something which is indeed called a {\em proprium} in the strict
sense.
\end{quote}

Martianus Capella, {\em The Marriage of Philology and Mercury} V (Rhetoric), 370 \cite[p.~124]{capellaII}:

\begin{quote}
[370] Quality accepts the idea of a more and
a less, but not in every instance. For nothing square is more square
than any other square thing; but something can be said to be more
white than another white thing. And it is a question often discussed,
whether one person may be said to be more just than another.
But there appear to be many who have given careful thought to the
question and say that the qualities themselves cannot accept the idea
of a more and a less, but only the items named after them. For instance,
justice is in itself one single perfect concept, so that we
cannot say {\em This is more justice than that}, but we can say {\em This man
is more just than that}. Similarly, we cannot say {\em This is more health
than that}, but we can say {\em This man is more healthy than that}. So it
happens that substance does not accept the idea of a more and a less,
but qualities can accept it through substances. 
\end{quote}

Simplicius, {\em in Cat.} 122--123 \cite[p.~101]{simpliciuscat5}, on 4b20:

\begin{quote}
Quantity is divided into the continuous ({\em to sunekhes}) and the
discrete ({\em to di\^{o}rismenon}), since these fall under the genus of Quantity.
For in their very essence, Quantity is predicated of them -- and
not as an accident or as a mere name; rather each partakes in
Quantity equally, since both admit in a similar manner the equal and
the unequal, and the double and the half. But this division is not
made into species of Quantity, but into differentiae, since the species
of Quantity are magnitude ({\em megethos}) and amount ({\em pl\^{e}thos}), and the
continuous and the discrete are its differentiae. For magnitude is
continuous quantity, and amount is discrete quantity. Aristotle himself
made number and speech species of Quantity according to the
differentia of the discrete, and line, surface and body according to that
of the continuous -- and also place and time, which is perhaps more
accurate. For it does not seem correct even to Iamblichus that amount
should be equated to the discrete, since speech is something discrete,
like number, but speech is not an amount. For even if speech is
manifold, even so the `being many' which partakes of amount is one
thing, just as people are many, and the `being an amount', when
characterized in this way, is something else.
\end{quote}

Simplicius, {\em in Cat.} 127--129 \cite[pp.~106--107]{simpliciuscat5}, on 4b20:

\begin{quote}
The supporters of Lucius and Nicostratus object to the division
firstly as wrongly calling even magnitude a quantity ({\em poson}). It should
have been described as `so much' ({\em p\^{e}likon}) and {$\langle$}only{$\rangle$} number as a
quantity. What is common {$\langle$}to magnitude and number{$\rangle$} should have
been called either something else, or `quantity' in a sense different
from that of one of its species. But even if the continuous is in the
broadest sense magnitude while the discrete is quantity, they are
often interchanged (at all events we call water, which is continuous,
a quantity, and not a magnitude -- for it is extensive, not large; and
we call time too a quantity); for this reason he quite reasonably did
not make two categories out of quantity and `so much', and did not
divide (it) according to `so much' and quantity, but according to the
continuous and the discrete, which are never interchanged.

They criticize also the fact that the division is {$\langle$}only{$\rangle$} into two. For
as a third species after number and size he should have established
weight or downward thrust, as Archytas and later Athenodorus
and Ptolemaeus the mathematician did. But it should be stated that
weight belongs to the category of Quality, like density and thickness
and their contraries, which are determined according to their quality,
not their quantity. But where would the mina and the talent, when
spoken of as weights, be included? We shall certainly not claim that
they belong in the category of prior quantities, but in that of {\em per
accidens} quantities; for they are not {$\langle$}determined{$\rangle$} according to
number or magnitude in an unqualified sense. But it should be noted
that perhaps downward thrust is not a {\em per accidens} quantity in the
way that white is, i.e. because the surface is a quantity, but is a
quantity {\em per se}, because it admits {\em per se} the particular characteristic
of Quantity, the equal and the unequal, just as other categories
admit excess and deficiency. For I think we should pay attention to
Archytas who also divides quantity in three ways. He writes as
follows: `There are three differentiae of quantity: one of them consists
of downward thrust, like the talent; one in magnitude, like a
length of two cubits; and one in amount, like the number ten.'
Iamblichus accepts this division, since it becomes the triad according
to the most perfect measure of quantity, and since it is in harmony
with realities. He writes: `For quantity in terms of downward thrust
is not the same as size or amount, but is considered rather in the case
of change, and possesses quantity in terms of weight or lightness.
This division is left in the following state: ``Of quantities some have
downward thrust, others do not''. It is clear that the division is not
the same as that into the continuous and the discrete, or that into
what has position and what does not. In the universe at large this
division seems evident, as being into the four elements which have
downward thrust, and the heavens which do not. In the case of
changes movements in a straight line happen with downward
thrust, having a beginning and an end, and as it takes place are
marked off at intervals by rest, while circular movement is continuous,
having no beginning and no end as if it were perpetual, and is
without downward thrust. Such a difference is evident also in the case
of bodiless quantities. For if someone were to posit the soul as a {\em per
se} quantity, it will have downward thrust where it inclines towards
the body, and upward thrust where it inclines away from the lower
world towards the intelligible. But intellect is a quantity without
gravity. Why then do we call the vocal intervals quantities, but the
degrees of downward thrust not quantities?�

In reply to Cornutus and Porphyry, who claim that downward
thrust considered in terms of weight and lightness is quality, {$\langle$}Iamblichus{$\rangle$}
says that downward thrust is not weight or lightness, but
the measure of weight and lightness. `For by themselves heavy or
light things would proceed to infinity if they had no boundary from
within themselves; but when the force of gravity resulting from the
measures produces a boundary and limit, it is then that they come to
a good proportion.' These then are the problems concerning quantity
in general, and their solutions.
\end{quote}

Simplicius, {\em in Cat.} 176 \cite[p.~31]{simpliciuscat7}:

\begin{quote}
He says: `it seems that relatives admit more and less', adding,
`seems' not because they do not in fact admit them but only seem to,
but because he is explaining an ancient doctrine. They admit more
and less in terms of similarity; for that which partakes of the same
form to a greater extent is more similar, and that which does so to a
lesser extent is less similar. But similarity is among relatives. So
much is clear. But why, instead of saying `also more and less unequal',
did he say `more unequal to a greater or lesser extent'? For if `more
unequal' is [the same as] `unequal' with the addition of `to greater
extent', why did he say `more unequal to a greater or lesser extent'?
The answer is that he introduced, on account of the rather unusual
language, the word `unequal' together with the word `more'. Why then
did he add `less'? Because the `more unequal' too allows for intension
and remission, so that [a thing] can be more unequal to a greater
or lesser degree. For we do say more similar to a greater or lesser
degree when one thing is more similar than something else, and this
more similar thing undergoes intension and remission.
\end{quote}

intension: {\em epitasis}, remission: {\em anesis}, latitude: {\em platos}

Simplicius, {\em in Cat.} 178 \cite[pp.~32--33]{simpliciuscat7}:

\begin{quote}
`But', they
would say, `no relationship admits of intension and remission; for [all
relationships] are [directed] in the same way towards something else'.
The answer is that they admit more and less not in so far as they are
relationships or relative to something, but in so far as they are of such
a kind and quality. `But', they say, `more and less are not observed
only in the qualities themselves, but also in quantities, as in the equal
and the unequal, the greater and the smaller'. The answer is that
these are quantitative qualities, and intension and remission occur
according to the quality. If anyone thinks that the double admits
intension and remission because of the increase or decrease of the
numbers in the same proportion, so as to think that the double in the
case of $200:100$ is more than that in the case of $4:2$, he fails to realise
that both in larger and smaller numbers the proportion is considered
the same, as admitting neither intension and remission.
\end{quote}

Simplicius, {\em in Cat.} 284 \cite[pp.~145--146]{simpliciuscat7}:

\begin{quote}
There are four schools of thought concerning the intension and remission of qualities and qualified things. Some, for example Plotinus
and other Platonists, declare that all qualities and qualified things admit more and less, because everything enmattered does -- so
matter has more and less because of its connate indeterminacy. There is another doctrine over and against this which states that there is no more or
less in the qualities themselves (such as justice
and whiteness) since each is a whole and stands according to a single definition, and for that reason does not admit more and less; intension
and remission are rather to be found in what participates, since participation involves latitude ({\em platos}) and some participate more, others less;
that is why states themselves are thought to admit more and less, although it is the qualified thing that admits them. Aristotle seems to hint at this
doctrine when he says: `one might wonder if
one justice is said to be more so than another justice', and later: `one justice is in no way said to be more or less a justice than another\dots
but things that are spoken of as qualified by them indisputably admit more [and less]; for one man is said to more literate, or more
just, or more healthy, than another.' But when he says `so too with literacy and the other conditions',
he lumped states and conditions together, just as he did a little earlier with qualities and things qualified.
\end{quote}

Simplicius, {\em in Cat.} 286 \cite[pp.~147--148]{simpliciuscat7}:

\begin{quote}
But why, they ask, is there such a difference of doctrine concerning intension and remission in qualities? The answer is that the cause of such a divergence of doctrines is the very account of Quality, what participates in it and what is compounded from it, since these have many differences among themselves and give rise to differing conceptions. For some think that the very account of Quality is fixed according to a single definition, while others think that it is fixed in some cases and not in others, and yet others that it is not fixed at all;
for they say that it is enmattered and with matter, and is altered together with matter�s indeterminate nature. Others divide it, positing one immaterial account, and another material one. All these are divided over the question of the differentiae of Quality. In another way the difference in doctrine occurs in terms of what participates in
and admits the account, when we say `partakes of' or `participates in' latitude, and that it participates more and less. A third difference would be the result of a combination of the two. For when two things combine, the predominance of the one over the other causes intension
and remission; when the form prevails `the more' occurs, and when
matter prevails, `the less' occurs. And when the form is predominant the account, being defined per se in a single form, is fixed; but when the recipient matter predominates, then there is alteration and internal change; when they are equally balanced, some of the parts are stable, others less stable -- and that is how the divergent doctrines
emerge. After this overview of the schools it would be wise to examine each in detail.
\end{quote}

Simplicius, {\em in Cat.} 287 \cite[pp.~148--149]{simpliciuscat7}:

\begin{quote}
But what sort of conditions does Aristotle see as perfect other than
those which are had? Does he mean separable forms such as Justice-itself? The answer is that this is not his usual way of thinking, nor is it in keeping with the matters in hand, nor would these be called dispositions. But having considered those that are participated in as
perfect by reference to them, he wants those that are wanting in some
way to admit greater and less. The more correct argument is that which in the case of all dispositions views the perfect in terms of the extreme, and the more and less in terms of participation in latitude.
\end{quote}

Aspasius, {\em in EN} 47--48 \cite[p.~48]{aspasius}:

\begin{quote}
That vice exists in excess and deficiency, but virtue is in the mean in respect to us, he tries to teach also through other arguments. He says that `erring is [possible] in many ways, but being correct in one only' (1106b28-31). That erring is [possible] in many ways the Pythagoreans too bear witness, when they say that evil is characteristic of the unlimited, `but the good is characteristic of the limited' (1106b30). This is why erring is easier: for it is possible for someone who is shooting to miss the target to the right and to the left and above it and below, but one does it correctly by hitting the target, and it is possible to hit the target in one way only. From this it is evident, then, that `excess and deficiency are characteristic of vice, whereas the mean is characteristic of virtue' (1106b33-4). For the mean is one, but excess and deficiency are two modes of vices. Further, in each excess it is possible to be excessive in many ways, if the excess is intensified and slackened; similarly in each deficiency, if the deficiency is less or more. But it is possible to be correct in one way only, in accord with the mean. For the
mean is one, as has been said.
\end{quote}

Aspasius, {\em in EN} 50 \cite[p.~50]{aspasius}:

\begin{quote}
Arguing for the [proposition] that there is no mean at all in the
above-mentioned actions, Aristotle moves on to wicked states, demonstrating
from these what he has proposed. For just as there is no excess,
deficiency, and mean concerning doing wrong and being dissolute and
being cowardly, but the entire such state is in error, so too in the
above-mentioned actions there is no excess, mean, or lack. It is inquired in what sense he said that there is no mean, excess, and deficiency concerning doing wrong and being cowardly. For if there are slackenings and intensifications in vices, there should be intensification and
slackening in cowardice, and likewise in dissoluteness, and thus nothing prevents there being one excess greater than another excess
and one deficiency greater than another deficiency. But excess and deficiency in
vices are spoken of in two ways, one being in the vice itself such that one exceeds more than another and one is less by more than another,
but excess and deficiency are spoken of in a different way as being in respect to the mean and virtue. According to the former of the ways
mentioned, then, it is possible for one vice to be exceeded by another vice and to be more intensified or slackened, but to one who considers
it in respect to the mean there is neither an excess nor a deficiency of an excess.
`For thus there will be' (1107a20), he says, a kind of mean and virtue in vice itself; but this is impossible. According to this argument, then,
there will neither be an excess of an excess nor a deficiency of a deficiency, so that there may not be means in vices.
\end{quote}

David the Invincible, {\em in Isag.} 21.8 \cite[p.~213]{david}:

\begin{quote}
Now what do we say? We say that one should put it like that: the difference
means items without any range of variation, like a triangle and a quadrangle,
for they are without any range of variation and do not admit the more and the
less. Likewise is an Athenian, who is without range of variation does not admit
the more and the less. Likewise the rational, being in a like manner always and
without any range of variation, does not admit the more and the less. And if
it has a range of variation, then it admits the more and the less. For white can
also be more and less white; for this reason it admits the more and the less.
\end{quote}

Averroes, {\em Middle Commentaries on Aristotle's Categories}, para. 40: \cite[p.~47]{butterworth}, on 5a38--5b10:

\begin{quote}
These primary genera of quantity are the ones which are truly and primarily quantity. Anything other than them ascribed to quantity is only
accidentally and secondarily said to be quantity--I mean, by means of one of these which we said were truly quantity. For example, we say that this
designated whiteness is large because it is in a large surface. Similarly, we say that the task is long because of it taking a long time. This becomes
evident when someone asks ``how extensive is this task,'' for the answer to that would be ``it is a year-long task.'' And if he were to ask ``how long is this white thing,'' it would be said ``three or four cubits long.'' So the task is limited and measured in terms of time, and the white thing is measured in terms of the scope of the plane--which is three or four cubits long.
If they were quantities essentially, they would be meas?ured in terms of themselves.
\end{quote}

Averroes, {\em Middle Commentaries on Aristotle's Categories}, para. 70: \cite[pp.~63--64]{butterworth}, on 9a27--35:

\begin{quote}
He said: there is a third genus of quality, and it is the one which is spoken of as affective qualities and affections. The species of this are tastes--like the sweet and the bitter--and colors--like black and white--and
tactile things--like heat, cold, moisture, and dryness. With all of these it is evident that they are such as to be qualities, for anything to which one of them is at? tributed may be asked about by using the particle ``how.'' For example, we would say ``how sweet is this honey'' and ``how white is this garment?'' And the reply would be that it is very sweet and very white or not very sweet or white.
\end{quote}

Averroes, {\em Middle Commentaries on Aristotle's Categories}, para. 71: \cite[p.~64]{butterworth}, on 9a36--9b8:

\begin{quote}
Now things like these are said to be affective qualities not because they occur in the things to which they are attributed by means of an affection, but be?cause they give rise to an affection in our senses. For example, the sweetness in honey and the bitterness in aloes are said to be affective qualities not because of an affection giving rise to sweetness in honey nor be?cause of an affection giving rise to bitterness in aloes, but because both of them give rise to an affection in the tongue. It is the same for heat and cold with the sense of touch.
\end{quote}

Averroes, {\em Middle Commentaries on Aristotle's Categories}, para. 78: \cite[pp.~66--67]{butterworth}, on 10a28--10b12:

\begin{quote}
He said: qualified things are those signified by names which signify the qualities themselves, that is,
the primary paradigms. In the Greek language that comes about for most of them by means of derivation--like with ``white'' which is derived from the noun ``whiteness,'' ``eloquent''
which is derived from the noun 	``eloquence,'' and ``just'' which is derived from the noun ``justice.'' For some exceptional ones, that is, those qual?ities taken apart from the subject, there are no names in Greek; so names for those qualities are derived in? sofar as they are in a subject. For example, the names they set down for things falling under what is said to occur by means of a natural faculty or not by means of a natural faculty are not derived from anything--like runner and boxer. For the names which signify these notions for them are not derived from ``running'' nor from ``boxing,'' as they are in the speech of the Arabs. Nor is it unusual to find verbs in the Arabic language which have no verbal nouns. However, in the Greek language it sometimes happens that there will be a name for a quality insofar as it is apart from a subject, and insofar as it is in a subject the name of that quality will be derived from another name. For example, they used to say ``diligent,'' not ``virtuous,'' when referring to virtue.
\end{quote}

Duns Scotus, {\em Questions on Aristotle's Categories}, Question 24, Art. 15 \cite[p.~205]{scotus}:

\begin{quote}
To the fourth argument, I say that it is one thing to receive
greater and less, and another thing to receive more and
less, since greater and less are said according to quantity; but
more and less express the intension and remission in the form
to which they are added. It must be conceded then that quantity
receives greater and less, but not more {$\langle$}and less{$\rangle$}.
\end{quote}

 












Simplicius, {\em in Phys.} 864.18--23

Plutarch, {\em De primo frigido}, {\em The Principle of Cold} 946d, Loeb LCL406, volume XII of {\em Moralia}, Cherniss and Hembold, pp.~233--235:

\begin{quote}
Besides, a negation does not permit degrees of
less or more. Surely nobody will affirm that one blind man is blinder than another, or one dumb man more
silent than another, or one corpse deader than its fellow ; but among cold things there is a wide range
of deviation from much to little, from very cold to not
very, and, generally speaking, in degrees of intensity
and remission, just as there is in hot things. This
occurs because the matter involved is in different cases acted upon by the opposing forces with more or less intensity ; it thus exhibits degrees of one or the other,
and so of hot and cold. There is, in fact, no such
thing as a blending of positive qualities with negative
ones, nor may any positive force accept the assault of the negation that corresponds to it or take it into partnership ; instead it gives place to it. Now hot
things do admit a blending with cold up to a point,
just as do black with white, high notes with low, sweet
tastes with sour ; and this harmonious association of colours and sounds, drugs and sauces, produces
many combinations that are pleasant and grateful
to the senses.
\end{quote}

Alexander, {\em in Metaph.} 418.19--22

Avicenna, {\em Sufficientia} \cite{avicenna}

Sextus {\em PH} II.40

Averroes, {\em Middle Commentary on De generatione et corruptione}  \cite{kurland}

Aquinas, {\em Summa Theologica} I-II, Q. 52, ``The increase of habits''.

Aquinas, {\em in Metaph.} Art. 919, Lesson 12, on Aristotle 1018a9--1018b8:

\begin{quote}
Some things are like, then, for three reasons. (1) First, they undergo or suffer the same thing; for example, two pieces of wood which are consumed by fire can be said to be like. (2) Second, several things are like merely because they are affected or undergo something, whether this be the same or different; for example, two men, one of whom is beaten and the other imprisoned, are said to be like in that they both undergo something or suffer. (3) Third, those things are said to be like which have one quality; for example, two white things are alike in whiteness, and two stars in the heaven are alike in brightness or in power.
\end{quote}

Aquinas, {\em in Metaph.} Art. 928, Lesson 12, on Aristotle 1018a9--1018b8:

\begin{quote}
For some things are contraries either because they actually possess contraries, as fire and water are called contraries because one is hot and the other cold; or because they are the potential recipients of contraries, as what is receptive of health and of disease; or because they are potentially causing contraries or undergoing them, as what is capable of heating and of cooling, and what is able to be heated and to be cooled; or because they are actually causing contraries or undergoing them, as what is heating and cooling or being heated and being cooled; or because they are expulsions or rejections or acquisitions of contraries, or even possessions or privations of them. For the privation of white is the opposite of the privation of black, just as the possession of the former is the opposite of that of the latter.
\end{quote}

Aquinas, {\em in Caelo}, Book I, Lecture 25, Art. 249:

\begin{quote}
249. To explain the first [180] he says that if a thing is capable of something great, for example, if a man can walk 100 stades or can lift a great weight, we always determine or describe his power in terms of the most he can do. For example, we say that the power of this man is that he can lift a weight of 100 talents or can walk a distance of 100 stades, even though he is capable of all the partial distances included in that quantity, since he can do what exceeds. But his power is not described by these parts -- we do not determine his power as being able to carry 50 talents or walk 50 stades, but by the most he can do. Consequently, the power of each thing is described with respect to the end, i.e., with respect to the ultimate, and to the maximum of which it is capable, and with respect to the strength of its excellence. Thus, too, the size of a thing is determined by what is greatest -- for example, in describing the size of something that is three cubits, we do not say that it is two cubits. Similarly, we assign as the notion of man that he is rational, not that he is sensible, because what is the ultimate and greatest in a thing is what completes it and puts upon it the stamp of its species.

Consequently, it is plain that one who can do what exceeds, necessarily can do what is less. For example, if a person can carry 100 talents, he can also carry two, and if he can walk 100 stades, he can also walk two; yet it is to what is excelling that the virtue of a thing is attributed, i.e., the virtue of a thing is gauged in terms of what is most excellent of all the things that can be done.

This is what is said in another translation, ``the virtue is the limit of a power,'' because, namely, the virtue of a thing is determined according to the ultimate it can do. And this applies also to the virtues of the soul: for a human virtue is that through which a man is capable of what is most excellent in human actions, i.e., in an action which is in accordance with reason.

250. Then at [181] he tells how something is said to be ``impossible'' to a thing. And he says that if some amount is impossible to someone if one takes what excels, it is plain that it will be impossible for him to carry or do more. For example, a person who cannot walk 100 stades clearly cannot walk 101. Hence, it is plain that just as the possibility is determined by the greatest that a thing can do -- which determines its virtue -- so what is impossible is determined by the least that it cannot do, and this determines its weakness. For example, if the most that someone can do is to go 20 stades, the least that he cannot go is 21 -- and it is from this that his weakness is to be determined, and not from his inability to walk 100 or 1,000 stades.
\end{quote}













Aristotle, {\em Physics} III.6, 206a9--b27 \cite[pp.~105--107]{aristotle}:

\begin{quote}
On the other hand it is clear that, if an infinite does not exist
at all, many impossibilities arise: time will have some beginning
and end, magnitudes will not be divisible into magnitudes, and
number will not be infinite. If therefore, when the case has been
set out as above, neither view appears to be admissible, we need an
arbitrator; clearly there is a sense in which the infinite exists and
another sense in which it does not.

Being means either being potentially, or being actually, and the
infinite is possible by way of addition as well as by way of division
(reading \textgreek{diair\'esei}). Now, as we have explained, magnitude is never
actually infinite, but it is infinite by way of division -- for it is not
difficult to refute the theory of indivisible lines -- the alternative
that remains, therefore, is that the infinite exists potentially. But
in what sense does it exist potentially? You may say, for example,
that a given piece of material is potentially a statue, because it will
(sometime) be a statue. Not so with something potentially infinite:
you must not suppose that it will be actually infinite. Being has
many meanings, and we say that the infinite ``is'' in the same sense
as we say ``it is day'' or ``the games are on'', namely in virtue of
one thing continually succeeding another. For the distinction 
between ``potentially'' and ``actually'' applies to these things too:
there {\em are} Olympian games both in the sense that the contests may
take place and that they do take place.

It is clear that the infinite takes different forms, as in time, in
generations of men, and in the division of magnitudes. For, 
generally speaking, the infinite is so in the sense that it is again and again
taking on something more, this something being always finite, but
different every time. In the case of magnitudes, however, what is
taken on during the process stays there; whereas in the case of time
and generations of men it passes away, but so that the source of
supply never gives out.

The infinite by way of addition is in a manner the same as the
infinite by way of division. Within a finite magnitude the infinite
by way of addition is realized in an inverse way (to that by way of
division); for, as we see the magnitude being divided {\em ad infinitum},
so, in the same way, the sum of the successive fractions when added
to one another (continually) will be found to tend towards a 
determinate limit. For if, in a finite magnitude, you take a determinate
fraction of it and then add to that fraction in the same ratio, and
so on [i.e. so that each part has to the preceding part the same ratio
as the part first taken has to the whole], but {\em not} each time including
(in the part taken) one and the same amount of the original whole,
you will not traverse (i.e. exhaust) the finite magnitude. But if you
increase the ratio so that it always includes one and the same
magnitude, whatever it is, you will traverse it, because any finite
magnitude can be exhausted by taking away from it continually any
definite magnitude however small. In no other sense, then, does the
infinite exist; but it does exist in this sense, namely potentially and
by way of diminution. In actuality it exists only in the sense
in which we say ``it is day'' or ``the games are on'', and potentially
it exists in the same way as matter, but not independently as the
finite does. Thus we may even have a potentially infinite by way of
addition of the kind we described, which, as we say, is in a certain
way the same as the infinite by way of division; it can always take
on something outside (the total for the time being), but the total
will never exceed every determinate magnitude (of the same kind)
in the way that, in the direction of division, it passes every 
determinate magnitude in smallness, and becomes continually smaller
and smaller. But in the sense of exceeding every (magnitude) by
way of addition, the infinite cannot exist even potentially, unless
there exists something actually infinite, but only incidentally so,
infinite, that is, in the sense in which natural philosophers declare
the body outside the universe to be infinite, whether its substance
be air or anything else of the kind. But if it is not possible that there
can be a sensible body actually infinite in this sense, it is manifest
that neither can there be such a body which is even potentially
infinite by way of addition, save in the sense which we have described
as the reverse of the infinite by way of division.
\end{quote}

Aristotle, {\em Physics} III.7, 207a33 \cite[p.~110]{aristotle}:

\begin{quote}
It is after all only reasonable that it should be thought that by
way of addition there is not an infinite such as to exceed every
magnitude but that by way of division there is. For the infinite,
like the matter (of a thing), is contained inside (what contains it)
and what contains both is the form.

Reasonable too it is that in number there is a limit in the direction
of the minimum, while in the direction of increase it may exceed
any number assigned from time to time, but that in the case of
magnitudes, on the contrary, it is possible to surpass any magnitude
in the direction of smallness, while in the direction of increase there
is no infinite magnitude. The reason is that the One is indivisible
whatever it may be that is one; for example, a man is one man and
not many; but a number is several ``ones'' or a certain quantity of
them. Hence number must stop at the indivisible; two and three
are derivative words, and so is every other number. But in the
direction of ``more'' we can always think of a greater number. The
possible bisections of a magnitude are infinite in number; this
infinite is potential, not actual, but you can always assume a number
(of such bisections) exceeding any assigned number. But this number
is not separable from the process of bisection, and its infinity is not
a stationary one but it is in process of coming to be, like time and
the number of time.

With magnitudes the contrary is the case; for the continuous
magnitude is divisible {\em ad infinitum}, but in the direction of increase
there is no infinite. Whatever its size potentially, that size it can be
actually; hence, since there is no sensible magnitude that is infinite, it
is not possible to have an excess over every determinate magnitude;
if it were, there would have to be something greater than the universe.
\end{quote}

Aristotle, {\em Physis} VIII.10, 266a24--b6 \cite[p.~153]{aristotle}:

\begin{quote}
And the general proposition that a finite magnitude cannot
possess an infinite force is clear from the following considerations.
Let us assume that it is always the greater force that produces an
equal effect in less time, in heating, for example, or sweetening, or
throwing, or generally in moving anything. Then that which is
acted on must be affected in some way by the finite magnitude
which we have supposed to possess infinite force, and that to a
greater extent than it would be by anything else, for the infinite
force is greater than any other. In that case the time taken must be
no time at all. For let $A$ be the time taken by the infinite force in
warming or thrusting the object acted on, and let $A+B$ be the time
in which some finite force does the same; then, if I make this finite
force greater and greater by continually adding another finite force,
I shall sometime arrive at the point of having completed the motion
in the time $A$, for if I add continually to a limited magnitude, I shall
at length exceed any assigned magnitude whatever, and if I 
continually subtract from it, I shall similarly make it fall short of any
assigned magnitude. Therefore, the finite force will move the object
in the same time as the infinite force does. But this is impossible;
therefore nothing finite can possess infinite force.
\end{quote}

Aristotle, {\em De caelo} I.1, 268a4--13 \cite[p.~159]{aristotle}:

\begin{quote}
Of things constituted by nature some are bodies and magnitudes, some possess body and magnitude, and some are the principles of
 things which possess these.
That is {\em continuous} which is divisible into parts continually divisible and that which is divisible every way is body. Of magnitude that which (extends)
one way is a line, that which (extends) two ways a plane, and that which (extends) three ways a body. And there is no magnitude besides these, because
the three dimensions are all that there are, and thrice extended means extended all ways. For, as the Pythagoreans say, the All and all things
in it are determined by three things; end, middle, and beginning give the number of the All, and these give the number of the Triad.
\end{quote}

Aristotle, {\em Posterior Analytics} I.5, 74a16 \cite[p.~41]{aristotle}

\begin{quote}
And, if there had been no triangle but an isosceles triangle, the
property that its angles are together equal to two right angles would
have been thought to belong to it {\em qua} isosceles. Another case is the
theorem about proportion, that you can take the terms alternately
(i.e. {\em alternando}); this theorem used at one time to be proved
separately for numbers, for lines, for solids, and for times, though it
admitted of proof by one demonstration. But because there was no
name comprehending all these things as one, I mean numbers,
lengths, times, and solids, which differ in species from one another,
they were treated separately. Now, however, the proposition is
proved universally; for the property did not belong to the subjects
{\em qua} lines or {\em qua} numbers, but {\em qua} having a particular character
which they are assumed to possess universally.
\end{quote}

Heath writes \cite[p.~44]{aristotle}:

\begin{quote}
Aristotle does not say what general term was used in his time to
cover the four categories of things; possibly no term had yet been
definitely agreed upon, for he merely observes that the common
character was `a certain something which ``they'' assume to inhere
universally'.
If he had 
suggested a term, it would presumably have been \textgreek{pos\'on}, quantity,
{\em how much}; for in the {\em Categories} he says that quantity may be either
discrete or continuous, and sometimes it is made up of parts having
a position relative to one another in space, sometimes of parts having
no local relation, while, as examples of the discrete, he mentions
number and ratio, and, as examples of the continuous, line, surface,
body, and again, besides these, time and space. Philoponus, too,
speaks of the unknown general term as being possibly \textgreek{pos\'on}
(`whether it is \textgreek{pos\'on} or anything else whatever'). Themistius in his
paraphrase suggests no name. Euclid, of course, uses the term
`magnitudes' (\textgreek{meg\'ejh}); and this may have been his personal 
contribution to the terminology.
\end{quote}

Aristotle, {\em Posterior Analytics} I.7, 75b1 \cite[pp.~44--45]{aristotle}

\begin{quote}
You cannot therefore when proving a thing pass from one genus
to another; e.g. you cannot prove a geometrical proposition by
arithmetic. For the things required in demonstrations are three:
(1) the conclusion that is being demonstrated, (2) the axioms -- it
is the axioms from which (the proof starts), (3) the underlying genus
or subject-matter, the properties and essential attributes of which
are made clear by the demonstration. Now the things from which
the proof starts (the axioms) may be the same (whatever the subject);
but where the genus is different, say in arithmetic and geometjy,
it is not possible to apply the arithmetical demonstration to the
properties of magnitudes unless the magnitudes are numbers. There
are, however, cases where such transfer is possible as will be 
explained hereafter.
\end{quote}

Heath writes \cite[p.~45]{aristotle}:

\begin{quote}
When Aristotle says that you cannot use arithmetic to prove
the properties of magnitudes `unless the magnitudes are numbers'
he does not express an opinion whether magnitudes can be numbers,
but the probabilities are that he would not include numbers among
magnitudes. Magnitude with Aristotle is generally connected with
{\em body}; it has one, two, or three dimensions, it is `continuous one,
or two, or three ways'; `all body has depth, this being the third
(kind of) magnitude'; i.e. a magnitude is a line, a plane, or a body;
two magnitudes cannot be in the same place; cf. `If it is not possible
that any magnitude can be invisible, but every magnitude is visible
at a certain distance'. Magnitude in fact corresponds to one of the
two divisions of {\em quantity}, \textgreek{pos\'on}, namely the continuous (as a line,
a surface, or a body), whereas a number is {\em discrete}.
\end{quote}













Aristoxenus, {\em Elementa Harmonica}. Pitch is a continuum and notes are points on pitch. The relations between notes are ``distances''
or ``intervals'', {\em diastemata}; Barker 4.

Aristoxenus \cite[p.~127]{GMWII}

{\em Sectio Canonis}, notes are entities one of whose attributes, pitch, varies quantiatively and can be expressed
in magnitudes. Intervals are ratios of magnitudes or numbers. Barker \cite[p.~7]{GMWII}














Aristotle, {\em Metaphysics} V.13, 1020a \cite{metaphysica}:

\begin{quote}
`Quantum' means that which is divisible into two or more constituent parts of which each is by nature a `one'
and a `this' . A quantum is a plurality if it is numerable,
a magnitude if it is measurable. `Plurality' means that which is divisible potentially into non-continuous parts,
`magnitude' that which is divisible into continuous parts; of magnitude, that which is continuous in one dimension is length, in two breadth, in three depth. Of these, limited plurality is number, limited length is a line, breadth a surface, depth a solid.

Again, some things are called quanta in virtue of their
own nature, others incidentally; e.g. the line is a quantum by its own nature, the musical is one incidentally. Of the things that are quanta by their own nature some are so as 
substances, e.g. the line is a quantum (for a `certain kind of quantum' is present in the definition which states what it is), and others are modifications and states of this kind of substance, e.g. much and little, long and short, broad and 
narrow, deep and shallow, heavy and light, and all other such attributes. And also great and small, and greater and smaller, both in themselves and when taken relatively to each other, are by their own nature attributes of what is quantitative; but these names are transferred to other things also. Of things that are quanta incidentally, some are so called in
the sense in which it was said that the musical and the white were quanta, viz. because that to which musicalness and whiteness belong is a quantum, and some are quanta in the way in which movement and time are so; for these also are
called quanta of a sort and continuous because the things of which these are attributes are divisible. I mean not that which is moved, but the space through which it is moved; for because that is a quantum movement also is a quantum, and because this is a quantum time is one.
\end{quote}

Aristotle, {\em Metaphysics} IX.1 \cite{metaphysica}:

\begin{quote}
Obviously, then, in a sense the potency of acting and of being acted on is one (for a thing may be `capable' either because it can itself be acted on or because something else can be acted on by it), but in a sense the potencies are different. For the one is in the thing acted on; it is because it contains a certain originative source, and because even the matter is an originative source, that the thing acted on is acted on, and one thing by one, another by another; for that which is oily can be burnt, and that which yields in a particular way can be crushed; and similarly in all other cases. But the other potency is in the agent, e.g. heat and the art of building are present, one in that which can produce heat and the other in the man who can build. And so, in so far as a thing is an organic unity, it cannot be acted on by itself; for it is one and not two different things. And `impotence'and `impotent' stand for the privation which is contrary to potency of this sort, so that every potency belongs to the same subject and refers to the same process as a corresponding impotence. Privation has several senses; for it means (1) that which has not a certain quality and (2) that which might naturally have it but has not it, either (a) in general or (b) when it might naturally have it, and either (a) in some particular way, e.g. when it has not it completely, or (b) when it has not it at all. And in certain cases if things which naturally have a quality lose it by violence, we say they have suffered privation. 
\end{quote}

Aristotle, {\em Metaphysics} XI.4, 1061b \cite{metaphysica}:

\begin{quote}
Since even the mathematician uses the common axioms only in a special application, it must be the business of first philosophy to examine the principles of mathematics also.
That when equals are taken from equals the remainders are equal, is common to all quantities, but mathematics  studies a part of its proper matter which it has detached, e.g. lines or angles or numbers or some other kind of quantity -- not, however, {\em qua} being but in so far as each of them is continuous in one or two or three dimensions; but 
philosophy does not inquire about particular subjects in so far as each of them has some attribute or other, but speculates about being, in so far as each particular thing {\em is}.--
\end{quote}

Aristotle, {\em Metaphysics} XIII.3, 1077b--1078a \cite{metaphysica}:

\begin{quote}
It has, then, been sufficiently pointed out that the objects of mathematics are not substances in a higher degree than
bodies are, and that they are not prior to sensibles in being, but only in definition, and that they cannot exist somewhere apart. But since it was not possible for them to exist {\em in} sensibles either, it is plain that they either do not exist at all or exist in a special sense and therefore do not exist without qualification. For `exist' has many
senses. For just as the universal propositions of mathematics
deal not with objects which exist separately, apart from extended magnitudes and from numbers, but with magnitudes and numbers, not however {\em qua} such as to have magnitude or
to be divisible, clearly it is possible that there should also be both propositions and demonstrations about sensible
magnitudes, not however {\em qua}  sensible but {\em qua}  possessed of certain definite qualities.
For as there are many propositions about things merely considered as in motion, apart from what each such thing is and from their accidents, and as it is not therefore necessary that there should be either a mobile separate from sensibles, or a distinct mobile entity in the sensibles, so too in the case of mobiles there will be propositions and sciences, which treat them however
not {\em qua}  mobile but only {\em qua}  bodies, or again only {\em qua}  planes, or only {\em qua}  lines, or {\em qua}  divisibles, or {\em qua}  indivisibles having position, or only {\em qua}  indivisibles. Thus since
it is true to say without qualification that not only things
which are separable but also things which are inseparable exist (for instance, that mobiles exist), it is true also to say without qualification that the objects of mathematics exist, and with the character ascribed to them by mathematicians.
And as it is true to say of the other sciences too, without
qualification, that they deal with such and such a subject
not with what is accidental to it (e.g. not with the pale, if
the healthy thing is pale, and the science has the healthy as its subject), but with that which is the subject of each
science with the healthy if it treats its object {\em qua} healthy, with man if {\em qua} man :-- so too is it with geometry; if its
subjects happen to be sensible, though it does not treat them {\em qua} sensible, the mathematical sciences will not for that reason be sciences of sensibles nor, on the other
hand, of other things separate from sensibles. 
Many properties attach to things in virtue of their own nature as possessed of each such character; e.g. there are attributes peculiar to the animal  {\em qua} female or  {\em qua} male (yet there is no female nor male separate from animals); so that there are also attributes which belong to things
merely as lengths or as planes. 
And in proportion as we
are dealing with things which are prior in definition and
simpler, our knowledge has more accuracy, i.e. simplicity. Therefore a science which abstracts from spatial magnitude
is more precise than one which takes it into account; and a science is most precise if it abstracts from movement, but if it takes account of movement, it is most precise if it deals with the primary movement, for this is the simplest; and of this again uniform movement is the simplest form.
\end{quote}














Walter of Odington was an English Benedictine active around 1280--1330 who wrote the alchemical work
{\em Icocedron} (having twenty chapters). In chapter 20 of the {\em Icocedron} \cite[p.~334]{odington}:
  
\begin{quote}
Similar qualities are remitted according to minutes but remain in the degree of the more intense. Contrary qualities are remitted according to degrees and are named by the quality
which is greater in degree. Weakening qualities hold their more intense powers according to the degree and return to degrees the more remiss according to minutes. Augmenting
qualities advance the more remiss according to minutes and retain the proper degree. A degree is the excess of any notable quality above the mean by a distance of
60 minutes. Quantity does not augment the degree intensively but only extensively in similar qualities, for such is the intensive heat in one handful of fire that it is the same in
 the whole sphere. Nevertheless, in contrary qualities, quantity changes the degree so that two hot to one cold in the same degree makes it hot in the first degree.
 Four hot to one cold, hot in the second degree. Eight hot to one cold, hot in the third degree. Sixteen hot to 
 one cold, hot in the fourth degree. And so a notable change always has to be considered according to the thing doubled, so that if it is doubled by the operation to the first 
 degree it will be quadrupled by comparison to the mean, and so with the others. The calcination of the qualities is evident in the following table of similar remitting
 qualities, of contrary qualities, of weakening qualities, [and] augmenting qualities.
 \end{quote}

Skabelund and Thomas \cite[p.~336]{odington}:

\begin{quote}
In the main body of Odington's text it becomes clear that he regards qualities as `saturating' at four degrees, which is in accord
with the medico-alchemical tradition. Evidently the idea was that nothing is hotter than elemental fire, dryer than elemental earth, more humid than air (strangely), or colder than water,
these maxima being set arbitrarily at four degrees.
\end{quote}












McVaugh \cite[pp.~89--122]{villanovaII} writes about the significance of medieval pharmacy in
medieval natural philosophy; 145, 157, 158, 196, 292.
Arnaldi de Villanova Opera medica omnia: Aphorismi de gradibus

Canonical Medicine: Gentile Da Foligno and Scholasticism, Roger Kenneth French

The Alphabet of Galen: Pharmacy from Antiquity to the Middle Ages, Nicholas Everett








Marenbon \cite[p.~153]{marenbon1} quotes Gilbert Crispin's (ca.~1045--1117/18) {\em De anima}, which Marenbon dates as probably from the 1090's:

\begin{quote}
One and the same soul exists
entire in the various parts and pieces of the human body. There is not more of it in a large part [of the body] or less of it in a smaller part, because everywhere [in the body]
it remains one in number.
\end{quote}











Peter Lombard's {\em Sentences} was a theological casebook, and scholastic writings often were commentaries on questions or problems from the {\em Sentences}.
Lombard, {\em Sentences} book I, distinction 17, chapter 5, question 1 \cite[p.~146]{sentences}, \cite[p.~92]{silano1}:

\begin{quote}
{\em Utrum concedendum sit quod Spiritus Sanctus augeatur in homine et magis vel minus habeatur vel detur.}

Whether it is to be granted that the Holy Spirit may be increased in a person and be had or given to a greater or lesser degree.
\end{quote}













Maier \cite[pp.~279--280]{maierIII}













Oresme's commentary on Aristotle's {\em De generatione et corruptione} \cite[p.~63, note~18]{clagett1968}.













In his discussion of chapter V, ``De maximo et minimo'', of  Heytesbury's {\em Regule solvendi sophismata},
Wilson \cite[p.~83]{wilson} explains:

\begin{quote}
If the latitude is difformly difform in such manner that each part of it is uniform, then each uniform part of the latitude contributes to the intensity of the whole
in proportion to the amount of the subject (a body or length of time) over which it extends; for example, if the whiteness which extends over one-third of a body
has an intensity of 2 degrees, and the whiteness of the remainder an intensity of 6 degrees, the intensity of the whole latitude is $2(1/3)+6(2/3)=\frac{14}{3}$, or four
and two-thirds.
\end{quote}

Galen, {\em De differentiis febrium libri II} VII.275

Galen, {\em Tegni} 4, 316K--317K \cite[pp.~173--175]{LCL523}:

\begin{quote}
And the range of health as a whole will be divided into
three parts, which themselves have considerable range.
The first part will be that of healthy bodies, the second
that of neither, and the third that of diseased bodies.
\end{quote}

range: {\em platos}

Galen, {\em De Temperamentis} 2.4, K\"uhn I.

Galen talked about whether old age is naturally wet or dry
and whether children are hotter than adults. 
Galen asserts that, ``children exude more moist heat, while those in their prime exhibit less of a dry, sharp heat;
and this can be determined only by long tactile practice'' \cite[p.~221]{hankinson}. Hankinson \cite[pp.~221--222]{hankinson} writes the following:
\begin{quote}
Here Galen is groping towards two important distinctions: between temperature and quantity of heat, and between temperature and experienced heat.
But he lacks the tools, both conceptual and physical, to make them properly rigorous. Here, as elsewhere,
Galen must rely on qualitative distinctions refined as far as possible by practice. At {\em On the Composition of Drugs according to Places} ({\em Comp.Med.Loc.}) XII 2--4, Galen distinguishes four different, empirically determined degrees of qualitative power, in another domain where absolute precision is unobtainable: the determination of the powers of various drugs. Take heating: the first degree is discovered by reason alone, since it is by definition subperceptible; the second is when the heat is plain to the touch, the third what heats vigorously without burning, and finally there is the heat that actually burns; and the same goes, {\em mutatis mutandis}, for the other properties. Drugs and foodstuffs are categorized according to their potential causal powers rather than their actual tactile properties, just as wood, being inflammable, is potentially hot even when it is not burning.
\end{quote}

Aetius of Amida, {\em  Libri medicinales}, {\em Preface} to Book I \cite[pp.~224--225]{scarborough}:

\begin{quote}
The variations of the individual effects of drugs are
due to each of them being to a certain sufficiency [{\em t\={o}
epi tosonde}: ``to a certain degree'' is the modern expression]
hot or cold or dry or wet, or each having fine [or
``small''] or coarse [or ``large''] particles [or ``parts''].
The extent/measure of the degree, however, of the attachments
[lit. ``fastenings-together''] in each of the
drugs cannot be expressed with truthful accuracy. But
we have attempted to encompass and characterize them
with adequately clear terms and definitions for use in
medical practice [{\em eis t\={e}n cheian t\={e}s techn\={e}s}]. We are
demonstrating that there is one kind [{\em hen genos}] of drugs
which is [{\em aphiknoumenon}: lit. ``arriving at'' or ``coming
into''] a same {\em krasis} as our bodies, when it has received
some {\em arch\={e}} of both change and alteration [{\em alli\={o}sis}] from
the hot in this kind of drugs, and, that there is another
kind of drugs which is hotter. From this, it seems to
me that four orders [{\em taxeis}] can be made: the first is
indistinct [{\em asaph\={e}}] to the senses, [and] detecting it necessarily
comes through pure reason [{\em logos}]; the second
is distinct and perceivable to the senses; the third is
rather hot, but not to the point of burning; the fourth
and last is the corrosive or caustic kind of drugs.
Likewise also for the cooling kind of drugs, the first
order must come from pure reason in demonstrating
its coldness, the second is cooling detectable by the
senses, the third is rather cold, and the fourth causes
necrosis. Analogous in these definitions are also the
wetting and drying drugs.
\end{quote}

Jabir ibn Hayyan, {\em Kitab Al-Ahjar} \cite{jabir}















{\em De generatione et corruptione} \cite{williams}, II 334b8--16

{\em De caelo} I.11, 273a22--27, 281a1--17, 281a18--27, 311a15--20

{\em Physics}, 5.2 (226b), 6.5, 7.3, 8.8

Philoponus, {\em in GC} 146 \cite[p.~48]{philoponus16}, on 324a14

Philoponus, {\em in Phys.}

Themistius, {\em in Phys.} \cite{themistius58}

Simplicius, {\em in Caelo} \cite[pp.~32--36]{simpliciuscaelo110}

Simplicius, {\em in Phys.}
























Peter of Spain, {\em Summaries of Logic} 3.12 \cite[p.~157]{copenhaver}:

\begin{quote}
Also, substance
does not take more and less. My point, however, is not that one substance is not more
underlying than another; my point is that each and every substance, regarding its own
being, is neither strengthened nor weakened, as when white is sometimes more white
and sometimes less. But Sortes is not more a man at one times than at another, nor is he
more a man than Plato.
\end{quote}











Campanus, {\em Elements} I, Definition ii \cite[p.~55]{campanusI}:

\begin{quote}
Linea est longitudo sine latitudine,
\end{quote}







\bibliographystyle{amsplain}
\bibliography{latitude}

\end{document}