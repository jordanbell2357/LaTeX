Parker, Demotic Mathematical Papyri \cite{DMP}, Problem 7:
\[
\sqrt{1500} \sim 38+\frac{2}{3}+\frac{1}{20}.
\]

Demotic Mathematical Papyri, Parker, Problem 32, Problem 36
\[
\sqrt{133+\frac{1}{3}} \sim 11+\frac{1}{2}+\frac{1}{20}.
\]

In {\em De Re Rustica} III.XVI.4--5 \cite[p.~501]{varro}, after stating that nature has given great talent and art to bees, Varro writes:
 
\begin{quote}
Bees are not of a solitary nature, as eagles are, but are like human beings. Even if jackdaws in this respect are the same, still it is not the same case; for in one there is a fellowship
in toil and in building which does not obtain in the other; in the one case there is reason and skill -- it is from these that men learn to toil, to build, to store up food. They have three 
tasks: food, dwelling, toil; and the food is not the same as the wax, nor the honey, nor the dwelling. Does not the chamber in the comb have six angles, the same number as the bee 
has feet? The geometricians prove that this hexagon inscribed in a circular figure encloses the greatest amount of space.
\end{quote}

IX, Introduction 4--5 \cite[p.~243]{vitruvius}:

\begin{quote}
[4] First of all, I will explain one of Plato's many exceptionally useful theorems as he formulated it. If there is a site
or square field, that is, one with equal sides, which we have to double, the solution can be found by drawing lines accurately,
since we will need a type of number that cannot be arrived at by multiplication. The proof of this is as follows: a square site
ten feet long and ten feet wide produces an area of a hundred square feet. If, then, we need to double it and produce a square
of two hundred feet, we must find out how long the side of the square would be to obtain from it the two hundred feet corresponding
to the doubling of the area. Nobody can discover this by calculation: for if we take the number fourteen, multiplication will give a 
hundred and ninety-six square feet; if we take fifteen, it will give two hundred and twenty-five square feet.

[5] Therefore, since we cannot solve this problem arithmetically, a diagonal line should be drawn in the ten foot square
from angle to angle so that it is divided into two triangles of equal size, each fifty feet in area; a square with equal sides
should be drawn along the length of this diagonal. In this way four triangles will be produced in the larger square of the same
size and number of feet as the two triangles of fifty square feet created by the diagonal in the smaller square. The problem of
doubling an area was solved by Plato with this procedure using geometrical methods, as is shown in the diagram at the foot
of the page.
\end{quote}

Then in IX, Introduction 6 \cite[pp.~243--244]{vitruvius}:

\begin{quote}
Again, Pythagoras demonstrated how to devise a set-square without the intervention of workmen; the results which workmen
arrive at when they make set-squares, with considerable effort but without great accuracy, can be arrived at with precision
using the principles and methods derived from his teachings. For if we take three rulers, three, four and five feet long,
and assemble them with their ends touching in the form of a triangle, they will form a perfect set-square. If squares with equal
sides are drawn along the lengths of each ruler, the three-foot side will produce an area of nine square feet, the four-foot
side an area of sixteen square feet and the five-foot side an area of twenty-five square feet.
\end{quote}

In {\em De Re Rustica} V.I.4--8 \cite[pp.~5--7]{columellaII}, Columella defines  measures of area:

\begin{quote}
But to return to my subject, the extent of every area is reckoned by measurement in {\em feet}, and a foot consists of 16 {\em fingers}. The multiplication of the foot produces 
successively the {\em pace}, the {\em actus}, the {\em clima}, the {\em iugerum}, the {\em stadium} and the {\em centuria}, and afterward still larger measurements. The {\em pace} 
contains five feet. The {\em smallest actus} (as Marcus Varro says) is four feet wide and 120 feet long. The {\em clima} is 60 feet each way. The {\em square actus} is bounded by
120 feet each way; when doubled it forms a {\em iugerum}, and it has derived the name of {\em iugerum} from the fact that it was formed by joining. This {\em actus} the country folk
of the province of Baetica call {\em acnua}; they also call a breadth of 30 feet and a length of 180 feet a {\em porca}. The Gauls give the name  {\em candetum} to areas of a
hundred feet in urban districts but to areas of 150 feet in rural districts they also call a half-{\em iugerum} an {\em arepennis}. Two {\em actus}, as I have said, form a {\em iugerum} 
240 feet long and 120 feet wide, which two numbers multiplied together make 28,800 square feet. Next a {\em stadium} contains 125 paces (that is to say 625 feet) which
multiplied by eight makes 1000 paces, which amount to 5000 feet. We now call an area of 200 {\em iugera} a {\em centuria}, as Varro again states; but formerly the {\em centuria}
was so called because it contained 100 {\em iugera}, but afterwards when it was doubled it retained the same name, just as the tribes were so called because the people were 
divided into three parts but now, though many times more numerous, still keep their old name. It was proper that we should begin by briefly mentioning these facts first, as being 
relevant to and closely connected with the system of calculation which we are going to set forth.
\end{quote}

Then in V.I.8--13 \cite[pp.~9--13]{columellaII} he defines different fractions of the {\em iugerum}:

\begin{quote}
Let us now come to our real purpose. We have not put down all the parts of the {\em iugerum} but only those which enter into the estimation of work done. For it was needless to
follow out the smaller fractions on which no business transaction depends. The {\em iugerum}, therefore, as we have said, contains 28,800 square feet, which number of feet is 
equivalent to 288 {\em scripula}. But to begin with the smallest fraction, the half-{\em scripulum}, the 576th part of a {\em iugerum}, contains 50 feet; it is the haif-{\em scripulum} of
the {\em iugerum}. The 288th part of the {\em iugerum} contains 100 feet; this is a {\em scripulum}. The 144th part contains 200 feet, that is two {\em scripula}. The 72nd part contains 
400 feet and is a {\em sextula}, in which there are four {\em scripula}. The 48th part, containing 600 feet, is a {\em sicilicus}, in which there
are six {\em scripula}. The 24th part, containing 1200 feet, is a {\em semi-uncia}, in which there are 12 {\em scripula}. The 12th part, containing 2400 feet, is the {\em uncia}, in
which there are 24 {\em scripula}. The 6th part, containing 4800 feet, is a {\em sextans}, in which there are 48 {\em scripula}. The 4th part, containing 7200 feet is a {\em quadrans},
in which there are 72 {\em scripula}. The 3rd part, containing 9600 feet, is a {\em triens}, in which there are 96 {\em scripula}. The 3rd part plus the 12th part, containing 12,000 feet,
is the {\em quincunx}, in which there are 120 {\em scripula}. The half of a {\em iugerum}, containing 14,400 feet, is a {\em semis}, in which there are 144 {\em scripula}. A half plus
a 12th part, containing 16,800 feet, is a {\em septunx}, in which there are 168 {\em scripula}. Two-thirds of a {\em iugerum}, containing 19,200 feet, is a {\em bes}, in which there are
192 {\em scripula}. Three-quarters, containing 21,600 feet, is a {\em dodrans}, in which there are 216 {\em scripula}. A half plus a third, containing 24,000 feet, is a {\em dextans},
in which there are 240 {\em scripula}. Two-thirds plus a quarter, containing 26,400 feet, is a {\em deunx}, in which there are 264 {\em scripula}. A {\em iugerum}, containing 28,800
feet, is the {\em as}, in which there are 288 {\em scripula}. If the form of the {\em iugerum} were always rectangular and, when measurements were being taken, were always
240 feet long and 120 feet wide, the calculation would be very quickly done; but since pieces of land of different shapes come to be the subjects of dispute, we will give below
specimens of every kind of shape which we will use as patterns.
\end{quote}

The area $A$ of an equilateral triangle whose sides have length $a$ is
\[
A=\sqrt{3} \cdot \frac{a^2}{4}.
\]
In V.II.5 \cite[pp.~15--17]{columellaII}, Columella writes:

\begin{quote}
But if you have to measure a triangle with three equal sides, you will follow this formula. Suppose the field to be triangular, three hundred feet on every side. Multiply this number by itself and the result is 90,000 feet. Take a third part of this sum, that is
30,000. Likewise take a tenth part, that is 9,000. Add the two numbers together; the result is 39,000. We shall say that this is the total number of square feet in this triangle, which measure makes a {\em iugerum}, plus a {\em triens} ($\frac{1}{3}$), plus a {\em sicilicus} ($\frac{1}{48}$).
\end{quote}

This amounts to 
\[
A \sim \frac{a^2}{3}+\frac{a^2}{10} = a^2 \cdot \frac{13}{30},
\]
which implies $\frac{\sqrt{3}}{4} \sim \frac{13}{30}$, or $\sqrt{3} \sim \frac{26}{15}$. 

The area of a regular hexagon whose sides have length $a$ is
\[
A = \sqrt{3} \cdot \frac{3a^2}{2}.
\]
In V.II.10 \cite[pp.~21--23]{columellaII}, Columella writes:

\begin{quote}
If the area has six angles, it is reduced to square  feet in the following manner. Let there be a hexagon, each side of which measures 30 feet. I multiply one side by itself:
30 times 30 makes 900. Of this sum I take one-third, which is 300, a tenth part of which is 90: total 390. This must be multiplied
by 6, because there are 6 sides: the product is 2310. We shall say, therefore, that this is the number of square feet. It will, then, be equivalent to an {\em uncia}
($\frac{1}{12}$ of a {\em iugerum}) less half a {\em scripulum} ($\frac{1}{596}$) plus $\frac{1}{10}$ of a {\em scripulum}.
\end{quote}

This amounts to
\[
A \sim 6 \cdot \left( \frac{a^2}{3}+\frac{a^2}{10} \right) = a^2 \cdot \frac{39}{15},
\]
which implies $\frac{3\sqrt{3}}{2} \sim  \frac{39}{15}$, or $\sqrt{3} \sim \frac{78}{45}$. 

Frontinus, {\em De Aquaeductu Urbis Romae} 24--25 \cite{rodgers}:

\begin{quote}
[24] Water pipes have been calibrated to measurement either in digits or in inches. Digits are employed in Campania and in most parts of Italy, but inches are still accepted
as standard in Apulia. (2) A digit, by convention, is one-sixteenth part of a foot, while an inch is one-twelfth. (3) Just as there is a distinction between the inch and the digit, there are
also two kinds of digits. (4) One is called square, the other round. (5) The square digit is larger than the round by three-fourteenths of its own size; the round digit is smaller than the
square by three-elevenths of its size (because, of course, the corners are taken away). [25] Later, a pipe called the 5-pipe ({\em quinaria}) came into use in the City to the exclusion 
of all former sizes. Its origin was based neither on the inch nor on either of the two kinds of digit. Some think that Agrippa was responsible for its introduction, others that this was done
by the lead-workers under the influence of the architect Vitruvius. (2) Those who credit Agrippa with its currency derive its name from the suggestion that into one such pipe were 
combined five of the slender ancient pipes (we might say little tubes) used for distributing the supply of water which in those times was not copious. Those who ascribe the 5-pipe to 
Vitruvius and the lead-workers suppose that its origin lay in producing a cylindrical pipe from a sheet of lead five digits in width. (3) The latter explanation is inexact, because in 
forming a cylindrical shape the inner surface is contracted while the outer surface is extended. (4) Most probable is the explanation that the name of the 5-pipe came from its diameter 
of five quarter-digits, (5) according to a system which remains consistent in pipes of increasing size up as far as the 20-pipe: the diameter of each increases in size by the addition of
one quarter-digit. For example, the 6-pipe has a diameter of six quarter-digits, the 7-pipe has seven, and so on by uniform increment up to a 20-pipe.
\end{quote}

See Rodgers \cite[pp.~209--211]{frontinus}.

26--29 \cite{rodgers}:

\begin{quote}
[26] The size of any pipe is determined either by its diameter, or its circumference, or the measure of its cross-section; from any one of these factors its capacity is evident. (2) That
we may more conveniently distinguish between the inch, the square digit, the round digit, and the 5-pipe itself, we need to treat ``the {\em quinaria}'' (5-pipe equivalent) as a unit of 
capacity, for its size is most accurate and its standard best established. (3) The inch pipe has a diameter of $1 \frac{1}{3}$ digits; its capacity is a little more than $1 \frac{1}{8}$
{\em quinariae}, the fraction being $\frac{1}{8}$ plus $\frac{3}{288}$ plus $\frac{2}{3}$ of another $\frac{1}{288}$. (4) A square digit converted to circular shape has a diameter of
$1 \frac{5}{36}$ digits; its capacity is $\frac{5}{6}$ of a {\em quinaria}. (5) A round digit has a diameter of 1 digit; its capacity is $\frac{23}{36}$ of a {\em quinaria}. [27] Now the
pipes based on the 5-pipe are increased in size in two ways. (2) One is by multiplying the 5-pipes themselves, that is by including the equivalent of several 5-pipes into one opening,
with the size of that opening increasing according to the addition of more 5-pipe equivalents. (3) This approach is more or less limited to instances where a number of {\em quinariae}
have been granted: to avoid tapping the conduit too often, a single pipe is used to lead the water into a delivery-tank, and from here individual persons draw off their respective
shares. [28] The second way does not involve an increase in pipe size related to a necessary number of 5-pipes. Instead, the increase is in the diameter of the pipe itself, a change
which alters both its name and its capacity. Take, for example, the 5-pipe: add a sixth quarter-digit to its diameter, and one has a 6-pipe, (2) but the capacity is not increased by an
entire 5-pipe equivalent (it has only $1 \frac{7}{16}$ {\em quinariae}). (3) By adding quarter-digits to the diameter in the same manner, as already explained, one gets larger pipes, a 7-pipe, an 8-pipe, and so on up to the 20-pipe. [29] Beyond the 20-pipe the gauging is based on the number of square digits which are contained in the cross-section, that is the
opening, of each pipe. From this same number the pipes also take their names. (2) Thus that pipe with an area of 25 square digits is called the 25-pipe; likewise the 30-pipe, and so
on by increase in square digits, up to the 120-pipe. 
\end{quote}

See Rodgers \cite[pp.~212--215]{frontinus}.

Faventinus, {\em De Diversis Fabricis Architectonicae} 28 \cite[p.~80]{plommer}:

\begin{quote}
Quoniam ad omnes usus normae ratio subtiliter inventa videtur, sine quo nihil utiliter fieri potest, hoc modo erit disponenda. sumantur itaque tres regulae, ita ut duae
sint pedibus binis et tertia habeat pedes duo uncias x. eae regulae aequali crassitudine compositae extremis acuminibus iungantur schema facientes trigoni. sic
fiet perite norma composita.
\end{quote}

A {\em norma} is a set-square, a right triangle.

Faventinus, {\em De Diversis Fabricis Architectonicae} 28 \cite[p.~81]{plommer}:

\begin{quote}
Since the principle of the square was a clever discovery and useful for all purposes -- since, indeed, nothing can be done very practically without it,
this is how you will prepare one. Take three scales, two of them 2 foot long, the third, 2 foot 10 inches. They are all to be of one uniform width, and are to be joined
at the ends to give the shape of a triangle. Your square will thus be made to professional standards.
\end{quote}

cf. {\em tegulae bipedales}

Vegetius, {\em Epitoma Rei Militaris} 
I.22 \cite[p.~24]{vegetius}: ``The camp should be built according to the number of soldiers and baggage-train, lest too great
a multitude be crammed in a small area, or a small force in too large a space be compelled to be spread
out more than is appropriate.''
I.23 \cite[p.~24]{vegetius}: ``Camps should be made sometimes square, sometimes triangular,
sometimes semicircular, according as the nature and demands of the site require.''

II.7 \cite[pp.~38--39]{vegetius}: ``Quartermasters measure out the places in camp according to the square footage
for the soldiers to pitch their tents, or else assign them billets in cities.''

III.8 \cite[p.~80]{vegetius}: ``When these conditions have been carefully and stringently investigated, you may build
the camp square, circular, triangular or oblong, as required by the site. Appearance should not prejudice utility, although those
whose length is one-third longer than the width are deemed more attractive. But surveyors should calculcate
the square footage of the site-plan so that the area enclosed corresponds to the size of the army. Cramped quarters
constrict the defenders, whilst unsuitably wide spaces spread them thinly.''

III.15 \cite[p.~97]{vegetius}: ``We said that 6 ft. ought to lie between each line in depth from the rear, and in fact each
warrior occupies 1 ft. standing still. Therefore, if you draw up six lines, an army of 10,000 men will take up 42 ft.
in depth and a mile in breadth. [If you decide to draw up three lines, an army of 10,000 will take up 21 ft. in depth and two
miles in breadth.] In accordance with this system, it will be possible to draw up even 20,000 or 30,000 infantry without the slightest
difficulty, if you follow the square footage for the size. The general does not go wrong when he knows what space
can hold how many fighting men.'' 

Palladius Rutilius Taurus Aemilianus, {\em Opus agriculturae} II.11, {\em De tabulis uinearum} \cite[p.~54]{palladii}:

\begin{quote}
Tabulas autem pro domini uoluptate uel loci ratione
faciemus siue integrum iugerum continentes seu medium
seu quaternariam tabulam, quae quartam iugeri partem
quadrata conficiet.
\end{quote}

The word {\em tabula} is said by Souter, {\em A Glossary of Later Latin to 600 A.D.}, s.v., to mean a ``stretch (of land) in a vineyard''.

Fitch \cite[p.~75]{fitch}:

\begin{quote}
We shall make the planting-beds in accordance with the owner's inclination or the requirements of the place, covering a whole juger or half or a quarter-bed, which consists of a fourth of a juger in square footage. 
\end{quote}

II.12, {\em De mensura pastini Italica} \cite[p.~55]{palladii}:

\begin{quote}
Mensura uero pastini haec est in tabula quadrata iugerali,
ut centeni octogeni pedes per singula latera dirigantur,
qui multiplicati trecentas uiginti et quattuor decempedas
quadratas per spatium omne conplebunt.
secundum hunc numerum omnia quae uolueris pastinare discuties.
decem et octo enum decempedae decies et octies subputatae
trecentas uiginti quattuor explebunt.
quo exemplo doceberis in maiore agri uel minore mensuram. 
\end{quote}

The word {\em iugeralis} is said by Souter, {\em A Glossary of Later Latin to 600 A.D.}, s.v., to mean ``of the land-measure called iugerum'' or ``very large''. 
Rodgers \cite[p.~96]{palladius}:

\begin{quote}
With 32400 sq.ft., P.'s {\em tabula iugeralis} is larger by 3200 sq.ft. than a normal {\em iugerum} ($240 \times 120$ ft.),
but P. is careful to explain that he calculates his {\em tabula} with 180 ft. on a side. I wonder if he arrived at the length of one side
of his ``squared'' {\em iugerum} by dividing the perimeter of a {\em iugerum} ($2 \times 240+2 \times 120=720$) by four equal
sides ($720/4=180$). He is at pains to tell us that the total area is 324 {\em decempedae quadratae}, and I suppose  it is
possible for him to say (2.11) that the {\em tabula} will contain an {\em integrum iugerum}. With {\em medium} (2.11) he must mean
half the area of a {\em iugerum} (traditionally called an {\em actus}, 120 ft. on a side) or 14400 sq.ft.; I doubt that he would
have been meaning half of 32400 sq.ft., which would be $10 \times \surd 162$ ft. on a side. His {\em quaternaria tabula}, I think,
would be 90 ft. on a side or 8100 sq.ft. (one-fourth of 32400) rather than one-fourth the area of a {\em iugerum}, 7200 sq.ft.,
$10 \times \surd 72$ ft. on a side. No-one, I am sure, would object to these rough approximations, least of all P. himself (for his mathematical inexactitude, see my note
on 3.9.9).
\end{quote}

Fitch \cite[p.~75]{fitch}:

\begin{quote}
In a square planting-bed covering one juger, the measurement of the prepared ground is 180 feet on each straight side; when multiplied this will yield 324 10-foot square units across the whole area. Using this figure, you will divide up all the ground you want to prepare. For 18 10-foot lengths multiplied 18 times will yield 324.
This example will show you how to measure a larger or smaller field.
\end{quote}

Folkerts \cite{folkerts}

{\em Podismus} \S 7 \cite[pp.~134--137]{guillaumin} states Heron's formula for the right triangle with sides
$6,8,10$; the area of the triangle is $26$.  

We refer to the tractate in the {\em Corpus agrimensorum} attributed to Epaphroditus and Vitruvius Rufus by {\em EVR}. {\em EVR}  
\S 10 \cite[pp.~140--141]{guillaumin}: let $ABCD$ be a right trapezium where $AB$ and $DC$ are parallel, $ADC$ is a right angle,
$AB=25$ feet, $DC=40$ feet, $DA=30$ feet; call $AB$ the summit, $BC$ the hypotenuse, $DC$ the base, and $AD$ the height.
The recipe given for finding the area
of the right triangle with height $AD$ and hypotenuse $BC$ is the following:
add the base $DC$ and the summit $AB$, getting $65$,  take half of this, getting $32 \frac{1}{2}$, and multiply this
by the height $AD$, getting $975$. The recipe given for finding the
hypotenuse $BC$ is the following: add the 
squares on the summit, the base, and the height, getting $3125$. 
Subtract from this twice the product of the base and the summit, i.e. subtract $2 \cdot 25 \cdot 40 = 2000$ from $3125$, getting $1125$. Then $BC$ is the side
of the square $1125$, namely $BC^2=1125$. That is, 
\[
BC^2 = AB^2+DC^2+AD^2 - 2  DC \cdot AB = (DC-AB)^2 + AD^2.
\]
It is stated that $BC$ is $33 \frac{1}{2}$; indeed, $33^2=1089$ and $34^2=1156$.

{\em EVR} \S 11 \cite[pp.~140--143]{guillaumin}: for an equilateral triangle whose sides are $30$ feet,
multiply a side by itself, getting $30\cdot 30 = 900$. Multiply half a side by itself, getting $15\cdot 15 = 225$. Then take away $225$ from
$900$, getting $675$, which is the area. It is stated that the side of the square $675$ is $26$. (Indeed, $25^2=625$ and $26^2=676$.)
This is the height of the triangle.
Then multiply  the height by half the base, getting $26\cdot 15 = 390$. This is the area of the triangle.

{\em EVR} \S 28 \cite[pp.~158--163]{guillaumin}: for an equilateral triangle whose sides are even numbers,
to find the area. Guillaumin explains that in \S \S 28, 30--37 
figurate numbers are being used:
for the triangular number whose each have $n$ pebbles, the figure contains $\frac{n^2+n}{2}$ pebbles; cf.
Nicomachus, {\em Introductio Arithmetica} II.7--12 \cite[pp~239--249]{nicomachus} and Heath \cite[p.~76]{HGMI}. 
The example is given of the equilateral triangle whose sides are $28$ feet, 
multiply a side by itself, getting $28\cdot 28=784$. Add a side to this, getting $784+28=812$. Take half of this, getting $406$. It is asserted that this
is the area of the triangle. (The height of the triangle is $h=\sqrt{28^2-14^2}=\sqrt{588}$, which satisfies $24<h<24 \frac{1}{4}$. 
Then the area of the triangle is half the product of the base and the height, i.e. $\frac{28\cdot h}{2}$, and using $h=24 \frac{1}{4}$ this is
$339 \frac{1}{2}$.)
Conversely the side of a triangle is found given the area.
Multiply the area by $8$, getting $8\cdot 406 = 3248$. Add $1$ to this, getting $3249$. The side of this square is 
$57$. Remove $1$ from this, getting $56$. Take half of this, getting $28$, which is the side of the triangle. 

For an $a$-gonal number with $n$ pebbles on each side, the figure contains 
\[
\frac{(2+(2n-1)(a-2))^2-(a-4)^2}{8(a-2)}
\]
pebbles; cf. Heath \cite[p.~516]{HGMII}.
Conversely, if the figure contains $P$ pebbles, then
\[
n=\frac{1}{2}\left( \frac{\sqrt{8P(a-2)+(a-4)^2}-2}{a-2}+1\right).
\]
{\em EVR} \S 29 \cite[pp.~164--167]{guillaumin} states that for a pentagon with equal sides, multiply a side by itself,
multiply this by $3$, then add one side, and that this gives the pentagon. If the sides are each $10$ feet, multiply
a side by itself, getting $100$. Multiply this by $3$, getting $300$. Add a side to this, getting $310$. Take
half of this, getting $155$, which is said to be the area of the pentagon. 
Conversely, if the area is $155$, to find the side do the following: multiply the area by $24$, getting
$24 \cdot 155 = 3720$. Add $1$ to this, getting $3721$. Find the side of the square $3721$, which is $61$. 
Remove $1$ from this number, getting $60$. Take a sixth of this, getting $10$, which is the said to be the side of the
pentagon.

{\em EVR} \S 31 \cite[pp.~172--177]{guillaumin}: for a hexagon with equal sides, multiply
a side by itself, multiply this by $4$, add twice a side to this, and then take half of this, and it is asserted
that this gives the pentagon. If the sides are $10$ feet, multiply a side by itself, getting $100$. 
Multiply this by $4$, getting $400$. Add twice a side to this, getting $400+2\cdot 10=420$. Take half
of this, getting $210$. It is asserted that this is the area of the hexagon. Conversely,
given the area of the hexagon, find the side. Multiply the area by $32$, getting $32 \cdot 210 = 6720$.
Add $4$ to this, getting $6724$. Find the side of the square $6724$, which is $82$. Remove $2$ from this,
getting $80$. Take an eighth of this, getting $10$. It is asserted that this is the side of the hexagon.

{\em EVR} \S 32 \cite[pp.~176--179]{guillaumin} states that for a heptagon with equal sides, multiply a side by
itself, multiply this by $5$, remove three times a side from this, and then take half of this, and it is asserted that
this is gives the heptagon. If the sides are $10$ feet,
multiply a side by itself, getting $100$. Multiply this by $5$, getting $500$. Remove three times a side from this,
getting $500-3\cdot 10=470$. Take half of this, getting $235$, which is asserted to be the area of the hexagon.
Conversely,
given the area of the heptagon, find the side. Multiply the area by $40$, getting $40 \cdot 235 = 9400$.
Add $9$ to this, getting $9409$. Find the side of the square $9409$, which is $97$. Add $3$ to this, getting $100$.
Take a tenth of this, getting $10$. It is asserted that this is the side of the heptagon.

{\em EVR} \S 33 \cite[pp.~178--179]{guillaumin} states that for an octagon with equal sides, 
multiply a side by itself, multiply this by $6$, remove four times a side from this, and then take half of this, and it is asserted that this gives the octagon.
If the sides are $10$ feet, multiply a side by itself, getting $100$. Multiply this by $6$, getting $600$. Remove four times a side from this,
getting $600-4\cdot 10=560$. Take half of this, getting $280$, which is asserted to be the area of the octagon. Conversely,
given the area of the octagon, find the side. Multiply the area by $48$, getting $48 \cdot 280 = 13440$. Add $16$ to this, getting $13456$. Find the side
of the square $13456$, which is $116$. Add $4$ to this, getting $120$. Take a twelfth of this, getting $10$, which is asserted to be the side
of the octagon. 

{\em EVR} \S \S 34--37 \cite[pp.~180--187]{guillaumin} treat respectively the enneagon, the decagon, the hendecagon, and the dodecagon. 

{\em De iugeribus metiundis} \S 54 \cite[pp.~198--201]{guillaumin}, cf.  \cite[p.~354--356]{blumeI}:

\begin{quote}
Castrense iugerum quadratas habet perticas CCLXXXVIII, pedes autem quadratos $\overline{\text{X}} \overline{\text{X}} \overline{\text{VIII}}$DCCC, 
id est per latus unum perticas XVIII, quae in quattuor latera faciunt perticas LXXII; habet itaque tabula una quadratas perticas LXXII. Si ergo
fuerit ager tetragonus isopleurus, habens per latus unum perticas L, ita eum metiri oportet ut sciamus quot iugera habeat intra se. Duco unum latus
per aliud: fiunt perticae $\overline{\text{II}}$D, quae faciunt iugera VIII, tabulas II, perticas LII. Itaque castrense iugerum capit k(astrenses) modios III.
\end{quote}

It is first stated that a {\em iugerum} contains $288$ square {\em perticae}. 
A {\em iugerum} is a rectangle with sides 240 feet and 120 feet,  thus whose area is $28800$ square feet. A {\em pertica} is 
a length of $10$ feet; see Balblus, {\em Expositio et ratio omnium formarum} \cite[p.~207]{campbell},
{\em Centuriarum quadratarum deformatio sive mensurarum diversarum ritus} \cite[p.~241]{campbell},
and {\em De mensuris agrorum} \cite[p.~271]{campbell}. (Thus, one iugerum contains $288$ square perticae.)
Next it is asserted that the side of the square $28800$ is $18$ perticae, whose perimeter is $72$ perticae.
In fact, $169^2<28800<170^2$, while
a square with side $18$ perticae contains $32400$ square feet. Guillaumin remarks
that the sides of the iugerum are 24 perticae and 12 perticae, and 
$18$ perticae is the arithmetic mean of these. If a rectangle has sides 
$a$ and $b$ with $b > a$, then the square with side $\frac{a+b}{2}$ has the same perimeter as the rectangle, namely
$a+a+b+b$, and has area $\left(\frac{a+b}{2}\right)^2 = \frac{a^2+b^2+2ab}{4}$, while the rectangle has area $ab$, for which
\[
\frac{a^2+b^2+2ab}{4}-ab = \frac{a^2+b^2-2ab}{4} = \frac{1}{4} (b-a)^2.
\]
Thus, the square with the same perimeter as the rectangle has greater area.
It is stated that one {\em tabula} contains $72$ square perticae, i.e., one {\em tabula} contains $7200$ square feet, namely, one {\em tabula} is a quarter of one iugerum.
Then, for a square field whose sides are $50$ perticae,  find how many iugera it contains. Multiply one side of the square by another, getting 
$2500$ square perticae. As $2500=8\cdot 288+196$, 
this field contains $8$ iugera and $196$ square perticae. As $196=2\cdot 72+52$, the remaining $196$ square feet contain
$2$ {\em tabulae} and $52$ square feet; thus the field contains $8$ iugera, 
$2$ {\em tabulae}, and $52$ square perticae.  

\S 56 \cite[pp.~202--203]{guillaumin}:

\begin{quote}
Ager si fuerit trigonus isopleurus, habens tria latera per
quae sexagenas perticas habeat, duco unum latus per alterius
lateris medietatem, id est LX per XXX: fiunt perticae MDCCC,
quae faciunt iugera VI, tabulam unam.
\end{quote}

If a field is an equilateral triangle whose sides are 60 perticae, multiply one side by half another, giving $60 \cdot 30=1800$ square perticae.
$1800=6 \cdot 288+72$, so this is $6$ iugera 1 tabula.

\S 57 \cite[pp.~202--203]{guillaumin}:

\begin{quote}
Ager si caput bubulum fuerit, id est duo trigona isopleura
iuncta, habentia per latus unum perticas L, unius trigoni latus in
alterius trigoni latus duco, id est L per L: fiunt $\overline{\text{II}}$D, quae sunt
iugera VIII, tabulae IIS, perticae XVI.
\end{quote}

If a field is two joined equilateral triangles (a ``head of beef''), whose sides are 50 perticae, multiply the side of one triangle by the side of the other triangle,
that is $50 \cdot 50=2500=8\cdot 288+2\cdot 72+36+16$. That is, the area is $2500$ square perticae, which is $8$ iugera, $2 \frac{1}{2}$ tabulae, 16 square perticae.

\S 63 \cite[pp.~210--211]{guillaumin}:

\begin{quote}
Ager si fuerit sex angulorum, in quadratos pedes sic
redigitur. Esto exagonum in quo sint per latus unum perticae XXX.
Latus unum in se multiplico, id est tricies triceni: fiunt perticae
DCCCC. Huius summae tertiam partem statuo, id est CCC. Nihilominus
ex eadem pleniori summa decimam partem tollo,  id est
XC. Quae pariter iunctae faciunt CCCXC. Quae sexies ducendae
sunt, quia sex latera habet: quae summa colligit perticas $\overline{\text{II}}$CCCXL.
Tot igitur quadratas perticas in hoc agro esse dicimus.
\end{quote}

For a field that is a hexagon where each side is 30 perticae. 
Multiply one side by itself, getting $30 \times 30=900$. Take a third of $900$, which is $300$, and a tenth of $900$, which is $90$. The sum of these two
is $300+90=390$. Multiply this by $6$, getting $2340$.
The area of the field is $2340$ square perticae. cf. Heron in Heath \cite[p.~327]{HGMII}






Marcus Junius Nipsus, {\em Limitis Repositio} \cite[p.~51]{bouma}:

\begin{quote}
In agris divisis subsiciva fiunt, in quibus trigona, trapezea et
pentagona sunt, et nihil alius nisi modus iugerum adsignatorum et
nomen scriptum est. Actus tamen in base sunt xx. Sic ut puta in
pentagono liis, bis ducti, faciunt cv. Qui in se ducti, faciunt
iugera lxv. Cathetum sic quaerimus semper. Embadum duco quater -- id
est lxv --; fiunt cclx. Huius summae pars vicesima fit xiii; erit
cathetus. In trigono sunt actus xlii, iugera cl. Insequentem actum
iunctum trigono ac trapezeo similiter. Quae si autem fuerint in
trapezeo iugera c, iugera ducta quater, erunt cccc. Horum pars
vicesima -- hoc est xx -- erit basis. Deducto contrario -- id est xx -- 
fit reliquum vii. Erit contraria basis actus vii. Similiter in
reliquius pedibus, si fuerint cc.
\end{quote}

Bouma \cite[p.~73]{bouma} translates:

\begin{quote}
When dividing land, pieces of land remain; these can be triangles,
trapeziums and pentagons; and nothing else but the number of {\em iugera}
assigned and their name has been written down (on the {\em forma}). Yet there are
20 {\em actus} at the base. Thus for instance $52 \frac{1}{2}$ ({\em actus}) in a pentagon make,
multiplied by two, 105 ({\em actus}). Together they comprise 65 {\em iugera}. We
always seek the perpendicular as described below. I multiply the area (of the
pentagon) --  that is 65 {\em iugera} -- by four. This makes 260 ({\em iugera}). From these
the 20th part makes 13. This will be the perpendicular.

In a triangle are 42 {\em actus}, 150 {\em iugera}. Likewise (we want to know) the next
(number of) {\em actus} of triangle and trapezium. If there are 100 {\em iugera} in a 
trapezium, there will be, when multiplied by four, 400 ({\em iugera}). The 20th
part of these (400 {\em iugera}) -- that is 20 -- will be the base. When subtracted
from the opposite base -- that is 20 --, 7 remain: The opposite base will be 7
{\em actus}. The same goes for the other feet, if they are 200.
\end{quote}

For an isosceles trapezium with base $a$, summit $b$, and sides $c$ and $c$, with $b>a$,
let $h$ be its height.
Then $h^2+\left( \frac{b-a}{2} \right)^2 = c^2$, and the area of the trapezium is 
$A=ah+\frac{1}{2}(b-a)h=\frac{1}{2}(b+a)h$. Now,
\[
\frac{b+a}{2} \cdot \frac{c+c}{2} - A
=\frac{b+a}{2} \cdot c -\frac{1}{2}(b+a)h
=\frac{b+a}{2} (c-h).
\]
Thus, $\frac{b+a}{2} \cdot \frac{c+c}{2}$ is greater than the area of the trapezium, as $c>h$.

{\em Vaticanus Palatinus graecus} 367, ff.~94r--97v, no.~23 \cite[p.~51]{fisc}:
for a trapezium with base 16 {\em orgyiai}, summit 
20 {\em orgyiai}, and sides each 25 {\em orgyiai}, the area is said to be
$\frac{16+20}{2} \cdot \frac{25+25}{2} = 18 \cdot 25 = 450$ square {\em orgyiai}, which is 
$2 \frac{1}{4}$ {\em modioi};
an {\em orgyia} is six feet, and a {\em modios} is an area equal to 200 square {\em orgyiai}. 
In fact the area is $54 \sqrt{69}$ square {\em orgyiai}, and $448<54 \sqrt{69}<449$. 


