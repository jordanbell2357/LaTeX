\documentclass{amsart}
\usepackage{amsmath,amssymb,graphicx,subfig,mathrsfs,amsthm,enumitem}
%\usepackage[T1]{fontenc}
\newtheorem{theorem}{Theorem}
\newtheorem{lemma}[theorem]{Lemma}
\newtheorem{proposition}[theorem]{Proposition}
\newtheorem{corollary}[theorem]{Corollary}
\theoremstyle{definition}
\newtheorem{definition}[theorem]{Definition}
\newtheorem{example}[theorem]{Example}
\begin{document}
\title[Swineshead's {\em Liber calculationum}]{An infinite series in Richard Swineshead's {\em Liber calculationum}}
\author{Jordan Bell}
\email{jordan.bell@gmail.com}
\address{Department of Mathematics, University of Toronto, Toronto, Ontario, Canada}
\author{Viktor Bl{\aa}sj{\"o}}
\email{V.N.E.Blasjo@uu.nl}
\address{Mathematisch Instituut, Universiteit Utrecht, Utrecht, The Netherlands}
\date{\today}

\maketitle

Aristotle, {\em Categories} 6, 5a38 \cite[p.~12]{ackrill}:

\begin{quote}
Only these we have mentioned are called quantities
strictly, all the others derivatively; for it is to these we
look when we call the others quantities. For example, we
speak of a large amount of white because the {\em surface} is
large, and an action or a change is called long because
the {\em time} is long. For it is not in its own right that each of
these others is called a quantity. For example, if one is to
say how long an action is, one will determine this by the
time, saying that it is a-year-long or something of that sort;
and in saying how white white one will determine it by
the surface--whatever the size of the surface one will say
that the white too is that size. 
\end{quote}

Porphyry, {\em in Cat.} 105 \cite[pp.~102--103]{strange}:

\begin{quote}
Q. What is quantity accidentally?

A. What is said to be a quantity in virtue of something else,
as white is said to be large in virtue of its surface and a man is
said to be tall insofar as his size is large. Again, we might say
that a fever is great if it lasts for a long time -- since if someone were to use `great' not of a protracted fever but of a intense
one, he would not be using this expression in the strict sense:
he would be speaking about a qualification rather than about
a quantity. For `intense' is a characteristic of quality. And if
we were to say that such-and-such a person has done a great
deal of running, we would be reckoning his motion ({\em kin\^{e}sis}) by 
the large amount of time that it had taken, and it would be derivatively from this time that we would say that he had
done a great deal of running.
\end{quote}


Ammonius, {\em in Cat.} 60--61 \cite[p.~71]{ammonius}, on 5b1:

\begin{quote}
The scientist's task is not only to investigate things he himself
proposes, but also to go through in detail and refute those that
seem to be so but in truth are not. Now white may seem to be a
quantity. For we speak of white as more and less, which is
characteristic of quantity; but then we also call an action long.
So Aristotle says that these are not quantities in the strict
sense, but only {\em per accidens}. For since white is in a surface,
and that can be more or less, we say that the white is more or
less. It is the same with an action; for example, a war is called
long {\em per accidens}. Thus, since a war goes on for a certain
period of time -- e.g. for ten years -- and we say that its time is
long, for this reason we say that the action, as well, is long, {\em per
accidens}. A change, too, is called large if its time is large. For
time is the measure of change. Thus, we call a revolution of
the moon a month, of the sun a year, and of the entire heaven
a day. So if someone were to ask how long an action is, the
answer is its time, e.g. ten years. It is the same with a surface;
one may say that the white is as large as the surface is. For if
we are asked how much white there is, we say two cubits if
that is how large the surface containing the white is.
Therefore, we mean that the surface, not the white itself, is
more or less. Indeed, the white in a one-cubit surface can be
whiter than that in a two-cubit surface, and then we do not
say that there is more white ({\em leukon pleon}) but that one white
is whiter than another ({\em leukon leukou mallon}).
\end{quote}

Martianus Capella, {\em The Marriage of Philology and Mercury} V (Rhetoric), 370 \cite[p.~124]{capellaII}:

\begin{quote}
[370] Quality accepts the idea of a more and
a less, but not in every instance. For nothing square is more square
than any other square thing; but something can be said to be more
white than another white thing. And it is a question often discussed,
whether one person may be said to be more just than another.
But there appear to be many who have given careful thought to the
question and say that the qualities themselves cannot accept the idea
of a more and a less, but only the items named after them. For instance,
justice is in itself one single perfect concept, so that we
cannot say {\em This is more justice than that}, but we can say {\em This man
is more just than that}. Similarly, we cannot say {\em This is more health
than that}, but we can say {\em This man is more healthy than that}. So it
happens that substance does not accept the idea of a more and a less,
but qualities can accept it through substances. 
\end{quote}

Averroes, {\em Middle Commentaries on Aristotle's Categories}, para. 40: \cite[p.~47]{butterworth}, on 5a38--5b10:

\begin{quote}
These primary genera of quantity are the ones which are truly and primarily quantity. Anything other than them ascribed to quantity is only
accidentally and secondarily said to be quantity--I mean, by means of one of these which we said were truly quantity. For example, we say that this
designated whiteness is large because it is in a large surface. Similarly, we say that the task is long because of it taking a long time. This becomes
evident when someone asks ``how extensive is this task,'' for the answer to that would be ``it is a year-long task.'' And if he were to ask ``how long is this white thing,'' it would be said ``three or four cubits long.'' So the task is limited and measured in terms of time, and the white thing is measured in terms of the scope of the plane--which is three or four cubits long.
If they were quantities essentially, they would be meas?ured in terms of themselves.
\end{quote}


Aristotle, {\em Metaphysics} V.13, 1020a \cite{metaphysica}:

\begin{quote}
`Quantum' means that which is divisible into two or more constituent parts of which each is by nature a `one'
and a `this' . A quantum is a plurality if it is numerable,
a magnitude if it is measurable. `Plurality' means that which is divisible potentially into non-continuous parts,
`magnitude' that which is divisible into continuous parts; of magnitude, that which is continuous in one dimension is length, in two breadth, in three depth. Of these, limited plurality is number, limited length is a line, breadth a surface, depth a solid.

Again, some things are called quanta in virtue of their
own nature, others incidentally; e.g. the line is a quantum by its own nature, the musical is one incidentally. Of the things that are quanta by their own nature some are so as 
substances, e.g. the line is a quantum (for a `certain kind of quantum' is present in the definition which states what it is), and others are modifications and states of this kind of substance, e.g. much and little, long and short, broad and 
narrow, deep and shallow, heavy and light, and all other such attributes. And also great and small, and greater and smaller, both in themselves and when taken relatively to each other, are by their own nature attributes of what is quantitative; but these names are transferred to other things also. Of things that are quanta incidentally, some are so called in
the sense in which it was said that the musical and the white were quanta, viz. because that to which musicalness and whiteness belong is a quantum, and some are quanta in the way in which movement and time are so; for these also are
called quanta of a sort and continuous because the things of which these are attributes are divisible. I mean not that which is moved, but the space through which it is moved; for because that is a quantum movement also is a quantum, and because this is a quantum time is one.
\end{quote}

In his discussion of chapter V, ``De maximo et minimo'', of  Heytesbury's {\em Regule solvendi sophismata},
Wilson \cite[p.~83]{wilson} explains:

\begin{quote}
If the latitude is difformly difform in such manner that each part of it is uniform, then each uniform part of the latitude contributes to the intensity of the whole
in proportion to the amount of the subject (a body or length of time) over which it extends; for example, if the whiteness which extends over one-third of a body
has an intensity of 2 degrees, and the whiteness of the remainder an intensity of 6 degrees, the intensity of the whole latitude is $2(1/3)+6(2/3)=\frac{14}{3}$, or four
and two-thirds.
\end{quote}

Al-Biruni, specific weight

\bibliographystyle{amsplain}
\bibliography{latitude}

\end{document}