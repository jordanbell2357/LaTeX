\documentclass{article}
\usepackage{amsmath,amssymb,mathrsfs,amsthm}
%\usepackage{tikz-cd}
%\usepackage{hyperref}
\newcommand{\inner}[2]{\left\langle #1, #2 \right\rangle}
\newcommand{\tr}{\ensuremath\mathrm{tr}\,} 
\newcommand{\Span}{\ensuremath\mathrm{span}} 
\def\Re{\ensuremath{\mathrm{Re}}\,}
\def\Im{\ensuremath{\mathrm{Im}}\,}
\newcommand{\id}{\ensuremath\mathrm{id}} 
\newcommand{\var}{\ensuremath\mathrm{var}} 
\newcommand{\Lip}{\ensuremath\mathrm{Lip}} 
\newcommand{\Sh}{\ensuremath\mathrm{Sh}} 
\newcommand{\GL}{\ensuremath\mathrm{GL}} 
\newcommand{\diam}{\ensuremath\mathrm{diam}} 
\newcommand{\sgn}{\ensuremath\mathrm{sgn}} 
\newcommand{\lcm}{\ensuremath\mathrm{lcm}} 
\newcommand{\supp}{\ensuremath\mathrm{supp}\,}
\newcommand{\dom}{\ensuremath\mathrm{dom}\,}
\newcommand{\upto}{\nearrow}
\newcommand{\downto}{\searrow}
\newcommand{\norm}[1]{\left\Vert #1 \right\Vert}
\newtheorem{theorem}{Theorem}
\newtheorem{lemma}[theorem]{Lemma}
\newtheorem{proposition}[theorem]{Proposition}
\newtheorem{corollary}[theorem]{Corollary}
\theoremstyle{definition}
\newtheorem{definition}[theorem]{Definition}
\newtheorem{example}[theorem]{Example}
\begin{document}
\title{L\'evy's inequality, Rademacher sums,  and Kahane's inequality}
\author{Jordan Bell\\ \texttt{jordan.bell@gmail.com}\\Department of Mathematics, University of Toronto}
\date{\today}

\maketitle

\section{L\'evy's inequality}
Let $(\Omega,\mathscr{A},P)$ be a probability space.
A \textbf{random variable} is a Borel measurable function $\Omega \to \mathbb{R}$. 
For a random variable $X$, we denote by 
$X_*P$ the pushforward measure of $P$ by $X$. $X_*P$ is a Borel probability measure on $\mathbb{R}$, called
the \textbf{distribution of $X$}.
A random variable $X$ is called \textbf{symmetric} when
the distribution of $X$ is equal to the distribution of $-X$.
Because the collection $\{(-\infty,a]: a \in \mathbb{R}\}$ generates the Borel $\sigma$-algebra of $\mathbb{R}$,
the statement that $X_*P=(-X)_*P$ is equivalent to the statement that for all $a \in \mathbb{R}$,
\[
P(\{\omega \in \Omega: X(\omega) \leq a\})=P(\omega \in \Omega: -X(\omega) \leq a\}).
\]

The following is \textbf{L\'evy's inequality}.\footnote{Joe Diestel, Hans Jarchow, and Andrew Tonge, {\em Absolutely Summing Operators},
p.~213, Theorem 11.3.}

\begin{theorem}[L\'evy's inequality]
Suppose that $\chi_k$, $k \geq 1$, are independent symmetric random variables, that $U$ is a real or complex Banach space, and that
$u_k \in U$, $k \geq 1$. Then for each $a>0$ and for each $n \geq 1$,
\[
P\left(\max_{1 \leq k \leq n} \norm{\sum_{1 \leq j \leq k} \chi_j u_j} \geq a\right) \leq 2\cdot P\left(\norm{\sum_{1 \leq j \leq n} \chi_j u_j}
\geq a\right).
\]
\label{levy}
\end{theorem}
\begin{proof}
Let $S_0=0$ and for $1 \leq k \leq n$,
\[
S_k(\omega) = \sum_{j=1}^k \chi_j(\omega) u_j, \qquad \omega \in \Omega.
\]
For $1 \leq k \leq n$, the function $\omega \mapsto (\chi_1(\omega),\ldots,\chi_k(\omega))$ is Borel measurable
$\Omega \to \mathbb{R}^k$.\footnote{Charalambos D. Aliprantis
and Kim C. Border, {\em Infinite Dimensional Analysis: A Hitchhiker�s Guide}, third ed., p.~152, Lemma 4.49.}
The function $(t_1,\ldots,t_k) \mapsto \sum_{j=1}^k t_j u_j$ is continuous $\mathbb{R}^k \to U$. And the function
$u \mapsto \norm{u}$ is continuous $U \to \mathbb{R}$. Therefore $\omega \mapsto 
\norm{S_k(\omega)}$, the composition of these functions, is Borel measurable $\Omega \to \mathbb{R}$. 
This then implies that $\omega \mapsto \max_{1 \leq k \leq n} \norm{S_k(\omega)}$ is Borel measurable $\Omega \to \mathbb{R}$. 
Let 
\[
A = \{\omega \in \Omega: \max_{1 \leq k \leq n} \norm{S_k(\omega)} \geq a\},
\quad B = \{\omega \in \Omega: \norm{S_n(\omega)} \geq a\},
\]
for which $B \subset A$.
For $1 \leq k \leq n$, let
\[
A_k = \bigcap_{0 \leq j < k} \{\omega \in \Omega: \textrm{$\norm{S_j(\omega)}<a$ and $\norm{S_k(\omega)} \geq a$}\}.
\]
It is apparent that 
that
$A_1,\ldots,A_n$ are pairwise disjoint and that $A = \bigcup_{k=1}^n A_k$. 

For $1 \leq k \leq n$, let
\[
T_{n,k}(\omega) = S_k(\omega) - \sum_{j=k+1}^n \chi_j(\omega) u_j
=\sum_{j=1}^k \chi_j(\omega) u_j - \sum_{j=k+1}^n \chi_j(\omega)u_j, \qquad \omega \in \Omega,
\]
in other words, $S_n+T_{n,k}=2S_k$. 
Let 
\[
U_k  = A_k \cap B,\qquad 
V_k = A_k \cap \{\omega \in \Omega: \norm{T_{n,k}(\omega)} \geq a\}.
\]
If $\omega \in A_k$, then 
\[
\norm{S_n(\omega)+T_{n,k}(\omega)} = 2\norm{S_k(\omega)}
\geq 2a,
\]
which implies that at least one of the inequalities $\norm{S_n(\omega)} \geq a$ or $\norm{T_{n,k}(\omega)} \geq a$ is true. Therefore
\[
A_k = U_k \cup V_k.
\]

Because $\chi_1,\ldots,\chi_n$  are independent,
the random vector $X=(\chi_1,\ldots,\chi_n):\Omega \to \mathbb{R}^n$ has the pushforward measure
\[	
X_*P = {\chi_1}_*P \times \cdots  \times {\chi_n}_*P,
\]
 and for each $1 \leq k \leq n$, the random vector
$X_k=(\chi_1,\ldots,\chi_k,-\chi_{k+1},\ldots,-\chi_n):\Omega \to \mathbb{R}^n$ has the pushforward measure
\[
{X_k}_*P={\chi_1}_*P \times \cdots {\chi_k}_*P \times {(-\chi_{k+1})}_*P \times \cdots {(-\chi_n)}_*P,
\]
and because each $\chi_j$ is symmetric, these pushforward measures are equal. 
Define $\sigma_k:\mathbb{R}^k \to \mathbb{R}$ by
\[
\sigma_k(t_1,\ldots,t_k) = \norm{\sum_{j=1}^k t_j u_j}, \qquad (t_1,\ldots,t_k) \in \mathbb{R}^k,
\]
define $\sigma_0=0$,
and set
\begin{align*}
H_k& = \left( \bigcap_{0 \leq j < k} \{(t_1,\ldots,t_n) \in \mathbb{R}^n: \sigma_j(t_1,\ldots,t_j)<a\} \right)\\
& \cap \{(t_1,\ldots,t_n) \in \mathbb{R}^n:
\sigma_k(t_1,\ldots,t_k) \geq a, \sigma_n(t_1,\ldots,t_n) \geq a\}.
\end{align*}
Because each $\sigma_j$ is continuous, $H_k$ is a Borel set in $\mathbb{R}^n$.
Then we have
\begin{align*}
P(U_k)&=P(A_k \cap B)\\
&=P(X^{-1}(H_k))\\
&=(X_*P)(H_k)\\
&=({X_k}_*P)(H_k)\\
&=P(X_k^{-1}(H_k))\\
&=P(A_k \cap \{\omega \in \Omega: \norm{T_{n,k}(\omega)} \geq a\})\\
&=P(V_k);
\end{align*}
among the above equalities, the two equalities that deserve chewing on are
\[
P(A_k \cap B) = P(X^{-1}(H_k)) \quad \textrm{and} \quad P(X_k^{-1}(H_k)) = P(A_k \cap \{\omega \in \Omega: \norm{T_{n,k}(\omega)} \geq a\}).
\]

Thus we have
\[
P(A_k) = P(U_k \cup V_k) \leq P(U_k)+P(V_k) = 2P(U_k) = 2P(A_k \cap B).
\]
Therefore
\begin{align*}
P(A)&=\sum_{k=1}^n P(A_k)\\
&\leq \sum_{k=1}^n 2P(A_k \cap B)\\
&=2 P(A \cap B)\\
&= 2P(B),
\end{align*}
proving the claim.
\end{proof}




\section{Rademacher sums}
Suppose that 
$\epsilon_n:(\Omega,\mathscr{A},P) \to (\mathbb{R},\mathscr{B}_{\mathbb{R}},\lambda)$, $n \geq 1$, are independent random variables each with the \textbf{Rademacher distribution}: for each $n$,
\[
{\epsilon_n}_*P = \frac{1}{2}\delta_{-1}+\frac{1}{2}\delta_1,
\]
in other words, $P(\epsilon_n=1)=\frac{1}{2}$ and $P(\epsilon_n=-1)=\frac{1}{2}$. 


We now use L\'evy's inequality
to prove the following for independent random variables with the Rademacher
distribution.\footnote{Joe Diestel, Hans Jarchow, and Andrew Tonge, {\em Absolutely Summing Operators},
p.~214, Lemma 11.4.}


\begin{theorem}
Suppose that 
$X$ is a real or complex Banach space, and that
$x_k \in X$, $k \geq 1$.
Then for each $a>0$ and for each
$n \geq 1$,
\[
P\left( \norm{\sum_{k=1}^n \epsilon_k x_k} \geq 2a \right)
\leq 4\left(P\left( \norm{\sum_{k=1}^n \epsilon_k x_k} \geq a \right)\right)^2.
\]
\label{rademacher}
\end{theorem}
\begin{proof}
Let $S_0=0$ and for $1 \leq k \leq n$, define 
\[
S_k(\omega) = \sum_{1 \leq j \leq k} \epsilon_j(\omega) x_j, \qquad \omega \in \Omega.
\]
Let
\[
A = \left\{ \max_{1 \leq k \leq n} \norm{S_k} \geq a\right\},
\quad B = \{\norm{S_n} \geq a\}, \quad C=\{\norm{S_n} \geq 2a\}.
\]
L\'evy's inequality tells us that $P(A) \leq 2P(B)$.

For $1 \leq k \leq n$, let
\[
A_k = \bigcap_{0 \leq j < k} \{\norm{S_j} < a\} \cap \{\norm{S_k} \geq a\}
\]
and
\[
C_k = \{\norm{S_n-S_{k-1}} \geq a\}.
\]
If $\omega \in A_k \cap C$, then
\[
\norm{S_n(\omega)-S_{k-1}(\omega)} \geq \norm{S_n(\omega)} - \norm{S_{k-1}(\omega)}
\geq 2a - a = a,
\]
hence $A_k \cap C \subset C_k$. 
Then because $C \subset A$ and because $A$ is the disjoint union of $A_1,\ldots,A_n$,
\[
P(C)=P(A \cap C)
=P\left( \bigcup_{k=1}^n (A_k \cap C) \right)
=\sum_{k=1}^n P(A_k \cap C)
\leq \sum_{k=1}^n P(A_k \cap C_k).
\]

Let $1 \leq k \leq n$. $P(\epsilon_k^2=1)=1$, so for almost all $\omega \in \Omega$,
\[
\norm{\sum_{j=k}^n \epsilon_j(\omega) x_j} =
\norm{\epsilon_k(\omega) \sum_{j=k}^n \epsilon_j(\omega) x_j}
=\norm{x_k+\sum_{j=k+1}^n \epsilon_k(\omega) \epsilon_j(\omega) x_j}.
\]
Thus, for
\[
D_k = \left\{\norm{x_k+\sum_{j=k+1}^n \epsilon_k \epsilon_j x_j} \geq a\right\},
\]
we have
\[
P(C_k \triangle D_k)=0.
\]
Let $(\delta_1,\ldots,\delta_n) \in \{+1,-1\}^n$. On the one hand,
because $\delta_j^2=1$ and using that $\epsilon_1,\ldots,\epsilon_n$ are independent,
\[
\begin{split}
&P(\epsilon_1=\delta_1,\ldots,\epsilon_k=\delta_k,\epsilon_k
\epsilon_{k+1}=\delta_{k+1},\ldots,
\epsilon_k \epsilon_n=\delta_n)\\
=&P(\epsilon_1=\delta_1,\ldots,\epsilon_k=\delta_k,\epsilon_{k+1}=\delta_k
\delta_{k+1},\ldots,\epsilon_n=\delta_k \delta_n)\\
=&P(\epsilon_1=\delta_1)\cdots P(\epsilon_k=\delta_k)
P(\epsilon_{k+1}=\delta_k \delta_{k+1})
\cdots P(\epsilon_n=\delta_k \delta_n)\\
=&2^{-n}.
\end{split}
\]
On the other hand, 
for $k+1 \leq j \leq n$ we have
\[
\begin{split}
&P(\epsilon_k \epsilon_j = \delta_j)\\
=&P(\epsilon_k \epsilon_j = \delta_j | \epsilon_k=1)P(\epsilon_k=1)
+P(\epsilon_k \epsilon_j = \delta_j | \epsilon_k=-1) P(\epsilon_k=-1)\\
=&\frac{1}{2}P(\epsilon_j=\delta_j)+\frac{1}{2}P(\epsilon_j=-\delta_j)\\
=&\frac{1}{2}\cdot \frac{1}{2}+\frac{1}{2}\cdot \frac{1}{2}\\
=&\frac{1}{2},
\end{split}
\]
and hence
\[
P(\epsilon_1=\delta_1)\cdots P(\epsilon_k=\delta_k)
P(\epsilon_k \epsilon_{k+1}=\delta_{k+1})
\cdots P(\epsilon_k \epsilon_n = \delta_n)
=2^{-n}.
\]
Therefore, for each $1 \leq k \leq n$ and for each $(\delta_1,\ldots,\delta_n) \in \{+1,-1\}^n$,
\[
\begin{split}
&P(\epsilon_1=\delta_1,\ldots,\epsilon_k=\delta_k,\epsilon_k
\epsilon_{k+1}=\delta_{k+1},\ldots,
\epsilon_k \epsilon_n=\delta_n)\\
=&P(\epsilon_1=\delta_1)\cdots P(\epsilon_k=\delta_k)
P(\epsilon_k \epsilon_{k+1}=\delta_{k+1})
\cdots P(\epsilon_k \epsilon_n = \delta_n).
\end{split}
\]
But for almost all $\omega \in \Omega$,
\[
(\epsilon_1(\omega),\ldots,\epsilon_k(\omega),\epsilon_k(\omega) \epsilon_{k+1}(\omega),
\ldots,\epsilon_k(\omega) \epsilon_n(\omega)) \in \{+1,-1\}^n,
\]
so it follows that
\[
\epsilon_1,\ldots,\epsilon_k,
\epsilon_k \epsilon_{k+1},
\ldots,\epsilon_k \epsilon_n
\]
are independent random variables.
We check that $A_k \in \sigma(\epsilon_1,\ldots,\epsilon_k)$ and
$D_k \in \sigma(\sigma_k \sigma_{k+1},\ldots,\sigma_k \sigma_n)$, and what we have just established
means that these  $\sigma$-algebras are
independent, so 
\[
P(A_k \cap D_k) = P(A_k) P(D_k).
\]
But
\[
A_k \cap (C_k \triangle D_k) = (A_k \cap C_k) \triangle (A_k \cap D_k),
\]
so, because $P(C_k \triangle D_k)=0$,
\[
P(A_k \cap C_k) = P(A_k \cap D_k) = P(A_k) P(D_k) = P(A_k) P(C_k).
\]
We had already established that $P(C) \leq \sum_{k=1}^n P(A_k \cap C_k)$. Using this with the above, and the fact that $A$ is the disjoint
union of $A_1,\ldots,A_n$, we obtain
\begin{align*}
P(C)&\leq \sum_{k=1}^n P(A_k \cap C_k)\\
&=\sum_{k=1}^n P(A_k) P(C_k)\\
&\leq \left(\sum_{k=1}^n P(A_k) \right) \max_{1 \leq k \leq n} P(C_k)\\
&=P\left(\bigcup_{k=1}^n A_k \right)  \max_{1 \leq k \leq n} P(C_k)\\
&=P(A)  \max_{1 \leq k \leq n} P(C_k).
\end{align*}
As we stated before, we have from L\'evy's inequality that $P(A) \leq 2P(B)$, with which
\[
P(C) \leq 2P(B) \max_{1 \leq k \leq n} P(C_k).
\]
To prove the claim it thus suffices to show that
\[
\max_{1 \leq k \leq n} P(C_k) \leq 2P(B).
\]

Let $1 \leq k \leq n$. For $\delta=(\delta_1,\ldots,\delta_{k-1}) \in \{+1,-1\}^{k-1}$, let
let $H_{k,\delta,+}$ be those
$(t_1,\ldots,t_n) \in \mathbb{R}^n$ such that (i) for each $1 \leq j \leq k-1$, $t_j=\delta_j$, (ii)
$\norm{\sum_{j=k}^n t_j x_j} \geq a$,
and (iii) 
\[
\norm{\sum_{j=1}^n t_j x_j} \geq a,
\]
and let $H_{k,\delta,-}$ be those $(t_1,\ldots,t_n) \in \mathbb{R}^n$ satisfying (i) and (ii) and 
\[
\norm{\sum_{j=1}^{k-1} t_j x_j - \sum_{j=k}^n t_j x_j} \geq a.
\]
Let
\[
X=(\epsilon_1,\ldots,\epsilon_n):\Omega \to \mathbb{R}^n
\] 
and let
\[
X_k=(\epsilon_1,\ldots,\epsilon_{k-1},-\epsilon_k,\ldots,-\epsilon_n):\Omega \to \mathbb{R}^n,
\]
which have the same distribution because $\epsilon_1,\ldots,\epsilon_n$ are independent and symmetric. 
Then
\begin{align*}
P(X^{-1}(H_{k,\delta,+}))&=(X_*P)(H_{k,\delta,+})\\
&=({X_k}_*P)(H_{k,\delta,+})\\
&=P(X_k^{-1}(H_{k,\delta,+}))\\
&=P(X^{-1}(H_{k,\delta,-})).
\end{align*}
Set
\[
C_{k,\delta,+} = \{X \in H_{k,\delta,+}\}, \qquad C_{k,\delta,-} = \{X \in H_{k,\delta,-}\},
\]
for which we thus have
\[
P(C_{k,\delta,+})=P(C_{k,\delta,-}).
\]
We can write $C_{k,\delta,+}$ and $C_{k,\delta,-}$ as
\[
C_{k,\delta,+} = \left( \bigcap_{0 \leq j < k} \{\epsilon_j=\delta_j\}\right) 
\cap C_k 
\cap \{\norm{S_n} \geq a\}
\]
and
\[
C_{k,\delta,-} = \left( \bigcap_{0 \leq j < k} \{\epsilon_j=\delta_j\}\right) 
\cap C_k 
\cap \{\norm{S_n-2S_{k-1}} \geq a\}.
\]
If $\omega \in C_k$ then, because $\norm{S_n(\omega)-S_{k-1}(\omega)} \geq a$, 
\begin{align*}
2a &\leq 2\norm{S_n(\omega)-S_{k-1}(\omega)}\\
&=\norm{S_n(\omega) + (S_n(\omega)-2S_{k-1}(\omega))}\\
&\leq \norm{S_n(\omega)}+\norm{S_n(\omega)-2S_{k-1}(\omega)},
\end{align*}
so at least one of the inequalities $\norm{S_n(\omega)} \geq a$ and $\norm{S_n(\omega)-2S_{k-1}(\omega)} \geq a$ is true, and hence
\[
C_k \subset \{\norm{S_n} \geq a\} \cup \{\norm{S_n-2S_{k-1}} \geq a\}.
\]
It follows that
\[
C_k \cap  \left( \bigcap_{0 \leq j < k} \{\epsilon_j=\delta_j\}\right)  = C_{k,\delta,+}
\cup C_{k,\delta,-}.
\]
Therefore,
using the fact that for almost all $\omega \in \Omega$,
\[
(\epsilon_1(\omega),\ldots,\epsilon_{k-1}(\omega)) \in \{+1,-1\}^{k-1},
\]
and
\[
C_{k,\delta,+} = \left(\bigcap_{0 \leq j < k} \{\epsilon_j=\delta_j\}\right)
\cap C_k \cap B,
\]
we get
\begin{align*}
P(C_k)&=\sum_\delta P\left(C_k \cap \bigcap_{0 \leq j < k} \{\epsilon_j=\delta_j\}\right)\\
&=\sum_\delta P(C_{k,\delta,+} \cup C_{k,\delta,-})\\
&=2\sum_\delta P(C_{k,\delta,+})\\
&\leq 2\sum_\delta P\left( B \cap \bigcap_{0 \leq j< k} \{\epsilon_j=\delta_j\}\right)\\
&=2P(B),
\end{align*}
and thus
\[
\max_{1 \leq k \leq n} P(C_k) \leq 2 P(B),
\]
which proves the claim.
\end{proof}







\section{Kahane's inequality}
By $E(X)^r$ we mean $(E(X))^r$. 
The following is \textbf{Kahane's inequality}.\footnote{Joe Diestel, Hans Jarchow, and Andrew Tonge, {\em Absolutely Summing Operators},
p.~211, Theorem 11.1.}

\begin{theorem}[Kahane's inequality]
For   $0<p,q<\infty$, there is some $K_{p,q}>0$ such that if $X$ is a real or complex Banach
space and $x_k \in X$, $k \geq 1$, then for each $n$,
\[
E\left(\norm{\sum_{k=1}^n \epsilon_k x_k}^q\right)^{1/q}
\leq K_{p,q}   \cdot E\left(\norm{\sum_{k=1}^n \epsilon_k x_k}^p\right)^{1/p}.
\]
\end{theorem}
\begin{proof}
Suppose that $0<p<q<\infty$; when $p \geq q$ the claim is immediate with $K_{p,q}=1$. Let
\[
M = E\left(\norm{\sum_{k=1}^n \epsilon_k x_k}^p\right)^{1/p};
\]
if $M=0$ we check that the claim is $0 \leq K_{p,q}\cdot 0$, which is true for, say, $K_{p,q}=1$. Otherwise, $M>0$,
and
let $u_k=\frac{x_k}{M}$, $1 \leq k \leq n$, for which
\begin{equation}
E\left(\norm{\sum_{k=1}^n \epsilon_k u_k}^p \right) = 
E\left(\norm{\sum_{k=1}^n \epsilon_k \frac{x_k}{M}}^p \right) 
= 1.
\label{normalized}
\end{equation}
Using Chebyshev's inequality,
\[
P\left( \norm{\sum_{k=1}^n \epsilon_k u_k} \geq 8^{1/p} \right)
=P\left( \norm{\sum_{k=1}^n \epsilon_k u_k}^p \geq 8 \right)
\leq \frac{1}{8} E\left(\norm{\sum_{k=1}^n \epsilon_k u_k}^p\right)
= \frac{1}{8}.
\]
Assume for induction that for some $l \geq 0$ we have 
\begin{equation}
P\left( \norm{\sum_{k=1}^n \epsilon_k u_k} \geq 2^l \cdot 8^{1/p} \right) 
\leq \frac{1}{4} \cdot 2^{-2^l};
\label{assumption}
\end{equation}
the above shows that this is true for $l=0$. Applying Theorem \ref{rademacher} and then \eqref{assumption},
\[
P\left( \norm{\sum_{k=1}^n \epsilon_k u_k} \geq 2^{l+1} \cdot 8^{1/p} \right) 
\leq 4\left(P\left( \norm{\sum_{k=1}^n \epsilon_k u_k} \geq 2^l \cdot 8^{1/p} \right)\right)^2
\leq \frac{1}{4} \cdot 2^{-2^{l+1}},
\]
which shows that \eqref{assumption} is true for all $l \geq 0$. 

Generally, for $0<q<\infty$, if $X:\Omega \to \mathbb{R}$ is a random variable for which
$P(X \geq 0)=1$, then
\[
E(X^q) = \int_0^\infty q s^{q-1} P(X \geq s) ds;
\]
the right-hand side is finite if and only if $X \in L^q(P)$. 
Using this,
\begin{equation}
E\left(\norm{\sum_{k=1}^n \epsilon_k u_k}^q\right) = 
\int_0^\infty qs^{q-1} P\left(\norm{\sum_{k=1}^n \epsilon_k u_k} \geq s\right) ds.
\label{parts}
\end{equation}
Let $\alpha_0=$ and for $l \geq 1$ let $\alpha_l=2^{l-1} \cdot 8^{1/p}$, and define
\[
f(s) = q s^{q-1} P\left(\norm{\sum_{k=1}^n \epsilon_k u_k} \geq s\right), \qquad s \geq 0.
\]
Using \eqref{parts} and then \eqref{assumption},
\begin{align*}
E\left(\norm{\sum_{k=1}^n \epsilon_k u_k}^q\right)&=\int_0^\infty f(s) ds\\
&=\int_0^{\alpha_1} f(s) ds
+
\sum_{l=0}^\infty 
\int_{\alpha_{l+1}}^{\alpha_{l+2}} f(s) ds\\
&\leq \int_0^{\alpha_1} qs^{q-1} ds + \sum_{l=0}^\infty \int_{\alpha_{l+1}}^{\alpha_{l+2}} qs^{q-1} P\left(\norm{\sum_{k=1}^n \epsilon_k u_k} \geq 
\alpha_{l+1}\right) ds\\
&\leq \alpha_1^q  +  \sum_{l=0}^\infty \int_{\alpha_{l+1}}^{\alpha_{l+2}} qs^{q-1} \frac{1}{4} \cdot 2^{-2^l} ds\\
&=8^{q/p} + \frac{1}{4} \sum_{l=0}^\infty 2^{-2^l} (\alpha_{l+2}^q-\alpha_{l+1}^q),
\end{align*}
and we define $K_{p,q}$ by taking $K_{p,q}^q$ to be equal to the above.
Thus
\[
E\left(\norm{\sum_{k=1}^n \epsilon_k u_k}^q\right)^{1/q} \leq K_{p,q},
\]
and therefore, by \eqref{normalized},
\[
E\left(\norm{\sum_{k=1}^n \epsilon_k u_k}^q\right)^{1/q} \leq K_{p,q} \cdot
E\left(\norm{\sum_{k=1}^n \epsilon_k u_k}^p \right)^{1/p}.
\]
Finally, as $u_k = \frac{x_k}{M}$,
\[
E\left(\norm{\sum_{k=1}^n \epsilon_k x_k}^q\right)^{1/q} \leq K_{p,q} \cdot
E\left(\norm{\sum_{k=1}^n \epsilon_k x_k}^p \right)^{1/p},
\]
which proves the claim.
\end{proof}

In the above proof of Kahane's inequality, for $p=1$ and $q=2$ we have
\begin{align*}
K_{1,2}^2 &= 8^2 + \frac{1}{4} \sum_{l=0}^\infty 2^{-2^l}(\alpha_{l+2}^2-\alpha_{l+1}^2)\\
&=64+16\sum_{l=0}^\infty 2^{-2^l} (2^{2l+2}-2^{2l})\\
&=64+48\sum_{l=0}^\infty 2^{-2^l} 2^{2l},
\end{align*}
for which 
\[
K_{1,2} = 14.006\ldots.
\]
In fact, the inequality is true with $K_{1,2}=\sqrt{2}=1.414\ldots$.\footnote{R. Lata\l{}a and 
K. Oleszkiewicz, 
{\em On the best constant in the Khinchin-Kahane inequality}, Studia Math. \textbf{109} (1994), no. 1,
101--104.}




\end{document}