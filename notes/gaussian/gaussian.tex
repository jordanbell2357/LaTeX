\documentclass{article}
\usepackage{amsmath,amssymb,mathrsfs,amsthm}
%\usepackage{tikz-cd}
%\usepackage{hyperref}
\newcommand{\inner}[2]{\left\langle #1, #2 \right\rangle}
\newcommand{\tr}{\ensuremath\mathrm{tr}\,} 
\newcommand{\Span}{\ensuremath\mathrm{span}} 
\def\Re{\ensuremath{\mathrm{Re}}\,}
\def\Im{\ensuremath{\mathrm{Im}}\,}
\newcommand{\id}{\ensuremath\mathrm{id}} 
\newcommand{\var}{\ensuremath\mathrm{var}} 
\newcommand{\Lip}{\ensuremath\mathrm{Lip}} 
\newcommand{\GL}{\ensuremath\mathrm{GL}} 
\newcommand{\diam}{\ensuremath\mathrm{diam}} 
\newcommand{\sgn}{\ensuremath\mathrm{sgn}\,} 
\newcommand{\lcm}{\ensuremath\mathrm{lcm}} 
\newcommand{\supp}{\ensuremath\mathrm{supp}\,}
\newcommand{\dom}{\ensuremath\mathrm{dom}\,}
\newcommand{\upto}{\nearrow}
\newcommand{\downto}{\searrow}
\newcommand{\norm}[1]{\left\Vert #1 \right\Vert}
\newtheorem{theorem}{Theorem}
\newtheorem{lemma}[theorem]{Lemma}
\newtheorem{proposition}[theorem]{Proposition}
\newtheorem{corollary}[theorem]{Corollary}
\theoremstyle{definition}
\newtheorem{definition}[theorem]{Definition}
\newtheorem{example}[theorem]{Example}
\begin{document}
\title{Gaussian measures, Hermite polynomials, and the Ornstein-Uhlenbeck semigroup}
\author{Jordan Bell\\ \texttt{jordan.bell@gmail.com}\\Department of Mathematics, University of Toronto}
\date{\today}

\maketitle

\section{Definitions}
For a topological space $X$, we denote by $\mathscr{B}_X$ the Borel $\sigma$-algebra of $X$.

We write $\overline{\mathbb{R}}  = \mathbb{R} \cup\{-\infty,\infty\}$. With the order topology, $\overline{\mathbb{R}}$ is a compact metrizable
space, and $\mathbb{R}$ has the subspace topology inherited from $\overline{\mathbb{R}}$, namely the inclusion map
is an embedding $\mathbb{R} \to \overline{\mathbb{R}}$. It follows that\footnote{Charalambos D. Aliprantis
and Kim C. Border, {\em Infinite Dimensional Analysis: A Hitchhiker�s Guide}, third ed., p.~138, Lemma 4.20.}
\[
\mathscr{B}_{\mathbb{R}} = \{E \cap \mathbb{R}: E \in \mathscr{B}_{\overline{\mathbb{R}}}\}.
\]

If $\mathscr{F}$ is a collection of functions $X \to \overline{\mathbb{R}}$ on a  set $X$, we define $\bigvee \mathscr{F}:X \to \overline{\mathbb{R}}$
and $\bigwedge \mathscr{F}:X \to \overline{\mathbb{R}}$ by
\[
\left(\bigvee \mathscr{F} \right)(x) = \sup\{f(x): f \in \mathscr{F}\}, \qquad x \in X
\]
and
\[
\left(\bigwedge \mathscr{F} \right)(x) = \inf\{f(x): f \in \mathscr{F}\}, \qquad x \in X.
\]
If $X$ is a measurable space and $\mathscr{F}$ is a countable collection of measurable functions $X \to \overline{\mathbb{R}}$,
it is a fact that $\bigwedge \mathscr{F}$ and $\bigvee \mathscr{F}$ are measurable $X \to \overline{\mathbb{R}}$. 




\section{Kolmogorov's inequality}
\textbf{Kolmogorov's inequality} is the following.\footnote{Gerald B. Folland, {\em Real Analysis: Modern Techniques and Their
Applications}, second ed., p.~322, Theorem 10.11.}

\begin{theorem}[Kolmogorov's inequality]
Suppose that $(\Omega,\mathscr{S},P)$ is a probability space, that
$X_1,\ldots,X_n \in L^2(P)$, that $E(X_1)=0,\ldots,E(X_n)=0$, and that
$X_1,\ldots,X_n$ are independent. Let
\[
S_k(\omega) = \sum_{j=1}^k X_j(\omega), \qquad \omega \in \Omega,
\]
for $1 \leq k \leq n$. Then for any $\lambda>0$,
\[
P\left(\left\{\omega \in \Omega: \bigvee_{k=1}^n |S_k(\omega)| \geq \lambda \right\}\right) \leq \frac{1}{\lambda^2} \sum_{j=1}^n V(X_j)
=\frac{1}{\lambda^2} V(S_n).
\]
\end{theorem}

\section{$\mathbb{R}$}
For real $a$ and $\sigma>0$, one computes that
\begin{equation}
\frac{1}{\sigma \sqrt{2\pi}} \int_{\mathbb{R}} \exp\left( - \frac{(t-a)^2}{2\sigma^2} \right) dt = 1.
\label{identity}
\end{equation}

Suppose that $\gamma$ is a Borel probability measure on $\mathbb{R}$. If 
\[
\gamma=\delta_a
\]
 for some
$a \in \mathbb{R}$ or has density
\[
p(t,a,\sigma^2) = \frac{1}{\sigma\sqrt{2\pi}}  \exp\left( - \frac{(t-a)^2}{2\sigma^2} \right),
\qquad t \in \mathbb{R},
\]
for some $a \in \mathbb{R}$ and some $\sigma>0$,
with respect to Lebesgue measure on $\mathbb{R}$, we say that $\gamma$ is a \textbf{Gaussian measure}.
We say that $\delta_a$ is a Gaussian measure with \textbf{mean $a$} and \textbf{variance $0$}, and
that a Gaussian measure with density $p(\cdot,a,\sigma^2)$ has \textbf{mean $a$} and \textbf{variance $\sigma^2$}.
A Gaussian measure with mean $0$ and variance $1$ is said to be \textbf{standard}.

One calculates that the \textbf{characteristic function} of a Gaussian measure $\gamma$ with density $p(\cdot,a,\sigma^2)$
is
\begin{equation}
\widetilde{\gamma}(y) = \int_{\mathbb{R}} \exp(i yx) d\gamma(x)
=\exp\left( iay - \frac{1}{2}\sigma^2 y^2 \right), \qquad y \in \mathbb{R}.
\label{characteristic}
\end{equation}

The \textbf{cumulative distribution function} of a standard Gaussian measure  $\gamma$ is, for
$t \in \mathbb{R}$,
\[
\Phi(t) = \gamma(-\infty,t] =\int_{-\infty}^t d\gamma(s) = 
\int_{-\infty}^t p(s,0,1) ds
=\int_{-\infty}^t \frac{1}{\sqrt{2\pi}} \exp\left(- \frac{s^2}{2} \right) ds.
\]
We define $\Phi(-\infty)=0$ and also define 
\[
\Phi(\infty) = \int_{-\infty}^\infty \frac{1}{\sqrt{2\pi}} \exp\left(- \frac{s^2}{2} \right) ds = 1,
\]
using \eqref{identity}.

$\Phi:\overline{\mathbb{R}} \to [0,1]$ is strictly increasing, thus 
$\Phi^{-1}:[0,1] \to \overline{\mathbb{R}}$ makes sense, and is itself strictly increasing. Then 
$1-\Phi$ is strictly decreasing.
By \eqref{identity},
\begin{align*}
1-\Phi(t) &=  \int_{-\infty}^\infty \frac{1}{\sqrt{2\pi}} \exp\left( - \frac{s^2}{2} \right) ds
-\int_{-\infty}^t \frac{1}{\sqrt{2\pi}} \exp\left(- \frac{s^2}{2} \right) ds\\
&=\int_t^\infty \frac{1}{\sqrt{2\pi}} \exp\left(- \frac{s^2}{2} \right) ds.
\end{align*}
The following lemma gives an estimate for
$1-\Phi(t)$ that tells us something substantial as $t \to +\infty$, beyond the immediate fact that $(1-\Phi)(\infty)=1-\Phi(\infty)=0$.\footnote{Vladimir I. Bogachev,
{\em Gaussian Measures}, p.~2, Lemma 1.1.3.}

\begin{lemma}
For $t>0$,
\[
\frac{1}{\sqrt{2\pi}} \left(\frac{1}{t}-\frac{1}{t^3}\right) e^{-t^2/2} \leq 1- \Phi(t) \leq \frac{1}{\sqrt{2\pi}} \frac{1}{t} e^{-t^2/2}.
\]
\end{lemma}
\begin{proof}
Integrating by parts,
\begin{align*}
1-\Phi(t) &= \int_t^\infty \frac{1}{\sqrt{2\pi}} \exp\left(- \frac{s^2}{2} \right) ds\\
&=\int_t^\infty \frac{1}{s \sqrt{2\pi}} \cdot s\exp\left(- \frac{s^2}{2} \right) ds\\
&=- \frac{1}{s\sqrt{2\pi}} \exp\left(- \frac{s^2}{2} \right) \bigg|_t^\infty
-\int_t^\infty \frac{1}{s^2 \sqrt{2\pi}}  \exp\left(- \frac{s^2}{2} \right) ds\\
&\leq \frac{1}{t\sqrt{2\pi}} \exp\left(-\frac{t^2}{2} \right).
\end{align*}
On the other hand, using the above work and again integrating by parts,
\begin{align*}
1-\Phi(t) &= \frac{1}{t\sqrt{2\pi}} \exp\left(-\frac{t^2}{2} \right)-\int_t^\infty \frac{1}{s^3 \sqrt{2\pi}} \cdot s  \exp\left(- \frac{s^2}{2} \right) ds\\
&=\frac{1}{t\sqrt{2\pi}} \exp\left(-\frac{t^2}{2} \right) +
\frac{1}{s^3 \sqrt{2\pi}} \exp\left(-\frac{s^2}{2} \right) \bigg|_t^\infty\\
&+\int_t^\infty \frac{3}{s^4\sqrt{2\pi}} \exp\left(-\frac{s^2}{2} \right) ds\\
&\geq \frac{1}{t\sqrt{2\pi}} \exp\left(-\frac{t^2}{2} \right) - \frac{1}{t^3 \sqrt{2\pi}} \exp\left(-\frac{t^2}{2} \right).
\end{align*}
\end{proof}

The following theorem shows that if the variances of a sequence of independent centered random variables
are summable then the sequence of random variables is summable almost surely.\footnote{Karl R. Stromberg, {\em Probability for Analysts}, p.~58, Theorem 4.6.} 

\begin{theorem}
Suppose that $\xi_j \in L^2(\Omega,\mathscr{S},P)$, $j \geq 1$, are independent random variables
each with mean $0$. If $\sum_{j=1}^\infty V(\xi_j)<\infty$,
then $\sum_{j=1}^\infty \xi_j$ converges almost surely.
\end{theorem}
\begin{proof}
Define $S_n:\Omega \to \mathbb{R}$ by
\[
S_n(\omega)=\sum_{j=1}^n \xi_j(\omega),
\]
 define
$Z_n:\Omega \to [0,\infty]$ by
\[
Z_n = \bigvee_{j=1}^\infty |S_{n+j}-S_n|,
\]
and define $Z:\Omega \to  [0,\infty]$ by
\[
Z = \bigwedge_{n=1}^\infty Z_n.
\]
If $S_n(\omega)$ converges and $\epsilon>0$,
there is some $n$ such that for all $j \geq 1$,
$|S_{n+j}(\omega)-S_n(\omega)|<\epsilon$ and so
$Z_n(\omega) \leq \epsilon$ and 
$Z(\omega) \leq \epsilon$. Therefore, if
$S_n(\omega)$ converges then $Z(\omega)=0$. On the other hand,
if $Z(\omega)=0$ and $\epsilon>0$, there
is some $n$ such that
$Z_n(\omega)<\epsilon$, hence 
$|S_{n+j}(\omega)-S_n(\omega)| < \epsilon$ for all $j \geq 1$.
That is, $S_n(\omega)$ is a Cauchy sequence in $\mathbb{R}$,
and hence converges. Therefore
\begin{equation}
\left\{\omega \in \Omega: \textrm{$S_n(\omega)$ converges}\right\}=
\{\omega \in \Omega: Z(\omega)=0\}.
\label{Z0}
\end{equation}
Let $\epsilon>0$. For any $n$ and $k$, using Kolmogorov's inequality with
$X_j=\xi_{n+j}$ for $j=1,\ldots,k$,
\[
P\left( \bigvee_{j=1}^k |S_{n+j}-S_n| \geq \epsilon \right)
\leq \frac{1}{\epsilon^2} \sum_{j=1}^k V(X_j)
\leq \frac{1}{\epsilon^2} \sum_{j=n+1}^\infty V(\xi_j).
\]
Because this is true for each $k$, it follows that
\[
P(Z_n \geq \epsilon) \leq \frac{1}{\epsilon^2} \sum_{j=n+1}^\infty V(\xi_j),
\]
hence, for each $n$,
\[
P(Z \geq \epsilon) \leq P(Z_n \geq \epsilon) \leq \frac{1}{\epsilon^2} \sum_{j=n+1}^\infty V(\xi_j).
\]
Because $\sum_{j=1}^\infty V(\xi_j)<\infty$, $\sum_{j=n+1}^\infty V(\xi_j) \to 0$ as $n \to \infty$, so
\[
P(Z \geq \epsilon)=0.
\]
Because this is true for all $\epsilon>0$, we get $P(Z>0)=0$, i.e. $P(Z=0)=1$. By \eqref{Z0},
this means that $S_n$ converges almost surely.
\end{proof}


The following theorem gives conditions under which the converse of the above theorem
holds.\footnote{Karl R. Stromberg, {\em Probability for Analysts}, p.~59, Theorem 4.7.}

\begin{theorem}
Suppose that $\xi_j \in L^2(\Omega,\mathscr{S},P)$, $j \geq 1$, are independent  random variables
each with mean $0$, and let
$S_n=\sum_{j=1}^n \xi_j$.
 If 
 \begin{equation}
P\left( \bigvee_{n=1}^\infty |S_n|<\infty \right)>0
\label{converges}
\end{equation}
and
 there is some $\beta \in [0,\infty)$ such that
$\bigvee_{j=1}^\infty |\xi_j| \leq \beta$ almost surely, then 
\[
\sum_{j=1}^\infty V(\xi_j)<\infty.
\]
\end{theorem}
\begin{proof}
By \eqref{converges}, there is some $\alpha \in [0,\infty)$ such that $P(A)>0$, for
\[
A = \left\{ \omega \in \Omega: \bigvee_{n=1}^\infty |S_n(\omega)| \leq \alpha \right\}.
\]
For $p \geq 1$, let
\[
A_p = \left\{ \omega \in \Omega: \bigvee_{n=1}^p |S_n(\omega)| \leq \alpha \right\},
\]
which satisfies $A_p \downarrow A$ as $p \to \infty$. For each $p$, the random variables
$\chi_{A_p} S_p$ and $\xi_{p+1}$ are independent and the random variables
$\chi_{A_p}$ and $\xi_{p+1}^2$ are independent,
whence
\begin{align*}
E(\chi_{A_p} S_{p+1}^2)&=E(\chi_{A_p} (S_p+\xi_{p+1})(S_p+\xi_{p+1}))\\
&=E(\chi_{A_p} S_p^2+2\chi_{A_p} S_p \xi_{p+1} + \chi_{A_p} \xi_{p+1}^2)\\
&=E(\chi_{A_p} S_p^2) + 2 E(\chi_{A_p} S_p) E(\xi_{p+1})+E(\chi_{A_p}) E(\xi_{p+1}^2)\\
&=E(\chi_{A_p} S_p^2) + P(A_p) V(\xi_{p+1})\\
&\geq E(\chi_{A_p} S_p^2) + P(A) V(\xi_{p+1}).
\end{align*}
Set $B_p = A_p \setminus A_{p+1}$. For $\omega \in A_p$, $|S_p(\omega)| \leq \alpha$, and for almost
all $\omega \in \Omega$, $|\xi_{p+1}(\omega)| \leq \beta$, so for almost all $\omega 
\in B_p$,
\[
|S_{p+1}(\omega)| \leq |S_p(\omega)| + |\xi_{p+1}(\omega) \leq \alpha+\beta,
\]
hence
\begin{align*}
P(A)V(\xi_{p+1})&\leq E((\chi_{B_p}+\chi_{A_{p+1}}) S_{p+1}^2) - E(\chi_{A_p} S_p^2)\\
&=E(\chi_{B_p} S_{p+1}^2)+E(\chi_{A_{p+1}}S_{p+1}^2)- E(\chi_{A_p} S_p^2)\\
&\leq P(B_p) (\alpha+\beta)^2 + E(\chi_{A_{p+1}}S_{p+1}^2)- E(\chi_{A_p} S_p^2).
\end{align*}
Adding the inequalities for $p=1,2,\ldots,n-1$, because $B_p$ are pairwise disjoint,
\begin{align*}
P(A) \sum_{p=1}^{n-1} V(\xi_{p+1})&= (\alpha+\beta)^2 \sum_{p=1}^{n-1} P(B_p) + E(\chi_{A_n} S_n^2) - 
E(\chi_{A_1} S_1^2)\\
&\leq (\alpha+\beta)^2 + E(\chi_{A_n} S_n^2)\\
&\leq (\alpha+\beta)^2 + \alpha^2.
\end{align*}
Because this is true for all $n$ and $P(A)>0$,
\[
\sum_{p=1}^\infty V(\xi_{p+1}) < \infty,
\]
and with $V(\xi_1)<\infty$ this completes the proof.
\end{proof}



\section{$\mathbb{R}^n$}
If $\mu$ is a finite Borel measure on $\mathbb{R}^n$, we define the \textbf{characteristic function of
$\mu$} by
\[
\widetilde{\mu}(y) = \int_{\mathbb{R}^n} e^{i\inner{y}{x}} d\mu(x), \qquad y \in \mathbb{R}^n.
\]

A Borel probability measure $\gamma$ on $\mathbb{R}^n$ is said to be \textbf{Gaussian} if
for each $f \in (\mathbb{R}^n)^*$, the pushforward measure
$f_* \gamma$ on $\mathbb{R}$ is a Gaussian measure on $\mathbb{R}$, where
\[
(f_* \gamma)(E)=\gamma(f^{-1}(E))
\]
 for $E$ a Borel set in
$\mathbb{R}$.

We now give a characterization of Gaussian measures on $\mathbb{R}^n$ and their densities.\footnote{Vladimir I. Bogachev,
{\em Gaussian Measures}, p.~3, Proposition 1.2.2; Michel Simonnet, {\em Measures and Probabilities},
p.~303, Theorem 14.5.} In the following theorem, the
vector $a \in \mathbb{R}^n$  is called the \textbf{mean of $\gamma$} and the 
linear transformation $K \in \mathscr{L}(\mathbb{R}^n)$ 
 is called the \textbf{covariance  operator of $\gamma$}. When $a=0 \in \mathbb{R}^n$ and $K=\id_{\mathbb{R}^n}$, we
 say that $\gamma$ is \textbf{standard}.

\begin{theorem}
A Borel probability measure $\gamma$ on $\mathbb{R}^n$ is Gaussian if and only if there is some
$a \in \mathbb{R}^n$ and some positive semidefinite  $K \in \mathscr{L}(\mathbb{R}^n)$ such that
\begin{equation}
\widetilde{\gamma}(y) = \exp\left(i \inner{y}{a} - \frac{1}{2} \inner{Ky}{y} \right), \qquad
y \in \mathbb{R}^n.
\label{GaussianRn}
\end{equation}

If $\gamma$ is a Gaussian measure whose  covariance operator $K$  is positive definite, then 
the density of $\gamma$ with respect to Lebesgue measure on $\mathbb{R}^n$ is 
\[
x \mapsto \frac{1}{\sqrt{(2\pi)^n \det K}} \exp\left(-\frac{1}{2}\inner{K^{-1}(x-a)}{x-a}\right),
\qquad x \in \mathbb{R}^n.
\]
\label{density}
\end{theorem}
\begin{proof}
Suppose  that
\eqref{GaussianRn} is satisfied.
Let $f \in (\mathbb{R}^n)^*$, i.e. a linear map $\mathbb{R}^n \to \mathbb{R}$, and put $\nu = f_* \gamma$. Using the change of variables formula,
the characteristic function of $\nu$ is
\begin{align*}
\widetilde{\nu}(t) = \int_{\mathbb{R}} e^{it s} d\nu(s) = 
\int_{\mathbb{R}^n} e^{it f(x)} d\gamma(x), \qquad t \in \mathbb{R}.
\end{align*}
Let $v$ be the unique element of $\mathbb{R}^n$ such that
$f(x) = \inner{v}{x}$ for all $x \in \mathbb{R}^n$. Then
\[
\widetilde{\nu}(t) = \int_{\mathbb{R}^n} e^{i \inner{tv}{x}} d\gamma(x) = 
\widetilde{\gamma}(tv).
\]
so by  \eqref{GaussianRn},
\[
\widetilde{\nu}(t) = \exp\left( i \inner{tv}{a} - \frac{1}{2} \inner{Ktv}{tv} \right)
=\exp\left(i f(a) t  - \frac{1}{2}  \inner{Kv}{v} t^2 \right).
\]
This implies that $\nu$ is a Gaussian measure on $\mathbb{R}$ with mean $f(a)$ and variance
$\inner{Kv}{v}$: if $\inner{Kv}{v}=0$ then 
$\nu=\delta_{f(a)}$, and if $\inner{Kv}{v}>0$ then $\nu$ has density
\[
\frac{1}{\sqrt{\inner{Kv}{v}} \sqrt{2\pi}} \exp\left(-\frac{(s-f(a))^2}{2 \inner{Kv}{v}} \right), \qquad
s \in \mathbb{R},
\]
with respect to Lebesgue measure on $\mathbb{R}$. That is, for any
$f \in (\mathbb{R}^n)^*$, the pushforward measure $f_* \gamma$ is a Gaussian measure
on $\mathbb{R}$, which is what it means for $\gamma$ to be a Gaussian measure on $\mathbb{R}^n$. 

Suppose that  $\gamma$ is Gaussian and let  $f \in (\mathbb{R}^n)^*$.  Then the pushforward
measure $f_* \gamma$ is a Gaussian measure on $\mathbb{R}$. Let
$a(f)$ be the mean of $f_* \gamma$ and let $\sigma^2(f)$ be the variance of $f_* \gamma$, and
 let $v_f$ be the unique element of $\mathbb{R}^n$ such that
$f(x) = \inner{x}{v_f}$ for all $x \in \mathbb{R}^n$. 
Using the change of variables formula,
\[
a(f) = \int_{\mathbb{R}} t d(f_* \gamma)(t) = 
\int_{\mathbb{R}^n} f(x) d\gamma(x)
\]
and
\begin{align*}
\sigma^2(f) &= \int_{\mathbb{R}} (t-a(f))^2 d(f_* \gamma)(t)\\
&=\int_{\mathbb{R}^n} (f(x)-a(f))^2 d\gamma(x)\\
&=\int_{\mathbb{R}^n} (f(x)^2-2f(x) a(f) + a(f)^2) d\gamma(x).
\end{align*}
Because $f \mapsto a(f)$ is linear $(\mathbb{R}^n)^* \to \mathbb{R}$, there is a unique
$a \in \mathbb{R}^n = (\mathbb{R}^n)^{**}$ such that
\[
a(f) = \inner{v_f}{a}, \qquad f \in (\mathbb{R}^n)^*.
\]
For $f,g \in (\mathbb{R}^n)^*$,
\begin{align*}
\sigma^2(f+g)&=\int_{\mathbb{R}^n} (f(x)^2+2f(x)g(x)+g(x)^2\\
&-2f(x)a(f)-2f(x)a(g)-2g(x)a(f)-2g(x)a(g) \\
&+a(f)^2+2a(f)a(g)+a(g)^2) d\gamma(x),
\end{align*}
so
\begin{align*}
\sigma^2(f+g)-\sigma^2(f)-\sigma^2(g)&=\int_{\mathbb{R}^n}  (2f(x)g(x)-2f(x)a(g)-2g(x)a(f)\\
&+2a(f)a(g)) d\gamma(x).
\end{align*}
 $B(f,g) = \frac{1}{2}(\sigma^2(f+g)-\sigma^2(f)-\sigma^2(g))$ is a symmetric bilinear form on $\mathbb{R}^n$, and
 \[
 B(f,f) = 2 \int_{\mathbb{R}^n} (f(x)-a(f))^2 d\gamma(x) \geq 0,
 \]
 namely, $B$ is positive semidefinite. It follows  that there is a unique positive semidefinite $K \in \mathscr{L}(\mathbb{R}^n)$ such that
 $B(f,g) = \inner{Kv_f}{v_g}$ for all $f,g \in (\mathbb{R}^n)^*$. 
 For $y \in \mathbb{R}^n$ and for $v_f=y$,  using the change of variables formula, using the fact that
 $f_* \gamma$ is a Gaussian measure on $\mathbb{R}$ with
 mean
\[
a(f) = \inner{v_f}{a} = \inner{y}{a}
\]
and variance
\[
\sigma^2(f) = B(f,f) = \inner{Kv_f}{v_f} = \inner{Ky}{y}
\]
 and  using \eqref{characteristic},
 \begin{align*}
 \widetilde{\gamma}(y)&=\int_{\mathbb{R}^n} e^{if(x)} d\gamma(x)\\
 &=\int_{\mathbb{R}} e^{it} d(f_*\gamma)(t)\\
 &=\exp\left(i\inner{y}{a}\cdot 1 - \frac{1}{2} \inner{Ky}{y}\cdot 1^2 \right)\\
 &=\exp\left(i\inner{y}{a}  - \frac{1}{2} \inner{Ky}{y} \right),
 \end{align*}
 which shows that \eqref{GaussianRn} is satisfied.
 
Suppose that $\gamma$ is a Gaussian measure and  further  that the covariance operator $K$ is positive definite. By the spectral theorem,
there is an orthonormal basis $\{e_1,\ldots,e_n\}$ for $\mathbb{R}^n$ such that $\inner{Ke_j}{e_j}>0$ for each
$1 \leq j \leq n$. Write $\inner{Ke_j}{e_j}=\sigma_j^2$, and for $y \in \mathbb{R}^n$ set
$y_j = \inner{y}{e_j}$, with which $y = y_1e_1 + \cdots + y_n e_n$ and then
\begin{align*}
\inner{Ky}{y} &= \inner{y_1Ke_1+\cdots+y_nKe_n}{y_1e_1+\cdots+y_ne_n}\\
&=\inner{y_1\sigma_1^2 e_1+ \cdots + y_n \sigma_n^2 e_n}{y_1e_1+\cdots+y_ne_n}\\
&=\sigma_1^2 y_1^2  + \cdots +  \sigma_n^2 y_n^2.
\end{align*}
And
\[
\inner{y}{a} = \inner{y_1e_1+\cdots+y_ne_n}{a_1e_1+\cdots+a_ne_n}
=a_1 y_1 + \cdots + a_n y_n.
\]
Let $\gamma_j$ be the Gaussian measure on $\mathbb{R}$ with mean $a_j$ and variance $\sigma_j^2$.
Because $\sigma_j^2>0$, the measure $\gamma_j$ has
density $p(\cdot,a_j,\sigma_j^2)$ with respect to Lebesgue measure on $\mathbb{R}$, and thus
\begin{align*}
\widetilde{\gamma}(y)&=\exp\left(i \inner{y}{a} - \frac{1}{2} \inner{Ky}{y} \right)\\
&= \exp\left(i \sum_{j=1}^n a_j y_j - \frac{1}{2} \sum_{j=1}^n \sigma_j^2 y_j^2 \right)\\
 &=\prod_{j=1}^n \exp\left(ia_jy_j - \frac{1}{2} \sigma_j^2 y_j^2 \right)\\
 &=\prod_{j=1}^n \widetilde{\gamma_j}(y_j)\\
 &=\prod_{j=1}^n \int_{\mathbb{R}} \exp(iy_j t) d\gamma_j(t)\\
 &=\prod_{j=1}^n \int_{\mathbb{R}} \exp(iy_j t) p(t,a_j,\sigma_j^2) dt\\
 &=\int_{\mathbb{R}^n} \prod_{j=1}^n \exp(iy_j x_j) p(x_j,a_j,\sigma_j^2) dx\\
 &=\int_{\mathbb{R}^n} e^{i\inner{y}{x}} \prod_{j=1}^n p(x_j,a_j,\sigma_j^2) dx.
\end{align*}
This implies that $\gamma$ has density
\[
x \mapsto \prod_{j=1}^n p(x_j,a_j,\sigma_j^2), \qquad x \in \mathbb{R}^n,
\]
with respect to Lebesgue measure on $\mathbb{R}^n$.
Moreover,
\begin{align*}
\inner{K^{-1}(x-a)}{x-a}&=\inner{\sum_{j=1}^n \sigma_j^{-2} (x_j-a_j)e_j}{\sum_{j=1}^n (x_j-a_j)e_j}\\
&=\sum_{j=1}^n  \frac{(x_j-a_j)^2}{\sigma_j^2},
\end{align*}
so we have, as $\det K = \prod_{j=1}^n \sigma_j^2$,
\begin{align*}
\prod_{j=1}^n p(x_j,a_j,\sigma_j^2) &= 
\prod_{j=1}^n  \frac{1}{\sigma_j \sqrt{2\pi}}  \exp\left( - \frac{(x_j-a_j)^2}{2\sigma_j^2} \right)\\
&=\frac{1}{\sqrt{(2\pi)^n \det K}} \exp\left(-\frac{1}{2} \inner{K^{-1}(x-a)}{x-a} \right).
\end{align*}
\end{proof}


Because $\mathbb{R}$ is a second-countable topological space, the Borel $\sigma$-algebra
$\mathscr{B}_{\mathbb{R}^n}$ is equal to the product 
$\sigma$-algebra $\bigotimes_{j=1}^n \mathscr{B}_{\mathbb{R}}$. 
The density of the standard Gaussian measure $\gamma_n$ with respect to Lebesgue measure on
$\mathbb{R}^n$ is, by Theorem \ref{density},
\[
x \mapsto \frac{1}{\sqrt{(2\pi)^n}} \exp\left(-\frac{1}{2}\inner{x}{x} \right), \qquad x \in \mathbb{R}^n.
\]
It follows that $\gamma_n$ is equal to the product measure $\prod_{j=1}^n \gamma_1$, and thus that the probability
space $(\mathbb{R}^n,\mathscr{B}_{\mathbb{R}^n},\gamma_n)$ is equal to the product $\prod_{j=1}^n (\mathbb{R},\mathscr{B}_{\mathbb{R}},
\gamma_1)$. 

For $f_1,\ldots,f_n \in L^2(\gamma_1)$, we define $f_1 \otimes \cdots \otimes f_n \in L^2(\gamma_n)$,
called the \textbf{tensor product of $f_1,\ldots,f_n$}, by
\[
(f_1 \otimes \cdots \otimes f_n)(x) = \prod_{j=1}^n f_j(x_j), \qquad x = (x_1,\ldots,x_n) \in \mathbb{R}^n.
\]
It is straightforward to check that for $f_1,\ldots,f_n,g_1,\ldots,g_n \in L^2(\gamma_1)$,
\[
\inner{f_1\otimes \cdots \otimes f_n}{g_1 \otimes \cdots \otimes g_n}_{L^2(\gamma_n)}
=\prod_{j=1}^n \inner{f_j}{g_j}_{L^2(\gamma_1)}.
\]
One proves that the linear span of the collection of all tensor products is dense in $L^2(\gamma_n)$, and that 
$\{v_k: k \geq 0\}$ is an orthonormal basis for $L^2(\gamma_1)$, then
\begin{equation}
\{v_{k_1} \otimes \cdots \otimes v_{k_n}: (k_1,\ldots,k_n) \in \mathbb{Z}_{\geq 0}^n\}
\label{tensor}
\end{equation}
is an orthonormal basis for $L^2(\gamma_n)$. 


We will later use the following statement about  centered Gaussian measures.\footnote{Vladimir I. Bogachev,
{\em Gaussian Measures}, p.~5, Lemma 1.2.5.}

\begin{theorem}
Let $\gamma$ be a Gaussian measure on $\mathbb{R}^n$ with mean $0$ and let $\theta \in \mathbb{R}$.
Then the pushforward of the product measure $\gamma \times \gamma$ on $\mathbb{R}^n \times \mathbb{R}^n$
under the mapping $(u,v) \mapsto u \sin \theta + v \cos\theta$, $\mathbb{R}^n \times \mathbb{R}^n \to
\mathbb{R}^n$, is equal to $\gamma$. 
\label{sincos}
\end{theorem}
\begin{proof}
Let $\mu$ be the pushforward of $\gamma \times \gamma$ under the above mapping. 
and let $K \in \mathscr{L}(\mathbb{R}^n)$ be the covariance operator of $\gamma$.
For $y \in \mathbb{R}^n$, using the change of variables formula,
\begin{align*}
\int_{\mathbb{R}^n} \exp\left( i \inner{y}{x} \right) d\mu(x)&=\int_{\mathbb{R}^n \times \mathbb{R}^n} \exp\left(i\inner{y}{u \sin \theta + v \cos\theta}\right)
d(\gamma \times \gamma)(u,v)\\
&=\left(\int_{\mathbb{R}^n} \exp\left(i  \inner{y \sin \theta}{u} \right) d\gamma(u) \right)\\
&\cdot \left(\int_{\mathbb{R}^n} \exp\left(i  \inner{y \cos \theta}{v} \right) d\gamma(v) \right)\\
&=\widetilde{\gamma}(y\sin\theta) \widetilde{\gamma}(y\cos\theta).
\end{align*}
By Theorem \ref{GaussianRn},
\begin{align*}
\widetilde{\gamma}(y\sin\theta) \widetilde{\gamma}(y\cos\theta)&=\exp\left(-\frac{1}{2}\inner{Ky\sin\theta}{y\sin\theta} \right)
\exp\left(-\frac{1}{2}\inner{Ky\cos\theta}{y\cos\theta} \right)\\
&=\exp\left(-\frac{1}{2}\sin^2 (\theta) \inner{Ky}{y}-\frac{1}{2}\cos^2(\theta) \inner{Ky}{y} \right)\\
&=\exp\left(-\frac{1}{2}\inner{Ky}{y}\right).
\end{align*}
Thus, the characteristic function of $\mu$ is
\[
\widetilde{\mu}(y) = \exp\left(-\frac{1}{2}\inner{Ky}{y}\right), \qquad y \in \mathbb{R}^n,
\]
which implies that $\mu$ is equal to the Gaussian measure with mean $0$ and covariance operator $K$, i.e.,
$\mu=\gamma$. 
\end{proof}



\section{Hermite polynomials}
For $k \geq 0$, we define the \textbf{Hermite polynomial $H_k$} by
\[
H_k(t) = \frac{(-1)^k}{\sqrt{k!}} \exp\left(\frac{t^2}{2}\right) \frac{d^k}{dt^k} \exp\left(-\frac{t^2}{2}\right),
\qquad t \in \mathbb{R}.
\]
It is apparent that $H_k(t)$ is a polynomial of  degree $k$. 

\begin{theorem}
For real $\lambda$ and $t$,
\[
\exp\left(\lambda t-\frac{1}{2}\lambda^2\right) = \sum_{k=0}^\infty  \frac{1}{\sqrt{k!}} H_k(t) \lambda^k.
\]
\label{generating}
\end{theorem}
\begin{proof}
For $u \in \mathbb{C}$, let $g(u)=\exp\left(-\frac{1}{2}u^2 \right)$. For $t \in \mathbb{R}$,
\begin{align*}
g(u) &= \sum_{k=0}^\infty \frac{g^{(k)}(t)}{k!} (u-t)^k\\
&= \sum_{k=0}^\infty \frac{\sqrt{k!}}{(-1)^k} \exp\left( - \frac{t^2}{2} \right) H_k(t) \frac{1}{k!} (u-t)^k\\
&= \exp\left( - \frac{t^2}{2} \right) \sum_{k=0}^\infty \frac{(-1)^k}{\sqrt{k!}} H_k(t) (u-t)^k.
\end{align*}
Therefore, for real $\lambda$ and $t$,
\begin{align*}
\exp\left(\lambda t-\frac{1}{2}\lambda^2\right)&=\exp\left(\frac{1}{2}t^2 - \frac{1}{2}(\lambda-t)^2 \right)\\
&=\exp\left(\frac{1}{2}t^2\right) g(\lambda-t)\\
&=\exp\left(\frac{1}{2}t^2\right) g(t-\lambda)\\
&=\exp\left(\frac{1}{2}t^2\right)  \exp\left( - \frac{t^2}{2} \right) \sum_{k=0}^\infty \frac{(-1)^k}{\sqrt{k!}} H_k(t) (-\lambda)^k\\
&=\sum_{k=0}^\infty  \frac{1}{\sqrt{k!}} H_k(t) \lambda^k.
\end{align*}
\end{proof}


\begin{theorem}
Let $\gamma_1$ be the standard Gaussian measure on $\mathbb{R}$, with density
$p(t,0,1)= \frac{1}{\sqrt{2\pi}}  \exp\left( - \frac{t^2}{2} \right)$. Then
\[
\{H_k: k \geq 0\}
\]
is an orthonormal basis for $L^2(\gamma_1)$.
\end{theorem}
\begin{proof}
For $\lambda,\mu \in \mathbb{R}$, on the one hand, using \eqref{identity} with $a=\lambda+\mu$ and $\sigma=1$,
\[
\begin{split}
&\int_{\mathbb{R}} \exp\left(\lambda t-\frac{1}{2}\lambda^2\right) \exp\left(\mu t-\frac{1}{2}\mu^2\right) d\gamma_1(t)\\
=&e^{\lambda \mu} \int_{\mathbb{R}} \frac{1}{\sqrt{2\pi}} \exp\left(-\frac{1}{2}(t-(\lambda+\mu))^2 \right) dt\\
=&e^{\lambda \mu}.
\end{split}
\]
On the other hand,  using Theorem \ref{generating},
\[
\begin{split}
&\int_{\mathbb{R}} \exp\left(\lambda t-\frac{1}{2}\lambda^2\right) \exp\left(\mu t-\frac{1}{2}\mu^2\right) d\gamma_1(t)\\
=&\int_{\mathbb{R}} \left( \sum_{k=0}^\infty  \frac{1}{\sqrt{k!}} H_k(t) \lambda^k\right)
\left(\sum_{l=0}^\infty  \frac{1}{\sqrt{l!}} H_l(t) \mu^l\right) d\gamma_1(t)\\
&=\int_{\mathbb{R}} \sum_{k, l \geq 0} \frac{1}{\sqrt{k! l!}} \lambda^k \mu^l H_k(t) H_l(t) d\gamma_1(t)\\
&=\sum_{k,l \geq 0} \frac{1}{\sqrt{k! l!}}  \lambda^k \mu^l \inner{H_k}{H_l}_{L^2(\gamma_1)}.
\end{split}
\]
Therefore
\[
\sum_{k,l \geq 0} \frac{1}{\sqrt{k! l!}}  \lambda^k \mu^l \inner{H_k}{H_l}_{L^2(\gamma_1)} = \sum_{k=0}^\infty \frac{1}{k!} \lambda^k \mu^k.
\]
From this, we get that if $k \neq l$ then 
$\frac{1}{\sqrt{k! l!}}  \inner{H_k}{H_l}_{L^2(\gamma_1)} = 0$,
i.e.
\[
\inner{H_k}{H_l}_{L^2(\gamma_1)}=0.
\]
 If $k=l$, then
$\frac{1}{\sqrt{k! l!}}  \inner{H_k}{H_l}_{L^2(\gamma_1)} = \frac{1}{k!}$, i.e.
\[
\inner{H_k}{H_k}_{L^2(\gamma_1)}=1.
\]
Therefore, $\{H_k: k \geq 0\}$ is an orthonormal set in $L^2(\gamma_1)$. 

Suppose that $f \in L^2(\gamma_1)$ satisfies $\inner{f}{H_k}_{L^2(\gamma_1)}=0$ for each
$k \geq 0$. 
Because $H_k(t)$ is a polynomial of  degree $k$,  for each $k \geq 0$ we have
\[
\Span\{H_0,H_1,H_2,\ldots,H_k\} = \Span\{1,t,t^2,\ldots,t^k\}.
\]
Hence for each $k \geq 0$, $\inner{f}{t^k}_{L^2(\gamma_1)}=0$. 
One then proves that $\Span\{1,t,t^2,\ldots\}$ is dense in $L^2(\gamma_1)$, from which it follows that
the linear span of the Hermite polynomials is dense in $L^2(\gamma_1)$ and thus that they are an orthonormal basis.
\end{proof}


\begin{lemma}
For $k \geq 1$,
\[
H_k'(t) = \sqrt{k} H_{k-1}(t), \qquad H_k'(t)= tH_k(t) - \sqrt{k+1} H_{k+1}(t).
\]
\end{lemma}
\begin{proof}
Theorem \ref{generating} says
\[
\exp\left(\lambda t-\frac{1}{2}\lambda^2\right) = \sum_{k=0}^\infty  \frac{1}{\sqrt{k!}} H_k(t) \lambda^k.
\]
On the one hand,
\begin{align*}
\frac{d}{dt} \exp\left(\lambda t-\frac{1}{2}\lambda^2\right) &= \lambda \exp\left(\lambda t-\frac{1}{2}\lambda^2\right)\\
&=\sum_{k=0}^\infty  \frac{1}{\sqrt{k!}} H_k(t) \lambda^{k+1}\\
&=\sum_{k=1}^\infty \frac{1}{\sqrt{(k-1)!}} H_{k-1}(t) \lambda^k.
\end{align*}
On the other hand,
\[
\frac{d}{dt} \exp\left(\lambda t-\frac{1}{2}\lambda^2\right)  = \sum_{k=0}^\infty \frac{1}{\sqrt{k!}} H_k'(t) \lambda^k.
\]
Therefore,  $H_0'(t)=0$, and for $k \geq 1$,
\[
\frac{1}{\sqrt{(k-1)!}} H_{k-1}(t) =  \frac{1}{\sqrt{k!}} H_k'(t),
\]
i.e.,
\[
H_k'(t) = \sqrt{k} H_{k-1}(t).
\]
\end{proof}

For $\alpha=(k_1,\ldots,k_n) \in \mathbb{Z}_{\geq 0}^n$, we define the \textbf{Hermite polynomial $H_\alpha$} by
\[
H_\alpha(x) = H_{k_1}(x_1) \cdots H_{k_n}(x_n), \qquad x = (x_1,\ldots,x_n) \in \mathbb{R}^n.
\]
Because the collection of all Hermite polynomials $H_k$ is an orthonormal basis for the Hilbert space $L^2(\gamma_1)$, following
\eqref{tensor} we have that the collection of all Hermite polynomials $H_\alpha$ is an orthonormal basis for the Hilbert
space $L^2(\gamma_n)$. 

\begin{theorem}
For $\gamma_n$ the standard Gaussian measure on $\mathbb{R}^n$, with mean $0 \in \mathbb{R}^n$ and covariance operator
$\id_{\mathbb{R}^n}$,
the collection
\[
\{H_\alpha: \alpha \in \mathbb{Z}_{\geq 0}^n\}
\]
is an orthonormal basis for $L^2(\gamma_n)$.
\end{theorem}

For $\alpha=(k_1,\ldots,k_n) \in \mathbb{Z}_{\geq 0}^n$, write $|\alpha|=k_1+\cdots+k_n$. 
For $k \geq 0$, we define
\[
\mathcal{X}_k = \Span\{H_\alpha: |\alpha|=k\},
\]
which is a subspace of $L^2(\gamma_n)$ of dimension
\[
\binom{k+n-1}{k}.
\]
As $\mathcal{X}_k$ 
is a finite dimensional subspace of $L^2(\gamma_n)$, it is closed. 
$L^2(\gamma_n)$ is equal to the orthogonal direct sum of the $\mathcal{X}_k$:
\[
L^2(\gamma_n) = \bigoplus_{k=0}^\infty \mathcal{X}_k.
\]
Let 
\[
I_k:L^2(\gamma_n) \to \mathcal{X}_k
\]
 be the  orthogonal projection onto $\mathcal{X}_k$. 


\section{The Ornstein-Uhlenbeck semigroup}
Let $\gamma$ be a Gaussian measure on $\mathbb{R}^n$ with mean $0$ and covariance operator
$K$. 
For $t \geq 0$, we define $M_t:\mathbb{R}^n \times \mathbb{R}^n \to \mathbb{R}^n$ by
\[
M_t(u,v) = e^{-t} u + \sqrt{1-e^{-2t}}v, \qquad (x,y) \in \mathbb{R}^n \times \mathbb{R}^n.
\]
By Theorem \ref{sincos}, ${M_t}_*(\gamma \times \gamma)=\gamma$. 
Therefore, for $p \geq 1$ and $f \in L^p(\gamma)$, using the change of variables formula,
\[
\int_{\mathbb{R}^n} |f(x)|^p d\gamma(x)=\int_{\mathbb{R}^n \times \mathbb{R}^n}
|f(M_t(u,v))|^p d(\gamma \times \gamma)(u,v).
\]
Applying Fubini's theorem, the function
\[
u \mapsto \int_{\mathbb{R}^n} |f(M_t(u,v))|^p d\gamma(v)
=\int_{\mathbb{R}^n} |f(e^{-t} u + \sqrt{1-e^{-2t}}v)|^p d\gamma(v)
\]
belongs to $L^1(\gamma)$. 
We define the \textbf{Ornstein-Uhlenbeck semigroup $\{T_t: t \geq 0\}$
on $L^p(\gamma)$}, $p \geq 1$, by
\[
T_t(f)(u) =
\int_{\mathbb{R}^n} f(M_t(u,v)) d\gamma(v)
= \int_{\mathbb{R}^n} f\left(e^{-t}u+\sqrt{1-e^{-2t}}v\right) d\gamma(v),
\]
for $u \in \mathbb{R}^n$. 


\begin{theorem}
Let $\gamma$ be a Gaussian measure on
$\mathbb{R}^n$ with mean $0$.
If $f \in L^1(\gamma)$, then
\[
\int_{\mathbb{R}^n} (T_t f)(x) d\gamma(x)
=\int_{\mathbb{R}^n} f(x) d\gamma(x).
\]
\label{invariant}
\end{theorem}
\begin{proof}
Using Fubini's theorem, then
the change of variables formula, then  Theorem \ref{sincos}, 
\begin{align*}
\int_{\mathbb{R}^n} (T_t f)(u) d\gamma(u) &=
\int_{\mathbb{R}^n} \left(\int_{\mathbb{R}^n} f(M_t(u,v)) d\gamma(v)\right)
d\gamma(u)\\
&=\int_{\mathbb{R}^n \times\mathbb{R}^n} f(M_t(u,v)) d(\gamma \times \gamma)(u,v)\\
&=\int_{\mathbb{R}^n} f(x) d({M_t}_*(\gamma \times \gamma))(x)\\
&=\int_{\mathbb{R}^n} f(x) d\gamma(x).
\end{align*}
\end{proof}


\begin{theorem}
Let $\gamma$ be a Gaussian measure on
$\mathbb{R}^n$ with mean $0$.
For $p \geq 1$ and $t \geq 0$, $T_t$ is a bounded linear operator
$L^p(\gamma) \to L^p(\gamma)$ with operator norm $1$.
\label{bounded}
\end{theorem}
\begin{proof}
For $f \in L^p(\gamma)$,
using
Jensen's  inequality and then Theorem \ref{invariant},
\begin{align*}
\norm{T_t f}_{L^p(\gamma)}^p& =
\int_{\mathbb{R}^n} \left| \int_{\mathbb{R}^n} f(M_t(u,v))
d\gamma(v) \right|^p d\gamma(u) \\
&\leq \int_{\mathbb{R}^n} \left( \int_{\mathbb{R}^n} | f(M_t(u,v))|^p d\gamma(v) \right) d\gamma(u)\\
&=\int_{\mathbb{R}^n} T_t(|f|^p)(u) d\gamma(u)\\
&=\int_{\mathbb{R}^n} |f|^p(u) d\gamma(u)\\
&=\norm{f}_{L^p(\gamma)}^p,
\end{align*}
i.e. $\norm{T_t f}_{L^p(\mu)} \leq \norm{f}_{L^p(\mu)}$.
This shows that the operator norm of $T_t$ is $\leq 1$. But, as $\gamma$ is a probability measure,
\[
T_t 1 = \int_{\mathbb{R}^n} 1 d\gamma(v)
=1,
\]
so $T_t$ has operator norm $1$. 
\end{proof}



For a Banach space $E$, we denote by $\mathscr{B}(E)$ the set of bounded linear operators
$E \to E$. The \textbf{strong operator topology on $E$} is the coarsest topology on $E$ such that
for each $x \in E$, the map $A \mapsto Ax$ is continuous $\mathscr{B}(E) \to \mathbb{E}$. 
To say that a map $Q:[0,\infty) \to \mathscr{B}(E)$ is \textbf{strongly continuous means} that
for each $t \in [0,\infty)$,
$Q(s) \to Q_t$ in the strong operator topology as $s \to t$, i.e., for each $x \in E$,
$Q(s) x \to Q(t) x$ in $E$. 

A \textbf{one-parameter semigroup in $\mathscr{B}(E)$} is a map
$Q:[0,\infty) \to \mathscr{B}(E)$ such that (i) $Q(0)=\id_E$ and (ii) for $s,t \geq 0$,
$Q(s+t)=Q(s) \circ Q(t)$. For a one-parameter semigroup to be strongly continuous,
one proves that it is equivalent that 
$Q(t) \to \id_E$ in the strong operator topology as $t \downarrow 0$, i.e. 
for each $x \in E$, $Q(t)x \to x$.\footnote{Walter Rudin, {\em Functional Analysis},
second ed., p.~376, Theorem 13.35.}

We now establish that the $\{T_t:t \geq 0\}$ is indeed a one-parameter semigroup and that it is strongly continuous.\footnote{Vladimir I. Bogachev,
{\em Gaussian Measures}, p.~10, Theorem 1.4.1.}

\begin{theorem}
Suppose $\mu$ is a Gaussian measure on $\mathbb{R}^n$ with mean $0$ and let $p\geq 1$. 
Then 
$\{T_t:t \geq 0\}$ is a strongly continuous one-parameter semigroup in $\mathscr{B}(L^p(\gamma))$. 
\label{semigroup}
\end{theorem}
\begin{proof}
For $f \in L^p(\gamma)$, because $\gamma$ is a probability measure,
\[
T_0(f)(u) = 
 \int_{\mathbb{R}^n} f(u) d\gamma(v)
=f(u),
\]
hence $T_0 = \id_{L^p(\mu)}$. For $s,t \geq 0$,
define $P:\mathbb{R}^n \times \mathbb{R}^n \to \mathbb{R}^n$ by
\[
P(u,v) = e^{-s} \frac{\sqrt{1-e^{-2t}}}{\sqrt{1-e^{-2t-2s}}} u + \frac{\sqrt{1-e^{-2s}}}{\sqrt{1-e^{-2t-2s}}}v.
\]
By Theorem \ref{sincos}, $P_*(\gamma \times \gamma)=\gamma$, whence
\[
\begin{split}
&(T_t(T_s f))(x)\\
=&\int_{\mathbb{R}^n} (T_s f)\left(e^{-t}x+\sqrt{1-e^{-2t}}y \right)d\gamma(y)\\
=&\int_{\mathbb{R}^n} \left( \int_{\mathbb{R}^n} 
f\left(e^{-s}\left(e^{-t}x+\sqrt{1-e^{-2t}}y\right)+\sqrt{1-e^{-2s}}w\right) d\gamma(w) \right) d\gamma(y)\\
=&\int_{\mathbb{R}^n \times \mathbb{R}^n} f\left(e^{-s-t}x+\sqrt{1-e^{-2t-2s}} P(y,w) \right)
d(\gamma \times \gamma)(y,w)\\
=&\int_{\mathbb{R}^n \times \mathbb{R}^n} (f \circ M_{s+t})(x,P(y,w)) d(\gamma \times \gamma)(y,w)\\
=&\int_{\mathbb{R}^n} (f \circ M_{s+t})(x,z)  d\gamma(z)\\
=&T_{s+t}(f)(x),
\end{split}
\]
hence $T_t \circ T_s = T_{s+t}$. This establishes that $\{T_t: t \geq 0\}$ is a semigroup. 

For $f \in C_b(\mathbb{R}^n)$, $u \in \mathbb{R}^n$, and $v \in \mathbb{R}^n$,
as $t \downarrow 0$ we have 
\[
f\left(e^{-t}u+\sqrt{1-e^{-2t}}v\right) - f(u) \to 0,
\]
thus by the dominated convergence theorem, since
\[
\left| f\left(e^{-t}u+\sqrt{1-e^{-2t}}v\right) - f(u) \right| \leq 2\norm{f}_\infty
\]
and $\gamma$ is a probability measure,
we have
\[
 \int_{\mathbb{R}^n} \left(f\left(e^{-t}u+\sqrt{1-e^{-2t}}v\right)  - f(u) \right) d\gamma(v) \to 0,
\]
and hence
\begin{align*}
(T_t f - T_0 f)(u)& = 
\int_{\mathbb{R}^n} f\left(e^{-t}u+\sqrt{1-e^{-2t}}v\right) d\gamma(v)-\int_{\mathbb{R}^n} f(u) d\gamma(v)\\
&= \int_{\mathbb{R}^n} \left(f\left(e^{-t}u+\sqrt{1-e^{-2t}}v\right)  - f(u) \right) d\gamma(v)\\
&\to 0.
\end{align*}
Because this is true for each $u \in \mathbb{R}^n$ and
\[
|(T_t f - T_0 f)(u)|  \leq \int_{\mathbb{R}^n} 2\norm{f}_\infty d\gamma(v) = 2\norm{f}_\infty,
\]
by the dominated convergence theorem we then have
\begin{equation}
\norm{T_t f -T_0 f}_{L^p(\gamma)} \to 0.
\label{dominated}
\end{equation}
Now let $f \in L^p(\gamma)$. There is a sequence $f_j \in C_b(\mathbb{R}^n)$ satisfying
$\norm{f_j-f}_{L^p(\gamma)} \to 0$, with $\norm{f_j}_{L^p(\gamma)} \leq 2 \norm{f}_{L^p(\gamma)}$ for all $j$.
For any $t \geq 0$,
\begin{align*}
\norm{T_t f - T_0 f}_{L^p(\gamma)}&\leq \norm{T_tf - T_t f_j}_{L^p(\gamma)}
+\norm{T_t f_j - T_0 f_j}_{L^p(\gamma)}
+\norm{T_0 f_j - T_0 f}_{L^p(\gamma)}\\
&= \norm{T_t (f-f_j)}_{L^p(\gamma)} + \norm{T_tf-T_0 f_j}_{L^p(\gamma)} + \norm{f_j-f}_{L^p(\gamma)}\\
&\leq  \norm{f-f_j}_{L^p(\gamma)} +  \norm{T_tf-T_0 f_j}_{L^p(\gamma)} +\norm{f_j-f}_{L^p(\gamma)}.
\end{align*}
Let $\epsilon>0$ and let $j$ be so large that $\norm{f-f_j}_{L^p(\gamma)}<\epsilon$.
Because $f_j \in C_b(\mathbb{R}^n)$, by \eqref{dominated} there is some $\delta>0$ such that 
when $0 < t < \delta$, 
$\norm{T_t f_j - f_j}_{L^p(\gamma)}<\epsilon$. Then when $0 < t < \delta$,
\[
\norm{T_t f - T_0 f}_{L^p(\gamma)} \leq \epsilon+\epsilon+\epsilon,
\]
which shows that for each $f \in L^p(\gamma)$, $\norm{T_t f - T_0f}_{L^p(\gamma)}$ as $t \downarrow 0$, which suffices
to establish that $\{T_t:t \geq 0\}$ is strongly continuous $[0,\infty) \to \mathscr{B}(L^p(\gamma))$. 
\end{proof}

For $t >0$, we define $L_t \in \mathscr{B}(L^p(\gamma))$ by 
\[
L_t f =\frac{1}{t}(T_t f -f), \qquad f \in L^p(\gamma).
\]
We define $\mathscr{D}(L)$ to be the set of those $f \in L^p(\gamma)$ such that $L_t f$ converges to
some element of $L^p(\gamma)$ as $t \downarrow 0$,
and we define 
$L:\mathscr{D}(L) \to L^p(\gamma)$. This is the \textbf{infinitesimal generator} of the semigroup $\{T_t: t \geq 0\}$, and the infinitesimal generator $L$ of the 
Ornstein-Uhlenbeck semigroup is called the \textbf{Ornstein-Uhlenbeck operator}.
Because the Ornstein-Uhlenbeck semigroup is strongly continuous, we get the following.\footnote{Walter Rudin, {\em Functional Analysis},
second ed., p.~376, Theorem 13.35.} 

\begin{theorem}
Suppose $\mu$ is a Gaussian measure on $\mathbb{R}^n$ with mean $0$, let $p\geq 1$, and let 
$L$ be the infinitesimal generator of the Ornstein-Uhlenbeck semigroup $\{T_t:t \geq 0\}$. Then:
\begin{enumerate}
\item $\mathscr{D}(L)$ is a dense linear subspace of $L^p(\gamma)$ and $L:\mathscr{D}(L) \to L^p(\gamma)$ is a closed operator.
\item For each $f \in \mathscr{D}(L)$ and for each $t \geq 0$,
\[
\frac{d}{dt} (T_tf) = (L \circ T_t) f = (T_t \circ L) f.
\]
\item For $f \in L^p(\gamma)$ and $K$ a compact subset of $[0,\infty)$, 
$(\exp(t L_\epsilon)f \to T_t f$ as $\epsilon \downarrow 0$ uniformly for $t \in K$. 
\item For $\lambda \in \mathbb{C}$ with $\Re \lambda>0$, $R(\lambda):L^p(\gamma) \to L^p(\gamma)$ defined by
\[
R(\lambda)f = \int_0^\infty e^{-\lambda t} T_t f dt, \qquad f \in L^p(\gamma),
\]
belongs to $\mathscr{B}(L^p(\gamma))$, the range of $R(\lambda)$ is equal to $\mathscr{D}(L)$, and
\[
((\lambda I-L) \circ R(\lambda))f = f, \quad f \in L^p(\gamma),
\qquad (R(\lambda) \circ (\lambda I-L))f = \mathscr{D}(L),
\]
where $I$ is the identity operator on $L^p(\gamma)$. 
\end{enumerate}
\label{generator}
\end{theorem}



We remind ourselves that if $H$ is a Hilbert space with inner product $\inner{\cdot}{\cdot}$,
an element $A$ of $\mathscr{B}(H)$ is said to be a \textbf{positive operator} when
$\inner{Ax}{x} \geq 0$ for all $x \in H$.
We prove that each $T_t$ is a positive operator on the Hilbert space $L^2(\gamma)$.\footnote{Vladimir I. Bogachev,
{\em Gaussian Measures}, p.~10, Theorem 1.4.1.}

\begin{theorem}
Suppose $\mu$ is a Gaussian measure on $\mathbb{R}^n$ with mean $0$. 
For each $t \geq 0$, $T_t \in \mathscr{B}(L^2(\mu))$
is a positive operator.
\end{theorem}
\begin{proof}
For $t \geq 0$, define $N_t:\mathbb{R}^n \times \mathbb{R}^n \to \mathbb{R}^n \times  \mathbb{R}^n$ by
\[
O_t(x,y) = \left(e^{-t}x+\sqrt{1-e^{-2t}}y,-\sqrt{1-e^{-2t}}x+e^{-t}y\right), \qquad (x,y) \in \mathbb{R}^n \times \mathbb{R}^n,
\]
whose transpose is the linear operator $N_t^*:\mathbb{R}^n \times \mathbb{R}^n \to \mathbb{R}^n \times  \mathbb{R}^n$ defined by
\[
O_t^*(u,v) = \left(e^{-t}u-\sqrt{1-e^{-2t}}v,\sqrt{1-e^{-2t}}u+e^{-t}v\right), \qquad (u,v) \in \mathbb{R}^n \times \mathbb{R}^n.
\]
For $(x,y) \in \mathbb{R}^n \times \mathbb{R}^n$, we calculate 
\[
\begin{split}
&\int_{\mathbb{R}^n \times \mathbb{R}^n} e^{i\inner{(x,y)}{(u,v)}} d({O_t}_*(\gamma \times \gamma))(u,v)\\
=&\int_{\mathbb{R}^n \times \mathbb{R}^n} e^{i\inner{(x,y)}{O_t(u,v)}} d(\gamma \times \gamma)(u,v)\\
=&\int_{\mathbb{R}^n \times \mathbb{R}^n}  e^{i\inner{O_t^*(x,y)}{(u,v)}} d(\gamma \times \gamma)(u,v)\\
=&\widetilde{\gamma \times \gamma}(O_t^*(x,y))\\
=&\widetilde{\gamma \times \gamma}(e^{-t}x-\sqrt{1-e^{-2t}}y,\sqrt{1-e^{-2t}}x+e^{-t}y)\\
=&\widetilde{\gamma}(e^{-t}x-\sqrt{1-e^{-2t}}y) \widetilde{\gamma}(\sqrt{1-e^{-2t}}x+e^{-t}y)\\
=&\exp\left(-\frac{1}{2}\inner{K(e^{-t}x-\sqrt{1-e^{-2t}}y)}{e^{-t}x-\sqrt{1-e^{-2t}}y}\right)\\
&\cdot \exp\left(-\frac{1}{2}\inner{K(\sqrt{1-e^{-2t}}x+e^{-t}y)}{\sqrt{1-e^{-2t}}x+e^{-t}y} \right)\\
=&\exp\left(-\frac{1}{2}\inner{Kx}{x}-\frac{1}{2}\inner{Ky}{y}\right)\\
=&\widetilde{\gamma}(x) \widetilde{\gamma}(y)\\
=&\widetilde{\gamma \times \gamma}(x,y),
\end{split}
\]
which shows that ${O_t}_*(\gamma \times \gamma)$ and $\gamma \times \gamma$ have equal characteristic functions and hence
are themselves equal. 

For $f,g \in L^2(\gamma)$ and $t \geq 0$,
\begin{align*}
\inner{T_t f}{g}_{L^2(\gamma)}&=\int_{\mathbb{R}^n} (T_t f)(x) g(x) d\mu(x)\\
&=\int_{\mathbb{R}^n \times \mathbb{R}^n} f\left(e^{-t}x+\sqrt{1-e^{-2t}}y\right) g(x) d(\gamma \times \gamma)(x,y)\\
&=\int_{\mathbb{R}^n \times \mathbb{R}^n} (f \circ \pi_1 \circ O_t)(x,y) (g \circ \pi_1 \circ O_t^{-1} \circ O_t)(x,y) d(\gamma \times \gamma)(x,y)\\
&=\int_{\mathbb{R}^n \times \mathbb{R}^n} (f \circ \pi_1)(u,v) (g \circ \pi_1 \circ O_t^{-1})(u,v) d({O_t}_* (\gamma \times \gamma))(u,v)\\
&=\int_{\mathbb{R}^n \times \mathbb{R}^n} (f \circ \pi_1)(u,v) (g \circ \pi_1 \circ O_t^{-1})(u,v) d(\gamma \times \gamma)(u,v)\\
&=\int_{\mathbb{R}^n \times \mathbb{R}^n} f(u)g\left(e^{-t}u-\sqrt{1-e^{-2t}}v\right) d(\gamma \times \gamma)(u,v)\\
&=\int_{\mathbb{R}^n} f(u) \left(\int_{\mathbb{R}^n} g(M_t(u,-v)) d\gamma(v) \right) d\gamma(u)\\
&=\int_{\mathbb{R}^n} f(u)  \left(\int_{\mathbb{R}^n} g(M_t(u,v)) d\gamma(v) \right) d\gamma(u)\\
&=\int_{\mathbb{R}^n} f(u) (T_tg)(u) d\gamma(u)\\
&=\inner{f}{T_tg}_{L^2(\gamma)},
\end{align*}
which establishes that $T_t$ is a self-adjoint operator on $L^2(\gamma)$. 

Furthermore, using that $T_t = T_{t/2} \circ T_{t/2}$ and that $T_{t/2}$ is self-adjoint,
\[
\inner{T_t f}{f}_{L^2(\gamma)}
=\inner{T_{t/2} T_{t/2} f}{f}_{L^2(\gamma)}
=\inner{T_{t/2} f}{T_{t/2}^* f}_{L^2(\gamma)}
=\inner{T_{t/2} f}{T_{t/2} f}_{L^2(\gamma)},
\]
which is $\geq 0$, which establishes that $T_t$ is a positive operator on $L^2(\gamma)$.
\end{proof}


We now write the Ornstein-Uhlenbeck semigroup using the orthogonal projections $I_k:L^2(\gamma_n) \to 
\mathcal{X}_k$, where $\gamma_n$ is the standard Gaussian measure on $\mathbb{R}^n$.\footnote{Vladimir I. Bogachev,
{\em Gaussian Measures}, p.~11, Theorem 1.4.4.}

\begin{theorem}
For each $t \geq 0$ and $f \in L^2(\gamma_n)$,
\[
T_t f = \sum_{k=0}^\infty e^{-kt} I_k(f).
\]
\label{projections}
\end{theorem}
\begin{proof}
Define $S_t:L^2(\gamma_n) \to L^2(\gamma_n)$ by $S_t f = \sum_{k=0}^\infty e^{-kt} I_k(f)$, which satisfies, using that
the subspaces $\mathcal{X}_k$ are pairwise orthogonal,
\[
\norm{S_t f}_{L^2(\gamma_n)}^2 = \sum_{k=0}^\infty e^{-kt} \norm{I_k(f)}_{L^2(\gamma_n)}^2
\leq  \sum_{k=0}^\infty \norm{I_k(f)}_{L^2(\gamma_n)}^2
=\norm{f}_{L^2(\gamma_n)}^2,
\]
so $S_t \in \mathscr{B}(L^2(\gamma_n))$. 
To prove that $T_t = S_t$, it suffices to prove that $T_t H_\alpha= S_t H_\alpha$ for each Hermite polynomial, which are
an orthonormal basis for $L^2(\gamma_n)$. 
For $\alpha=(k_1,\ldots,k_n)$ with $k=|\alpha|=k_1+\cdots+k_n$,
\[
S_t H_\alpha = e^{-kt} H_\alpha,
\]
and
\begin{align*}
(T_t H_\alpha)(x)&=\int_{\mathbb{R}^n} H_\alpha\left(e^{-t}x+\sqrt{1-e^{-2t}}y\right)d\gamma_n(y)\\
&=\int_{\mathbb{R}^n} \prod_{j=1}^n H_{k_j}\left(e^{-t}x_j+\sqrt{1-e^{-2t}}y_j\right) d\gamma_n(y)\\
&=\prod_{j=1}^n \int_{\mathbb{R}} H_{k_j}\left(e^{-t}x_j+\sqrt{1-e^{-2t}}y_j\right) d\gamma_1(y_j).
\end{align*}
To prove that $T_t H_\alpha = e^{-kt} H_\alpha$, it thus suffices to prove that for any 
$t$, for any $k_j$, and for any $x_j$,
\begin{equation}
\int_{\mathbb{R}} H_{k_j}\left(e^{-t}x_j+\sqrt{1-e^{-2t}}y_j\right) d\gamma_1(y_j) = e^{-k_j t} H_{k_j}(x_j).
\label{kj}
\end{equation}
For $k_j=0$, as $H_0=1$ and $\gamma_1$ is a probability measure, \eqref{kj} is true. 
Suppose that \eqref{kj} is true for  $\leq k_j$. That is, for each $0 \leq h \leq k_j$,
$T_t H_{h} = e^{-h t} H_{h}$. For any $l$, because the Hermite polynomial $H_l$ is a polynomial of degree
$l$, one checks that $T_t H_l(x_j)$ is a polynomial of degree $l$: using the binomial formula,
\[
\int_{\mathbb{R}} (e^{-t}x_j+\sqrt{1-e^{-2t}}y_j)^l \exp\left(-\frac{y_j^2}{2}\right) d\gamma_1(y_j)
\]
is a polynomial in $x_j$ of degree $l$. 
Hence $T_t H_l$ a linear combination of $H_0,H_1,\ldots,H_l$.  
For $0 \leq h \leq k_j$,
\[
\inner{T_t H_{k_j+1}}{H_h}_{L^2(\gamma_1)}=
\inner{H_{k_j+1}}{T_t H_h}_{L^2(\gamma_1)}
=\inner{H_{k_j+1}}{e^{-ht}H_h}_{L^2(\gamma_1)}
=0.
\]
Therefore there is some $c \in \mathbb{R}$ such that
$T_t H_{k_j+1} = c H_{k_j+1}$. Then check that
$c=e^{-(k_j+1)t}$. 
\end{proof}

We now give an explicit expression for the domain $\mathscr{D}(L)$ of the Ornstein-Uhlenbeck operator $L$ and for $L$ applied to an element of
its domain.\footnote{Vladimir I. Bogachev, {\em Gaussian Measures}, p.~12, Proposition 1.4.5.}

\begin{theorem}
\[
\mathscr{D}(L) = \left\{f \in L^2(\gamma_n) : \sum_{k=0}^\infty k^2 \norm{I_k(f)}_{L^2(\gamma_n)}^2<\infty\right\}.
\]
For $f \in \mathscr{D}(L)$,
\[
Lf = -\sum_{k=0}^\infty k I_f(f).
\]
\end{theorem}
\begin{proof}
Let $f \in \mathscr{D}(L)$, i.e. $\frac{T_t f - f}{t} \to Lf$ in $L^2(\gamma_n)$ as $t \downarrow 0$. For any $k \geq 0$, using
Theorem \ref{projections},
\begin{align*}
I_k L f&=I_k\left(\lim_{t \downarrow 0} \frac{T_t f-f}{t} \right)\\
&=\lim_{t \downarrow 0} \frac{I_kT_t f - I_kf}{t}\\
&=\lim_{t \downarrow 0} \frac{T_t I_k f - I_k f}{t}\\
&=\lim_{t \downarrow 0}  \frac{e^{-kt} I_k f - I_k f}{t}\\
&=\left( \lim_{t \downarrow 0} \frac{e^{-kt}-1}{t}\right) I_k f\\
&=\left(e^{-kt}\right)'\big|_{t=0} I_kf\\
&=-kI_kf.
\end{align*}
Using this, 
\begin{align*}
\sum_{k=0}^\infty k^2 \norm{I_k f}_{L^2(\gamma_n)}^2&=\sum_{k=0}^\infty \norm{I_k Lf}_{L^2(\gamma_n)}^2\\
&=\norm{\sum_{k=0}^\infty I_k Lf}_{L^2(\gamma_n)}^2\\
&=\norm{Lf}_{L^2(\gamma_n)}^2\\
&<\infty.
\end{align*}
Moreover,
\[
Lf = L\left(\sum_{k=0}^\infty I_k f \right)
=\sum_{k=0}^\infty LI_k f
=\sum_{k=0}^\infty I_kLf
=\sum_{k=0}^\infty -kI_f.
\]

Let $f \in L^2(\gamma_n)$ satisfy 
\[
\sum_{k=0}^\infty k^2 \norm{I_k f}_{L^2(\gamma_n)}^2<\infty.
\]
For $t>0$,
\begin{align*}
\norm{\frac{T_t f-f}{t} + \sum_{k=0}^\infty kI_kf}_{L^2(\gamma_n)}^2&=
\norm{\sum_{k=0}^\infty \left( \frac{e^{-kt} I_k f - I_k f}{t} + kI_kf\right)}_{L^2(\gamma_n)}^2\\
&=\sum_{k=0}^\infty \left|\frac{e^{-kt}-1}{t}+k\right|^2 \norm{I_k f}_{L^2(\gamma_n)}^2.
\end{align*}
For $t>0$ and $k \geq 0$,
\[
|t^{-1}(e^{-kt}-1)| \leq k,
\]
and thus
\[
\sum_{k=0}^\infty \left|\frac{e^{-kt}-1}{t}+k\right|^2 \norm{I_k f}_{L^2(\gamma_n)}^2
\leq \sum_{k=0}^\infty (2k)^2  \norm{I_k f}_{L^2(\gamma_n)}^2
<\infty.
\]
For each $k \geq 0$, as $t \downarrow 0$, 
\[
\frac{e^{-kt}-1}{t}+k \to 0,
\]
thus as $t \downarrow 0$,
\[
\sum_{k=0}^\infty \left|\frac{e^{-kt}-1}{t}+k\right|^2 \norm{I_k f}_{L^2(\gamma_n)}^2  \to 0
\]
and hence
\[
\norm{\frac{T_t f-f}{t} + \sum_{k=0}^\infty kI_kf}_{L^2(\gamma_n)}^2 \to 0.
\]
This means that
$\frac{T_t f -f}{t}$ converges in $L^2(\gamma_n)$ to $-\sum_{k=0}^\infty kI_kf$ as $t \downarrow 0$, and since
$\frac{T_t f-f}{t}$ converges, $f \in \mathscr{D}(L)$. 
\end{proof}

\end{document}