\documentclass{article}
\usepackage{amsmath,amssymb,mathrsfs,amsthm}
%\usepackage{tikz-cd}
\usepackage{hyperref}
\newcommand{\inner}[2]{\left\langle #1, #2 \right\rangle}
\newcommand{\tr}{\ensuremath\mathrm{tr}\,} 
\newcommand{\Span}{\ensuremath\mathrm{span}} 
\def\Re{\ensuremath{\mathrm{Re}}\,}
\def\Im{\ensuremath{\mathrm{Im}}\,}
\newcommand{\id}{\ensuremath\mathrm{id}} 
\newcommand{\var}{\ensuremath\mathrm{var}} 
\newcommand{\Lip}{\ensuremath\mathrm{Lip}} 
\newcommand{\GL}{\ensuremath\mathrm{GL}} 
\newcommand{\Hess}{\ensuremath\mathrm{Hess}\,} 
\newcommand{\diam}{\ensuremath\mathrm{diam}} 
\newcommand{\sgn}{\ensuremath\mathrm{sgn}\,} 
\newcommand{\lcm}{\ensuremath\mathrm{lcm}} 
\newcommand{\supp}{\ensuremath\mathrm{supp}\,}
\newcommand{\dom}{\ensuremath\mathrm{dom}\,}
\newcommand{\upto}{\nearrow}
\newcommand{\downto}{\searrow}
\newcommand{\norm}[1]{\left\Vert #1 \right\Vert}
\newtheorem{theorem}{Theorem}
\newtheorem{lemma}[theorem]{Lemma}
\newtheorem{proposition}[theorem]{Proposition}
\newtheorem{corollary}[theorem]{Corollary}
\theoremstyle{definition}
\newtheorem{definition}[theorem]{Definition}
\newtheorem{example}[theorem]{Example}
\begin{document}
\title{Gaussian integrals}
\author{Jordan Bell\\ \texttt{jordan.bell@gmail.com}\\Department of Mathematics, University of Toronto}
\date{\today}

\maketitle

\section{One dimensional Gaussian integrals}
For $p \in \mathbb{C}$, let\footnote{Eberhard Zeidler,
{\em Quantum Field Theory I: Basics in Mathematics and Physics},
p.~493, Problem 7.1.}
\[
h(p) = \int_{\mathbb{R}} e^{-x^2/2} e^{-ipx} dx.
\]
Then we check that
\[
h'(p) = -i \int_{\mathbb{R}} x e^{-x^2/2} e^{-ipx} dx = i\int_{\mathbb{R}} \frac{d}{dx} \left( e^{-x^2/2} \right)
e^{-ipx} dx.
\]
Integrating by parts yields
\[
h'(p) = -p \int_{\mathbb{R}} e^{-x^2/2} e^{-ipx} dx = -p h(p).
\]
Since $h'(p)=-p h(p)$,\footnote{cf. Einar Hille, {\em Ordinary Differential Equations in the Complex Domain}.}
\[
h(p) = h(0) e^{-p^2/2}.
\]
Now, using Fubini's theorem and then polar coordinates,
\begin{align*}
h(0)^2 &= \int_{\mathbb{R}} \left( \int_{\mathbb{R}} e^{-x^2/2} dx\right) e^{-y^2/2} dy\\
&=\int_{\mathbb{R}^2} e^{-(x^2+y^2)/2} dxdy\\
&=\int_0^\infty \left( \int_{S^1} e^{-r^2/2} e^{-r^2/2} d\sigma(\theta) \right) r dr\\
&=2\pi \int_0^\infty r e^{-r^2/2} dr\\
&=2\pi,
\end{align*}
so
\[
h(p) = (2\pi)^{1/2} e^{-p^2/2}.
\]
For $a>0$ and $p \in \mathbb{C}$, doing the change of variable $y = a^{1/2} x$,
\begin{align*}
(2\pi)^{-1/2} \int_{\mathbb{R}} e^{-ax^2/2} e^{-ipx} dx&=(2\pi)^{-1/2} a^{-1/2} \int_{\mathbb{R}} e^{-y^2/2} e^{-ipa^{-1/2} y} dy\\
&=(2\pi)^{-1/2} a^{-1/2} h(pa^{-1/2})\\
&=a^{-1/2} e^{-a^{-1} p^2 /2}.
\end{align*}

For $t>0$ and $m \in \mathbb{R}$, doing the change of variable $y=x-m$, and using the above with $a=t^{-2}$ and $p=0$,
\begin{align*}
(2\pi t^2)^{-1/2} \int_{\mathbb{R}} \exp\left(-\frac{(x-m)^2}{2t^2}\right) dx&=(2\pi)^{-1/2}  t^{-1} \int_{\mathbb{R}} e^{-ay^2/2} dx\\
&=t^{-1} \cdot a^{-1/2}\\
&=1.
\end{align*}

\begin{theorem}
For $a > 0$ and $p \in \mathbb{C}$,
\[
(2\pi)^{-1/2} \int_{\mathbb{R}} e^{-ax^2/2} e^{-ipx} dx = a^{-1/2} e^{-a^{-1} p^2 /2}.
\]
For $t>0$ and $m \in \mathbb{R}$,
\[
(2\pi t^2)^{-1/2} \int_{\mathbb{R}} \exp\left(-\frac{(x-m)^2}{2t^2}\right) dx = 1.
\]
\label{gaussian1d}
\end{theorem}

For $t>0$ and $x \in \mathbb{R}$, let
\[
p_t(x) = (2\pi t^2)^{-1/2} \exp\left( - \frac{x^2}{2t^2}\right).
\]
For $\phi \in \mathscr{S}(\mathbb{R})$,
doing the change of variable $x=ty$,
\[
\int_{\mathbb{R}} \phi(x) p_t(x) dx = (2\pi)^{-1/2} \int_{\mathbb{R}} \phi(ty) e^{-y^2/2} dy
=\int_{\mathbb{R}} \phi(tx) p_1(x) dx.
\]
Then as $t \downarrow 0$, using the dominated convergence theorem,
\[
\int_{\mathbb{R}} \phi(x) p_t(x) dx \to \int_{\mathbb{R}} \phi(0) p_1(x) dx = \phi(0).
\]

For $\phi \in L^1(\mathbb{R}^N)$, let
\[
\widehat{\phi}(\xi) = (2\pi)^{-N/2} \int_{\mathbb{R}^N} e^{-i\inner{\xi}{x}} \phi(x) dx,\qquad
\xi \in \mathbb{R}^N.
\]
By Theorem \ref{gaussian1d}, with $a=t^{-2}$,
\begin{align*}
\widehat{p}_t(\xi) &= (2\pi t^2)^{-1/2} \cdot (2\pi)^{-1/2}  \int_{\mathbb{R}} e^{-ax^2/2} e^{-i\xi x}  dx\\
&=(2\pi t^2)^{-1/2} a^{-1/2} e^{-a^{-1} \xi^2/2}\\
&=(2\pi)^{-1/2} e^{-t^2 \xi^2/2}.
\end{align*}



\section{Moments}
For $a>0$, define for $Z \in \mathbb{C}$,
\[
Z(J) = a^{1/2} (2\pi)^{-1/2} \int_{\mathbb{R}} e^{-ax^2/2} e^{iJx} dx.
\]
By Theorem \ref{gaussian1d},
\begin{equation}
Z(J) = e^{-a^{-1} J^2/2}.
\label{ZJ}
\end{equation}
By the dominated convergence theorem,
\[
Z^{(n)}(J) = a^{1/2} (2\pi)^{-1/2} i^n  \int_{\mathbb{R}} x^n e^{-ax^2/2} e^{iJx} dx,
\]
and so 
\[
a^{1/2} (2\pi)^{-1/2} \int_{\mathbb{R}} x^n e^{-ax^2/2} dx = i^{-n} \frac{dZ}{dJ}(0).
\]
From \eqref{ZJ} 
we calculate
\[
Z'(J) = -a^{-1}J Z(J),
\quad Z''(J) = -a^{-1} Z(J) + a^{-2} J^2 Z(J),
\]
so $Z''(0)=-a^{-1} Z(0) = -a^{-1}$, and thus for $t>0$ and $a=t^{-1}$,
\[
(2\pi t)^{-1/2} \int_{\mathbb{R}} x^2 e^{-t^{-2} x^2/2} dx = t,
\]
i.e.
\[
\int_{\mathbb{R}} x^2 p_t(x) dx = t.
\]


\section{$N$-dimensional Gaussian integrals}
Let $S(x) = \frac{\inner{x}{x}}{2}$ for $x \in \mathbb{R}^N$. 
For $\chi \in \mathscr{D}(\mathbb{R}^N)$ and $t>0$,
\textbf{Laplace's method}\footnote{cf.
\url{http://individual.utoronto.ca/jordanbell/notes/stationaryphase.pdf}, p.~5, \S 5.}
tells us that 
\[
\int_{\mathbb{R}^N} e^{-tS(x)} \chi(x) dx 
= (2\pi t^{-1})^{N/2} (\det \Hess S(0))^{-1/2} e^{-tS(0)} \chi(0) (1+O(t^{-1}))
\]
as $t \to \infty$. 
Here, $\Hess S(x) = I$ for all $x$ and $S(0)=0$, so
\[
\int_{\mathbb{R}^N} e^{-t \frac{\inner{x}{x}}{2}} \chi(x) dx = (2\pi t^{-1})^{N/2} \chi(0) (1+O(t^{-1}))
\]
as $t \to \infty$. 

For $A$ an $N \times N$ matrix, we write $A>0$ if $A$ is symmetric and has positive eigenvalues.
 It
 is proved that\footnote{\url{http://individual.utoronto.ca/jordanbell/notes/stationaryphase.pdf},
Theorem 2 and Theorem 3.}
\[
\begin{split}
&\int_{\mathbb{R}^N} \exp\left(-\frac{1}{2} \inner{Ax}{x} - i\inner{\xi}{x} \right) dx\\
=&(\det A)^{-1/2} (2\pi)^{N/2} \exp\left( -\frac{1}{2} \inner{A^{-1} \xi}{\xi} \right)
\end{split}
\]
for all $\xi \in \mathbb{R}^N$, and 
\[
\int_{\mathbb{R}^N} \exp\left(-\frac{1}{2}\inner{Ax}{x}+\inner{b}{x}\right) dx = (\det A)^{-1/2} (2\pi)^{N/2}
\exp\left(\frac{1}{2}\inner{A^{-1}b}{b}\right).
\]
for all $b \in \mathbb{R}^N$. 
Let
\[
Z_A = \int_{\mathbb{R}^N} e^{-\frac{1}{2}\inner{Ax}{x}} dx
=(\det A)^{-1/2} (2\pi)^{N/2}.
\]
Let $\lambda_N$ be Lebesgue measure on $\mathbb{R}^N$ and
let $\mu_A$ be the following Borel probability measure on $\mathbb{R}^N$:\footnote{See \url{http://individual.utoronto.ca/jordanbell/notes/gaussian.pdf},
p.~6, \S 4.}
\[
d\mu_A(x) = \frac{1}{Z_A} e^{-\frac{1}{2}\inner{Ax}{x}} d\lambda_N(x)
=(\det A)^{1/2} (2\pi)^{-N/2} e^{-\frac{1}{2}\inner{Ax}{x}} d\lambda_N(x).
\]
For $\xi \in \mathbb{R}^N$, 
\begin{align*}
\int_{\mathbb{R}^N} e^{-i\inner{\xi}{x}} d\mu_A(x)&=
(\det A)^{1/2} (2\pi)^{-N/2} \int_{\mathbb{R}^N} e^{-\frac{1}{2}\inner{Ax}{x}} e^{-i\inner{\xi}{x}} d\lambda_N(x)\\
&=(\det A)^{1/2} (2\pi)^{-N/2}  \cdot (\det A)^{-1/2} (2\pi)^{N/2} e^{-\frac{1}{2}\inner{A^{-1}\xi}{\xi}}\\
&=e^{-\frac{1}{2}\inner{A^{-1}\xi}{\xi}},
\end{align*}
and for $b \in \mathbb{R}^N$,
\begin{align*}
\int_{\mathbb{R}^N} e^{\inner{b}{x}} d\mu_A(x)&=
(\det A)^{1/2} (2\pi)^{-N/2}  \int_{\mathbb{R}^N} e^{\inner{b}{x}}
e^{-\frac{1}{2}\inner{Ax}{x}} d\lambda_N(x)\\
&=(\det A)^{1/2} (2\pi)^{-N/2}   \cdot
(\det A)^{-1/2} (2\pi)^{N/2}
e^{\frac{1}{2}\inner{A^{-1}b}{b}}\\
&=e^{\frac{1}{2}\inner{A^{-1}b}{b}}.
\end{align*}

\begin{theorem}
For $\xi \in \mathbb{R}^N$,
\[
\int_{\mathbb{R}^N} e^{-i\inner{\xi}{x}} d\mu_A(x) = e^{-\frac{1}{2}\inner{A^{-1}\xi}{\xi}},
\]
and for $b \in \mathbb{R}^N$,
\[
\int_{\mathbb{R}^N} e^{\inner{b}{x}} d\mu_A(x) = e^{\frac{1}{2}\inner{A^{-1}b}{b}}.
\]
\label{gaussianRd}
\end{theorem}


Let\footnote{See \url{http://www.math.ucsd.edu/~bdriver/247A-Winter2012/}}
\[
L=L^A = \sum_{j,k=1}^N A_{j,k}^{-1} \partial_j \partial_k.
\]
We work out the semigroup whose infinitesimal generator is $L/2$.\footnote{cf. the Ornstein-Uhlenbeck semigroup,
\url{http://individual.utoronto.ca/jordanbell/notes/gaussian.pdf}, \S 6.}

\begin{theorem}
For $f \in C^1(\mathbb{R}^N)$ that is $\mu_A$-integrable and for $t>0$,
\[
(e^{t L/2} f)(x) = \int_{\mathbb{R}^N} f(x-t^{1/2} y) d\mu_A(y),
\qquad x \in \mathbb{R}^N.
\]
\end{theorem}
\begin{proof}
For $\xi \in \mathbb{R}^N$ define
 $f(x) = e^{\inner{\xi}{x}}=e^{\xi_1 x_1 + \cdots + \xi_N x_N}$. On the one hand,
\[
Lf=\sum_{j,k=1}^n A_{j,k}^{-1} \xi_j \xi_k f
= \inner{A^{-1}\xi}{\xi} f.
\]
Then
\[
\exp(tL/2) f = \exp\left( \frac{1}{2}t \inner{A^{-1}\xi}{\xi} \right) f.
\]
On the other hand, for $x \in \mathbb{R}^N$, applying Theorem \ref{gaussianRd},
\begin{align*}
\int_{\mathbb{R}^N} f(x-t^{1/2}y) d\mu_A(y)&=\int_{\mathbb{R}^N}
e^{\inner{\xi}{x-t^{1/2}y}} d\mu_A(y)\\
&=e^{\inner{\lambda}{x}} \int_{\mathbb{R}^N} e^{\inner{-t^{1/2}\xi}{y}} d\mu_A(y)\\
&=e^{\inner{\lambda}{x}} e^{\frac{1}{2}\inner{A^{-1} (-t^{1/2} \xi)}{(-t^{1/2}\xi)}}\\
&= e^{\frac{1}{2}t\inner{A^{-1}\xi}{\xi}} f.
\end{align*}
Therefore
\[
\int_{\mathbb{R}^N} f(x-t^{1/2}y) d\mu_A(y) = e^{tL/2} f.
\]
\end{proof}



\section{Concentration of measure}
Let $\gamma_N$ be the Borel probability measure on $\mathbb{R}^N$ defined by
\[
d\gamma_N(x) = (2\pi)^{-N/2} e^{-\frac{1}{2} \inner{x}{x}} d\lambda_N(x).
\]
We estimate the mass $\gamma_N$ assigns to a spherical shell about the sphere of radius $N^{1/2}$.\footnote{Alexander
Barvinok, {\em Measure Concentration}, \url{http://www.math.lsa.umich.edu/~barvinok/total710.pdf},
p.~5, Proposition 2.2.}


\begin{theorem}
For $\delta \geq 0$,
\[
\gamma_N\{x \in \mathbb{R}^N: \norm{x}^2 \geq N+\delta \} \leq \left( \frac{N}{N+\delta} \right)^{-N/2} e^{-\delta/2},
\]
and for $0<\delta \leq N$,
\[
\gamma_N\{x \in \mathbb{R}^N: \norm{x}^2 \leq N-\delta\} \leq \left(\frac{N}{N-\delta} \right)^{-N/2} e^{\delta/2}.
\]
\end{theorem}
\begin{proof}
For $0<\lambda<1$, if $\norm{x}^2 \geq N+\delta$ then $\lambda \norm{x}^2/2 \geq \lambda(N+\delta)/2$
and then $e^{\lambda \norm{x}^2/2} \geq e^{\lambda(N+\delta)/2}$.
Hence
\begin{align*}
\gamma_N\{x \in \mathbb{R}^N: \norm{x}^2 \geq N+\delta\}&=e^{-\lambda(N+\delta)/2} \int_{\norm{x}^2 \geq N+\delta} e^{\lambda(N+\delta)/2} d\gamma_N(x)\\
&\leq e^{-\lambda(N+\delta)/2} \int_{\norm{x}^2 \geq N+\delta} e^{\lambda \norm{x}^2/2} d\gamma_N(x)\\
&\leq e^{-\lambda(N+\delta)/2} \int_{\mathbb{R}^N} e^{\lambda \norm{x}^2/2} d\gamma_N(x)\\
&=  e^{-\lambda(N+\delta)/2} 
\cdot  (2\pi)^{-N/2}
\int_{\mathbb{R}^N} e^{\lambda \norm{x}^2/2}    e^{-\frac{1}{2}\norm{x}^2} d\lambda_N(x)\\
&=  e^{-\lambda(N+\delta)/2}  \cdot \prod_{k=1}^N (2\pi)^{-1/2} \int_{\mathbb{R}} e^{(\lambda-1) u^2/2} du.
\end{align*}
For $a=-\lambda+1>0$, we have by Theorem \ref{gaussian1d}
\[
(2\pi)^{-1/2} \int_{\mathbb{R}}  e^{-au^2/2} du = a^{-1/2},
\]
so
\[
\gamma_N\{x \in \mathbb{R}^N: \norm{x}^2 \geq N+\delta\} \leq
e^{-\lambda(N+\delta)/2}  a^{-N/2} = e^{-\lambda(N+\delta)/2}   (1-\lambda)^{-N/2}.
\]
For $\lambda=\frac{\delta}{N+\delta}$ this is
\[
\gamma_N\{x \in \mathbb{R}^N: \norm{x}^2 \geq N+\delta\} \leq e^{-\delta/2} \left( \frac{N}{N+\delta} \right)^{-N/2}.
\]
\end{proof}


Let $\Sigma_N=\{x \in \mathbb{R}^N: \norm{x} = N^{1/2}\}$, and let $\mu_N$ be the unique
$SO(N)$-invariant  Borel probability measure on $S^{N-1}$ (any Borel probability measure on a metric space is regular so we need not explicitly demand this to ensure uniqueness).\footnote{\url{http://individual.utoronto.ca/jordanbell/notes/harmonicpolynomials.pdf},
\S 2.}
Let $\pi_N:\Sigma_N \to \mathbb{R}$ be the projection 
\[
\pi_N(x) = \pi_N(x_1,\ldots,x_N) = x_1,
\]
and let $\nu_N=(\pi_N)_* \mu_N$, the pushforward measure which is itself a Borel probability measure on $\mathbb{R}$. 
The following theorem states that the measures $\nu_N$ converges strongly to the standard Gaussian measure $\gamma_1$.\footnote{Alexander
Barvinok, {\em Measure Concentration}, \url{http://www.math.lsa.umich.edu/~barvinok/total710.pdf},
p.~54, Theorem 13.2.}

\begin{theorem}
For $A$ a Borel set in $\mathbb{R}$,
\[
\nu_N(A) \to \gamma_1(A)
\]
as $N \to \infty$. 
\end{theorem}






\section{Zeta functions}
Let $A>0$, with eigenvalues
 $\lambda_1,
\ldots,\lambda_N$, counted according to multiplicity.
For $s \in \mathbb{C}$, define\footnote{Eberhard Zeidler,
{\em Quantum Field Theory I: Basics in Mathematics and Physics},
p.~434, \S 7.23.3.}
\[
\zeta_A(s) = \sum_{k=1}^N \lambda_k^{-s} = \sum_{k=1}^N e^{-s \log \lambda_k}.
\]
The derivative of $\zeta_A$ is
\[
\zeta_A'(s) = \sum_{k=1}^N -\log \lambda_k \cdot \lambda_k^{-s},
\]
so
\[
\zeta_A'(0) = - \sum_{k=1}^N \log \lambda_k,
\]
hence
\[
e^{-\zeta_A'(0)} = \prod_{k=1}^N \lambda_k = \det A.
\]

\begin{theorem}
For $\xi \in \mathbb{R}^N$,
\[
(2\pi)^{-N/2} \int_{\mathbb{R}^N}  \exp\left(-\frac{1}{2} \inner{Ax}{x} - i\inner{\xi}{x} \right) dx
=e^{\zeta_A'(0)/2}  \exp\left( -\frac{1}{2} \inner{A^{-1} \xi}{\xi} \right).
\]
\end{theorem}

Let $\lambda_k>0$, $k \geq 1$, and
let $Ae_k = \lambda_k e_k$, and if it makes sense let\footnote{cf. \url{http://individual.utoronto.ca/jordanbell/notes/fredholm.pdf}}
\[
\det A = \prod_{k=1}^\infty \lambda_k.
\]
For those complex $s$ for which the expression makes sense,
let\footnote{cf. \url{http://individual.utoronto.ca/jordanbell/notes/functionaldeterminant.pdf}}
\[
\zeta_A(s) = \sum_{k=1}^\infty \lambda_k^{-s} = \sum_{k=1}^\infty e^{-s \log \lambda_k}.
\]
Then, if the above makes sense in a neighborhood of $s=0$,
\[
\zeta_A'(0) =  - \sum_{k=1}^\infty \log \lambda_k,
\]
so
\[
e^{-\zeta_A'(0)} = \det A.
\]

We calculate, doing the change of variables $t=\lambda_k u$,
\begin{align*}
\Gamma(s) \zeta_A(s)&=\int_0^\infty t^{s-1} e^{-t} dt \cdot \sum_{k=1}^\infty \lambda_k^{-s}\\
&=\sum_{k=1}^\infty  \lambda_k^{-s}  \int_0^\infty t^{s-1} e^{-t}dt\\
&=\sum_{k=1}^\infty \lambda_k^{-s} \int_0^\infty (\lambda_k u)^{s-1} e^{-\lambda_k u} \lambda_k du\\
&=\sum_{k=1}^\infty \int_0^\infty u^{s-1} e^{-\lambda_k u} du.
\end{align*}
Thus
\[
\zeta_A(s) = \frac{1}{\Gamma(s)} \int_0^\infty u^{s-1} \sum_{k=1}^\infty e^{-\lambda_k u} du.
\]
For $\gamma>0$, the eigenvalues of $\gamma A$ are $\gamma \lambda_k$, and doing the change of variables
$v=\gamma u$,
\begin{align*}
\zeta_{\gamma A}(s) &=\frac{1}{\Gamma(s)} \int_0^\infty u^{s-1} \sum_{k=1}^\infty e^{-\gamma \lambda_k u} du\\
&= \frac{1}{\Gamma(s)} \int_0^\infty \gamma^{-s} v^{s-1} \sum_{k=1}^\infty e^{-\lambda_k v} dv\\
&=\gamma^{-s} \zeta_A(s).
\end{align*}
Taking the derivative,
\[
\zeta_{\gamma A}'(s) = -\log \gamma \cdot \gamma^{-s} \cdot \zeta_A(s) + \gamma^{-s} \gamma_A'(s),
\]
and then
\[
\zeta_{\gamma A}'(0) = -\log \gamma \cdot \zeta_A(0) + \zeta_A'(0). 
\]
Then
\[
\det (\gamma A) = e^{-\zeta_{\gamma A}'(0)} = e^{\log \gamma \cdot \zeta_A(0) - \zeta_A'(0)}
=\gamma^{\zeta_A(0)} \det A.
\]

\end{document}