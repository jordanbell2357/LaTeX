\documentclass{article}
\usepackage{amsmath,amssymb,mathrsfs,amsthm}
%\usepackage{tikz-cd}
\usepackage{hyperref}
\newcommand{\inner}[2]{\left\langle #1, #2 \right\rangle}
\newcommand{\tr}{\ensuremath\mathrm{tr}\,} 
\newcommand{\Span}{\ensuremath\mathrm{span}} 
\def\Re{\ensuremath{\mathrm{Re}}\,}
\def\Im{\ensuremath{\mathrm{Im}}\,}
\newcommand{\id}{\ensuremath\mathrm{id}} 
\newcommand{\var}{\ensuremath\mathrm{var}} 
\newcommand{\Lip}{\ensuremath\mathrm{Lip}} 
\newcommand{\Sh}{\ensuremath\mathrm{Sh}} 
\newcommand{\GL}{\ensuremath\mathrm{GL}} 
\newcommand{\diam}{\ensuremath\mathrm{diam}} 
\newcommand{\sgn}{\ensuremath\mathrm{sgn}}
\newcommand{\Var}{\ensuremath\mathrm{Var}} 
\newcommand{\lcm}{\ensuremath\mathrm{lcm}} 
\newcommand{\cl}{\ensuremath\mathrm{cl}} 
\newcommand{\supp}{\ensuremath\mathrm{supp}\,}
\newcommand{\dom}{\ensuremath\mathrm{dom}\,}
\newcommand{\arctanh}{\ensuremath\mathrm{arctanh}\,}
\newcommand{\upto}{\nearrow}
\newcommand{\downto}{\searrow}
\newcommand{\norm}[1]{\left\Vert #1 \right\Vert}
\newtheorem{theorem}{Theorem}
\newtheorem{lemma}[theorem]{Lemma}
\newtheorem{proposition}[theorem]{Proposition}
\newtheorem{corollary}[theorem]{Corollary}
\theoremstyle{definition}
\newtheorem{definition}[theorem]{Definition}
\newtheorem{example}[theorem]{Example}
\begin{document}
\title{Varadhan's lemma for large deviations}
\author{Jordan Bell\\ \texttt{jordan.bell@gmail.com}\\Department of Mathematics, University of Toronto}
\date{\today}

\maketitle


\section{Large deviation principles}
Let $(\mathcal{X},d)$ be a Polish space.
A function
$f:\mathcal{X} \to [-\infty,\infty]$ is called \textbf{lower semicontinuous} if  $c \in \mathbb{R}$ implies that
$f^{-1}(c,\infty]$ is an open set in $\mathcal{X}$;
equivalently, $c \in \mathbb{R}$ implies that $f^{-1}[-\infty,c]$ is a closed set in $\mathcal{X}$;
equivalently,
for any convergent sequence $x_n$ in $\mathcal{X}$,\footnote{\url{http://individual.utoronto.ca/jordanbell/notes/semicontinuous.pdf},
p.~2, Theorem 3.}
\[
f(\lim_{n \to \infty} x_n) \leq \liminf_n f(x_n).
\]
If $K$ is a nonempty compact subset of $\mathcal{X}$ and $f:\mathcal{X} \to [-\infty,\infty]$ is lower semicontinuous, 
there is some $x_0 \in K$ such that $f(x_0) = \inf_{x \in K} f(x)$.\footnote{\url{http://individual.utoronto.ca/jordanbell/notes/semicontinuous.pdf},
p.~5, Theorem 8.}

A \textbf{rate function} is a function $I:\mathcal{X} \to [0,\infty]$ such that (i)
there is some $x \in \mathcal{X}$ with $I(x) < \infty$ and (ii)
for any $c \in [0,\infty]$, $f^{-1}([0,c])$ is compact (namely, the sublevel sets of $I$ are compact).
Condition (ii) implies that a rate function is lower semicontinuous.

For $S \in \mathscr{B}_\mathcal{X}$, the Borel $\sigma$-algebra of $\mathcal{X}$, we define
\[
I(S) = \inf_{x \in S} I(x).
\]

Denote by $\mathscr{P}(\mathcal{X})$ the collection of Borel probability measures on $\mathcal{X}$ and let $I$
be a rate function. A sequence
$P_n \in \mathscr{P}(\mathcal{X})$  is said to \textbf{satisfy a large deviation principle with rate $n$ and   with rate function $I$}
if for any closed subset $C$ of $\mathcal{X}$,
\[
\limsup_{n \to \infty} \frac{1}{n} \log P_n(C) \leq -I(C) 
\]
and for any open subset $U$ of $\mathcal{X}$,
\[
\liminf_{n \to \infty} \frac{1}{n} \log P_n(U) \geq -I(U).
\]
Unless we say otherwise, when we speak about a large deviation principle we shall use the rate $n$.
Because $P_n(\mathcal{X})=1$ for each $n$, it follows that $I(\mathcal{X})=0$ and thus that $\inf_{x \in \mathcal{X}} I(x)=0$.

We first establish that if a sequence of Borel probability measures satisfies a large deviation principle then its rate function is unique.\footnote{Frank
den Hollander, {\em Large Deviations}, p.~30,Theorem III.8.}

\begin{lemma}
If $P_n$ satisfy a large deviation principle with rate functions $I$ and $J$, then $I=J$.
\end{lemma}
\begin{proof}
Let $x \in \mathcal{X}$ and let $B_N = B_{1/N}(x)$, the open ball with center $x$ and radius $1/N$. 
Then, because $I(B_{N+1}) \leq I(\cl (B_{N+1}))$ and because $\cl(B_{N+1}) \subset B_N$ and so
$I(\cl(B_{N+1})) \leq I(B_N)$,
\begin{align*}
-I(x)&\leq -I(B_{N+1})\\
&\leq \liminf_{n \to \infty} \frac{1}{n} \log P_n(B_{N+1})\\
&\leq -J(\cl(B_{N+1}))\\
&\leq -J(B_N).
\end{align*}
Because $J$ is lower semicontinuous, as $N \to \infty$,
\[
J(x) \leq \liminf_{N \to \infty} J(B_N),
\]
hence
\[
-I(x) \leq -J(x),
\]
i.e. $J(x) \leq I(x)$. Likewise we obtain $I(x) \leq J(x)$, so $I(x)=J(x)$, for any $x \in \mathcal{X}$.
\end{proof}

\section{Varadhan's lemma}
For sequences $\alpha_n$ and $\beta_n$ of positive real numbers, 
we write 
\[
\alpha_n \simeq \beta_n
\]
if
\[
\lim_{n \to \infty} \frac{1}{n}(\log \alpha_n - \log \beta_n) = 0.
\]

\begin{lemma}
$\alpha_n+\beta_n \simeq \alpha \vee \beta_n$.
\label{equivalent}
\end{lemma}



\begin{lemma}
If $A$ is a set and $f,g:A \to [-\infty,\infty]$ are functions, then
\[
\inf_A f - \inf_A g  \leq \sup_A (f-g).
\]
\label{infsup}
\end{lemma}
\begin{proof}
\[
\inf_A g = \inf_A (g-f+f) \geq \inf_A (g-f) + \inf_A f,
\]
hence
\[
\inf_A f - \inf_A g \leq -\inf_A (g-f) = \sup_A (f-g).
\]
\end{proof}


We now prove \textbf{Varadhan's lemma}.\footnote{Frank den Hollander, {\em Large Deviations}, p.~32, Theorem III.13.}

\begin{theorem}[Varadhan's lemma]
Suppose that $P_n$ satisfies a large deviation principle with rate function $I$. If $F:\mathcal{X}
\to \mathbb{R}$ is continuous and bounded  above, then
\[
\lim_{n \to \infty} \frac{1}{n} \log \int_\mathcal{X} e^{nF(x)} dP_n(x)
=\sup_{x \in \mathcal{X}} (F(x)-I(x)).
\]
\end{theorem}
\begin{proof}
Let
\[
b=\sup_{x \in \mathcal{X}} F(x), \qquad a = \sup_{x \in \mathcal{X}} (F(x)-I(x)).
\]
Because $F$ is bounded  above, $-\infty<b<\infty$.
Because $I$ is not identically $\infty$,
$-\infty<a \leq b$. 
For $n \geq 1$ and
$S \in \mathscr{B}_{\mathcal{X}}$, 
define
\[
J_n(S) = \int_S e^{nF(x)} dP_n(x).
\]

Let $C=F^{-1}([a,b])$. For $N \geq 1$ and $0 \leq j \leq N$, define
\[
c_{N,j} = a + \frac{j}{N}(b-a),
\]
for which
\[
[a,b] = \bigcup_{j=1}^N [c_{N,j-1},c_{N,j}].
\]
For $1 \leq j \leq N$, define
\[
C_{N,j} = F^{-1}([c_{N,j-1},c_{N,j}]),
\]
for which
\[
C=F^{-1}([a,b]) = \bigcup_{j=1}^N C_{N,j}.
\]
Because $F$ is continuous, each $C_{N,j}$ is closed, and hence applying the large deviation principle,
\[
\limsup_{n \to \infty} \frac{1}{n} \log P_n(C_{N,j}) \leq -I(C_{N,j}).
\]
For $x \in C_{N,j}$, $F(x) \leq c_{N,j}$, and hence, for each $N$,
\[
J_n(C) = \int_C e^{nF(x)} dP_n(x)
\leq \sum_{j=1}^N \int_{C_{N,j}} e^{nF(x)} dP_n(x)
\leq \sum_{j=1}^N e^{nc_{N,j}} P_n(C_{N,j}).
\]
Applying Lemma \ref{equivalent},
\[
\lim_{n \to \infty} \frac{1}{n} \left(\log\left(\sum_{j=1}^N e^{nc_{N,j}} P_n(C_{N,j})\right)
-\log\left(\bigvee_{j=1}^n e^{nc_{N,j}} P_n(C_{N,j}) \right) \right)=0,
\]
i.e.
\[
\lim_{n \to \infty} \frac{1}{n} \left( \log\left(\sum_{j=1}^N e^{nc_{N,j}} P_n(C_{N,j})\right)
-\bigvee_{j=1}^N \log(e^{nc_{N,j}} P_n(C_{N,j})) \right)=0.
\]
Then applying the large deviation principle,
\begin{align*}
\limsup_{n \to \infty} \frac{1}{n}\log J_n(C)&\leq \limsup_{n \to \infty} \frac{1}{n} \bigvee_{j=1}^N \left(nc_{N,j}
+\log P_n(C_{N,j})\right)\\
&\leq \bigvee_{j=1}^N \left( c_{N,j} + \limsup_{n \to \infty} \frac{1}{n}  \log P_n(C_{N,j})\right)\\
&\leq  \bigvee_{j=1}^N \left( c_{N,j}  - I(C_{N,j})\right).
\end{align*}
For $x \in C_{N,j}$, $F(x) \geq c_{N,j-1}$, and as $c_{N,j} = c_{N,j-1}+\frac{1}{N}(b-a)$,
\[
c_{N,j} \leq \inf_{x \in C_{N,j}} F(x) + \frac{1}{N}(b-a),
\]
and using this and Lemma \ref{infsup} we get
\begin{align*}
\limsup_{n \to \infty} \frac{1}{n}\log J_n(C)& \leq   \bigvee_{j=1}^N \left( \inf_{x \in C_{N,j}} F(x) + \frac{1}{N}(b-a)
-I(C_{N,j})\right)\\
&=\frac{1}{N}(b-a)+ \bigvee_{j=1}^N \left( \inf_{x \in C_{N,j}} F(x) -  \inf_{x \in C_{N,j}} I(x) \right)\\
&\leq \frac{1}{N}(b-a)+ \bigvee_{j=1}^N \sup_{x \in C_{N,j}} (F(x)-I(x))\\
&= \frac{1}{N}(b-a)+\sup_{x \in C} (F(x)-I(x))\\
&\leq \frac{1}{N}(b-a)+a.
\end{align*}
Because this is true for all $N$,
\[
\limsup_{n \to \infty} \frac{1}{n}\log J_n(C) \leq a.
\]
On the other hand, for $x \in \mathcal{X} \setminus C$ we have 
$F(x)<a$ and hence
\[
J_n(\mathcal{X} \setminus C) = 
\int_{\mathcal{X} \setminus C} e^{nF(x)} dP_n(x)
\leq e^{na},
\]
so
\[
\limsup_{n \to \infty} \frac{1}{n} \log J_n(\mathcal{X} \setminus C)
\leq a.
\]
Lemma \ref{equivalent} tells us that $J_n(C)+J_n(\mathcal{X}\setminus C) \simeq 
J_n(C) \vee J_n(\mathcal{X} \setminus C)$:
\[
\lim_{n \to \infty} \frac{1}{n} (\log(J_n(C)+J_n(\mathcal{X}\setminus C)) - \log(J_n(C) \vee J_n(\mathcal{X} \setminus C))),
\]
whence
\[
\limsup_{n \to \infty} \frac{1}{n} \log J_n(\mathcal{X}) = \limsup_{n \to \infty} \frac{1}{n}
 \log(J_n(C) \vee J_n(\mathcal{X} \setminus C))
 \leq a,
\] 
i.e.
\[
\limsup_{n \to \infty} \frac{1}{n} \log \int_\mathcal{X} e^{nF(x)} dP_n(x) \leq \sup_{x \in \mathcal{X}} (F(x)-I(x)).
\]

Let $x \in \mathcal{X}$ and let $\epsilon>0$, and define
\[
U_{x,\epsilon} = \{y \in \mathcal{X}: F(y)>F(x)-\epsilon\}
=F^{-1}(F(x)-\epsilon,\infty),
\]
which is an open subset of $\mathcal{X}$ because $F$ is continuous. 
Applying the large deviation principle,
\[
\liminf_{n \to \infty} \frac{1}{n} \log P_n(U_{x,\epsilon}) \geq -I(U_{x,\epsilon}).
\]
Because $x \in U_{x,\epsilon}$, $I(U_{x,\epsilon}) \leq I(x)$,
\[
\liminf_{n \to \infty} \frac{1}{n} \log P_n(U_{x,\epsilon}) \geq -I(x).
\]
Together with
\[
J_n(U_{x,\epsilon}) = \int_{U_{x,\epsilon}} e^{nF(y)} dP_n(y)
\geq \int_{U_{x,\epsilon}} e^{n(F(x)-\epsilon)} dP_n(y)
= e^{n(F(x)-\epsilon)} P_n(U_{x,\epsilon}),
\]
this yields
\begin{align*}
\liminf_{n \to \infty} \frac{1}{n} \log J_n(\mathcal{X})&\geq \liminf_{n \to \infty} \frac{1}{n} \log J_n(U_{x,\epsilon})\\
&\geq \liminf_{n \to \infty} \frac{1}{n} \log \left(e^{n(F(x)-\epsilon)} P_n(U_{x,\epsilon})\right)\\
&=F(x)-\epsilon+\liminf_{n \to \infty} \frac{1}{n} \log P_n(U_{x,\epsilon})\\
&\geq F(x)-\epsilon-I(x).
\end{align*}
Because this is true for all $\epsilon>0$,
\[
\liminf_{n \to \infty} \frac{1}{n} \log J_n(\mathcal{X}) \geq F(x)-I(x).
\]
Because this is true for all $x \in \mathcal{X}$,
\[
\liminf_{n \to \infty} \frac{1}{n} \log J_n(\mathcal{X}) \geq \sup_{x \in \mathcal{X}} (F(x)-I(x)),
\]
i.e.,
\[
\liminf_{n \to \infty} \frac{1}{n} \log \int_\mathcal{X} e^{nF(x)} dP_n(x) \geq  \sup_{x \in \mathcal{X}} (F(x)-I(x)),
\]
which completes the proof.
\end{proof}


The following theorem states that if a sequence of Borel probability measures satisfies a large deviation principle then
the \textbf{tilted} sequence of Borel probability measures satisfies a large deviation principle with a 
\textbf{tilted} rate function.\footnote{Frank
den Hollander, {\em Large Deviations}, p.~34,Theorem III.17.}

\begin{theorem}
Suppose that $P_n \in \mathscr{P}(\mathcal{X})$ satisfy a large deviation principle with rate function
$I$, and let $F:\mathcal{X} \to \mathbb{R}$ be continuous and bounded  above.
Define for $n \geq 1$ and $S \in \mathscr{B}_{\mathcal{X}}$,
\[
J_n(S) = \int_S e^{nF(x)} dP_n(x).
\]
Then the sequence $P_n^F \in \mathscr{P}(\mathcal{X})$ defined by
\[
P_n^F(S) = \frac{J_n(S)}{J_n(\mathcal{X})}, \qquad S \in \mathscr{B}_{\mathcal{X}},
\]
satisfies a large deviation principle with rate function
\[
I^F(x) = \sup_{y \in \mathcal{X}} (F(y)-I(y)) - (F(x)-I(x)).
\]
\end{theorem}



\end{document}