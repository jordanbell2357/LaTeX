\documentclass{article}
\usepackage{amsmath,amssymb,graphicx,subfig,mathrsfs,amsthm}
\newtheorem{theorem}{Theorem}
\newcommand{\norm}[1]{\Vert #1 \Vert}
\newtheorem{remark}[theorem]{Remark}
\newtheorem{lemma}[theorem]{Lemma}
\newtheorem{prop}[theorem]{Proposition}
\begin{document}
\title{Arnold's theorem on analytic circle diffeomorphisms}
\author{Jordan Bell\\ \texttt{jordan.bell@gmail.com}\\Department of Mathematics, University of Toronto}

\date{\today}

\maketitle

We are reorganizing and expanding on the presentation in \cite{wayne}. Arnold's paper: \cite{malye}. Other sources that present the theorem, as I come across them: \cite{MR947141}, \cite{MR1230383}. 

Let $S_\sigma=\{z \in \mathbb{C}: |\Im z |< \sigma\}$,
let $\norm{\eta}_\sigma=\sup_{|\Im z|<\sigma} |\eta(z)|$, 
and let \[
B_\sigma=\{\eta: \eta \ \text{is holomorphic on} \ S_\sigma, \ \eta(x+1)=\eta(x), \ \text{and} \ \norm{\eta}_\sigma<\infty\}.
\]
For each $\sigma>0$, the set $B_\sigma$ is a Banach space with the norm $\norm{\cdot}_\sigma$.

We say that $\rho \in \mathbb{R}$ is of type $(K,\nu)$ if $|\rho-\frac{m}{n}|>K|n|^{-\nu}$ for all
$(m,n) \in \mathbb{Z}^2$ with $n \neq 0$.
If $\rho$ is of type $(K,\nu)$ with $K >1$ then $\rho$ is also of type $(1,\nu)$,  and thus we can assume that $\rho$ is of type $(K,\nu)$ with $K \leq 1$.
Suppose that $\rho \in \mathbb{R} \setminus \mathbb{Q}$. Let $\epsilon>0$, and let $K>0$. It follows from Dirichlet's approximation theorem \cite[p. 155, Theorem 185]{MR2445243} that there is some $(m,n) \in \mathbb{Z}^2$ with $n \geq \frac{1}{K^{1/\epsilon}}$ such that $|\rho - \frac{m}{n}| < \frac{1}{n^2}$. Then
\[
|\rho-\frac{m}{n}|<\frac{1}{n^2}=\frac{1}{n^{2-\epsilon}} \frac{1}{n^\epsilon}
\leq \frac{1}{n^{2-\epsilon}} K.
\]
Hence $\rho$ is not of type $(K,2-\epsilon)$. Therefore if $\rho \in \mathbb{R} \setminus \mathbb{Q}$ is of type $(K,\nu)$ then $\nu \geq 2$.

Let $F:\mathbb{R} \to \mathbb{R}$ be an increasing homeomorphism that satisfies $F(x+1)=F(x)+1$ for all $x \in \mathbb{R}$. The {\em rotation number} of $F$ is defined to be
\[
\lim_{n \to \infty} \frac{F^n(x)-x}{n}.
\]
This limit exists for all $x \in \mathbb{R}$ and is the same for all $x \in \mathbb{R}$ \cite[p. 387, Proposition 11.1.1]{katok}. If $H:\mathbb{R} \to \mathbb{R}$ is an increasing homeomorphism that satisfies $H(x+1)=H(x)+1$ for all $x \in \mathbb{R}$, then $H^{-1} \circ F \circ H$ has the same rotation number as $F$ \cite[p. 388, Proposition 11.1.3]{katok}.

\begin{prop}
\label{hestimate}
Let $\eta \in B_\sigma$, let $\rho \in \mathbb{R} \setminus \mathbb{Q}$ be of type $(K,\nu)$ and define $h$ by
$\widehat{h}(k)=\frac{\widehat{\eta}(k)}{e^{2\pi i k\rho}-1}$ for $k \neq 0$ and $\widehat{h}(0)=0$.
Then for any $0<\delta<\frac{1}{2\pi}$ we have
 $h \in B_{\sigma-\delta}$ and
\[
\norm{h}_{\sigma-\delta} < \frac{\Gamma(\nu)}{K(2\pi \delta)^\nu}\norm{\eta}_\sigma.
\]
\end{prop}
\begin{proof}

We first show that $|\widehat{\eta}(n)| \leq \norm{\eta}_\sigma e^{-2\pi \sigma|n|}$ for all $k$. Say $k>0$.
For any $\epsilon>0$ by the residue theorem we have
\[
\widehat{\eta}(k)=\int_0^1 e^{-2\pi i kz} \eta(z) dz = 
\int_0^{-i(\sigma-\epsilon)}+\int_{-i(\sigma-\epsilon)}^{1-i(\sigma-\epsilon)}
+\int_{1-i(\sigma-\epsilon)}^1.
\]

Since $\eta(z+1)=\eta(z)$,
\[
\int_{1-i(\sigma-\epsilon)}^1 e^{-2\pi i kz} \eta(z) dz
= \int_{-i(\sigma-\epsilon)}^0 e^{-2\pi ikz}\eta(z)dz
=-\int_0^{-i(\sigma-\epsilon)} e^{-2\pi ikz}\eta(z)dz,
\]
so
\[
\widehat{\eta}(k)=\int_{-i(\sigma-\epsilon)}^{1-i(\sigma-\epsilon)}  e^{-2\pi i kz} \eta(z) dz
= e^{-2\pi  k(\sigma-\epsilon)} \int_0^1 e^{-2\pi i kx} \eta(x-i(\sigma-\epsilon)) dx,
\]
and hence
\[
|\widehat{\eta}(k)| \leq e^{-2\pi  k(\sigma-\epsilon)} \norm{\eta}_\sigma.
\]
This is true for all $\epsilon>0$, so we have
\[
|\widehat{\eta}(k)| \leq e^{-2\pi  k\sigma} \norm{\eta}_\sigma.
\]

For $k<0$ we use a contour in the upper half-plane rather than a contour in the lower half-plane and
get $|\widehat{\eta}(k)| \leq e^{2\pi  k\sigma} \norm{\eta}_\sigma$, proving that $|\widehat{\eta}(k)| \leq \norm{\eta}_\sigma e^{-2\pi \sigma|k|}$ for all $k$.

Let $k \neq 0$, and let $m$ be such that $|\rho k-m|\leq \frac{1}{2}$. We have
\[
|e^{2\pi i \rho k}-1|
=|e^{2\pi i\rho k}-e^{2\pi im}|
=|e^{2\pi i(\rho k-m)}-1|
=2|\sin \pi(\rho k-m)|.
\]
For $|x| \leq \frac{\pi}{2}$ we have $|\sin x| \geq \frac{2}{\pi}|x|$, so
\[
2|\sin \pi(\rho k-m)| \geq 2 \frac{2}{\pi} |\pi(\rho k-m)|
=4|n||\rho-\frac{m}{k}|.
\]
But $\rho$ is of type $(K,\nu)$ so we get for all $k \neq 0$ that
\[
|e^{2\pi i \rho k}-1| \geq 4K|k|^{-(\nu-1)}.
\]

Let $\delta>0$.
If $|\Im z| \leq \sigma-\delta$ then 
\begin{eqnarray*}
|h(z)|&=&\Big| \sum_{k \neq 0}  e^{2\pi ikz} \frac{\widehat{\eta}(k)}{e^{2\pi i k\rho}-1} \Big|\\
&\leq&\sum_{k \neq 0} e^{2\pi |k|(\sigma-\delta)} \frac{e^{-2\pi  k\sigma} \norm{\eta}_\sigma}{4K|k|^{-(\nu-1)}}\\
&=&\frac{\norm{\eta}_\sigma}{4K} \sum_{k \neq 0} e^{-2\pi |k|\delta} |k|^{\nu-1}.
\end{eqnarray*}

One can check that $\frac{\Gamma(\nu)}{(2\pi \delta)^\nu}= \int_0^\infty y^{\nu-1} e^{-2\pi \delta y} dy$,
and because $2\pi\delta<\nu-1$ we have that for $y \geq 1$ the integrand is decreasing.
 Therefore, since $\nu \geq 2$,
 \[
\sum_{k \geq 1} e^{-2\pi k\delta} k^{\nu-1} \leq e^{-2\pi \delta} +\frac{\Gamma(\nu)}{(2\pi \delta)^\nu}
<2\frac{\Gamma(\nu)}{(2\pi \delta)^\nu},
 \]
 and so
 \[
  \sum_{k \neq 0} e^{-2\pi |k|\delta} |k|^{\nu-1} < 4\frac{\Gamma(\nu)}{(2\pi \delta)^\nu}.
 \]
 \end{proof}
 
 \begin{prop}
 \label{gestimate}
 Let $\eta \in B_\sigma$, let $\rho \in \mathbb{R} \setminus \mathbb{Q}$ be of type $(K,\nu)$, define $h$ by
$\widehat{h}(k)=\frac{\widehat{\eta}(k)}{e^{2\pi i k\rho}-1}$ for $k \neq 0$ and $\widehat{h}(0)=0$,
and let $H(z)=z+h(z)$.
If $0<\delta<\frac{1}{2\pi}$ and $\frac{2\pi \Gamma(\nu)}{K(2\pi \delta)^{\nu+1}}\norm{\eta}_{\sigma}<1$, then there is a holomorphic $H^{-1}:
S_{\sigma-3\delta} \to S_{\sigma-2\delta}$,
and $H^{-1}(z)=z-h(z)+g(z)$ with $g \in B_{\sigma-4\delta}$ and
\[
\norm{g}_{\sigma-4\delta} < \frac{2\pi \Gamma(\nu)^2}{K^2(2\pi \delta)^{2\nu+1}}\norm{\eta}_\sigma^2.
\]
  \end{prop}
 \begin{proof}
 By Proposition \ref{hestimate} we have $h \in B_{\sigma-\delta}$. 
 Using the maximum modulus principle and Cauchy's integral formula we get that
$\norm{h'}_{\sigma-2\delta} \leq \frac{\norm{h}_{\sigma-\delta}}{\delta}$, and by Proposition \ref{hestimate} this is $< \frac{2\pi \Gamma(\nu)}{K(2\pi \delta)^{\nu+1}}\norm{\eta}_{\sigma}$, which by hypothesis is $<1$ (and so $\norm{h}_{\sigma-\delta} <\delta$).
Then for $z \in
 S_{\sigma-2\delta}$ we have $|H'(z)|=|1+h'(z)|>0$, so by the inverse function theorem there is a holomorphic
 $H^{-1}:H( S_{\sigma-2\delta}) \to  S_{\sigma-2\delta}$.

Let $a \in S_{\sigma-3\delta}$, let $K \subset S_{\sigma-2\delta}$ be the circle about $a$ of radius $\delta$, and let $f(z)=z-a$. Then for $z \in K$ we have $|h(z)| <\delta=|f(z)|$, so by Rouch\'e's theorem $f$ and $f+h$ have the same number of zeros in the interior of $K$. Of course $f$ has one zero in the interior of $K$ so too $f+h=H-a$ has one zero in the interior of $K$. Thus $a \in H(S_{\sigma-2\delta})$, and we conclude $S_{\sigma-3\delta} \subseteq H(S_{\sigma-2\delta})$. Therefore $H^{-1}:
S_{\sigma-3\delta} \to S_{\sigma-2\delta}$.

Now define $g:S_{\sigma-3\delta} \to \mathbb{C}$ by $g(z)=H^{-1}(z)-z+h(z)$. For 
$z \in B_{\sigma-3\delta}$ we have
\begin{eqnarray*}
z+1&=&H(H^{-1}(z))+1\\
&=&H^{-1}(z)+1+h(H^{-1}(z))\\
&=&H^{-1}(z)+1+h(H^{-1}(z)+1)\\
&=&H(H^{-1}(z)+1),
\end{eqnarray*}
so $H^{-1}(z+1)=H^{-1}(z)+1$, from which it follows that $g(z+1)=g(z)$.

Let $a \in S_{\sigma-4\delta}$, and again using Rouch\'e's theorem we get $a \in H(S_{\sigma-3\delta})$. Hence $S_{\sigma-4\delta} \subseteq H(S_{\sigma-3\delta})$. Therefore if
$\xi \in S_{\sigma-4\delta}$ then $H^{-1}(\xi) \in S_{\sigma-3\delta}$.
Let $\xi \in S_{\sigma-4\delta}$ and let $z=H^{-1}(\xi) \in S_{\sigma-3\delta}$.
We have
\begin{eqnarray*}
z&=&H^{-1}(H(z))\\
&=&H(z)-h(H(z))+g(H(z))\\
&=&z+h(z)-h(z+h(z))+g(H(z)),
\end{eqnarray*}
and so
\[
g(\xi)=\int_0^1 h'(H^{-1}(\xi)+sh(H^{-1}(\xi))) h(H^{-1}(\xi)) ds.
\]
Because $\norm{h}_{\sigma-\delta} < \delta$, for $0 \leq s \leq 1$ we have $H^{-1}(\xi)+sh(H^{-1}(\xi)) \in B_{\sigma-2\delta}$. Then since
$\norm{h}_{\sigma-\delta} < \frac{\Gamma(\nu)}{K(2\pi \delta)^\nu}\norm{\eta}_\sigma$
and
$\norm{h'}_{\sigma-2\delta} < \frac{2\pi \Gamma(\nu)}{K(2\pi \delta)^{\nu+1}}\norm{\eta}_{\sigma}$, we get
\[
|g(\xi)| <  \frac{2\pi \Gamma(\nu)^2}{K^2(2\pi \delta)^{2\nu+1}}\norm{\eta}_\sigma^2,
\]
and so
\[
\norm{g}_{\sigma-4\delta} < \frac{2\pi \Gamma(\nu)^2}{K^2(2\pi \delta)^{2\nu+1}}\norm{\eta}_\sigma^2.
\]
 \end{proof}

\begin{prop}
\label{muestimate}
Let $\eta \in B_\sigma$, let $\rho \in \mathbb{R} \setminus \mathbb{Q}$ be of type $(K,\nu)$,  
let $\phi(z)=z+\rho+\eta(z)$, 
define $h$ by
$\widehat{h}(k)=\frac{\widehat{\eta}(k)}{e^{2\pi i k\rho}-1}$ for $k \neq 0$ and $\widehat{h}(0)=0$,
define $H(z)=z+h(z)$,
define $\psi(z)=H^{-1} \circ \phi \circ H(z)$, and define $\mu$ by $\psi(z)=z+\rho+\mu(z)$. If $\phi$ has rotation number $\rho$,
$0<\delta<\frac{1}{2\pi}$,
and
$\frac{2\pi \Gamma(\nu)}{K(2\pi \delta)^{\nu+1}}\norm{\eta}_{\sigma}<1$,
then $\mu \in B_{\sigma-6\delta}$ and
\[
\norm{\mu}_{\sigma-6\delta} < \frac{16\pi \Gamma(\nu)^2}{K^2(2\pi \delta)^{2\nu+1}} \norm{\eta}_\sigma^2.
\]
\end{prop}
\begin{proof}
For $z$ we have 
\begin{eqnarray*}
\psi(z)&=&H^{-1} \circ \phi (z+h(z))\\
&=&H^{-1}(z+h(z)+\rho+\eta(z+h(z)))\\
&=&z+h(z)+\rho+\eta(z+h(z))-h(z+h(z)+\rho+\eta(z+h(z)))\\
&&+g(z+h(z)+\rho+\eta(z+h(z)))\\
&=&z+\rho+\Big(h(z)-h(z+\rho)+\eta(z)\Big)+\Big(\eta(z+h(z))-\eta(z)\Big)\\
&&+\Big(h(z+\rho)-h(z+h(z)+\rho+\eta(z+h(z)))\Big)\\
&&+g(z+h(z)+\rho+\eta(z+h(z)))\\
&=&z+\rho+A(z)+B(z)+C(z)+D(z).
\end{eqnarray*}

We have
\[
\mu(z)=A(z)+B(z)+C(z)+D(z).
\]

First, $h(x+\rho)-h(x)=\eta(x)-\widehat{\eta}(0)$, so $A(z)=\widehat{\eta}(0)$.
Second,
\[
B(z)=\int_0^1 \eta'(z+sh(z))h(z)ds.
\]
Since $\frac{2\pi \Gamma(\nu)}{K(2\pi \delta)^{\nu+1}}\norm{\eta}_{\sigma}<1$ we have by Proposition \ref{hestimate}
that $\norm{h}_{\sigma-\delta}<\delta$, so for $z \in S_{\sigma-2\delta}$ and $0 \leq s \leq 1$ we get
$z+sh(z) \in S_{\sigma-\delta}$.  Using the maximum modulus principle and Cauchy's integral formula we get that $\norm{\eta'}_{\sigma-\delta} \leq \frac{\norm{\eta}_\sigma}{\delta}$. 
Therefore
\[
\norm{B}_{\sigma-2\delta}< \frac{\norm{\eta}_\sigma}{\delta} \frac{\Gamma(\nu)}{K(2\pi \delta)^\nu}\norm{\eta}_\sigma=\frac{2\pi \Gamma(\nu)}{K(2\pi\delta)^{\nu+1}}\norm{\eta}_\sigma^2.
\]

Third,
\[
C(z)=\int_0^1 h'(z+\rho+s(h(z)+\eta(z+h(z))))(h(z)+\eta(z+h(z))ds.
\]
We have that
\[
\norm{\eta}_\sigma < \frac{K(2\pi \delta)^{\nu+1}}{2\pi \Gamma(\nu)}
\leq \frac{K \cdot \delta}{\Gamma(\nu)}
\leq K\delta \leq\delta.
\]
For $z \in S_{\sigma-4\delta}$ and $0 \leq s \leq 1$ we have 
\[
z+\rho+s(h(z)+\eta(z+h(z))) \in S_{\sigma-2\delta}.
\]
We have
$\norm{h'}_{\sigma-2\delta} \leq \frac{\norm{h}_{\sigma-\delta}}{\delta}$, and so by Proposition \ref{hestimate} we get $\norm{h'}_{\sigma-2\delta}< \frac{2\pi \Gamma(\nu)}{K(2\pi \delta)^{\nu+1}}\norm{\eta}_{\sigma}$. Therefore for $z \in S_{\sigma-4\delta}$ we get
\[
|C(z)| <  \frac{2\pi \Gamma(\nu)}{K(2\pi \delta)^{\nu+1}}\norm{\eta}_{\sigma}
\Big(\frac{\Gamma(\nu)}{K(2\pi \delta)^\nu}\norm{\eta}_\sigma
+\norm{\eta}_\sigma \Big)
< \frac{4\pi \Gamma(\nu)^2 }{K^2 (2\pi \delta)^{2\nu+1}} \norm{\eta}_\sigma^2.
\]

Fourth, if $z \in S_{\sigma-6\delta}$ then $z+h(z)+\rho+\eta(z+h(z)) \in S_{\sigma-4\delta}$. Thus
by Proposition \ref{gestimate} we get
\[
\norm{D}_{\sigma-6\delta} <
\frac{2\pi \Gamma(\nu)^2}{K^2(2\pi \delta)^{2\nu+1}}\norm{\eta}_\sigma^2.
\]

Since $\psi$ is conjugate to $\phi$ it has rotation number $\rho$, so there is some $x_0 \in \mathbb{R}$ such that $\psi(x_0)=x_0+\rho$. Thus 
\[
x_0+\rho=x_0+\rho+\widehat{\eta}(0)+B(x_0)+C(x_0)+D(x_0),
\]
so
\[
\widehat{\eta}(0)=-B(x_0)-C(x_0)-D(x_0).
\]
Of course $x_0 \in S_{\sigma-6\delta}$, so by what we've done so far in this proof,
\[
|\widehat{\eta}(0)| <
\frac{2\pi \Gamma(\nu)}{K(2\pi\delta)^{\nu+1}}\norm{\eta}_\sigma^2
+
\frac{4\pi}{K^2 (2\pi \delta)^{2\nu+1}} \Gamma(\nu)^2 \norm{\eta}_\sigma^2
+
\frac{2\pi \Gamma(\nu)^2}{K^2(2\pi \delta)^{2\nu+1}}\norm{\eta}_\sigma^2.
\]
Therefore
\begin{eqnarray*}
\norm{\mu}_{\sigma-6\delta}
&<&2\cdot\Big(\frac{2\pi \Gamma(\nu)}{K(2\pi\delta)^{\nu+1}}\norm{\eta}_\sigma^2
+
\frac{4\pi \Gamma(\nu)^2 }{K^2 (2\pi \delta)^{2\nu+1}} \norm{\eta}_\sigma^2\\
&&+
\frac{2\pi \Gamma(\nu)^2}{K^2(2\pi \delta)^{2\nu+1}}\norm{\eta}_\sigma^2\Big)\\
&\leq&2\cdot\Big( \frac{2\pi \Gamma(\nu)^2}{K^2 (2\pi \delta)^{2\nu+1}} \norm{\eta}_\sigma^2
+\frac{4\pi \Gamma(\nu)^2}{K^2(2\pi \delta)^{2\nu+1}}\norm{\eta}_\sigma^2\\
&&+\frac{2\pi \Gamma(\nu)^2}{K^2(2\pi \delta)^{2\nu+1}}\norm{\eta}_\sigma^2\Big)\\
&=&\frac{16\pi \Gamma(\nu)^2}{K^2(2\pi \delta)^{2\nu+1}} \norm{\eta}_\sigma^2.
\end{eqnarray*}
\end{proof}

\begin{lemma}
\label{induction}
Let $\rho \in \mathbb{R} \setminus \mathbb{Q}$ be of type $(K,\nu)$.
Let $\eta_0 \in B_{\sigma_0}$ and let $\epsilon_0=\norm{\eta_0}_{\sigma_0}$.
Suppose that 
\begin{equation}
\label{epsilon0}
\epsilon_0 < \Big( \frac{K}{16\pi \Gamma(\nu)} \Big( \frac{\sigma_0}{36} \Big)^{\nu+1} \Big)^8,
\end{equation}
and that $\frac{\sigma_0}{36}<\frac{1}{2\pi}$.

Let $\phi_0(z)=z+\rho+\eta_0(z)$, and suppose that $\phi_0$ has rotation number $\rho$. For $n \geq 0$ let
\begin{itemize}
\item $\widehat{h_n}(k)=\frac{\widehat{\eta_n}(k)}{e^{2\pi i k\rho}-1}$ for $k \neq 0$ and $\widehat{h_n}(0)=0$
\item $H_n(z)=z+h_n(z)$, and $g_n(z)=H_n^{-1}(z)-z+h_n(z)$
\item $\phi_{n+1}=H_n^{-1} \circ \phi_n \circ H_n$
\item $\eta_{n+1}(z)=\phi_{n+1}(z)-z-\rho$
\item $\delta_n=\frac{\sigma_0}{36(1+n^2)}$
\item $\sigma_{n+1}=\sigma_n-6\delta_n$
\item $\epsilon_{n+1}=\epsilon_0^{(3/2)^{n+1}}$
\end{itemize}

Then for $n \geq 0$ we have that
\begin{itemize}
\item $\norm{\eta_{n+1}}_{\sigma_{n+1}} \leq \epsilon_{n+1}$
\item $\norm{h_n}_{\sigma_n-\delta_n} < \frac{\Gamma(\nu)\epsilon_n}{K(2\pi \delta_n)^\nu}$
\item $\norm{g_n}_{\sigma_n-4\delta_n} < \frac{2\pi\Gamma(\nu)^2\epsilon_n^2}{K^2(2\pi \delta_n)^{2\nu+1}}$
\end{itemize}
\end{lemma}
\begin{proof}
We first verify the claim for $n=0$. 
First,
$\delta_0=\frac{\sigma_0}{36}<\frac{1}{2\pi}$, and we have
\[
\epsilon_0 < \Big( \frac{K}{16\pi \Gamma(\nu)} \Big( \frac{\sigma_0}{36} \Big)^{\nu+1} \Big)^8
\leq \frac{K}{16\pi \Gamma(\nu)} \Big( \frac{\sigma_0}{36} \Big)^{\nu+1} 
=\frac{K \delta_0^{\nu+1}}{2\pi \Gamma(\nu)} \frac{1}{8}
<\frac{K(2\pi \delta_0)^{\nu+1}}{2\pi\Gamma(\nu)},
\]
which gives us that
$\frac{2\pi \Gamma(\nu)}{K(2\pi \delta_0)^{\nu+1}}\norm{\eta_0}_{\sigma_0}<1$.  
Thus by Proposition \ref{muestimate} we have
\[
\norm{\eta_1}_{\sigma_1}=\norm{\eta_1}_{\sigma_0-6\delta_0}
< \frac{16\pi \Gamma(\nu)^2}{K^2(2\pi \delta_0)^{2\nu+1}} \norm{\eta_0}_{\sigma_0}^2
=\frac{16\pi \Gamma(\nu)^2 \epsilon_0^2 }{K^2(2\pi \delta_0)^{2\nu+1}}.
\]
By \eqref{epsilon0} it follows that
$\frac{16\pi \Gamma(\nu)^2 \epsilon_0^2 }{K^2(2\pi \delta_0)^{2\nu+1}} \leq \epsilon_0^{3/2}$,
and so we get
$\norm{\eta_1}_{\sigma_1} \leq \epsilon_1$.

By Proposition \ref{hestimate} we get
\[
\norm{h_0}_{\sigma_0-\delta_0} < \frac{\Gamma(\nu)}{K(2\pi \delta_0)^\nu}\norm{\eta_0}_{\sigma_0}
= \frac{\Gamma(\nu)\epsilon_0}{K(2\pi \delta_0)^\nu},
\]
and by Proposition \ref{gestimate} we get
\[
\norm{g}_{\sigma_0-4\delta_0} < \frac{2\pi \Gamma(\nu)^2}{K^2(2\pi \delta_0)^{2\nu+1}}\norm{\eta_0}_{\sigma_0}^2
=\frac{2\pi\Gamma(\nu)^2\epsilon_0^2}{K^2(2\pi \delta_0)^{2\nu+1}}.
\]
This verifies the claim for $n=0$. Now we suppose that the claim is true for $n \leq N$, and we shall show that the claim is true for $n=N+1$. 

By assumption we have $\norm{\eta_{N+1}}_{\sigma_{N+1}} \leq \epsilon_{N+1}$.
Then by
Proposition \ref{hestimate} we have
\[
\norm{h_{N+1}}_{\sigma_{N+1}-\delta_{N+1}} <
\frac{\Gamma(\nu)}{K(2\pi \delta_{N+1})^\nu}\norm{\eta_{N+1}}_{\sigma_{N+1}}
\leq \frac{\Gamma(\nu)  \epsilon_{N+1}}{K(2\pi \delta_{N+1})^\nu}.
\]
\end{proof}
One can prove by induction that $\epsilon_n<\frac{K(2\pi \delta_n)^{\nu+1}}{2\pi\Gamma(\nu)}$ for all
$n \geq 0$, from which we have $\frac{2\pi \Gamma(\nu)}{K(2\pi \delta_{N+1})^{\nu+1}}\norm{\eta_{N+1}}_{\sigma_{N+1}}<1$. Therefore by Proposition \ref{gestimate} we get
\[
\norm{g_{N+1}}_{\sigma_{N+1}-4\delta_{N+1}} < \frac{2\pi \Gamma(\nu)^2}{K^2(2\pi \delta_{N+1})^{2\nu+1}}\norm{\eta_{N+1}}_{\sigma_{N+1}}^2 \leq
\frac{2\pi\Gamma(\nu)^2\epsilon_{N+1}^2}{K^2(2\pi \delta_{N+1})^{2\nu+1}}.
\]

Finally, we have by Proposition \ref{muestimate} and by assumption that
\[
\norm{\eta_{N+2}}_{\sigma_{N+1}-6\delta_{N+1}} < \frac{16\pi \Gamma(\nu)^2}{K^2(2\pi \delta_{N+1})^{2\nu+1}} \norm{\eta_{N+1}}_{\sigma_{N+1}}^2
\leq \frac{16\pi \Gamma(\nu)^2 \epsilon_{N+1}^2}{K^2(2\pi \delta_{N+1})^{2\nu+1}}.
\]
By \eqref{epsilon0} it follows that $\frac{16\pi \Gamma(\nu)^2 \epsilon_{N+1}^2}{K^2(2\pi \delta_{N+1})^{2\nu+1}} < \epsilon_{N+1}^{3/2}$, and so we get $\norm{\eta_{N+2}}_{\sigma_{N+2}}
\leq \epsilon_{N+2}$, completing the induction.

\begin{theorem}
Arnold's theorem
\end{theorem}
\begin{proof}
Let $\mathcal{H}_N=H_0 \circ H_1 \circ \cdots H_N$. By Lemma \ref{induction}, $\mathcal{H}_N$ is
holomorphic on $S_{\sigma_N-2\delta_N}$. And for $z \in S_{\sigma_N-2\delta_N}$,
\[
|\mathcal{H}_N(z)-z| = |h_N(z)+\cdots+h_0(\cdots)| \leq \sum_{n=0}^N \norm{h_n}_{\sigma_n-\delta_n}
<\sum_{n=0}^N \frac{\Gamma(\nu)\epsilon_n}{K(2\pi\delta_n)^\nu}.
\]
Let $\Delta=\sum_{n=0}^\infty \frac{\Gamma(nu)\epsilon_n}{K(2\pi\delta_n)^\nu}$.
\end{proof}

\bibliographystyle{amsplain}
\bibliography{arnold}

\end{document}