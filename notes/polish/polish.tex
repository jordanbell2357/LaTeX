\documentclass{article}
\usepackage{amsmath,amssymb,graphicx,subfig,mathrsfs,amsthm,siunitx}
%\usepackage{tikz-cd}
%\usepackage{hyperref}
\newcommand{\inner}[2]{\left\langle #1, #2 \right\rangle}
\newcommand{\tr}{\ensuremath\mathrm{tr}\,} 
\newcommand{\Span}{\ensuremath\mathrm{span}} 
\def\Re{\ensuremath{\mathrm{Re}}\,}
\def\Im{\ensuremath{\mathrm{Im}}\,}
\newcommand{\id}{\ensuremath\mathrm{id}} 
\newcommand{\rank}{\ensuremath\mathrm{rank\,}} 
\newcommand{\diam}{\ensuremath\mathrm{diam}} 
\newcommand{\osc}{\ensuremath\mathrm{osc}} 
\newcommand{\Int}{\ensuremath\mathrm{Int}} 
\newcommand{\co}{\ensuremath\mathrm{co}\,} 
\newcommand{\cco}{\ensuremath\overline{\mathrm{co}}\,}
\newcommand{\supp}{\ensuremath\mathrm{supp}\,}
\newcommand{\ext}{\ensuremath\mathrm{ext}\,}
\newcommand{\ba}{\ensuremath\mathrm{ba}\,}
\newcommand{\cl}{\ensuremath\mathrm{cl}\,}
\newcommand{\dom}{\ensuremath\mathrm{dom}\,}
\newcommand{\Cyl}{\ensuremath\mathrm{Cyl}\,}
\newcommand{\extreals}{\overline{\mathbb{R}}}
\newcommand{\upto}{\nearrow}
\newcommand{\downto}{\searrow}
\newcommand{\norm}[1]{\left\Vert #1 \right\Vert}
\newtheorem{theorem}{Theorem}
\newtheorem{lemma}[theorem]{Lemma}
\newtheorem{proposition}[theorem]{Proposition}
\newtheorem{corollary}[theorem]{Corollary}
\theoremstyle{definition}
\newtheorem{definition}[theorem]{Definition}
\newtheorem{example}[theorem]{Example}
\begin{document}
\title{Polish spaces and Baire spaces}
\author{Jordan Bell\\ \texttt{jordan.bell@gmail.com}\\Department of Mathematics, University of Toronto}
\date{\today}

\maketitle


\section{Introduction}
These notes consist of me working through those parts of the first chapter of Alexander S. Kechris, {\em Classical Descriptive Set Theory}, 
that I think are important in analysis. Denote by  $\mathbb{N}$ the set of positive integers.
I do not talk about universal spaces like the Cantor space $2^\mathbb{N}$, the Baire space $\mathbb{N}^\mathbb{N}$, and the
Hilbert cube $[0,1]^\mathbb{N}$, or ``localization'', or about Polish groups. 

If $(X,\tau)$ is a topological space, the \textbf{Borel $\sigma$-algebra of $X$},
denoted by $\mathscr{B}_X$, is the smallest $\sigma$-algebra of subsets of $X$ that contains $\tau$. 
$\mathscr{B}_X$ contains $\tau$, and is closed under complements and countable unions, and rather than talking merely about \textbf{Borel sets} (elements of the Borel $\sigma$-algebra),
we can be more specific by talking about open sets, closed sets, and sets that are obtained by taking countable unions and complements. 

\begin{definition}
An \textbf{$F_\sigma$ set}
is a countable union of closed sets.

A \textbf{$G_\delta$ set} is a complement of an $F_\sigma$ set. Equivalently, it is a countable intersection of open sets.
\end{definition}

If $(X,d)$ is a metric space, the \textbf{topology induced by the metric $d$} is the topology generated by the collection of open balls. 
If $(X,\tau)$ is a topological space, a metric $d$ on the set $X$ is said to be \textbf{compatible with $\tau$} if $\tau$ is the topology induced by $d$. A \textbf{metrizable
space} is a topological space whose topology is induced by some metric, and   a \textbf{completely metrizable space} is a topological space whose topology is induced by some
complete metric. One proves that being metrizable  and being completely metrizable are topological properties, i.e., are preserved by homeomorphisms.


If $X$ is a topological space, a \textbf{subspace of $X$} is a subset of $X$ which is a topogical space with the subspace topology inherited from $X$. 
Because any topological space is a closed subset of itself, when we say that a \textbf{subspace is closed} we mean that 
it is a closed subset of its parent space, and similarly for open, $F_\sigma$, $G_\delta$.
A subspace of a compact Hausdorff space is compact if and only if it is closed;
a subspace of a metrizable space is metrizable; and a subspace of a completely metrizable space is completely metrizable if and only if it is closed.

A topological space is said to be \textbf{separable} if it has a countable dense subset, and \textbf{second-countable} if it has a countable basis for its topology.
It is straightforward to check that being second-countable implies being separable, but a separable topological space need not be second-countable. However, one checks
that a separable metrizable space is  second-countable. 
A subspace of a second-countable topological space is second-countable, and because a subspace of a metrizable space is metrizable, it follows that a subspace
of a separable metrizable space is separable.

A \textbf{Polish space} is a separable completely metrizable space. My own interest in Polish spaces is because one can prove many things about Borel
probability measures on a Polish space that one cannot prove for other types of topological spaces. Using the fact (the \textbf{Heine-Borel theorem}) that a compact
metric space is complete and totally bounded, one proves that a compact metrizable space is Polish, but for many
purposes we do not need  a metrizable space to be compact, only Polish, and using compact spaces rather than Polish spaces
excludes, for example, $\mathbb{R}$. 


\section{Separable Banach spaces}
Let $K$ denote either $\mathbb{R}$ or $\mathbb{C}$. 
If $X$ and $Y$ are Banach spaces over $K$, we denote by $\mathscr{B}(X,Y)$ the set of bounded linear operators $X \to Y$. With the operator norm, 
this is a Banach space.
We shall be interested in the \textbf{strong operator topology}, which is the initial topology on
$\mathscr{B}(X,Y)$ induced by the family $\{T \mapsto Tx: x \in X\}$. 
One proves that the strong operator topology on $\mathscr{B}(X,Y)$ is induced by the family of
seminorms $\{T \mapsto \norm{Tx}: x \in X\}$, and because this is a separating family of seminorms, $\mathscr{B}(X,Y)$ with the strong operator topology is
a \textbf{locally convex space}. 
A basis of convex sets for the strong operator topology consists of those sets of the form
\[
\{S \in \mathscr{B}(X,Y): \norm{Sx_1-T_1x_1}<\epsilon,\ldots,\norm{Sx_n-T_nx_n}<\epsilon\},
\]
for $x_1,\ldots,x_n \in X$, $\epsilon>0$, $T_1,\ldots,T_n \in \mathscr{B}(X,Y)$. 
 
We  prove conditions under which the closed unit ball in $\mathscr{B}(X,Y)$ with the strong operator topology is Polish.\footnote{Alexander S. Kechris, {\em Classical Descriptive Set Theory}, p.~14.}
 
\begin{theorem}
Suppose that $X$ and $Y$ are separable Banach spaces. Then the closed unit ball
\[
B_1 = \{T \in \mathscr{B}(X,Y): \norm{T} \leq 1\}
\]
with the subspace topology inherited from $\mathscr{B}(X,Y)$ with the strong operator topology is Polish.
\end{theorem} 
\begin{proof}
Let $E$ be $\mathbb{Q}$ or $\{a+ib: a,b \in \mathbb{Q}\}$, depending on whether $K$ is $\mathbb{R}$ or $\mathbb{C}$,
 let $D_0$ be a countable dense subset of $X$, and let
$D$ be the span of $D_0$ over $K$. $D$ is countable and $Y$ is Polish, so the product $Y^D$ is Polish. 
Define $\Phi:B_1 \to Y^D$ by $\Phi(T) = T \circ \iota$, where $\iota:D \to X$ is the inclusion map. 
If $\Phi(S)=\Phi(T)$, then because $D$ is dense in $X$ and $S,T:X \to Y$ are continuous, $X=Y$, showing that $\Phi$ is one-to-one.
We check that
$\Phi(B_1)$ consists of those $f \in Y^D$ such that both (i) if $x,y \in D$ and $a,b \in E$ then $f(ax+by)=af(x)+bf(y)$, and (ii)
if $x \in D$ then $\norm{f(x)} \leq \norm{x}$. One proves that $\Phi(B_1)$ is a closed subset of $Y^D$, and because
$Y^D$ is Polish this implies that $\Phi(B_1)$ with the subspace topology inherited from $Y^D$ is Polish. Then one proves that $\Phi:B_1 \to \Phi(B_1)$ is a
homeomorphism, where $B_1$ has the subspace topology inherited from $\mathscr{B}(X,Y)$ with the strong operator
topology, which tells us that $B_1$ is Polish.
\end{proof}

If $X$ is a Banach space over $K$, where $K$ is $\mathbb{R}$ or $\mathbb{C}$, we write $X^*=\mathscr{B}(X,K)$. The strong
operator topology on $\mathscr{B}(X,K)$ is called the \textbf{weak-*} topology on $X^*$.
\textbf{Keller's theorem}\footnote{Alexander S. Kechris, {\em Classical Descriptive Set Theory}, p.~64, Theorem 9.19.}
 states that if $X$ is a separable infinite-dimensional Banach space, then the closed unit ball in $X^*$ with the subspace
topology inherited from $X^*$ with the weak-* topology is homeomorphic to the Hilbert cube $[0,1]^\mathbb{N}$.

\section{$G_\delta$ sets}
If $(X,d)$ is a metric space and $A$ is a subset of $X$, we define
\[
\diam(A) = \sup\{ d(x,y): x,y \in A\},
\]
with $\diam(\emptyset)=0$, and if $x \in X$ we define
\[
d(x,A) = \inf\{d(x,y): y \in A\},
\]
with $d(x,\emptyset)=\infty$. 
We also define
\[
B_d(A,\epsilon) = \{x \in X: d(x,A)<\epsilon\}.
\]

If $X$ and $Y$ are topological spaces and $f:X \to Y$ is a function, the \textbf{set of continuity} of $f$ is the set of all points in $X$ at which $f$ is continuous.
To say that $f$ is continuous is equivalent to saying that its set of continuity is $X$.

If $X$ is a topological space, $(Y,d)$ is a metric space, $A \subset X$, and $f:A \to Y$ is a function, for $x \in X$ we define the \textbf{oscillation of $f$
at $x$} as
\[
\osc_f(x) = \inf\{ \diam(f(U \cap A)): \textrm{$U$ is an open neighborhood of $x$}\}.
\]
To say that $f:A \to Y$ is continuous at $x \in A$ means that for every $\epsilon>0$ there is some open neighborhood $U$ of $x$ such that
$y \in U \cap A$ implies that $d(f(y),f(x))<\epsilon$, and this implies that $\diam(f(U \cap A)) \leq 2\epsilon$. Hence if $f$ is continuous at $x$ then
$\osc_f(x)=0$. On the other hand, suppose that $\osc_f(x)=0$ and let $\epsilon>0$. There is then some open neighborhood $U$ of $x$
such that $\diam(f(U \cap A))<\epsilon$, and this implies that $d(f(y),f(x))<\epsilon$ for every $y \in U \cap A$, showing that $f$ is continuous at $x$.
Therefore, the set of continuity of $f:A \to Y$ is
\[
\{x \in A: \osc_f(x)=0\}.
\] 
As well, if $x \in X \setminus \overline{A}=\overline{A}^c$, then $\overline{A}^c$ is an open neighborhood of $x$ and $f(\overline{A}^c \cap A) = 
f(\emptyset)=\emptyset$ and $\diam(\emptyset)=0$, so in this case $\osc_f(x)=0$.


The following theorem shows that the set of points where a function taking values in a metrizable space 
has zero oscillation 
is a $G_\delta$ set.\footnote{Alexander S. Kechris, {\em Classical Descriptive Set Theory}, p.~15, Proposition 3.6.}

\begin{theorem}
Suppose that $X$ is a topological space, $Y$ is a metrizable space, $A \subset X$, and $f:A \to Y$ is a function. Then $\{x \in X: \osc_f(x)=0\}$ is a $G_\delta$ set.
\label{continuityset}
\end{theorem}
\begin{proof}
Let $d$ be a metric on $Y$ that induces its topology
and let $A_\epsilon=\{x \in X: \osc_f(x)<\epsilon\}$. For $x \in A_\epsilon$, there is an open neighborhood $U$ of $x$ such that
$\osc_f(x) \leq \diam(f(U \cap A))  < \epsilon$. But if $y \in U$ then $U$ is an open neighborhood of $y$ and $\diam(f(U \cap A))<\epsilon$, 
so $\osc_f(y) < \epsilon$ and hence $y \in A_\epsilon$, showing that $A_\epsilon$ is open.
Finally,
\[
\{x \in X: \osc_f(x)=0\} = \bigcap_{n \in \mathbb{N}} A_{1/n},
\]
which is a $G_\delta$ set, completing the proof.
\end{proof}




In a metrizable space, the only closed sets that are open are $\emptyset$ and the space itself, but we can show that
any closed set is a countable intersection of open sets.\footnote{Alexander S. Kechris, {\em Classical Descriptive Set Theory}, p.~15, Proposition 3.7.}

\begin{theorem}
If $X$ is a metrizable space, then any closed subset of $X$ is a $G_\delta$ set.
\label{closedGdelta}
\end{theorem}
\begin{proof}
Let $d$ be a metric on $X$ that induces its topology. Suppose that $A$ is a nonempty subset of $X$ and that $x,y \in X$. We have
$d(x,A) \leq d(x,y)+d(y,A)$ and $d(y,A) \leq d(y,x) + d(x,A)$, so
\[
|d(x,A) - d(y,A)| \leq d(x,y).
\]
It follows that $B_d(A,\epsilon)$ is open. But if $F$ is a closed subset of $X$ then check that 
\[
F = \bigcap_{n \in \mathbb{N}} B_d(F,1/n),
\]
which is an $F_\sigma$ set, completing the proof. (If we did not know that $F$ was closed then  $F$ would be contained in this intersection, but
need not be equal to it.)
\end{proof}

Kechris attributes the following theorem\footnote{Alexander S. Kechris, {\em Classical Descriptive Set Theory}, p.~16,
Theorem 3.8.}
 to Kuratowski. It and the following theorem are about extending continuous functions from a set to a $G_\delta$ 
set that contains it, and  we will use the following theorem in the proof of Theorem \ref{Polishsubspace}.

\begin{theorem}
Suppose that $X$ is metrizable, $Y$ is completely metrizable, $A$ is a subspace of $X$, and $f:A \to Y$ is continuous. Then there is a $G_\delta$ set
$G$ in $X$ such that $A \subset G \subset \overline{A}$ and a continuous function $g:G \to Y$ whose restriction to $A$ is equal to $f$.
\label{kuratowski}
\end{theorem}
\begin{proof}
Let $G=\overline{A} \cap \{x \in X: \osc_f(x)=0\}$.
Theorem \ref{closedGdelta} tells us that the first set is $G_\delta$ and 
Theorem \ref{continuityset} tells us that the second set is $G_\delta$, so $G$ is $G_\delta$. Because $f:A \to Y$ is continuous,
$A \subset \{x \in X: \osc_f(x)=0\}$, and hence $A \subset G$.

Let $x \in G \subset \overline{A}$, and let $x_n, t_n \in A$ with $x_n \to x$ and $t_n \to x$. Because $\osc_f(x)=0$, 
for every $\epsilon>0$ there is some open neighborhood $U$ of $x$ such that 
$\diam(f(U \cap A))<\epsilon$. But then there is some $n$ such that $k \geq n$ implies that $x_k, t_k \in U$, and thus
$\diam(f(\{x_k,t_k: k \geq n\}))<\epsilon$. Hence
$\diam(f(\{x_k,t_k: k \geq n\})) \to 0$ as $n \to \infty$, and
this is equivalent to the sequence $f(x_1),f(t_1),f(x_2),f(t_2),\ldots$ being Cauchy. Because $Y$ is completely metrizable this sequence converges to some
$y \in Y$
and therefore the subsequence $f(x_n)$ and the subsequence $f(t_n)$ both converge to $y$.
Thus it makes sense to define $g:G \to Y$ by
\[
g(x) = \lim_{n \to \infty} f(x_n),
\]
and the restriction of $g$ to $A$ is equal to $f$.  It remains to prove that $g$ is continuous.

If $U$ is an open subset of $X$, then $g(U \cap G) \subset \overline{f(U \cap A)}$, hence
\[
\diam( g(U \cap G)) \leq \diam(\overline{f(U \cap A)}) = \diam(f(U \cap A)).
\]
For any $x \in G$ this and $\osc_f(x)=0$ yield
\[
\osc_g(x) \leq \osc_f(x) = 0,
\]
showing that the set of continuity of $g$ is $G$, i.e. that $g$ is continuous.
\end{proof}

The following shows that a homeomorphism between  subsets of metrizable spaces can be extended to a homeomorphism of
 $G_\delta$ sets.\footnote{Alexander S. Kechris, {\em Classical Descriptive Set Theory}, p.~16, Theorem 3.9.}

\begin{theorem}[Lavrentiev's theorem]
Suppose that $X$ and $Y$ are completely metrizable spaces, that $A$ is a subspace of $X$, and that $B$ is a subspace of $Y$. If $f:A \to B$
is a homeomorphism, then there are $G_\delta$ sets $G \supset A$ and $H \supset B$ and a homeomorphism $G \to H$ whose restriction
to $A$ is equal to $f$.
\label{lavrentiev}
\end{theorem}
\begin{proof}
Theorem \ref{kuratowski} tells us that there is a $G_\delta$ set $G_1 \supset A$ and a continuous function
$g_1:G_1 \to Y$ whose restriction to $A$ is equal to $f$, and there is a $G_\delta$ set $H_1 \supset B$ and a continuous function
$h_1:H_1 \to X$ whose restriction to $B$ is equal to $f^{-1}$. 
Let
\[
R = \{(x,y) \in G_1 \times Y: y=g_1(x)\}, \qquad S = \{(x,y) \in X \times H_1 : x=h_1(y)\}.
\]
Because $g_1:G_1 \to Y$ is continuous,
$R$ is a closed subset of $X \times Y$, and because $h_1:H_1 \to X$ is continuous, $S$ is a closed subset of $X \times Y$.
Let
\[
G = \pi_X(R \cap S), \qquad H = \pi_Y(R \cap S),
\]
where $\pi_X:X \times Y \to X$ and $\pi_Y:X \times Y \to Y$ are the projection maps. 
If $x \in A$ then $h_1(g_1(x))=f^{-1}(f(x))=x$, and hence $x \in G$, and if $y \in B$ then $g_1(h_1(y))=f(f^{-1}(y))=y$, and hence
$y \in H$, so we have
\[
A \subset G \subset G_1, \qquad B \subset H \subset H_1.
\]

The map 
$E_1:G_1 \to X \times Y$ defined by $E_1(x) = (x,g_1(x))$ is continuous because $g_1:G_1 \to Y$ is continuous, and hence
\[
E_1^{-1}(S)=\{x \in G_1: x = h_1(g_1(x))\}=G
\] 
is a closed subset of $G_1$, and thus by Theorem \ref{closedGdelta} is a $G_\delta$ set in $G_1$. But $G_1$ is a $G_\delta$
subset of $X$, so $G$ is a $G_\delta$ set in $X$ also.
Define $E_2:H_1 \to X \times Y$ by $E_2(y)=(h_1(y),y)$, which is continuous because $h_1$
is continuous. Then
\[
E_2^{-1}(R) = \{y \in H_1: y=g_1(h_1(y))\} = H
\]
is a closed subset of $H_1$, and hence is $G_\delta$ in $H_1$.
But $H_1$ is a $G_\delta$ subset of $Y$, so $H_1$ is a $G_\delta$ set in $Y$ also.

Check  that the restriction of $g_1$ to $G_1$ is a homeomorphism $G_1 \to H_1$ whose restriction
to $A$ is equal to $f$, completing the proof.
\end{proof}


If a topological space has some property and $Y$ is a subset of $X$, one wants to know conditions under which $Y$ with the subspace
topology inherited from $X$ has the same property.
For example, a subspace of a compact Hausdorff space is compact if and only if it is closed, and a subspace of a completely metrizable space
is completely metrizable if and only if it is  closed.
The following theorem shows in particular that a subspace of a Polish space is Polish if and only if it is $G_\delta$.\footnote{Alexander S. Kechris, {\em Classical Descriptive Set Theory}, p.~17, Theorem 3.11.} (The statement
of the theorem is about completely metrizable spaces and we obtain the conclusion about Polish spaces because any subspace
of a separable metrizable space is itself separable.)

\begin{theorem}
Suppose that $X$ is a metrizable space and $Y$ is a subset of $X$ with the subspace topology. If $Y$ 
is completely metrizable then $Y$ is a $G_\delta$ set in $X$. If $X$ is completely metrizable and $Y$ is a $G_\delta$ set in $X$ then
$Y$ is completely metrizable.
\label{Polishsubspace}
\end{theorem}
\begin{proof}
Suppose that $Y$ is completely metrizable. The map $\id_Y:Y \to Y$ is continuous, so Theorem \ref{kuratowski} tells
us that there is a $G_\delta$ set $Y \subset G \subset \overline{Y}$ and a continuous function $g:G \to Y$ whose restriction
to $Y$ is equal to $\id_Y$. For $x \in G \subset \overline{Y}$, there are $y_n \in Y$ with $y_n \to x$, and because
$g$ is continuous we get $\id_Y(y_n)=g(y_n) \to g(x)$,  i.e. $y_n \to g(x)$, hence $g(x)=x$. But $g:G \to Y$ so $x \in Y$, showing
that $G = Y$ and hence that $Y$ is a $G_\delta$ set.

Suppose that $X$ is completely metrizable and that $Y$ is a $G_\delta$ subset of $X$,
and let $d$ be a complete metric
on $X$ that is compatible with the topology of $X$;
if we restrict this metric to $Y$ then it is a metric on $Y$ that is compatible with the subspace topology on $Y$ inherited from $X$,
but it need not be a complete metric.
 Let  $U_n$ be open sets in $X$ with $Y = \bigcap_{n \in \mathbb{N}} U_n$, let
 $F_n = X \setminus U_n$, and for $x,y \in Y$ define
\[
d_1(x,y) = d(x,y)+\sum_{n \in \mathbb{N}} \min\left\{2^{-n},\left| \frac{1}{d(x,F_n)}-\frac{1}{d(y,F_n)} \right| \right\}.
\]
One proves that $d_1$ is a metric on $Y$ and that it is compatible with the subspace topology on $Y$. Suppose that
$y_n \in Y$ is Cauchy in $(Y,d_1)$. Because $d \leq d_1$, this is also a Cauchy sequence in $(X,d)$, and because $(X,d)$
is complete, there is some $y \in X$ such that $y_n \to y$ in $(X,d)$. Then one proves that $y_n \to y$ in $(Y,d_1)$, from which
we have that $(Y,d_1)$ is a complete metric space. 
\end{proof}



\section{Continuous functions on a compact space}
\label{CXY}
If $X$ and $Y$ are topological spaces, we denote by $C(X,Y)$ the set of continuous functions $X \to Y$. 
If $X$ is a compact topological space and $(Y,\rho)$ is a metric space, we define 
\[
d_\rho(f,g) = \sup_{x \in X} \rho(f(x),g(x)), \qquad f,g \in C(X,Y),
\]
which is a metric on $C(X,Y)$, which we call
the \textbf{$\rho$-supremum
metric}. One proves that $d_\rho$ is a complete metric on $C(X,Y)$ if and only if
$\rho$ is a complete metric on $Y$.\footnote{Charalambos D. 
Aliprantis and Kim C. Border, {\em Infinite Dimensional Analysis: A Hitchhiker's Guide}, third ed., p.~124, Lemma 3.97.} 
It follows that if $Y$ is a Banach space then so is $C(X,Y)$ with the supremum norm $\norm{f}_\infty = \sup_{x \in X} \norm{f(x)}_Y$.

Suppose that $X$ is a compact topological space and that $Y$ is a  metrizable space. 
If $\rho_1,\rho_2$ are metrics on $Y$ that induce its topology,
then $d_{\rho_1},d_{\rho_2}$ are metrics on $C(X,Y)$, and it can be proved
that they induce the same topology,\footnote{Charalambos D. 
Aliprantis and Kim C. Border, {\em Infinite Dimensional Analysis: A Hitchhiker's Guide}, third ed., p.~124,
Lemma 3.98.}  which we call the \textbf{topology of uniform convergence}. 


Finally, if $X$ is a compact metrizable space and $Y$ is a separable metrizable space, it can be proved that
$C(X,Y)$ is separable.\footnote{Charalambos D. 
Aliprantis and Kim C. Border, {\em Infinite Dimensional Analysis: A Hitchhiker's Guide}, third ed., p.~125,
Lemma 3.99.} 

Thus, using what we have stated above, suppose that $X$ is a compact metrizable space and that $Y$ is a Polish
space. Because $X$ is a compact metrizable space and $Y$ is a separable metrizable space, $C(X,Y)$
is separable. Because $X$ is a compact topological space and $Y$ is a completely metrizable space,
$C(X,Y)$ is completely metrizable, and hence Polish.


\section{$C([0,1])$}
$C^1(\mathbb{R})$ consists of those functions $F:\mathbb{R} \to \mathbb{R}$ such that for each $x_0 \in \mathbb{R}$, there is some 
$F'(x_0) \in \mathbb{R}$ such that
\[
F'(x_0)= \lim_{x \to x_0} \frac{F(x)-F(x_0)}{x-x_0},
\]
and such that this function $F'$ belongs to $C(\mathbb{R})$. We define $C^1([0,1])$ to be those functions $[0,1] \to \mathbb{R}$ that are the restriction
to $[0,1]$ of some element of $C^1(\mathbb{R})$. 
We shall prove that $C^1([0,1])$ is an $F_{\sigma \delta}$ set in $C([0,1])$.\footnote{Alexander S. Kechris, {\em Classical Descriptive Set Theory}, p.~70.}

Suppose that $f \in C^1([0,1])$. For each $x \in [0,1]$, 




\section{Meager sets and Baire spaces}
Let $X$ be a topological space.  A subet $A$ of $X$ is called \textbf{nowhere dense} if 
the interior of $\overline{A}$ is $\emptyset$. A subset $A$ of $X$ is called \textbf{meager} if it is a countable union of nowhere dense sets. A meager
set is also said to be \textbf{of first category}, and a nonmeager is said to be \textbf{of second category}.
Meager is a good name  for at least two reasons: it is descriptive and  the word is not already used to name anything else.
First category and second category are bad names for at least four reasons: the words describe nothing, they are phrases rather than single
words, they suggests an ordering, and they conflict with
reserving the word ``category'' for category theory. 
A complement of a meager  is said to be \textbf{comeager}. 

If $X$ is a set, an \textbf{ideal on $X$} is a collection of subsets of $X$ that includes $\emptyset$ and is closed under subsets and finite unions. A
\textbf{$\sigma$-ideal on $X$} is an ideal that is closed under countable unions.

\begin{lemma}
The collection of meager subsets of a topological space is a $\sigma$-ideal.
\end{lemma}


If $X$ is a topological space and $x \in X$, we say that $x$ is \textbf{isolated} if $\{x\}$ is open. We say $X$ is \textbf{perfect}
if it has no isolated points, and a \textbf{$T_1$ space} if $\{x\}$ is closed for each $x \in X$. Suppose that $X$ is a perfect $T_1$ space and let $A$ be a countable subset of $X$. 
For each $x \in A$, because $X$ is $T_1$, the closure of $\{x\}$ is $\{x\}$, and because $X$ is perfect, the interior of $\{x\}$ is $\emptyset$, and hence
$\{x\}$ is nowhere dense. $A=\bigcup_{x \in A} \{x\}$ is a countable union of nowhere dense sets, hence is meager. Thus we have proved
that any countable subset of a perfect $T_1$ space is meager.   

Suppose that $X$ is a topological space. If every comeager set in $X$ is dense, we say that $X$ is a \textbf{Baire space}. 

\begin{lemma}
A topological space is a Baire space if and only if the intersection of any countable family of  dense open sets is dense.
\end{lemma}

We prove that open subsets of Baire spaces are Baire spaces.\footnote{Alexander S. Kechris, {\em Classical Descriptive Set Theory}, p.~41, Proposition
8.3.}

\begin{theorem}
If $X$ is a Baire space and $U$ is an open subspace of $X$, then $U$ is a Baire space.
\end{theorem}
\begin{proof}
Because $U$ is open, an open subset of $U$ is an open subset of $X$ that is contained in $U$. 
Suppose that $U_n$, $n \in \mathbb{N}$, are dense open subsets of $U$. So they are each open subsets of $X$, 
and $U_n \cup (X \setminus \overline{U})$ is a dense open subset of $X$ for each $n \in \mathbb{N}$. 
Then because $X$ is a Baire space, 
\[
\bigcap_{n \in \mathbb{N}} (U_n \cup (X \setminus \overline{U})) = \left( \bigcap_{n \in \mathbb{N}} U_n \right) \cup (X \setminus \overline{U})
\]
is dense in $X$. It follows that $\bigcap_{n \in \mathbb{N}} U_n$ is dense in $U$, showing that $U$ is a Baire space.
\end{proof} 


The following is the \textbf{Baire category theorem}.\footnote{Alexander S. Kechris, {\em Classical Descriptive Set Theory}, p.~41, Theorem
8.4.}

\begin{theorem}[Baire category theorem]
Every completely metrizable space is a Baire space. Every locally compact Hausdorff space is a Baire space.
\end{theorem}
\begin{proof}
Let  $X$ be a completely metrizable space and let $d$ be a complete metric on $X$ compatible with the topology. 
Suppose that $U_n$ are dense open subsets of $X$. To show that $\bigcap_{n \in \mathbb{N}} U_n$ is dense it suffices
to show that for any nonempty open subset $U$ of $X$,
\[
\bigcap_{n \in \mathbb{N}} (U_n \cap U) = U \cap \bigcap_{n \in \mathbb{N}} U_n \neq \emptyset. 
\]
Because $U$ is a nonempty open set it contains an open ball $B_1$ of radius $<1$ with $\overline{B_1} \subset U$. 
Since $U_1$ is dense and $B_1$ is open, $B_1 \cap U_1 \neq \emptyset$ and is open because both $B_1$ and $U_1$ are open.
As $B_1 \cap U_1$ is a nonempty open set it contains an open ball $B_2$ of radius $<\frac{1}{2}$ with
$\overline{B_2} \subset B_1 \cap U_1$. Suppose that $n>1$ and that $B_n$ is an open ball of radius $<\frac{1}{n}$ with
$\overline{B_n} \subset B_{n-1} \cap U_{n-1}$. Since $U_n$ is dense and $B_n$ is open,
$B_n \cap U_n \neq \emptyset$ and is open because both $B_n$ and $U_n$ are open.
As $B_n \cap U_n$ is a nonempty open set it contains an open ball $B_{n+1}$ of radius $<\frac{1}{n+1}$ with
$\overline{B_{n+1}} \subset B_n \cap U_n$. Then, we have $B_{n+1} \subset B_n$ for each $n \in \mathbb{N}$. 
Letting $x_i$ be the center of $B_i$, we have $d(x_j,x_i)<\frac{1}{i}$ for $j >i$, and hence $x_i$ is a Cauchy sequence.
Since $(X,d)$ is a complete metric space, there is some $x \in X$ such that $x_i \to x$. For
any $m$ there is some $i_0$ such that $i \geq i_0$ implies that
$d(x_i,x)<\frac{1}{m}$, and hence 
$x \in B_m = \bigcap_{n=1}^m B_n$. Therefore
\[
x \in \bigcap_{n \in \mathbb{N}} B_n \subset \bigcap_{n \in \mathbb{N}} (U_n \cap U),
\]
which shows that $\bigcap_{n \in \mathbb{N}} U_n$ is dense and hence that $X$ is a Baire space.


Let  $X$ be a locally compact Hausdorff space.  Suppose that $U_n$ are dense
open subsets of $X$ and that $U$ is a nonempty open set.  
Let $x_1 \in U$, and because $X$ is a locally compact Hausdorff space there is an open neighborhood
$V_1$ of $x_1$ with $\overline{V_1}$ compact and $\overline{V_1} \subset U$. 
Since $U_1$ is dense and $V_1$ is open, there is some $x_2 \in V_1 \cap U_1$.
As $V_1 \cap U_1$ is open, there is an open neighborhood $V_2$ of $x_2$ with $\overline{V_2}$ compact and 
$\overline{V_2} \subset V_1 \cap U_1$. Thus, $\overline{V_n}$ are compact and satisfy 
$\overline{V_{n+1}} \subset \overline{V_n}$ for each $n$, and
hence
\[
\bigcap_{n \in \mathbb{N}} \overline{V_n} \neq \emptyset. 
\]
This intersection is contained in $\bigcap_{n \in \mathbb{N}} (U_n \cap U)$ which is therefore nonempty,
showing that $\bigcap_{n \in \mathbb{N}} U_n$
is dense and hence that $X$ is a Baire space.
\end{proof} 



\section{Nowhere differentiable functions}
From what we said in \S \ref{CXY},
because $[0,1]$ is a compact metrizable space and $\mathbb{R}$ is a Polish space, $C([0,1])=
C([0,1],\mathbb{R})$ with the topology of uniform convergence is Polish. This topology is induced by the norm $\norm{f}_\infty = \sup_{x \in [0,1]} |f(x)|$,
with which $C([0,1])$ is thus a separable Banach space.

For a function $F:\mathbb{R} \to \mathbb{R}$ to be differentiable at a point $x_0$ means that
there is some $F'(x_0) \in \mathbb{R}$ such that
\[
\lim_{x \to x_0} \frac{F(x)-F(x_0)}{x-x_0} = F'(x_0).
\]
If $f:[0,1] \to \mathbb{R}$ is a function and $x_0 \in [0,1]$, we say that $f$ is \textbf{differentiable at $x_0$} if there is some function $F:\mathbb{R} \to \mathbb{R}$ that
is differentiable at $x_0$ and whose restriction to $[0,1]$ is equal to $f$, and we write $f'(x_0)=F'(x_0)$. The purpose of speaking in this way 
is to be precise about what we mean by $f$ being differentiable at the endpoints of the interval $[0,1]$. 

If $f:[0,1] \to \mathbb{R}$ is differentiable at $x_0 \in [0,1]$, then there is some $\delta>0$ such that if $0<|x-x_0| < \delta$ and $x \in [0,1]$, then
\[
\left|\frac{f(x)-f(x_0)}{x-x_0} - f'(x_0)\right|<1,
\]
and hence
\[
|f(x)-f(x_0)|< (1+|f'(x_0)|)|x-x_0|.
\]
On the other hand, if $f \in C([0,1])$ then $\{x \in [0,1]: |x-x_0| \geq \delta\}$ is a compact set on which $x \mapsto \frac{f(x)-f(x_0)}{x-x_0}$ is continuous, and hence the absolute value of this
function is bounded
by some $M$. Thus, if $|x-x_0| \geq \delta$ and $x \in [0,1]$, then
\[
\left|\frac{f(x)-f(x_0)}{x-x_0} \right| \leq M,
\]
hence
\[
|f(x)-f(x_0)| \leq M|x-x_0|.
\]
Therefore, if $f \in C([0,1])$ is differentiable at $x_0 \in [0,1]$ then there is some positive integer $N$ such that
\[
|f(x)-f(x_0)| \leq N|x-x_0|, \qquad x \in [0,1].
\]
For $N \in \mathbb{N}$, let $E_N$ be those $f \in C([0,1])$ for which there is some $x_0 \in [0,1]$ such that
\[
|f(x)-f(x_0)| \leq N|x-x_0|, \qquad x \in [0,1].
\]
We have established that if $f \in C([0,1])$ and there is some $x_0 \in [0,1]$ such that $f$ is differentiable at $x_0$, then
there is some $N \in \mathbb{N}$ such that $f \in E_N$. 
Therefore, the set of those $f \in C([0,1])$ that are differentiable at some point in $[0,1]$ is contained in 
\[
\bigcup_{N \in \mathbb{N}} E_N,
\]
and hence to prove that the set of $f \in C([0,1])$ that are nowhere differentiable is comeager in $C([0,1])$, it suffices to prove
that each $E_N$ is nowhere dense.
To show this we shall follow the proof in Stein and Shakarchi.\footnote{Elias M. Stein and Rami Shakarchi, {\em Functional Analysis}, p.~163, Theorem 1.5.} 

\begin{lemma}
For each $N \in \mathbb{N}$, $E_N$ is a closed subset of the Banach space $C([0,1])$. 
\end{lemma}
\begin{proof}
$C([0,1])$ is a metric space, so to show that  $E_N$ is closed it suffices to prove that if $f_n \in E_N$ is a sequence tending to $f \in C([0,1])$, then $f \in E_N$. 
For each $n$, let $x_n \in [0,1]$ be such that
\[
|f_n(x)-f_n(x_n)| \leq N|x-x_n|, \qquad x \in [0,1].
\]
Because $x_n$ is a sequence in the compact set $[0,1]$, it has subsequence $x_{a(n)}$ that converges to some $x_0 \in [0,1]$. 
For all $x \in [0,1]$ we have
\begin{eqnarray*}
|f(x)-f(x_0)|&\leq&|f(x)-f_{a(n)}(x)|+|f_{a(n)}(x)-f_{a(n)}(x_0)|\\
&&+|f_{a(n)}(x_0)-f(x_0)|.
\end{eqnarray*}
Let $\epsilon>0$. Because $\norm{f_n-f}_\infty \to 0$, there is some $n_0$ such that when $n \geq n_0$, the first and third terms on the right-hand side
are each $<\epsilon$. For the second term on the right-hand side, we use
\[
|f_{a(n)}(x)-f_{a(n)}(x_0)| \leq |f_{a(n)}(x)-f_{a(n)}(x_{a(n)})| + |f_{a(n)}(x_{a(n)})-f_{a(n)}(x_0)|.
\]
But $f_{a(n)} \in E_N$, so this is $\leq$
\[
N|x-x_{a(n)}| + N|x_{a(n)}-x_0|.
\]
Putting everything together, for $n \geq n_0$ we have
\[
|f(x)-f(x_0)| < 2\epsilon + N|x-x_{a(n)}| + N|x_{a(n)}-x_0|.
\]
Because $x_{a(n)} \to x_0$, we get
\[
|f(x)-f(x_0)| \leq 2\epsilon + N|x-x_0|.
\]
But this is true for any $\epsilon>0$, so
\[
|f(x)-f(x_0)| \leq N|x-x_0|,
\]
showing that $f \in E_N$.
\end{proof}

For $M \in \mathbb{N}$  let $P_M$ be the set of those $f \in C([0,1])$ that are piecewise linear and whose line segments have slopes
with absolute value $\geq M$. If $M, N \in \mathbb{N}$, $M>N$, and $f \in P_M$, then for any $x_0 \in [0,1]$, this $x_0$ is the abscissa of a point
on at least one line segment whose
slope has absolute value $\geq M$ (the point will be on two line segments when it is their common endpoint), and then there is another point on this line segment,
with abscissa $x$, such that $|f(x)-f(x_0)| \geq M|x-x_0|>N|x-x_0|$, and the fact that for every $x_0 \in [0,1]$ there is such $x \in [0,1]$ means that
$f \not \in E_N$. Therefore, if $M>N$ then $P_M \cap E_N = \emptyset$. 

\begin{lemma}
For each $M \in \mathbb{N}$, $P_M$ is dense in $C([0,1])$. 
\end{lemma}
\begin{proof}
Let $f \in C([0,1])$ and $\epsilon>0$. Because $f$ is continuous on the compact set $[0,1]$ it is uniformly continuous, so there is some
positive integer $n$ such that $|x-y| \leq \frac{1}{n}$ implies that $|f(x)-f(y)| \leq \epsilon$. We define $g:[0,1] \to \mathbb{R}$
to be linear on the intervals $[\frac{k}{n},\frac{k+1}{n}]$, $k=0,\ldots,n-1$ and to satisfy
\[
g\left(\frac{k}{n}\right)=f\left(\frac{k}{n}\right), \qquad k=0,\ldots,n.
\]
This nails down $g$, and for any $x \in [0,1]$ there is some $k=0,\ldots,n-1$ such that $x$ lies in the interval $[\frac{k}{n},  \frac{k+1}{n}]$.
But since $g$ is linear on this interval and we know its values at the endpoints, for any $y$ in this interval we have
\begin{eqnarray*}
g(y)& =& \frac{f\left(\frac{k+1}{n}\right)-f\left(\frac{k}{n}\right)}{\frac{k+1}{n}-\frac{k}{n}} y + f\left(\frac{k}{n}\right)-
 \frac{f\left(\frac{k+1}{n}\right)-f\left(\frac{k}{n}\right)}{\frac{k+1}{n}-\frac{k}{n}} \cdot \frac{k}{n}\\
 &=&n\left(f\left(\frac{k+1}{n}\right)-f\left(\frac{k}{n}\right)\right)y +f\left(\frac{k}{n}\right) - k\left(f\left(\frac{k+1}{n}\right)-f\left(\frac{k}{n}\right)\right),
\end{eqnarray*}
so 
\begin{eqnarray*}
|g(x)-f(x)| &\leq& |g(x)-g(k/n)|+|g(k/n)-f(k/n)| +|f(k/n)-f(x)|\\
&=&|g(x)-f(k/n)|+|f(k/n)-f(x)|\\
&=&n\left|\left(f\left(\frac{k+1}{n}\right)-f\left(\frac{k}{n}\right)\right)\left(x- \frac{k}{n}\right)\right|+|f(k/n)-f(x)|\\
&\leq&\left|f\left(\frac{k+1}{n}\right)-f\left(\frac{k}{n}\right)\right|+|f(k/n)-f(x)|\\
&\leq&2\epsilon.
\end{eqnarray*}
This is true for all $x \in [0,1]$, so
\[
\norm{g-f}_\infty \leq 2\epsilon.
\]

Now that we know that we can approximate any $f \in C([0,1])$ with continuous piecewise linear functions, we 
shall show that we can approximate any continuous piecewise linear function with  elements of $P_M$, from which it will follow
that $P_M$ is dense in $C([0,1])$. Let $g$ be a continuous piecewise linear function. We can write
$g$ in the following way: there is some positive integer $n$ and $a_0,\ldots,a_{n-1}, b_0,\ldots,b_{n-1} \in \mathbb{R}$ such that
 $g$ is linear on the intervals $[\frac{k}{n},\frac{k+1}{n}]$, $k=0,\ldots,n-1$, and 
satisfies $g(x)=a_kx+b_k$ for $x \in [\frac{k}{n},\frac{k+1}{n}]$; this can be satisfied  precisely when
$a_k \frac{k+1}{n}+b_k = a_{k+1}\frac{k+1}{n}+b_{k+1}$ for each $k=0,\ldots,n-1$. 
For $\epsilon>0$,
let
\[
\phi_\epsilon(x)=g(x)+\epsilon, \qquad \psi_\epsilon(x)=g(x)-\epsilon, \qquad x \in [0,1].
\]
We shall define a function $h:[0,1] \to \mathbb{R}$ by describing its graph.
We start at $(0,g(0))$, and then the graph of $h$ is a line segment of slope $M$ until it intersects the graph of $\phi_\epsilon$, at which
point the graph of $h$ is a line segment of slope $-M$ until it intersects the graph of $\psi_\epsilon$. We repeat this until we hit
the point $(\frac{1}{n},h(\frac{1}{n}))$; we remark that it need not be the case that  $h(\frac{1}{n})=g(\frac{1}{n})$. If $(\frac{1}{n},h(\frac{1}{n}))$
lies on the graph of $\phi_\epsilon$ then we start a line segment of slope $-M$, and if it lies on the graph
of $\psi_\epsilon$ then we start a line segment of slope $M$, and otherwise we continue the existing line segment until it intersects
$\phi_\epsilon$ or $\psi_\epsilon$ and we repeat this until the point $(\frac{2}{n},h(\frac{2}{n}))$, and then repeat this procedure.
This constructs a function $h \in P_M$ such that $\norm{h-g}_\infty \leq \epsilon$. But for any $f \in C([0,1])$ and $\epsilon>0$,
we have shown that there is some continuous piecewise linear $g$ such that $\norm{g-f}_\infty < \epsilon$, and now we know that there
is some $h \in P_M$ such that $\norm{h-g}_\infty<\epsilon$, so $\norm{h-f}_\infty < 2 \epsilon$, showing that $P_M$ is dense in
$C([0,1])$.
\end{proof}

Let $N \in \mathbb{N}$,
suppose that $f \in E_N$, and let $\epsilon>0$. Let $M>N$, and because $P_M$ is dense in $C([0,1])$,
there is some $h \in P_M$ such that $\norm{f-h}_\infty<\epsilon$.
But $P_M \cap E_N = \emptyset$ because $M>N$, so $h \not \in E_N$, showing that there is no open ball
with center $f$ that is contained in $E_N$, which shows that $E_N$ has empty interior.
But we have shown that $E_N$ is closed, so the interior of the closure of $E_N$ is empty, namely,
$E_N$ is nowhere dense, which completes the proof.


\section{The Baire property}
Suppose that $X$ is a topological space and that $\mathscr{I}$ is the $\sigma$-ideal of meager sets in $X$. 
For $A,B \subset X$, write
\[
A \bigtriangleup B =(A \setminus B) \cup (B \setminus A).
\]
We write $A=^* B$ if $A \bigtriangleup B \in \mathscr{I}$.
One proves that if $A=^* B$ then $X \setminus A =^* X \setminus B$, and that if
$A_n=^* B_n$ then $\bigcap_{n \in \mathbb{N}} A_n =^* \bigcap_{n \in \mathbb{N}} B_n$ and
$\bigcup_{n \in \mathbb{N}} A_n =^* \bigcup_{n \in \mathbb{N}} B_n$.
A subset $A$ of $X$ is said to have the \textbf{Baire property} if
there is an open set $U$ such that $A=^* U$. (It is a common practice to talk about things that are equal to a thing that is somehow easy to work with
modulo things that are considered small.) The following theorem characterizes the collection of subsets with the Baire property of a topological space.\footnote{Alexander S. Kechris, {\em Classical Descriptive Set Theory}, p.~47, Proposition 8.22.}

\begin{theorem}
Let $X$ be a topological space and let $\mathscr{B}$ be the collection of subsets of $X$ with the Baire property. Then $\mathscr{B}$ is a $\sigma$-algebra
on $X$, and is the algebra generated by all open sets and all meager sets.
\end{theorem}
\begin{proof}
If $F$ is closed, then
$F \setminus \Int(F)$ is closed and has empty interior, so is nowhere dense and therefore meager. Thus,
if $F$ is closed then $F=^* \Int(F)$.

$\emptyset =^* \emptyset$ and  $\emptyset$ is open so $\emptyset$ has the Baire
property, and so belongs to $\mathscr{B}$. 
Suppose that $B \in \mathscr{B}$. This means that there is some open set $U$ such that $B=^* U$,
which implies that $X \setminus B =^* X \setminus U$. But $X \setminus U$ is closed, hence
$X \setminus U =^* \Int(X \setminus U)$, so $X \setminus B =^* \Int(X \setminus U)$. As
$\Int(X \setminus U)$ is open, this shows that $X \setminus B$ has the Baire property, that is, $X \setminus B \in \mathscr{B}$. 

Suppose that $B_n \in \mathscr{B}$. So there are open sets $U_n$ such that
$B_n =^* U_n$, and it follows that
$\bigcup_{n \in \mathbb{N}} B_n =^* \bigcup_{n \in \mathbb{N}} U_n$. The union on the right-hand side is open,
so $\bigcup_{n \in \mathbb{N}}$ has the Baire property and thus belongs to $\mathscr{B}$. This shows that $\mathscr{B}$
is a $\sigma$-algebra. 

Suppose that $\mathscr{A}$ is an algebra containing all open sets and all meager sets, and let
$B \in \mathscr{B}$. Because $B$ has the Baire property there is some open set $U$ such that
$B=^* U$, which means that $M=B \bigtriangleup U=(B\setminus U) \cup (U \setminus B)$ is meager. 
But $B = M \bigtriangleup U = (M \setminus U) \cup (U \setminus M)$, and because $\mathscr{A}$ is an algebra
and $U,M \in \mathscr{A}$ we get $B \in \mathscr{A}$, showing that $\mathscr{B} \subset \mathscr{A}$. 
\end{proof}



If $X_n$ is a sequence of sets, we call $A \subset \prod_{n \in \mathbb{N}} X_n$ a \textbf{tail set}
if for all $(x_n) \in A$ and $(y_n) \in \prod_{n \in \mathbb{N}} X_n$, $\{n \in \mathbb{N}: y_n \neq x_n\}$ 
being finite implies that $(y_n) \in A$. The following theorem states is a \textbf{topological zero-one law},\footnote{Alexander S. Kechris, {\em Classical Descriptive Set Theory}, p.~55, Theorem 8.47.} whose proof uses the \textbf{Kutatowski-Ulam theorem},\footnote{Alexander S. Kechris, {\em Classical Descriptive Set Theory}, p.~53, Theorem 8.41.} which is about
meager sets in a product of two second-countable topological spaces. Since, from the Baire category theorem, any completely metrizable space is a Baire space and
a separable metrizable space is second-countable,
we can in particular use the following theorem when the  $X_n$ are Polish spaces.

\begin{theorem}
Suppose that $X_n$ is a sequence of second-countable Baire spaces. If $A \subset \prod_{n \in \mathbb{N}} X_n$ has the Baire
property and is a tail set, then $A$ is either meager or comeager.
\end{theorem}


\end{document}