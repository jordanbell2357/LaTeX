\documentclass{article}
\usepackage{amsmath,amssymb,graphicx,subfig,mathrsfs,amsthm}
\usepackage{tikz-cd}
\newcommand{\HSnorm}[1]{\left\Vert #1 \right\Vert_{\textrm{HS}}}
\newcommand{\HSinner}[2]{\left\langle #1, #2 \right\rangle_{\textrm{HS}}}
\newcommand{\inner}[2]{\langle #1, #2 \rangle}
\newcommand{\jap}[1]{\left\langle #1 \right\rangle}
\newcommand{\alg}{\otimes_{\textrm{alg}}} 
\newcommand{\HS}{\otimes_{\textrm{HS}}} 
\newcommand{\tr}{\textrm{tr}} 
\newcommand{\Span}{\textrm{span}} 
\newcommand{\id}{\textrm{id}} 
\newcommand{\Hom}{\textrm{Hom}} 
\newcommand{\norm}[1]{\left\Vert #1 \right\Vert}
\newtheorem{theorem}{Theorem}
\newtheorem{lemma}[theorem]{Lemma}
\newtheorem{corollary}[theorem]{Corollary}
\begin{document}
\title{Projective limits of topological vector spaces}
\author{Jordan Bell\\ \texttt{jordan.bell@gmail.com}\\Department of Mathematics, University of Toronto}
\date{\today}

\maketitle


\section{Definitions}
Let $(I,\preceq)$ be a {\em directed poset}: if $i \in I$ then $i \preceq i$; if $i,j,k \in I$, $i \preceq j$, and $j \preceq k$, then $i \preceq k$: if $i,j \in I$, $i \preceq j$, and $j \preceq i$, then $i=j$; if
$i,i \in I$, then there is some $k \in I$ such that $i \preceq k$ and $j \preceq k$.\footnote{I am using the first chapter of L. Ribes and Z. Zalesskii, {\em Profinite Groups}, which has
a clean presentation of projective and direct limits, and
also Paul Garrett's notes {\em Functions on circles} and {\em Basic categorical constructions}, which are on his homepage. I have not used them,
but J. L. Taylor, {\em Notes on locally convex topological vector spaces} looks readable and comprehensive.}

A {\em projective system} in a category $\mathfrak{C}$ is a set $\{X_i:i \in I\}$ of objects in $\mathfrak{C}$ and a set $\{\phi_{ij}:X_i \to X_j:i,j \in I, i \succeq j\}$ of morphisms, such that
if $i \in I$ then $\phi_{ii}=\id_{X_i}$  and such that
if $i,j,k \in I$ and $k \preceq j \preceq i$ then
\begin{center}
\begin{tikzcd}[column sep=small]
X_i    \arrow{dr}[swap]{\phi_{ij}} \arrow{rr}{\phi_{ik}}    &         & X_k \\
   &                  X_j \arrow{ur}[swap]{\phi_{jk}}     & 
\end{tikzcd}
\end{center}
commutes.
If $Y$ is an object in $\mathfrak{C}$ and for each $i \in I$, $\psi_i:Y \to X_i$ is a morphism, we say that the morphisms $\psi_i$ are {\em compatible}
with the projective system $\{X_i,\phi_{ij},I\}$ if $\phi_{ij} \circ \psi_i = \psi_j$ whenever $j \preceq i$, that is, 
\begin{center}
\begin{tikzcd}[column sep=small]
X_i   \arrow{rr}{\phi_{ij}}    &         & X_j \\
   &                 Y \arrow{ul}{\psi_i} \arrow{ur}[swap]{\psi_j}     & 
\end{tikzcd}
\end{center}
commutes.

Suppose that $X$ is an object in $\mathfrak{C}$ and the morphisms $\phi_i:X \to X_i$ are compatible with the projective system $\{X_i,\phi_{ij},I\}$.
We say that $X$ is a 
{\em projective limit} of the projective system if whenever $Y$ is an object in $\mathfrak{C}$ and there are morphisms
 $\psi_i:Y \to X_i$ that are compatible with the projective
system, there is one and only one morphism
$\psi:Y \to X$ such that
\[
\phi_i \circ \psi = \psi_i
\]
for all $i \in I$. This is a definition of a projective limit by a {\em universal 
property}, but it is not clear which projective systems have projective limits; but any object and morphisms that satisfy this property will be called
a projective limit.

\section{Direct products of topological vector spaces}
A {\em topological vector space} is a vector space $V$ over $\mathbb{C}$ that is a Hausdorff topological space, and such that addition
$V \times V \to V$ is continuous and scalar multiplication $\mathbb{C} \times V \to V$ is continuous.

Let $I$ be a set and let $V_i$ be topological vector spaces, where $\alpha_i:V_i \times V_i \to V_i$ is the addition map and
$\mu_i:\mathbb{C} \times V_i \to V_i$ is the scalar multiplication map. Let
\[
V=\prod_{i \in I} V_i = \Big\{v:I \to \bigcup_{i \in I} V_i : \textrm{if $i \in I$ then $v_i \in V_i$} \Big\},
\]
the Cartesian product
of the sets $V_i$, and let $V$ have the product topology. $V$ is a Hausdorff topological space. 
For each $i \in I$, define
$p_i:V \to V_i$ 
by $p_i(v)=v_i$, the projection from $V$ to $V_i$; 
$p_i$ is continuous. 

Define $\alpha:V \times V \to V$ by $\alpha(v,w)_i=\alpha_i(v_i,w_i)$, and define $\mu:\mathbb{C} \times V \to V$ by
$\mu(\lambda,v)_i=\mu_i(\lambda, v_i)$. $V$ is a vector space with addition $\alpha$ and scalar multiplication $\mu$. 
Let $U_i$ be an open set in $V_i$ and let  $U=p_i^{-1}(U_i)$.
Let $W_j=\alpha_i^{-1}(U_i)$ for $j=i$ and $W_j=V_j \times V_j$ for $j \neq i$, and let
$W=\prod_{j \in I} W_j$: $W=\alpha^{-1}(U)$ and $W$ is an open set in $V \times V$. Since $\alpha^{-1}(U)$ is an open set in 
$V \times V$ for each subbasic open set $U$ in $V$, it follows that $\alpha$ is continuous. 
Moreover, let $U_i$ be an open set in $V_i$  and let  $U=p_i^{-1}(U_i)$.
Let $W_j=\alpha_i^{-1}(U_i)$ for $j=i$ and $W_j=\mathbb{C} \times V_j$ for $j \neq i$, and let $W=\prod_{j \in I} W_j$:
$W=\mu_i^{-1}(U)$ and $W$ is an open set in $\mathbb{C} \times V$. It follows that $\mu$ is continuous.
Therefore $V$ is a topological vector space.\footnote{Paul Garrett in his notes gives a diagrammatic proof that
the direct product of topological vector spaces is a topological vector space.}


\section{Projective limits of topological vector spaces}
Let $\mathfrak{C}$ be the category of topological vector spaces: an object of $\mathfrak{C}$ is a topological vector space, and a morphism $V \to W$ is a linear map that is continuous. 
Let $\{V_i,\phi_{ij},I\}$ be a projective system in $\mathfrak{C}$. 
Let
\[
X= \prod_{j \in I} V_j,
\]
and 
let $p_i: X \to V_i$ be the projection map.
Define
\[
V=\Big\{v \in X: \textrm{if $i,j \in I$ and $j \preceq i$ then $\phi_{ij}(p_i(v))=p_j(v)$}   \Big\}.
\]
$V$ is a Hausdorff topological space with the subspace topology. Since both $p_i$ and $\phi_{ij}$ are linear, it follows that $V$ is a vector subspace of $X$. 
Let $\alpha:X \times X \to X$ be addition and
$\mu:\mathbb{C} \times X \to X$ be scalar multiplication,
and let $\iota: V \hookrightarrow X$ be the inclusion map. $\iota$  is continuous, so
$\iota \times \iota:V \times V \to  X \times X$
is continuous, and hence $\alpha \circ (\iota \times \iota):V \times V \to V$ 
and $\mu \circ (\iota \times \iota):V \times V \to V$ are continuous; that their codomains are $V$ follows from $V$ being a vector subspace of $X$. Thus $V$ is a topological vector space. 

For $i \in I$, define $\phi_i:V \to V_i$ by $\phi_i=p_i \circ \iota$, which is a morphism. 
If $j \preceq i$ then $\phi_{ij} \circ \phi_i=\phi_j$, which follows from the definition of $V$, so
$\phi_i:V \to V_i$ are compatible with the projective
system $\{V_i,\phi_{ij},I\}$.
Suppose that $Y$ is a topological vector space and $\psi_i:Y \to V_i$ are compatible with the projective system  $\{V_i,\phi_{ij},I\}$. Define 
$\psi:Y \to V$ by $\psi(y)_i=\psi_i(y)$; that the codomain of $\psi$ is $V$ follows from $\psi_i:Y \to V_i$ being compatible with the projective
system. Since each $\psi_i$ is linear, $\psi$ is linear. Let $U_i$ be open in $V_i$. We have
\[
\psi^{-1}(V \cap p_i^{-1}(U_i))=\psi_i^{-1}(U_i),
\]
hence $\psi^{-1}(U)$ for each subbasic open set $U$ in $V$, and so $\psi$ is continuous. Let $y \in Y$.
\[
\phi_i(\psi(y))=p_i (\iota ( \psi(y)))=\psi(y)_i=\psi_i(y),
\]
so $\phi_i \circ \psi = \psi_i$ for each in $i \in I$. 
Suppose that $\chi:Y \to V$ is a morphism such that 
\[
\phi_i \circ \chi=\psi_i
\]
for all $i \in I$. On the one hand $\phi_i(\chi(y))=\chi(y)_i$, and on the other hand $\psi_i(y)=\psi(y)_i$, so
$\phi_i \circ \chi = \psi$ for all $i \in I$. Therefore $V$ is a projective limit of the projective system
 $\{V_i,\phi_{ij},I\}$.
 
 In any category: Let $\{X_i,\phi_{ij},I\}$ be a projective system. If $(X,\phi_i)$ and $(Y,\psi_i)$ are projective limits, then there
 is a unique isomorphism $\phi:X \to Y$ such that $\psi_i \circ \phi = \phi_i$ for all $i$. This can be cleanly proved in a straightforward way; a proof
 is written down in Ribes and Zalesskii for the category of topological spaces, but it doesn't use anything that is special to that category.\footnote{Ribes and Zalesskii's book is about profinite groups, which are projective limits of finite discrete
topological groups; indeed there exist projective limits in the category of topological groups. An example of an infinite profinite group is
the $p$-adic integers $\mathbb{Z}_p$, for $p$ prime, which is the projective limit of the groups $\mathbb{Z}/p^n$, where the map
$\mathbb{Z}/p^n \to \mathbb{Z}/p^m$, $n \geq m$, is taking the remainder modulo $p^m$.}

Thus in the category of topological vector spaces, a projective limit of a projective system exists and is unique up to unique isomorphism.
We denote the projective limit of a projective system $\{V_i,\phi_{ij},I\}$ of topological vector spaces by
\[
\varprojlim_{i \in I} V_i.
\]

Let $v \in \prod_{i \in I} V_i \setminus \varprojlim_{i \in I} V_i$. Since $v \not \in  \varprojlim_{i \in I} V_i$, there are
some $r,s \in I, r \succeq s$ such that $\phi_{rs}(v_r) \neq v_s$. As $\phi_{rs}(v_r)$ and $v_s$ are distinct
points in the Hausdorff space $V_s$, there are disjoint open sets $U_1,U_2$ in $V_s$ with $\phi_{rs}(v_r) \in U_1$
and $v_s \in U_2$. Let $U_1'=\phi_{rs}^{-1}(U_1)$; this is an open set in $V_r$. Define $W_r=U_1'$, $W_s=U_2$, and
$W_i = V_i$ if $i \neq r,s$. $\prod_{i \ in I} W_i$ is open in $\prod_{i \in I} V_i$, and $w \in \prod_{i \in I} W_i$. Let
$w \in \prod_{i \in I} W_i$. $\phi_{rs}(w_r) \in U_1$ and $w_s \in U_2$, and $U_1$ and $U_2$ are disjoint, so
$\phi_{rs}(w_r) \neq w_s$. Thus $w \not \in \varprojlim_{i \in I} V_i$. Therefore, $\varprojlim_{i \in I} V_i$ is a closed
subset of $\prod_{i \in I} V_i$. 



\section{Locally convex topological vector spaces}
We say that a topological vector space $V$ is {\em locally convex} if for every open neighborhood $U$ of $0$, there is some convex open $0 \in U' \subseteq U$.
Let $(V_j,\phi_{ij},I)$ be a projective system of locally convex topological vector spaces and let $V$ be the projective limit in the category of topological vector spaces. 
We shall show that $V$ is locally convex.
Let $J \subset I$ be finite and let $U_j$ be open in $V_j$ for each $j \in J$. Since $V_j$ is locally convex, for each $j \in J$ there is some convex open $U_j' \subseteq U_j$ such that $0 \in U_j'$. Define $W_j = U_j'$ for $j \in J$ and $W_i=V_i$ for $i \in I \setminus J$, and let $W = \prod_{i \in I} W_i$. $W$ is open and convex. This shows that
$\prod_{i \in I} V_i$ is locally convex. It is apparent that a vector subspace of a locally convex topological vector space is itself a locally convex topological vector
space with the product topology. Thus $V$ is locally convex. It follows that a projective system in the category of locally convex topological vector spaces has a
projective limit that is unique up to unique isomorphism. (A morphism of locally convex topological vector spaces is the same as a morphism of topological vector spaces.) 

\section{Cofinality}
Let $(I,\preceq)$ be a directed poset, and suppose that $J \subseteq I$ is also  directed with $\preceq$.\footnote{This section follows Paul Garrett's notes
{\em Basic categorical constructions}.} We say that $J$ is {\em cofinal} in $I$ if 
for all $i \in I$ there is some $j \in J$ such that $i \preceq j$. Let $\mathfrak{C}$ be a category in which there exists a projective
limit of any projective system; we have shown above that the category of topological vector spaces and the category
of locally convex topological vector spaces are such categories. This projective limit will be unique up to unique isomorphism.  

Let $(X_i,\phi_{ij},I)$ be a projective system and let $J$ be cofinal in $I$. Then $(X_i,\phi_{ij},J)$ is a projective system. Let $(X_I,\phi_i,I)$ be a projective limit of
the first system and let $(X_J,\phi_i',J)$ be a projective limit of the second system. Let $i \in I$,  take $j \in J$ with $j \succeq i$, and define $\psi_i:X_J \to X_i$ by 
$\psi_i=\phi_{ji} \circ \phi_j'$; if $j,j' \succeq i$ then $\phi_{j'i} \circ \phi_{j'}'=\phi_{ji} \circ \phi_j'$. Let $i',i \in I$ with $i' \preceq i$, and let $j \in J$ with $i \preceq j$. 
Because $(X_i,\phi_{ij},I)$ is  a projective system, $\phi_{ii'} \circ\phi_{ji}=\phi_{ji'}$, and hence
\[
\phi_{ii'} \circ \psi_i=\phi_{ii'} \circ \phi_{ji} \circ \phi_j'=\phi_{ji'} \circ\phi_j'=\psi_{i'},
\]
showing that the morphisms $\psi_i:X_J \to X_i$ are compatible with the projective
system $(X_i,\phi_{ij},I)$. As $(X_I,\phi_i,I)$ is a projective limit of
 $(X_i,\phi_{ij},I)$,
it follows that there is a unique morphism $\psi:X_J \to X_I$
such that $\phi_i \circ \psi= \psi_i$ for all $i \in I$.

Since $\phi_i:X_I \to X_i$, $i \in I$, are compatible with the projective system
$(X_i,\phi_{ij},I)$, it follows that $\phi_j:X_I \to X_j$, $j \in J$, are compatible
with $(X_i,\phi_{ij},J)$. As $(X_J,\phi_i',J)$ is a projective limit of $(X_i,\phi_{ij},I)$, 
there is a unique morphism $\phi:X_I \to X_J$ such that $\phi_j' \circ \phi = \phi_j$ for all $j \in J$. 

On the one hand,
$\phi \circ \psi: X_J \to X_J$ is a morphism such that $\phi_j' \circ (\phi \circ \psi) = \phi_j'$ for all $j \in J$. On the other hand,
$\id_{X_J}:X_J \to X_J$ is a morphism such that $\phi_j' \circ \id_{X_J} = \phi_j'$ for all $j \in J$. By the definition of projective limit, we get
$\phi \circ \psi=\id_{X_J}$. Likewise, $\psi \circ \phi = \id_{X_I}$. Thus $\psi:X_J \to X_I$ and $\phi:X_I \to X_J$ are  isomorphisms.
Since $X_J$ and $X_I$ are isomorphic, 
$(X_J,\psi_i,I)$
is a projective limit of $(X_i,\phi_{ij},I)$. This is similar to how the limit of a convergent sequence is the same as the limit of any infinite subsequence.

\end{document}
