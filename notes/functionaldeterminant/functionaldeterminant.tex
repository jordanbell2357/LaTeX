\documentclass{article}
\usepackage{amsmath,amssymb,graphicx,subfig,mathrsfs,amsthm,siunitx}
%\usepackage{tikz-cd}
\usepackage{hyperref}
\newcommand{\inner}[2]{\left\langle #1, #2 \right\rangle}
\newcommand{\tr}{\ensuremath\mathrm{tr}\,} 
\newcommand{\Span}{\ensuremath\mathrm{span}} 
\def\Re{\ensuremath{\mathrm{Re}}\,}
\def\Im{\ensuremath{\mathrm{Im}}\,}
\newcommand{\id}{\ensuremath\mathrm{id}} 
\newcommand{\sgn}{\ensuremath\mathrm{sgn}\,} 
\newcommand{\rank}{\ensuremath\mathrm{rank\,}} 
\newcommand{\co}{\ensuremath\mathrm{co}\,} 
\newcommand{\cco}{\ensuremath\overline{\mathrm{co}}\,}
\newcommand{\supp}{\ensuremath\mathrm{supp}\,}
\newcommand{\epi}{\ensuremath\mathrm{epi}\,}
\newcommand{\lsc}{\ensuremath\mathrm{lsc}\,}
\newcommand{\ext}{\ensuremath\mathrm{ext}\,}
\newcommand{\cl}{\ensuremath\mathrm{cl}\,}
\newcommand{\dom}{\ensuremath\mathrm{dom}\,}
\newcommand{\LSC}{\ensuremath\mathrm{LSC}}
\newcommand{\USC}{\ensuremath\mathrm{USC}}
\newcommand{\Cyl}{\ensuremath\mathrm{Cyl}\,}
\newcommand{\extreals}{\overline{\mathbb{R}}}
\newcommand{\upto}{\nearrow}
\newcommand{\downto}{\searrow}
\newcommand{\norm}[1]{\left\Vert #1 \right\Vert}
\newtheorem{theorem}{Theorem}
\newtheorem{lemma}[theorem]{Lemma}
\newtheorem{proposition}[theorem]{Proposition}
\newtheorem{corollary}[theorem]{Corollary}
\theoremstyle{definition}
\newtheorem{definition}[theorem]{Definition}
\newtheorem{example}[theorem]{Example}
\begin{document}
\title{The functional determinant}
\author{Jordan Bell\\ \texttt{jordan.bell@gmail.com}\\Department of Mathematics, University of Toronto}
\date{\today}

\maketitle

\section{Gaussians}
Let $A \in \mathscr{B}(\mathbb{R}^n)$ have   positive spectrum. Because $A$ is positive, it has a unique positive square root $\sqrt{A}$, which also
has positive spectrum. Then, using the fact  that $\int_{\mathbb{R}} \exp(-x^2) dx=\sqrt{\pi}$,
\begin{eqnarray*}
\int_{\mathbb{R}^n} \exp(-\pi \inner{x}{Ax}) dx&=&\int_{\mathbb{R}^n} \exp\left(-\pi \inner{x}{\sqrt{A}\sqrt{A}x}\right) dx\\
&=&\int_{\mathbb{R}^n} \exp\left(-\inner{\sqrt{\pi} \sqrt{A}x}{\sqrt{\pi} \sqrt{A}x}\right) dx\\
&=&\int_{\mathbb{R}^n} \exp(-\inner{x}{x}) |\det (\sqrt{\pi} \sqrt{A})^{-1}| dx\\
&=&\frac{1}{\det(\sqrt{\pi} I)} \frac{1}{\det(\sqrt{A})} \int_{\mathbb{R}^n} \exp(-\inner{x}{x}) dx\\
&=&\pi^{-d/2} (\det A)^{-1/2}\int_{\mathbb{R}^n} \exp(-\inner{x}{x}) dx\\
&=&(\det A)^{-1/2};
\end{eqnarray*}
$\frac{1}{\det(\sqrt{A})}=(\det A)^{-1/2}$ because $A=\sqrt{A}\sqrt{A}$ and $\det(\sqrt{A})>0$.



\section{Zeta functions}
Suppose that $A \in \mathscr{B}(\mathbb{R}^n)$ has positive spectrum: $0<\lambda_1 \leq \cdots \leq \lambda_n$.
Define
\[
\zeta_A(s) = \sum_{k=1}^n \frac{1}{\lambda_k^s}=\sum_{k=1}^n \exp(-s\log \lambda_k), \qquad s \in \mathbb{C}.
\]
As
\[
\zeta_A'(s) = \sum_{k=1}^n -\log (\lambda_k) \exp(-s\log \lambda_k)
=\sum_{k=1}^n -\frac{\log \lambda_k}{\lambda_k^s},
\]
we have
\[
-\zeta_A'(0)=\sum_{k=1}^n \log\lambda_k,
\]
 hence
\[
\exp(-\zeta_A'(0)) = \prod_{k=1}^n \lambda_k=\det A.
\]

\section{Further reading}
Leon A. Takhtajan, {\em Quantum Mechanics for Mathematicians}, p.~262.

Nicole Berline, Ezra Getzler and Mich\`ele Vergne, {\em Heat Kernels and Dirac Operators}, p.~296.

J\"urgen Jost, {\em Geometry and Physics}, p.~101.

Eberhard Zeidler, {\em Quantum Field Theory II: Quantum Electrodynamics}, p.~570.

John Baez, Week 127, \url{http://math.ucr.edu/home/baez/week127.html}

Klaus Kirsten, {\em Functional determinants in higher dimensions using contour integrals}, \url{http://arxiv.org/abs/1005.2595},
and {\em Basic zeta functions and some applications in physics}, \url{http://arxiv.org/abs/1005.2389}

H. Kumagai, {\em The determinant of the Laplacian on the $n$-sphere}, Acta Arith. \textbf{91} (1999), no. 3, 199-208. 

Predrag Cvitanovi\'c, {\em Spectral determinants}, \url{http://chaosbook.org/chapters/det.pdf}

Steven Rosenberg, {\em The Laplacian on a Riemannian Manifold: An Introduction to Analysis on Manifolds}, Chapter 5.

\end{document}