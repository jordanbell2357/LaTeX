\documentclass{article}
\usepackage{amsmath,amssymb,graphicx,subfig,mathrsfs,amsthm,siunitx}
%\usepackage{tikz-cd}
%\usepackage{hyperref}
\newcommand{\inner}[2]{\left\langle #1, #2 \right\rangle}
\newcommand{\tr}{\ensuremath\mathrm{tr}\,} 
\newcommand{\Span}{\ensuremath\mathrm{span}} 
\def\Re{\ensuremath{\mathrm{Re}}\,}
\def\Im{\ensuremath{\mathrm{Im}}\,}
\newcommand{\id}{\ensuremath\mathrm{id}} 
\newcommand{\sgn}{\ensuremath\mathrm{sgn}\,} 
\newcommand{\rank}{\ensuremath\mathrm{rank\,}} 
\newcommand{\Res}{\ensuremath\mathrm{Res}} 
\newcommand{\diam}{\ensuremath\mathrm{diam}} 
\newcommand{\supp}{\ensuremath\mathrm{supp}\,}
\newcommand{\sech}{\ensuremath\mathrm{sech}\,}
\newcommand{\dom}{\ensuremath\mathrm{dom}\,}
\newcommand{\upto}{\nearrow}
\newcommand{\downto}{\searrow}
\newcommand{\norm}[1]{\left\Vert #1 \right\Vert}
\newtheorem{theorem}{Theorem}
\newtheorem{lemma}[theorem]{Lemma}
\newtheorem{proposition}[theorem]{Proposition}
\newtheorem{corollary}[theorem]{Corollary}
\theoremstyle{definition}
\newtheorem{definition}[theorem]{Definition}
\newtheorem{example}[theorem]{Example}
\begin{document}
\title{The Fourier transform of holomorphic functions}
\author{Jordan Bell\\ \texttt{jordan.bell@gmail.com}\\Department of Mathematics, University of Toronto}
\date{\today}

\maketitle

For $f \in L^1(\mathbb{R})$, define
\[
\widehat{f}(\xi) = \int_{-\infty}^\infty e^{-2\pi i\xi x} f(x) dx,
\qquad \xi \in \mathbb{R}.
\]

For $a>0$, write
\[
S_a = \{z \in \mathbb{C}: |\Im z|<a\}.
\]
We define $\mathfrak{F}_a$ to be the set of functions $f$ that are holomorphic on $S_a$ and for which there
is some $A>0$ such that
\begin{equation}
|f(x+iy)| \leq \frac{A}{1+x^2}, \qquad x+iy \in S_a.
\label{Sa}
\end{equation}
For example, for $f(z)=e^{-\pi z^2}$, 
\[
|f(z)| = |e^{-\pi(x+iy)^2}| = |e^{-\pi x^2 - 2\pi i xy + \pi y^2}|
=e^{-\pi x^2} e^{\pi y^2},
\]
and for any $a>0$, $f \in S_a$. 

The following is from Stein and Shakarchi.\footnote{Elias M. Stein and Rami Shakarchi, {\em Complex Analysis},
p.~114, Theorem 2.1.}

\begin{theorem}
If $a>0$ and $f \in \mathfrak{F}_a$, then for any $0 \leq b < a$,
\[
\widehat{f}(\xi) = e^{-2\pi |\xi| b} \int_{-\infty}^\infty e^{-2\pi i\xi x} f(x-i\cdot  \sgn \xi \cdot b) dx,
\qquad \xi \in \mathbb{R}.
\]
\label{bleqa}
\end{theorem}
\begin{proof}
If $b=0$ then the claim is immediate. If $0<b<a$, 
we define $g(z)=e^{-2\pi i\xi z} f(z)$.
Because $f \in \mathfrak{F}_a$ there is some $A>0$ such that $f$ satisfies \eqref{Sa}. 
We
prove the claim separately for $\xi \geq 0$ and $\xi \leq 0$. 
For $\xi \geq 0$, with $R>0$,
\begin{align*}
\left| \int_{-R-ib}^{-R} g(z) dz \right|&\leq \int_{-R-ib}^{-R} |e^{-2\pi i\xi z} f(z)| dz\\
&=\int_{-b}^0 |e^{-2\pi i\xi(-R+iy)} f(-R+iy)| dy\\
&=\int_{-b}^0 e^{2\pi \xi y} |f(-R+iy)| dy\\
&\leq \int_{-b}^0 e^{2\pi \xi y} \frac{A}{1+R^2} dy\\
&=O(R^{-2})
\end{align*}
and likewise
\[
\left| \int_{R}^{R-ib} g(z) dz \right|=O(R^{-2}).
\]
$g$ is holomorphic on $S_a$, so by Cauchy's integral theorem, taking $R \to \infty$,
\[
\int_{-\infty}^\infty g(z) dz=
\int_{-\infty-ib}^{\infty-ib} g(z) dz,
\]
i.e.,
\begin{align*}
\widehat{f}(\xi)&=\int_{-\infty-ib}^{-\infty-ib} e^{-2\pi i\xi z} f(z)dz\\
&=\int_{-\infty}^\infty e^{-2\pi i\xi(x-ib)} f(x-ib) dx\\
&=e^{-2\pi \xi b} \int_{-\infty}^\infty e^{-2\pi i\xi x} f(x-ib) dx.
\end{align*}

For $\xi \leq 0$, with $R>0$,
\begin{align*}
\left| \int_{-R+ib}^{-R} g(z) dz \right|&\leq
\int_{-R}^{-R+ib}  |e^{-2\pi i\xi z} f(z)| dz\\
&=\int_{0}^{b} |e^{-2\pi i\xi(-R+iy)} f(-R+iy)| dy\\
&=\int_0^b e^{2\pi \xi y} |f(-R+iy)| dy\\
&\leq \int_0^b e^{2\pi \xi y} \frac{A}{1+R^2} dy\\
&=O(R^{-2}),
\end{align*}
and likewise
\[
\left| \int_{R}^{R+ib} g(z) dz \right|=O(R^{-2}).
\]
By Cauchy's integral theorem, taking $R \to \infty$,
\[
\int_{-\infty}^\infty g(z) dz = \int_{-\infty+ib}^{\infty+ib} g(z) dz,
\]
i.e.,
\begin{align*}
\widehat{f}(\xi)&= \int_{-\infty+ib}^{\infty+ib} e^{-2\pi i\xi z} f(z) dz\\
&=\int_{-\infty}^\infty e^{-2\pi i\xi(x+ib)} f(x+ib) dx\\
&=e^{2\pi \xi b} \int_{-\infty}^\infty e^{-2\pi i\xi x} f(x+ib) dx.
\end{align*}
\end{proof}



\begin{corollary}
If $a>0$ and $f \in \mathfrak{F}_a$, then for any $0 \leq b < a$ there is some $B$ such that
\[
|\widehat{f}(\xi)| \leq B e^{-2\pi b|\xi|}, \qquad \xi \in \mathbb{R}.
\]
\end{corollary}
\begin{proof}
Because $f \in \mathfrak{F}_a$ there is some $A>0$ such $f$ satisfies \eqref{Sa}. Put
\[
B=A\int_{-\infty}^\infty \frac{1}{1+x^2} dx = \pi A.
\]
By Theorem \ref{bleqa}, 
\begin{align*}
|\widehat{f}(\xi)| &\leq e^{-2\pi |\xi| b} \int_{-\infty}^\infty |e^{-2\pi i\xi x} f(x-i\cdot  \sgn \xi \cdot b)| dx\\
&=e^{-2\pi |\xi| b} \int_{-\infty}^\infty |f(x-i\cdot  \sgn \xi \cdot b)| dx\\
&\leq e^{-2\pi |\xi| b}  \int_{-\infty}^\infty \frac{A}{1+x^2} dx\\
&= e^{-2\pi |\xi| b} \cdot B.
\end{align*}
\end{proof}


Define
\[
\mathfrak{F} = \bigcup_{a>0} \mathfrak{F}_a.
\]
We now prove the \textbf{Fourier inversion formula} for functions belonging to $\mathfrak{F}$.\footnote{Elias M. Stein and Rami Shakarchi, {\em Complex Analysis},
p.~115, Theorem 2.2.}

\begin{theorem}
If $f \in \mathfrak{F}$, then
\[
f(x) = \int_{-\infty}^\infty e^{2\pi ix\xi} \widehat{f}(\xi) d\xi, \qquad x\in \mathbb{R}.
\]
\end{theorem}
\begin{proof}
Say $f \in \mathfrak{F}_a$, write
\[
\int_{-\infty}^\infty e^{2\pi i x\xi} d\xi = 
\int_{0}^\infty e^{2\pi i x\xi} d\xi+
\int_{-\infty}^0 e^{2\pi i x\xi} d\xi 
 = 
I_1+I_2,
\]
and take $0<b<a$. 
First we handle $I_1$. By Theorem \ref{bleqa}, for $\xi > 0$,
\[
\widehat{f}(\xi) = e^{-2\pi \xi b} \int_{-\infty}^\infty e^{-2\pi i\xi u} f(u-ib) du
=\int_{-\infty}^\infty e^{-2\pi i \xi(u-ib)} f(u-ib) du,
\]
with which, because $\xi b>0$,
\begin{align*}
\int_0^\infty e^{2\pi ix\xi} \widehat{f}(\xi) d\xi&=\int_0^\infty e^{2\pi ix\xi} \left(\int_{-\infty}^\infty e^{-2\pi i \xi(u-ib)} f(u-ib) du\right)
d\xi\\
&= \int_{-\infty}^\infty  f(u-ib) \int_0^\infty e^{-2\pi i\xi(u-ib-x)}  d\xi du\\
&=\int_{-\infty}^\infty  f(u-ib) \frac{1}{2\pi i(u-ib-x)} du\\
&=\frac{1}{2\pi i} \int_{L_1} \frac{f(\zeta)}{\zeta-x} d\zeta.
\end{align*}
where $L_1=\{u-ib: u \in \mathbb{R}\}$ traversed left to right. 
Now we handle $I_2$. By Theorem \ref{bleqa}, for $\xi < 0$,
\[
\widehat{f}(\xi)=e^{2\pi \xi b} \int_{-\infty}^\infty e^{-2\pi i \xi u} f(u+i  b) dx
=\int_{-\infty}^\infty e^{-2\pi i\xi(u+ib)} f(u+ib) dx,
\]
with which, because $\xi b<0$,
\begin{align*}
\int_{-\infty}^0 e^{2\pi ix \xi} \widehat{f}(\xi) d\xi&=\int_{-\infty}^0 e^{2\pi ix\xi} 
\left(\int_{-\infty}^\infty e^{-2\pi i\xi(u+ib)} f(u+ib) du \right) d\xi\\
&=\int_{-\infty}^\infty f(u+ib)  \int_{-\infty}^0 e^{-2\pi i\xi(u+ib-x)} d\xi du\\
&=\int_{-\infty}^\infty f(u+ib) \frac{-1}{2\pi i(u+ib-x)} du\\
&=-\frac{1}{2\pi i} \int_{L_2} \frac{f(\zeta)}{\zeta-x} d\xi,
\end{align*}
where $L_2=\{u+ib: u \in \mathbb{R}\}$ traversed left to right. Thus
\begin{equation}
\int_{-\infty}^\infty e^{2\pi ix \xi} \widehat{f}(\xi) d\xi = \frac{1}{2\pi i} \int_{L_1} \frac{f(\zeta)}{\zeta-x} d\zeta
-\frac{1}{2\pi i} \int_{L_2} \frac{f(\zeta)}{\zeta-x} d\xi.
\label{L1L2}
\end{equation}
Let $\gamma-R$ be the rectangle starting at $-R-ib$, going to $R-ib$, going to
$R+ib$, going to $-R+ib$, going to $-R-ib$. Because this rectangle and its interior are contained in $S_a$, on which $f$ is holomorphic,
by the residue theorem we have, for $R>|x|$,
\[
\int_{\gamma_R} \frac{f(\zeta)}{\zeta-x} d\zeta = 2\pi i \cdot \Res_{\zeta=x} \frac{f(\zeta)}{\zeta-x}=
2\pi i \cdot f(x).
\]
We estimate the integrand on the vertical sides of $\gamma_R$. For the left side, taking $A$ such that $f$ satisfies \eqref{Sa},
\[
\left| \int_{-R+ib}^{-R-ib} \frac{f(\zeta)}{\zeta-x} d\zeta \right|
\leq \int_{-b}^b \left| \frac{f(-R+iy)}{-R+iy-x} \right| dy
\leq \int_{-b}^b \frac{A}{1+R^2} \cdot \frac{1}{R-|x|} dy
=O(R^{-3}).
\]
For the right side,
\[
\left| \int_{R-ib}^{R+ib} \frac{f(\zeta)}{\zeta-x} d\zeta \right|
\leq \int_{-b}^b \left| \frac{f(R+iy)}{R+iy-x}\right| dy
\leq \int_{-b}^b \frac{A}{1+R^2} \cdot \frac{1}{R-|x|} dy
=O(R^{-3}).
\]
Thus, taking $R \to \infty$ we get
\[
\int_{L_1} \frac{f(\zeta)}{\zeta-x} d\zeta
-\int_{L_2} \frac{f(\zeta)}{\zeta-x} d\zeta
=2\pi i \cdot f(x),
\]
which by \eqref{L1L2} is
\[
\int_{-\infty}^\infty e^{2\pi ix \xi} \widehat{f}(\xi) d\xi  = f(x),
\]
proving the claim.
\end{proof}


We now prove the \textbf{Poisson summation formula}.\footnote{Elias M. Stein and Rami Shakarchi, {\em Complex Analysis},
p.~118, Theorem 2.4.}

\begin{theorem}
If $f \in \mathfrak{F}$, then
\[
\sum_{n \in \mathbb{Z}} f(n) = \sum_{n \in \mathbb{Z}} \widehat{f}(n).
\]
\end{theorem}
\begin{proof}
Say $f \in \mathfrak{F}_a$, take $0<b<a$, and for $N$ a positive integer
 let $\gamma_N$ be the rectangle starting at $-N-\frac{1}{2}-ib$, going to $N+\frac{1}{2}-ib$, going to
 $N+\frac{1}{2}+ib$, going to $-N-\frac{1}{2}+ib$, going to $-N-\frac{1}{2}-ib$. Because $f \in \mathfrak{F}_a$,
$\frac{f(z)}{e^{2\pi iz}-1}$ is meromorphic on a region containing $\gamma_N$ and its interior, and has poles
at $z=-N,\ldots,N$, with residues
\[
\Res_{z=n} \frac{f(z)}{e^{2\pi iz}-1}
=\frac{f(n)}{2\pi i e^{2\pi in}}
=\frac{f(n)}{2\pi i}.
\]
Thus the residue theorem gives us
\begin{equation}
\int_{\gamma_N} \frac{f(z)}{e^{2\pi iz}-1} dz = 2\pi i \sum_{|n| \leq N} \frac{f(n)}{2\pi i}
=\sum_{|n| \leq N} f(n).
\label{gammaN}
\end{equation}
For the left side of $\gamma_N$, with $z=-N-\frac{1}{2}+iy$, $-b \leq y \leq b$,
\[
|e^{2\pi iz}-1| = |e^{-2\pi iN - \pi i -2\pi y} - 1|
=|-e^{-2\pi y}-1| \geq 1,
\]
so, taking $A>0$ such that $f$ satisfies \eqref{Sa},
\begin{align*}
\left| \int_{-N-\frac{1}{2}+ib}^{-N-\frac{1}{2}-ib} \frac{f(z)}{e^{2\pi iz}-1} dz\right|&
\leq \int_{-b}^b \left| f\left(-N-\frac{1}{2}+iy\right) \right|dy\\
&\leq \int_{-b}^b \frac{A}{1+\left(-N-\frac{1}{2}\right)^2} dy\\
&=O(N^{-2}).
\end{align*}
Likewise,
\[
\left| \int_{N+\frac{1}{2}-ib}^{N+\frac{1}{2}+ib} \frac{f(z)}{e^{2\pi iz}-1} dz\right| = O(N^{-2}).
\]
Therefore, taking $N \to \infty$, \eqref{gammaN} becomes
\[
\int_{L_1}  \frac{f(z)}{e^{2\pi iz}-1} dz
- \int_{L_2}  \frac{f(z)}{e^{2\pi iz}-1} dz = \sum_{n \in \mathbb{Z}} f(n),
\]
where $L_1=\{x-ib: x \in \mathbb{R}\}$, traversed left to right, and $L_2=\{x+ib:x \in \mathbb{R}\}$, traversed left to right.
Then, as $b>0$,
\begin{align*}
\sum_{n \in \mathbb{Z}} f(n)&=\int_{L_1}  f(z) \frac{e^{-2\pi iz}}{1-e^{-2\pi iz}} dz
+ \int_{L_2} f(z) \frac{1}{1-e^{2\pi iz}} dz\\
&=\int_{L_1} f(z) e^{-2\pi iz} \sum_{n =0}^\infty (e^{-2\pi iz})^n dz
+\int_{L_2} f(z) \sum_{n=0}^\infty (e^{2\pi iz})^n\\
&=\sum_{n=0}^\infty \int_{L_1} e^{-2\pi i(n+1)z} f(z) dz
+\sum_{n=0}^\infty \int_{L_2} e^{2\pi inz} f(z) dz\\
&=\sum_{n=1}^\infty \int_{-\infty}^\infty e^{-2\pi in(x-ib)}f(x-ib) dx
+\sum_{n=0}^\infty \int_{-\infty}^\infty e^{2\pi in(x+ib)}f(x+ib) dx\\
&=\sum_{n=1}^\infty e^{-2\pi nb} \int_{-\infty}^\infty e^{-2\pi inx} f(x-ib) dx\\
&+\sum_{n=0}^\infty e^{-2\pi nb} \int_{-\infty}^\infty e^{2\pi inx} f(x+ib) dx.
\end{align*}
Using Theorem \ref{bleqa} this becomes
\[
\sum_{n \in \mathbb{Z}} f(n) = \sum_{n=1}^\infty \widehat{f}(n) + \sum_{n=0}^\infty \widehat{f}(-n)
=\sum_{n \in \mathbb{Z}} \widehat{f}(n),
\]
proving the claim.
\end{proof}

Take as granted that
\[
\int_{-\infty}^\infty e^{-2\pi i\xi x} e^{-\pi x^2} dx = e^{-\pi \xi^2}, 
\qquad \xi \in \mathbb{R}.
\]
For $t>0$ and $a \in \mathbb{R}$, with $y=t^{1/2}(x+a)$,
\begin{align*}
\int_{-\infty}^\infty e^{-2\pi i\xi x} e^{-\pi t(x+a)^2} dx&=\int_{-\infty}^\infty 
e^{-2\pi i\xi(t^{-1/2}y-a)} e^{-\pi y^2} t^{-1/2} dy\\
&=e^{2\pi i\xi a} t^{-1/2} \int_{-\infty}^\infty e^{-2\pi i\xi t^{-1/2} y} e^{-\pi y^2} dy\\
&=e^{2\pi i\xi a} t^{-1/2}  e^{-\pi\xi^2 t^{-1}}.
\end{align*}
With $f(x)=e^{-\pi t(x+a)^2}$, this shows us that 
\[
\widehat{f}(\xi) = e^{2\pi i\xi a} t^{-1/2}  e^{-\pi\xi^2 t^{-1}},
\]
and applying the Poisson summaton gives
\begin{equation}
\sum_{n \in \mathbb{Z}} e^{-\pi t(n+a)^2} = \sum_{n \in \mathbb{Z}}  e^{2\pi in a} t^{-1/2}  e^{-\pi n^2 t^{-1}}.
\label{thetaa}
\end{equation}
Define
\[
\vartheta(t) = \sum_{n \in \mathbb{Z}} e^{-\pi n^2 t}, \qquad t>0.
\]
Using \eqref{thetaa} with $a=0$ gives
\[
\vartheta(t)=
t^{-1/2} \vartheta\left(\frac{1}{t}\right).
\]


\end{document}