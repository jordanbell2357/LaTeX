\documentclass{amsart}
\usepackage{amsmath,amssymb,graphicx,subfig,mathrsfs}
%\usepackage{tikz-cd}
\newcommand{\inner}[2]{\langle #1, #2 \rangle}
\newcommand{\jap}[1]{\left\langle #1 \right\rangle}
\newcommand{\tr}{\textrm{tr}} 
\newcommand{\Span}{\textrm{span}} 
\newcommand{\id}{\textrm{id}} 
\newcommand{\Hom}{\textrm{Hom}} 
\newcommand{\norm}[1]{\left\Vert #1 \right\Vert}
\newtheorem{theorem}{Theorem}
\newtheorem{lemma}[theorem]{Lemma}
\newtheorem{corollary}[theorem]{Corollary}
\begin{document}
\title{The heat equation on $\mathbb{T}$}
\author{Jordan Bell}
\email{jordan.bell@gmail.com}
\address{Department of Mathematics, University of Toronto, Toronto, Ontario, Canada}
\date{\today}

\maketitle

Here I am working out some material following Steve Shkoller's {\em MAT218: Lecture Notes on Partial Differential Equations}. 

Fix $T>0$. Let $f \in L^2(0,T;L^2(\mathbb{T}))$ and $g \in H^1(\mathbb{T})$; as $H^1(\mathbb{T}) \subset C^0(\mathbb{T})$, we can
speak about the value of $g$ at every point rather than merely almost all points.

For almost all $t$, define $f_n$ by
\[
f_n(x,t)=\sum_{k=-n}^n \hat{f}(k,t) e^{ikx}.
\]
Define $g_n$ by
\[
g_n(x)=\sum_{k=-n}^n \hat{g}(k) e^{ikx}.
\]
We are going to find a solution $u_n$ to the Cauchy problem $u_t(x,t)-u_{xx}(x,t)=f_n(x,t)$ for almost all $x$ and for almost all $t$, 
$u(x,0)=g_n(x)$ for all $x$. 

\section{Energy estimate}
For almost all $x$ and for almost all $t$,
\begin{equation}
u_{nt}(x,t)-u_{nxx}(x,t)=f_n(x,t),
\label{npde}
\end{equation}
and for all $x$
\[
u_n(x,0)=g_n(x).
\]

Multiply \eqref{npde} by $u_n(x,t)$ and integrate over $\mathbb{T}$. We get
\[
\int_\mathbb{T} u_{nt}(x,t)u_n(x,t) dx- \int_\mathbb{T} u_{nxx}(x,t) u_n(x,t) dx = \int_\mathbb{T} f_n(x,t) u_n(x,t) dx.
\]
Integrating by parts,
\[
\int_\mathbb{T} u_{nt}(x,t) u_n(x,t) dx + \int_\mathbb{T} u_{nx}(x,t) u_{nx}(x,t) dx = \int_\mathbb{T} f_n(x,t) u_n(x,t) dx,
\]
which is
\[
\pi \cdot \partial_t \frac{1}{2\pi} \int_\mathbb{T} u_n(x,t)^2 dx + 2\pi \cdot \frac{1}{2\pi} \int_\mathbb{T} u_{nx}(x,t)^2 dx = \int_\mathbb{T} f_n(x,t) u_n(x,t) dx.
\]
Writing this using norms,
\[
\pi \cdot \partial_t \norm{u_n(\cdot,t)}_{L^2}^2 + 2\pi \cdot \norm{u_{nx}(\cdot,t)}_{L^2}^2 =  \int_\mathbb{T} f_n(x,t) u_n(x,t) dx.
\]
Integrating from $0$ to $t$,
\[
\pi \cdot \norm{u_n(\cdot,t)}_{L^2}^2 - \pi \cdot \norm{u_n(\cdot,0)}_{L^2}^2
+ 2\pi \int_0^t \norm{u_{nx}(\cdot,s)}_{L^2}^2 ds = \int_0^t  \int_\mathbb{T} f_n(x,s) u_n(x,s) dx ds.
\]

For almost all $s$,
\begin{eqnarray*}
 \int_\mathbb{T} |f_n(x,s) u_n(x,s)| dx &=& 
 2\pi \cdot \frac{1}{2\pi}  \int_\mathbb{T} |f_n(x,s) u_n(x,s)| dx ds\\
 & \leq& 
 2\pi \cdot \norm{f_n(\cdot,s)}_{L^2} \norm{u_n(\cdot,s)}_{L^2}\\
 &\leq&2\pi \left( \frac{ \norm{f_n(\cdot,s)}_{L^2}^2}{2} + \frac{\norm{u_n(\cdot,s)}_{L^2}^2}{2} \right)\\
 &=&\pi \cdot  \norm{f_n(\cdot,s)}_{L^2}^2 + \pi\cdot \norm{u_n(\cdot,s)}_{L^2}^2.
\end{eqnarray*}
Hence for all $t$,
\[
\begin{split}
&\pi \cdot \norm{u_n(\cdot,t)}_{L^2}^2 - \pi \cdot \norm{u_n(\cdot,0)}_{L^2}^2
+ 2\pi \int_0^t \norm{u_{nx}(\cdot,s)}_{L^2}^2 ds\\
 \leq& \int_0^t \pi \cdot  \norm{f_n(\cdot,s)}_{L^2}^2 + \pi\cdot \norm{u_n(\cdot,s)}_{L^2}^2 ds.
 \end{split},
\]
so, as $u_n(x,0)=g_n(x)$,
\[
 \norm{u_n(\cdot,t)}_{L^2}^2+2 \int_0^t \norm{u_{nx}(\cdot,s)}_{L^2}^2 ds \leq  \norm{g_n}_{L^2}^2 +  \int_0^t  \norm{f_n(\cdot,s)}_{L^2}^2  + \norm{u_n(\cdot,s)}_{L^2}^2 ds.
\]
Let
\[
y(t)= \norm{u_n(\cdot,t)}_{L^2}^2+2 \int_0^t \norm{u_{nx}(\cdot,s)}_{L^2}^2 ds.
\]
By the inequality we just established we have
\begin{eqnarray*}
y(t)& \leq&   \norm{g_n}_{L^2}^2 +  \int_0^t  \norm{f_n(\cdot,s)}_{L^2}^2 ds +\int_0^t \norm{u_n(\cdot,s)}_{L^2}^2 ds\\
&\leq&   \norm{g_n}_{L^2}^2 +  \int_0^t  \norm{f_n(\cdot,s)}_{L^2}^2 ds + \int_0^t y(s) ds.
\end{eqnarray*}
By Gronwall's inequality, we obtain
\[
 y(t) \leq \left(\norm{g_n}_{L^2}^2 + \int_0^t  \norm{f_n(\cdot,s)}_{L^2}^2 ds \right) e^t.
\]
As $\norm{g_n}_{L^2} \leq \norm{g}_{L^2}$ and $\norm{f_n(\cdot,s)}_{L^2} \leq \norm{f(\cdot,s)}_{L^2}$ (these two
facts follow from Parseval's identity), it follows that
\[
 y(t) \leq \left(\norm{g}_{L^2}^2 + \int_0^t  \norm{f(\cdot,s)}_{L^2}^2 ds \right) e^t.
\]
Therefore, there is some $C=C(T)<\infty$ such that if $0 \leq t \leq T$ then
\[
\norm{u_n(\cdot,t)}_{L^2}^2+2 \int_0^t \norm{u_{nx}(\cdot,s)}_{L^2}^2 ds \leq C  \left(\norm{g}_{L^2}^2 + \int_0^t  \norm{f(\cdot,s)}_{L^2}^2 ds \right).
\]
Because $\norm{g}_{L^2} < \infty$ and $\int_0^t  \norm{f(\cdot,s)}_{L^2}^2 ds<\infty$, it follows that there is some $M=M(T,f,g)<\infty$ such that 
if $0 \leq t \leq T$ then
\[
\norm{u_n(\cdot,t)}_{L^2}^2+2 \int_0^t \norm{u_{nx}(\cdot,s)}_{L^2}^2 ds \leq M.
\]
By Parseval's identity,
\[
\sum_{k \in \mathbb{Z}} |\widehat{u_n}(k,t)|^2+2\int_0^t \sum_{k \in \mathbb{Z}} |\widehat{u_{nx}}(k,s)|^2 ds \leq M,
\] 
hence for all $t$,
\[
\sum_{k \in \mathbb{Z}} |\widehat{u_n}(k,t)|^2+2\int_0^t \sum_{k \in \mathbb{Z}} k^2 |\widehat{u_n}(k,s)|^2 ds \leq M.
\]

If $k \leq n \leq m$, then $\widehat{u_n}(k,t)=\widehat{u_m}(k,t)$ for all $t$. Define $\hat{u}(k,t)$ by 
\[
\hat{u}(k,t)
=\lim_{n \to \infty} \widehat{u_n}(k,t)=\widehat{u_k}(k,t).
\]
Thus
\[
\sum_{k \in \mathbb{Z}} |\hat{u}(k,t)|^2+2\int_0^t \sum_{k \in \mathbb{Z}} k^2 |\hat{u}(k,s)|^2 ds \leq M.
\]
Then, for $M'=M'(f,g,T)$,
\[
\int_0^T \sum_{k \in \mathbb{Z}} |\hat{u}(k,t)|^2+ \sum_{k \in \mathbb{Z}} k^2 |\hat{u}(k,t)|^2 dt \leq M'.
\]
It follows that for almost all $t$, there is some $u \in H^1(\mathbb{T})$ whose Fourier coefficients are $\hat{u}(k,t)$, and that we have
\[
\int_0^T \norm{u(\cdot,t)}_{H^1}^2 dt \leq M'.
\]
We have
\[
\lim_{n \to \infty} \int_0^T \norm{u_n(\cdot,t)-u(\cdot,t)}_{H^s}^2 dt=0,
\]
i.e.
\[
\lim_{n \to \infty} \norm{u_n -u}_{L^2(0,T;H^s(\mathbb{T}))}^2 = 0.
\]



\end{document}