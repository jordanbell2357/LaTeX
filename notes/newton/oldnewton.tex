\documentclass{amsart}
\usepackage{amssymb,latexsym,amsmath,amsthm,mathrsfs}
\usepackage{graphicx}
\newtheorem{theorem}{Theorem}
\newtheorem{remark}[theorem]{Remark}
\newtheorem{conjecture}[theorem]{Conjecture}
\newtheorem{construction}[theorem]{Construction}
\begin{document}
\title{Newton's identities and the pentagonal number theorem}
\author{Jordan Bell}
\thanks{The author is supported by NSERC}
\email{jordan.bell@gmail.com}
\address{Department of Mathematics, University of Toronto, Toronto, Ontario, Canada}
\date{\today}
\begin{abstract}
Assuming the recurrence relation for the sum of divisors function $\sigma(n)$,
we prove the pentgonal number theorem using Newton's identities for the sums
of the powers of the roots of polynomials and the coefficients of the polynomials.
\end{abstract}
\maketitle

\section{Introduction}

Rademacher \cite{MR0364103} gives recurrence for $\sigma(n)$ from PNT
outline proof of $\sigma(n)$ recurrence from PNT

history Euler \cite{me}


then by integrating we get PNT

cite $\sigma(n)$ direct proof

\section{Results}


$\prod_{i=1}^n (1+TX_i)=\sum_{k=0}^n s_k T^k$.

$p_k=\sum_{i=1}^n X_i^k$ for $k \geq 1$. 

Then Newton's identities \cite[Chap. IV, \S 6, Lemma 4]{bourbaki} are
\[
p_k=s_1p_{k-1}-s_2p_{k-2}+\ldots+(-1)^k s_{k-1}p_1+(-1)^{k+1}ks_k,\quad
1 \leq k \leq n
\] 

For $n$ a positive integer, let $\sigma(n)$ be the sum of the positive divisors
of $n$, and for $n \leq 0$ let $\sigma(n)=0$. For example, $\sigma(6)=12$.

The recurrence relation for $\sigma(n)$ is as follows.
For $n \neq \frac{m(3m-1)}{2}$, 
\begin{equation}
\label{sigma1}
\sigma(n)=\sum_{j=1}^\infty (-1)^{j-1}(\sigma(n-\omega_j)+\sigma(n-\omega_{-j})),
\end{equation}
and for $n=\frac{m(3m-1)}{2}$,
\begin{equation}
\label{sigma2}
\sigma(n)=(-1)^{m-1}n+\sum_{j=1}^\infty (-1)^{j-1}(\sigma(n-\omega_j)+
\sigma(n-\omega_{-j})).
\end{equation}


Let
\[
\prod_{i=1}^n (1-T^i)=\prod_{i=1}^{\frac{n^2+n}{2}} (1-TX_i)=
\sum_{k=0}^{\frac{n^2+n}{2}} s_{k,n}(-T)^k.
\]
Also let $p_{k,n}=\sum_{i=1}^{\frac{n^2+n}{2}}X_i^k$.
Then,
\begin{eqnarray}
\nonumber
p_{k,n}&=&\sum_{i=1}^{\frac{n^2+n}{2}}X_i^k\\
\nonumber
&=&\sum_{r=1}^n \sum_{s=1}^r e^{\frac{2\pi isk}{r}}\\
\nonumber
&=&\sum_{r=1}^n \begin{cases}
r,&\textrm{if }\, r|k,\\
0,&\textrm{otherwise}
\label{pkn}
\end{cases}
\\
&=&\sigma(k) \quad \textrm{if $k\leq n$}.
\end{eqnarray}
This is because for a character $\chi$ of a group $G$, $\sum_{g \in G} \chi(g)$ is $|G|$ if $\chi$ is trivial and $0$ if $\chi$ is nontrivial, i.e. orthogonality.

As well, define
\[
(-1)^k s_k=\begin{cases}
0,&k \neq \frac{j(3j-1)}{2},\\
(-1)^j,& k=\frac{j(3j-1)}{2}.
\end{cases}
\]
The pentagonal number theorem is that
\[
\prod_{i=1}^\infty (1-T^i)=\sum_{k=0}^\infty (-1)^k s_k T^k, \quad |T|<1.
\]

Since $\prod_{i=1}^{n+1}(1-T^i)=(1-T^{n+1})\sum_{k=0}^{\frac{n^2+n}{2}}
s_{k,n}(-T)^k$, $s_{k,n}=s_{k,n+1}$ if $k \leq n$.
Therefore it suffices to prove that $s_{n,n}=s_n$ for all $n$.

First, $s_1=1$ from the definition of $(-1)^ks_k$. Also, since
$1-T=s_{0,1}-s_{1,1}T$, $s_{1,1}=1$. 
We now assume that $s_{k,n}=s_k$ for
all $k \leq n$.

By Newton's identities,
\begin{eqnarray*}
p_{n+1,n+1}&=&s_{1,n+1}p_{n,n+1}-s_{2,n+1}p_{n-1,n+1}+\ldots
+(-1)^{n+1}s_{n,n+1}p_{1,n+1}\\
&&+(-1)^{n+2}(n+1)s_{n+1,n+1}.
\end{eqnarray*}
Then since $p_{k,n}=\sigma(k)$ for $k \leq n$,
\[
\sigma(n+1)=s_{1,n+1}\sigma(n)-s_{2,n+1}\sigma(n-1)+\ldots+(-1)^ns_{n,n+1}\sigma(1)
+(-1)^{n+2}(n+1)s_{n+1,n+1}.
\]
Then as $s_{k,n+1}=s_k$ for $k \leq n$,
\[
\sigma(n+1)=s_1\sigma(n)-s_2\sigma(n-1)+\ldots+(-1)^{n+1}s_n\sigma(1)
+(-1)^{n+2}(n+1)s_{n+1,n+1}.
\]

Now because 
\[
(-1)^ks_k=\begin{cases}
0,&k \neq \frac{j(3j-1)}{2},\\
(-1)^j,&k=\frac{k(3j-1)}{2},
\end{cases}
\]
we have
\[
\sigma(n+1)=\sum_{j=1}^\infty (-1)^{j+1}(\sigma(n+1-\omega_j)
+\sigma(n+1-\omega_{-j}))
+(-1)^{n+2}(n+1)s_{n+1,n+1}.
\]

We deal with the two possible cases $n+1\neq \frac{m(3m-1)}{2}$ and
$n+1=\frac{m(3m-1)}{2}$. In the first case,
by \eqref{sigma1} we get
\[
\begin{split}
&\sum_{j=1}^\infty (-1)^{j-1}(\sigma(n+1-\omega_j)+\sigma(n+1-\omega_{-j}))\\
=&\sum_{j=1}^\infty (-1)^{j+1}(\sigma(n+1-\omega_j)
+\sigma(n+1-\omega_{-j}))+(-1)^{n+2}(n+1)s_{n+1,n+1}.
\end{split}
\]
Hence $s_{n+1,n+1}=0=s_{n+1}$, since $n+1 \neq \frac{m(3m-1)}{2}$.

In the second case by \eqref{sigma2} we get
\[
\begin{split}
&(-1)^{m-1}(n+1)+\sum_{j=1}^\infty (-1)^{j-1}(\sigma(n+1-\omega_j)+
\sigma(n+1-\omega_{-j}))\\
=&\sum_{j=1}^\infty (-1)^{j+1}(\sigma(n+1-\omega_j)
+\sigma(n+1-\omega_{-j}))+(-1)^{n+2}(n+1)s_{n+1,n+1}.
\end{split}
\]
Hence $(-1)^{m-1}(n+1)=(-1)^{n+2}(n+1)s_{n+1,n+1}$, so
$s_{n+1,n+1}=(-1)^{m-n-3}$, so $(-1)^{n+1}s_{n+1,n+1}=(-1)^m$,
which is $=s_{n+1}$ since $n+1=\frac{m(3m-1)}{2}$.
Therefore in both cases $s_{n+1,n+1}=s_{n+1}$, which proves the claim.

Let $f_n(T)=\prod_{i=1}^n (1-T^i)$. Then
\[
f_n=\sum_{k=0}^n (-1)^k s_k T^k + \sum_{k=n+1}^{\frac{n^2+n}{2}} s_{k,n}T^k.
\]  

Let $K$ be a compact subset of the open unit disc.
Then $\sum_{n=1}^\infty |T|^n$ converges uniformly
in $K$. Hence \cite[Chap. V, \S 3.1]{cartan}
the product $\prod_{n=1}^\infty (1-T^n)$ converges uniformly in $K$.
Thus \cite[Chap. V, \S 1.2, Theorem 1]{cartan} the sequence of analytic functions $f_n(T)=\prod_{i=1}^n (1-T^i)$
converges to the analytic function $f(T)=\prod_{i=1}^\infty (1-T^i)$.
Therefore the sequence of derivatives $f_n'$ converges to $f'$ uniformly
on compact sets \cite[Chap. V, \S 1.2, Theorem 2]{cartan}. 
Then for $f^{(r)}$ the $r$th derivative of $f$, $f_n^{(r)}$ converges to
$f^{(r)}$.

Let
\[
f(T)=\sum_{k=0}^\infty a_kT^k, \quad |T|<1.
\]

$f_n^{(k)}(0) \to f^{(k)}(0)$ as $n \to \infty$.
For a fixed $k$, for all $n \geq k$ one has $(-1)^ks_{k,n}=(-1)^ks_k$.
Now, $f_n^{(k)}(0)=k!(-1)^ks_{k,n}$. Thus
$f_n^{(k)}(0)=k!(-1)^ks_k$ for all $n \geq k$. Hence $f^{(k)}(0)=(-1)^ks_k$. That is,
$a_k=(-1)^ks_k$ for all $k$.

Therefore,
\[
\prod_{i=1}^\infty (1-T^i)=\sum_{k=0}^\infty (-1)^k s_k T^k, \quad |T|<1,
\]
the pentagonal number theorem.

\section{Conclusion}
Compare to usual proof of the PNT

in general: reconstruct analytic functions on the unit disc by knoiwng
their boundary values, in particular where they are 0

uses: discovering recurrences (what would recurrence have to be to get PNT)

investigating other products, say $\prod_{i=1}^\infty (1-T^i)^4$

Also interesting that the usual proof goes up in steps of two, while this goes one by one

\bibliographystyle{amsplain}
\bibliography{newton}

\end{document}
