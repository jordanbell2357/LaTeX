\documentclass{article}
\usepackage{amsmath,amssymb,mathrsfs,amsthm}
%\usepackage{tikz-cd}
\usepackage{hyperref}
\newcommand{\inner}[2]{\left\langle #1, #2 \right\rangle}
\newcommand{\tr}{\ensuremath\mathrm{tr}\,} 
\newcommand{\Span}{\ensuremath\mathrm{span}} 
\def\Re{\ensuremath{\mathrm{Re}}\,}
\def\Im{\ensuremath{\mathrm{Im}}\,}
\newcommand{\id}{\ensuremath\mathrm{id}} 
\newcommand{\var}{\ensuremath\mathrm{var}} 
\newcommand{\Lip}{\ensuremath\mathrm{Lip}} 
\newcommand{\GL}{\ensuremath\mathrm{GL}} 
\newcommand{\diam}{\ensuremath\mathrm{diam}} 
\newcommand{\sgn}{\ensuremath\mathrm{sgn}\,} 
\newcommand{\lcm}{\ensuremath\mathrm{lcm}} 
\newcommand{\supp}{\ensuremath\mathrm{supp}\,}
\newcommand{\dom}{\ensuremath\mathrm{dom}\,}
\newcommand{\upto}{\nearrow}
\newcommand{\downto}{\searrow}
\newcommand{\norm}[1]{\left\Vert #1 \right\Vert}
\newtheorem{theorem}{Theorem}
\newtheorem{lemma}[theorem]{Lemma}
\newtheorem{proposition}[theorem]{Proposition}
\newtheorem{corollary}[theorem]{Corollary}
\theoremstyle{definition}
\newtheorem{definition}[theorem]{Definition}
\newtheorem{example}[theorem]{Example}
\begin{document}
\title{Positive definite functions, completely monotone functions, the Bernstein-Widder theorem, and Schoenberg's theorem}
\author{Jordan Bell\\ \texttt{jordan.bell@gmail.com}\\Department of Mathematics, University of Toronto}
\date{\today}

\maketitle

\section{Linear operators}
For a complex Hilbert space $H$ let $\mathscr{L}(H)$ be the bounded linear operators
$H \to H$. It is a fact that $A \in \mathscr{L}(H)$ is self-adjoint if and only if $\inner{Ah}{h} \in \mathbb{R}$ for all
$h \in H$.\footnote{John B. Conway, {\em A Course in Functional Analysis}, second ed., p.~33, Proposition 2.12.}
For a bounded self-adjoint operator $A$ it is a fact
that\footnote{John B. Conway, {\em A Course in Functional Analysis}, second ed., p.~34, Proposition 2.13.}
\[
\norm{A} = \sup_{\norm{h} = 1} |\inner{Ah}{h}|.
\]
$A \in \mathscr{L}(H)$ is called \textbf{positive} if it is self-adjoint and 
\[
\inner{Ah}{h} \geq 0, \qquad h \in H;
\]
because we have taken $H$ to be a complex Hilbert space,  for $A$ to be positive it suffices that the inequality is satisfied. 

For $A,B \in \mathscr{L}(\mathbb{C}^n)$, we define
their \textbf{Hadamard product} $A * B \in \mathscr{L}(H)$ by
\[
(A*B)e_i = \sum_{j=1}^n \inner{Ae_i}{e_j} \inner{Be_i}{e_j} e_j.
\]
So,
\[
\inner{(A*B)e_i}{e_j} = \inner{Ae_i}{e_j} \inner{Be_i}{e_j}.
\]
The \textbf{Schur product theorem} states that if $A,B \in \mathscr{L}(\mathbb{C}^n)$ are positive then
their Hadamard product $A * B$ is positive.\footnote{Ward Cheney and Will Light, {\em A Course
in Approximation Theory}, p.~81, chapter 12.}


\section{Positive definite functions}
Let $X$ be a real or complex linear space, let  $f:X \to \mathbb{C}$ be a function, and for
$x_1,\ldots,x_n \in X$, define $F_{f;x_1,\ldots,x_n} \in \mathscr{L}(\mathbb{C}^n)$ by
\[
F_{f;x_1,\ldots,x_n}e_i = \sum_{j=1}^n f(x_i-x_j) e_j,
\]
where $\{e_1,\ldots,e_n\}$ is the standard basis for $\mathbb{C}^n$.
 Thus for
$u=\sum_{i=1}^n u_i e_i \in \mathbb{C}^n$,
\begin{align*}
\inner{F_{f;x_1,\ldots,x_n}u}{u}&=\inner{\sum_{i=1}^n u_i \sum_{j=1}^n f(x_i-x_j) e_j}{\sum_{k=1}^n u_k e_k}\\
&=\sum_{i=1}^n u_i \sum_{j=1}^n f(x_i-x_j) \inner{e_j}{\sum_{k=1}^n u_k e_k}\\
&=\sum_{i=1}^n  \sum_{j=1}^n u_i \overline{u_j} f(x_i-x_j).
\end{align*}
We call $f$  \textbf{positive definite} if
for all $x_1,\ldots,x_n \in X$,
$F_{f;x_1,\ldots,x_n}$ is a positive operator, i.e. for
 $u \in \mathbb{C}^n$,
\[
\inner{F_{f;x_1,\ldots,x_n}u}{u} \geq 0.
\]
We call $f$
\textbf{strictly positive definite} for all distinct $x_1,\ldots,x_n \in X$ and nonzero $u \in \mathbb{C}^n$,
\[
\inner{F_{f;x_1,\ldots,x_n}u}{u} >0.
\]


\section{Completely monotone functions}
A function $f:[0,\infty) \to \mathbb{R}$ is called \textbf{completely monotone} if
\begin{enumerate}
\item $f \in C[0,\infty)$
\item $f \in C^\infty(0,\infty)$
\item $(-1)^k f^{(k)}(x) \geq 0$ for $k \geq 0$ and $x \in (0,\infty)$
\end{enumerate}
Because a completely monotone function is continuous, $f(x)$ tends to $f(0)$ as  $x \downarrow 0$.
Because a completely monotone function is nonincreasing and convex, 
$f(x)$ has a limit, which we call $f(\infty)$, as $x \uparrow \infty$. 

The \textbf{Bernstein-Widder theorem} states that a function $f$ satisfying
$f(0)=1$ is completely monotone if and only
if it is the Laplace transform of a Borel probability measure on $[0,\infty)$.\footnote{Peter D. Lax, {\em Functional Analysis}, p.~138, chapter 14, Theorem 3;
\url{http://djalil.chafai.net/blog/2013/03/23/the-bernstein-theorem-on-completely-monotone-functions/}}


\begin{theorem}[Bernstein-Widder theorem]
A function $f:[0,\infty) \to \mathbb{R}$ satisfies $f(0)=1$ and is completely monotone if and only if
there is a Borel probability measure $\mu$ on $[0,\infty)$ such that
\[
f(x) = \int_0^\infty e^{-xt} d\mu(t),\qquad x \in [0,\infty).
\]
\label{bernstein}
\end{theorem}
\begin{proof}
If $f$ is the Laplace transform of some probability measure $\mu$ on $\mathscr{B}_{[0,\infty)}$,
then using the dominated convergence theorem yields that $f$ is continuous and by induction that
$f \in C^\infty(0,\infty)$. For $k \geq 0$ and for $x \in (0,\infty)$,
\[
f^{(k)}(x) = \int_0^\infty (-t)^k e^{-xt} d\mu(t),
\]
as $\int_0^\infty t^k e^{-xt} d\mu(t) \geq 0$ so $(-1)^k f^{(k)}(x) \geq 0$. Hence $f$ is completely monotone,
and $f(0)=\int_0^\infty d\mu(t)=1$. 

If $f$ satisfies $f(0)=1$ and is completely monotone, then for each $k \geq 0$, 
the function $(-1)^{k} f^{(k)}:(0,\infty) \to \mathbb{R}$ is nonnegative and is nonincreasing, so for $k \geq 1$ and
$t \in (0,\infty)$, using that $(-1)^k f^{(k)}$ is nondecreasing and that
$(-1)^{k-1} f^{(k-1)}$ is nonnegative,
\begin{align*}
(-1)^k f^{(k)}(t)&\leq \frac{2}{t} \int_{t/2}^t (-1)^k f^{(k)}(u) du\\
&=\frac{2}{t} (-1)^k \left(f^{(k-1)}(t)-f^{(k-1)}(t/2)\right)\\
&\leq \frac{2}{t} (-1)^{k-1} f^{(k-1)}(t/2).
\end{align*}
Doing induction, for any $k \geq 1$,
\begin{align*}
(-1)^k f^{(k)}(t)&\leq \prod_{j=1}^{k-1} \left(\frac{2^j}{t}\right) \cdot f'\left(\frac{t}{2^{k-1}}\right)\\
&\leq  \prod_{j=1}^{k-1} \left(\frac{2^j}{t}\right) \cdot \frac{2^k}{t}  \left(f\left(\frac{t}{2^{k-1}}\right)-f\left(\frac{t}{2^{k}}\right)\right)\\
&=t^{-k} 2^{k(k-1)/2}  \left(f\left(\frac{t}{2^{k-1}}\right)-f\left(\frac{t}{2^{k}}\right)\right).
\end{align*}
Because $f(x) \to f(0)$ as $x \downarrow 0$,
\[
f\left(\frac{t}{2^{k-1}}\right)-f\left(\frac{t}{2^{k}}\right) \to 0,\qquad t \downarrow 0,
\]
and because $f(x) \to f(\infty)$ as $x \uparrow \infty$,
\[
f\left(\frac{t}{2^{k-1}}\right)-f\left(\frac{t}{2^{k}}\right) \to 0,\qquad t \uparrow \infty.
\]
Hence for each $k \geq 1$,
\begin{equation}
|f(t)| = o_k(t^{-k}), \qquad t \downarrow 0,
\label{0limit}
\end{equation}
and
\begin{equation}
|f(t)| = o_k(t^{-k}),\qquad t \uparrow \infty.
\label{infinitylimit}
\end{equation}
Furthermore, for any $x \in (0,\infty)$, $f^{(k)}(t) \to f^{(k)}(x)$ as $t \to x$, so it is immediate that
\begin{equation}
(t-x)^k f^{(k)}(t) \to 0,\qquad t \to x.
\label{xlimit}
\end{equation}
For $x \geq 0$ and $k \geq 1$,
integrating by parts, using  \eqref{infinitylimit} and \eqref{0limit} or  \eqref{xlimit} 
respectively as $x=0$ or $x>0$,
\begin{align*}
f(x)-f(\infty)&=-\int_x^\infty f'(t) dt\\
&=-(t-x)f'(t)\Big|_x^\infty + \int_x^\infty f''(t) (t-x) dt\\
&=\int_x^\infty f''(t) (t-x) dt\\
&=\frac{(t-x)^2}{2} f''(t) \Big|_x^\infty - \int_x^\infty f'''(t) \frac{(t-x)^2}{2} dt\\
&=- \int_x^\infty f'''(t) \frac{(t-x)^2}{2} dt\\
&=(-1)^k \int_x^\infty f^{(k)}(t) \frac{(t-x)^{k-1}}{(k-1)!} dt.
\end{align*}
Hence for $x \geq 0$ and $n \geq 0$, 
\[
f(x)-f(\infty) = \frac{(-1)^{n+1}}{n!} \int_x^\infty f^{(n+1)}(t) (t-x)^n dt.
\]
Define
\[
\phi_n(y) = (1-y/n)^n 1_{[0,n]}(y).
\]
For $n \geq 1$,
by change of variables,
\begin{align*}
f(x)-f(\infty)&=\frac{(-1)^{n+1}}{n!} \int_{x/n}^\infty f^{(n+1)}(nu) (nu-x)^n ndu\\
&=\frac{(-1)^{n+1}}{(n-1)!} \int_{x/n}^\infty f^{(n+1)}(nu) (nu)^n \left(1-\frac{x}{nu}\right)^n du\\
&=\frac{(-1)^{n+1}}{(n-1)!} \int_0^\infty (nu)^n \phi_n(x/u) f^{(n+1)}(nu)  du\\
&=\frac{(-1)^{n+1}}{(n-1)!} \int_0^\infty (n/t)^n \phi_n(xt) f^{(n+1)}(n/t) t^{-2}dt.
\end{align*}
For $t \geq 0$, define
\[
s_n(t) = \frac{(-1)^{n+1}}{(n-1)!} \int_{1/t}^\infty  (nu)^n f^{(n+1)}(nu) du,
\]
where $s_n(0)=0$, 
and for $t<0$ let $s_n(t)=0$. $s_n$ is continuous
and because $f$ is completely monotone, $s_n$ is nondecreasing, so there is a unique positive
measure $\sigma_n$ on $\mathscr{B}_{\mathbb{R}}$ such that\footnote{Charalambos D. Aliprantis
and Kim C. Border, {\em Infinite Dimensional Analysis: A Hitchhiker's Guide}, third ed., p.~393, Theorem 10.48.}
\[
\sigma_n((a,b])=s_n(b)-s_n(a),\qquad a \leq b.
\]
On the other hand, $s_n$ is absolutely continuous, so $\sigma_n$ is absolutely continuous with respect to
Lebesgue measure $\lambda_1$, and for $\lambda_1$-almost all $t \in \mathbb{R}$,\footnote{H. L. Royden, {\em Real Analysis},
third ed., p.~303, Exercise 16.}
\[
\frac{d\sigma_n}{d\lambda_1}(t) = s_n'(t).
\]
Now for $t>0$, by the fundamental theorem of calculus and the chain
rule,
\[
s_n'(t) = \frac{(-1)^{n+1}}{(n-1)!} (n/t)^n f^{(n+1)}(n/t) \cdot t^{-2},
\]
and therefore
\begin{align*}
f(x)-f(\infty)&=\int_0^\infty \phi_n(xt) s_n'(t) d\lambda_1(t)\\
&=\int_0^\infty \phi_n(xt) \frac{d\sigma_n}{d\lambda_1}(t) d\lambda_1(t)\\
&=\int_0^\infty \phi_n(xt) d\sigma_n(t).
\end{align*}
The \textbf{total variation} of $\sigma_n$ is equal to the total variation of $s_n$, and because $s_n$ is nondecreasing,
\[
\norm{\sigma_n} = \int_0^\infty |s_n'(t)| dt =  \int_0^\infty s_n'(t) dt 
=s_n(\infty) - s_n(0) = s_n(\infty),
\]
which is
\[
\norm{\sigma_n} = \frac{(-1)^{n+1}}{(n-1)!} \int_0^\infty (nu)^n f^{(n+1)}(nu) du
=f(0)-f(\infty),
\]
showing that $\{\sigma_n: n \geq 1\}$ is bounded for the total variation norm. 
We claim that $\{\sigma_n: n \geq 1\}$ is \textbf{tight}: for each $\epsilon>0$ there is a compact subset
$K_\epsilon$ of $\mathbb{R}$ such that $\sigma_n(K_\epsilon^c)<\epsilon$ for all $n$. 
Taking this for granted, \textbf{Prokhorov's theorem}\footnote{V. I. Bogachev, {\em Measure Theory},
volume II, p.~202, Theorem 8.6.2.} states that there is a subsequence
$\sigma_{k_n}$ of $\sigma_n$  that converges \textbf{narrowly} to some positive measure $\sigma$ on
$\mathscr{B}_{\mathbb{R}}$. Finally, the sequence
$t \mapsto \phi_n(xt)$ tends in $C_b([0,\infty))$ to $t \mapsto e^{-xt}$, and it thus follows that\footnote{cf. Charalambos D. Aliprantis
and Kim C. Border, {\em Infinite Dimensional Analysis: A Hitchhiker's Guide}, third ed., p.~511, Corollary 15.7.}
\[
\int_0^\infty \phi_n(xt) d\sigma_n(t) \to \int_0^\infty e^{-xt} d\sigma(t),
\]
so
\[
f(x)-f(\infty) =  \int_0^\infty e^{-xt} d\sigma(t).
\]
Let
\[
\mu=\sigma+f(\infty) \delta_0,
\]
with which
\[
 \int_0^\infty e^{-xt} d\mu(t) =  \int_0^\infty e^{-xt} d\sigma(t) + f(\infty),
\]
hence 
\[
f(x) =  \int_0^\infty e^{-xt} d\mu(t).
\]
Because $f(0)=1$, $\int_0^\infty d\mu(t)=1$, showing that $\mu$ is a probability measure.
\end{proof}


\section{Fourier transforms}
For a topological space $X$ and a positive Borel measure $\mu$ on $X$, 
$F \subset X$ is called a support of $\mu$ if (i) $F$ is closed, 
(ii) $\mu(F^c)=0$, and (iii) if $G$ is open and $G \cap F \neq \emptyset$ then
$\mu(G \cap F)>0$.
 If $F_1$ and $F_2$ are supports of $\mu$, it is straightforward that $F_1=F_2$.
 It is a fact that if $X$ is second-countable then $\mu$ has a support, which we denote by
 $\supp \mu$.\footnote{Charalambos D. Aliprantis
and Kim C. Border, {\em Infinite Dimensional Analysis: A Hitchhiker's Guide}, third ed., p.~442, Theorem 12.14.}

\begin{lemma}
 If $\mu$ is a Borel measure on a topological space $X$ and $\mu$ has a support
 $\supp \mu$, if $f:X \to [0,\infty)$ is continuous and $\int_X f d\mu=0$ then $f(x)=0$ for all
 $x \in \supp \mu$. 
 \label{supplemma}
 \end{lemma}
 \begin{proof}
 Let $F = \supp \mu$ and let $E=\{x \in X: f(x) \neq 0\}$. $E$ is an open subset of $X$. Suppose by contradiction
 that there is some $x \in E \cap F$, i.e. that $E \cap F \neq  \emptyset$. Because $f$ is continuous and $f(x)>0$,
 there is some open neighborhood $G$ of $x$ for which $f(y) > f(x)/2$ for $y \in U$. 
Then $x \in G \cap F$, so $G \cap F \neq \emptyset$ and because $F$ is the support of $\mu$,
$\mu(G \cap F)>0$ and a fortiori $\mu(G)>0$. Then
\[
0=\int_X f d\mu \geq \int_G f(y) d\mu(y) \geq \int_G \frac{f(x)}{2} d\mu(y)
=\frac{f(x)}{2} \mu(G)>0, 
\]
 a contradiction. Therefore $E \cap F = \emptyset$, i.e. for all $x \in F$, $f(x)=0$. 
 \end{proof}
 
 
The following lemma asserts that a certain function is nonzero $\lambda_d$-almost everywhere, where $\lambda_d$ is Lebesgue
measure on $\mathbb{R}^d$.\footnote{Ward Cheney and Will Light, {\em A Course
in Approximation Theory}, p.~91, chapter 13, Lemma 6.}
 
\begin{lemma}
Let $x_1,\ldots,x_n$ be distinct points in $\mathbb{R}^d$, let $u \in \mathbb{C}^n$ not be the zero vector, and define
\[
g(y) = \sum_{j=1}^n u_j e^{-2\pi ix_j \cdot y}, \qquad y \in \mathbb{R}^d.
\]
For $\lambda_d$-almost all $y \in \mathbb{R}^d$, $g(y) \neq 0$.
 \label{lebesguelemma}
\end{lemma}


 
The following theorem gives conditions under which the Fourier transform of a Borel measure on $\mathbb{R}^d$ is strictly
positive definite.\footnote{Ward Cheney and Will Light, {\em A Course
in Approximation Theory}, p.~92, chapter 13, Theorem 3.}

\begin{theorem}
If $\mu$ is a finite Borel measure on $\mathbb{R}^d$ and $\lambda_d(\supp \mu)>0$, then 
$\hat{\mu}:\mathbb{R}^d \to \mathbb{C}$ is strictly positive definite.
\label{fourier}
\end{theorem} 
\begin{proof}
For distinct $x_1,\ldots,x_n \in \mathbb{R}^d$ and for nonzero $u \in \mathbb{C}^n$, 
\begin{align*}
\sum_{j=1}^n \sum_{k=1}^n u_j \overline{u_k} \hat{\mu}(x_j-x_k)&=
\sum_{j=1}^n \sum_{k=1}^n u_j \overline{u_k} \int_{\mathbb{R}^d} e^{-2\pi i(x_j-x_k)\cdot y} d\mu(y)\\
&=\int_{\mathbb{R}^d} \left(\sum_{j=1}^n u_j e^{-2\pi ix_j\cdot y} \right) \overline{\left( \sum_{k=1}^n u_k e^{-2\pi ix_k\cdot y} \right)} d\mu(y)\\
&=\int_{\mathbb{R}^d} \left| \sum_{j=1}^n u_j e^{-2\pi ix_j\cdot y} \right|^2 d\mu(y)\\
&=\int_{\mathbb{R}^d} |g(y)|^2 d\mu(y).
\end{align*}
It is apparent that this is nonnegative. If it is equal to $0$ then because $g$ is continuous we obtain from Lemma \ref{supplemma}
that 
$|g(y)|^2=0$ for all $y \in \supp \mu$, i.e.
$g(y)=0$ for all $y \in \supp \mu$. In other words,
\[
\supp \mu \subset \{y \in \mathbb{R}^d: g(y)=0\}.
\]
But by Lemma \ref{lebesguelemma}, $\lambda_d(\{y \in \mathbb{R}^d: g(y)=0\})=0$, so $\lambda_d(\supp \mu)=0$,
contradicting the hypothesis $\lambda_d(\supp \mu)>0$. Therefore
\[
\int_{\mathbb{R}^d} |g(y)|^2 d\mu(y)>0,
\]
which shows that $\hat{\mu}$ is strictly positive definite.
\end{proof}


\section{Schoenberg's  theorem}
Let $(X,\inner{\cdot}{\cdot})$ be a real  inner product space. We call a function $F:X \to \mathbb{R}$ \textbf{radial} when 
$\norm{x}=\norm{y}$ implies that $F(x)=F(y)$. 


An identity that is worth memorizing is that for $y \in \mathbb{R}$,
\[
\int_{\mathbb{R}} e^{-\pi x^2} e^{-2\pi ixy} dx = e^{-\pi y^2}.
\]
Using this and Fubini's theorem yields,
 $y \in \mathbb{R}^d$,
\[
\int_{\mathbb{R}^d} e^{-\pi |x|^2} e^{-2\pi \inner{x}{y}} = e^{-\pi |y|^2}.
\]

\begin{lemma}
For $\alpha>0$ and $y \in \mathbb{R}^d$,
\[
\int_{\mathbb{R}^d} \left(\frac{\pi}{\alpha}\right)^{d/2} \exp\left(-\frac{\pi^2}{\alpha} |x|^2\right)
e^{-2\pi i\inner{x}{y}} dx  = e^{-\alpha |y|^2}.
\]
\label{gaussian}
\end{lemma}
\begin{proof}
Define $T:\mathbb{R}^d \to \mathbb{R}^d$ by
\[
T(x) =  \sqrt{\frac{\pi}{\alpha}}x, \qquad x \in \mathbb{R}^d.
\]
$T'(x)= \sqrt{\frac{\pi}{\alpha}}I \in \mathscr{L}(\mathbb{R}^d)$ and $J_T(x)=\det T'(x) = \left(\frac{\pi}{\alpha}\right)^{d/2}$. 
Let $u \in \mathbb{R}^d$ and define  $f(x) = e^{-\pi |x|^2} e^{-2\pi i\inner{x}{u}}$.
By the change of variables formula,\footnote{Charalambos D. Aliprantis and Owen Burkinshaw, {\em Principles of Real Analysis}, third ed.,
p.~393, Theorem 40.7.}
\[
 \int_{\mathbb{R}^d} (f \circ T) \cdot  |J_T| d\lambda_d=
\int_{T(\mathbb{R}^d)} fd\lambda_d,
\]
and because $T$ is self-adjoint this is
\[
 \int_{\mathbb{R}^d} 
e^{-\pi |T(x)|^2} e^{-2\pi i\inner{x}{Tu}}
 \left(\frac{\pi}{\alpha}\right)^{d/2} dx=
\int_{\mathbb{R}^d} e^{-\pi |x|^2} e^{-2\pi i\inner{x}{u}} dx,
\]
and therefore
\[
\int_{\mathbb{R}^d}  \left(\frac{\pi}{\alpha}\right)^{d/2}  \exp\left(-\frac{\pi^2}{\alpha} |x|^2 \right)  e^{-2\pi i\inner{x}{Tu}} dx
=e^{-\pi|u|^2}.
\]
For $u = T^{-1}(y)=\sqrt{\frac{\alpha}{\pi}}y$ this is
\[
\int_{\mathbb{R}^d}  \left(\frac{\pi}{\alpha}\right)^{d/2}  \exp\left(-\frac{\pi^2}{\alpha} |x|^2 \right)  e^{-2\pi i\inner{x}{y}} dx
=e^{-\alpha |y|^2},
\]
proving the claim.
\end{proof}


We now prove that on a real inner product space, $x \mapsto e^{-\alpha \norm{x}^2}$ is strictly positive definite whenever
$\alpha>0$.\footnote{Ward Cheney and Will Light, {\em A Course
in Approximation Theory}, p.~104, chapter 15, Theorem 2.}

\begin{theorem}
Let $(X,\inner{\cdot}{\cdot})$ be a real inner product space. If $\alpha>0$, then
\[
x \mapsto  e^{-\alpha  \norm{x}^2},\qquad x \in X,
\]
is radial and strictly positive definite.
\label{ealpha}
\end{theorem}
\begin{proof}
Let $x_1,\ldots,x_n$ be distinct points in $X$. There is an $n$-dimensional linear subspace
$V$ of $X$ that contains $x_1,\ldots,x_n$. By the Gram-Schmidt process, $V$ has an orthonormal basis
$\{v_1,\ldots,v_n\}$. Define $T:V \to \mathbb{R}^n$ by $Tv_j=e_j$, where $\{e_1,\ldots,e_n\}$ is the standard
basis for $\mathbb{R}^n$, which is an orthogonal transformation, and define
\[
f(u) = e^{-\alpha |u|^2}, \qquad  u \in \mathbb{R}^d.
\]
For $u \in \mathbb{C}^n$, $u \neq 0$,
\begin{align*}
\sum_{j=1}^n \sum_{k=1}^n u_j \overline{u_k} e^{-\alpha \norm{x_j-x_k}^2}
&=\sum_{j=1}^n \sum_{k=1}^n u_j \overline{u_k} \exp\left(-\alpha |T(x_j-x_k)|^2 \right)\\
&=\sum_{j=1}^n \sum_{k=1}^n u_j \overline{u_k} f(Tx_j-Tx_k).
\end{align*}
Now, let $\mu$ be the Borel measure on $\mathbb{R}^d$ whose density with respect to $\lambda_d$ is
\[
y \mapsto \left( \frac{\pi}{\alpha} \right)^{d/2} \exp\left(-\frac{\pi^2}{\alpha} |y|^2 \right).
\]
Because $\mu$ is absolutely continuous with respect to $\lambda_d$, 
$\lambda_d(\supp \mu)>0$, so Theorem \ref{fourier} states that the Fourier transform
$\hat{\mu}:\mathbb{R}^d \to \mathbb{C}$ is strictly positive definite. 
Applying Lemma \ref{gaussian},
the Fourier transform of $\mu$ is
\[
\hat{\mu}(u) = \int_{\mathbb{R}^d} \left( \frac{\pi}{\alpha} \right)^{d/2} \exp\left(-\frac{\pi^2}{\alpha} |y|^2 \right) 
e^{-2\pi  i\inner{y}{u}} dy
=e^{-\alpha |u|^2}=f(u),
\]
so $f$ is strictly positive definite. Because $T$ is an orthogonal transformation it is in particular one-to-one, so
$Tx_1,\ldots,Tx_n$ are distinct points in $\mathbb{R}^d$. Thus the fact that $f$ is strictly positive definite means that
\[
\sum_{j=1}^n \sum_{k=1}^n u_j \overline{u_k} e^{-\alpha \norm{x_j-x_k}^2} = \sum_{j=1}^n \sum_{k=1}^n u_j \overline{u_k} f(Tx_j-Tx_k)
>0,
\]
which establishes that $x \mapsto e^{-\alpha \norm{x}^2}$ is strictly positive definite. 
\end{proof}


The following is \textbf{Schoenberg's theorem}.\footnote{Ward Cheney and Will Light, {\em A Course
in Approximation Theory}, p.~101, chapter 15, Theorem 1;
Ren\'e L. Schilling, Renming Song, and Zoran Vondra\v{c}ek, {\em Bernstein Functions: Theory and Applications}, p.~142, Theorem 12.14;
William F. Donoghue Jr., {\em Distributions and Fourier Transforms}, p.~205, \S 41.}

\begin{theorem}[Schoenberg's theorem]
Let $(X,\inner{\cdot}{\cdot})$ be a real  inner product space.
If $f:[0,\infty) \to \mathbb{R}$ is completely monotone, $f(0)=1$,  and $f$ is not constant, then
\[
x \mapsto f(\norm{x}^2), \qquad X \to [0,\infty),
\]
is radial and strictly positive definite.
\end{theorem}
\begin{proof}
Because $f$ is completely monotone, the Bernstein-Widder theorem (Theorem \ref{bernstein}) tells us that
there is a Borel probability measure $\mu$ on $[0,\infty)$ such that
\[
f(t) = \int_0^\infty e^{-st} d\mu(s),\qquad t \in [0,\infty),
\]
that is, $f$ is the Laplace transform of $\mu$. Now, the Laplace transform of $\delta_0$ is $t \mapsto 1$,
and because $f$ is not constant, the Laplace transform of $\mu$ is not equal to the Laplace transform of $\delta_0$,
which implies that $\mu \neq \delta_0$.\footnote{Bert Fristedt and Lawrence Gray, {\em A Modern Approach
to Probability Theory},
p.~218, \S 13.5, Theorem 6.} Therefore $\mu((0,\infty))>0$. 


Let $x_1,\ldots,x_n$ be distinct points in $X$ and let $u \in \mathbb{C}^n$, $u \neq 0$. Then, because
$\sum_{j=1}^n \sum_{k=1}^n u_j \overline{u_k} \geq 0$,
\begin{align*}
\sum_{j=1}^n \sum_{k=1}^n u_j \overline{u_k} f(\norm{x_j-x_k}^2)&=
\sum_{j=1}^n \sum_{k=1}^n u_j \overline{u_k} \int_0^\infty \exp\left(-s \norm{x_j-x_k}^2 \right) d\mu(s)\\
&=\int_0^\infty \sum_{j=1}^n \sum_{k=1}^n u_j \overline{u_k} \exp\left(-s \norm{x_j -  x_k}^2\right) d\mu(s)\\
&=\sum_{j=1}^n \sum_{k=1}^n u_j \overline{u_k} \mu(\{0\})\\
&+\int_0^\infty 1_{(0,\infty)}(s)  \sum_{j=1}^n \sum_{k=1}^n u_j \overline{u_k} \exp\left(-s \norm{x_j -  x_k}^2\right) d\mu(s)\\
&\geq\int_0^\infty 1_{(0,\infty)}(s)  \sum_{j=1}^n \sum_{k=1}^n u_j \overline{u_k} \exp\left(-s \norm{x_j -  x_k}^2\right) d\mu(s)\\
&=\int_0^\infty g(s) d\mu(s).
\end{align*}
Assume by contradiction that $\int_0^\infty g(s) d\mu(s)=0$. Because
$g \geq 0$, this implies that
$\mu(\{s \in [0,\infty): g(s)>0\})=0$.\footnote{Charalambos D. Aliprantis
and Kim C. Border, {\em Infinite Dimensional Analysis: A Hitchhiker's Guide}, third ed., p.~411, Theorem 11.16.}
By Theorem \ref{ealpha}, for each $s>0$,
\[
\sum_{j=1}^n \sum_{k=1}^n u_j \overline{u_k} \exp\left(-s \norm{x_j -  x_k}^2\right) >0,
\]
so $g(s)>0$ when  $s>0$. Thus $\mu((0,\infty))=0$, a contradiction. Therefore,
\[
\sum_{j=1}^n \sum_{k=1}^n u_j \overline{u_k} f(\norm{x_j-x_k}^2) = \int_0^\infty g(s) d\mu(s)>0,
\]
which shows that $x \mapsto f(\norm{x}^2)$ is strictly positive definite. 
\end{proof}

\end{document}