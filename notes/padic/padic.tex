\documentclass{article}
\usepackage{amsmath,amssymb,mathrsfs,amsthm,hyperref}
\newcommand{\inner}[2]{\left\langle #1, #2 \right\rangle}
\newcommand{\tr}{\ensuremath\mathrm{tr}\,} 
\newcommand{\Span}{\ensuremath\mathrm{span}} 
\def\Re{\ensuremath{\mathrm{Re}}\,}
\def\Im{\ensuremath{\mathrm{Im}}\,}
\newcommand{\id}{\ensuremath\mathrm{id}} 
\newcommand{\rank}{\ensuremath\mathrm{rank\,}} 
\newcommand{\diam}{\ensuremath\mathrm{diam}} 
\newcommand{\osc}{\ensuremath\mathrm{osc}} 
\newcommand{\co}{\ensuremath\mathrm{co}\,} 
\newcommand{\cco}{\ensuremath\overline{\mathrm{co}}\,}
\newcommand{\supp}{\ensuremath\mathrm{supp}\,}
\newcommand{\ext}{\ensuremath\mathrm{ext}\,}
\newcommand{\ba}{\ensuremath\mathrm{ba}\,}
\newcommand{\cl}{\ensuremath\mathrm{cl}\,}
\newcommand{\dom}{\ensuremath\mathrm{dom}\,}
\newcommand{\Cyl}{\ensuremath\mathrm{Cyl}\,}
\newcommand{\extreals}{\overline{\mathbb{R}}}
\newcommand{\upto}{\nearrow}
\newcommand{\downto}{\searrow}
\newcommand{\norm}[1]{\left\Vert #1 \right\Vert}
\theoremstyle{plain}
\newtheorem{theorem}{Theorem}
\newtheorem{lemma}[theorem]{Lemma}
\newtheorem{proposition}[theorem]{Proposition}
\newtheorem{corollary}[theorem]{Corollary}
\theoremstyle{definition}
\newtheorem{definition}[theorem]{Definition}
\newtheorem{example}[theorem]{Example}
\begin{document}
\title{Hensel's lemma, valuations, and $p$-adic numbers}
\author{Jordan Bell}
\date{November 2, 2014}

\maketitle

\section{Hensel's lemma}
Let $p$ be prime and $f(x) \in \mathbb{Z}[x]$.\footnote{Hua Loo Keng, {\em Introduction to Number Theory}, Chapter 15, ``$p$-adic numbers''.}
 Suppose that $0 \leq a_0 < p$, satisfies
\[
f(a_0) \equiv 0 \pmod{p}
\]
 and
 \[
 f'(a_0) \not \equiv 0 \pmod{p}.
 \] 
Using the power series expansion
\[
f(a_0+h) = f(a_0)+f'(a_0)h + \frac{f''(a_0)}{2}h^2+\cdots,
\]
 for any $y \in \mathbb{Z}$ we have
 \[
f(a_0+py) = f(a_0)+f'(a_0)py+\frac{f''(a_0)}{2}p^2 y^2 + \cdots
 \]
so
\[
\frac{f(a_0+py)}{p} = \frac{f(a_0)}{p}+f'(a_0)y+\frac{f''(a_0)}{2}p y^2 + \cdots.
\]
Because $f(a_0) \equiv 0 \pmod{p}$, each term on the right-hand side is an integer. 
Then, $f(a_0+py) \equiv 0 \pmod{p^2}$ is equivalent to
\[
\frac{f(a_0)}{p}+f'(a_0)y+\frac{f''(a_0)}{2}p y^2 + \cdots \equiv 0 \pmod{p},
\]
i.e.,
\[
f'(a_0)y \equiv -\frac{f(a_0)}{p} \pmod{p}.
\]
Because $f'(a_0) \not \equiv 0 \pmod{p}$, there is a unique $y \pmod{p}$ that solves the above congruence, 
so there is a unique $y \pmod{p}$ that solves $f(a_0+py) \equiv 0 \pmod{p^2}$. 
This $y$ is 
\[
y \equiv -\frac{f(a_0)}{p} (f'(a_0))^{-1} \pmod{p}.
\]
Let $0 \leq a_1 < p$ be $a_1 \equiv y \pmod{p}$. 

Suppose that 
\[
x=a_0+a_1p+a_2p^2+\cdots+a_{l-2}p^{l-2}, \qquad
0 \leq a_j < p,
\]
satisfies
\[
f(x) \equiv 0 \pmod{p^{l-1}}
\]
and
\[
f'(x) \not \equiv 0 \pmod{p}.
\]
Using the power series expansion
\[
f(x+h)=f(x)+f'(x)h+\frac{f''(x)}{2}h^2+\cdots,
\]
for any $y \in \mathbb{Z}$ we have
\[
f(x+p^{l-1}y) = f(x)+f'(x)p^{l-1}y+\frac{f''(x)}{2}p^{2l-2} y^2 + \cdots,
\]
i.e.
\[
\frac{f(x+p^{l-1}y)}{p^{l-1}} = \frac{f(x)}{p^{l-1}}+f'(x)y+\frac{f''(x)}{2}p^{l-1}y^2+\cdots.
\]
Because $f(x) \equiv 0 \pmod{p^{l-1}}$, each term on the right-hand side is an integer. Then, 
$f(x+p^{l-1}y) \equiv 0 \pmod{p^l}$ is equivalent to
\[
 \frac{f(x)}{p^{l-1}}+f'(x)y+\frac{f''(x)}{2}p^{l-1}y^2+\cdots \equiv 0 \pmod{p},
\]
i.e.,
\[
f'(x)y \equiv - \frac{f(x)}{p^{l-1}} \pmod{p}.
\]
Because $f'(x) \not \equiv 0 \pmod{p}$, there is a unique $y \pmod{p}$ that solves the above congruence, so there
is a unique $y \pmod{p}$ that solves $f(x+p^{l-1}y) \equiv 0 \pmod{p^l}$. This $y$ is
\[
y \equiv - \frac{f(x)}{p^{l-1}} (f'(x))^{-1} \pmod{p}.
\]
Let $0 \leq a_{l-1} < p$ be $a_{l-1} \equiv y \pmod{p}$. 

We have thus inductively defined a sequence $a_0,a_1,a_2,\ldots$, with $0 \leq a_j < p$, such that for any
$l$, 
\[
f(a_0+a_1p+\cdots+a_{l-1}p^{l-1}) \equiv 0 \pmod{p^l}.
\]

We wish to make sense of the infinite expression
\[
a_0+a_1p+a_2p^2+a_3p^3+\cdots
\]
Calling this $x$, it ought to be the case that $f(x) \equiv 0 \pmod{p}$, $f(x) \equiv 0 \pmod{p^2}$, $f(x) \equiv 0 \pmod{p^3}$, etc.



\begin{example}
Take $p=3$ and $f(x)=x^2-7$, $f'(x)=2x$. The two conditions $f(x) \equiv 0 \pmod{p}$ and $f'(x) \not \equiv 0 \pmod{p}$
are satisfied both by $a_0=1$ and $a_0=2$. Take $a_0=1$. Then
\[
a_1 \equiv -\frac{f(1)}{3} (f'(1))^{-1} 
\equiv
-\frac{-6}{3} (2)^{-1}
\equiv 1  \pmod{3}.
\]
So $a_1=1$. Then,
\[
a_2 \equiv -\frac{f(1+1\cdot 3)}{3^2} (f'(1+1\cdot 3))^{-1}
\equiv -\frac{9}{9}(8)^{-1} \equiv -2 \equiv 1 \pmod{3}.
\]
So $a_2=1$.
Then,
\[
a_3 \equiv -\frac{f(1+1\cdot 3+1\cdot 3^2)}{3^3} (f'(1+1\cdot 3+1\cdot 3^2))^{-1}
\equiv -6 \cdot 2 
\equiv 0 \pmod{3}.
\]
So, $a_3=0$. Then, 
\[
a_4 \equiv -\frac{f(1+1\cdot 3+1\cdot 3^2+0\cdot 3^3)}{3^4}(f'(1+1\cdot 3+1\cdot 3^2+0\cdot 3^3))^{-1}
\equiv -2 \cdot 2 \equiv 2 \pmod{3}.
\]
So, $a_4=2$, etc.
\end{example}


\section{Absolute values on fields}
If $K$ is a field, an \textbf{absolute value on $K$} is a map $|\cdot|:K \to \mathbb{R}_{\geq 0}$ such that
$|x|=0$ if and only if $x=0$, $|xy|=|x||y|$, and $|x+y| \leq |x|+|y|$. The \textbf{trivial absolute value on $K$}
is $|0|=0$ and $|x|=1$ for all nonzero $x \in K$. 

If $|\cdot|$ is an absolute value on $K$, then $d(x,y)=|x-y|$ is a metric on $K$. The trivial absolute value yields the discrete
metric. 
Two absolute values $|\cdot|_1,|\cdot|_2$ on $K$ are said to be \textbf{equivalent} if 
they induce the same topology on $K$. 

The following theorem characterizes equivalent absolute values.\footnote{{\em Absolute values, valuations and completion}, \url{https://www.math.ethz.ch/education/bachelor/seminars/fs2008/algebra/Crivelli.pdf}}

\begin{theorem}
Two nontrivial absolute values $|\cdot|_1,|\cdot|_2$ are equivalent if and only if there is some  real $s>0$
such that
\[
|x|_1=|x|_2^s, \qquad x \in K.
\]
\label{equivalent}
\end{theorem}
\begin{proof}
Suppose that $s>0$ and that $|x|_1=|x|_2^s$ for all $x \in K$. Then
\begin{align*}
B_{d_1}(x,r) &= \{y \in K: |y-x|_1 < r\}\\
&=\{y \in K: |y-x|_2^s < r\}\\
&=\{y \in K: |y-x|_2 < r^{1/s}\}\\
&=B_{d_2}(x,r^{1/s}).
\end{align*}
Since the collection of open balls for $d_1$ is equal to the collection of open balls for $d_2$, the absolute values $|\cdot|_1,|\cdot|_2$ induce the same
topology on $K$. 

Suppose that $|\cdot|_1,|\cdot|_2$ are equivalent.
If $|x|_1<1$ then $d_1(x^n,0)=|x^n|_1= |x|_1^n   \to 0$ as $n \to \infty$. Thus $x^n \to 0$ in $d_1$ and
hence, because the topologies induced by $|\cdot|_1$ and $|\cdot|_2$ are equal,
$x^n \to 0$ in $d_2$, i.e. $|x|_2^n = |x^n|_2 = d_2(x^n,0) \to 0$. Therefore $|x|_2<1$. Thus, $|x|_1 < 1$ if and only if 
$|x|_2 < 1$. 

Let $y \in K$ such that $|y|_1>1$ (there is such an element because $|\cdot|_1$ is nontrivial and
 $|y^{-1}|_1 = |y|_1^{-1}$) and
let $x \in K$ with $|x|_1 \neq 0, 1$. There is some nonzero $\alpha \in \mathbb{R}$ such that $|x|_1=|y|_1^\alpha$. 
Let $\frac{m_i}{n_i} \in \mathbb{Q}$ all be greater than $\alpha$ and converge to $\alpha$. 
Then, because $|y|_1>1$,
we have $|x|_1 = |y|_1^\alpha < |y|_1^{\frac{m_i}{n_i}}$, hence
$|x|_1^{n_i} < |y|_1^{m_i}$, hence
$\frac{|x^{n_i}|_1}{ |y^{m_i}|_1}<1$, hence
\[
\left| \frac{x^{n_i}}{y^{m_i}} \right|_1 < 1.
\]
Because $|\cdot|_1$ and $|\cdot|_2$ are equivalent, 
\[
\frac{|x|_2^{n_i}}{|y|_2^{m_i}}
=
 \left| \frac{x^{n_i}}{y^{m_i}} \right|_2 < 1,
\]
so $|x|_2<|y|_2^{\frac{m_i}{n_i}}$. Taking $i \to \infty$ gives
\[
|x|_2 \leq |y|_2^\alpha.
\]
Similarly, we check that 
\[
|x|_2 \geq |y|_2^\alpha.
\]
Therefore,
\[
|x|_2 = |y|_2^\alpha.
\]
Using this and $|x|_1=|y|_1^\alpha$, we have
\[
\log |x|_1 = \alpha \log |y|_1, \qquad \log |x|_2 = \alpha \log |y|_2,
\]
and so, as $\alpha \neq 0$,
\[
\frac{\log |x|_1}{\log |x|_2} = \frac{\log |y|_1}{\log |y|_2}.
\]
This is true for any $x \in K$ with $|x|_1 \neq 0, 1$. We define $s \in \mathbb{R}$ to be this common value. The fact that
$|y|_1>1$ implies, because $|\cdot|_1$ and $|\cdot|_2$ are equivalent, that $|y|_2>1$, and so
$s>0$. 

Now take $x \in K$. If $x=0$ then $|x|_1=0=0^s=|x|_2^s$. Because $|\cdot|_1$ and $|\cdot|_2$ are equivalent, $|x|_2 >1$ implies that
$|x|_1>1$ and $|x|_2<1$ implies that $|x|_1<1$, so if $|x|_1=1$ then $|x|_2=1$ and hence
$|x|_1 =1=1^s= |x|_2^s$. If $|x|_1 \neq 0, 1$, then the above shows that
\[
\frac{\log |x|_1}{\log |x|_2} = s,
\]
i.e., $|x|_1 = |x|_2^s$, proving the claim.
\end{proof}




An absolute value $|\cdot|:K \to  \mathbb{R}_{\geq 0}$ is said to be \textbf{non-Archimedean} if
\[
|x+y| \leq \max\{|x|,|y|\}, \qquad x,y \in K.
\]
An absolute value is called \textbf{Archimedean} if it is not non-Archimedean. For example, the absolute value on the field
$\mathbb{R}$ is Archimedean, since, for example, $|1+1|=2>\max\{|1|,|1|\}=1$. 

\begin{lemma}
If $|\cdot|$ is a non-Archimedean absolute value on a field $K$ and $|x| \neq |y|$, then
\[
|x+y| = \max\{|x|,|y|\}.
\]
\end{lemma}





\section{Valuations}
A \textbf{valuation} on a field $K$ is a function $v:K \to \mathbb{R} \cup \{\infty\}$
satisfying $v(x)=\infty$ if and only if $x=0$, $v(xy)=v(x)+v(y)$, and
\[
v(x+y) \geq \min\{v(x),v(y)\}.
\]
The \textbf{trivial valuation} is $v(x)=0$ for $x \neq 0$ and $v(0)=\infty$. 

\begin{lemma}
Let $v$ be a valuation on a field $K$. If $v(x) \neq v(y)$, then $v(x+y)=\min\{v(x),v(y)\}$.
\end{lemma}
\begin{proof}
Take $v(y)<v(x) \leq \infty$. For $x=0$,
\[
v(x+y)=v(y)=\min\{\infty,v(y)\} = \min\{v(x),v(y)\}.
\]
For $x \neq 0$, assume by contradiction that $\min\{v(x+y),v(x)\}=v(x)$. Then, since $v(-x)=v(-1\cdot x)=v(-1)+ v(x)=v(x)$,
\[
v(x)>v(y)
=v(x+y-x)
\geq \min\{v(x+y),v(x)\}
=v(x),
\]
a contradiction. Hence $\min\{v(x+y),v(x)\}=v(x+y)$. Then
\begin{align*}
v(y)&=v(x+y-x)\\
&\geq \min\{v(x+y),v(x)\}\\
&=v(x+y)\\
&\geq \min\{v(x),v(y)\}\\
&=v(y).
\end{align*}
Hence $v(x+y)=v(y)=  \min\{v(x),v(y)\}$, completing the proof.
\end{proof}


\begin{theorem}
Let $K$ be a field. If $|\cdot|$  is a non-Archimedean absolute value on $K$
 and $s>0$, then $v_s:K \to \mathbb{R} \cup \{\infty\}$ defined by
 $v_s(x)=-s \log |x|$ for $x \neq 0$ and $v_s(0)=\infty$ is a valuation on $K$.
 
 If $v$ is a valuation on $K$ and $q>1$, then the function $|\cdot|_q:K \to \mathbb{R}_{\geq 0}$ defined by
 $|x|_q=q^{-v(x)}$ for $x \neq 0$ and $|0|_q=0$ is a non-Archimedean absolute value on $K$.
\end{theorem}
\begin{proof}
Suppose that $|\cdot|$ is a non-Archimedean absolute value on $K$ and that $s>0$. 
Let $x,y \in K$. If either is $0$, then it is immediate that $v_s(xy)=\infty=v_s(x)+v_s(y)$. 
If neither is $0$, then
\[
v_s(xy)=-s \log |xy| = -s \log(|x||y|)
=-s\log|x|-s\log|y|=v_s(x)+v_s(y).
\]
Now, if both $x,y$ are $0$ then
\[
v_s(x+y)=v_s(0)=\infty = \min\{\infty,\infty\} = \min\{v_s(x),v_s(y)\}.
\]
If $x=0$ and $y \neq 0$ then
\[
v_s(x+y)=v_s(y)=-s\log|y| = \min\{-s\log|y|,\infty\}=\min\{v_s(y),v_s(x)\}.
\]
If neither $x,y$ is $0$ but $x = -y$, then
\[
v_s(x+y)=v_s(0)=\infty \geq \min\{v_s(x),v_s(y)\}.
\]
Finally, if neither $x,y$ is $0$ and $x \neq -y$, then, because $|\cdot|$ is non-Archimedean,
\begin{align*}
v_s(x+y)&=-s\log|x+y|\\
&\geq -s\log (\max\{|x|,|y|\})\\
&=\min\{-s\log|x|,-s\log|y|\}\\
&=\min\{v_s(x),v_s(y)\}.
\end{align*}
Thus $v_s$ is a valuation on $K$.

Suppose that $v$ is a valuation on $K$ and that $q>1$. If $x,y$ are nonzero, then 
\[
|xy|_q=q^{-v(xy)} = q^{-v(x)-v(y)}=q^{-v(x)} q^{-v(y)} = |x|_q |y|_q.
\]
Let $x,y \in K$. To show that $|x+y|_q \leq |x|_q + |y|_q$, it suffices to show that
$|x+y|_q \leq \max\{|x|_q,|y|_q\}$; proving this will establish that $|\cdot|_q$ is an absolute value and furthermore
that $|\cdot|_q$ is non-Archimedean. If $x,y$ are both $0$, then
$|x+y|_q=|0|_q = 0 = \max\{0,0\}=\max\{|x|_q,|y|_q\}$. If $x =0$ and $y \neq 0$, then
$|x+y|_q=|y|_q = q^{-v(y)} = \max\{q^{-v(y)},0\}=\max\{|y|_q,|x|_q\}$. If neither $x,y$ is $0$ but
$x=-y$, then
\[
|x+y|_q=|0|_q=0 \leq \max\{|x|_q,|y|_q\}.
\]
Finally, if neither $x,y$ is $0$ and $x \neq -y$, then
\begin{align*}
|x+y|_q&=q^{-v(x+y)}\\
&\leq q^{-\min\{v(x),v(y)\}}\\
&=\max\{q^{-v(x)},q^{-v(y)}\}\\
&=\max\{|x|_q,|y|_q\}.
\end{align*}
\end{proof}


Two valuations $v_1,v_2$ on a field $K$ are said to be \textbf{equivalent} if there is some
real $s>0$ such that
\[
v_1=sv_2.
\]

A valuation $v$ on a field $K$ is said to be \textbf{discrete} if there is some real $s>0$ such that
\[
v(K^*) = s\mathbb{Z}.
\]
A valuation is said to be \textbf{normalized} if
\[
v(K^*)=\mathbb{Z}.
\]


\section{Valuation rings}
\begin{theorem}
If $K$ is a field and $v$ is a nontrivial valuation on $K$, then 
\[
\mathcal{O}_v = \{x \in K: v(x) \geq 0\}
\]
is a maximal proper subring of $K$, and for all $x \neq 0$, $x \in \mathcal{O}_v$ or $x^{-1} \in
\mathcal{O}_v$. The set
\[
\{x \in K: v(x)=0\}
\]
is the group of invertible elements of $\mathcal{O}_v$, and the set 
\[
\mathfrak{p}_v=\{x \in K: v(x)>0\}
\]
is the unique maximal ideal  of $\mathcal{O}_v$.
\label{valring}
\end{theorem}
\begin{proof}
It is immediate that $0,1 \in \mathcal{O}_v$. For $x \in \mathcal{O}_v$, $v(-x)=v(x) \geq 0$, so
$-x \in \mathcal{O}_v$. For $x,y \in \mathcal{O}_v$, $v(xy)=v(x)+v(y) \geq 0$, so $xy \in \mathcal{O}_v$. 
And
$v(x+y) \geq \min\{v(x),v(y)\} \geq 0$, so $x+y \in \mathcal{O}_v$.
Thus
$\mathcal{O}_v$ is a subring of $K$. 
For nonzero $x \in K$, if $v(x) \geq 0$ then $x \in \mathcal{O}_v$, and if
$v(x)<0$ then $v(x^{-1})=-v(x)>0$, so $x^{-1} \in \mathcal{O}_v$.

Since $v$ is nontrivial, there is some $x \in K$ with $v(x) \neq 0, \infty$. If $x \in \mathcal{O}_v$ then
$v(x)>0$ and
so $v(x^{-1})=-v(x)<0$, giving $x^{-1} \not \in \mathcal{O}_v$. Hence $\mathcal{O}_v \neq K$, showing
that $\mathcal{O}_v$ is a proper subring of $K$. 

To show that $\mathcal{O}_v$ is a maximal proper subring, it suffices to show that
if $z \in K \setminus \mathcal{O}_v$ then $\mathcal{O}_v[z] = K$, i.e., that the smallest ring containing
$\mathcal{O}_v$ and $z$ is $K$. As $z \not \in \mathcal{O}_v$, $v(z)<0$.
Let $y \in K$. For any positive integer $j$ we have $v(y z^{-j})=v(y)-jv(z)$, and because
$v(z)<0$, there is some
$j=j(y)$ such that $v(yz^{-j})>0$. For this $j$, $yz^{-j} \in \mathcal{O}_v$. Hence
$y \in \mathcal{O}_v[z]$, and so $\mathcal{O}_v[z] = K$, showing that $\mathcal{O}_v$ is a maximal
proper subring.

Suppose that $x \in \mathcal{O}_v$ and $x^{-1} \in \mathcal{O}_v$. If $v(x)>0$, then
$v(x^{-1}=-v(x)<0$, contradicting that $x^{-1} \in \mathcal{O}_v$. Hence $v(x)=0$. 
If $v(x)=0$, then, as $x^{-1} \in K$, $v(x^{-1})=-v(x)=0$, so $x^{-1} \in \mathcal{O}_v$, hence
$x$ is an element of $\mathcal{O}_v$ whose inverse is in $\mathcal{O}_v$.

Let $x,y \in  \mathfrak{p}_v$. Then, since $v(x)>0$ and $v(y)>0$,
\[
v(x-y) \geq \min\{v(x),v(-y)\}=
\min\{v(x),v(y)\} >0,
\]
showing that $x-y \in \mathfrak{p}_v$, and thus that $\mathfrak{p}_v$ is an additive subgroup
of $\mathcal{O}_v$. Let $x \in \mathfrak{p}_v$ and $z \in \mathcal{O}_v$.  
Then, since $v(z) \geq 0$ and $v(x)>0$,
\[
v(zx)=v(z)+v(x) \geq v(x)>0,
\]
showing that $zx \in \mathfrak{p}_v$. Therefore $\mathfrak{p}_v$ is an ideal in the ring
$\mathcal{O}_v$. Since $v(1)=0$, $1 \not \in \mathfrak{p}_v$, so $\mathfrak{p}_v$ is a proper
ideal. 

The fact that $\mathfrak{p}_v$ is maximal follows from it being the set of noninvertible elements
of $\mathcal{O}_v$. Suppose that $B$ is a maximal ideal $B$ of
$\mathcal{O}_v$.  Because $B$ is a proper ideal it contains
no invertible elements, and hence is contained in $\mathfrak{p}_v$, the set of noninvertible
elements of $\mathcal{O}_v$. Since $B$ is maximal, it must be that $B=\mathfrak{p}_v$.
Therefore, any maximal ideal of $\mathcal{O}_v$ is $\mathfrak{p}_v$, showing that
$\mathfrak{p}_v$ is the unique maximal ideal of $\mathcal{O}_v$.
\end{proof}

The above ring $\mathcal{O}_v$
is called the \textbf{valuation ring}. Generally, a ring that has a unique maximal ideal is called a \textbf{local ring}, and thus the above
theorem shows that the valuation ring is a local ring.
We call the quotient $\mathcal{O}_v / \mathfrak{p}_v$ the \textbf{residue field of $\mathcal{O}_v$}. 


\begin{lemma}
If $v$ is a normalized valuation on a field $K$ then for all nonzero $x \in K$
and $t \in \mathfrak{p}_v$, $v(t)=1$,
there is some $u \in \mathcal{O}_v^*$ such that
\[
x=ut^n, \qquad n = v(x).
\]
\label{utnlemma}
\end{lemma}
\begin{proof}
Since $x \neq 0$, $v(x) =n \in \mathbb{Z}$. 
Hence $v(xt^{-n})=v(x)-nv(t)=v(x)-n=0$, and therefore
$u=xt^{-n}  \in \mathcal{O}^*$. Then $x=ut^n$, completing the proof.
\end{proof}







\begin{theorem}
If $v$ is a normalized valuation on a field $K$, then $\mathcal{O}_v$ is a principal
ideal domain. If $A$ is a nonzero ideal of $\mathcal{O}_v$, then there is some
$t \in \mathfrak{p}$, $v(t)=1$ and $n \geq 0$ such that 
\[
A = t^n \mathcal{O}_v = \{x \in K: v(x) \geq n\}=\mathfrak{p}_v^n,
\]
and
\[
\mathfrak{p}_v^n/ \mathfrak{p}_v^{n+1} \cong \mathcal{O}_v / \mathfrak{p}_v,
\]
as $ \mathcal{O}_v / \mathfrak{p}_v$-linear vector spaces.
\end{theorem}
\begin{proof}
Let $A \neq \{0\}$ be an ideal of $\mathcal{O}_v$. For any $y \in A$, $v(y) \geq 0$, and we take
$x \in A$ such that
\begin{equation}
v(x) = \min\{v(y): y \in A\}.
\label{minv}
\end{equation}
Since
$v(K^*)=\mathbb{Z}$,  there is  some $t \in K$ with
$v(t)=1$, and because $v(t)>0$, $t \in \mathfrak{p}_v$. 
By Lemma \ref{utnlemma}, there is some $u \in \mathcal{O}^*$ such that
$x=ut^n$, $n=v(x)$. 
For any $z \in \mathcal{O}$, $xz \in A$ and so $t^nz \in A$. Thus $t^n \mathcal{O}_v \subset A$. 
On the other hand, let $y \in A$. Then also by Lemma \ref{utnlemma} there is some $w \in \mathcal{O}_v^*$
such that $y=wt^m$, $m=v(y)$. By \eqref{minv}, $m=v(y) \geq v(x) = n$, so $v(t^{m-n})=(m-n)v(t)=m-n \geq 0$
so $t^{m-n} \in \mathcal{O}_v$, giving
\[
y=wt^m = t^n (wt^{m-n})  \in t^n \mathcal{O}_v.
\]
Therefore $A \subset t^n\mathcal{O}_v$, and so $A=t^n \mathcal{O}_v$. That is, $A$ is the principal ideal generated by $t^n$,
which shows that $\mathcal{O}_v$ is a principal ideal domain. 

Let $t \in \mathfrak{p}_v$ with $v(t)=1$, and 
define $\phi:\mathfrak{p}_v^n \to \mathcal{O}_v / \mathfrak{p}_v$ by
$v(at^n) = a+\mathfrak{p}$, for $a \in \mathcal{O}_v$. 
\end{proof}


\begin{lemma}
If $v_1,v_2$ are discrete valuations on a field $K$ such that $\mathcal{O}_{v_1}=\mathcal{O}_{v_2}$, then 
$v_1$ and $v_2$ are equivalent. 
\end{lemma}


\section{$p$-adic valuations}
Fix a prime number $p$. For nonzero $a \in \mathbb{Q}$, there are unique integers $n,r,s$ satisfying
\[
a=\frac{r}{s}p^n,
\]
where $r,s$ are coprime, $s>0$, and $p \nmid rs$. 
We define $v_p(a)=n$. 
Furthermore, we define $v_p(0)=\infty$. 


\begin{theorem}
$v_p:\mathbb{Q} \to \mathbb{R} \cup \{\infty\}$ is a normalized valuation.
\end{theorem}
\begin{proof}
For nonzero $a,b \in \mathbb{Q}$,
write
\[
a=\frac{r_1}{s_1} p^m, \qquad b = \frac{r_2}{s_2}p^n,
\]
where $\gcd(r_1,s_1)=\gcd(r_2,s_2)=1$, $s_1,s_2>0$, and $p
\nmid r_1s_1, p \nmid r_2s_2$. Then,
\[
ab = \frac{r_1r_2}{s_1s_2} p^{m+n},
\]
where $p \nmid r_1s_1r_2s_2$; the fraction $\frac{r_1r_2}{s_1s_2}$ need not be in lowest terms. 
So $v_p(ab)=m+n=v_p(a)+v_p(n)$.

Suppose that $v_p(a) \leq   v_p(b)$. Then
\[
a+b = \frac{r_1}{s_1} p^m +  \frac{r_2}{s_2}p^n
= \left( \frac{r_1}{s_1} +\frac{r_2}{s_2} p^{n-m}\right)p^m
=\frac{r_1s_2+r_2s_1p^{n-m}}{s_1s_2} p^m.
\]
Since $p \nmid s_1$ and $p \nmid s_2$, 
then
\[
v_p(a+b) \geq m = v_p(a) = \min\{v_p(a),v_p(b)\}.
\]
\end{proof}

We call $v_p$ the \textbf{$p$-adic valuation}.
The valuation ring of $\mathbb{Q}$ corresponding to  $v_p$ is
\[
\mathcal{O}_p = \{x \in \mathbb{Q}: v_p(x) \geq 0\}, 
\]
in other words, those rational numbers such that in lowest terms, $p$ does not divide their denominator. For example,
$\frac{11}{169}, -\frac{9}{35} \in \mathcal{O}_3$, and $\frac{5}{3} \not \in \mathcal{O}_3$. 
By Theorem \ref{valring}, the group of units of the valuation ring $\mathcal{O}_p$ is
\[
\mathcal{O}_p^*= \{x \in \mathbb{Q}: v_p(x)=0\},
\]
in other words, those rational numbers such that in lowest terms, $p$ divides neither their numerator nor their denominator. 
As well by Theorem \ref{valring}, $\mathcal{O}_p$ is a local ring whose unique maximal ideal is
\[
\mathfrak{p}_p = \{x \in \mathbb{Q}: v_p(x)>0\},
\]
in other words, those rational numbers such that in lowest terms, $p$ divides their numerator and does not divide their
denominator. 
We see that $p \in \mathfrak{p}_p$ and $v_p(p)=1$, so the nonzero ideals of $\mathcal{O}_p$ are of the form
\[
p^n \mathcal{O}_p.
\]


\begin{lemma}
$\mathcal{O}_p/\mathfrak{p}_p \cong \mathbb{Z}/p\mathbb{Z}$.
\end{lemma}


\section{$p$-adic absolute values and metrics}
We define $|\cdot|_p:\mathbb{Q} \to \mathbb{R}_{\geq 0}$ by $|a|_p=p^{-v_p(n)}$ for $a \neq 0$ and $|0|_p=0$. 
This is a non-Archimedean absolute value on $\mathbb{Q}$, which we call
 the \textbf{$p$-adic absolute value}. 

\begin{example}
For 
$p=3$ and 
$a=-\frac{57}{10}$,
we have $n=1, r=-19, s=10$. 
Thus
$\left|-\frac{57}{10}\right|_3=3^{-1}$.

For $p=5$ and $a=\frac{28}{75}$, we have $n=-2, r=28, s=3$. Thus 
$\left| \frac{28}{75} \right|_5 = 5^2$.
\end{example}

We define $d_p(x,y)=|x-y|_p$.  The sequences $x_l = a_0+a_1p+a_2p^2+\cdots+a_{l-1} p^{l-1}$ constructed when applying Hensel's
lemma satisfy, for $m < n$,
\[
x_n-x_m = a_mp^m+a_{m+1}p^{m+1}+\cdots
+a_{n-1}p^{n-1} \equiv 0 \pmod{p^m},
\]
so
\[
|x_n-x_m|_p \leq p^{-m},
\]
and 
\[
f(x_n) \equiv 0 \pmod{p^n},
\]
so
\[
|f(x_n)|_p \leq p^{-n}.
\]
Thus, $x_n$ is a Cauchy sequence in the $p$-adic metric $d_p(x,y)=|x-y|_p$, and 
$f(x_n) \to 0$ as $n \to \infty$. 

\begin{lemma}
If $x_n$ and $y_n$ are  Cauchy sequences
in $(\mathbb{Q},d_p)$, then $x_n+y_n$ and $x_n\cdot y_n$ are Cauchy sequences in
$(\mathbb{Q},d_p)$.
\label{cauchysum}
\end{lemma}
\begin{proof}
The claim follows from
\[
|x_n+y_n-(x_m+y_m)|_p
\leq |x_n-x_m|_p + |y_n-y_m|_p
\]
and
\begin{align*}
|x_n\cdot y_n - x_m\cdot y_m|_p &=|x_n \cdot y_n - x_m \cdot y_n
+x_m\cdot y_n  - x_m \cdot y_m|_p\\
&\leq |x_n-x_m|_p |y_n|_p + |x_m|_p |y_n-y_m|_p,
\end{align*}
and the fact that $x_n,y_n$ being Cauchy implies that $|x_n|_p,|y_n|_p$ are bounded. 
\end{proof}


\section{Completions of metric spaces}
If $(X,d)$ is a metric space, a \textbf{completion} of $X$ is a complete metric space $(Y,\rho)$ and an
isometry $i:X \to Y$ such that for every metric space $(Z,r)$ and isometry $j:X \to Z$, there is a unique
isometry $J:Y \to Z$ such that $J \circ i = j$. It is a fact that any metric space has a completion,
and that if $(Y_1,\rho_1)$ and $(Y_2,\rho_2)$ are completions then there is a unique
isometric isomorphism $f:Y_1 \to Y_2$. 

For $p$ prime, let $(\mathbb{Q}_p,d_p)$ be the completion of $(\mathbb{Q},d_p)$. Elements
of $\mathbb{Q}_p$ are called \textbf{$p$-adic numbers}.
For $x,y \in \mathbb{Q}_p$, there are Cauchy sequences $x_n,y_n$ in $(\mathbb{Q},d_p)$ such that
$x_n \to x$ and $y_n \to y$ in $(\mathbb{Q}_p,d_p)$. 
We define addition and multiplication on the set $\mathbb{Q}_p$ by
\[
x+y = \lim (x_n+y_n), \qquad x\cdot y= \lim (x_n\cdot y_n);
\]
that these limits exists follows from Lemma \ref{cauchysum}. 
If $x \in \mathbb{Q}_p$, $x \neq 0$, then there is a sequence $x_n \in \mathbb{Q}$, each term of which is $\neq 0$, 
such that $x_n \to x$ in $(\mathbb{Q}_p,d_p)$. 
Then $x_n^{-1}$ is a Cauchy sequence in $(\mathbb{Q},d_p)$ hence converges to some $y \in \mathbb{Q}_p$ which
satisfies $x\cdot y =1$. Therefore $\mathbb{Q}_p$ is a field. 

We define $v_p:\mathbb{Q}_p \to \mathbb{R} \cup \{\infty\}$ 
\[
v_p(x) = \lim v_p(x_n), \qquad x_n \to x.
\]
One proves that $v_p$ is a normalized valuation on the field $\mathbb{Q}_p$.\footnote{cf. Paul Garrett, {\em Classical definitions of $\mathbb{Z}_p$ and $\mathbb{A}$},
\url{http://www.math.umn.edu/~garrett/m/mfms/notes/05_compare_classical.pdf}}
We then 
define $|\cdot|_p:\mathbb{Q}_p \to \mathbb{R}_{\geq 0}$ by
$|x|_p=p^{-v_p(x)}$ for $x \neq 0$ and $|0|_p=\infty$. 



\section{The exponential function}
\begin{lemma}
For $a_1,\ldots,a_r \in \mathbb{Q}_p$,
\[
|a_1+\cdots+a_r|_p \leq \max\{|a_1|,\ldots,|a_r|\}.
\]
\end{lemma}

\begin{lemma}
A sequence $a_i \in \mathbb{Q}_p$ is Cauchy if and only if $a_{i+1}-a_i \to 0$ as $i \to \infty$.
\end{lemma}
\begin{proof}
Assume that $a_{i+1}-a_i \to 0$ and let $\epsilon>0$. Then there is some $i_0$ such that $i \geq i_0$ implies
$|a_{i+1}-a_i|_p < \epsilon$. For $i_0 \leq i < j$,
\begin{align*}
|a_j-a_i|_p&=|a_j-a_{j-1}+a_{j-1}+\cdots-a_{i+1}+a_{i+1}-a_i|_p\\
&=|(a_j-a_{j-1})+\cdots+(a_{i+1}-a_i)|_p\\
&\leq \max\{|a_j-a_{j-1}|,\ldots,|a_{i+1}-a_i|\}\\
&<\epsilon.
\end{align*}
\end{proof}

The above shows that if $a_i \to 0$ in $(\mathbb{Q}_p,d_p)$ then the series $\sum a_i$ converges in
$(\mathbb{Q}_p,d_p)$. 

\begin{lemma}[Exponential power series]
If $v_p(x)>\frac{1}{p-1}$, then
\[
\sum_{n=0}^\infty \frac{x^n}{n!}
\]
converges in $(\mathbb{Q}_p,d_p)$. 
\end{lemma}
\begin{proof}
\[
v_p(n!) = \sum_{j=1}^\infty \left[ \frac{n}{p^j} \right]
\leq \sum_{j=1}^\infty \frac{n}{p^j}
=\frac{1}{np} \frac{1}{1-\frac{1}{p}}
=\frac{n}{p-1}.
\]
Then
\[
v_p\left( \frac{x^n}{n!} \right) = n v_p(x) - v_p(n!) 
\geq n v_p(x) - \frac{n}{p-1}
=n\left(v_p(x)-\frac{1}{p-1}\right).
\]
As $n \to \infty$ this tends to $+\infty$, hence
\[
\left| \frac{x^n}{n!} \right|_p = p^{-v_p\left( \frac{x^n}{n!} \right)} \to 0,
\]
and thus the series $\sum_{n=0}^\infty \frac{x^n}{n!}$ converges. 
\end{proof}


\begin{lemma}[Logarithm power series]
If $v_p(x)>0$, then
\[
\sum_{n=1}^\infty (-1)^{n+1} \frac{x^n}{n}
\]
converges in $(\mathbb{Q}_p,d_p)$. 
\end{lemma}
\begin{proof}
For $n$ a positive integer we have $v_p(n) \leq \log_p n$. Then,
\[
v_p\left(\frac{x^n}{n}\right)=
nv_p(x)-v_p(n) \geq 
nv_p(x)-\log_p n.
\]
If $v_p(x)>0$ then this tends to $+\infty$ as $n \to \infty$.
\end{proof}



\section{Topology}
We define $\mathbb{Z}_p$ to be the valuation ring of $\mathbb{Q}_p$. Elements of $\mathbb{Z}_p$ are called
\textbf{$p$-adic integers}. 
For $x \in \mathbb{Q}_p$ and real $r>0$, write
\[
\overline{B}_p(r,x) = \{y \in \mathbb{Q}_p:|x-y|_p \leq r\}
=\{y \in \mathbb{Q}_p: v_p(x-y) \geq -\log_p r\}.
\]
In particular,
\[
\overline{B}_p(0,1) = \mathbb{Z}_p.
\]
Because $v_p$ is discrete, there is some $\epsilon>0$ such that
\[
\{y \in \mathbb{Q}_p: |x-y|_p \leq r\} = 
\{y \in \mathbb{Q}_p: |x-y|_p < r+\epsilon\}.
\]
This shows that $\overline{B}_p(x,r)$ is open in the topology induced by $v_p$, and thus is both closed and open. 
It follows that $\mathbb{Q}_p$ is \textbf{totally disconnected}.\footnote{Gerald B. Folland, {\em A Course in Abstract Harmonic Analysis}, pp.~34--36.}


\begin{theorem}
$\mathbb{Z}_p$ is totally bounded.
\end{theorem}

The fact that $\mathbb{Z}_p$ is a totally bounded subset of a complete metric space implies that $\mathbb{Z}_p$ is compact. Then because
\[
\overline{B}_d(0,p^k)=\{y \in \mathbb{Q}_p: |y|_p \leq p^k\} = \{y \in \mathbb{Q}_p: |p^k y|_p \leq 1\}
=p^{-k} \mathbb{Z}_p
\]
and translation is a homeomorphism, any closed ball in $\mathbb{Q}_p$ is compact. 
Therefore $\mathbb{Q}_p$ is locally compact. 

$\mathbb{Q}_p$ is a locally compact abelian group under addition, and we take Haar measure on it
satisfying $\mu(\mathbb{Z}_p)=1$. One can explicitly calculate the characters on $\mathbb{Q}_p$.\footnote{Gerald B. Folland, {\em A Course in Abstract Harmonic Analysis}, pp.~91--93, 104. Cf. Keith Conrad, {\em The character group of $\mathbf{Q}$}, \url{http://www.math.uconn.edu/~kconrad/blurbs/gradnumthy/characterQ.pdf}}



\end{document}