\documentclass{article}
\usepackage{amsmath,amssymb,mathrsfs,amsthm}
%\usepackage{tikz-cd}
%\usepackage{hyperref}
\newcommand{\inner}[2]{\left\langle #1, #2 \right\rangle}
\newcommand{\tr}{\ensuremath\mathrm{tr}\,} 
\newcommand{\Span}{\ensuremath\mathrm{span}} 
\def\Re{\ensuremath{\mathrm{Re}}\,}
\def\Im{\ensuremath{\mathrm{Im}}\,}
\newcommand{\id}{\ensuremath\mathrm{id}} 
\newcommand{\var}{\ensuremath\mathrm{var}} 
\newcommand{\Lip}{\ensuremath\mathrm{Lip}} 
\newcommand{\GL}{\ensuremath\mathrm{GL}} 
\newcommand{\diam}{\ensuremath\mathrm{diam}} 
\newcommand{\sgn}{\ensuremath\mathrm{sgn}\,} 
\newcommand{\lcm}{\ensuremath\mathrm{lcm}} 
\newcommand{\supp}{\ensuremath\mathrm{supp}\,}
\newcommand{\dom}{\ensuremath\mathrm{dom}\,}
\newcommand{\upto}{\nearrow}
\newcommand{\downto}{\searrow}
\newcommand{\norm}[1]{\left\Vert #1 \right\Vert}
\newtheorem{theorem}{Theorem}
\newtheorem{lemma}[theorem]{Lemma}
\newtheorem{proposition}[theorem]{Proposition}
\newtheorem{corollary}[theorem]{Corollary}
\theoremstyle{definition}
\newtheorem{definition}[theorem]{Definition}
\newtheorem{example}[theorem]{Example}
\begin{document}
\title{$p$-adic test functions}
\author{Jordan Bell\\ \texttt{jordan.bell@gmail.com}\\Department of Mathematics, University of Toronto}
\date{\today}

\maketitle


\section{$\mathbb{Q}_p^n$}
Let $p$ be prime, let $N_p = \{0,\ldots,p-1\}$, and let $\mathbb{Q}_p \subset \prod_{\mathbb{Z}} N_p$ be the $p$-adic numbers.
For $x \in \mathbb{Q}_p$ let
\[
v_p(x) = \inf\{k \in \mathbb{Z}: x(k) \neq 0\},\qquad |x|_p = p^{-v_p(x)}.
\]
For $r>0$ and $a \in \mathbb{Q}_p$ let
\[
B_{\leq r}(a) = \{x \in \mathbb{Q}_p : |x-a|_p \leq r\},
\]
and let 
\[
\mathbb{Z}_p = \{x \in \mathbb{Q}_p : v_p(x) \geq 0\} = B_{\leq 1}(0).
\]
For $l \in \mathbb{Z}$, $v_p(p^l) = l$, $|p^l|_p = p^{-l}$, and
\[
p^l \mathbb{Z}_p = \{x \in \mathbb{Q}_p : v_p(x) \geq l\} = B_{\leq p^{-l}}(0).
\]
Let $\mu$ be the Haar measure on the additive group $\mathbb{Q}_p$ with $\mu(\mathbb{Z}_p)=1$. It is a fact
that if $A$ is a Borel set in $\mathbb{Q}_p$ and $x \in \mathbb{Q}_p$ then
\[
\mu(x \cdot A) = |x|_p \mu(A).
\]
In particular, for $l \in \mathbb{Z}$ and $x = p^l$, $\mu(p^l \cdot A) = |p^l|_p \mu(A) = p^{-l} \mu(A)$
and so
\[
\mu(p^l \mathbb{Z}_p) = p^{-l},\qquad l \in \mathbb{Z}.
\]


Let $n \geq 1$. For $x \in \mathbb{Q}_p^n$ let
\[
|x|_p = \max\{|x_j|_p : 1 \leq j \leq n\}.
\]
For $r>0$ and $a \in \mathbb{Q}_p$ let
\[
B_{\leq r}^n(a) = \{x \in \mathbb{Q}_p^n : |x-a|_p \leq r\} = \prod_{j=1}^n B_{\leq r}(a_j).
\]
For $l \in \mathbb{Z}$,
\[
p^l \mathbb{Z}_p^n = p^l B_{\leq 1}^n(0) = B_{\leq p^{-l}}^n(0)
=(p^l \mathbb{Z}_p)^n.
\]
Let $\mu_n = \bigotimes_{j=1}^n \mu$, the product measure on the Borel $\sigma$-algebra of $\mathbb{Q}_p^n$. 
Then
\[
\mu_n(p^l \mathbb{Z}_p^n) =  \prod_{j=1}^n  \mu(p^l \mathbb{Z}_p)
=\prod_{j=1}^n p^{-l} = p^{-nl}.
\]




\section{Locally constant functions}
Let $O$ be an open set in $\mathbb{Q}_p^n$. A function $\psi:O \to \mathbb{C}$ is called \textbf{locally constant}
if for each $x \in O$ there is some neighborhood $N_x$ of $x$ such that $\psi(y)=\psi(x)$ for $y \in N_x$. 
In this case, there is some $l(x) \in \mathbb{Z}$ such that $x+p^{l(x)} \mathbb{Z}_p^n \subset N_x$, and so
\[
\psi(x+h) = \psi(x),\qquad x \in O,\quad h \in p^{l(x)} \mathbb{Z}_p^n.
\]
It is immediate that a locally constant function is continuous. Let $\mathcal{E}(O)$ be the collection of locally
constant functions $O \to \mathbb{C}$. 

Because locally constant functions are used often in $p$-adic analysis, it is worthwhile working out some facts about
them.\footnote{S. Albeverio, A. Yu Khrennikov, and V. M. Shelkovich,
{\em Theory of $p$-adic Distributions: Linear and Nonlinear Models},
p.~55, Lemma 4.2.1.}

\begin{lemma}
If $\psi \in \mathcal{E}(\mathbb{Q}_p^n)$ and $K$ is a compact set in $\mathbb{Q}_p^n$, then there is some
$l \in \mathbb{Z}$ such that
\[
\psi(x+h) = \psi(x),\qquad  x \in K,\qquad h \in p^l \mathbb{Z}_p^n.
\]
\end{lemma}
\begin{proof}
Because $K$ is compact it is bounded and so is contained in $p^N \mathbb{Z}_p^n$ for some $m \in \mathbb{Z}$. 
Now, $\{x+p^{l(x)} \mathbb{Z}_p^n: x \in p^N \mathbb{Z}_p^n\}$ is an open cover of $p^N \mathbb{Z}_p^n$, and because
$p^N \mathbb{Z}_p^n$ is compact there are $x^1,\ldots,x^m \in p^N \mathbb{Z}_p^n$ such that
$K \subset \bigcup_{k=1}^m  (x^k+p^{l(x^k)} \mathbb{Z}_p^n)$. We further specify that these sets are pairwise disjoint, which we can
because two balls in $\mathbb{Q}_p^n$ have nonempty intersection if and only if one is contained in another.\footnote{This
is not transparent but is straightforward to check.} 
Let $l=\max\{l(x^k): 1 \leq k \leq m\}$. 
For $x \in K$ there is some $k$ for which
$x \in x^k+p^{l(x^k)} \mathbb{Z}_p^n$, and as $x-x^k \in p^{l(x^k)} \mathbb{Z}_p^n$, for  $h \in p^l \mathbb{Z}_p^n$, 
\[
|x-x^k + h|_p \leq \max(|x-x^k|_p,|h|_p) \leq \max(p^{-l(x^k)},p^{-l}) = p^{-l(x^k)},
\]
i.e. $x-x^k + h \in p^{l(x^k)} \mathbb{Z}_p^n$. 
Then using that $\psi$ is locally constant, with $O=x^k + p^{l(x^k)} \mathbb{Z}_p^n$,
\[
\psi(x+h) = \psi(x^k+(x-x^k+h)) = \psi(x^k).
\]
And $x-x^k \in p^{l(x^k)} \mathbb{Z}_p^n$ means that $\psi(x^k+x-x^k)=\psi(x^k)$, i.e. $\psi(x)=\psi(x^k)$, showing
$\psi(x+h)=\psi(x)$. 
\end{proof}




\section{$p$-adic test functions}
Let $\mathcal{D}(\mathbb{Q}_p^n)$ be the set of those $\psi \in \mathcal{E}(\mathbb{Q}_p^n)$ such that $\supp \psi$ is a compact set.
Elements of $\mathcal{D}(\mathbb{Q}_p^n)$ are called \textbf{$p$-adic test functions}.



\end{document}