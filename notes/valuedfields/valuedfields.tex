\documentclass{article}
\usepackage{amsmath,amssymb,mathrsfs,amsthm}
%\usepackage{tikz-cd}
%\usepackage{hyperref}
\newcommand{\inner}[2]{\left\langle #1, #2 \right\rangle}
\newcommand{\tr}{\ensuremath\mathrm{tr}\,} 
\newcommand{\Span}{\ensuremath\mathrm{span}} 
\def\Re{\ensuremath{\mathrm{Re}}\,}
\def\Im{\ensuremath{\mathrm{Im}}\,}
\newcommand{\id}{\ensuremath\mathrm{id}} 
\newcommand{\var}{\ensuremath\mathrm{var}} 
\newcommand{\Lip}{\ensuremath\mathrm{Lip}} 
\newcommand{\GL}{\ensuremath\mathrm{GL}}
\newcommand{\diam}{\ensuremath\mathrm{diam}} 
\newcommand{\sgn}{\ensuremath\mathrm{sgn}\,} 
\newcommand{\lcm}{\ensuremath\mathrm{lcm}} 
\newcommand{\supp}{\ensuremath\mathrm{supp}\,}
\newcommand{\dom}{\ensuremath\mathrm{dom}\,}
\newcommand{\upto}{\nearrow}
\newcommand{\downto}{\searrow}
\newcommand{\norm}[1]{\left\Vert #1 \right\Vert}
\newtheorem{theorem}{Theorem}
\newtheorem{lemma}[theorem]{Lemma}
\newtheorem{proposition}[theorem]{Proposition}
\newtheorem{corollary}[theorem]{Corollary}
\theoremstyle{definition}
\newtheorem{definition}[theorem]{Definition}
\newtheorem{example}[theorem]{Example}
\begin{document}
\title{Valued fields}
\author{Jordan Bell\\ \texttt{jordan.bell@gmail.com}\\Department of Mathematics, University of Toronto}
\date{\today}

\maketitle

\section{Absolute values}
Let $F$ be a field. An \textbf{absolute value on $F$} is a function $|\cdot|:F \to \mathbb{R}_{\geq 0}$ 
such that (i) $|x|=0$ implies $x=0$, (ii) $|x+y| \leq |x|+|y|$, and (iii) $|xy|=|x||y|$.
The pair $F$ and $|\cdot|$ is called a \textbf{valued field}.
An absolute value is
called \textbf{nonarchimedean} if
\[
|x+y| \leq \max(|x|,|y|),
\]
and \textbf{archimedean} otherwise. 
Because $|xy|=|x||y|$, 
$x \mapsto |x|$ is a group homomorphism $F^* \to \mathbb{R}_{>0}$. 
The \textbf{trivial absolute value} is $|x|=0$ for $x=0$ and $|x|=1$ for $x \neq 1$, which is nonarchimedean.

The \textbf{value group} of $F$ is the image of $F^*$ under $x \mapsto |x|$; it is a subgroup of $\mathbb{R}_{>0}$. 
If the subspace topology on the value group inherited from $\mathbb{R}_{> 0}$ is discrete, then the absolute value is called
\textbf{discrete}. For example, it is a fact that the value group of $(\mathbb{Q}_p,|\cdot|_p)$ is $\{p^{n}: n \in \mathbb{Z}\}$, which is discrete. 

If $|\cdot|$ is an absolute value on a field $F$, let $d(x,y) = |x-y|$. This is a metric on $F$, and with the topology induced
by $d$, $F$ is a topological field.\footnote{Anthony W. Knapp,
{\em Advanced Algebra}, p.~334, Proposition 6.13; the proof is straightforward.}
We call the valued field $F$ \textbf{complete} if $d$ is a complete metric.

If $(F,|\cdot|_F)$ and $(K,|\cdot|_K)$ are valued fields, a \textbf{homomorphism of valued fields} from $F$ to $K$ 
is a field homomorphism $\phi:F \to K$ such that
$|\phi(x)|_K = |x|_F$ for all $x \in F$. If $\phi$ is onto then $\phi$ is called an \textbf{isomorphism of valued fields}.

If $F$ is a field with a nontrivial absolute value $|\cdot|_F$, 
then there is a complete valued field $(K,|\cdot|_K)$ and a homomorphism of valued fields
$\iota:F \to K$ such that $\iota(F)$ is dense in $K$.\footnote{Anthony W. Knapp,
{\em Advanced Algebra}, p.~342, Theorem 6.24; 
W. H. Schikhof, {\em Ultrametric calculus: An introduction to $p$-adic analysis}, p.~15, Theorem 6.3.}
We call $(K,|\cdot|_K)$ a \textbf{completion of $F$}, and if $F$ is nonarchimedean then it has a nonarchimedean completion.
Completions of valued fields have the following \textbf{universal property}: 
if $\iota:(F,|\cdot|_F) \to (K,|\cdot|_K)$ is a completion of the valued field $F$, if 
$(L,|\cdot|_L)$ is a complete valued field, and if $\phi:F \to L$ is a homomorphism of valued fields, then
there is a unique homomorphism of valued fields 
$\Phi:K \to L$ such that $\phi = \Phi \circ \iota$.\footnote{Anthony W. Knapp,
{\em Advanced Algebra}, p.~343, Theorem 6.25.}
It is often permissible to talk \textbf{the completion} $(K,|\cdot|_K)$ of a valued field 
$(F,|\cdot|_F)$ and $F \subset K$ where the restriction of $|\cdot|_K$ to $F$ is $|\cdot|_F$, rather than
$\iota:F \to K$; for some arguments it is necessary to speak about $\iota:F \to K$ rather than $F \subset K$.  


\section{Nonarchimedean valued fields}
Let $F$ be a field with a nontrivial nonarchimedean absolute value $|\cdot|_F$. For $a \in F$ and $r>0$ let
\[
B_{\leq r}(a) = \{x \in F : |x-a|_F \leq r\},
\qquad B_{< r}(a) = \{x \in F : |x-a|_F < r\}.
\]


We now prove that $B_{\leq 1}(0)$ is a local ring whose unique maximal ideal is $B_{<1}(0)$.\footnote{W. H. Schikhof, {\em Ultrametric calculus: An introduction to $p$-adic analysis}, p.~25, Proposition 11.1.}
(A commutative ring $R$ is a \textbf{local ring} if it has a unique maximal ideal.)

\begin{lemma}
$B_{\leq 1}(0)$ is a local ring and 
$B_{<1}(0)$ is the unique maximal ideal in $B_{\leq 1}(0)$.
\end{lemma}
\begin{proof}
If  $x,y \in B_{\leq 1}(0)$ then $|x+y|_F \leq \max(|x|_F,|y|_F) \leq 1$ so $x+y \in B_{\leq 1}(0)$. 
$|-x|_F = |x|_F \leq 1$ so $-x \in B_{\leq 1}(0)$. $|xy|_F = |x|_F |y|_F \leq 1$ so
$xy \in B_{\leq 1}(0)$. $|1|_F = 1$ so $1 \in B_{\leq 1}(0)$. Therefore $B_{\leq 1}(0)$ is a subring of $K$.

For $x,y \in B_{<1}(0)$, $|x+y|_F \leq \max(|x|_F,|y|_F) < 1$ so $x+y \in B_{<1}(0)$. For $x \in B_{<1}(0)$ and $y \in B_{\leq 1}(0)$,
$|xy|_F = |x|_F |y|_F < 1$ so $xy \in B_{<1}(0)$ and therefore $B_{<1}(0)$ is an ideal in the ring $B_{\leq 1}(0)$. 
Now, if $|x|_F =1$ then $x^{-1} \in F$ satisfies $|x^{-1}|_F = 1$. 
Therefore, $B_{<1}(0)$ is the set of elements in $B_{\leq 1}(0)$ which do not have an inverse in $B_{\leq 1}(0)$. Generally,
if $R$ is a commutative ring and the set $M$ of noninvertible elements is an ideal, then $R$ is a local ring with unique maximal ideal $M$.
\end{proof}

The \textbf{residue class field of $F$} is the field
\[
B_{\leq 1}(0)/B_{<1}(0).
\]
It can be proved that a complete nonarchimedean field is locally compact if and only if its residue class field is finite and the absolute value is
discrete.\footnote{W. H. Schikhof, {\em Ultrametric calculus: An introduction to $p$-adic analysis}, p.~29, Corollary 12.2.}


\section{Algebraic closures}
Let $(F,|\cdot|_F)$ be a nonarchimedean valued field and let $K$ be a field containing $F$. 
It can be proved that there is a nonarchimedean absolute value on $K$ whose restriction to $F$ is equal to
$|\cdot|_F$.\footnote{W. H. Schikhof, {\em Ultrametric calculus: An introduction to $p$-adic analysis}, p.~34, Theorem 14.1.}
Furthermore, if $F$ is complete and $K$ is algebraic over $F$ then this absolute value on $K$ is
unique.\footnote{W. H. Schikhof, {\em Ultrametric calculus: An introduction to $p$-adic analysis}, p.~35, Theorem 14.2.}

If $(F,|\cdot|_F)$ is a complete nonarchimedean valued field and $K$ is an algebraic closure of $F$, then by the above
there is a unique nonarchimedean absolute value $|\cdot|_K$ on $K$ such that $|x|_K=|x|_F$ for $x \in F$.
If $(K,|\cdot|_K)$ is not complete,
then we have stated earlier that it has a completion $(L,|\cdot|_L)$. 
It can in fact be proved that the field $L$ is algebraically closed.\footnote{W. H. Schikhof, {\em Ultrametric calculus: An introduction to $p$-adic analysis}, p.~45, Theorem 17.1.}


Let $\mathbb{Q}_p^a$ be an algebraic closure of $\mathbb{Q}_p$. Then there is a unique nonarchimedean absolute value on $\mathbb{Q}_p^a$ whose
restriction to $\mathbb{Q}_p$ is equal to $|\cdot|_p$. 
One proves that the valued field $\mathbb{Q}_p^a$ is not
complete.\footnote{W. H. Schikhof, {\em Ultrametric calculus: An introduction to $p$-adic analysis}, p.~43, Theorem 16.6.}
Let $\mathbb{C}_p$ be the completion of the valued field $\mathbb{Q}_p^a$, which by what we have said is
an algebraically closed nonarchimedean valued field. $\mathbb{Q}_p \subset \mathbb{C}_p$,
and  $|x|_{\mathbb{C}_p} = |x|_p$ for $x \in \mathbb{Q}_p$. 
One further proves that the residue class field of $\mathbb{C}_p$ is the algebraic closure of $\mathbb{F}_p$, and hence
$\mathbb{C}_p$ is not locally compact.  The value group of $\mathbb{C}_p$ is $\{p^r: r \in \mathbb{Q}\}$. Finally, $\mathbb{C}_p$ is a separable metric
space.

It turns out to be fruitful to work with functions $\mathbb{Z}_p \to \mathbb{C}_p$, and because $\mathbb{C}_p$ is an occult object it is useful to 
become familiar with it before working out this machinery.

\end{document}