\documentclass{article}
\usepackage{amsmath,amssymb,graphicx,subfig,mathrsfs,amsthm,siunitx}
%\usepackage{tikz-cd}
%\usepackage{hyperref}
\newcommand{\inner}[2]{\left\langle #1, #2 \right\rangle}
\newcommand{\tr}{\ensuremath\mathrm{tr}\,} 
\newcommand{\Span}{\ensuremath\mathrm{span}} 
\def\Re{\ensuremath{\mathrm{Re}}\,}
\def\Im{\ensuremath{\mathrm{Im}}\,}
\newcommand{\id}{\ensuremath\mathrm{id}} 
\newcommand{\rank}{\ensuremath\mathrm{rank\,}} 
\newcommand{\co}{\ensuremath\mathrm{co}\,} 
\newcommand{\cco}{\ensuremath\overline{\mathrm{co}}\,}
\newcommand{\supp}{\ensuremath\mathrm{supp}\,}
\newcommand{\ext}{\ensuremath\mathrm{ext}\,}
\newcommand{\cl}{\ensuremath\mathrm{cl}\,}
\newcommand{\dom}{\ensuremath\mathrm{dom}\,}
\newcommand{\Cyl}{\ensuremath\mathrm{Cyl}\,}
\newcommand{\extreals}{\overline{\mathbb{R}}}
\newcommand{\upto}{\nearrow}
\newcommand{\downto}{\searrow}
\newcommand{\norm}[1]{\left\Vert #1 \right\Vert}
\newtheorem{theorem}{Theorem}
\newtheorem{lemma}[theorem]{Lemma}
\newtheorem{proposition}[theorem]{Proposition}
\newtheorem{corollary}[theorem]{Corollary}
\theoremstyle{definition}
\newtheorem{definition}[theorem]{Definition}
\newtheorem{example}[theorem]{Example}
\begin{document}
\title{The Kolmogorov extension theorem}
\author{Jordan Bell}
\date{June 21, 2014}

\maketitle

\section{$\sigma$-algebras and semirings}
If $X$ is a nonempty set, an \textbf{algebra} of sets on $X$ is a  collection $\mathscr{A}$ of subsets of
$X$ such that if $\{A_i\} \subset \mathscr{A}$ is finite then $\bigcup_i A_i \in \mathscr{A}$,
and if $A \in \mathscr{A}$ then $X \setminus A \in \mathscr{A}$. An algebra
$\mathscr{A}$ is called a \textbf{$\sigma$-algebra} if $\{A_i\} \subset \mathscr{A}$ being countable
implies that $\bigcup_i A_i \in \mathscr{A}$.

If $X$ is a set and  $\mathscr{G}$ is a collection of subsets of $X$,
we denote by $\sigma(\mathscr{G})$ the smallest $\sigma$-algebra containing $\mathscr{G}$, and we say that $\sigma(\mathscr{G})$ is the
\textbf{$\sigma$-algebra generated by $\mathscr{G}$}. 

Later  we will also use the following notion. If $X$ is a nonempty set, a \textbf{semiring} of sets on $X$ is a collection $\mathscr{S}$ of subsets of $X$
such that (i) $\emptyset \in \mathscr{S}$, (ii) if $A,B \in \mathscr{S}$ then $A \cap B \in \mathscr{S}$, and (iii) if $A,B \in \mathscr{S}$ then
there are pairwise disjoint $S_1,\ldots,S_n \in \mathscr{S}$ such that $A \setminus B = \bigcup_{i=1}^n S_i$; we do not demand that this union
itself belong to $\mathscr{S}$. We remark that a semiring on $X$ need not include $X$.

If $\mathscr{S}$ is a semiring of sets and $\mu_0:\mathscr{S} \to [0,\infty]$, we say that $\mu_0$ is
\textbf{finitely additive} if $\{S_i\} \subset \mathscr{S}$ being finite, pairwise disjoint, and satisfying $\bigcup_i S_i \in \mathscr{S}$ 
implies that $\mu_0\left(\bigcup_i S_i \right) = \sum_i \mu_0(S_i)$, and
\textbf{countably additive} if $\{S_i\} \subset \mathscr{S}$ being countable, pairwise disjoint, and satisfying $\bigcup_i S_i \in \mathscr{S}$ 
implies that $\mu_0\left(\bigcup_i S_i \right) = \sum_i \mu_0(S_i)$. 
If $\mathscr{G}$ is a collection of subsets of $X$, 
 the \textbf{algebra generated by $\mathscr{G}$} is the smallest algebra containing $\mathscr{G}$.
We shall use the following lemma in the proof of Lemma \ref{neveu}.

\begin{lemma}
\label{semiringgen}
If $\mathscr{S}$ is a semiring on a set $X$ and $X \in \mathscr{S}$, then the algebra generated by
$\mathscr{S}$ is equal to the collection of finite unions of members of $\mathscr{S}$. 
\end{lemma}

For a bounded countably additive function, the \textbf{Carath\'eodory extension theorem} states the following.\footnote{Ren\'e L. Schilling,
{\em Measures, Integrals and Martingales}, p.~37, Theorem 6.1. If we had not specified that $\mu_0:\mathscr{S} \to [0,1]$ but rather had
talked about $\mu_0:\mathscr{S} \to [0,\infty]$, then the Carath\'eodory extension theorem shows that there is some extension of
$\mu_0$ to $\sigma(\mathscr{S})$, but  this extension need not be unique.}

\begin{theorem}[Carath\'eodory extension theorem]
Suppose that $X$ is a nonempty set, that $\mathscr{S}$ is a semiring on $X$, and that $\mu_0:\mathscr{S} \to [0,1]$ is countably
additive. Then there is one and only one measure on $\sigma(\mathscr{S})$ whose restriction to $\mathscr{S}$ is equal to $\mu_0$.
\end{theorem}



\section{Product $\sigma$-algebras}
\label{productsection}
Suppose that $X$ is a set, that
$\{(Y_i,\mathscr{M}_i): i \in I\}$ is a family of measurable spaces,  and that $f_i:X \to Y_i$ are functions.
The smallest $\sigma$-algebra on $X$ such that each $f_i$ is measurable is called the \textbf{$\sigma$-algebra generated by
$\{f_i:i \in I\}$}.
This is analogous to the initial topology induced by a family of functions on a
set.
Calling this $\sigma$-algebra $\mathscr{M}$ and
 supposing that 
 $\sigma(\mathscr{G}_i)=\mathscr{M}_i$ for each $i \in I$, we check then that
\begin{equation}
\mathscr{M} = \sigma\left( \{f_i^{-1}(A): i \in I, A \in \mathscr{G}_i\} \right).
\label{initialgen}
\end{equation}

Suppose that $\{(X_i,\mathscr{M}_i): i \in I\}$  is a family of measurable spaces.  Let
\[
X=\prod_{i \in I} X_i,
\]
 the cartesian product of the sets $X_i$, and let $\pi_i:X \to X_i$ be the projection maps. The \textbf{product $\sigma$-algebra} on
$X$ is the $\sigma$-algebra $\mathscr{M}$ generated by $\{\pi_i: i \in I\}$, and is denoted
\[
\mathscr{M}=\bigotimes_{i \in I} \mathscr{M}_i.
\]
This is analogous to the product topology on a cartesian product of topological spaces, which has the initial topology induced by the 
family of projection maps. 


For $H \subset I$, we define
\[
X_H = \prod_{i \in H} X_i.
\]
Thus, $X_I = X$ and $X_{\emptyset}=\{\emptyset\}$, and for $G \subset H$,
\[
X_H = X_G \times X_{H \setminus G}.
\]
For $H \subset I$, let
\[
\mathscr{M}_H=\bigotimes_{i \in H} \mathscr{M}_i,
\]
 the product $\sigma$-algebra on $X_H$.  Thus,  $\mathscr{M}_I=\mathscr{M}$ and
 $\mathscr{M}_{\emptyset}=\{\emptyset,\{\emptyset\}\}$, and for $G \subset H$ we have
 \[
 \mathscr{M}_H = \mathscr{M}_G \otimes \mathscr{M}_{H \setminus G}.
 \]
 
 
 For $G \subset H$, we define $P_{H,G}:X_H \to X_G$ to be the projection map: an element of $X_H$ is a function $x$ on $H$ such that $x(i) \in X_i$
for all $i \in H$,
and $P_{H,G}(x)$ is the restriction of $x$ to $G$.
 
 \begin{lemma}
For $G \subset H$,  $P_{H,G}:(X_H,\mathscr{M}_H) \to 
 (X_G,\mathscr{M}_G)$ is measurable. 
 \label{projmeasurable}
 \end{lemma}
 

If $F$ is a finite subset of $I$ and $A \in \mathscr{M}_F$, we call $A \times X_{I \setminus F} \in
\mathscr{M}$
an  \textbf{$F$-cylinder set}. Cylinder sets for the product $\sigma$-algebra are analogous to the usual basic open sets for the product topology.

\begin{lemma}
The collection of all cylinder sets 
is an algebra of sets on $\prod_{i \in I} X_i$,
and  this collection generates the product $\sigma$-algebra $\bigotimes_{I \in I} \mathscr{M}_i$.
\label{cylinderlemma}
\end{lemma}

 The product $\sigma$-algebra
can in fact be generated by a smaller collection of sets. (The following collection of sets is not a minimal collection of sets that generates the product $\sigma$-algebra,
but it is smaller than the collection of all cylinder sets and it has the structure of a semiring, which will turn out to be useful.)
An intersection of finitely many sets of the form $A \times \mathscr{M}_{I \setminus \{t\}}$, $A \in \mathscr{M}_t$, is called a \textbf{product cylinder}.

\begin{lemma}
The collection of all  product cylinders is a semiring of sets on $\prod_{i \in I} X_i$, and this collection
generates the product $\sigma$-algebra  $\bigotimes_{i \in I} \mathscr{M}_i$.
\label{specialcylinder}
\end{lemma}



\section{Borel $\sigma$-algebras}
If $(X,\tau)$ is a topogical space, the \textbf{Borel $\sigma$-algebra} on $X$ is $\sigma(\tau)$, and is denoted $\mathscr{B}_X$. A member of $\mathscr{B}_X$ is called a \textbf{Borel
set}. 

\begin{lemma}
If $X$ is a topological space and $\mathscr{G}$ is a countable subbasis for the topology of $X$, then 
\[
\mathscr{B}_X = \sigma(\mathscr{G}).
\]
\label{subbasis}
\end{lemma}


A separable metrizable space is second-countable, so we can apply the following theorem to such spaces.

\begin{theorem}
Suppose that $X_i$, $i \in \mathbb{N}$, are second-countable topological spaces and let $X=\prod_{i \in \mathbb{N}} X_i$, with the product topology. Then
\[
\mathscr{B}_X = \bigotimes_{i \in \mathbb{N}} \mathscr{B}_{X_i}.
\]
\label{secondcountable}
\end{theorem}
\begin{proof}
For each $i \in \mathbb{N}$, let $\mathscr{G}_i$ be a countable subbasis for the topology of
$X_i$.
Because $\mathscr{G}_i$ is a subbasis for the topology of $X_i$ for each $i$, we check that
$\mathscr{G}$ is a subbasis for the product topology of $X$, where
\[
\mathscr{G} = \{\pi_i^{-1}(A): i \in \mathbb{N}, A \in \mathscr{G}_i\}.
\]
Because each $\mathscr{G}_i$ is countable and $\mathbb{N}$ is countable, $\mathscr{G}$ is countable. Hence by Lemma \ref{subbasis},
\[
\mathscr{B}_X = \sigma(\mathscr{G}).
\]
On the other hand,
for each $i \in \mathbb{N}$ we have by  Lemma \ref{subbasis} that $\mathscr{B}_{X_i}=\sigma(\mathscr{G}_i)$, and 
so by \eqref{initialgen},
\[
\bigotimes_{i \in \mathbb{N}} \mathscr{B}_{X_i}=\sigma(\mathscr{G}).
\]
\end{proof}




\section{Product measures}
If $\{(X_i,\mathscr{M}_i,\mu_i): 1 \leq i \leq n\}$ are $\sigma$-finite measure spaces, let $X=\prod_{i=1}^n X_i$ and $\mathscr{M}=\bigotimes_{i=1}^n
\mathscr{M}_i$. 
It is a fact that there is a unique  measure $\mu$ on $\mathscr{M}$ such that for $A_i \in \mathscr{M}_i$,
\[
\mu\left( \prod_{i=1}^n A_i \right) = \prod_{i=1}^n \mu_i(A_i),
\]
and $\mu$ is a $\sigma$-finite measure. We write $\mu=\prod_{i=1}^n \mu_i$ and call $\mu$ the \textbf{product measure}.\footnote{Gerald
B. Folland, {\em Real Analysis: Modern Techniques and Their Applications}, second ed., pp.~64--65, and p.~31, Theorem 1.14.}



\section{Compact classes}
If $X$ is a set and $\mathscr{C}$ is a collection of subsets of $X$, we say that $\mathscr{C}$ is a \textbf{compact class}
if every countable subset of $\mathscr{C}$ with the finite intersection property has nonempty intersection. We remind ourselves
that a collection $\mathscr{E}$ of sets is said to have the \textbf{finite intersection property} if for any finite subset $\mathscr{F}$ of $\mathscr{E}$ we have
$\bigcap_{A \in \mathscr{F}} A \neq \emptyset$. Usually one speaks about a collection of sets having the finite intersection property in the following setting:
A topological space $Y$ is compact if and only if every collection of closed sets that has the finite intersection property
has nonempty intersection.

We will employ the following lemma in the proof of the Kolmogorov extension theorem.\footnote{V. I. Bogachev,
{\em Measure Theory}, volume I, p.~50,
Proposition 1.12.4.} 

\begin{lemma}
Suppose that $\mathscr{C}^0$ is a compact class of subsets of a set $X$ and let
$\mathscr{C}$ be the collection of countable intersections of finite unions of members of $\mathscr{C}^0$. $\mathscr{C}$ is the smallest
 collection of subsets of $X$ containing $\mathscr{C}^0$ that is closed
under finite unions and countable intersections, and $\mathscr{C}$ is itself  a compact class.
\label{countablegen}
\end{lemma}


We state the following result that gives conditions under which a finitely additive functions on an algebra of sets is in fact countably additive,\footnote{Charalambos D. 
Aliprantis and Kim C. Border, {\em Infinite Dimensional Analysis: A Hitchhiker's Guide}, third ed., p.~378,
Theorem 10.13.} and then use it to prove an analogous result for semirings. 

\begin{lemma}
Suppose that $\mathscr{A}$ is an algebra of sets on a set $X$, and that $\mu_0:\mathscr{A} \to [0,\infty)$ is finitely additive and
$\mu_0(X)<\infty$.
If there is a compact class $\mathscr{C} \subset \mathscr{A}$ such that
\[
\mu_0(A) = \sup\{\mu_0(C): \textrm{$C \in \mathscr{C}$ and $C \subset A$}\}, \qquad A \in \mathscr{A},
\]
then $\mu_0$ is countably additive.
\label{algebraneveu}
\end{lemma}


The following lemma gives conditions under which a finitely additive function on a semiring of sets is in fact countably additive.\footnote{Charalambos D. 
Aliprantis and Kim C. Border, {\em Infinite Dimensional Analysis: A Hitchhiker's Guide}, third ed., p.~521,
Lemma 15.25.} 


\begin{lemma}
Suppose that $\mathscr{S}$ is a semiring of sets on $X$ with $X \in \mathscr{S}$ and
that $\mu_0:\mathscr{S} \to [0,\infty)$
is finitely additive and $\mu_0(X)<\infty$. If there is a compact class $\mathscr{C} \subset \mathscr{S}$ such that
\[
\mu_0(A) = \sup\{\mu_0(C) : \textrm{$C \in \mathscr{C}$ and $C \subset A$}\}, \qquad A \in \mathscr{S},
\]
then $\mu_0$ is a countably additive.
\label{neveu}
\end{lemma}
\begin{proof}
Let $\mathscr{C}_u$ be the collection of finite unions of members of $\mathscr{C}$. $\mathscr{C}_u$ is a subset of the compact class
produced in Lemma \ref{countablegen}, hence is itself a compact class. Let $\mathscr{A}$ be the collection of finite unions of members of $\mathscr{S}$, which
 by Lemma \ref{semiringgen} is the algebra generated by $\mathscr{S}$. 
Because $\mathscr{C} \subset \mathscr{S} \subset \mathscr{A}$ and $\mathscr{A}$ is closed under
finite unions, it follows that $\mathscr{C}_u \subset \mathscr{A}$. 

Because $\mathscr{S}$ is a semiring, it is a fact that if $A_1,\ldots,A_n,A \in \mathscr{S}$, then there are pairwise disjoint $S_1,\ldots,S_m \in \mathscr{S}$
such that $A \setminus \bigcup_{i=1}^n A_i = \bigcup_{i=1}^m S_i$.\footnote{Charalambos D. 
Aliprantis and Kim C. Border, {\em Infinite Dimensional Analysis: A Hitchhiker's Guide}, third ed., p.~134,
Lemma 4.7.} Thus, if $A_1,\ldots,A_n \in \mathscr{S}$, defining $E_i = A_i \setminus \bigcup_{j=1}^{i-1} A_j$, with $E_1=A_1 \setminus \emptyset =A_1$,
 the sets $E_1,\ldots,E_n$ are pairwise disjoint, and for each $i$ there are pairwise disjoint $S_{i,1},\ldots,S_{i,a_i} \in \mathscr{S}$ 
such that $E_i = \bigcup_{j=1}^{a_i} S_{i,j}$. Then the sets $S_{i,j}$, $1 \leq i \leq n$, $1 \leq j \leq a_i$ are pairwise disjoint and their
union is equal to $\bigcup_{i=1}^n A_i$. This shows that any element of $\mathscr{A}$ can be written as a union of pairwise disjoint elements of $\mathscr{S}$.


Furthermore, because $\mathscr{S}$ is a semiring, if $A_1,\ldots,A_N \in \mathscr{S}$, there are pairwise disjoint
$S_1,\ldots,S_k \in \mathscr{S}$ such that for each $1 \leq i \leq k$ there is some $1 \leq n \leq N$ such that
$S_i \in A_n$, and for each $1 \leq n \leq N$, there is a subset $F \subset \{1,\ldots,k\}$ such that
$A_n = \bigcup_{i \in F} S_i$.\footnote{Charalambos D. 
Aliprantis and Kim C. Border, {\em Infinite Dimensional Analysis: A Hitchhiker's Guide}, third ed., p.~134,
Lemma 4.8.}

Let  $E \in \mathscr{A}$ and suppose that
$E=\bigcup_{n=1}^N A_n$, where $A_1,\ldots,A_N \in \mathscr{S}$ are pairwise disjoint,
and that $E=\bigcup_{m=1}^M B_m$, where $B_1,\ldots,B_M \in \mathscr{S}$ are pairwise disjoint. 
There are pairwise disjoint
$S_1,\ldots,S_k \in \mathscr{S}$ such that for each $1 \leq i \leq k$ there is some $1 \leq n \leq N$ or  $1 \leq m \leq M$
such that, respectively, $S_i \in A_n$ or $S_i \in B_m$, and for each $1 \leq n \leq N$ there is some subset $F \subset \{1,\ldots,k\}$
such that $A_n = \bigcup_{i \in F} S_i$, and for each $1 \leq m \leq M$ there is some subset $F \subset \{1,\ldots,k\}$
such that $B_m = \bigcup_{i \in F} S_i$. 
It follows that $E=\bigcup_{i=1}^k S_i$, and because $\mu_0$ is finitely additive,
\[
\sum_{n=1}^N \mu_0(A_n) = \sum_{i=1}^k \mu_0(S_i) = \sum_{m=1}^M \mu_0(B_m).
\]
Therefore, for $E \in \mathscr{A}$ it makes sense to define 
\[
\mu(E) = \sum_{n=1}^n \mu_0(A_i),
\]
where $A_1,\ldots,A_n$ are pairwise disjoint elements of $\mathscr{S}$ whose union is equal to $E$.
Also, $\mu(X)=\mu_0(X)<\infty$.

We shall now show that the function $\mu:\mathscr{A} \to [0,\infty)$ is finitely additive.
If $E_1,\ldots,E_N \in \mathscr{A}$ are pairwise disjoint, for each $n$ there are pairwise disjoint
$A_{n,1},\ldots,A_{n,a_n} \in \mathscr{S}$ such that $E_n = \bigcup_{j=1}^{a_n} A_{n,j}$, 
and there are pairwise disjoint $S_1,\ldots,S_k \in \mathscr{S}$ such that for each
$1 \leq i \leq k$, there is some $1 \leq n \leq N$ and some $1 \leq j \leq a_n$ such that
$S_i \in A_{n,j}$, and for each $1 \leq n \leq N$ and each $1 \leq j \leq a_n$ there is some subset $F \subset
\{1,\ldots,k\}$ such that $A_{n,j} = \bigcup_{i \in F} S_i$. It follows that
$\bigcup_{n=1}^N E_n = \bigcup_{i=1}^k S_i$, and
\[
\mu\left(\bigcup_{n=1}^N E_n \right) = \mu\left( \bigcup_{i=1}^k S_i \right) 
=\sum_{i=1}^k \mu_0(S_i) = \sum_{n=1}^N \sum_{j=1}^{a_n} \mu_0(A_{n,j})
=\sum_{n=1}^N \mu(E_n),
\]
showing that $\mu$ is finitely additive. 

For $E=\bigcup_{i=1}^n A_i \in \mathscr{A}$
with pairwise disjoint $A_1,\ldots,A_n \in \mathscr{S}$, let $\epsilon>0$, and for each $1 \leq i \leq n$ let
$C_i \in \mathscr{C}$ with
$\mu_0(C_i)>\mu_0(A_i)+\frac{\epsilon}{n}$ and $C_i \subset A_i$. Then $C=\bigcup_{i=1}^n C_i \in \mathscr{C}_u$. 
As $A_1,\ldots,A_n$ are pairwise disjoint and $C_i \subset A_i$, $C_1,\ldots,C_n$ are pairwise disjoint, so
because $\mu$ is finitely additive on $\mathscr{A}$,
\[
\mu(C) = \sum_{i=1}^n \mu(C_i) = \sum_{i=1}^n \mu_0(C_i) > \sum_{i=1}^n \left(\mu_0(A_i)+\frac{\epsilon}{n}\right)
=\mu(E)+\epsilon.
\]
 Lemma \ref{algebraneveu} tells us now that $\mu:\mathscr{A} \to [0,\infty)$ is countably additive, and therefore $\mu_0$, its restriction to
 the semiring $\mathscr{S}$, is countably additive.
\end{proof}




\section{Kolmogorov consistent families}
Suppose that $\{(X_i,\mathscr{M}_i): i \in I\}$ is a family of measurable spaces. 
The collection $D$ of all finite subsets of $I$, ordered by set
inclusion, is a directed set.
Suppose that for each $F \in D$,
$\mu_F$ is a probability measure on $\mathscr{M}_F$; we defined the notation $\mathscr{M}_F$ in \S \ref{productsection} and we use that here.
 We say that the family of measures $\{\mu_F: F \in D\}$ is \textbf{Kolmogorov consistent} if whenever $F,G \in D$
with $F \subset G$, it happens that ${P_{G,F}}_* \mu_G = \mu_F$, where $f_* \mu$ denotes the \textbf{pushforward of
$\mu$ by $f$}, i.e. $f_* \mu = \mu \circ f^{-1}$. It makes sense to talk about ${P_{G,F}}_* \mu_G$ because
$P_{G,F}:(X_G,\mathscr{M}_G) \to (X_F,\mathscr{M}_F)$ is measurable, as stated in Lemma \ref{projmeasurable}.

We are now prepared to prove the \textbf{Kolmogorov extension theorem}.\footnote{Charalambos D. Aliprantis and Kim C. Border, {\em Infinite Dimensional Analysis: A Hitchhiker's Guide}, third ed., p.~522,
Theorem 15.26.}

\begin{theorem}[Kolmogorov extension theorem]
Suppose that $\{(X_i,\mathscr{M}_i): i \in I\}$ is a family of measurable spaces
and suppose that for each $F \in D$, $\mu_F$ is a probability measure
on $\mathscr{M}_F$. If the family of probability measures $\{\mu_F: F \in D\}$
is Kolmogorov consistent and if for each $i \in I$ there is a compact class $\mathscr{C}_i \subset
\mathscr{M}_i$ satisfying
\begin{equation}
\mu_i (A) = \sup\{\mu_i(C) : \textrm{$C \in \mathscr{C}_i$ and $C \subset A$}\}, \qquad A \in \mathscr{M}_i,
\label{compacthypothesis}
\end{equation}
then there is a unique probability measure on $\mathscr{M}_I$ such that for
each $F \in D$, the pushforward of $\mu$ by the projection map $P_{I,F}:X_I \to X_F$ is equal to $\mu_F$. 
\label{kolmogorovtheorem}
\end{theorem}
\begin{proof}
Define
\[
\mathscr{C}^0 = \{C \times X_{I \setminus \{i\}}: i \in I, C \in \mathscr{C}_i\}.
\]
We shall show that $\mathscr{C}^0$ is a compact class. 
Suppose 
\[
 \{C_{n} \times X_{I \setminus \{i_n\}}: n \in \mathbb{N}, C_n \in \mathscr{C}_{i_n} \} \subset \mathscr{C}^0
\]
has empty intersection.
For each $i \in I$, let
\[
Q_i = \bigcap_{i_n = i} C_n,
\]
and if there are no such $i_n$, then $Q_i = X_i$. Then
\[
\bigcap_{n \in \mathbb{N}} C_n  \times X_{I \setminus \{i_n\}} = \prod_{i \in I} Q_i.
\]
Because this intersection is equal to $\emptyset$,  one of the factors in the product is equal to $\emptyset$. (For some purposes one wants
to keep track of where the axiom of choice is used, so we mention that concluding that some factor of any empty cartesian product is itself
empty is equivalent to  the axiom of choice). No $X_i$ is empty, so this empty $Q_i$  must be of the form
$\bigcap_{i_n = i} C_n$ for which at least one $i_n$ is equal to $i$. But if $i_n = i$ then $C_n \in \mathscr{C}_i$, and because $\mathscr{C}_i$ is a compact class,
$\bigcap_{i_n =i}  C_n = \emptyset$ implies that there are finitely many $a_1,\ldots,a_N$ such that
$\bigcap_{n=1}^N C_{a_n} = \emptyset$, and this yields $\bigcap_{n=1}^N C_{a_n} \times X_{I \setminus \{i_{a_n}\}} = \emptyset$. 
We have thus proved that if an intersection of countably many members of $\mathscr{C}^0$ is empty then some intersection
of finitely many of these is empty, showing that $\mathscr{C}^0$ is a compact class.
Let $\mathscr{C}^1$ be the smallest collection of subsets of $X_I$ containing $\mathscr{C^0}$ that is closed under finite unions and countable
intersections, and by Lemma \ref{countablegen} we know that $\mathscr{C}^1$ is a compact class; we use the notation $\mathscr{C}^1$ because presently
we will use a subset of $\mathscr{C}^1$.

Let $\mathscr{A}$ be the collection of all cylinder sets of the product $\sigma$-algebra $\mathscr{M}_I$. Explicitly,
\[
\mathscr{A} = \{A \times X_{I \setminus F}: F \in D, A \in \mathscr{M}_F\}.
\]
Suppose $F,G \in D$, $F \subset G$, $A \in \mathscr{M}_F, B \in \mathscr{M}_G$, and that
$A \times X_{I \setminus F} = B \times X_{I \setminus G}$. 
It follows that $B = A \times X_{G \setminus F}$, and then
using ${P_{G,F}}_* \mu_G = \mu_F$ we get
\[
\mu_G(B)=\mu_G(A \times X_{G \setminus F}) = \mu_G(P_{G,F}^{-1}(A)) = \mu_F(A).
\]
Therefore it makes sense to define
$\mu_0:\mathscr{A} \to [0,1]$ as follows: for $A \times X_{I \setminus F} \in \mathscr{A}$,
\[
\mu_0(A \times X_{I \setminus F}) = \mu_F(A).
\]
Let  $F_1,\ldots,F_n \in D$ and $A_1 \in \mathscr{M}_{F_1}, \ldots,A_n \in \mathscr{M}_{F_n}$, and suppose that 
$A_1 \times X_{I \setminus F_1},\ldots,A_n \times X_{I \setminus F_n} \in \mathscr{G}$ are pairwise disjoint. With $F=\bigcup_{j=1}^n F_j \in D$, 
\[
\bigcup_{j=1}^n A_j \times X_{I \setminus F_j} = \left(\bigcup_{j=1}^n A_j \times X_{F \setminus F_j} \right) \times X_{I \setminus F},
\]
which is an $F$-cylinder set. Then,
\begin{eqnarray*}
\mu_0\left(\bigcup_{j=1}^n A_j \times X_{I \setminus F_j}\right) &=& \mu_F  \left(\bigcup_{j=1}^n A_j \times X_{F \setminus F_j} \right)\\
&=&\sum_{j=1}^n \mu_F (A_j \times X_{F \setminus F_j})\\
&=&\sum_{j=1}^n \mu_0 (A_j \times X_{I \setminus F_j}),
\end{eqnarray*}
showing that $\mu_0:\mathscr{A} \to [0,1]$ is finitely additive. 

Let $\mathscr{G}$ be the collection of all product cylinder sets of the product $\sigma$-algebra $\mathscr{M}_I$. Explicitly,
\[
\mathscr{G} = \left\{\bigcap_{i \in F} A_i \times X_{I \setminus \{i\}} : \textrm{$F \in D$, and $A_i \in \mathscr{M}_i$ for 
$i \in F$}\right\}.
\]
It is apparent that $\mathscr{C}^0 \subset \mathscr{G}$. Let $\mathscr{C}$ be the intersection of $\mathscr{C}^1$ and $\mathscr{G}$. A subset of a compact
class is a compact class, so $\mathscr{C}$ is a compact class, and
$\mathscr{C}^0 \subset \mathscr{C}$. 

Suppose that $E \in \mathscr{G}$: there is some $F \in D$ and $A_i \in \mathscr{M}_i$ for each $i \in F$ such that
\[
E = \bigcap_{i \in F} A_i \times X_{I \setminus \{i\}} = \left( \prod_{i \in F} A_i \right) \times X_{I \setminus F}.
\]
Take $n=|F|$, and
let $\epsilon>0$. Then, for each $i \in F$, by \eqref{compacthypothesis} there is some $C_i \in \mathscr{C}_i$  such that
$C_i \subset A_i$ and $\mu_i(A_i)<\mu_i(C_i)+\frac{\epsilon}{n}$, and we set
\[
C = \bigcap_{i \in F} C_i \times X_{I \setminus \{i\}}= \left( \prod_{i \in F} C_i \right) \times X_{I \setminus F},
\] 
which is a finite intersection of members of $\mathscr{C}^0$ and hence belongs to $\mathscr{C}^1$, and which visibly belongs to $\mathscr{G}$, and hence
belongs to $\mathscr{C}$. 
We have
\begin{eqnarray*}
E \setminus C &=&\left( \bigcup_{i \in F} (A_i \setminus C_i) \times \prod_{j \in F \setminus \{i\}}
A_j \right) \times X_{I \setminus F}\\
&\subset&\bigcup_{i \in F} (A_i \setminus C_i ) \times X_{I \setminus \{i\}}.
\end{eqnarray*}
Both $E \setminus C$ and the above union are cylinder sets so it makes sense to apply $\mu_0$ to them, and 
because $\mu_0$ is finitely additive,
\begin{eqnarray*}
\mu_0(E \setminus C)&\leq&\sum_{i \in F} \mu_0((A_i \setminus C_i ) \times X_{I \setminus \{i\}})\\
&=&\sum_{i \in F} \mu_i(A_i \setminus C_i)\\
&=&\sum_{i \in F} \mu_i(A_i)  - \mu_i(C_i)\\
&<&\sum_{i \in F} \frac{\epsilon}{n}\\
&=&\epsilon.
\end{eqnarray*}
Hence $\mu_0(E) - \mu_0(C) = \mu_0(E \setminus C) < \epsilon$, i.e. 
$\mu_0(E) < \mu_0(C) + \epsilon$. Thus, we have proved that for each $\epsilon>0$, there is some
$C \in \mathscr{C}$  such that $C \subset E$ and $\mu_0(E)<\mu_0(C)+\epsilon$, which means that
\[
\mu_0(E) = \sup\{\mu_0(C): \textrm{$C \in \mathscr{C}$ and $C \subset E$}\}.
\]
Using
the restriction of $\mu_0:\mathscr{A} \to [0,1]$ to the semiring $\mathscr{G}$ and the compact class $\mathscr{C}$, the conditions of
Lemma \ref{neveu} are satisfied, and therefore the restriction of $\mu_0$ to $\mathscr{G}$ is countably additive. 

By Lemma \ref{specialcylinder},
$\mathscr{M}_I = \sigma(\mathscr{G})$.
Because the restriction of $\mu_0$ to the semiring $\mathscr{G}$ is countably additive, we can
apply the Carath\'eodory extension theorem, which tells us that there is a unique measure $\mu$
on $\sigma(\mathscr{G})=\mathscr{M}_I$ whose restriction to $\mathscr{G}$ is equal to the restriction of $\mu_0$ to $\mathscr{G}$. Check that the restriction
of $\mu$ to $\mathscr{A}$ is equal to $\mu_0$. For $F \in D$ and $A \in \mathscr{M}_F$, 
\[
{P_{I,F}}_* \mu (A) = \mu(P_{I,F}^{-1}(A)) = \mu(A \times X_{I \setminus F}) = \mu_0(A \times X_{I \setminus F}) = \mu_F(A),
\]
showing that ${P_{I,F}}_* \mu = \mu_F$. Certainly $\mu(X_I)=1$, namely, $\mu$ is a probability measure. 
If $\nu$ is a probability measure on $X_I$ whose pushforward by $P_{I,F}$ is equal to $\mu_F$ for each $F \in D$, then
check that the restriction of $\nu$ to $\mathscr{G}$ is equal to the restriction of $\mu_0$ to $\mathscr{G}$, and then
by the assertion of uniqueness in Carath\'eodory's theorem, $\nu=\mu$, completing the proof.
\end{proof}

If $X$ is a Hausdorff space, we say that a Borel measure $\mu$ on $X$ is \textbf{tight} if for every $A \in \mathscr{B}_X$,
\[
\mu(A) = \sup\{\mu(K): \textrm{$K$ is compact and $K \subset A$}\}.
\]
A \textbf{Polish space} is a topological space that is homeomorphic to a complete separable metric space, and 
it is a fact that a finite Borel measure on a Polish space is tight.\footnote{Charalambos D. 
Aliprantis and Kim C. Border, {\em Infinite Dimensional Analysis: A Hitchhiker's Guide}, third ed., p.~438,
Theorem 12.7.} In particular, any Borel probability measure on a Polish space is tight. We use this  in the proof
of the following version of the Kolmogorov extension theorem, which  applies for instance to the case
where $X_i=\mathbb{R}$ for each $i \in I$, with $I$ any index set. 


\begin{corollary}
Suppose that $\{X_i: i \in I\}$ is a family of Polish spaces and suppose that for each $F \in D$, $\mu_F$ is a Borel probability
measure on $X_F$. If the family of measures $\{\mu_F: F \in D\}$ is Kolmogorov consistent, then there is a unique probability
measure on $\mathscr{M}_I=\bigotimes_{i \in I} \mathscr{B}_{X_i}$ such that for each $F \in D$, the pushforward of $\mu$ by the projection map $P_{I,F}:X_I \to X_F$ is
equal to $\mu_F$. 
\end{corollary}
\begin{proof}
For each $i \in I$, let $\mathscr{C}_i$ be the collection of all compact subsets of $X_i$. In any topological space, check that
a collection
of compact sets is a compact class. The fact that $\mu_i$ is a Borel probability measure on a Polish
space then implies that it is tight, which we can write as
\[
\mu_i(A) = \sup\{\mu_i(C): \textrm{$K \in \mathscr{C}_i$ and $K \subset A$}\}, \qquad A \in \mathscr{M}_i.
\]
Therefore the conditions of Theorem \ref{kolmogorovtheorem} are satisfied, so the claim follows.
\end{proof}


If the index set $I$ in the above corollary is countable, then by Theorem \ref{secondcountable} the product $\sigma$-algebra $\bigotimes_{i \in I}
\mathscr{B}_{X_i}$ is equal
to the Borel $\sigma$-algebra of the product $\prod_{t \in T} X_i$, so that the probability measure $\mu$ on the product $\sigma$-algebra
is in this case a Borel measure.

\end{document}