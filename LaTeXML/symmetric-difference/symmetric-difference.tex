\documentclass{article}
\usepackage{amsmath,amssymb,mathrsfs,amsthm}
%\usepackage{tikz-cd}
%\usepackage{hyperref}
\newcommand{\inner}[2]{\left\langle #1, #2 \right\rangle}
\newcommand{\tr}{\ensuremath\mathrm{tr}\,} 
\newcommand{\Span}{\ensuremath\mathrm{span}} 
\def\Re{\ensuremath{\mathrm{Re}}\,}
\def\Im{\ensuremath{\mathrm{Im}}\,}
\newcommand{\id}{\ensuremath\mathrm{id}} 
\newcommand{\diam}{\ensuremath\mathrm{diam}} 
\newcommand{\lcm}{\ensuremath\mathrm{lcm}} 
\newcommand{\supp}{\ensuremath\mathrm{supp}\,}
\newcommand{\dom}{\ensuremath\mathrm{dom}\,}
\newcommand{\upto}{\nearrow}
\newcommand{\downto}{\searrow}
\newcommand{\norm}[1]{\left\Vert #1 \right\Vert}
\newtheorem{theorem}{Theorem}
\newtheorem{lemma}[theorem]{Lemma}
\newtheorem{proposition}[theorem]{Proposition}
\newtheorem{corollary}[theorem]{Corollary}
\theoremstyle{definition}
\newtheorem{definition}[theorem]{Definition}
\newtheorem{example}[theorem]{Example}
\begin{document}
\title{The symmetric difference metric}
\author{Jordan Bell}
\date{April 12, 2015}

\maketitle

Let $(\Omega,\Sigma,\mu)$ be a probability space. 
For $A,B \in \Sigma$, define 
\[
d_\mu(A,B) = \mu(A \triangle B).
\]
This is a pseduometric on $\Sigma$: 
\begin{align*}
d_\mu(A,C)&=\mu( A \triangle C)\\
&=\mu((A \triangle B) \triangle (B \triangle C))\\
&\leq \mu((A \triangle B) \cup (B \triangle C))\\
&\leq \mu(A \triangle B) + \mu(B \triangle C)\\
&=d_\mu(A,B)+d_\mu(B,C).
\end{align*}
The relation $A \sim B$ if and only if
$d_\mu(A,B)=0$ is an equivalence relation on $\Sigma$, and
$d_\mu([A],[B])=d_\mu(A,B)$ is a metric on the collection $\Sigma_\mu$ of equivalence classes.
We call $d_\mu$ the \textbf{symmetric difference metric}.

The following theorem shows that
$(\Sigma_\mu,d_\mu)$ is a complete metric space.\footnote{V. I. Bogachev, {\em Measure Theory}, volume I,
p.~54, Theorem 1.12.16.}

\begin{theorem}
If $(\Omega,\Sigma,\mu)$ is a probability space, then 
$(\Sigma_\mu, d_\mu)$ is a complete metric space.
\end{theorem}
\begin{proof}
Suppose that $[B_n]$ is a Cauchy sequence in 
$(\Sigma_\mu, d_\mu)$. As for any Cauchy sequence in a metric space, there is a subsequence $[A_n]$ of
$[B_n]$
such that $d_\mu([A_k],[A_n]) < 2^{-n}$ for $k \geq n$. 
Define
\[
E_n = \bigcup_{k \geq n} A_k.
\]
We have
\begin{align*}
E_n \setminus A_n &=\bigcup_{k=n+1}^\infty (A_k \setminus A_n)\\
&=\bigcup_{k=n+1} \left(A_k \setminus \bigcup_{j=n}^{k-1} A_j \right)\\
&\subset \bigcup_{k=n+1} (A_k \setminus A_{k-1})\\
&= \bigcup_{k=n}^\infty (A_{k+1} \setminus A_k),
\end{align*}
hence
\begin{equation}
\mu(E_n \triangle A_n)= \mu(E_n \setminus A_n)
\leq \sum_{k=n}^\infty \mu(A_{k+1} \setminus A_k)
<\sum_{k=n}^\infty 2^{-k}
=2^{-n+1}.
\label{En}
\end{equation}
Now, define
\[
A = \limsup_{n \to \infty} A_n = \bigcap_{n=1}^\infty \bigcup_{k=n}^\infty A_k =
\bigcap_{n=1}^\infty E_n,
\]
for which
\begin{align*}
\mu(A_n \triangle A)&=\mu(A_n \setminus A)\\
&=\mu\left(A_n \cap \left( \bigcap_{k=1}^\infty E_k \right)^c \right)\\
&=\mu\left( A_n \cap \bigcup_{k=1}^\infty E_k^c \right)\\
&=\mu\left( \bigcup_{k=1}^\infty (A_n \cap E_k^c) \right)\\
&= \lim_{k \to \infty} \mu(A_n \cap E_k^c)\\
&=\lim_{k \to \infty} \mu\left( \bigcap_{j \geq k} (A_n \setminus A_j) \right)\\
&\leq \lim_{k \to \infty} \mu(A_n \setminus A_k)\\
&<2^{-n}.
\end{align*}
Using \eqref{En},
\[
d_\mu(A_n,A) \leq 
\mu(E_n \triangle A_n) + \mu(A_n \triangle A) 
< 2^{-n+1}+2^{-n}=3 \cdot 2^{-n},
\]
showing that $[A_n]$ converges to $[A]$ 
as $n \to \infty$, and because $[A_n]$ is a subsequence of the Cauchy
sequence $[B_n]$, it follows that $[B_n]$ converges to $[A]$ and therefore
that $(\Sigma_\mu,d_\mu)$ is a complete metric space.
\end{proof}

\begin{lemma}
For $A, B \in \Sigma$,
\[
|\mu(A)-\mu(B)| \leq \mu(A \triangle B).
\]
\end{lemma}
\begin{proof}
\begin{align*}
|\mu(A)-\mu(B)|&=|(\mu(A \setminus B) + \mu(A \cap B)) - 
(\mu(B \setminus A) + \mu(B \cap B))|\\
&=|\mu(A \setminus B)-\mu(B \setminus A)|\\
&\leq \mu(A \setminus B)+\mu(B \setminus A)\\
&=\mu((A \setminus B) \cup (B \setminus A))\\
&\leq \mu(A \triangle B).
\end{align*}
\end{proof}


The following theorem connects the metric space $(\Sigma_\mu,d_\mu)$ with the Banach space
$L^1(\mu)$.\footnote{John B. Conway, {\em A Course in Abstract Analysis}, p.~90, Proposition 2.7.13.}

\begin{theorem}
If $(\Sigma_\mu,d_\mu)$ is separable then $L^1(\mu)$ is separable.
\end{theorem}



\end{document}