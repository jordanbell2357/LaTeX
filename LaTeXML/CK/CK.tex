\documentclass{article}
\usepackage{amsmath,amssymb,mathrsfs,amsthm}
%\usepackage{tikz-cd}
%\usepackage{hyperref}
\newcommand{\inner}[2]{\left\langle #1, #2 \right\rangle}
\newcommand{\tr}{\ensuremath\mathrm{tr}\,} 
\newcommand{\Span}{\ensuremath\mathrm{span}} 
\def\Re{\ensuremath{\mathrm{Re}}\,}
\def\Im{\ensuremath{\mathrm{Im}}\,}
\newcommand{\id}{\ensuremath\mathrm{id}} 
\newcommand{\ca}{\ensuremath\mathrm{ca}} 
\newcommand{\var}{\ensuremath\mathrm{var}} 
\newcommand{\Lip}{\ensuremath\mathrm{Lip}} 
\newcommand{\GL}{\ensuremath\mathrm{GL}} 
\newcommand{\diam}{\ensuremath\mathrm{diam}} 
\newcommand{\sgn}{\ensuremath\mathrm{sgn}\,} 
\newcommand{\lcm}{\ensuremath\mathrm{lcm}} 
\newcommand{\supp}{\ensuremath\mathrm{supp}\,}
\newcommand{\dom}{\ensuremath\mathrm{dom}\,}
\newcommand{\upto}{\nearrow}
\newcommand{\downto}{\searrow}
\newcommand{\norm}[1]{\left\Vert #1 \right\Vert}
\newcommand{\HS}[1]{\left\Vert #1 \right\Vert_{\mathrm{HS}}}
\newtheorem{theorem}{Theorem}
\newtheorem{lemma}[theorem]{Lemma}
\newtheorem{proposition}[theorem]{Proposition}
\newtheorem{corollary}[theorem]{Corollary}
\theoremstyle{definition}
\newtheorem{definition}[theorem]{Definition}
\newtheorem{example}[theorem]{Example}
\begin{document}
\title{$C[0,1]$: the Faber-Schauder basis,  the Riesz representation theorem, and the  Borel $\sigma$-algebra}
\author{Jordan Bell}
\date{July 24, 2015}

\maketitle

\section{Introduction}
In this note I work out some results about the space of continuous functions $[0,1] \to \mathbb{R}$. 
In some cases these are instances of general results about continuous functions on compact Hausdorff spaces or compact metrizable
spaces. 


\section{$C(K)$}
Let $K$ be a compact topological space (not necessarily Hausdorff) and let $C(K)$ be the collection of continuous functions $K \to \mathbb{R}$.
With the norm
\[
\norm{f} = \sup_{t \in K} |f(t)|,\qquad f \in C(K),
\]
this is a real Banach algebra, with unity $1:K \to \mathbb{R}$ defined by $1(t)=1$ for all $t \in K$.\footnote{Charalambos D. Aliprantis
and Kim C. Border, {\em Infinite Dimensional Analysis: A Hitchhiker's Guide}, third ed., p.~124, Lemma 3.97.}
Generally, if $X$ is a compact metrizable space and 
$Y$ is a separable metrizable space then $C(X,Y)$ is separable.\footnote{Charalambos D. Aliprantis
and Kim C. Border, {\em Infinite Dimensional Analysis: A Hitchhiker's Guide}, third ed., p.~125, Lemma 3.99.}
Thus, when $K$ is a compact metrizable space, $C(K)$ is a separable unital Banach algebra. 


\section{Schauder bases}
If $X$ is a real normed space, a \textbf{Schauder basis for $X$} is a 
sequence $(h_k)$ in $X$ such that for each $x \in X$ there is a unique sequence of real numbers $(c_k(x))$ such that 
$\sum_{k=1}^n c_k(x) h_k \to x$. A sequence $(h_k)$ in $X$ is called a \textbf{basic sequence} if it is a Schauder basis for the closure of
its linear span.\footnote{Because Schauder bases are not as familiar objects as orthonormal bases (Hilbert space
bases) or vector space
bases (Hamel bases), it is worth explicitly checking their properties to make ourselves familiar with how they work.}

If $(h_k)$ is a Schauder basis, then
$c_k(0)=0$ for all $k$. Suppose that for some real numbers $a_1,\ldots,a_n$ we have
  $\sum_{k=1}^n a_k h_k=0$. Then $a_k=c_k(0)=0$ for $1 \leq k \leq n$, which means that the set $\{h_1,\ldots,h_n\}$ is linearly independent. 
Therefore a Schauder basis is linearly independent. 
It is immediate that the linear span of a Schauder basis is a dense linear subspace of $X$. 
The following shows that a normed space with a Schauder basis is separable.\footnote{In particular, a Banach
space with a Schauder basis is separable. There is a celebrated counterexample found by Per Enflo of a separable
Banach space for which there does not exist a Schauder basis.}

\begin{lemma}
If $(h_k)$ is a Schauder basis for a normed space $X$ then
\[
\left\{\sum_{k=1}^n a_k h_k: n \geq 1, a_1,\ldots,a_n \in \mathbb{Q}\right\}
\]
is dense in $X$. 
\end{lemma}
\begin{proof}
Let $x \in X$ and let $\epsilon>0$. There is some $n$ for which 
\[
\norm{\sum_{k=1}^n c_k(x) h_k - x} \leq \epsilon.
\]
For each $1 \leq k \leq n$, let $a_k \in \mathbb{Q}$ satisfy $|a_k-c_k(x)| \leq \frac{\epsilon}{n \norm{h_k}}$. Then
\begin{align*}
\norm{\sum_{k=1}^n a_k h_k -x} &\leq 
\norm{\sum_{k=1}^n (a_k-c_k(x))h_k} + \epsilon\\
&\leq \sum_{k=1}^n |a_k-c_k(x)| \norm{h_k} + \epsilon\\
&\leq 2\epsilon,
\end{align*}
which proves the claim.
\end{proof}


For each $k$ we define $h_k^*:X \to \mathbb{R}$ by $h_k^*(x)=c_k(x)$. 
For $x,y \in X$, $\sum_{k=1}^n h_k^*(x)h_k \to x$ and $\sum_{k=1}^n h_k^*(y)h_k \to y$,
so $\sum_{k=1}^n (h_k^*(x)+h_k^*(y))h_k \to x+y$ and therefore $h_k^*(x+y)=h_k^*(x)+h_k^*(y)$, and for $\alpha \in \mathbb{R}$,
$\sum_{k=1}^n \alpha h_k^*(x) h_k \to \alpha x$ and therefore $h_k^*(\alpha x)=\alpha h_k^*(x)$.
This shows that $h_k^*:X \to \mathbb{R}$ is linear. 
It is apparent that $h_k^*(h_j)=\delta_{j,k}$. 

We define $P_n:X \to X$ by
\[
P_n x=\sum_{k=1}^n h_k^*(x) h_k,
\]
 and it is immediate that $P_n$ is linear and that
for each $x \in X$, $P_nx \to x$ as $n \to \infty$. It is also immediate that $P_n$ is a \textbf{finite-rank operator}: $P_n(X)$ is a finite-dimensional linear subspace
of $X$.
For $m,n \geq 1$,
\begin{align*}
P_n P_m x &= P_n \sum_{k=1}^m h_k^*(x) h_k\\
& = \sum_{k=1}^m h_k^*(x) P_n h_k\\
&=\sum_{k=1}^m h_k^*(x) \sum_{j=1}^n h_j^*(h_k) h_j\\
&=\sum_{k=1}^m h_k^*(x) \sum_{j=1}^n \delta_{j,k} h_j\\
&=P_{\min(m,n)} x,
\end{align*}
showing that 
\[
P_n P_m = P_{\min(m,n)}.
\]
In particular, $P_n^2 = P_n$, namely, $P_n$ is a \textbf{projection operator}. 
We prove that when $X$ is a Banach space, $P_n$ is continuous.\footnote{N. L. Carothers, {\em A Short Course
in Banach Space Theory}, p.~26, Theorem 3.1.} We indicate explicitly in the proof
when we use that $X$ is a Banach space rather than merely a normed space. 

\begin{theorem}
If $X$ is a Banach space and $(h_n)$ is a Schauder basis for $X$, then each $P_n$ is continuous, and 
$\sup_{n \geq 1} \norm{P_n}<\infty$.  Furthermore, each $h_k^*$ is continuous.
\label{projections}
\end{theorem}
\begin{proof}
For $x \in X$, $P_n x \to x$, so $\norm{P_n x} \to \norm{x}$, which implies that $\sup_{n \geq 1} \norm{P_n x}<\infty$.
It thus makes sense to define
\[
p(x) = \sup_{n \geq 1} \norm{P_n x}.
\]
It is immediate that $p(\alpha x) = |\alpha| p(x)$ and that $p(x+y) \leq p(x) + p(y)$. If $p(x)=0$, then $P_nx=0$ for all $n$, 
which implies that $x=0$. Therefore $p$ is a norm on $X$. 

For $n \geq 1$ and $x \in X$,
\[
\norm{P_n x} \leq
p(x),
\]
showing that $P_n:(X,p) \to (X,\norm{\cdot})$ is a bounded linear operator with operator norm $\leq 1$. 
Suppose that $(x_k)$ is a Cauchy sequence in the norm $p$. Then we have for each $n$  that 
$P_n x_k$ is a Cauchy sequence in the norm
$\norm{\cdot}$, and because $(X,\norm{\cdot})$ is a Banach space, there is some
$y_n \in X$ such that $P_n x_k \to y_n$ in the norm $\norm{\cdot}$ as $k \to \infty$. 
For $\epsilon>0$ there is some $k_\epsilon$ such that when
$j,k \geq k_\epsilon$, $p(x_j-x_k) \leq \epsilon$, and thus for
$k \geq k_\epsilon$ and for any $n$,
\begin{equation}
\norm{P_n x_k-y_n} =  \lim_{j \to \infty} \norm{P_n x_k-P_n x_j}
\leq \limsup_{j \to \infty} p(x_k-x_j)
\leq \epsilon.
\label{Pnxk}
\end{equation}
Now, for any $m,n$ and any $k$,
\[
\norm{y_n-y_m} \leq \norm{y_n - P_n x_k}
+\norm{P_n x_k-x_k}+ \norm{P_m x_k-x_k} + \norm{P_m x_k- y_m},
\]
so using the above, for $k \geq k_\epsilon$, 
\[
\norm{y_n-y_m}  \leq 2\epsilon+\norm{P_n x_k-x_k}+ \norm{P_m x_k-x_k}.
\]
Because $P_n x_k \to x_k$ as $n \to \infty$, there is some  $n_\epsilon$ such that
$\norm{P_n x_k-x_k} \leq \epsilon$ when $n \geq n_\epsilon$, and in this case
for $n,m \geq n_\epsilon$,
\[
\norm{y_n-y_m}  \leq 4\epsilon,
\]
which shows that $y_n$ is a Cauchy sequence in the norm $\norm{\cdot}$, and because
$(X,\norm{\cdot})$ is a Banach space there is some $y \in X$ such that
$y_n \to y$ in the norm $\norm{\cdot}$. 

For any $n,m$, the restriction of the linear map $P_n:X \to X$ to the finite-dimensional linear subspace $P_m(X)$ is
continuous in the norm $\norm{\cdot}$,
thus
\begin{align*}
P_n y_m &=P_n\left( \lim_{k \to \infty} P_m x_k\right)\\
&=\lim_{k \to \infty} P_n P_m x_k\\
&=\lim_{k \to \infty} P_{\min(m,n)} x_k\\
&=y_{\min(m,n)}.
\end{align*}
Using this, we find by induction that for each $n$,
\[
y_n = \sum_{k=1}^n h_k^*(y_k) h_k,
\]
and because $y_n \to y$ in the norm $\norm{\cdot}$ as $n \to \infty$ and $(h_k)$ is a Schauder basis,
this implies that $h_k^*(y)=h_k^*(y_k)$ for all $k$, and thus $P_n y=y_n$. 
Therefore
\[
p(x_k-y) = \sup_{n \geq 1} \norm{P_n(x_k-y)}
= \sup_{n \geq 1} \norm{P_n x_k-y_n}.
\]
For $\epsilon>0$ and $k \geq k_\epsilon$, by \eqref{Pnxk} we have $\norm{P_n x_k-y_n} \leq \epsilon$
and therefore
$p(x_k-y) \leq \epsilon$, which shows that $x_k \to y$ in the norm $p$ as $k \to \infty$. Thus,
$(X,p)$ is a Banach space. 

The identity map $\id_X:X \to X$ is a linear isomorphism, and
for $x \in X$,
\[
\norm{\id_X x} = \norm{x} = \lim_{n \to \infty} \norm{P_n x}
\leq p(x),
\]
showing that $\id_X$ is continuous $(X,p) \to (X,\norm{\cdot})$. 
Because $(X,p)$ and $(X,\norm{\cdot})$ are Banach spaces and
$\id_X$ is a continuous linear isomorphism, by the open mapping theorem\footnote{Walter Rudin,
{\em Functional Analysis}, second ed., p.~49, Corollary 2.12c.}
there is some $a>0$ such that
\[
\norm{x}=\norm{\id_X} \geq ap(x),\qquad x \in X.
\]
Then
\[
\norm{P_n x} \leq p(x) \leq \frac{1}{a} \norm{x},
\]
which means that $P_n:(X,\norm{\cdot}) \to (X,\norm{\cdot})$ is a bounded linear operator with
operator norm $\leq \frac{1}{a}$. 
\end{proof}

Each $P_n$ is a bounded linear operator on $X$, and for each $x \in X$, $P_nx \to x$, which means
that $P_n \to \id_X$ in the \textbf{strong operator topology}. 
The fact that the functions $h_k^*:X \to \mathbb{R}$ are linear and continuous means that they belong to the dual
space
$X^*$. 

In Theorem \ref{projections}, if $\sup_{n \geq 1} \norm{P_n}=1$, we call the Schauder basis $(x_k)$  \textbf{monotone}. 



The following theorem gives sufficient and necessary conditions under which a sequence in a Banach space $X$ is a basic sequence.\footnote{Joseph Diestel, {\em Sequences and Series in Banach Spaces}, p.~36, Chapter V, Theorem 1.} Thus if a sequence satisfies this condition and its linear span is dense in $X$
space, then it is a Schauder basis.

\begin{theorem}
Suppose that $X$ is a Banach space and that $(x_k)$ is a sequence of nonzero elements of $X$. 
There is some $K$ such that for any sequence of real numbers $(a_k)$ and any $n<N$,
\begin{equation}
\norm{\sum_{k=1}^n a_k x_k} \leq K \norm{\sum_{k=1}^N a_k x_k}
\label{Kinequality}
\end{equation}
if and only if $(x_k)$ is a basic sequence.
\end{theorem}
\begin{proof}
It is apparent that $K \geq 1$.
Let $S$ be the linear span of $(x_k)$, and let $S_n$ be the linear span of $\{x_1,\ldots,x_n\}$.
For $n<N$, 
\begin{equation}
|a_n| \norm{x_n} \leq
\norm{\sum_{k=1}^{n-1} a_k x_k}
+\norm{\sum_{k=1}^n a_k x_k}
\leq 2K \norm{\sum_{k=1}^N a_k x_k}.
\label{2K}
\end{equation}
Thus if $\sum_{k=1}^N a_k x_k=0$ then using the above with
$a_{N+1}=0$, for $1 \leq n \leq N$ we get $a_n=0$, showing that
$(x_k)$ is linearly independent. 
Because $(x_k)$ is linearly independent, it makes sense to define a linear map $Q_n:S \to S_n$ by
\[
Q_n x_k = \begin{cases}
x_k&k \leq n\\
0&k > n.
\end{cases}
\]
For $x=\sum_{k=1}^N a_k x_k \in S$, if $n<N$ then by \eqref{Kinequality},
$\norm{Q_n x} \leq K \norm{x}$, and if $n \geq N$ then $Q_n x = x$. Thus 
for each $n$, $Q_n:S \to S_n$ is a bounded linear operator with operator norm $\leq 
K$, and because $X$ is a Banach space and $\overline{S}_n=S_n$,
there is a unique bounded linear operator $P_n:\overline{S} \to S_n$ whose restriction to $S$ is equal
to $Q_n$, which satisfies  $\norm{P_n}=\norm{Q_n} \leq K$.\footnote{Gert K. Pedersen, {\em Analysis Now}, revised
printing, p.~47, Proposition 2.1.11.}
For $s \in S$, $Q_n Q_m  s = Q_{\min(m,n)} s$, and
for $x \in \overline{S}$, there is a sequence $s_k \in S$ 
that tends to $x$, thus
\begin{equation}
P_nP_m x = \lim_{k \to \infty} P_n P_m s_k = 
\lim_{k \to \infty} P_{\min(m,n)} s_k
=P_{\min(m,n)} x.
\label{Pmin}
\end{equation}


For $x \in \overline{S}$ and $\epsilon>0$, there is 
some $s \in S$ with $\norm{s-x} \leq \epsilon$. 
Then there are some $a_1,\ldots,a_m \in \mathbb{R}$ for which $s = \sum_{i=1}^m a_i x_i$, and for $n>m$,
\begin{align*}
\norm{x-P_n x}&\leq \norm{x-s}+\norm{s-P_n s}+\norm{P_n s-  P_n x}\\
&=  \norm{x-s} +\norm{P_n s -P_n x}\\
&\leq \epsilon+K \norm{s-x}\\
&\leq (K+1)\epsilon,
\end{align*}
showing that $P_n x \to x$. It follows from this and \eqref{Pmin} that for each $x \in \overline{S}$ there is a sequence
of real numbers $(c_k)$ such that $\sum_{k=1}^n c_k x_k \to x$. 
If $(b_k)$ is another such sequence, let $a_k=c_k-b_k$, and then we obtain from \eqref{2K}
that $a_k=0$ for each $k$. Therefore, $(c_k)$ is the unique sequence of real numbers such that
$\sum_{k=1}^n c_k x_k \to x$, which establishes that $(x_k)$ is a Schauder basis for $\overline{S}$.
\end{proof}



\section{The Haar system and the Faber-Schauder system}
For $k \geq 0$, for $1 \leq i \leq 2^k$, and for $n=2^k+i$, write
\[
\Delta_n = \Delta_k^i = \left( \frac{i-1}{2^k},\frac{i}{2^k}\right),
\]
and we write $\Delta_1 = \Delta_0^0 = (0,1)$.\footnote{We are partly following the presentation in
B. S. Kashin and A. A. Saakyan, {\em Orthogonal Series}, p.~61, Chapter III.}
Thus
\[
\Delta_1 = \Delta_0^0 = (0,1),\; \Delta_2 = \Delta_0^1 = (0,1),
\; \Delta_3 = \Delta_1^1 = \Big(0,\frac{1}{2}\Big),
\; \Delta_4 = \Delta_1^2 = \Big(\frac{1}{2},
1\Big). 
\]
If $(a,b) \subset [0,1]$, let $(a,b)^- = \left(a,\frac{a+b}{2}\right)$ and $(a,b)^+=\left(\frac{a+b}{2},b\right)$. 
Thus
\[
\Delta_n^-=(\Delta_k^i)^-=\left(\frac{i-1}{2^k}, \frac{2i-1}{2^{k+1}}\right) 
=\left(\frac{2i-2}{2^{k+1}},  \frac{2i-1}{2^{k+1}}\right) 
=\Delta_{k+1}^{2i-1}
\]
and
\[
\Delta_n^+=(\Delta_k^i)^+=\left(\frac{2i-1}{2^{k+1}}, \frac{i}{2^{k}}\right) 
=\left(\frac{2i-1}{2^{k+1}},  \frac{2i}{2^{k+1}}\right) 
=\Delta_{k+1}^{2i}.
\]


\begin{lemma}
If $n \geq m$ then either $\Delta_n \subset \Delta_m$ or $\Delta_n \cap \Delta_m = \emptyset$.
\end{lemma}
\begin{proof}
Let
$n=2^k+i$ and $m=2^l+j$, with $k \geq l$.
Then
\[
\Delta_l^j = \left(\frac{j-1}{2^l},\frac{j}{2^l}\right)
=\left(\frac{2^{k-l}(j-1)}{2^k},\frac{2^{k-l}j}{2^k}\right).
\] 
There are three cases: 
(i) $i \leq 2^{k-l}(j-1)$,
(ii) $2^{k-l}(j-1)<i \leq 2^{k-l}j$,
(iii) $i>2^{k-l}j$. In the first
case, $\Delta_k^i \cap \Delta_l^j = \emptyset$,
i.e. $\Delta_n \cap \Delta_m = \emptyset$. 
 In the second case,
$i-1 \geq 2^{k-l}(j-1)$ and $i \leq 2^{k-l}$, so $\Delta_k^i \subset \Delta_l^j$, i.e.
$\Delta_n \subset \Delta_m$. 
In the third case, $i-1 \geq 2^{k-l}j$ so $\Delta_k^i \cap \Delta_l^j = \emptyset$, i.e. $\Delta_n \cap \Delta_m = \emptyset$. 
\end{proof}

We define $\chi_1=1$,
and for $k \geq 0$,  $1 \leq i \leq 2^k$, and  $n=2^k+i$, we define
\[
\chi_n(t)=\begin{cases}
1&t \in \Delta_n^- =  \left(\frac{2i-2}{2^{k+1}},\frac{2i-1}{2^{k+1}}\right)\\
-1&t \in \Delta_n^+ = \left(\frac{2i-1}{2^{k+1}},  \frac{2i}{2^{k+1}}\right)\\
0&\textrm{otherwise}.
\end{cases}
\]
For example,
\[
\chi_2(t) = 
\begin{cases}
1&t \in \left(0,\frac{1}{2}\right)\\
-1&t \in \left(\frac{1}{2},1\right)\\
0&\textrm{otherwise}
\end{cases}
\]
and
\[
\chi_3(t)
=\begin{cases}
1&t \in \left(0,\frac{1}{4}\right)\\
-1&t \in \left(\frac{1}{4},\frac{1}{2}\right)\\
0&\textrm{otherwise}
\end{cases}
\]
and
\[
\chi_4(t)
=\begin{cases}
1&t \in \left(\frac{1}{2},\frac{3}{4}\right)\\
-1&t \in \left(\frac{3}{4},1\right)\\
0&\textrm{otherwise}.
\end{cases}
\]
We call $(\chi_n)$ the \textbf{Haar system}. It is a fact that $(\chi_n)$ is a monotone Schauder basis for the Banach space
$L^1[0,1]$ with the norm $\norm{f}_{L^1}=\int_0^1 |f(t)| dt$.\footnote{Joram Lindenstrauss and Lior
Tzafriri, {\em Classical Banach Spaces I and II}, p.~3.}



Now we define $\phi_1=1$ and for $n>1$ we define $\phi_n:[0,1] \to \mathbb{R}$ by
\[
\phi_n(t) = \int_0^t \chi_{n-1}(u) du.
\]
Each $\phi_n$ belongs to $C[0,1]$, and we call $(\phi_n)$ the \textbf{Faber-Schauder system}. 
For example,
\[
\phi_2(t) = \int_0^t \chi_1(u) du = t,
\]
 and
\[
 \phi_3(t) = \begin{cases}
\int_0^t 1du&t \in \left[0,\frac{1}{2}\right]\\
 \int_0^{1/2} 1du + \int_{1/2}^t -1 du&t \in \left[\frac{1}{2},1\right]
 \end{cases}
 =\begin{cases}
t &t \in \left[0,\frac{1}{2}\right]\\
-t+1 &t \in \left[\frac{1}{2},1\right]
 \end{cases}
\]
and
\[
 \phi_4(t) = \begin{cases}
\int_0^t 1du&t \in \left[0,\frac{1}{4}\right]\\
 \int_0^{1/4} 1du + \int_{1/4}^t -1 du&t \in \left[\frac{1}{4},\frac{1}{2} \right]\\
 0&t \in \left[\frac{1}{2},1\right]
 \end{cases}
 =\begin{cases}
t &t \in \left[0,\frac{1}{4}\right]\\
-t+\frac{1}{2} &t \in \left[\frac{1}{4},\frac{1}{2} \right]\\
 0&t \in \left[\frac{1}{2},1\right]
 \end{cases}
\]
and
\[
 \phi_5(t) = \begin{cases}
 0&t \in \left[0,\frac{1}{2}\right]\\
\int_{1/2}^t 1du&t \in \left[\frac{1}{2},\frac{3}{4}\right]\\
 \int_{1/2}^{3/4} 1du + \int_{3/4}^t -1 du&t \in \left[\frac{3}{4},1 \right]
  \end{cases}
 =\begin{cases}
  0&t \in \left[0,\frac{1}{2}\right]\\
t -\frac{1}{2}&t \in\left[\frac{1}{2},\frac{3}{4}\right]\\
-t+1 &t \in \left[\frac{3}{4},1 \right].
 \end{cases}
\]
Generally  for $k \geq 0$, for $1 \leq i \leq 2^k$, and for $n=2^k+i$,
\[
\phi_{n+1}(t) = \begin{cases}
t-\frac{i-1}{2^k}&t \in \Delta_{n}^-=\left(\frac{i-1}{2^k}, \frac{2i-1}{2^{k+1}}\right)\\
-t+ \frac{i}{2^{k}}&t \in \Delta_{n}^+=\left(\frac{2i-1}{2^{k+1}}, \frac{i}{2^{k}}\right)\\
0&\textrm{otherwise}.
\end{cases}
\]
We remark that 
\[
\norm{\phi_{n+1}} = \frac{2i-1}{2^{k+1}}-\frac{i-1}{2^k} = 2^{-k-1}.
\]


\section{The Riesz representation theorem}
Let $(X,\mathfrak{M})$ be a measurable space. 
A \textbf{signed measure} is a function $\mu:\mathfrak{M} \to [-\infty,\infty]$ such that
(i) $\mu(\emptyset)=0$, (ii) $\mu$ assumes at most one of the values $-\infty,\infty$, and (iii)
 if $(E_j)$ is a sequence of disjoint elements of $\mathfrak{M}$
then $\mu\left(\bigcup E_j\right) = \sum \mu(E_j)$. 
A \textbf{finite signed measure} is a signed measure whose image is contained in $\mathbb{R}$. 
We denote by 
$\ca(\mathfrak{M})$ 
the collection of all finite signed measures on $\mathfrak{M}$. 
For $\mu,\lambda \in \ca(\mathfrak{M})$ and for $c \in \mathbb{R}$, define
\[
(\mu+\lambda)(E) = \mu(E)+\lambda(E),
\qquad (c\mu)(E) = c \mu(E)
\]
for $E \in \mathfrak{M}$, and we check that with addition and scalar multiplication thus defined $\ca(\mathfrak{M})$ is a real vector space.
A \textbf{positive measure} is a 
signed measure whose imaged is contained in $[0,\infty]$. For $\mu,\lambda \in \ca(\mathfrak{M})$, we write 
\[
\mu \geq \lambda
\]
if $\mu-\lambda$ is a positive measure; in any case $\mu-\lambda \in \ca(\mathfrak{M})$. We check that $\leq$ is a partial order
on $\ca(\mathfrak{M})$ with which $\ca(\mathfrak{M})$ is an \textbf{ordered vector space}.
Finally, a \textbf{probability measure} is a positive measure satisfying $\mu(X)=1$, which in
particular belongs to $\ca(\mathfrak{M})$. 

For $E \in \mathfrak{M}$, a \textbf{partition of $E$} is a countable subset $\{E_i\}$
of $\mathfrak{M}$ whose members are pairwise disjoint. 
For $\mu \in \ca(\mathfrak{M})$ and $E \in \mathfrak{M}$ we define
\[
|\mu|(E) = \sup \left\{\sum_{i=1}^\infty |\mu(E_i)|: \textrm{$\{E_i\}$ is a partition of $E$}\right\}.
\]
It is immediate that $|\mu(E)| \leq |\mu|(E)$. 
It is proved that $|\mu|$ is a finite positive measure on $\mathfrak{M}$,
called the \textbf{total variation measure of $\mu$}.\footnote{Walter Rudin, {\em Real
and Complex Analysis}, third ed., pp.~117--118, Theorem 6.2 and Theorem 6.4.}
For $\mu \in \ca(\mathfrak{M})$, define
\[
\mu^+ = \frac{1}{2}(|\mu|+\mu),\qquad
\mu^-=\frac{1}{2}(|\mu|-\mu).
\]
Because $|\mu|$ is a finite positive measure and $|\mu(E)| \leq |\mu(E)|$, 
$\mu^+$ and $\mu^-$ are finite positive measures, called the \textbf{positive and
negative variations of $\mu$}. Then $\mu=\mu^+ -\mu^-$, called the \textbf{Jordan
decomposition of $\mu$}. 


For $\mu \in \ca(\mathfrak{M})$,
 define
\[
\norm{\mu} = |\mu|(X) = \mu^+(X) + \mu^-(X).
\]
One checks that $\norm{\cdot}$ is a norm on $\ca(\mathfrak{M})$.

For a compact Hausdorff space $K$ and
for $f,g \in C(K)$, write $g \geq f$ when $(g-f)(t) \geq 0$ for all $t \in K$. A \textbf{positive linear functional} is a linear map $\phi:C(K) \to \mathbb{R}$ such that
$\phi(f) \geq 0$ when $f \geq 0$. In this case, because $\norm{f}\cdot 1+f \geq 0$,
\[
\norm{f}\phi(1)+\phi(f) = \phi(\norm{f}\cdot 1 + f) \geq 0
\]
i.e. $-\phi(f) \leq \phi(1) \norm{f}$,
and 
because $\norm{f} \cdot 1 - f \geq 0$,
\[
\norm{f}\phi(1)-\phi(f) = \phi(\norm{f}\cdot 1 - f) \geq 0,
\]
i.e. $\phi(f) \leq \phi(1) \norm{f}$, showing because $\norm{1}=1$ that the operator norm of $\phi$ is
$\norm{\phi} = \phi(1)$, and in particular that $\phi \in C(K)^*$. 
 
 For normed spaces $(V,\norm{\cdot}_V)$ and $(W,\norm{\cdot}_W)$, an \textbf{isometric
 isomorphism} from $V$ to $W$ is a linear isomorphism $T:V \to W$ satisfying $\norm{Tv}_W = 
 \norm{v}_V$ for all $v \in V$. The simplest version of the Riesz representation theorem is for compact
 metrizable spaces, for which we do not need to speak about regular Borel measures or continuous functions vanishing at infinity.\footnote{Walter
 Rudin, {\em Real and Complex Analysis}, third ed., p.~130, Theorem 6.19.}


\begin{theorem}[Riesz representation theorem for compact metrizable spaces]
Let $K$ be a compact metrizable space and define $\Lambda:\ca(\mathscr{B}_K) \to C(K)^*$ by 
\[
\Lambda(\mu)(f) = \int_K f d\mu,\qquad \mu \in \ca(\mathscr{B}_K),\quad f \in C(K).
\]
$\Lambda$ is an isometric isomorphism, and is order preserving: if $\mu \geq 0$ then $\Lambda(\mu) \in C(K)^*$ is a positive linear functional, and thus
if $\mu \geq \lambda$ then $\Lambda(\mu) \geq \Lambda(\lambda)$. 
\end{theorem}




\section{The Borel $\sigma$-algebra of $C[0,1]$}
Let $I=[0,1]$, with the relative topology inherited from $\mathbb{R}$, with which $I$ is a compact metric space.
For $t_1,\ldots,t_n \in I$, we define $\pi_{t_1,\ldots,t_n}:C(I) \to \mathbb{R}^n$ by
\[
\pi_{t_1,\ldots,t_n}(x)=(x(t_1),\ldots,x(t_n)), \qquad x \in C(I),
\]
which is continuous.



For a set $X$ and a collection of functions $f_t:X \to \mathbb{R}$,
the coarsest $\sigma$-algebra on $X$ such that each $f_t$ is measurable
$X \to \mathbb{R}$, where $\mathbb{R}$ has the Borel $\sigma$-algebra, is called
the \textbf{$\sigma$-algebra generated by $\{f_t:t \in I\}$}, and is
denoted
by $\sigma(\{f_t: t \in I\})$. 
We show that the Borel $\sigma$-algebra of $C(I)$ is equal to the $\sigma$-algebra generated by the family
of projection maps $C(I) \to \mathbb{R}$.\footnote{K. R. Parthasarathy, {\em Probability Measures on
Metric Spaces}, p.~212, Theorem 2.1.}


\begin{theorem}
Let $\mathscr{A}$ be the $\sigma$-algebra generated by the family $\{\pi_t:t \in [0,1]\}$. Then
$\mathscr{B}_{C[0,1]}=\mathscr{A}$.
\end{theorem}
\begin{proof}
Because $C(I)$ is a separable metric space it is second-countable, and so if $U$ is an open subset of $C(I)$ then $U$ is equal to the union of countably many open balls. Each open ball
is equal to the union of countably many closed balls: $B(x,r)=\bigcup \overline{B(x,r-1/n)}$. Therefore each open subset of $C(I)$ is equal to
the union of countably many closed balls, and
 to prove that $\mathscr{B}_{C(I)} \subset \mathscr{A}$ it suffices to prove that all closed balls belong to $\mathscr{A}$. 
To this end, 
let $q_1,q_2,\ldots$ be an enumeration of $[0,1] \cap \mathbb{Q}$, 
let $x \in C(I)$, and let $r>0$.  
Suppose that $|y(q_n)-x(q_n)| \leq r$ for all $n$, and take $t \in [0,1]$. Then there is a subsequence
$q_{a_n}$ of $q_n$ that tends to $t$, 
and because $y-x:[0,1] \to \mathbb{R}$ is continuous, 
$|y(q_{a_n})-x(q_{a_n})| \to |y(t)-x(t)|$, and then because for each $n$ we have
$|y(q_{a_n})-x(q_{a_n})| \leq r$ it follows that $|y(t)-x(t)| \leq r$. 
This establishes
\[
\{y \in C(I): \norm{y-x} \leq r\}
=\bigcap_{n=1}^\infty \{y \in C(I): |y(q_n)-x(q_n)| \leq r\}.
\]
But
\[
\{y \in C(I): |y(q_n)-x(q_n)| \leq r\} = \pi_{q_n}^{-1}([\pi_{q_n}(x)-r,\pi_{q_n}(x)+r]),
\]
which belongs to $\mathscr{A}$. Thus $\overline{B(x,r)}$ 
is a countable intersection of elements of $\mathscr{A}$ and so belongs to $\mathscr{A}$, which shows
that $\mathscr{B}_{C(I)} \subset \mathscr{A}$.

On the other hand, for each $t \in I$ the map $\pi_t:C(I) \to \mathbb{R}$ is continuous and hence is
measurable $\mathscr{B}_{C(I)} \to \mathscr{B}_{\mathbb{R}}$.\footnote{Charalambos D. Aliprantis
and Kim C. Border, {\em Infinite Dimensional Analysis: A Hitchhiker's Guide}, third ed., p.~140, Corollary 4.26.}
Therefore  $\mathscr{A} \subset \mathscr{B}_{C(I)}$.
\end{proof}





\section{Relatively compact subsets of $C[0,1]$}
If $(M,d)$ is a metric space, a subset $A$ of $M$ is called \textbf{totally bounded} if for 
each $\epsilon>0$ there are finitely many points $x_1,\ldots,x_n \in M$ such that for any 
point $x \in M$ there is some $i$ for which $d(x,x_i)<\epsilon$. 
It is immediate that a compact metric space is totally bounded, and the \textbf{Heine-Borel theorem} states that
a metric space is compact if and only if it is complete and totally bounded.\footnote{Charalambos D. Aliprantis
and Kim C. Border, {\em Infinite Dimensional Analysis: A Hitchhiker's Guide}, third ed., p.~86, Theorem 3.28.}
On the other hand, one checks that if $(M,d)$ is a metric space and $A$ is a totally bounded subset of $M$, then
the closure $\overline{A}$ is a totally bounded subset of $M$. 
Thus, if $A$ is a totally bounded subset of a complete metric space $(M,d)$, then 
the closure $\overline{A}$ is itself a complete metric space (because $(M,d)$ is a complete metric space), and because $\overline{A}$ is complete and totally bounded, by the Heine-Borel
theorem it is compact.

If $X$ is a topological space and $\mathscr{F}$ is a subset of $C(X)$, we say that
$\mathscr{F}$ is \textbf{equicontinuous at $x \in X$} if for each $\epsilon>0$ there
is a neighborhood $U_{x,\epsilon}$ of $x$ such that $|f(x)-f(y)|<\epsilon$ for all $f \in \mathscr{F}$ and for all
$y \in U_{x,\epsilon}$, and we call $\mathscr{F}$ \textbf{equicontinuous} if it is equicontinuous at each $x \in X$.  
We call $\mathscr{F}$ \textbf{pointwise bounded} if for each $x \in X$, $\{f(x): f \in \mathscr{F}\}$ is a bounded subset
of $\mathbb{R}$. 
The \textbf{Arzel\`a-Ascoli theorem} states that for a compact Hausdorff space $X$ and for $\mathscr{F} \subset C(X)$,
$\mathscr{F}$ is equicontinuous and pointwise bounded if and only if $\mathscr{F}$ is totally bounded.\footnote{Gerald B. Folland, {\em Real
Analysis: Modern Techniques and Their Applications}, second ed., p.~137, Theorem 4.43.}
Then the closure $\overline{\mathscr{F}}$ is totally bounded and is itself a complete metric space, and hence
is a compact metric space. That is, for a compact Hausdorff space $X$, $\mathscr{F}$ is equicontinuous and pointwise bounded if and only if  $\mathscr{F}$ is relatively compact  in $C(X)$.

For $x \in C[0,1]$ and $\delta>0$, define
\[
\omega_x(\delta) = \sup_{s,t \in I, |s-t| \leq \delta} |x(s)-x(t)|.
\]
For $x \in C[0,1]$ because $x:I \to \mathbb{R}$ is continuous and $I$ is compact,
$x$ is uniformly continuous on $I$. Thus
for $\epsilon>0$ there is some $\delta_\epsilon>0$ such that when $|s-t| \leq \delta_\epsilon$,
$|x(s)-x(t)| \leq \epsilon$, hence
if $\delta \leq \delta_\epsilon$ then $\omega_x(\delta) \leq \epsilon$, i.e.
$\omega_x(\delta) \to 0$ as $\delta \to 0$.  We give sufficient and necessary conditions for a subset of $C[0,1]$ to be relatively compact.


\begin{lemma}
Let $A \subset C[0,1]$. $A$ is relatively compact if and only if 
\begin{equation}
\sup_{x \in A} |x(0)|<\infty
\label{x0}
\end{equation}
and
\begin{equation}
\lim_{\delta \downarrow 0} \sup_{x \in A} \omega_x(\delta)=0.
\label{delta0}
\end{equation}
\end{lemma}
\begin{proof}
For $t \in I$ and for $r>0$, let
\[
B_r(t) = \{s \in I: |s-t|<r\},
\]
which is an open neighborhood of $t$.
The Arzel\`a-Ascoli theorem tells us that $A$ is relatively compact
if and only if $A$ is equicontinuous and pointwise bounded. If $A$ is relatively compact, then being pointwise bounded
yields \eqref{x0}. Let $\epsilon>0$. Because $A$ is equicontinuous, for each $t \in I$ there is some $\delta_t>0$ 
such that for all
$x \in A$ and for all
$s \in B_{\delta_t}(t)$ we have 
$|x(s)-x(t)|<\epsilon/2$. 
Then because $I$ is compact, 
there are $t_1,\ldots,t_n \in I$ such that $I = \bigcup_{i=1}^n B_{\delta_{t_i}}(t_i)$.
Let $\delta_\epsilon>0$ be the \textbf{Lebesgue number} for this open cover: for each $t \in I$ there is some
$i$ for which $B_{\delta_\epsilon}(t) \subset B_{\delta_{t_i}}(t_i)$.
For $0<\delta \leq \delta_\epsilon$, for $x \in A$,  and for $|s-t| \leq \delta/2$, 
there is some $i$ for which $B_\delta(t) \subset B_{\delta_{t_i}}(t_i)$, so $s,t \in B_{\delta_{t_i}}(t_i)$, which means that
$|x(s)-x(t_i)|<\epsilon/2$ and $|x(t)-x(t_i)|<\epsilon/2$, and thus
$|x(s)-x(t)| \leq |x(s)-x(t_i)|+|x(t_i)-x(t)| < \epsilon$. 
Therefore $\omega_{\delta/2}(x) \leq \epsilon$, and this is true for all $x \in A$ which proves \eqref{delta0}.

Suppose now that \eqref{x0} and \eqref{delta0} are true. 
By \eqref{delta0}, there is some $m \geq 1$ such that
$\sup_{x \in A} \omega_x(1/m) \leq 1$, and therefore
for $x \in A$, $\norm{x} \leq |x(0)|+m$.
With \eqref{x0} this yields
$\sup_{x \in A} \norm{x}<\infty$, whence $A$ is pointwise bounded. 
Let $t \in I$ and let $\epsilon>0$. By \eqref{delta0} there is some $\delta>0$ such that
$\sup_{x \in A} \omega_x(\delta) < \epsilon$. Thus for 
$x \in A$ and for $|s-t| \leq \delta$, $|x(s)-x(t)|<\epsilon$, which shows that $A$ is equicontinuous at $t$. 
This is true for all $t \in I$, so $A$ is equicontinuous and by the Arzel\`a-Ascoli theorem we get that
$A$ is relatively compact in $C[0,1]$.
\end{proof}






\section{Continuously differentiable functions}
Let $f:[0,1] \to \mathbb{R}$ be a function. We say that $f$ is \textbf{differentiable at $t \in [0,1]$}
if  there is some $f'(t) \in \mathbb{R}$ such that\footnote{cf.
Nicolas Bourbaki, {\em Elements of Mathematics. Functions of a Real Variable: Elementary Theory},
p.~3, Chapter I, \S 1, no.~1, Definition 1.}
\[
\lim_{s \to t, s \in [0,1] \setminus \{t\}} \frac{f(s)-f(t)}{s-t} = f'(t).
\]
If $f$ is differentiable at each $t \in [0,1]$, we say that $f$ is \textbf{differentiable on $[0,1]$}
and define $f':[0,1] \to \mathbb{R}$ by $t \mapsto f'(t)$. 
We define $C^1[0,1]$ to be the collection of those $f:[0,1] \to \mathbb{R}$ such that
$f$ is differentiable on $[0,1]$ and $f':[0,1] \to \mathbb{R}$ is continuous.
$C^1[0,1]$ is contained in $C[0,1]$, and it turns out that $C^1[0,1]$ is a Borel set in $C[0,1]$.\footnote{Alexander S. Kechris, {\em Classical
Descriptive Set Theory}, p.~70, \S 11.B, Example 2.}
On the other hand, the collection differentiable functions $[0,1] \to \mathbb{R}$, which is a subset of $C[0,1]$,
 is not a Borel set in $C[0,1]$.\footnote{S. M. Srivastava, {\em A Course on Borel Sets}, p.~139, Proposition 4.2.7.}




\end{document}