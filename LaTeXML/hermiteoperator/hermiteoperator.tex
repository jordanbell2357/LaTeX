\documentclass{article}
\usepackage{amsmath,amssymb,mathrsfs,amsthm}
%\usepackage{tikz-cd}
\usepackage[draft]{hyperref}
\newcommand{\inner}[2]{\left\langle #1, #2 \right\rangle}
\newcommand{\tr}{\ensuremath\mathrm{tr}\,} 
\newcommand{\Span}{\ensuremath\mathrm{span}} 
\def\Re{\ensuremath{\mathrm{Re}}\,}
\def\Im{\ensuremath{\mathrm{Im}}\,}
\newcommand{\id}{\ensuremath\mathrm{id}} 
\newcommand{\var}{\ensuremath\mathrm{var}} 
\newcommand{\Lip}{\ensuremath\mathrm{Lip}} 
\newcommand{\GL}{\ensuremath\mathrm{GL}} 
\newcommand{\diam}{\ensuremath\mathrm{diam}} 
\newcommand{\sgn}{\ensuremath\mathrm{sgn}\,} 
\newcommand{\lcm}{\ensuremath\mathrm{lcm}} 
\newcommand{\supp}{\ensuremath\mathrm{supp}\,}
\newcommand{\dom}{\ensuremath\mathrm{dom}\,}
\newcommand{\upto}{\nearrow}
\newcommand{\downto}{\searrow}
\newcommand{\norm}[1]{\left\Vert #1 \right\Vert}
\newcommand{\HS}[1]{\left\Vert #1 \right\Vert_{\mathrm{HS}}}
\theoremstyle{definition}
\newtheorem{theorem}{Theorem}
\newtheorem{lemma}[theorem]{Lemma}
\newtheorem{proposition}[theorem]{Proposition}
\newtheorem{corollary}[theorem]{Corollary}
\theoremstyle{definition}
\newtheorem{definition}[theorem]{Definition}
\newtheorem{example}[theorem]{Example}
\begin{document}
\title{Schwartz functions, Hermite functions, and the Hermite operator}
\author{Jordan Bell}
\date{July 17, 2015}

\maketitle


\section{Schwartz functions}
For $\phi \in C^\infty(\mathbb{R},\mathbb{C})$ and $p \geq 0$,
let 
\[
|\phi|_p =\sup_{0 \leq k \leq p} \sup_{u \in \mathbb{R}} (1+u^2)^{p/2} |\phi^{(k)}(u)|.
\]
We define $\mathscr{S}$ to be the set of those 
$\phi \in C^\infty(\mathbb{R},\mathbb{C})$ such that $|\phi|_p<\infty$ for all $p \geq 0$. 
$\mathscr{S}$ is a complex vector space 
and each $|\cdot|_p$ is a norm, and
because each $|\cdot|_p$ is a norm, a fortiori $\{|\cdot|_p:  p \geq 0\}$ is a separating family of seminorms.
With the topology induced by this family of seminorms, $\mathscr{S}$ is a Fr\'echet space.\footnote{Walter
Rudin, {\em Functional Analysis}, second ed., p.~184, Theorem 7.4.}
Furthermore, $D:\mathscr{S} \to \mathscr{S}$ defined by
\[
(D\phi)(x) = \phi'(x),\qquad x \in \mathbb{R}
\]
and $M:\mathscr{S} \to \mathscr{S}$ defined by
\[
(M\phi)(x) = x\phi(x),\qquad x \in \mathbb{R}
\]
are continuous linear maps. 

Let $\mathscr{S}'$ be the collection of continuous linear maps $\mathscr{S}
\to \mathbb{C}$.
For $\phi \in \mathscr{S}$, define $e_\phi:\mathscr{S}' \to \mathbb{C}$ by
\[
e_\phi(\omega) = \omega(\phi), \qquad \omega \in \mathscr{S}'.
\]
The initial topology for the collection
$\{e_\phi: \phi \in \mathscr{S}\}$ is called the \textbf{weak-* topology} on $\mathscr{S}'$. 
With this topology, $\mathscr{S}'$ is a locally convex space whose dual space
is $\{e_\phi: \phi \in \mathscr{S}\}$.



\section{{\em L\textsuperscript{2}} norms}
\label{L2norms}
For $p \geq 0$ and $\phi,\psi \in \mathscr{S}$, let 
\[
[\phi,\psi]_p = \sum_{k=0}^p \int_{\mathbb{R}} (1+u^2)^{p} \phi^{(k)}(u)  \overline{\psi^{(k)}(u)} du,
\]
and let
\[
[\phi]_p^2 = [\phi,\phi]_p=  \sum_{k=0}^p \int_{\mathbb{R}} (1+u^2)^p |\phi^{(k)}(u)|^2 du.
\]
Because $(1+u^2)^p \leq (1+u^2)^q$ when $p \leq q$, it is immediate that
$[\phi]_p \leq [\phi]_q$ when $p \leq q$. 


We relate the norms $|\cdot|_p$ and the norms $[\cdot]_p$.\footnote{Takeyuki Hida,
{\em Brownian Motion}, p.~305, Lemma A.1.}

\begin{lemma}
For each $p \geq 1$, for all
$\phi \in \mathscr{S}$,
\[
\frac{1}{p\sqrt{\pi}} |\phi|_{p-1} \leq [\phi]_p \leq  \sqrt{(p+1)\pi} |\phi|_{p+1}.
\]
\label{L2equiv}
\end{lemma}
\begin{proof}
For $0 \leq k \leq p$,
\begin{align*}
\int_{\mathbb{R}} (1+u^2)^p |\phi^{(k)}(u)|^2 du &\leq \sup_{u \in \mathbb{R}} ((1+u^2)^{p+1} |\phi^{(k)}(u)|^2) 
\int_{\mathbb{R}} (1+u^2)^{-1} du\\
&=  \sup_{u \in \mathbb{R}} ((1+u^2)^{p+1} |\phi^{(k)}(u)|^2)  \cdot \pi\\
&\leq \pi |\phi|_{p+1}^2,
\end{align*}
hence
\begin{align*}
[\phi]_p^2&=  \sum_{k=0}^p \int_{\mathbb{R}} (1+u^2)^p |\phi^{(k)}(u)|^2 du\\
&\leq \sum_{k=0}^p \pi |\phi|_{p+1}^2\\
&=(p+1)\pi  |\phi|_{p+1}^2.
\end{align*}


For $0 \leq k \leq p-1$ and $u \in \mathbb{R}$, using the fundamental theorem of calculus and the Cauchy-Schwarz inequality,
\begin{align*}
|(1+u^2)^{(p-1)/2}\phi^{(k)}(u)|&=\left| \int_{-\infty}^u ((1+t^2)^{(p-1)/2} \phi^{(k)}(t))' dt\right|\\
&\leq \int_{\mathbb{R}} |(p-1)t (1+t^2)^{(p-1)/2-1} \phi^{(k)}(t)| dt\\
&+\int_{\mathbb{R}} |(1+t^2)^{(p-1)/2} \phi^{(k+1)}(t)| dt\\
&\leq (p-1) \int_{\mathbb{R}} (1+t^2)^{-1/2} (1+t^2)^{(p-1)/2} |\phi^{(k)}(t)| dt\\
&+\int_{\mathbb{R}} (1+t^2)^{-1/2} (1+t^2)^{p/2} |\phi^{(k+1)}(t)| dt\\
&\leq (p-1) \left(\int_{\mathbb{R}} (1+t^2)^{-1} dt \right)^{1/2} \left( \int_{\mathbb{R}} (1+t^2)^{p-1} |\phi^{(k)}(t)|^2 dt\right)^{1/2}\\
&+\left(\int_{\mathbb{R}} (1+t^2)^{-1} dt\right)^{1/2} \left(\int_{\mathbb{R}} (1+t^2)^p |\phi^{(k+1)}(t)|^2 dt \right)^{1/2}\\
&\leq (p-1)\sqrt{\pi} [\phi]_{p-1} + \sqrt{\pi} [\phi]_p\\
&\leq p\sqrt{\pi} [\phi]_p,
\end{align*}
which shows that
\[
|\phi|_{p-1} \leq p\sqrt{\pi} [\phi]_p.
\]
\end{proof}

 


\section{Hermite functions}
Let $\lambda$ be Lebesgue measure on $\mathbb{R}$,
and let 
\[
(f,g)_{L^2} = \int_{\mathbb{R}} f \overline{g} d\lambda.
\]
$L^2(\lambda)$ with the inner product $(\cdot,\cdot)_{L^2}$ is a separable Hilbert space. 
For $n \geq 0$, let
\[
h_n(x) = (-1)^n(2^n n! \sqrt{\pi})^{-1/2} e^{x^2/2} D^n e^{-x^2},
\]
the \textbf{Hermite functions}, the set of which is an orthonormal basis for $L^2(\lambda)$.
We remark that the Hermite functions belong to $\mathscr{S}$.
For $n<0$ we define
\[
h_n=0,
\]
to write some expressions in a uniform way.

We calculate that for $n \geq 0$,
\[
Dh_n = \sqrt{\frac{n}{2}} h_{n-1}-\sqrt{\frac{n+1}{2}} h_{n+1}.
\]


We define the \textbf{Hermite operator} $A:\mathscr{S} \to \mathscr{S}$ by
\[
A=-D^2+M^2+1.
\]
$A$ is a densely defined operator in $L^2(\lambda)$ that is symmetric and positive, and  satisfies
\[
Ah_n = (2n+2)h_n.
\]
There is a unique bounded linear operator $T:L^2(\lambda) \to L^2(\lambda)$ satisfying
\[
Th_n=A^{-1}h_n = (2n+2)^{-1} h_n,\qquad n \geq 0.
\]
The operator norm of $T$ is $\norm{T}=\frac{1}{2}$, and $T$ is self-adjoint. For $p \geq 1$, 
$T^p$ is a Hilbert-Schmidt operator with Hilbert-Schmidt norm $\HS{T^p}=2^{-p} \sqrt{\zeta(2p)}$. 

We define the \textbf{creation operator} $B:\mathscr{S} \to \mathscr{S}$ by
\[
B=D+M
\]
and we define the \textbf{annihilation operator} $C:\mathscr{S} \to \mathscr{S}$ by
\[
C=-D+M,
\]
which are continuous linear maps. They satisfy, for $n \geq 0$,
\[
Bh_n = (2n)^{1/2}h_{n-1},\qquad Ch_n=(2n+2)^{1/2}h_{n+1}.
\]
(We remind ourselves that we have defined $h_{-1}=0$.)
It is immediate that $BC=A$ and that $B-C=2D$. 
Using the creation operator, we can write the Hermite functions as
\[
h_n=(2^n n!)^{-1/2} C^n h_0 = \pi^{-1/4} (2^n n!)^{-1/2} C^n(e^{-x^2/2}).
\]

For $\phi,\psi \in \mathscr{S}$, using integration by parts,
\[
(D\phi,\psi)_{L^2}=\int_{\mathbb{R}} \phi'(x) \overline{\psi(x)} dx
=-\int_{\mathbb{R}} \phi(x) \overline{\psi'(x)} dx
=(\phi,(-D)\psi)_{L^2},
\]
and 
\[
(M\phi,\psi)_{L^2}=\int_{\mathbb{R}} x \phi(x) \overline{\psi(x)} dx
=(\phi,M\psi)_{L^2}.
\]
Thus,
\begin{align*}
(B\phi,\psi)_{L^2}&=(D\phi,\psi)_{L^2} + (M\phi,\psi)_{L^2}\\
&=(\phi,(-D)\psi)_{L^2}+(\phi,M\psi)_{L^2}\\
&=(\phi,C\psi)_{L^2}
\end{align*}
and
\[
(C\phi,\psi)_{L^2}=(\phi,B\psi)_{L^2}.
\]
We shall use these calculations to obtain the following lemma.


\begin{lemma}
For $p \geq 0$ and for $\phi \in \mathscr{S}$,
\[
B^p \phi = 2^{p/2}\sum_{n=0}^\infty \left( \frac{(n+p)!}{n!} \right)^{1/2} (\phi,h_{n+p})_{L^2}   h_n
\]
and
\[
C^p \phi =2^{p/2} \sum_{n=0}^\infty (\phi,h_{n-p})_{L^2}  \left( \frac{n!}{(n-p)!} \right)^{1/2} h_n.
\]
\end{lemma}
\begin{proof}
Because $Ch_n=(2n+2)^{1/2}h_{n+1}$,
\[
(\phi,C^p h_n)_{L^2}
=(\phi,h_{n+p})_{L^2} \prod_{j=n}^{n+p-1}(2j+2)^{1/2} 
=(\phi,h_{n+p})_{L^2} 2^{p/2}  \left( \frac{(n+p)!}{n!} \right)^{1/2}.
\]
With
\[
\phi = \sum_{n=0}^\infty (\phi,h_n)_{L^2} h_n,
\]
and because $(B\phi,\psi)_{L^2}=(\phi,C\psi)_{L^2}$, we have
\begin{align*}
B^p \phi &= \sum_{n=0}^\infty (B^p \phi,h_n)_{L^2} h_n\\
&=\sum_{n=0}^\infty  (\phi,C^p h_n)_{L^2} h_n\\
&=\sum_{n=0}^\infty (\phi,h_{n+p})_{L^2} 2^{p/2} \left( \frac{(n+p)!}{n!} \right)^{1/2} h_n.
\end{align*}

Because $Bh_n=(2n)^{1/2}h_{n-1}$, and reminding ourselves that we define $h_n=0$ for $n<0$,
\[
(\phi,B^ph_n)_{L^2}=(\phi,h_{n-p})_{L^2} \prod_{j=n-p+1}^n (2j)^{1/2}
=(\phi,h_{n-p})_{L^2} 2^{p/2} \left( \frac{n!}{(n-p)!} \right)^{1/2}.
\]
Because $(C\phi,\psi)_{L^2}=(\phi,B\psi)_{L^2}$, we have
\begin{align*}
C^p\phi&=\sum_{n=0}^\infty (C^p\phi,\psi)_{L^2} h_n\\
&=\sum_{n=0}^\infty (\phi,B^p\psi)_{L^2} h_n\\
&=\sum_{n=0}^\infty (\phi,h_{n-p})_{L^2} 2^{p/2} \left( \frac{n!}{(n-p)!} \right)^{1/2} h_n.
\end{align*}
\end{proof}

We define the Fourier transform $\mathscr{F}:\mathscr{S} \to \mathscr{S}$ by
\[
(\mathscr{F}\phi)(\xi) = \int_{\mathbb{R}} \phi(x) e^{-i\xi x} \frac{dx}{(2\pi)^{1/2}},\qquad \xi \in \mathbb{R}.
\]
$\mathscr{F}:\mathscr{S} \to \mathscr{S}$ is a continuous linear map, and satisfies
\[
\mathscr{F}M = iD\mathscr{F},\qquad \mathscr{F}D = iM\mathscr{F}. 
\]
From these we obtain
\[
\mathscr{F}A=A\mathscr{F},\qquad \mathscr{F}B=iB\mathscr{F},\qquad \mathscr{F}C=-iC\mathscr{F},
\]
and one proves the following using the above.

\begin{lemma}
For $n \geq 0$,
\[
\mathscr{F}h_n = (-i)^n h_n.
\] 
\end{lemma}

We further remark that for $\phi \in \mathscr{S}$,
\begin{equation}
\norm{\phi}_{\infty} \leq 2^{-1/2}(\norm{\phi}_{L^2}^2+\norm{\phi'}_{L^2}^2).
\label{supremum}
\end{equation}

Finally, there is a unique Hilbert space isomorphism $\mathscr{F}:L^2(\lambda) \to L^2(\lambda)$ whose
restriction to $\mathscr{S}$ is equal to $\mathscr{F}$ as already defined. Thus for $f \in L^2(\lambda)$, as
\[
f=\sum_{n=0}^\infty (f,h_n)_{L^2} h_n,
\]
we get
\[
\mathscr{F}f = \sum_{n=0}^\infty (f,h_n)_{L^2} (-i)^n h_n.
\]


\section{Hermite operator}
For $p \geq 0$ and $f \in L^2(\lambda)$, we define
\[
\norm{f}_p^2 = \sum_{n=0}^\infty (2n+2)^{2p} |(f,h_n)_{L^2}|^2.
\]
We define 
\[
\mathscr{S}_p = \{f \in L^2(\lambda): \norm{f}_p < \infty\},
\]
and for $f,g \in \mathscr{S}_p$ we define
\[
(f,g)_p = \sum_{n=0}^\infty (2n+2)^{2p} (f,h_n)_{L^2} \overline{(g,h_n)_{L^2}},
\]
for which
\[
\norm{f}_p^2 = (f,f)_p. 
\]

\begin{lemma}
For $\phi \in \mathscr{S}$, for each $p \geq 0$, $\phi \in \mathscr{S}_p$,
and 
\[
\norm{\phi}_p = \norm{A^p \phi}_{L^2}.
\]
\label{phermite}
\end{lemma}
\begin{proof}
$A^p \phi \in \mathscr{S}$, so
$\norm{A^p \phi}_{L^2}<\infty$. 
Because $A$ is a symmetric operator and
as $Ah_n=(2n+2) h_n$,
\begin{align*}
\norm{A^p\phi}_{L^2}^2&=\sum_{n=0}^\infty |(A^p\phi,h_n)_{L^2}|^2\\
&=\sum_{n=0}^\infty |(\phi,A^ph_n)_{L^2}|^2\\
&=\sum_{n=0}^\infty (2n+2)^{2p} |(\phi,h_n)_{L^2}|^2\\
&=\norm{\phi}_p^2.
\end{align*}
\end{proof}

For $f,g \in L^2(\lambda)$, because $T$ is self-adjoint,
\begin{align*}
(T^p f,T^p g)_p &=\sum_{n=0}^\infty (2n+2)^{2p} (T^p f,h_n)_{L^2} \overline{(T^p f,h_n)_{L^2}}\\
&=\sum_{n=0}^\infty (2n+2)^{2p} (f,T^p h_n)_{L^2} \overline{(g,T^p h_n)_{L^2}}\\
&=\sum_{n=0}^\infty (2n+2)^{2p} (f,(2n+2)^{-p} h_n)_{L^2} \overline{(g,(2n+2)^{-p} h_n)_{L^2}} \\
&=\sum_{n=0}^\infty (f,h_n)_{L^2} \overline{(g,h_n)_{L^2}}\\
&=(f,g)_{L^2},
\end{align*}
and so $\norm{T^p f}_p = \norm{f}_{L^2}$, which shows that 
\[
T^p L^2(\lambda) = \mathscr{S}_p.
\]
If $f_i \in \mathscr{S}_p$ is a Cauchy sequence in the norm $\norm{\cdot}_p$, 
then as $\norm{T^{-p} f_i-T^{-p} f_j}_{L^2}=\norm{f_i-f_j}_p$, 
$T^{-p}f_i$ is a Cauchy sequence in the norm $\norm{\cdot}_{L^2}$ and so there is  some
$g \in L^2(\lambda)$ for which $\norm{T^{-p}f_i-g}_{L^2} \to 0$. We have $T^p g \in \mathscr{S}_p$, and
\[
\norm{f_i-T^p g}_p = \norm{T^{-p} f_i-g}_{L^2} \to 0,
\]
thus $f_i \to T^p g$ in the norm $\norm{\cdot}_p$, showing that 
$(\mathscr{S}_p,(\cdot,\cdot)_p)$ is a Hilbert space. 
Furthermore, $T^p:L^2(\lambda) \to \mathscr{S}_p$ is an isomorphism of Hilbert spaces, and thus
$\{T^p h_n: n \geq 0\}$ is an orthonormal basis for $(\mathscr{S}_p,(\cdot,\cdot)_p)$. 



For $p \leq q$,
\[
\norm{f}_p \leq \norm{f}_q,
\]
so $\mathscr{S}_q \subset \mathscr{S}_p$. 
For $p \geq q$, let $i_{q,p}:\mathscr{S}_q \to \mathscr{S}_p$ be the inclusion
map.\footnote{Hui-Hsiung Kuo, {\em White Noise Distribution Theory}, p.~18, Lemma 3.3.}

\begin{theorem}
For $p < q$, the inclusion map $i_{q,p}:\mathscr{S}_q \to \mathscr{S}_p$ is
a Hilbert-Schmidt operator, with Hilbert-Schmidt norm
\[
\HS{i_{q,p}} = 2^{-q+p} \sqrt{\zeta(2q-2p)}.
\]
\end{theorem}
\begin{proof}
$\{T^q h_n: n \geq 0\}$ is an orthonormal basis for $(\mathscr{S}_q,(\cdot,\cdot)_q)$, and
\begin{align*}
\HS{i_{q,p}}^2&=\sum_{n=0}^\infty \norm{i_{q,p} T^q h_n}_p^2\\
&=\sum_{n=0}^\infty \norm{T^q h_n}_p^2\\
&=\sum_{n=0}^\infty \norm{(2n+2)^{-q} h_n}_p^2\\
&=\sum_{n=0}^\infty (2n+2)^{-2q} (2n+2)^{2p}\\
&=2^{-2q+2p} \zeta(2q-2p).
\end{align*}
\end{proof}




\section{The Hilbert spaces {\em S\textsubscript{p}}}
For $f \in L^2(\lambda)$,
\[
f = \sum_{n=0}^\infty (f,h_n)_{L^2}h_n,
\]
and for $N \geq 0$  we define $f_N:\mathbb{R} \to \mathbb{C}$ by
\[
f_N(x) = \sum_{n=0}^N (f,h_n)_{L^2}h_n(x),\qquad x \in \mathbb{R},
\]
which belongs to $\mathscr{S}$. 

For $k \geq 0$, we define $C_b^k(\mathbb{R})$ to be the set of those functions
$\mathbb{R} \to \mathbb{C}$ that are $k$-times differentiable and
 such that for each $0 \leq j \leq k$, $f^{(j)}$  is continuous and bounded. With the norm
\[
\norm{f}_{C_b^k} = \sum_{j=0}^k \norm{f^{(j)}}_\infty
\]
this is a Banach space. 
Because the Hermite functions belong to $\mathscr{S}$, for $f \in L^2(\lambda)$ and for any $k$ 
and $N$, the function $f_N$ belongs to $C_b^k(\mathbb{R})$.  

\begin{lemma}
If $p \geq 1$ and $f \in \mathscr{S}_p$, then there is some $F \in C_b^{p-1}(\mathbb{R})$ such that
$f$ is equal almost everywhere to $F$.
\label{cramer}
\end{lemma}
\begin{proof}
\textbf{Cram\'er's inequality} states that there is a constant $K_0$ such that for all
$n$, $\norm{h_n}_\infty \leq K_0$. 
For $M < N$, using this and the Cauchy-Schwarz inequality,
\begin{align*}
\norm{f_N-f_M}_{C_b^0}&=\norm{ \sum_{n=M+1}^N (f,h_n)_{L^2} h_n}_\infty\\
&\leq K_0 \sum_{n=M+1}^N |(f,h_n)_{L^2}|\\
&= K_0 \sum_{n=M+1}^N (2n+2)^{-1} (2n+2)  |(f,h_n)_{L^2}|\\
&\leq \left( \sum_{n=M+1}^N (2n+2)^{-2}\right)^{1/2} 
\left( \sum_{n=M+1}^N (2n+2)^2   |(f,h_n)_{L^2}|^2 \right)^{1/2}\\
&= \left( \sum_{n=M+1}^N (2n+2)^{-2}\right)^{1/2} \norm{f_N-f_M}_1.
\end{align*}
Because $f \in \mathscr{S}_p \subset \mathscr{S}_1$, $f_N$ is a Cauchy sequence in $\mathscr{S}_1$, hence
$f_N$ is a Cauchy sequence in $C_b^0(\mathbb{R})$, so there is some $F \in C_b^0(\mathbb{R})$ such that
$f_N$ converges to $F$ in $C_b^0(\mathbb{R})$. 
We assert that $F=f$ as elements of $L^2(\lambda)$.

Using
\[
Dh_n= \sqrt{\frac{n}{2}} h_{n-1}-\sqrt{\frac{n+1}{2}} h_{n+1},
\]
we calculate
\begin{align*}
f_N' &= -\sqrt{\frac{N}{2}} (f,h_{N-1})_{L^2} h_N - \sqrt{\frac{N+1}{2}} (f,h_N)_{L^2} h_{N+1}\\
&+\sum_{n=0}^{N-1} \left( \sqrt{\frac{n+1}{2}} (f,h_{n+1})_{L^2}  - \sqrt{\frac{n}{2}} (f,h_{n-1})_{L^2}\right)h_n,
\end{align*}
hence for $M<N$,
\begin{align*}
f_N'-f_M'&= -\sqrt{\frac{N}{2}} (f,h_{N-1})_{L^2} h_N - \sqrt{\frac{N+1}{2}} (f,h_N)_{L^2} h_{N+1}\\
& +\sqrt{\frac{M}{2}} (f,h_{M-1})_{L^2} h_M + \sqrt{\frac{M+1}{2}} (f,h_M)_{L^2} h_{M+1}\\
&+\sum_{n=M}^{N-1} \left( \sqrt{\frac{n+1}{2}} (f,h_{n+1})_{L^2}  - \sqrt{\frac{n}{2}} (f,h_{n-1})_{L^2}\right)h_n,
\end{align*}
and for $N \geq M+2$,
\begin{align*}
\norm{f_N'-f_M'}_{1}&=(2N+2)^2 \frac{N}{2} |(f,h_{N-1})|_{L^2}^2 +(2N+4)^2 \frac{N+1}{2} |(f,h_{N-1})|_{L^2}^2\\
&(2M+2)^2 \frac{M+1}{2} |(f,h_{M+1})|_{L^2}^2+(2M+4)^2 \frac{M+2}{2} |(f,h_{M+1})|_{L^2}^2\\
&+\sum_{n=M+2}^{N-1} (2n+2)^2 \left| \sqrt{\frac{n+1}{2}} (f,h_{n+1})_{L^2}  - \sqrt{\frac{n}{2}} (f,h_{n-1})_{L^2}\right|^2\\
&=O(\norm{f_N-f_M}_2),
\end{align*}
whence
$f_N'$ is a Cauchy sequence in $C_b^0(\mathbb{R})$, and so $f_N$ is a Cauchy sequence in $C_b^1(\mathbb{R})$. 
\end{proof}




We prove that for $p \geq 1$ the derivatives of the partial sums $f_N$ are a Cauchy sequence
in $L^2(\lambda)$.\footnote{Jeremy J. Becnel and Ambar N. Sengupta,
{\em The Schwartz space: a background to white noise analysis},
\url{https://www.math.lsu.edu/~preprint/2004/as20041.pdf}, Lemma 7.1.}

\begin{lemma}
For $p \geq 1$ and $f \in \mathscr{S}_p$, $f_N'$ is a Cauchy sequence in $L^2(\lambda)$.
\label{cauchyL2}
\end{lemma}
\begin{proof}
Because $f_N \in \mathscr{S}$,
\[
f_N' =Df_N= \frac{B-C}{2}f_N.
\]
Then
\[
\norm{f_N'-f_M'}_{L^2} \leq \frac{1}{2}\norm{Bf_N-Bf_M}_{L^2} + \frac{1}{2}\norm{Cf_N-Cf_M}_{L^2}.
\]
For $M<N$, as $Bh_n=(2n)^{1/2}h_{n-1}$,
\begin{align*}
\norm{Bf_N-Bf_M}_{L^2}^2&=\norm{B \sum_{n=M+1}^N (f,h_n)_{L^2} h_n}_{L^2}^2\\
&=\norm{ \sum_{n=M+1}^N (f,h_n)_{L^2} (2n)^{1/2} h_{n-1}}_{L^2}^2\\
&=\sum_{n=M+1}^N |(f,h_n)_{L^2}|^2 (2n)\\
&\leq \sum_{n=M+1}^N (2n+2)^2 |(f,h_n)_{L^2}|^2,
\end{align*}
and as $Ch_n=(2n+2)^{1/2}h_{n+1}$,
\begin{align*}
\norm{Cf_N-Cf_M}_{L^2}^2&=\norm{C \sum_{n=M+1}^N (f,h_n)_{L^2} h_n}_{L^2}^2\\
&=\norm{ \sum_{n=M+1}^N (f,h_n)_{L^2}(2n+2)^{1/2}h_{n+1}}_{L^2}^2\\
&=\sum_{n=M+1}^N |(f,h_n)_{L^2}|^2 (2n+2)\\
&\leq \sum_{n=M+1}^N (2n+2)^2 |(f,h_n)_{L^2}|^2.
\end{align*}
Thus
\[
\norm{f_N'-f_M'}_{L^2} \leq \frac{1}{2}\norm{f_N-f_M}_1 +  \frac{1}{2}\norm{f_N-f_M}_1
= \norm{f_N-f_M}_1.
\]
Because $f \in \mathscr{S}_p$ and $p \geq 1$, 
the series $\sum_{n=0}^\infty (2n+2)^2 |(f,h_n)_{L^2}|^2$ converges, from
which the claim follows.
\end{proof}



Now we establish that if
$p \geq 1$ and $f \in \mathscr{S}_p$ then there is some $F \in C_b^0(\mathbb{R})$ such that $f$ is equal almost everywhere to $F$,
$F$ is differentiable almost everywhere,
and $F'  \in \mathscr{S}_{p-1}$.\footnote{Jeremy J. Becnel and Ambar N. Sengupta,
{\em The Schwartz space: a background to white noise analysis},
\url{https://www.math.lsu.edu/~preprint/2004/as20041.pdf}, Theorem 7.3.}

\begin{theorem}
For $p \geq 1$ and $f \in \mathscr{S}_p$,
there is some $F \in C_b^0(\mathbb{R})$
such that $f$ is equal almost everywhere to $F$, 
$F$ is differentiable almost everywhere,
$f_N'$ converges to $F'$ in the norm
$\norm{\cdot}_{L^2}$, and    
 $F' \in \mathscr{S}_{p-1}$. 
\end{theorem}
\begin{proof}
Lemma \ref{cauchyL2} tells us that $f_N'$ is a Cauchy sequence in the norm $\norm{\cdot}_{L^2}$, and
hence there is some $g \in L^2(\lambda)$ to which $f_N'$ converges in
 the norm $\norm{\cdot}_{L^2}$. 
For $x \leq y$, by the fundamental theorem of calculus,
\[
f_N(y) = f_N(x) + \int_0^1 f_N'(x+t(y-x))\cdot (y-x) dt.
\]
By the Cauchy-Schwarz inequality,
\[
\begin{split}
&\int_0^1|f_N'(x+t(y-x))\cdot (y-x)-g(x+t(y-x))\cdot (y-x)| dt\\
=&\int_x^y |f_N'(u)-g(u)| du\\
\leq&\sqrt{y-x} \norm{f_N'-g}_{L^2}.
\end{split}
\]
Because $\norm{f_N'-g}_{L^2} \to 0$ as $N \to \infty$,
\[
\int_0^1f_N'(x+t(y-x))\cdot (y-x) dt
\to \int_0^1 g(x+t(y-x))\cdot (y-x) dt.
\]
Then by Lemma \ref{cramer}, taking $N \to \infty$, for any $y>x$ we have
\[
F(y)=F(x)+\int_0^1 g(x+t(y-x))\cdot (y-x) dt = F(x) + \frac{1}{y-x} \int_x^y g(s) ds.
\]
By the \textbf{Lebesgue differentiation theorem},
for almost all $x \in \mathbb{R}$,
\[
\frac{1}{y-x} \int_x^y g(s) ds \to g(x),\qquad y \to x.
\]
Therefore for almost all $x \in \mathbb{R}$,
\[
F'(x) = g(x).
\]
Thus $F'=g$ in $L^2(\lambda)$,
and as $f_N' \to g$ in $L^2(\lambda)$,
\begin{align*}
F' &= \lim_{N \to \infty} f_N' \\
&= \lim_{N \to \infty} \left(\frac{B-C}{2}\right)
\sum_{n=0}^N (f,h_n)_{L^2} h_n\\
&= \frac{1}{2}\sum_{n=0}^\infty  (f,h_n)_{L^2} ((2n)^{1/2}h_{n-1}-(2n+2)^{1/2}h_{n+1})\\
&= \frac{1}{2}\sum_{n=0}^\infty \bigg( (2n+2)^{1/2} (f,h_n)_{L^2}-(2n)^{1/2} (f,h_{n-1})_{L^2}\bigg) h_n,
\end{align*}
for which
\begin{align*}
\norm{F'}_{p-1}^2&=\frac{1}{4} \sum_{n=0}^\infty (2n+2)^{2p-2} \bigg| (2n+2)^{1/2} (f,h_n)_{L^2}-(2n)^{1/2} (f,h_{n-1})_{L^2}\bigg|^2\\
&\leq \frac{1}{2} \sum_{n=0}^\infty (2n+2)^{2p-2}\bigg((2n+2) |(f,h_n)_{L^2}|^2+2n |(f,h_{n-1})_{L^2}|^2\bigg),
\end{align*}
which is finite because $f \in \mathscr{S}_p$. Therefore $F' \in \mathscr{S}_{p-1}$. 
\end{proof}


\end{document}

