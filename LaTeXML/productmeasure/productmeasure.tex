\documentclass{article}
\usepackage{amsmath,amssymb,mathrsfs,amsthm}
%\usepackage{tikz-cd}
%\usepackage{hyperref}
\newcommand{\inner}[2]{\left\langle #1, #2 \right\rangle}
\newcommand{\tr}{\ensuremath\mathrm{tr}\,} 
\newcommand{\Span}{\ensuremath\mathrm{span}} 
\def\Re{\ensuremath{\mathrm{Re}}\,}
\def\Im{\ensuremath{\mathrm{Im}}\,}
\newcommand{\id}{\ensuremath\mathrm{id}} 
\newcommand{\var}{\ensuremath\mathrm{var}} 
\newcommand{\Lip}{\ensuremath\mathrm{Lip}} 
\newcommand{\GL}{\ensuremath\mathrm{GL}} 
\newcommand{\diam}{\ensuremath\mathrm{diam}} 
\newcommand{\sgn}{\ensuremath\mathrm{sgn}\,} 
\newcommand{\lcm}{\ensuremath\mathrm{lcm}} 
\newcommand{\supp}{\ensuremath\mathrm{supp}\,}
\newcommand{\dom}{\ensuremath\mathrm{dom}\,}
\newcommand{\upto}{\nearrow}
\newcommand{\downto}{\searrow}
\newcommand{\norm}[1]{\left\Vert #1 \right\Vert}
\theoremstyle{definition}
\newtheorem{theorem}{Theorem}
\newtheorem{lemma}[theorem]{Lemma}
\newtheorem{proposition}[theorem]{Proposition}
\newtheorem{corollary}[theorem]{Corollary}
\theoremstyle{definition}
\newtheorem{definition}[theorem]{Definition}
\newtheorem{example}[theorem]{Example}
\begin{document}
\title{Infinite product measures}
\author{Jordan Bell}
\date{May 10, 2015}

\maketitle

\section{Introduction}
The usual proof that the product of a collection of probability measures exists uses Fubini's theorem. This is unsatisfying because one ought not need to use Fubini's theorem
to prove things having only to do with $\sigma$-algebras and measures. In this note I work through the proof given by Saeki of the existence of the product of a collection
of probability measures.\footnote{Sadahiro Saeki, {\em A Proof of the Existence of Infinite Product Probability Measures}, Amer. Math. Monthly \textbf{103} (1996), no.~8, 682--682.}
 We  speak only about the Lebesgue integral of characteristic functions.

\section{Rings of sets and Hopf's extension theorem}
If $X$ is a set and $\mathscr{R}$ is a collection of subsets of $X$, we call $\mathscr{R}$ a \textbf{ring of sets} when
(i) $\emptyset \in \mathscr{R}$ and (ii) if $A$ and $B$ belong to $\mathscr{R}$ then $A \cup B$ and $A \setminus B$ belong to $\mathscr{R}$.
If $\mathscr{R}$ is a ring of sets and $A, B \in \mathscr{R}$, then $A \cap B = A \setminus (A \setminus B) \in \mathscr{R}$. Equivalently, one checks
that a collection of subsets $\mathscr{R}$ of $X$ is a ring of sets if and only if (i) $\emptyset \in \mathscr{R}$ and (ii) if
$A$ and $B$ belong to $\mathscr{R}$ then $A \triangle B$ and $A \cap B$ belong to $\mathscr{R}$, where
$A \triangle B = (A \setminus B) \cup (B \setminus A)$ is the \textbf{symmetric difference}. One checks that
indeed a ring of sets is a ring with addition $\triangle$ and multiplication $\cap$.
If $\mathscr{S}$ is a nonempty collection of subsets of $X$, one proves that there is a unique ring of sets $\mathscr{R}(\mathscr{S})$ 
that (i) contains $\mathscr{S}$  and (ii) is contained in any ring of sets that contains $\mathscr{S}$. We call $\mathscr{R}(\mathscr{S})$ the \textbf{ring
of sets generated by $\mathscr{S}$}.

If $\mathscr{A}$ is a ring of subsets of a set $X$, we call $\mathscr{A}$ an \textbf{algebra of sets} when $X \in \mathscr{A}$. Namely, an algebra of sets is a unital
ring of sets. If $\mathscr{S}$ is a nonempty collection of subsets of $X$, one proves that there is a unique algebra of sets
$\mathscr{A}(\mathscr{S})$ that (i) contains $\mathscr{S}$ and (ii) is contained in any algebra of sets that contains $\mathscr{S}$. We call
$\mathscr{A}(\mathscr{S})$ the \textbf{algebra of sets generated by $\mathscr{S}$}.

For a nonempty collection $\mathscr{G}$ of subsets of a set $X$, we denote by $\sigma(\mathscr{G})$ the smallest $\sigma$-algebra
of subsets of $X$ such that $\mathscr{G} \subset \sigma(\mathscr{G})$.

If $\mathscr{R}$ is a ring of subsets of a set $X$ and $\tau:\mathscr{R} \to [0,\infty]$ is a function such that
(i) $\mu(\emptyset)=0$ and (ii) when $\{A_n\}$ is a countable subset of $\mathscr{R}$ whose members are pairwise
disjoint and which satisfies $\bigcup_{n=1}^\infty A_n \in \mathscr{R}$, then
\[
\tau\left( \bigcup_{n=1}^\infty A_n \right) = \sum_{n=1}^\infty \tau(A_n),
\]
we call $\tau$ a \textbf{measure on $\mathscr{R}$}. The following is \textbf{Hopf's extension theorem}.\footnote{Karl Stromberg, {\em Probability for Analysts}, p.~52, Theorem A3.6.}


\begin{theorem}[Hopf's extension theorem]
Suppose that $X$ is a set, that $\mathscr{R}$ is a ring of subsets of $X$, and that $\tau$ is a measure
on $\mathscr{R}$. If there is a countable subset $\{E_n\}$ of $\mathscr{R}$ with $\tau(E_n)<\infty$ for each $n$ and such
that $\bigcup_{n=1}^\infty E_n = X$, then there is a unique measure $\mu:\sigma(\mathscr{R}) \to [0,\infty]$ whose
restriction to $\mathscr{R}$ is equal to $\tau$.
\end{theorem}



\section{Semirings of sets}
If $X$ is a set and $\mathscr{S}$ is a collection of subsets of $X$, we call $\mathscr{S}$ a \textbf{semiring of sets} when
(i) $\emptyset \in \mathscr{S}$, (ii) if $A$ and $B$ belong to $\mathscr{S}$ then $A \cap B \in \mathscr{S}$, and (iii)
if $A$ and $B$ belong to $\mathscr{S}$ then there are pairwise disjoint $C_1,\ldots,C_n \in \mathscr{S}$ such that
\[
A \setminus B = \bigcup_{i=1}^n C_i.
\]

If $\mathscr{S}$ is a semiring of subsets of a set $X$, we call $\mathscr{S}$ a \textbf{semialgebra of sets} when $X \in \mathscr{S}$. One proves
that if $\mathscr{S}$ is a semialgebra, then the collection $\mathscr{A}$ of all finite unions of elements of $\mathscr{S}$ is equal to the algebra
generated by $\mathscr{S}$, and that each element of $\mathscr{A}$ is equal to a finite union of pairwise disjoint elements of $\mathscr{S}$.\footnote{V. I. Bogachev, {\em Measure Theory}, volume I, p.~8, Lemma 1.2.14.} 

\section{Cylinder sets}
Suppose that $\{(\Omega_i,\mathscr{F}_i,P_i): i \in I\}$ is a nonempty collection of probability spaces and let
\[
\Omega = \prod_{i \in I} \Omega_i.
\]
If $A_i \in \mathscr{F}_i$ for each $i \in I$ and $\{i \in I: A_i \neq \Omega_i\}$ is finite,
we call
\[
A = \prod_{i \in I} A_i
\]
a \textbf{cylinder set}. Let $\mathscr{C}$ be the collection of all cylinder sets. 
One checks that $\mathscr{C}$ is a semialgebra of sets.\footnote{S. J. Taylor, {\em Introduction to Measure and Integration},
p.~136, \S 6.1, Lemma.}



\begin{lemma}
Suppose that $P:\mathscr{C} \to [0,1]$ is a function such that 
\[
\sum_{n=1}^\infty P(A_n)=1
\]
whenever $A_n$ are pairwise disjoint elements of $\mathscr{C}$ whose union is equal to $\Omega$. Then there is a unique probability measure
on $\sigma(\mathscr{C})$ whose restriction to $\mathscr{C}$ is equal to $P$. 
\label{cylinderextension}
\end{lemma}
\begin{proof}
Let $\mathscr{A}$ be the collection of all finite unions of cylinder sets. Because $\mathscr{C}$ is a semialgebra of sets, $\mathscr{A}$ is the algebra of sets generated
by $\mathscr{C}$, and any element of $\mathscr{A}$ is equal to a finite union of pairwise disjoint elements of $\mathscr{C}$. 
Let $A \in \mathscr{A}$.
There are pairwise disjoint $B_1,\ldots,B_j \in \mathscr{C}$ whose union is equal to $A$. Suppose also that
$\{C_i\}$ is a countable subset of  $\mathscr{C}$ with pairwise disjoint members whose union is equal to $A$.
Moreover, as $\Omega \setminus A \in \mathscr{A}$ there are pairwise disjoint $W_1,\ldots,W_p \in \mathscr{C}$ such that
$\Omega \setminus A = \bigcup_{i=1}^p W_i$.
On the one hand, $W_1,\ldots,W_p,B_1,\ldots,B_j$ are pairwise disjoint cylinder sets with union $\Omega$, so
\[
\sum_{i=1}^j P(B_i) + \sum_{i=1}^p P(W_i)=1.
\]
On the other hand, $W_1,\ldots,W_p,C_1,C_2,\ldots$ are pairwise disjoint cylinder sets with union $\Omega$, so
\[
\sum_{i=1}^\infty P(C_i) + \sum_{i=1}^p P(W_i)=1.
\]
Hence,
\[
\sum_{i=1}^j P(B_i) = \sum_{i=1}^\infty P(C_i);
\]
this conclusion does not involve $W_1,\ldots,W_p$. Thus it makes sense to define $\tau(A)$ to be this common value,
and this defines a function
$\tau:\mathscr{A} \to [0,1]$. For $C \in \mathscr{C}$, $\tau(C)=P(C)$, i.e. the restriction of $\tau$ to $P$ is equal to $\mathscr{C}$. 

If $\{A_n\}$ is a countable subset of $\mathscr{A}$ whose members are pairwise disjoint and
$A=\bigcup_{n=1}^\infty A_n \in \mathscr{A}$, 
for each $n$ let $C_{n,1}, \ldots, C_{n,j(n)} \in \mathscr{C}$ be pairwise disjoint cylinder sets with union
$A_n$. Then
\[
\{C_{n,i}: n \geq 1, 1 \leq i \leq j(n)\}
\]
 is a countable subset of $\mathscr{C}$ whose elements are pairwise
disjoint and with union $A$, so
\[
\tau(A) = \sum_{n=1}^\infty \sum_{i=1}^{j(n)} P(C_{n,i}).
\]
But for each $n$, 
\[
\tau(A_n) = \sum_{i=1}^{j(n)} P(C_{n,i}),
\]
so
\[
\tau(A) = \sum_{n=1}^\infty \tau(A_n).
\]
This shows that $\tau:\mathscr{A} \to [0,1]$ is a measure. 
Then applying Hopf's extension theorem, we get that there is a unique measure $\mu:\sigma(\mathscr{A}) \to [0,1]$ 
whose restriction to $\mathscr{A}$ is equal to $\tau$. 
It is apparent that the $\sigma$-algebra generated by a semialgebra is equal to the $\sigma$-algebra generated by the algebra generated
by the semialgebra, so $\sigma(\mathscr{A})=\sigma(\mathscr{C})$.  Because the restriction of $\tau$ to $\mathscr{C}$ is equal to
$P$, the restriction of $\mu$ to $\mathscr{C}$ is equal to $P$. Now suppose that
$\nu:\sigma(\mathscr{A}) \to [0,1]$ is a measure whose restriction to $\mathscr{C}$ is equal to $P$.
For $A \in \mathscr{A}$, there are disjoint $C_1,\ldots,C_n \in \mathscr{C}$ with $A=\bigcup_{i=1}^n C_i$. 
Then
\[
\nu(A) = \sum_{i=1}^n \nu(C_i) = \sum_{i=1}^n P(C_i) = \sum_{i=1}^n \mu(C_i)
=\mu(A),
\]
showing that the restriction of $\nu$ to $\mathscr{A}$ is equal to the restriction of $\mu$ to $\mathscr{A}$, from which it follows
that $\nu=\mu$.
\end{proof}



\section{Product measures}
Suppose that $\{(\Omega_i,\mathscr{F}_i,P_i): i \in I\}$ is a nonempty collection of probability spaces. The \textbf{product $\sigma$-algebra} 
is $\sigma(\mathscr{C})$, the $\sigma$-algebra generated by the cylinder sets. We define $P:\mathscr{C} \to [0,1]$ by
\[
P(A) = \prod_{i \in I_A} P_i(A_i) = \prod_{i \in I} P_i(A_i),
\]
for $A \in \mathscr{C}$ and with $I_A=\{i \in I: A_i \neq \Omega_i\}$, which is finite. 

\begin{lemma}
Suppose that $I$ is the set of positive integers. If $\{A_n\}$ is a countable subset of $\mathscr{C}$ with pairwise disjoint elements whose union is equal to $\Omega$, then
\[
\sum_{n=1}^\infty P(A_n)=1.
\]
\label{countable}
\end{lemma}
\begin{proof}
For each $k \geq 1$, there is some $i_k$ and $A_{k,1} \in \mathscr{F}_1, \ldots,A_{k,i_k} \in \mathscr{F}_{i_k}$ such that
\[
A_k = \prod_{i=1}^\infty A_{k,i},
\]
with $A_{k,i}=\Omega_i$ for $i>i_k$. 
Let $m \geq 1$, let $x=(x_i) \in A_m$, and let $n \geq 1$. If $n=m$,  
\[
\left( \prod_{i=1}^{i_m} \chi_{A_{n,i}}(x_i) \right) \left( \prod_{i>i_m} P_i(A_{n,i}) \right) 
=1=\delta_{m,n}.
\]
If $m \neq n$ and $y_i \in \Omega_i$ for each $i>i_m$ and we set
$y_i=x_i$ for $1 \leq i \leq i_m$,
then  because $A_m$ and $A_n$ are disjoint and $y \in A_m$,
we have $y \not \in A_n$ and therefore there is some $i$,
$1 \leq i \leq i_n$, such that $y_i \not \in A_{n,i}$. 
Thus
\begin{equation}
\left( \prod_{i=1}^{i_m} \chi_{A_{n,i}}(x_i) \right) \left( \prod_{i>i_m} \chi_{A_{n,i}}(y_i) \right) = 
\prod_{i=1}^\infty \chi_{A_{n,i}}(y_i) = 0.
\label{equals0}
\end{equation}
Either $i_n \leq i_m$ or $i_n>i_m$. In the case $i_n \leq i_m$ we have
$A_{n,i} = \Omega_i$ for $i > i_m$ and thus
\[
\left( \prod_{i=1}^{i_m} \chi_{A_{n,i}}(x_i) \right) \left( \prod_{i>i_m} \chi_{A_{n,i}}(y_i) \right) 
= \prod_{i=1}^{i_m} \chi_{A_{n,i}}(x_i),
\]
hence by \eqref{equals0},
\[
\left( \prod_{i=1}^{i_m} \chi_{A_{n,i}}(x_i) \right)  \left( \prod_{i>i_m} P_i(A_{n,i}) \right)
=
\prod_{i=1}^{i_m} \chi_{A_{n,i}}(x_i)
 = 0 = \delta_{m,n}.
\]
In the case $i_n>i_m$, we have $A_{n,i}=\Omega_i$ for $i>i_n$ and thus
\[
\left( \prod_{i=1}^{i_m} \chi_{A_{n,i}}(x_i) \right) \left( \prod_{i>i_m} \chi_{A_{n,i}}(y_i) \right)=
\left( \prod_{i=1}^{i_m} \chi_{A_{n,i}}(x_i) \right) 
\left( \prod_{i=i_m+1}^{i_n} \chi_{A_{n,i}}(y_i) \right),
\]
hence by \eqref{equals0} we have that for $y_i \in \Omega_i$, $i > i_m$,
\[
\left( \prod_{i=1}^{i_m} \chi_{A_{n,i}}(x_i) \right) 
\left( \prod_{i=i_m+1}^{i_n} \chi_{A_{n,i}}(y_i) \right)=0.
\]
Therefore, integrating over $\Omega_i$ for $i=i_m+1,\ldots,i_n$,
\[
\left( \prod_{i=1}^{i_m} \chi_{A_{n,i}}(x_i) \right)  \left( \prod_{i=i_m+1}^{i_n} P_i(A_{n,i}) \right)
=0,
\]
so
\[
\left( \prod_{i=1}^{i_m} \chi_{A_{n,i}}(x_i) \right)  \left( \prod_{i>i_m} P_i(A_{n,i}) \right)=0=
\delta_{m,n}.
\]
We have thus established that for any $m \geq 1$, $x \in A_m$, and $n \geq 1$,
\begin{equation}
\left( \prod_{i=1}^{i_m} \chi_{A_{n,i}}(x_i) \right)  \left( \prod_{i>i_m} P_i(A_{n,i}) \right)=
\delta_{m,n}.
\label{deltamn}
\end{equation}

Suppose by contradiction that
\[
\sum_{n=1}^\infty P(A_n) < 1,
\]
i.e.
\begin{equation}
\sum_{n=1}^\infty \prod_{i=1}^{\infty} P_i(A_{n,i})<1.
\label{neq1}
\end{equation}
If 
\[
\sum_{n=1}^\infty \chi_{A_{n,1}}(x_1) \prod_{i=2}^{\infty} P_i(A_{n,i}) = 1
\]
for all $x_1 \in \Omega_1$, then integrating over $\Omega_1$ we contradict \eqref{neq1}. Hence
there is some $x_1 \in \Omega_1$ such that
\[
\sum_{n=1}^\infty \chi_{A_{n,1}}(x_1) \prod_{i=2}^{\infty} P_i(A_{n,i}) <1.
\]
Suppose by induction that for some $j \geq 1$, $x_1 \in \Omega_1,\ldots,x_j \in \Omega_j$ and 
\[
\sum_{n=1}^\infty \left( \prod_{i=1}^j \chi_{A_{n,i}}(x_i) \right) \left(\prod_{i=j+1}^{\infty} P_i(A_{n,i})\right) <1.
\]
If
\[
\sum_{n=1}^\infty \left( \prod_{i=1}^{j+1} \chi_{A_{n,i}}(x_i) \right) \left(\prod_{i=j+2}^{\infty} P_i(A_{n,i})\right) =1
\]
for all $x_{j+1} \in \Omega_{j+1}$, then integrating over $\Omega_{j+1}$ we contradict
\eqref{neq1}. Hence there is some $x_{j+1} \in \Omega_{j+1}$ such that
\[
\sum_{n=1}^\infty \left( \prod_{i=1}^{j+1} \chi_{A_{n,i}}(x_i) \right) \left(\prod_{i=j+2}^{\infty} P_i(A_{n,i})\right) <1.
\]
Therefore, by induction we obtain that for any $j$, there are $x_1 \in \Omega_1,\ldots,x_j \in \Omega_j$ such that
\begin{equation}
\sum_{n=1}^\infty \left( \prod_{i=1}^j \chi_{A_{n,i}}(x_i) \right) \left(\prod_{i=j+1}^{\infty} P_i(A_{n,i})\right) <1.
\label{induction}
\end{equation}
Write $x=(x_1,x_2,\ldots) \in \Omega$. Because $\Omega= \bigcup_{m=1}^\infty A_m$, there is some $m$ for which
$x \in A_m$. For $j=i_m$, \eqref{induction} states
\[
\sum_{n=1}^\infty \left( \prod_{i=1}^{i_m} \chi_{A_{n,i}}(x_i) \right) \left(\prod_{i>i_m} P_i(A_{n,i})\right) <1.
\]
But \eqref{deltamn} tells us
\[
\sum_{n=1}^\infty \left( \prod_{i=1}^{i_m} \chi_{A_{n,i}}(x_i) \right)  \left( \prod_{i>i_m} P_i(A_{n,i}) \right)=
\sum_{n=1}^\infty \delta_{m,n} =1,
\]
a contradiction. Therefore, 
\[
\sum_{n=1}^\infty P(A_n) =1,
\]
proving the claim.
\end{proof}


\begin{lemma}
Suppose that $I$ is an uncountable set. 
If $\{A_n\}$ is a countable subset of $\mathscr{C}$ with pairwise disjoint elements whose union is equal to $\Omega$, then
\[
\sum_{n=1}^\infty P(A_n)=1.
\]
\end{lemma}
\begin{proof}
For each $n$,
there are $A_{n,i} \in \mathscr{F}_i$ with $A_{n,i} = \Omega_i$, and
$I_n = \{i \in I: A_i \neq \Omega_i\}$ is finite. 
Then $J=\bigcup_{n=1}^\infty I_n$ is countable. Let $\Omega_J=\prod_{i \in J} \Omega_i$, let
$\mathscr{C}_J$ be the collection of cylinder sets corresponding to the probability
spaces $\{(\Omega_i,\mathscr{F}_i,P_i): i \in J\}$, and define $P_J:\mathscr{C}_J \to [0,1]$ by 
\[
P_J(B) = \prod_{i \in J_B} P_i(B_i) = \prod_{i \in J} P_i(B_i),
\]
for $B \in \mathscr{C}_J$ and with $J_B=\{i \in J: B_i \neq \Omega_i\}$, which is finite. $P_J$ satisfies
\[
P_J(B) = P\left(B \times \prod_{i \in I \setminus J} \Omega_i \right), \qquad B \in \mathscr{C}_J. 
\]
Let $B_n=\prod_{i \in J} A_{n,i}$,
i.e. $A_n = B_n \times \prod_{i \in I \setminus J} A_{n,i}$. 
 Then $\{B_n\}$ is a countable subset of $\mathscr{C}_J$ with pairwise disjoint elements whose
union is equal to $\Omega_J$, and
applying Lemma \ref{countable} we get that 
\[
\sum_{n=1}^\infty P_J(B_n)=1,
\]
and therefore
\[
\sum_{n=1}^\infty P(A_n) = 1.
\]
\end{proof}

Now by Lemma \ref{cylinderextension} and the above lemma, there is a unique probability measure $\mu$ on
$\sigma(\mathscr{C})$ whose restriction to $\mathscr{C}$ is equal to $P$. That is, 
when $\{(\Omega_i,\mathscr{F}_i,P_i): i \in I\}$ are probability spaces and
$\mathscr{C}$ is the collection of cylinder sets corresponding to these probability spaces, with
$\Omega=\prod_{i \in I} \Omega_i$ and $P:\mathscr{C} \to [0,1]$ defined by
\[
P(A) = \prod_{i \in I} P(A_i)
\]
for $A = \prod_{i \in I} A_i \in \mathscr{C}$, 
then there is a unique probability
measure $\mu$ on the the product $\sigma$-algebra such that
$\mu(A) = P(A)$ for each cylinder set $A$. We call
$\mu$ the \textbf{product measure}, and write 
\[
\bigotimes_{i \in I} \mathscr{F}_i = \sigma(\mathscr{C})
\]
and
\[
\prod_{i \in I} P_i
=
\mu. 
\]



\end{document}