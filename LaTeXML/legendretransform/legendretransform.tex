\documentclass{article}
\usepackage{amsmath,amssymb,graphicx,subfig,mathrsfs,amsthm,siunitx}
%\usepackage{tikz-cd}
%\usepackage{hyperref}
\newcommand{\inner}[2]{\left\langle #1, #2 \right\rangle}
\newcommand{\tr}{\ensuremath\mathrm{tr}\,} 
\newcommand{\Span}{\ensuremath\mathrm{span}} 
\def\Re{\ensuremath{\mathrm{Re}}\,}
\def\Im{\ensuremath{\mathrm{Im}}\,}
\newcommand{\id}{\ensuremath\mathrm{id}} 
\newcommand{\rank}{\ensuremath\mathrm{rank\,}} 
\newcommand{\co}{\ensuremath\mathrm{co}\,} 
\newcommand{\cco}{\ensuremath\overline{\mathrm{co}}\,}
\newcommand{\supp}{\ensuremath\mathrm{supp}}
\newcommand{\epi}{\ensuremath\mathrm{epi}\,}
\newcommand{\lsc}{\ensuremath\mathrm{lsc}\,}
\newcommand{\ext}{\ensuremath\mathrm{ext}\,}
\newcommand{\cl}{\ensuremath\mathrm{cl}\,}
\newcommand{\dom}{\ensuremath\mathrm{dom}\,}
\newcommand{\LSC}{\ensuremath\mathrm{LSC}}
\newcommand{\USC}{\ensuremath\mathrm{USC}}
\newcommand{\extreals}{\overline{\mathbb{R}}}
\newcommand{\upto}{\nearrow}
\newcommand{\downto}{\searrow}
\newcommand{\norm}[1]{\left\Vert #1 \right\Vert}
\newtheorem{theorem}{Theorem}
\newtheorem{lemma}[theorem]{Lemma}
\newtheorem{proposition}[theorem]{Proposition}
\newtheorem{corollary}[theorem]{Corollary}
\theoremstyle{definition}
\newtheorem{definition}[theorem]{Definition}
\newtheorem{example}[theorem]{Example}
\begin{document}
\title{The Legendre transform}
\author{Jordan Bell}
\date{April 25, 2014}

\maketitle



\section{Convexity}
Write $\extreals=[-\infty,\infty]$. 
We define $\infty+\infty=\infty$, $-\infty-\infty=-\infty$, and $\infty-\infty$ is nonsense. If $a \in \mathbb{R}$, we define $a+\infty=\infty$ and
$a-\infty=-\infty$.

If $X$ is a vector space and $f:X \to \extreals$ is a function, we define the {\em epigraph of $f$} to be the set
\[
\epi f = \{(x,\alpha) \in X \times \mathbb{R}: \alpha \geq f(x)\},
\]
and
if $\epi f$ is a convex subset of the vector space $X \times \mathbb{R}$, we say that $f$ is {\em convex}. We
define the {\em effective domain} of a convex function $f$ to be
\[
\dom f = \{x \in X: f(x)<\infty\}.
\]
We say that a convex function $f:X \to \extreals$ is {\em proper} if $\dom f  \neq \emptyset$ and $f$ does not take the value $-\infty$.
If $C$ is a  convex subset of $X$ and $f:C \to \mathbb{R}$ is a function, we extend 
$f$ to $X$ by defining $f(x)=\infty$ for $x \in X \setminus C$. 

\begin{lemma}
If $X$ is a vector space, $C$ is a convex subset of $X$, and $f:C \to \mathbb{R}$ satisfies
\[
f((1-t)x_1+tx_2) \leq (1-t)f(x_1)+tf(x_2), \qquad x_1,x_2 \in C, \quad 0 \leq t \leq 1,
\]
then $f:X \to \extreals$ is convex.
\end{lemma}
\begin{proof}
Let $(x_1,\alpha_1),(x_2,\alpha_2) \in \epi f$ and $0 \leq t \leq 1$. The fact that the pairs
 $(x_i,\alpha_i)$ belong to $\epi f$ means in particular
that $f(x_i)<\infty$, and hence that $x_i \in C$, as otherwise we would have $f(x_i)=\infty$. 
But
\[
(1-t)(x_1,\alpha_1)+t(x_2,\alpha_2)=((1-t)x_1+tx_2,(1-t)\alpha_1+t\alpha_2),
\]
and, as $x_1,x_2 \in C$,
\[
f((1-t)x_1+tx_2) \leq (1-t)f(x_1)+tf(x_2) \leq (1-t)\alpha_1+t\alpha_2,
\]
showing that $(1-t)(x_1,\alpha_1)+t(x_2,\alpha_2) \in \epi f$, and hence that $f:X \to \extreals$ is convex.
\end{proof}


\section{Definition of the Legendre transform}
Let $V$ be a locally convex space and let $V^*$ be the dual space of $V$, i.e. the set of all continuous linear maps
$V \to \mathbb{R}$. With the weak-* topology, $V^*$ is itself a locally convex space and $V=(V^*)^*$, with the  isomorphism
of locally convex spaces $x \mapsto (\lambda \mapsto \lambda x)$.
If $f:V \to \extreals$ is a function, the {\em Legendre transform}  or  {\em convex conjugate} of $f$
is the function $f^*:V^* \to \extreals$ defined by
\[
f^*(\lambda) = \sup \{\lambda v - f(v):v \in V\} = \sup\{\lambda v-f(v):v \in \dom f\}.
\]
Like how the Fourier transform of a function from a locally compact abelian group to $\mathbb{C}$ is itself a function
from the Pontryagin dual of the group to $\mathbb{C}$, the Legendre transform of a function from a locally convex space to $\extreals$ is
itself a function from the dual space to $\extreals$.




\begin{theorem}
If $V$ is a locally convex space and $f:V \to \extreals$ is convex, then its Legendre transform
$f^*:V^* \to \extreals$ is convex.
\label{legendreconvex}
\end{theorem}
\begin{proof}
Let $(\lambda_1,\alpha_1),(\lambda_2,\alpha_2) \in \epi f^*$, and let $0 \leq t \leq 1$. 
We have
 \begin{eqnarray*}
f^*((1-t)\lambda_1+t\lambda_2)&=&\sup\{((1-t)\lambda_1+t\lambda_2)v-f(v):v \in \dom f\}\\
&=&\sup\{(1-t)(\lambda_1 v-f(v))+t(\lambda_2 v-f(v)):v \in \dom f\}\\
&\leq&\sup\{(1-t)(\lambda_1 v-f(v)): v\in \dom f\}\\
&&+\sup\{t(\lambda_2 v-f(v)):v \in \dom f\}\\
&=&(1-t)f^*(\lambda_1)+tf^*(\lambda_2)\\
&\leq&(1-t)\alpha_1+t\alpha_2,
 \end{eqnarray*}
 which means that $(1-t)(\lambda_1,\alpha_1)+t(\lambda_2,\alpha_2) \in \epi f^*$, and hence that $f^*$ is convex.
\end{proof}


\section{Lower semicontinuity}
If $X$ is a topological vector space and $f:X \to \extreals$ is a function, we say that $f$ is {\em lower semicontinuous} if $\epi f$ is a closed subset
of $X \times \mathbb{R}$. 

Theorem \ref{legendreconvex} shows that the Legendre transform of a convex function is itself convex. The following
lemma states that if a proper convex function is lower semicontinuous, then its Legendre transform is proper; one
proves the lemma using the Hahn-Banach separation theorem.\footnote{Viorel Barbu and Teodor Precupanu, {\em Convexity and Optimization in Banach Spaces},
fourth ed., p.~78, Corollary 2.21.}
 We
use this lemma in the proof of the theorem that comes after.

\begin{lemma}
If $V$ is a locally convex space and $f:V \to \extreals$ is a lower semicontinuous proper convex function, then its Legendre
transform $f^*:V^* \to \extreals$ is proper.
\label{properlemma}
\end{lemma}


\begin{theorem}
If $V$ is a locally convex space and $f:V \to \extreals$ is a lower semicontinuous proper convex convex, then $f^{**}=f$.
\end{theorem}
\begin{proof}
For any $\lambda \in V^*$ we have $f^*(\lambda)=\sup\{\lambda v  - f(v): v\in V\}$, and hence
for any $v \in V$ we have
$f^*(\lambda) \geq \lambda v - f(v)$.
Thus, for any $v \in V$ and $\lambda \in V^*$ we have
\[
\lambda v -f^*(\lambda) \leq f(v).
\]
Using this, for any $v \in V$ we have
\[
f^{**}(v)=\sup\{v\lambda - f^*(\lambda): \lambda \in \dom f^*\}
=\sup\{\lambda v-f^*(\lambda):\lambda \in \dom f^*\}
 \leq f(v).
\]

Suppose by contradiction that there were some $v_0 \in V$ for which $f^{**}(v_0)<f(v_0)$. First,  
by Lemma \ref{properlemma} we have that $f^*$ is proper, so in particular $\dom f^* \neq \emptyset$,
and this tells us that $f^{**}$ does not take the value $-\infty$. Hence $-\infty<f^{**}(v_0)<f(v_0)$, which tells us that $(v_0,f^{**}(v_0)) \in V \times \mathbb{R} \setminus  \epi f$.
Therefore, $\epi f$ and the singleton $\{(v_0,f^{**}(v_0))\}$ are disjoint closed convex sets ($\epi f$ is closed because
$f$ is lower semicontinuous), and so we can apply the Hahn-Banach separation
theorem: there is some $\Lambda \in (V \times \mathbb{R})^*$ and some $\gamma \in \mathbb{R}$ for which
\[
\Lambda (v,\alpha) < \gamma < \Lambda (v_0,f^{**}(v_0)), \qquad (v,\alpha) \in \epi f.
\]
As $\Lambda \in (V \times \mathbb{R})^*$, there is some $\lambda \in V^*$ and some $\beta \in \mathbb{R}^*=\mathbb{R}$ for which
$\Lambda (v,\alpha) = \lambda v + \beta \alpha$, and so 
\begin{equation}
\lambda v + \beta \alpha < \gamma < \lambda v_0 + \beta f^{**}(v_0), \qquad (v,\alpha) \in \epi f.
\label{beta}
\end{equation}
If $\beta>0$ then we get a contradiction because for a fixed $v \in \dom f$ there are arbitrarily large positive $\alpha$ for which
$(v,\alpha) \in \epi f$. Hence, $\beta \leq 0$. Assume by contradiction that $\beta=0$, and hence that
\begin{equation}
\lambda v < \gamma <\lambda v_0, \qquad v \in \dom f.
\label{beta0}
\end{equation}
 If $v_0 \in \dom f$ then we get
$\lambda v_0 < \gamma<\lambda v_0$, a contradiction. If $v_0 \not \in \dom f$, we shall still obtain a contradiction.
Let $\mu \in \dom f^*$. For any $h>0$,
\begin{eqnarray*}
f^*(\mu+h\lambda)&=&\sup\{(\mu+h\lambda)v-f(v):v \in \dom f\}\\
&=&\sup\{\mu v -f(v) + h\lambda v:v \in \dom f\}\\
&\leq&\sup\{\mu v-f(v): v\in \dom f\} + \sup\{h\lambda v: v \in \dom f\}\\
&=&f^*(\mu)+h \sup\{\lambda v: v \in \dom f\}.
\end{eqnarray*}
Therefore,
\begin{eqnarray*}
f^{**}(v_0)&\geq&(\mu+h\lambda)v_0-f^*(\mu+h\lambda)\\
&\geq&(\mu+h\lambda)v_0-f^*(\mu)-h \sup\{\lambda v: v \in \dom f\}\\
&=&\mu v_0 - f^*(\mu) + h\big(\lambda v_0 - \sup\{\lambda v: v \in \dom f\}\big).
\end{eqnarray*}
But by  \eqref{beta0} we have $\lambda v_0 - \sup\{\lambda v: v \in \dom f\}>0$, and
therefore the right-hand side of 
\[
f^{**}(v_0) \geq \mu v_0 - f^*(\mu) + h\big(\lambda v_0 - \sup\{\lambda v: v \in \dom f\}\big)
\]
can be  an arbitrarily large positive number (as $f^*(\mu) < \infty$), contradicting
that $f^{**}(v_0)<\infty$. Therefore, $\beta<0$, and dividing \eqref{beta} by $\beta$  then yields
\[
\frac{1}{\beta} \lambda v + \alpha> \frac{\gamma}{\beta} > \frac{1}{\beta} \lambda v_0 + f^{**}(v_0), \qquad
(v,\alpha) \in \epi f.
\]
Hence,
\begin{eqnarray*}
\left(-\frac{1}{\beta}\lambda\right)v_0-f^{**}(v_0)&>&\sup\left\{-\frac{1}{\beta}\lambda v -\alpha: (v,\alpha) \in \epi f\right\}\\
&=&\sup\left\{ -\frac{1}{\beta} \lambda v - f(v): v\in \dom f \right\}\\
&=&f^*\left(-\frac{1}{\beta} \lambda \right),
\end{eqnarray*}
which contradicts 
\[
\left(-\frac{1}{\beta}\lambda\right)v_0 -f^{**}(v_0) \leq f^*\left(-\frac{1}{\beta} \lambda \right).
\]
Therefore, there is no $v_0 \in V$ for which $f^{**}(v_0)<f(v_0)$, i.e., for all $v \in V$ we have
\[
f(v_0) \leq f^{**}(v_0).
\]
 \end{proof}
 
\section{Example in \textbf{R}\textsuperscript{{\em n}}}
Let $V:\mathbb{R}^n \to \mathbb{R}$ be a function,
 let $A$ be an $n \times n$ symmetric positive definite matrix, 
and define $L:T\mathbb{R}^n \to \mathbb{R}$ by
\[
L(q,v) = \frac{1}{2}\inner{v}{Av} - V(q).
\]
Fix any $q \in \mathbb{R}^n$,  let $X = T_q \mathbb{R}^n = \mathbb{R}^n$, 
and write $L_q(v)=L(q,v)$, for which $L_q:X \to \mathbb{R}$. The Legendre transform of $L_q$ is
$L_q^*:X^* \to \extreals$, defined by
\[
L_q^*(\lambda) = \sup\{\lambda v - L_q(v): v \in X\}.
\]
Because $A$ is symmetric, for any $v \in X$ we obtain
\[
D(\lambda  -L_q)(v) = \lambda - Av.
\]
Hence, $D(\lambda - L_q)(v)=0$ is equivalent to 
\[
v = A^{-1} \lambda,
\]
and  with this,
\[
L_q(A^{-1}\lambda) = \frac{1}{2}\inner{A^{-1}\lambda}{AA^{-1}\lambda}-V(q) = \frac{1}{2}\inner{A^{-1}\lambda}{\lambda}-V(q),
\]
and therefore
\[
L_q^*(\lambda) = \lambda(A^{-1}\lambda)- \frac{1}{2}\inner{A^{-1}\lambda}{\lambda}+V(q) = \frac{1}{2}\inner{A^{-1}\lambda}{\lambda}+V(q).
\]



\section{Derivatives}
Let $\Omega$ be a domain in $\mathbb{R}^n$ and let $f \in C^s(\Omega)$ for some $s \geq 2$. 
We define $\phi:\Omega \to \mathbb{R}^n$ by $\phi(x)=(D f)(x)$; we have $\phi \in C^{s-1}(\Omega,\mathbb{R}^n)$.
Following Giaquinta and Hildebrandt,\footnote{Mariano
Giaquinta and Stefan Hildebrandt, {\em Calculus of Variations II}, p.~6.} we call $\phi$ a {\em gradient mapping}. The following theorem gives conditions under which $\phi:\Omega \to \phi(\Omega)$ is 
  invertible.\footnote{Mariano
Giaquinta and Stefan Hildebrandt, {\em Calculus of Variations II}, p.~6, Lemma 1.} (To be {\em locally invertible}
means that for each point there is an open neighborhood such that the restriction of $\phi$ to that neighborhood is invertible.)

\begin{theorem}
If 
\[
\det (D^2 f)(x) \neq 0, \qquad x \in \Omega,
\]
then $\phi$ is locally invertible on $\Omega$. If $\Omega$ is convex and for all $x \in \Omega$ the matrix $(D^2 f)(x)$ is positive definite,
then $\phi:\Omega \to \phi(\Omega)$ is a $C^{s-1}$ diffeomorphism.
\end{theorem}
\begin{proof}
Because $\phi \in C^{s-1}(\Omega,\mathbb{R}^n)$ and $\det (D\phi)(x) = \det (D^2 f)(x) \neq 0$ for all $x \in \Omega$, by the inverse function theorem\footnote{Jerrold E.
Marsden and Michael J. Hoffman, {\em Elementary Classical Analysis}, second ed., p.~393, Theorem 7.1.1.} we have that
$\phi:\Omega \to \phi(\Omega)$ is a local $C^{s-1}$ diffeomorphism. 

Suppose that $\phi(x_1)=\phi(x_2)$ for some distinct $x_1,x_2 \in \Omega$. Put $x=x_2-x_1$.
Because $x_1,x_1+x=x_2 \in \Omega$ and $\Omega$ is convex, for any $0 \leq t \leq 1$ we have $x_1+tx \in \Omega$. Now define
\[
A(t) = (D^2 f) (x_1+tx),\qquad 0 \leq t \leq 1;
\]
because $f\in C^s(\Omega)$ with $s \geq 2$, we have that $A$ is continuous. Because
\[
\frac{d}{dt}\phi(x_1+tx) = (D\phi) (x_1+tx)x =(D^2 f)(x_1+tx)x= A(t)x,
\]
 we have
\begin{eqnarray*}
\inner{x}{\phi(x_2)-\phi(x_1)}&=&\inner{x}{\int_0^1 \frac{d}{dt}\phi(x_1+tx) dt}\\
&=&\inner{x}{\int_0^1 A(t) dt}\\
&=&\int_0^1 \inner{x}{A(t)x} dt.
\end{eqnarray*}
For each $0 \leq t \leq 1$ we have that $A(t)$ is a positive definite matrix, and because $x \neq 0$, this gives us that
$\inner{x}{A(t)x} > 0$. Moreover, $t \mapsto \inner{x}{A(t)x}$ is continuous, so it follows that
\[
\int_0^1 \inner{x}{A(t)x} dt > 0.
\]
But $\inner{x}{\phi(x_2)-\phi(x_1)}=0$, a contradiction. Therefore, $\phi$ is one-to-one.
It is a fact that a local diffeomorphism that is one-to-one is a diffeomorphism, thus $\phi:\Omega \to \phi(\Omega)$ is a $C^{s-1}$
diffeomorphism.
\end{proof}

Suppose that the gradient mapping $\phi:\Omega \to \phi(\Omega)$ is a $C^{s-1}$ diffeomorphism. We write $\psi=\phi^{-1}$, so $\psi:\phi(\Omega) \to \Omega$
is a $C^{s-1}$ diffeomorphism.
The following theorem gives an explicit expression for the Legendre transform of certain functions.\footnote{Mariano
Giaquinta and Stefan Hildebrandt, {\em Calculus of Variations II}, p.~9.}

\begin{theorem}
If $\Omega$ is a convex domain in $\mathbb{R}^n$, $f \in C^2(\Omega)$, 
and for all $x \in \Omega$ the matrix $(D^2 f)(x)$ is positive definite,
then
\[
f^*(\xi) = \xi  \psi(\xi) -f(\psi(\xi)), \qquad \xi \in \phi(\Omega).
\]
\label{legendreformula}
\end{theorem}
\begin{proof}
Fix $\xi \in \phi(\Omega)$ and define $g:\Omega \to \mathbb{R}$ by 
\[
g(x) = \xi x - f(x).
\]
We have $g \in C^2(\Omega)$, and we have $(Dg)(x) = \xi-(Df)(x)$ and $(D^2 g)(x) = -(D^2 f)(x)$. Thus, for each $x \in
\Omega$, the matrix
$(D^2 g)(x)$ is negative definite. It follows that if there is a point $x_0 \in \Omega$ at which
$(Dg)(x_0)=0$, then for all other $x \in \Omega$ we have $g(x) < g(x_0)$. To have $(Dg)(x_0)=0$ is equivalent
$(Df)(x_0)=\xi$, and because $\phi:\Omega \to \phi(\Omega)$ is a bijection, there is indeed a unique
$x_0 \in \Omega$ for which $(Df)(x_0)=\phi(x_0)=\xi$. Therefore,
\begin{eqnarray*}
f^*(\xi) &=& \sup\{\xi x-f(x): x \in \Omega\} \\
&=&\sup\{g(x):x \in \Omega\} \\
&=& g(x_0)\\
&=&\xi x_0-f(x_0)\\
&=&\xi \psi(\xi) - f(\psi(\xi)).
\end{eqnarray*}
\end{proof}



Using the above expression for the Legendre transform of a $C^2$ function with positive definite Hessian,
we show in the following theorem that the Legendre transform of a $C^s$ function with positive definite Hessian is itself a $C^2$ function.\footnote{Mariano
Giaquinta and Stefan Hildebrandt, {\em Calculus of Variations II}, p.~7, Lemma 2.}


\begin{theorem}
If $\Omega$ is a convex domain in $\mathbb{R}^n$, $f \in C^s(\Omega)$ with $s \geq 2$, and for all $x \in \Omega$ the matrix
$(D^2 f)(x)$ is positive definite, then $Df^*=\psi$ and $f^* \in C^s(\phi(\Omega))$.
\end{theorem}
\begin{proof}
For all $\xi \in \phi(\Omega)$ we have by Theorem \ref{legendreformula},
\[
f^*(\xi) = \xi  \psi(\xi) -f(\psi(\xi)).
\]
Thus,
\begin{eqnarray*}
(D f^*)(\xi) &=& \psi(\xi) +\xi (D \psi)(\xi) -(Df)(\psi(\xi)) (D\psi)(\xi)\\
&=&\psi(\xi) + \xi (D\psi)(\xi) - \phi(\psi(\xi)) (D\psi)(\xi)\\
&=&\psi(\xi) + \xi (D\psi)(\xi) -\xi(D\psi)(\xi)\\
&=&\psi(\xi).
\end{eqnarray*}
Hence $Df^*=\psi$. Because $\psi \in C^{s-1}(\phi(\Omega))$, it follows that $f \in C^s(\phi(\Omega))$.
\end{proof}

For all $x \in \Omega$, because $\psi(\phi(x))=x$ we have $(D\psi)(\phi(x)) (D\phi)(x) = I$, i.e. $(D^2 f^*)(\phi(x)) (D^2 f)(x) = I$, so
\[
(D^2f)(x)=((D^2 f^*)(\phi(x)))^{-1}.
\]
For all $\xi \in \phi(\Omega)$, because $\phi(\psi(\xi))=\xi$, we have $(D\phi)(\psi(\xi)) (D\psi)(\xi)=I$, i.e. $(D^2 f)(\psi(\xi)) (D^2 f^*)(\xi)=I$, so
\[
(D^2 f^*)(\xi) = ((D^2 f)(\psi(\xi)))^{-1}.
\]



\section{Example in \textbf{R}\textsuperscript{2}}
Suppose that $\Omega$ is a convex domain in $\mathbb{R}^2$, that $f \in C^2(\Omega)$, and that for all $x \in \Omega$, the matrix $(D^2 f)(x)$ is positive definite.
Write $\rho(x)=\det (D^2 f)(x)$; $\rho(x)>0$ for all $x \in \Omega$.  
Because $(D^2 f)(x) = ((D^2 f^*)(\phi(x)))^{-1}$ for all $x \in \Omega$, we have
\begin{eqnarray*}
\begin{pmatrix}
f_{11}(x)&f_{12}(x)\\
f_{21}(x)&f_{22}(x)
\end{pmatrix}
&=&\begin{pmatrix}
f^*_{11}(\phi(x))&f^*_{12}(\phi(x))\\
f^*_{21}(\phi(x))&f^*_{22}(\phi(x))
\end{pmatrix}^{-1}\\
&=&
\frac{1}{\det (D^2 f^*)(\phi(x))} \begin{pmatrix}f^*_{22}(\phi(x))&-f^*_{12}(\phi(x))\\
-f^*_{21}(\phi(x))&f^*_{11}(\phi(x))
\end{pmatrix}\\
&=&
\rho(x) \begin{pmatrix}f^*_{22}(\phi(x))&-f^*_{12}(\phi(x))\\
-f^*_{21}(\phi(x))&f^*_{11}(\phi(x)).
\end{pmatrix}
\end{eqnarray*}

Giaquinta and Hildebrandt\footnote{Mariano
Giaquinta and Stefan Hildebrandt, {\em Calculus of Variations II}, p.~14.} give the following consequence of what we have just written out.
If $f$ satisfies the above conditions and satisfies  the equation
\[
(1+f_2^2)f_{11} - 2f_1 f_2 f_{12} + (1+f_1^2)f_{22} = 2H(1+f_1^2+f_2^2)^{3/2},
\]
on $\Omega$, where $H$ is some constant, then 
\[
\begin{split}
&(1+f_2(x)^2) \rho(x) f^*_{22}(\phi(x))+2f_1(x)f_2(x)\rho(x)f^*_{12}(\phi(x))\\
&+(1+f_1(x)^2)\rho(x)f^*_{11}(\phi(x))\\
=&2H(1+f_1(x)^2+f_2(x)^2)^{3/2}
\end{split}
\]
for all $x \in \Omega$. Therefore,
\[
\begin{split}
&(1+\xi_2^2) \rho(\psi(\xi)) f^*_{22}(\xi) + 2\xi_1\xi_2 \rho(\psi(\xi)) f^*_{12}(\xi) +(1+\xi_1^2)\rho(\psi(\xi)) f^*_{11}(\xi)\\
=&2H(1+\xi_1^2+\xi_2^2)^{3/2}
\end{split}
\]
for all $\xi \in \phi(\Omega)$, $\xi=(\xi_1,\xi_2)$. In the case where $H=0$, then, dividing by $\rho(\psi(\xi))$, which is $>0$, we obtain
\[
(1+\xi_2^2) f^*_{22}(\xi) + 2\xi_1\xi_2  f^*_{12}(\xi) +(1+\xi_1^2) f^*_{11}(\xi)=0.
\]
In the case $H=0$, the equation satisfied by $f$ is called the {\em minimal surface equation}, and we have thus found a partial
differential equation satisfied by 
the Legendre transform of a solution of the minimal surface equation that satisfies the conditions we imposed at the start of the example.
Writing the equation satisfied by $f^*$ in the form
\[
A f^*_{11} + 2Bf^*_{12}+Cf^*_{22}=0,
\]
we have $A=(1+\xi_1^2)$, $B=\xi_1\xi_2$, $C=(1+\xi_2^2)$, with which
\[
B^2-AC = \xi_1^2\xi_2^2-(1+\xi_1^2)(1+\xi_2^2)=-1-\xi_1^2-\xi_2^2 \leq -1,
\]
which means that partial differential equation satisfied by $f^*$ is elliptic.


\section{Lagrangians and Hamiltonians}
Theorem \ref{legendreformula} states that if 
$\Omega$ is a convex domain in $\mathbb{R}^n$, $f \in C^2(\Omega)$, 
and for all $x \in \Omega$ the matrix $(D^2 f)(x)$ is positive definite, then
\[
f^*(\xi) = \xi  \psi(\xi) -f(\psi(\xi)), \qquad \xi \in \phi(\Omega).
\]
Suppose that $L:\mathbb{R}^n \times \mathbb{R}^n \times \mathbb{R}$ is a function
such that for each $q \in \mathbb{R}^n$ and $t \in \mathbb{R}$, the function $v \mapsto
L(q,v,t)$ satisfies the above conditions. Fix $q \in \mathbb{R}^n$ and $t \in \mathbb{R}$. 
With $DL=(\frac{\partial L}{\partial q},\frac{\partial L}{\partial v},\frac{\partial L}{\partial t})$ and
 $\phi(v) = \frac{\partial L}{\partial v} (q,v,t)$, $\psi=\phi^{-1}$, we have
 \[
 L^*(q,p,t) = p \psi(p)-L(q,\psi(p),t), \qquad p \in \phi(\mathbb{R}^n),
 \]
 or with $H=L^*$,
 \[
 H(q,p,t) = p\psi(p)-L(q,\psi(p),t), \qquad p \in \phi(\mathbb{R}^n).
 \]
We have
\[
\frac{\partial H}{\partial q}(q,p,t) = -\frac{\partial L}{\partial q}(q,\psi(p),t),
\]
and
\begin{eqnarray*}
\frac{\partial H}{\partial p}(q,p,t) &=& \psi(p)+p(D\psi)(p) - \frac{\partial L}{\partial v}(q,\psi(p),t) (D\psi)(p)\\
&=&\psi(p)+p(D\psi)(p)-\phi(\psi(p)) (D\psi)(p)\\
&=&\psi(p)+p(D\psi)(p)-p(D\psi)(p)\\
&=&\psi(p),
\end{eqnarray*}
and
\[
\frac{\partial H}{\partial t}(q,p,t) = -\frac{\partial L}{\partial t}(q,\psi(p),t).
\]

For a path $(q(t),v(t),t)$ to satisfy the
 {\em Euler-Lagrange equation} means that 
\[
\frac{d}{dt}\left( \frac{\partial L}{\partial v} (q(t),v(t),t) \right)  = \frac{\partial L}{\partial q} (q(t),v(t),t).
\]
With $p(t)=\phi(v(t))$, this yields
\[
\frac{dp}{dt}(t) = \frac{\partial L}{\partial q} (q(t),\psi(p(t)),t),
\]
and hence
\[
\frac{dp}{dt}(t) = -\frac{\partial H}{\partial q}(q(t),p(t),t).
\]





\section{Physical units}
First, if a Lagrangian $L$, $L(q,v,t)$, has units \si{J}, then the Hamiltonian $H=L^*$, $H(q,p,t)$, has the same units \si{J}, and it follows
that $p\psi(p)$ has units \si{J}.  Second, $[\psi(p)]=[v]$, and so $[H]=[p][\psi(p)]=[p][v]$. Therefore, $[p][v]=\si{J}$. If we take $[v]=\si{m/s}$, then
this implies that $[p]=\si{kg.m/s}$.

\section{More books}
V. I. Arnold,
{\em Mathematical Methods of Classical Mechanics}, second ed., p.~61, \S 14; 
Ralph Abraham and Jerrold E. Marsden, {\em Foundations of Mechanics}, second ed., p.~218, \S 3.6;
Jerrold E. Marsden and Tudor S. Ratiu, {\em Introduction to Mechanics and Symmetry}, second ed., p.~183, \S 7.2; 
J\"urgen Jost and Xianqing Li-Jost, {\em Calculus of Variations}, chapter 4; David Yang Gao, {\em Duality Principles in Nonconvex Systems: Theory, Methods and Applications}.



\section{History}
As best as I can tell, the thing we call the Legendre transform is named after Legendre because of the following paper:
Adrien-Marie Legendre, {\em M\'emoire sur l'int\'egration de quelques 
\'Equations aux diff\'erences partielles}, 
Histoire de l'Acad\'emie royale des sciences (1787), 309--351.
The following is a partial bibliography of works that refer to this paper of Legendre's. No
historical summary of the Legendre transform exists in the literature, and the following is presented as an aid to the preparation of one.
To properly  tell the story of the Legendre transform, one would be well served by 
carefully digging through
sources and attentively reading Legendre's original paper, and also by making oneself comfortable with how it appears in 
convex analysis, minimal surfaces, contact geometry, thermodynamics, etc.
Such a comprehensive history would require meticulously scanning through 
Legendre's monumental {\em Traite} on elliptic integrals lest
relevant material is included there. The best biography of Legendre that exists
is the one by Itard in the {\em Dictionary of Scientific Biography},
who mentions that something relevant to the Legendre transform appears in volume II
of the 1826 {\em Traite}, concerning arc lengths. One should also scan through the work of Lagrange, including
his  1788 {\em M\'echanique analitique}, and the work of Euler on the calculus of variations.

{\em Correspondance de Leonhard Euler avec A. C. Clairaut, J. d'Alembert et J. L. Lagrange},
pp.~440--441, Note 6; 
S. S. Demidov, {\em The study of partial differential equations of the first order in the 18th and 19th centuries},
Arch. Hist. Exact Sci. \textbf{26} (1982), no. 4, 325--350; Erwin Kreyszig, {\em On the Theory of Minimal Surfaces}, 
The Problem of Plateau (Themistocles M. Rassias, ed.), 1992, 138--164, p.~145;
Julian Lowell Coolidge, {\em A History of Geometrical Methods}, p.~377; Alfred Enneper,
{\em Bemerkungen \"uber einige Fl\"achen von constantem Kr\"ummungsmaa\ss}, Nachrichten von der K\"onigl. Gesellschaft der Wissenschaften und der Georg-Augusts-Universit\"at zu
G\"ottingen (1876), 597--619, p.~614; Alfred Enneper, {\em Ueber Fl\"achen mit besonderen Meridiancurven}, Abhandlungen der K\"oniglichen Gesellschaft der Wissenschaften in
G\"ottingen \textbf{29} (1882), 3--87, p.~6; Gaston Darboux, {\em Le\c{c}ons sur la th\'eorie g\'en\'erale des surfaces}, vol. 1, p.~271, \S 177;  \'Edouard Goursat,
{\em Le\c{c}ons sur l'int\'egration des \'equations aux d\'eriv\'ees partielles du second ordre, \`a deux variables ind\'ependantes}, tome 2, p.~32, chapter V, \S 113;
Ren\'e Taton, {\em L'{\oe}uvre scientifique de Monge}, p.~262; 
Karin Reich,
{\em Die Geschichte der Differentialgeometrie von Gau{\ss} bis Riemann (1828--1868)},
Arch. Hist. Exact Sci. \textbf{11} (1973), no.~4, 273--376, p.~315;
Ivor Grattan-Guinness, {\em Convolutions in French Mathematics, 1800-1840}, vol. I, p.~152; 
Morris Kline, {\em Mathematical Thought From Ancient to Modern Times},
chapter 22; Jo\~{a}o Caramalho Domingues, {\em Lacroix and the Calculus}, p.~223;
Ernst Hairer,
Syvert Paul N{\o}rsett and Gerhard Wanner,
{\em Solving Ordinary Differential Equations I: Nonstiff Problems}, p.~32;
Paul Mansion, {\em Th\'eorie des \'equations aux d\'eriv\'ees partielles du premier ordre}, p.~76; 
A. R. Forsyth, 
{\em A Treatise on Differential Equations}, sixth ed., pp.~418, 476; Lagrange, {\em M\'echanique analitique} (1788),
tome 1, partie 2, \S IV; Ernesto Pascal, {\em Die Variationsrechnung}, p.~125; Bernhard Riemann,
{\em Ueber die Fl\"ache vom kleinsten Inhalt bei gegebener Begrenzung}; Courant and Hilbert, vol. II; Cornelius Lanczos, {\em The Variational Principles of Mechanics},
fourth ed., \S VI.1; Ed. Combescure, {\em Remarques sur un M\'emoire de
Legendre}, Comptes rendus
hebdomadaires des s\'eances de l'Acad\'emie des sciences \textbf{74} (1872), 798--802;
Johannes C. C. Nitsche,
{\em Vorlesungen über Minimalflächen}, p.~147; 
A. W. Conway and J. L. Synge (ed.), {\em The Mathematical Papers of Sir William Rowan Hamilton}, vol. I, (1931), p.~474.

\end{document}
