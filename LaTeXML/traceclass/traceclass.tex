\documentclass{article}
\usepackage{amsmath,amssymb,graphicx,subfig,mathrsfs,amsthm}
%\usepackage{tikz-cd}
%\usepackage{hyperref}
\newcommand{\innerL}[2]{\langle #1, #2 \rangle_{L^2}}
\newcommand{\inner}[2]{\langle #1, #2 \rangle}
\newcommand{\HSinner}[2]{\left\langle #1, #2 \right\rangle_{\ensuremath\mathrm{HS}}}
\newcommand{\tr}{\ensuremath\mathrm{tr}\,} 
\newcommand{\Span}{\ensuremath\mathrm{span}} 
\def\Re{\ensuremath{\mathrm{Re}}\,}
\def\Im{\ensuremath{\mathrm{Im}}\,}
\newcommand{\id}{\ensuremath\mathrm{id}} 
\newcommand{\Hom}{\ensuremath\mathrm{Hom}}
\newcommand{\norm}[1]{\Vert #1 \Vert}
\newtheorem{theorem}{Theorem}
\newtheorem{lemma}[theorem]{Lemma}
\newtheorem{proposition}[theorem]{Proposition}
\newtheorem{corollary}[theorem]{Corollary}
\begin{document}
\title{Trace class operators and Hilbert-Schmidt operators}
\author{Jordan Bell}
\date{April 18, 2016}

\maketitle

\section{Introduction}
If $X,Y$ are normed spaces, let $\mathscr{B}(X,Y)$ be the set of all bounded linear maps $X \to Y$. If $T:X \to Y$ is a linear map,
I take it as known that $T$ is bounded if and only if it is continuous if and only if $E \subseteq X$ being bounded implies that $T(E) \subseteq Y$ is bounded.
I also take it as known that $\mathscr{B}(X,Y)$ is a normed space with the operator norm,
 that if $Y$ is a Banach space then $\mathscr{B}(X,Y)$ is a Banach space,  that if $X$ is a Banach space then $\mathscr{B}(X)=\mathscr{B}(X,X)$ is a Banach algebra, and that if $H$ is a Hilbert space then $\mathscr{B}(H)$ is a 
 $C^*$-algebra. An {\em ideal} $I$ of a Banach algebra is an ideal of the algebra: to say that $I$ is an ideal does not demand that $I$ is  a Banach subalgebra, i.e. does not demand
 that $I$ is a closed subset of the Banach algebra. $I$ is a  {\em $*$-ideal} of a $C^*$-algebra if $I$ is an ideal of the algebra and if $A \in I$ implies that $A^* \in I$.

If $X$ and $Y$ are normed spaces, we take as known that the {\em strong operator topology} on $\mathscr{B}(X,Y)$ is coarser than the norm topology on $\mathscr{B}(X,Y)$, and thus if $T_n \to T$
in the operator norm, then $T_n \to T$ in the strong operator topology.

If $X$ is a normed space, $M$ is a dense subspace of $X$, $Y$ is a Banach space and $T:M \to Y$ is a bounded linear operator, then there is a unique element of $\mathscr{B}(X,Y)$
whose restriction to $M$ is equal to $T$, and we also denote this  by $T$.\footnote{This is an instance of a result about topological vector spaces and
Fr\'echet spaces. See Walter Rudin, {\em Functional Analysis}, second ed., p.~40, chapter 1, ex. 19.} 

If $X$ is a normed space, define $\inner{\cdot}{\cdot}:X \times X^* \to \mathbb{C}$ by 
\[
\inner{x}{\lambda}=\lambda(x), \qquad x \in X, \lambda \in X^*.
\]
This is called the {\em dual pairing}. If $X$ and $Y$ are normed spaces and $T \in \mathscr{B}(X,Y)$, it can be proved that there is a unique $T^* \in \mathscr{B}(Y^*,X^*)$ such that
\[
\inner{Tx}{\lambda}=\inner{x}{T^*\lambda}, \qquad x \in X, \lambda \in Y^*,
\]
called the {\em adjoint} of $T$, and that the adjoint satisfies $\norm{T}=\norm{T^*}$.\footnote{Walter Rudin,
{\em Functional Analysis}, second ed., p.~ 98, Theorem 4.10.}

I give precise statements of any statement that I want to use without proof. If a fact is not straightforward to prove and is not easy to look up (perhaps because it
does not have a standardized name), I give a citation to a statement that is the precise version  I use.
I am making a point to write out full proofs of some tedious but essential arguments about Hilbert-Schmidt operators and trace class operators.\footnote{Many of the proofs that I give are vastly expanded from what is written in the references I used; I simply decided to write down every step that I did, and thus a reader should
be able to read this note without needing having to work out calculations on paper or without realizing partway through that I was tacitly identifying  things or that
what I said is true only under conditions I left unstated because I thought them too obvious.  Indeed there is no royal road through mathematics, but we do not need to break up
the asphalt and destroy the signage to make travelling what roads there are a trial of one's skill.}




\section{Finite rank operators}
In this section, $X$ and $Y$ denote Banach spaces.
We say that a linear map $T:X \to Y$ has {\em finite rank} if $T(X)$ is a finite dimensional subspace of $Y$. 
A finite rank operator need not be bounded:
If $X$ is infinite dimensional, let $\mathscr{E}$ be a Hamel basis for $X$ and  let $\{e_n:n \in \mathbb{N}\}$ be a subset of this basis. Defining a linear
map on a Hamel basis determines it on the vector space. Define $T:X \to \mathbb{C}$ by $Te_n=n \norm{e_n}$ and $Te=0$ if $e$ is not in the countable subset. Then $T$ is not bounded; but its range has dimension 1, so $T$ has finite rank. 
We define $\mathscr{B}_{00}(X,Y)$ to be the set of $T \in \mathscr{B}(X,Y)$ that have finite rank. It is apparent that $\mathscr{B}_{00}(X,Y)$ is a vector space.

If $f\in X^*$ and $y \in Y$, we define
$y \otimes f:X \to Y$ by
\[
y \otimes f(x)=f(x)y.
\]
$y \otimes f:X \to Y$ is linear, and
\[
\norm{y \otimes f} = \sup_{\norm{x} \leq 1} \norm{(y \otimes f)(x)} =
\sup_{\norm{x} \leq 1} \norm{f(x)y} = \norm{y} \sup_{\norm{x} \leq 1} |f(x)| \leq
\norm{y} \sup_{\norm{x} \leq 1} \norm{f} \norm{x},
\]
so $\norm{y \otimes f} \leq \norm{f} \norm{y}$, so $y \otimes f$ is bounded. $(y \otimes f)(H) \subseteq \Span\{y\}$, so $y \otimes f$ has finite rank.
Therefore $y \otimes f \in \mathscr{B}_{00}(X,Y)$.

The following theorem gives a representation for bounded finite rank operators.\footnote{Y. A. Abramovich and C. D. Aliprantis,
{\em An Invitation to Operator Theory}, p.~124, Lemma 4.2. In this reference, what we write as $y \otimes f$ they write as $f \otimes y$.} 

\begin{theorem}
If $T \in \mathscr{B}_{00}(X,Y)$ and $w_1,\ldots,w_k$ is a basis for $T(X)$, then there are unique $f_1,\ldots,f_k \in X^*$ such that
\[
T = \sum_{j=1}^k w_j \otimes f_j.
\]
\label{finiteranksum}
\end{theorem}


For $y \in Y$, define $F_y:Y^* \to \mathbb{C}$ by $F_y(\lambda)=\lambda(y)$. One checks that $F_y \in (Y^*)^*$. 
Often one writes 
$y=F_y$, which is fine as long as we keep in mind whether we are using $y$ as an element of $Y$ or as an element of $(Y^*)^*$. 

\begin{theorem}
If there are $w_1,\ldots,w_k \in Y$ and $f_1,\ldots,f_k \in X^*$ such that
\[
T = \sum_{j=1}^k w_j \otimes f_j,
\]
then
\[
T^* = \sum_{j=1}^k f_j \otimes w_j.
\]
\label{finiterankadjoint}
\end{theorem}
\begin{proof}
Let $y \in Y$ and $f \in X^*$. If $x \in X$ and $g \in Y^*$, then
\[
\inner{(y \otimes f)x}{g}=\inner{f(x)y}{g}=f(x)\inner{y}{g}=f(x)g(y),
\]
where $\inner{\cdot}{\cdot}:Y \times Y^* \to \mathbb{C}$  is the dual pairing. But
\[
f(x)g(y)=\inner{x}{g(y)f} = \inner{x}{F_y(g)f} = \inner{x}{(F_y \otimes f)(g)}.
\]
where $\inner{\cdot}{\cdot}:X \times X^* \to \mathbb{C}$  is the dual pairing. We have $F_y \otimes f \in \mathscr{B}(Y^*,X^*)$, and hence
\[
(y \otimes f)^*=F_y \otimes f=y \otimes f.
\]
This shows that the adjoint of each term $w_j \otimes f_j$ in $T$ is $f_j \otimes w_j$, and the adjoint of a sum is the sum of the adjoints of the terms, completing the proof.
\end{proof}

The above two theorems together show that if $T \in \mathscr{B}_{00}(X,Y)$ then $T^* \in \mathscr{B}_{00}(Y^*,X^*)$.


\begin{theorem}
If $X$ is a Banach space, then $\mathscr{B}_{00}(X)$ is a two sided ideal in the Banach algebra $\mathscr{B}(X)$. 
\end{theorem}
\begin{proof}
We have stated already that $\mathscr{B}_{00}(X,Y)$ is a vector space, and here $Y=X$. If $A \in \mathscr{B}_{00}(X)$ and $T \in \mathscr{B}(X)$, then $AT \in \mathscr{B}(X)$, and 
\[
A(T(X)) \subseteq A(X),
\]
which is finite dimensional, so $AT \in \mathscr{B}_{00}(X)$. $TA \in \mathscr{B}(X)$, and $T(A(X))$ is the image of a finite dimensional subspace under $T$, and so is itself finite dimensional. Hence $TA \in \mathscr{B}_{00}(X)$. 
\end{proof}






\section{Compact operators}
We say that a metric space $M$ is {\em totally bounded} if for every $\epsilon$ there are finitely many balls of radius $\epsilon$ whose union equals $M$.
The Heine-Borel theorem states that a metric space is compact if and only if it is complete and totally bounded. If $S$ is a subset of a complete metric space $M$ and
$\overline{S}$ is compact, then by the Heine-Borel theorem it is totally bounded, and any subset of a totally bounded metric space is itself a totally bounded metric space,
so $S$ is totally bounded. On the other hand, if $S \subseteq M$ is totally bounded, then one proves that $\overline{S}$ is also totally bounded. As $\overline{S}$ is a closed
subset of the complete metric space $M$, $\overline{S}$ is a complete metric space, and hence by the Heine-Borel theorem it is compact. 
 We say that a subset of a metric space is {\em precompact} if its closure is compact, and thus
a subset of a complete metric
space is precompact if and only if it is totally bounded.

Let $X$ and $Y$ be Banach spaces. If $T:X \to Y$ is a linear map and $U$ is the open unit ball in $X$, we say that $T$ is {\em compact} if the closure
of $T(U)$ in $Y$ is compact.
Therefore, to say that a linear map $T:X \to Y$ is compact is to say that $T(U)$ is totally bounded.

It doesn't take long to show that if $T:X \to Y$ is linear and compact then it is bounded, so there is no difference between stating that something is a bounded
compact operator and stating that it is a compact operator.
The following is often a convenient characterization of a compact operator.

\begin{theorem}
A linear map $T:X \to Y$ is compact if and only if for every bounded sequence $x_n \in X$ there is a subsequence
$x_{a(n)}$ such that $Tx_{a(n)}$ converges in $Y$.
\end{theorem}

We denote the set of compact operators $X \to Y$ by $\mathscr{B}_0(X,Y)$. It is apparent that $\mathscr{B}_0(X,Y)$ is a vector space.
For $T \in \mathscr{B}(X,Y)$, it is a fact that $T \in \mathscr{B}_0(X,Y)$  if and only if $T^* \in \mathscr{B}_0(Y^*,X^*)$.\footnote{Walter
Rudin, {\em Functional Analysis}, second ed., p.~105, Theorem 4.19.} 

The following theorem states if a sequence of compact operators converges to a bounded operator, then that operator is compact.\footnote{Walter
Rudin, {\em Functional Analysis}, second ed., p.~104, Theorem 4.18 (c).} Since $\mathscr{B}(X,Y)$ is a Banach space, this implies that $\mathscr{B}_0(X,Y)$ is a Banach space with the operator
norm.

\begin{theorem}
If $X$ and $Y$ are Banach spaces, then
$\mathscr{B}_0(X,Y)$ is a closed subspace of the Banach space $\mathscr{B}(X,Y)$.
\end{theorem} 



The following theorem shows that a bounded finite rank operator is a compact operator. Since a limit of compact operators is a compact operator, it follows from this that
a limit of bounded finite rank operators is a compact operator.


\begin{theorem}
If $T \in \mathscr{B}_{00}(X,Y)$ then $T \in \mathscr{B}_0(X,Y)$. 
\label{finiterankdensecompact}
\end{theorem}
\begin{proof}
Let $U$ be the open unit ball in $X$. Since $T$ is bounded  and $U$ is a bounded set in $X$, $T(U)$ is a bounded set in $Y$. But $T(X)$
is finite dimensional, so $T(U)$ is a bounded set in a finite dimensional vector space and hence by the Heine-Borel theorem,
 the closure of $T(U)$ in $T(X)$ is a compact subset of $T(X)$. $T(X)$ is finite
dimensional so is a closed subset of $Y$, and hence the closure of $T(U)$ in $T(X)$ is equal to the  closure  of $T(U)$ in $Y$. Hence the closure of $T(U)$ in $Y$
is a compact subset
of $Y$. (If $E$ is a topological space and $C \subseteq D \subseteq E$, then the subspace topology $C$ inherits from $E$ is the same as the subspace
topology it inherits from $D$, so if $K \subseteq T(X) \subseteq Y$ then to say that $K$ is compact in $T(X)$ is equivalent to saying that $K$ is compact in $Y$.)
\end{proof}

\begin{theorem}
If $X$ is a Banach space, then $\mathscr{B}_0(X)$ is a  two sided ideal in the Banach algebra $\mathscr{B}(X)$.
\end{theorem}
\begin{proof}
We have stated already that $\mathscr{B}_0(X,Y)$ is a closed subspace of $\mathscr{B}(X,Y)$, and here $Y=X$.
Let $K \in \mathscr{B}_0(X)$ and $T \in \mathscr{B}(X)$. On the one hand, if $x_n$ is a bounded sequence in $X$, then $\norm{Tx_n} \leq \norm{T} \norm{x_n}$, so
$Tx_n$ is a bounded sequence in $X$. Hence there is a subsequence $Tx_{a(n)}$ such that $K(Tx_{a(n)})=(KT)x_{a(n)}$ converges, showing that
$KT$ is compact.

On the other hand, if $x_n$ is a bounded sequence in $X$, then there is a subsequence $x_{a(n)}$ such that
$Kx_{a(n)}$ converges to some $x$. $T$ is continuous, so $T(Kx_{a(n)})$ converges  to $Tx$,
showing that $TK$ is compact.
\end{proof}

If $X$ is not separable, then
the image of the identity map $\id_X$ is not separable, so the image of a bounded linear operator need not be separable. However, the following theorem shows that the  image of a compact operator is separable. Check that if a subset of a metric space is separable then its closure
is separable. From this and Theorem \ref{separable}  we get that the closure of the image of a compact operator is separable.

\begin{theorem}
If $K \in \mathscr{B}_0(X,Y)$, then $K(X)$ with the subspace topology from $Y$ is separable.
\label{separable}
\end{theorem}
\begin{proof}
Let $U_n=\{x \in X: \norm{x} < n\}$. Then $\overline{K(U_n)}$ is compact. 
It is a fact that a compact metric space is separable, hence $\overline{K(U_n)}$ is separable. A subset of a separable metric space is itself separable with 
the subspace topology, so
 let $L_n$ be a countable dense subset of $K(U_n)$. 
Let $L=\bigcup_{n=1}^\infty L_n$, which is countable. It is straightforward to verify that $L$ is a dense subset of
\[
K(X) = \bigcup_{n=1}^\infty K(U_n).
\]
Therefore, $K(X)$ is separable.
\end{proof}






\section{Hilbert spaces}
We showed that if $X$ is a Banach space then both $\mathscr{B}_{00}(X)$ and $\mathscr{B}_0(X)$ are  two sided ideals in the Banach algebra $\mathscr{B}(X)$. We also showed
that if $A \in \mathscr{B}_{00}(X)$ then $A^* \in \mathscr{B}_{00}(X^*)$, and that if $A \in \mathscr{B}_0(X)$ then $A^* \in \mathscr{B}_0(X^*)$. If $H$ is a Hilbert space then $\mathscr{B}(H)$ is a $C^*$-algebra (as the adjoint
of $A \in \mathscr{B}(H)$ is not just an element of $\mathscr{B}(H^*)$, but can be identified with an element of $\mathscr{B}(H)$)
and what we have shown implies that $\mathscr{B}_{00}(H)$ and $\mathscr{B}_0(H)$ are two sided $*$-ideals in the $C^*$-algebra $\mathscr{B}(H)$.


If $S_\alpha, \alpha \in I$ are subsets of a Hilbert space $H$, we denote by $\bigvee_{\alpha \in I} S_\alpha$ the closure of the span
of $\bigcup_{\alpha \in I} S_\alpha$.
If $S_1,S_2$ are subsets of $H$, we write $S_1 \perp S_2$ if for every $s_1 \in S_1$ and
$s_2 \in S_2$ we have $\inner{s_1}{s_2}=0$.
If $V$ is a closed subspace of $H$, then $H = V \oplus V^\perp$, and the {\em orthogonal projection} onto $V$ is the map
$P:H \to H$ defined by $P(v+w)=v$ for $v \in V, w\in V^\perp$. $P(H)=V$, so rather than first fixing a closed subspace and then
talking about the orthogonal projection onto that subspace, one often speaks about an orthogonal projection, which is the orthogonal projection onto its image.
It is straightforward to check that if an orthogonal projection is nonzero then it has norm $1$, and
that an orthogonal projection is a positive operator.

The following theorem is an explicit version of Theorem \ref{finiteranksum} for orthogonal projections.

\begin{theorem}
Let $\{e_1,\ldots,e_n\}$ be orthonormal in $H$ and let $M=\bigvee_{k=1}^n \{e_j\}$. If $P$ is the orthogonal projection onto $M$, then
\[
P = \sum_{k=1}^n e_k \otimes e_k, \qquad Ph = \sum_{k=1}^n \inner{h}{e_k} e_k, \quad h \in H.
\]
\label{projectionsum}
\end{theorem}
\begin{proof}
 As $M$ is finite dimensional it is closed, and hence $H=M \oplus M^\perp$. Let $h = h_1 + h_2$, $h_1 \in M, h_2 \in M^\perp$; as
 $P$ is the orthogonal projection onto $M$, we have $Ph=h_1$.

Let $Qh = \sum_{k=1}^n \inner{h}{e_k}e_k$. We have to show that $Qh =Ph$. For each $1 \leq  j \leq n$, using that $\inner{e_k}{e_j}=\delta_{k,j}$ we get
\[
\inner{Qh}{e_j} = \sum_{k=1}^n \inner{h}{e_k}\inner{e_k}{e_j} = \inner{h}{e_j}. 
\]
Hence, if $1 \leq j \leq n$ then $\inner{h-Qh}{e_j}=0$. As $\{e_j\}$ are an orthonormal basis for $M$, this implies that
$h-Qh \perp M$, and so $h-Qh \in M^\perp$. That is,
\[
h_1+h_2-Qh \in M^\perp,
\]
and as $h_2 \in M^\perp$ it follows that $h_1 - Qh \in M^\perp$. But $h_1-Qh \in M$ ($h_1 \in M$ by definition, and $Qh$ is a sum of elements
in $M$), so $h_1-Qh=0$. As $Ph=h_1$, this means that $Ph=Qh$, completing the proof.
\end{proof}



We say that a Banach space $X$ has the {\em approximation property} if for each $A \in \mathscr{B}_0(X)$ there is a sequence $A_n \in \mathscr{B}_{00}(X)$ such that
$A_n \to A$. A result of Per Enflo shows that there are Banach spaces that do not have the approximation property. However, in the following theorem we show that
every Hilbert space has the approximation property.

\begin{theorem}
If $H$ is a Hilbert space and $A \in \mathscr{B}_0(H)$, then there is a sequence $A_n \in \mathscr{B}_{00}(H)$ such that $A_n \to A$.
\label{finiteranklimit}
\end{theorem}
\begin{proof}
By Theorem \ref{separable}, $V=\overline{A(H)}$ is separable. $V$ is a closed subspace of the Hilbert space $H$, so is itself a Hilbert space. If $V$ has finite dimension
then $A$ is itself finite rank. Otherwise,
let $\{e_n: n \geq 1\}$ be an orthonormal basis for $V$, and let $P_n$ be the orthogonal projection onto $\bigvee_{j=1}^n \{e_j\}$. 
 $P_n \in \mathscr{B}_{00}(H)$, and define $A_n=P_nA \in \mathscr{B}_{00}(H)$. 


In any Hilbert space, if $v_\alpha$ is an orthonormal basis then $\id_H = \sum_{\alpha} v_\alpha \otimes v_\alpha$, where the series converges in the strong operator
topology (see \S \ref{diagonalizable}). Thus, for any $h \in V$ we have
\[
Ah = \sum_{k=1}^\infty \inner{Ah}{e_k}e_k,
\]
where the series converges in $H$. By Theorem \ref{projectionsum},
\[
A_n h = P_nAh = \sum_{k=1}^n \inner{Ah}{e_k}e_k, 
\]
and therefore $\norm{A_n h -Ah} \to 0$ as $n \to \infty$. What we have shown is that $A_n \to A$ in the strong operator topology.

Let $B$ be the closed unit ball in $H$.
Because $A$ is a compact operator, by the Heine-Borel theorem $A(B)$ is totally bounded:  If $\epsilon>0$, then
  there is some $m$ and $h_1,\ldots,h_m \in
H$ such that
\[
A(B) \subseteq \bigcup_{j=1}^m B_{\epsilon}(Ah_j).
\]
If $h \in B$, there is some $1 \leq j \leq m$ with $Ah \in B_\epsilon(Ah_j)$, i.e. $\norm{Ah-Ah_j}<\epsilon$. If $n \geq 1$, then, as
$\norm{P_n} \leq 1$,
\begin{eqnarray*}
\norm{Ah-A_nh}&\leq&\norm{Ah-Ah_j}+\norm{Ah_j-A_nh_j}+\norm{A_nh_j-A_nh}\\
&=&\norm{Ah-Ah_j}+\norm{Ah_j-A_nh_j}+\norm{P_n(Ah_j-Ah)}\\
&\leq&2\norm{Ah-Ah_j}+\norm{Ah_j-A_nh_j}\\
&\leq&2\epsilon+\norm{Ah_j-A_nh_j}.
\end{eqnarray*}
As $A_n \to A$ in the strong operator topology, for each $1 \leq j \leq m$ there is some $N(j)$ such that if $n \geq N(j)$ then
$\norm{Ah_j-A_nh_j}<\epsilon$ and so, if $n \geq N(j)$,
\[
\norm{Ah-A_nh}<3\epsilon.
\]
Let $N=\max_{1 \leq j \leq m} N(j)$, whence for all $h \in B$, if $n \geq N$ then
\[
\norm{Ah-A_nh}<3\epsilon.
\]
But
\[
\norm{A-A_n}=\sup_{\norm{h} \leq 1} \norm{(A-A_n)h},
\]
so if $n \geq N$ then
\[
\norm{A-A_n} < 3\epsilon,
\]
showing that $A_n \to A$.
\end{proof}




\section{Diagonalizable operators}
\label{diagonalizable}
If $\mathscr{E}$ is an orthonormal set in a Hilbert space $H$, which  we do not demand  be separable, then $\mathscr{E}$ is an orthonormal basis for $H$ if and only if for every $h \in H$ we have 
$h=\sum_{e \in \mathscr{E}} \inner{h}{e}e$.\footnote{John B. Conway, {\em A Course in Functional Analysis}, second ed., p.~16, Theorem 4.13.}
In other words, if $\mathscr{E}$ is an orthonormal set in $H$, then $\mathscr{E}$ is an orthonormal basis for $H$ if and only if
\[
\id_H = \sum_{e \in \mathscr{E}} e \otimes e,
\]
where the series converges in the strong operator topology. 


We say that a linear map $A:H \to H$ is {\em diagonalizable} if there is an orthonormal basis of $H$ each element of which is an eigenvector of $A$. 
If $A$ is a bounded linear operator on $H$, $A$ is diagonalizable if and only
if there is an orthonormal basis $e_i, i \in I$, of $H$ and
$\lambda_i \in \mathbb{C}$ such that the series
\[
\sum_{i \in I} \lambda_i e_i \otimes e_i
\]
converges to $A$ in the strong operator topology.  One checks that the series $\sum_{i \in I} \overline{\lambda_i} e_i \otimes e_i$ converges to $A^*$ in the strong
operator topology.

If
$A \in \mathscr{B}(H)$ is diagonalizable with eigenvalues $\{\lambda_i: i \in I\}$, it is a fact that
\begin{equation}
\norm{A}=\sup_{i \in I} |\lambda_i|.
\label{diagonalnorm}
\end{equation}

\begin{theorem}
Let  $H$ be a Hilbert space with orthonormal basis $\{e_i: i \in I\}$, let $\lambda_i \in \mathbb{C}$,
and define a linear map  $A:\Span\{e_i: i \in I\} \to \Span\{e_i: i \in I\}$ by $Ae_i = \lambda_i e_i$.  
If $\sup_{i \in I} |\lambda_i| < \infty$, then $A$ extends to a unique element of $\mathscr{B}(H)$.
\end{theorem}
\begin{proof}
Let $M=\sup_{i \in I} |\lambda_i| < \infty$.
If $J$ is a finite subset of $I$ and $x=\sum_{i \in J} \alpha_i e_i$, then, as the $e_i$ are orthonormal,
\[
\norm{Ax}^2=\norm{\sum_{i \in J} \alpha_i Ae_i}^2 = \norm{\sum_{i \in J} \alpha_i \lambda_i e_i}^2
=\sum_{i \in J} \norm{\alpha_i \lambda_i e_i}^2 \leq M^2 \sum_{i \in J} |\alpha_i|^2.
\]
But
\[
\norm{x}^2 = \norm{\sum_{i \in J} \alpha_i e_i}^2 =\sum_{i \in J} \norm{\alpha_i e_i}^2= \sum_{i \in J} |\alpha_i|^2.
\]
So
\[
\norm{Ax} \leq M \norm{x}.
\]
It follows that $A$ is a bounded operator on $\Span\{e_i:i \in I\}$,
which is dense in $H$. Then there is a unique element of $\mathscr{B}(H)$ whose
restriction to $\Span\{e_i: i \in I\}$ is equal to $A$, and we denote this element of $\mathscr{B}(H)$ by $A$.
\end{proof}




In Theorem \ref{separable} we proved that the image of a compact operator is separable. Hence if a compact operator is diagonalizable then it has only countably many nonzero eigenvalues.

\begin{theorem}
If $H$ is a separable Hilbert space with orthonormal basis $e_n, n \geq 1$, and if $A:H \to H$ is linear and
$Ae_n=\lambda_n e_n$ for all $n$,
then $A \in \mathscr{B}_0(H)$ if and only if $\lambda_n \to 0$ as $n \to \infty$.
\end{theorem}
\begin{proof}
Suppose that $A$ is compact. Then $A^*$ is compact, and as $A$ is diagonalizable so is $A^*$. The proof of Theorem \ref{finiteranklimit} wasn't generally useful enough to be worth putting
into a lemma, but to understand the following it will be necessary to read that proof.
Let $\lambda_{a(n)}$ be the nonzero eigenvalues of $A$, and let $\mu_n=\lambda_{a(n)}$, $f_n=e_{a(n)}$. Check that $f_n, n \geq 1$ is an orthonormal basis for $\overline{A^*(H)}$,
 $P_n$ be the projection onto $\bigvee_{j=1}^n \{f_j\}$.
 Then using the argument in the proof of Theorem \ref{finiteranklimit} we get $P_n A^* \to A^*$; this completes our trip back to that proof.  
Let $A_n = A-AP_n$. As $\norm{P_nA^*-A^*} \to 0$ as $n \to \infty$,
\begin{eqnarray*}
\norm{A_n}&=&\norm{(A-AP_n)^*}\\
&=&\norm{A^*-P_n^*A^*}\\
&=&\norm{A^*-P_n A^*}\\
&\to&0.
\end{eqnarray*}
If $1 \leq j \leq n$ then $A_n f_j = Af_j-Af_j = 0$, and if $j > n$ then
$A_n f_j = Af_j=\mu_j f_j$. Hence $A_n$ is a diagonalizable operator, and by \eqref{diagonalnorm} we get $\norm{A_n}=\sup_{j>n} |\mu_j|$. 
Together with $\norm{A_n} \to 0$ as $n \to \infty$, this means that $\limsup_{n \to \infty} |\mu_n| = 0$. As $\mu_n$ are precisely the nonzero $\lambda_n$, we obtain from this
that
$\lim_{n \to \infty} |\lambda_n|=0$.

On the other hand, suppose that $\lambda_n \to 0$ as $n \to \infty$.
Because $\lambda_n \to 0$, the absolute values of the eigenvalues of $A$ are bounded and hence 
$A \in \mathscr{B}(H)$. 
Let $P_n$ be the projection onto
$\bigvee_{j=1}^n \{e_j\}$. If $1 \leq j \leq n$ then $AP_n e_j =Ae_j=\lambda_j e_j$, and if
$j>n$ then $AP_n e_j = 0$. Therefore $AP_n \in \mathscr{B}_{00}(H)$. For $A_n=A-AP_n$, we have
$\norm{A_n} = \sup_{j>n} |\lambda_j|$, from which it follows that
\[
\lim_{n \to \infty} \norm{A_n} =0.
\]
Hence $AP_n \to A$, and as $AP_n$ are bounded finite rank operators, $A$ is a compact operator.
\end{proof}



If $A \in \mathscr{B}(H)$ is diagonalizable, say $Ae_i = \lambda_i e_i$, $i \in I$, then $A^*e_i = \overline{\lambda_i} e_i$, $i \in I$, and for $j \in I$,
\[
AA^*e_j=A\sum_{i \in I} \overline{\lambda_i} \inner{e_j}{e_i} e_i
=A \left( \overline{\lambda_j} e_j \right)= \overline{\lambda_j} \lambda_j e_j = |\lambda_j|^2 e_j.
\]
Likewise we get $A^*Ae_j=|\lambda_j|^2 e_j$, so $AA^*=A^*A$, that is, a bounded diagonalizable operator is  {\em normal}.
The following theorem states an implication in the other direction.\footnote{Gert K. Pedersen, {\em Analysis Now}, revised printing, p.~108, Theorem 3.3.8.}
This is an instance of the {\em spectral theorem}.

\begin{theorem}
If $H$ is a Hilbert space over $\mathbb{C}$ and $T$ is a normal compact operator on $H$, then $T$ is diagonalizable.
\label{spectraltheorem}
\end{theorem}





\section{Polar decomposition}
It is a fact that
$A \in \mathscr{B}(H)$ is self-adjoint if and only if
 $\inner{Ax}{x} \in \mathbb{R}$ for all $x \in H$.
$A \in \mathscr{B}(H)$ is said to be {\em positive} if $A$ is self-adjoint and $\inner{Ax}{x} \geq 0$ for all $x \in H$.
It is a fact that if $A \in \mathscr{B}(H)$ is positive then there is a unique positive element of $\mathscr{B}(H)$, call it $A^{1/2}$, such that
$(A^{1/2})^2=A$; namely, a bounded positive operator has a unique positive square root in $\mathscr{B}(H)$.\footnote{Gert K. Pedersen, {\em Analysis Now}, revised printing, p.~92, Proposition 3.2.11.}
It is straightforward to check that if $A \in \mathscr{B}(H)$ then $A^*A$  is positive, and hence has a square root $(A^*A)^{1/2} \in \mathscr{B}(H)$, which we denote
by $|A|$. $|A|$ satisfies $\norm{Ax}=\norm{|A|x}$ for all $x \in H$. 


An {\em isometry} from one Hilbert space to another is a linear map $A:H_1 \to H_2$ such that if $x \in H_1$ then $\norm{Ax}=\norm{x}$. 
If $A:H \to H $ is linear and the restriction of $A$ to $(\ker A)^\perp$ is an isometry, then we say that $A$ is a {\em partial isometry}.
One checks that a partial isometry is an element
of $\mathscr{B}(H)$,  if it is not the zero map then it has norm $1$, and that its image is closed.
We call
$(\ker A)^\perp$ the {\em initial space} of $A$, and $A(H)$ the {\em final space} of $A$. We can prove that
if $A$ is a partial isometry then $A^*$ is a partial isometry whose initial space is the final space
of $A$ and whose final space is the initial space of $A$.
For example, an orthogonal projection is a partial isometry whose initial space is the image of the orthogonal projection and whose final
space is equal to its initial space.


It is a fact that if $A \in \mathscr{B}(H)$ then there is a partial isometry $U$
with $\ker U = \ker A$ satisfying $A=U|A|$, and that if $V$ is a partial isometry with $\ker V = \ker A$ that satisfies
$A=V|A|$, then $V=U$.\footnote{Gert K. Pedersen, {\em Analysis Now}, revised printing, p.~96, Theorem 3.2.17.} $A=U|A|$ is called the {\em polar decomposition} of $A$. The polar decomposition satisfies
\begin{equation}
U^*U|A|=|A|, \quad U^*A=|A|, \quad UU^*A=A, \quad A^*=|A|U^*, \quad |A^*|=U|A|U^*.
\label{polardecomposition}
\end{equation}
We will use these formulas repeatedly when we are working with trace class operators, and we have numbered the above
equation to refer to it and also to draw the eye to it.


If $I$ is an ideal of $\mathscr{B}(H)$ and $A \in I$ then $|A|=U^*A \in I$. In particular, if $A \in \mathscr{B}_{00}(H)$ then $|A| \in \mathscr{B}_{00}(H)$ and if
$A \in \mathscr{B}_0(H)$ then $|A| \in \mathscr{B}_0(H)$.




\section{Hilbert-Schmidt operators}
Let $H$ be a Hilbert space, and $\{e_i: i \in I\}$ an orthonormal basis of $H$. We say that $A \in \mathscr{B}(H)$ is a {\em Hilbert-Schmidt}
operator if 
\[
\sum_{i \in I} \norm{Ae_i}^2 < \infty.
\]
The following theorem shows that if $A \in \mathscr{B}(H)$ is a Hilbert-Schmidt operator using one basis it will also be one using any other basis, that  if $A$ is not a Hilbert-Schmidt operator using one basis it will not be one using
any other basis, and that $A$ is Hilbert-Schmidt if and only if its adjoint $A^*$ is Hilbert-Schmidt.

\begin{theorem}
Let $H$ be a  Hilbert space. If $\{e_\alpha: \alpha \in I\}$ and $\{f_\alpha: \alpha \in J\}$ are orthonormal bases for $H$ and $A \in \mathscr{B}(H)$, then
\[
 \sum_{\alpha \in I} \norm{Ae_\alpha}^2 = \sum_{\alpha \in J} \norm{A^*f_\alpha}^2= \sum_{\alpha \in J} \norm{Af_\alpha}^2.
 \]
 \label{basisindependent}
\end{theorem}
\begin{proof}
For each $\beta \in J$, using Parseval's identity we have
\[
Af_\beta = \sum_{\alpha \in J} \inner{Af_\beta}{f_\alpha}f_\alpha, \qquad \norm{Af_\beta}^2=\sum_{\alpha \in J} |\inner{Af_\beta}{f_\alpha}|^2.
\]
Using this we get 
\begin{eqnarray*}
\sum_{\beta \in J} \norm{Af_\beta}^2&=&\sum_{\beta \in J} \sum_{\alpha \in J} |\inner{Af_\beta}{f_\alpha}|^2\\
&=&\sum_{\alpha \in J} \sum_{\beta \in J} |\inner{Af_\beta}{f_\alpha}|^2\\
&=&\sum_{\alpha \in J} \sum_{\beta \in J} |\inner{f_\beta}{A^*f_\alpha}|^2\\
&=&\sum_{\alpha \in J} \sum_{\beta \in J} |\inner{A^*f_\alpha}{f_\beta}|^2\\
&=&\sum_{\alpha \in J} \norm{A^* f_\alpha}^2.
\end{eqnarray*}

For each $\beta \in I$, Parseval's identity gives us
\[
\norm{Ae_\beta}^2=\sum_{\alpha \in J} |\inner{Ae_\beta}{f_\alpha}|^2,
\]
and using this we obtain
\begin{eqnarray*}
\sum_{\beta \in I} \norm{Ae_\beta}^2&=&\sum_{\beta \in I} \sum_{\alpha \in J} |\inner{Ae_\beta}{f_\alpha}|^2\\
&=&\sum_{\alpha \in J} \sum_{\beta \in I} |\inner{Ae_\beta}{f_\alpha}|^2\\
&=&\sum_{\alpha \in J} \sum_{\beta \in I}|\inner{e_\beta}{A^*f_\alpha}|^2\\
&=&\sum_{\alpha \in J} \sum_{\beta \in I}|\inner{A^*f_\alpha}{e_\beta}|^2\\
&=&\sum_{\alpha \in J} \norm{A^* f_\alpha}^2.
\end{eqnarray*}
\end{proof}


Let $\mathscr{B}_2(H)$ be the set of Hilbert-Schmidt operators in $\mathscr{B}(H)$.
If $\{e_i:i \in I\}$ is an orthonormal basis for $H$ and $A \in \mathscr{B}(H)$, we define
\[
\norm{A}_2 = \left( \sum_{ i \in I} \norm{Ae_i}^2\right)^{1/2}.
\]
To say that $A \in \mathscr{B}_2(H)$ is to say that $A \in \mathscr{B}(H)$ and $\norm{A}_2 < \infty$. 
Using the triangle inequality in $\ell^2(I)$,
one checks that $\mathscr{B}_2(H)$ is a vector space and that $\norm{\cdot}_2$ is a norm on $\mathscr{B}_2(H)$, which we call
the {\em Hilbert-Schmidt norm}.

Because $\norm{|A|x}=\norm{Ax}$ for all $x \in H$, $A$ is Hilbert-Schmidt if and only if $|A|$ is Hilbert-Schmidt, and
$\norm{A}_2 = \norm{|A|}_2$.
From Theorem \ref{basisindependent} we obtain that if $A \in \mathscr{B}_2(H)$ then $A^* \in \mathscr{B}_2(H)$, and 
$\norm{A}_2 = \norm{A^*}_2$.


\begin{theorem}
$B_2(H)$ is a two sided $*$-ideal in the $C^*$-algebra $\mathscr{B}(H)$, and if $A \in \mathscr{B}_2(H)$ and $T \in \mathscr{B}(H)$ then
\[
\norm{AT}_2 \leq \norm{A}_2 \cdot \norm{T}, \qquad \norm{TA}_2 \leq \norm{T} \cdot \norm{A}_2.
\]
\label{HSideal}
\end{theorem}
\begin{proof}
Let $\mathscr{E}$ be an orthonormal basis for $H$.
If $A \in \mathscr{B}_2(H)$ and $T \in \mathscr{B}(H)$, then
\[
\norm{TA}_2^2=\sum_{e \in \mathscr{E}} \norm{TAe}^2 \leq \sum_{e \in \mathscr{E}} \left( \norm{T} \norm{Ae}\right)^2 = \norm{T}^2\norm{A}_2^2 < \infty.
\]
On the other hand, using the above and the fact that if $A$ is Hilbert-Schmidt then $A^*$ is Hilbert-Schmidt,
\[
\norm{AT}_2^2 = \norm{(AT)^*}_2^2 = \norm{T^*A^*}_2^2 = \norm{T^*}^2 \norm{A^*}_2^2 < \infty.
\]
\end{proof}


The following theorem shows that the operator norm is dominated by the Hilbert-Schmidt norm, and therefore that the topology on the normed space
$\mathscr{B}_2(H)$ with the Hilbert-Schmidt norm is finer than the subspace topology it inherits from $\mathscr{B}(H)$ (i.e. its topology as a normed space with the operator norm).


\begin{theorem}
If $A \in \mathscr{B}_2(H)$ then $\norm{A} \leq \norm{A}_2$.
\label{HSdominated}
\end{theorem}
\begin{proof}
Let $\epsilon>0$. We have
\[
\norm{A} = \sup_{\norm{v} = 1} \norm{Av}.
\]
Take $v \in H$, $\norm{v}=1$, with $\norm{Av}^2+\epsilon>\norm{A}^2$. There
 is an orthonormal basis $\{e_\alpha: \alpha \in I\}$ for $H$ that includes
$v$; one proves this using Zorn's lemma. Then
\[
\norm{A}^2 <\epsilon+ \norm{Av}^2 \leq \epsilon + \norm{Av}^2 + \sum_{e_\alpha \neq v} \norm{Ae_\alpha}^2
=\epsilon+\sum_{\alpha \in I} \norm{Ae_\alpha}^2 = \epsilon + \norm{A}_2^2.
\]
As this is true for all $\epsilon>0$, it follows that $\norm{A} \leq \norm{A}_2$.
\end{proof}


The following theorem states that every bounded finite rank operator is a bounded Hilbert-Schmidt operator, and that every bounded Hilbert-Schmidt operator
is the limit in the Hilbert-Schmidt norm  of a sequence of bounded finite rank operators.

\begin{theorem}
$\mathscr{B}_{00}(H)$ is a dense subset of the normed space $\mathscr{B}_2(H)$ with the Hilbert-Schmidt norm.
\end{theorem}
\begin{proof}
If $A \in \mathscr{B}_{00}(H)$, then there is an orthonormal basis $\{e_i: i \in I\}$  for $H$ and a finite subset $J$ of $I$
such that if $i \in I \setminus J$ then
$Ae_i=0$.
From this it follows that $\norm{A}_2<\infty$, and thus $A \in \mathscr{B}_2(H)$, so $\mathscr{B}_{00}(H)$ is a subset of $\mathscr{B}_2(H)$.

Let $\{e_i:i \in I\}$ be an orthonormal basis for $H$, let 
$A \in \mathscr{B}_2(H)$, and let $\epsilon>0$.   As $\sum_{i \in I} \norm{Ae_i}^2 < \infty$,
there is some finite subset $J$ of $I$ such that 
\[
\sum_{i \in I \setminus J} \norm{Ae_i}^2 < \epsilon.
\] 
Let $P$ be the orthogonal projection onto $\Span\{e_i: i \in J\}$ (which is finite dimensional and hence closed), and define
$B \in \mathscr{B}_{00}(H)$ by $B=AP$.
We have
\[
\norm{A-B}_2^2 = \sum_{i \in I \setminus J} \norm{Ae_i}^2 < \epsilon,
\]
showing that $\mathscr{B}_{00}(H)$ is dense in $\mathscr{B}_2(H)$.
\end{proof}


If $A \in \mathscr{B}_2(H)$, then by the above theorem there is a sequence of bounded finite rank operators $A_n$ such that
$\norm{A_n-A}_2 \to 0$ as $n \to \infty$. But by Theorem \ref{HSdominated}, $\norm{A_n-A} \leq \norm{A_n-A}_2$,
and by Theorem \ref{finiterankdensecompact}, a limit of bounded finite rank operators is a compact operator, so $A$ is compact.
Thus, a bounded Hilbert-Schmidt operator is a compact operator.


We are going to define an inner product on $\mathscr{B}_2(H)$ and we will show that with this inner product $\mathscr{B}_2(H)$ is a Hilbert space. However
the cleanest way I know to do this is by defining the trace of an operator. Moreover, we care just as much about trace class operators as we do Hilbert-Schmidt operators.



\section{Trace class operators}
 If $\{e_\alpha: \alpha \in I\}$ and $\{f_\alpha: \alpha \in J\}$ are orthonormal bases for $H$
and $A \in \mathscr{B}(H)$, then using Theorem \ref{basisindependent} we have
\begin{eqnarray*}
\sum_{\alpha \in I} \inner{|A|e_\alpha}{e_\alpha}&=&\sum_{\alpha \in I} \inner{|A|^{1/2}e_\alpha}{|A|^{1/2}e_\alpha}\\
&=&\sum_{\alpha \in I} \norm{|A|^{1/2} e_\alpha}^2\\
&=&\sum_{\alpha \in J} \norm{|A|^{1/2} f_\alpha}^2\\
&=&\sum_{\alpha \in J} \inner{|A|^{1/2} f_\alpha}{|A|^{1/2} f_\alpha}\\
&=&\sum_{\alpha \in J} \inner{|A|f_\alpha}{f_\alpha}.
\end{eqnarray*}
If $\{e_i: i \in I\}$ is an orthonormal basis for $H$, we say that $A \in \mathscr{B}(H)$ is {\em trace class} if
\[
\sum_{i \in I} \inner{|A|e_i}{e_i} < \infty.
\]
We denote the set of trace class operators in $\mathscr{B}(H)$ by $\mathscr{B}_1(H)$. For $A \in \mathscr{B}(H)$, define
\[
\norm{A}_1 =\sum_{i \in I} \inner{|A|e_i}{e_i}.
\]
To say that $A \in \mathscr{B}_1(H)$ is to say that $A \in \mathscr{B}(H)$ and that $\norm{A}_1 < \infty$. 
As $|A|^{1/2}$ is self-adjoint,
it is apparent that $\norm{A}_1 = \norm{|A|^{1/2}}_2^2$.
We
will prove that $\mathscr{B}_1(H)$ is a vector space and that $\norm{\cdot}_1$ is a norm on this vector space, but this takes a surprising amount of work and we will not do this yet.




The following theorem gives different characterizations of bounded trace class operators.\footnote{John B. Conway,
{\em A Course in Operator Theory}, p.~88, Proposition 18.8.} This theorem shows in particular that if $A \in \mathscr{B}_1(H)$ then
$A$ is the product of two bounded Hilbert-Schmidt operators, and thus, as $\mathscr{B}_2(H)$ is an ideal in $\mathscr{B}(H)$,  that $A \in \mathscr{B}_2(H)$.
In particular, as a consequence Theorem \ref{HSdominated}, every bounded Hilbert-Schmidt operator is compact, so every bounded trace
class operator is compact.


\begin{theorem}
If $A \in \mathscr{B}(H)$, then the following are equivalent.
\begin{itemize}
\item $A \in \mathscr{B}_1(H)$.
\item $|A|^{1/2} \in \mathscr{B}_2(H)$.
\item $A$ is the product of two elements of $\mathscr{B}_2(H)$.
\item $|A|$ is the product of two elements of $\mathscr{B}_2(H)$.
\end{itemize}
\label{traceTFAE}
\end{theorem}
\begin{proof}
Let $e_i, i \in I$ be an orthonormal basis for $H$.
Suppose that $A \in \mathscr{B}_1(H)$. We have
\[
\norm{|A|^{1/2}}_2^2=\sum_{i \in I} \norm{|A|^{1/2}e_i}^2 = \sum_{i \in I} \inner{|A|^{1/2}e_i}{|A|^{1/2}e_i}=\sum_{i \in I} \inner{|A|e_i}{e_i} = \norm{A}_1,
\]
so $|A|^{1/2} \in \mathscr{B}_2(H)$.

Suppose that $|A|^{1/2} \in \mathscr{B}_2(H)$. $A=U|A|=(U|A|^{1/2}) |A|^{1/2}$. As $\mathscr{B}_2(H)$ is an ideal,
we get $U|A|^{1/2} \in \mathscr{B}_2(H)$, hence $A$ is the product of two elements of $\mathscr{B}_2(H)$.

Suppose that $A=BC$, with $B,C \in \mathscr{B}_2(H)$. Let $A=U|A|$ be the polar decomposition of $A$. By 
\eqref{polardecomposition}, the polar decomposition satisfies $U^*U|A|=|A|$. 
But $U|A|=BC$ implies that $U^*U|A|=U^*BC$, hence $|A|=U^*BC=(U^*B)C$. As $\mathscr{B}_2(H)$ is an ideal, we have $U^*B \in \mathscr{B}_2(H)$, so we have written $|A|$ as a product of two elements of
$\mathscr{B}_2(H)$.

Suppose that $|A|=BC$, with  $B,C \in \mathscr{B}_2(H)$; so $B^* \in \mathscr{B}_2(H)$ too. We have, using the Cauchy-Schwarz inequality first in $H$ and next in $\ell^2(I)$,
\begin{eqnarray*}
\norm{A}_1&=&\sum_{i \in I} \inner{|A|e_i}{e_i}\\
&=&\sum_{i \in I} \inner{BCe_i}{e_i}\\
&=&\sum_{i \in I} \inner{Ce_i}{B^*e_i}\\
&\leq&\sum_{i \in I} \norm{Ce_i}\norm{B^*e_i}\\
&\leq&\left( \sum_{i \in I} \norm{Ce_i}^2 \right)^{1/2} \left( \sum_{i \in I} \norm{B^*e_i}^2 \right)^{1/2}\\
&=&\norm{C}_2 \norm{B^*}_2\\
&<&\infty.
\end{eqnarray*}
Hence $A \in \mathscr{B}_1(H)$, completing the proof.
\end{proof}


The following theorem shows that if $A \in \mathscr{B}_1(H)$ then  sums similar to $\norm{A}_1$ are also finite, and
 that the series $\sum_{e \in \mathscr{E}} \inner{Ae}{e}$ does not depend on the orthonormal basis $\mathscr{E}$.\footnote{John 
B. Conway, {\em A Course in Operator Theory}, p.~88, Proposition 18.9.}

\begin{theorem}
If $A \in \mathscr{B}_1(H)$ and $\mathscr{E}$ is an orthonormal basis for $H$, then
\[
\sum_{e \in \mathscr{E}} |\inner{Ae}{e}|< \infty,
\]
and if $\mathscr{F}$ is an orthonormal basis for $H$ then
\[
\sum_{e \in \mathscr{E}} \inner{Ae}{e} = \sum_{f \in \mathscr{F}} \inner{Af}{f}.
\]
\label{absconvergent}
\end{theorem}
\begin{proof}
By Theorem \ref{traceTFAE}, there are $B,C \in \mathscr{B}_2(H)$ such that $A=C^*B$. If $\lambda \in \mathbb{C}$ and $e \in \mathscr{E}$ then
\begin{eqnarray*}
\norm{(B-\lambda C)e}^2 &=& \inner{(B-\lambda C)e}{(B-\lambda C)e}\\
&=&\inner{Be}{Be}-\inner{Be}{\lambda Ce}-\inner{\lambda Ce}{Be}+
\inner{\lambda Ce}{\lambda Ce}\\
&=&\norm{Be}^2-\inner{Be}{\lambda Ce}-\overline{\inner{Be}{\lambda Ce}}+|\lambda|^2 \norm{Ce}^2\\
&=&\norm{Be}^2 - 2 \Re \inner{Be}{\lambda Ce} +|\lambda|^2 \norm{Ce}^2.
\end{eqnarray*}
As $\norm{(B-\lambda C)e}^2 \geq 0$, 
\[
2 \Re \inner{Be}{\lambda Ce} \leq \norm{Be}^2+|\lambda|^2 \norm{Ce}^2,
\]
so
\[
2 \Re\left(\overline{\lambda} \inner{Be}{ Ce}\right) \leq \norm{Be}^2+|\lambda|^2 \norm{Ce}^2.
\]
This is true for any $\lambda \in \mathbb{C}$ and $e \in \mathscr{E}$. Take $|\lambda|=1$, depending on $e$, such that
\[
\overline{\lambda}\inner{Be}{Ce}=|\inner{Be}{Ce}|,
\]
which gives
\[
2\Re |\inner{Be}{Ce}| \leq \norm{Be}^2+\norm{Ce}^2,
\]
i.e.,
\[
|\inner{Be}{Ce}| \leq \frac{1}{2}\left( \norm{Be}^2+\norm{Ce}^2 \right).
\]
Since this inequality doesn't involve $\lambda$, the fact that we chose $\lambda$ depending on $e$ doesn't matter, and the above inequality holds
for any $e \in \mathscr{E}$. Therefore
\[
\sum_{e \in \mathscr{E}} |\inner{Ae}{e}| = \sum_{e \in \mathscr{E}} |\inner{C^*Be}{e}| = \sum_{e \in \mathscr{E}} |\inner{Be}{Ce}|
\leq \frac{1}{2} \norm{B}_2^2 + \frac{1}{2} \norm{C}_2^2 < \infty,
\]
which is the first statement we wanted to prove and which was necessary to prove even to make sense of the second statement.

Because $\sum_{e \in \mathscr{E}} |\inner{Ae}{e}|<\infty$, the series
$\sum_{e \in \mathscr{E}} \inner{Ae}{e}$ converges. We have to show that its value does not depend on the orthonormal basis $\mathscr{E}$.
If $e \in \mathscr{E}$, then
\begin{eqnarray*}
\norm{(B+C)e}^2 - \norm{(B-C)e}^2&=&\norm{Be}^2 + \inner{Be}{Ce}+\inner{Ce}{Be}+\norm{Ce}^2\\
&&-\left( \norm{Be}^2 - \inner{Be}{Ce}-\inner{Ce}{Be}+\norm{Ce}^2 \right)\\
&=&2\inner{Be}{Ce} + 2\inner{Ce}{Be}\\
&=&4\Re \inner{Be}{Ce},
\end{eqnarray*}
which gives us
\begin{eqnarray*}
\sum_{e \in \mathscr{E}} \Re \inner{Ae}{e}  &=& \sum_{e \in \mathscr{E}}\Re \inner{Be}{Ce}\\
&=&\frac{1}{4} \sum_{e \in \mathscr{E}} \norm{(B+C)e}^2 - \norm{(B-C)e}^2\\
&=&\frac{1}{4} \norm{B+C}_2^2 - \frac{1}{4} \norm{B-C}_2^2.
\end{eqnarray*}
Applying this to $iA=iC^*B$ gives us, as $(iC^*)^*=-iC$,
\[
\sum_{e \in \mathscr{E}} \Re \inner{iAe}{e} = \frac{1}{4} \norm{B-iC}_2^2 - \frac{1}{4} \norm{B+iC}_2^2.
\]
But $\Re \inner{iAe}{e} = \Re (i \inner{Ae}{e}) = - \Im \inner{Ae}{e}$, as $\Re (i(a+ib))=\Re (ia-b)=-b$. Therefore
\[
\sum_{e \in \mathscr{E}} -\Im \inner{Ae}{e}= \frac{1}{4} \norm{B-iC}_2^2 - \frac{1}{4} \norm{B+iC}_2^2,
\]
i.e.
\[
\sum_{e \in \mathscr{E}} \Im \inner{Ae}{e} = \frac{1}{4} \norm{B+iC}_2^2 -  \frac{1}{4} \norm{B-iC}_2^2.
\]
Thus
\[
\sum_{e \in \mathscr{E}} \inner{Ae}{e} = \frac{1}{4} \norm{B+C}_2^2 - \frac{1}{4} \norm{B-C}_2^2 + \frac{i}{4} \norm{B+iC}_2^2
 -  \frac{i}{4} \norm{B-iC}_2^2,
\]
and the right-hand side does not depend on the orthonormal basis $\mathscr{E}$, completing the proof.
\end{proof}



If $A \in \mathscr{B}_1(H)$ and $\mathscr{E}$ is an orthonormal basis for $H$, we define the {\em trace} of $A$, written $\tr A$, to be
\[
\tr A = \sum_{e \in \mathscr{E}} \inner{Ae}{e}.
\]
It is apparent that $\tr:\mathscr{B}_1(H) \to \mathbb{C}$ is a  {\em positive linear functional}: $\tr$ is a  linear functional $\mathscr{B}_1(H) \to \mathbb{C}$, and if $A \in \mathscr{B}_1(H)$ is a positive operator, then
$\tr A$ is real and $\geq 0$. If $A \in \mathscr{B}_1(H)$ is a positive operator then it is diagonalizable (being a bounded trace class operator implies that it is compact): there is an orthonormal
basis $\{e_i: i \in I\}$ for $H$ such that
\[
A=\sum_{i \in I} \inner{Ae_i}{e_i} e_i \otimes e_i,
\] 
where the series converges in the strong operator topology. 
Since $A$ is positive, $\inner{Ae_i}{e_i}$ is a real nonnegative number for each $i \in I$. $\tr A=0$ means that
\[
\sum_{i \in I} \inner{Ae_i}{e_i}=0,
\]
and as this is a series of nonnegative terms they must all be $0$. Putting these into the expression for $A$ gives us $A=0$, showing that
$\tr:\mathscr{B}_1(H) \to \mathbb{C}$ is a {\em positive definite} linear functional. We haven't yet proved that $\tr$ is a bounded linear functional. This follows
from Theorem \ref{tracesymmetric}, which we prove later in this section.




In the following we show that the set of bounded trace class operators is a normed vector space with norm $\norm{\cdot}_1$,
which we call the {\em trace norm}.\footnote{John
B. Conway, {\em A Course in Operator Theory}, p.~89, Theorem 18.11 (a).}

\begin{theorem}
$\mathscr{B}_1(H)$ is a normed vector space with the norm $\norm{\cdot}_1$.
\end{theorem}
\begin{proof}
Let $A,B \in \mathscr{B}_1(H)$, and let their polar decompositions be $A=U|A|, B=V|B|$, and $A+B = W|A+B|$. 
As bounded trace class operators are compact and as the compact operators are a vector space,
$A+B$ is a compact operator. We have already stated that if $T$ is compact then $|T|$ is compact, which follows
from the polar decomposition of $T$ and the fact that the compact operators are an ideal. Thus
$|A+B|$ is a compact operator. 
$|A+B|$ is a positive operator, so  by the spectral theorem for normal compact operators that we stated as Theorem \ref{spectraltheorem},
$|A+B|$ is diagonalizable: there is a countable subset $\{e_n: n \geq 1\}$ of an orthonormal basis $\mathscr{E}$ for $H$,
and $\lambda_n \in \mathbb{C}$, such that
$|A+B|e_n = \lambda_n e_n$ and $|A+B|e=0$ if $e \in \mathscr{E}$ is not a member of this countable subset.
As $|A+B|$ is a positive operator, the eigenvalues $\lambda_n$ are real and nonnegative. 
We have, in the strong operator topology,
\[
|A+B| = \sum_{n=1}^\infty \lambda_n e_n \otimes e_n.
\]

First,
as $|A+B|=W^*(A+B)$ and using the Cauchy-Schwarz inequality in $H$,
\begin{eqnarray*}
\norm{A+B}_1&=&\sum_{e \in \mathscr{E}} \inner{|A+B|e}{e}\\
&=&\sum_{n \geq 1} \inner{|A+B|e_n}{e_n}\\
&=&\sum_{n \geq 1} \inner{(A+B)e_n}{We_n}\\
&=&\sum_{n \geq 1} \inner{U|A|e_n}{We_n}+\inner{V|B|e_n}{We_n}\\
&=&\sum_{n \geq 1} \inner{|A|e_n}{U^*We_n}+\inner{|B|e_n}{V^*We_n}\\
&=&\sum_{n \geq 1} \inner{|A|^{1/2}e_n}{|A|^{1/2}U^*We_n}+\inner{|B|^{1/2}e_n}{|B|^{1/2}V^*We_n}\\
&\leq&\sum_{n \geq 1} \norm{|A|^{1/2}e_n} \cdot \norm{|A|^{1/2}U^*We_n} + \norm{|B|^{1/2}e_n} \cdot \norm{|B|^{1/2}V^*We_n}.
\end{eqnarray*}
Applying the Cauchy-Schwarz inequality in $\ell^2$ to this gives
\begin{eqnarray*}
\cdots&\leq&\left( \sum_{n \geq 1}  \norm{|A|^{1/2}e_n}^2 \right)^{1/2} \left( \sum_{n \geq 1} \norm{|A|^{1/2}U^*We_n}^2 \right)^{1/2}\\
&&+\left(\sum_{n \geq 1} \norm{|B|^{1/2}e_n}^2 \right)^{1/2} \left(  \sum_{n \geq 1}  \norm{|B|^{1/2}V^*We_n}^2 \right)^{1/2}.\\
&\leq&\left( \sum_{e \in \mathscr{E}}  \norm{|A|^{1/2}e}^2 \right)^{1/2} \left( \sum_{e \in \mathscr{E}} \norm{|A|^{1/2}U^*We}^2 \right)^{1/2}\\
&&+\left(\sum_{e \in \mathscr{E}} \norm{|B|^{1/2}e}^2 \right)^{1/2} \left(  \sum_{e \in \mathscr{E}}  \norm{|B|^{1/2}V^*We}^2 \right)^{1/2}.
\end{eqnarray*}
So we have
\[
\norm{A+B}_1 \leq \norm{|A|^{1/2}}_2 \cdot \norm{|A|^{1/2} U^* W}_2 + \norm{|B|^{1/2}}_2 \cdot \norm{|B|^{1/2}V^*W}_2.
\]
By Theorem \ref{traceTFAE}, $|A|^{1/2}, |B|^{1/2} \in \mathscr{B}_2(H)$, and then using
 the inequality in Theorem \ref{HSideal} gives us
\[
\norm{|A|^{1/2} U^* W}_2 \leq \norm{|A|^{1/2}}_2 \cdot \norm{U^*} \cdot \norm{W} \leq \norm{|A|^{1/2}}_2,
\]
where we used the fact that $U^*$ and $W$ are partial isometries and hence either have norm 1 or 0, depending on whether they are the zero map.
Likewise,
\[
\norm{|B|^{1/2}V^*W}_2 \leq \norm{|B|^{1/2}}_2.
\]
Therefore we have obtained
\[
\norm{A+B}_1 \leq \norm{|A|^{1/2}}_2^2 + \norm{|B|^{1/2}}_2^2 = \norm{A}_1 + \norm{B}_1.
\]
so $A+B \in \mathscr{B}_1(H)$, and $\norm{\cdot}_1$ 
satisfies the triangle inequality.

If $A \in \mathscr{B}(H)$ and $\alpha \in \mathbb{C}$ then $|\alpha A|^{1/2} = |\alpha|^{1/2} |A|^{1/2}$, and from this it follows
that if $A \in \mathscr{B}_1(H)$ and $\alpha \in \mathbb{C}$ then $\norm{\alpha A}_1 = |\alpha| \norm{A}_1$ and so $\alpha A \in \mathscr{B}_1(H)$. 
Therefore $\mathscr{B}_1(H)$ is a vector space.

If $\norm{A}_1 = 0$, then $\tr|A|=0$. We have shown that $\tr$ is a positive definite linear functional: hence $|A|=0$, and so $A=0$. This completes the proof
that $\norm{\cdot}_1$ is a norm on $\mathscr{B}_1(H)$.
\end{proof}


We have shown that $\mathscr{B}_1(H)$ with the trace norm $\norm{\cdot}_1$ is a normed space. In the following theorem we show that
a bounded finite rank operator is a bounded trace class operator, and that $\mathscr{B}_{00}(H)$ is a dense subset of $\mathscr{B}_1(H)$. 


\begin{theorem}
$\mathscr{B}_{00}(H)$ is a dense subset of the normed space $\mathscr{B}_1(H)$ with the trace norm.
\end{theorem}
\begin{proof}
If $A \in \mathscr{B}_{00}(H)$, then there is an 
orthonormal basis  $\{e_i: i \in I\}$ for $H$ and a finite subset $J$ of $I$ such that
such that if $i \in I \setminus J$ then $Ae_i=0$, and so $|A|e_i=0$. This gives $\norm{A}_1<\infty$, so $A \in \mathscr{B}_1(H)$.

Let $\{e_i:i \in I\}$ be an orthonormal basis for $H$.
We've shown that the bounded finite rank operators are contained in the bounded trace class operators, and now we have to show that
if $A \in \mathscr{B}_1(H)$ and $\epsilon>0$ then there is some $B \in \mathscr{B}_{00}(H)$ satisfying $\norm{A-B}_1 < \epsilon$.
As $\sum_{i \in I} \inner{|A|e_i}{e_i}<\infty$, there is a finite subset $J$ of $I$ such that
\[
\sum_{i \in I \setminus J} \inner{|A|e_i}{e_i}<\epsilon.
\]
Let $P$ be the orthogonal projection onto $\Span \{e_i:i \in J\}$, and define $B \in \mathscr{B}_{00}(H)$ by $B=AP$, which gives us
\[
\norm{A-B}_1 = \sum_{i \in I} \inner{|A-B|e_i}{e_i} = \sum_{i \in I \setminus J} \inner{|A|e_i}{e_i} < \epsilon.
\]
\end{proof}


Taking an adjoint of an operator on a Hilbert space and taking the complex conjugate of a complex number ought to be interchangeable where it makes sense. We show
in the following theorem that the adjoint of an element of $\mathscr{B}_1(H)$ is also an element of $\mathscr{B}_1(H)$ and that the trace of the adjoint is the complex conjugate of the trace.

\begin{theorem}
If $A \in \mathscr{B}_1(H)$ then $A^* \in \mathscr{B}_1(H)$ and 
\[
\tr A^* = \overline{\tr A}.
\]
\label{traceadjoint}
\end{theorem}
\begin{proof}
As $A \in \mathscr{B}_1(H)$, by Theorem \ref{traceTFAE}  there are
$B,C \in \mathscr{B}_2(H)$ such that $A=C^*B$. Then $A^*=B^*C$ is
 a product of two bounded Hilbert-Schmidt operators, and so by the same theorem is itself an element of $\mathscr{B}_1(H)$.

We're going to extract something from our proof of Theorem \ref{absconvergent}, which showed that the trace of an
operator does not depend on the orthonormal basis that we use:
We proved that, with $A=C^*B$,
\[
\tr(A)= \frac{1}{4} \norm{B+C}_2^2 - \frac{1}{4} \norm{B-C}_2^2 + \frac{i}{4} \norm{B+iC}_2^2
 -  \frac{i}{4} \norm{B-iC}_2^2,
\]
Applying this to the adjoint $A^*=B^*C$, and as $i(C+iB)=iC-B$,
\begin{eqnarray*}
\tr A^*&=& \frac{1}{4} \norm{C+B}_2^2 - \frac{1}{4} \norm{C-B}_2^2 + \frac{i}{4} \norm{C+iB}_2^2
 -  \frac{i}{4} \norm{B-iC}_2^2\\
 &=&\frac{1}{4} \norm{B+C}_2^2 - \frac{1}{4} \norm{B-C}_2^2 + \frac{i}{4} \norm{iC-B}_2^2
 -  \frac{i}{4} \norm{iB+C}_2^2\\
 &=&\overline{\tr A}.
\end{eqnarray*}
\end{proof}



It is familiar to us that if $A$ and $B$ are matrices then $\tr(AB)=\tr(BA)$. In the next theorem we show that this is true for bounded linear operators providing one
of the two is a bounded
trace class operator.\footnote{John
B. Conway, {\em A Course in Operator Theory}, p.~89, Theorem 18.11 (e).} The first thing we prove in this theorem is that $\mathscr{B}_1(H)$ is an ideal of the
algebra $\mathscr{B}(H)$, so that it makes sense to talk about the trace of a product of two bounded linear operators only one of which is a bounded
trace class operator. The theorem also shows that $\tr:\mathscr{B}_1(H) \to \mathbb{C}$, which we have already shown is a positive definite linear functional, is bounded.


\begin{theorem}
If $A \in \mathscr{B}_1(H)$ and $T \in \mathscr{B}(H)$, then $AT, TA \in \mathscr{B}_1(H)$, and $\tr(AT)=\tr(TA)$, and $|\tr(TA)| \leq \norm{T} \norm{A}_1$.
\label{tracesymmetric}
\end{theorem}
\begin{proof}
$A$ is the product of two bounded Hilbert-Schmidt operators, say $A=C^*B$ (we write it this way because this will be handy later in the proof). Hence
$AT=C^*(BT)$. As $\mathscr{B}_2(H)$ is an ideal of $\mathscr{B}(H)$, we have $BT \in \mathscr{B}_2(H)$, showing that
$AT$ is a product of two bounded Hilbert-Schmidt operators, which implies that $AT \in \mathscr{B}_1(H)$.
Similarly, $TA \in \mathscr{B}_1(H)$.
We use from the proof of Theorem \ref{absconvergent} the following: (we wrote $A=C^*B$ to match the way we wrote $A$ in that theorem) 
\[
\tr(C^*B)= \frac{1}{4} \norm{B+C}_2^2 - \frac{1}{4} \norm{B-C}_2^2 + \frac{i}{4} \norm{B+iC}_2^2
 -  \frac{i}{4} \norm{B-iC}_2^2.
\] 
Applying this to $CB^*$, and using that the norm of the adjoint of an operator is equal to the norm of the operator itself and that
$(iC^*)^*=-iC$,
\begin{eqnarray*}
\tr(CB^*)&=&  \frac{1}{4} \norm{B^*+C^*}_2^2 - \frac{1}{4} \norm{B^*-C^*}_2^2 + \frac{i}{4} \norm{B^*+iC^*}_2^2
 -  \frac{i}{4} \norm{B^*-iC^*}_2^2\\
 &=&\frac{1}{4} \norm{B+C}_2^2 - \frac{1}{4} \norm{B-C}_2^2 + \frac{i}{4} \norm{B-iC}_2^2
 -  \frac{i}{4} \norm{B+iC}_2^2\\
 &=&\overline{\tr(C^*B)}.
\end{eqnarray*}
Using this we obtain
\begin{eqnarray*}
\tr(TA)&=&\tr((TC^*)B)\\
&=&\overline{\tr((TC^*)^* B^*)}\\
&=&\overline{\tr(CT^*B^*)}\\
&=&\overline{\tr(C(BT)^*)}\\
&=&\tr(C^*(BT))\\
&=&\tr(AT),
\end{eqnarray*}
which we wanted to show.

We still have to prove that $|\tr(TA)| \leq \norm{T} \norm{A}_1$. 
Let $A=U|A|$ be the polar decomposition of $A$, and let $\mathscr{E}$ be an orthonormal basis for $H$.
By Theorem \ref{traceTFAE}, $|A|^{1/2} \in \mathscr{B}_2(H)$. (We mention this  to justify talking about the Hilbert-Schmidt norm of $|A|^{1/2}$.)
We have, using the Cauchy-Schwarz inequality in $H$ and in $\ell^2$,
\begin{eqnarray*}
|\tr(TA)|&=&\left| \sum_{e \in \mathscr{E}} \inner{TU|A|e}{e} \right|\\
&=&\left| \sum_{e \in \mathscr{E}} \inner{|A|^{1/2}e}{|A|^{1/2}U^*T^*e} \right|\\
&\leq&\sum_{e \in \mathscr{E}} |  \inner{|A|^{1/2}e}{|A|^{1/2}U^*T^*e}  |\\
&\leq&\sum_{e \in \mathscr{E}}   \norm{|A|^{1/2}e} \cdot \norm{|A|^{1/2}U^*T^*e}\\
&\leq&\left( \sum_{e \in \mathscr{E}}   \norm{|A|^{1/2}e}^2 \right)^{1/2} \left( \sum_{e \in \mathscr{E}}  \norm{|A|^{1/2}U^*T^*e}^2 \right)^{1/2}\\
&=&\norm{|A|^{1/2}}_2 \cdot \norm{|A|^{1/2}U^* T^*}_2.
\end{eqnarray*}
By Theorem \ref{HSideal}, and as $U^*$ is a partial isometry,
\[
 \norm{|A|^{1/2}U^* T^*}_2 \leq \norm{|A|^{1/2}}_2 \cdot \norm{U^*} \cdot \norm{T^*} \leq \norm{|A|^{1/2}}_2 \cdot \norm{T^*}.
\]
Therefore
\[
|\tr(TA)| \leq \norm{|A|^{1/2}}_2^2 \cdot \norm{T^*}=\norm{|A|^{1/2}}_2^2 \cdot \norm{T}= \norm{A}_1 \cdot \norm{T},
\]
completing the proof.
\end{proof}



In Theorem \ref{traceadjoint} we proved that the adjoint of a bounded trace class operator is itself trace class, and we now prove that they have the same trace norm.

\begin{theorem}
If $A \in \mathscr{B}_1(H)$ then 
\[
\norm{A^*}_1 = \norm{A}_1.
\]
\label{adjointtracenorm}
\end{theorem}
\begin{proof}
The polar decomposition $A=U|A|$ satisfies
$|A^*|=U|A|U^*$ and $U^*U|A|=|A|$, as we stated in \eqref{polardecomposition}. 
We have, using this and Theorem \ref{tracesymmetric},
\[
\norm{A^*}_1 = \tr |A^*| = \tr ((U|A|)U^*) = \tr(U^*(U|A|)) = \tr |A| = \norm{A}_1.
\]
\end{proof}




\begin{theorem}
If $A \in \mathscr{B}_1(H)$ and $T \in \mathscr{B}(H)$, then
\[
\norm{AT}_1 \leq \norm{A}_1 \cdot \norm{T}, \qquad \norm{TA}_1 \leq \norm{T} \cdot \norm{A}_1.
\]
\end{theorem}
\begin{proof}
Let $A$ have the polar decomposition $A=U|A|$ and let $TA$ have the polar decomposition
$TA=W|TA|$. We have, from \eqref{polardecomposition},
\[
|TA|=W^*(TA)=W^*TU|A|.
\]
It follows from Theorem \ref{tracesymmetric} that, putting $S=W^*TU$,
\[
\norm{TA}_1 = \tr(|TA|) = \tr(S|A|) \leq \norm{S}\cdot \norm{A}_1.
\]
As $W^*$ and $U$ are partial isometries, $\norm{S} \leq \norm{T}$, thus
\begin{equation}
\norm{TA}_1 \leq \norm{T} \cdot \norm{A}_1.
\label{internalinequality}
\end{equation}

On the other hand, Theorem \ref{adjointtracenorm} tells us
\[
\norm{AT}_1 = \norm{T^*A^*}_1.
\]
As $A^* \in \mathscr{B}_1(H)$, by \eqref{internalinequality} we get 
\[
\norm{T^*A^*}_1 \leq \norm{T^*} \cdot \norm{A^*}_1 = \norm{T} \cdot \norm{A}_1,
\]
 using Theorem \ref{adjointtracenorm} again.
\end{proof}


In Theorem \ref{HSdominated} we proved that if $A \in \mathscr{B}_2(H)$ then $\norm{A} \leq \norm{A}_2$. Now we prove that the trace norm also dominates the operator norm, so
that the topology on the normed space $\mathscr{B}_1(H)$ with the trace norm is finer than its topology as a subspace of $\mathscr{B}(H)$.

\begin{theorem}
If $A \in \mathscr{B}_1(H)$, then 
\[
\norm{A} \leq \norm{A}_1.
\]
\label{tracedominates}
\end{theorem}
\begin{proof}
Let $A$ have polar decomposition $A=U|A|$. As $|A|$ is a compact positive, it is diagonalizable: there is an orthonormal basis
$\{e_i : i \in I\}$ for $H$ and $\lambda_i \in \mathbb{C}$ such that $|A|e_i = \lambda_i e_i$. On the one hand, the operator norm of a diagonalizable operator
is the supremum of the absolute values of its eigenvalues, as we  stated  in \eqref{diagonalnorm}.
On the other hand,
\[
\norm{|A|}_1= \sum_{i \in I} |\lambda_i|.
\]
Certainly then $\norm{|A|} \leq \norm{|A|}_1$. But
\[
\norm{A} \leq \norm{U} \cdot \norm{|A|} \leq \norm{|A|} \leq \norm{|A|}_1 = \norm{A}_1,
\]
which is what we wanted to prove.
\end{proof}

We have already shown that the trace class operators with the trace norm are a normed space. We now prove that they are a Banach space. We do this by showing that there is an
isometric isomorphism $\rho:\mathscr{B}_1(H) \to \mathscr{B}_0(H)^*$.\footnote{John B. Conway, {\em A Course in Operator Theory}, p.~93, Theorem 19.1.}  The latter space is a Banach space, so if $A_n \in \mathscr{B}_1(H)$ is a Cauchy sequence, its image
$\rho(A_n)$ is a Cauchy sequence in $\mathscr{B}_0(H)^*$ and hence has a limit, call it $B$. Since $\rho$ is surjective, there is some $A \in \mathscr{B}_1(H)$ such that
$\rho(A) = B$, and one checks that $A_n \to A$. 

For $A \in \mathscr{B}_1(H)$ and $C \in \mathscr{B}_0(H)$, define 
\[
\Phi_A(C)=\tr(CA).
\]

\begin{theorem}
The map $\rho:\mathscr{B}_1(H) \to \mathscr{B}_0(H)^*$ defined by
\[
\rho(A)=\Phi_A, \qquad A \in \mathscr{B}_1(H),
\]
is an isometric isomorphism.
\end{theorem}
\begin{proof}
Let $A \in \mathscr{B}_1(H)$. It is apparent that $\Phi_A:\mathscr{B}_0(H) \to \mathbb{C}$ is a linear map. Using Theorem \ref{tracesymmetric}, its operator norm is
\[
\norm{\Phi_A} = \sup_{\norm{C} \leq 1} |\Phi_A(C)| = \sup_{\norm{C} \leq 1} |\tr(CA)| \leq \sup_{\norm{C} \leq 1} \norm{C} \norm{A}_1
=\norm{A}_1,
\]
where the supremum is taken
over compact operators. Hence $\Phi_A \in \mathscr{B}_0(H)^*$ (if $H=\{0\}$ then the final equality is $\leq$). We have $\norm{\Phi_A} \leq \norm{A}_1$, so to show that $\rho$ is an isometric isomorphism, we have to show
that if $A \in \mathscr{B}_1(H)$ then $\norm{\Phi_A} \geq \norm{A}_1$, and that $\rho$ is surjective (as being injective is implied by being an isometry).

Let $\Phi \in \mathscr{B}_0(H)^*$. For $g,h \in H$, define
\[
B(g,h)=\Phi(g \otimes h), \qquad B(g,h)v=g \otimes h (v)=\inner{v}{h}g.
\]
It is apparent from this that $B$ is a sesquilinear form on $H$. A sesquilinear form $B$ is said to be {\em bounded} if
$M = \sup_{\norm{g}, \norm{h} =1} |B(g,h)| < \infty$.
For $\norm{g}=\norm{h}=1$,
\[
|B(g,h)| = |\Phi(g \otimes h)| \leq \norm{\Phi} \norm{g \otimes h} \leq \norm{\Phi} \norm{g} \norm{h}=\norm{\Phi}.
\]
Thus $B$ is a bounded sesquilinear form on $H$, and we can therefore apply
the {\em Riesz representation theorem},\footnote{Walter Rudin,
{\em Functional Analysis}, second ed., p.~310, Theorem 12.8.} which states that there is a unique $T \in \mathscr{B}(H)$ such that 
\[
B(g,h)=\inner{g}{Th}, \qquad g,h \in H,
\]
and that this $T$ satisfies $\norm{T}=M$.

Let $A=T^*$,  let
 $A=U|A|$ be the polar decomposition of $A$, and let $\mathscr{E}$ be an orthonormal basis for $H$. If $\mathscr{E}_0$ is a finite subset of $\mathscr{E}$,
define
\[
C_{\mathscr{E}_0} = \left( \sum_{e \in \mathscr{E}_0} e \otimes e \right) U^*= \sum_{e \in \mathscr{E}_0} e \otimes Ue.
\]
It is apparent that $C_{\mathscr{E}_0} \in \mathscr{B}_{00}(H)$, and one checks that $\norm{C_{\mathscr{E}_0}} \leq 1$. 
We have
\begin{eqnarray*}
\sum_{e \in \mathscr{E}_0} \inner{|A|e}{e}&=&\left| \sum_{e \in \mathscr{E}_0} \inner{|A|e}{e} \right|\\
&=&\left| \sum_{e \in \mathscr{E}_0} \inner{U^*Ae}{e} \right|\\
&=&\left| \sum_{e \in \mathscr{E}_0} \inner{e}{TUe} \right|\\
&=&\left| \sum_{e \in \mathscr{E}_0} B(e,Ue) \right|\\
&=&\left| \sum_{e \in \mathscr{E}_0} \Phi(e \otimes Ue) \right|\\
&=&\left| \Phi (C_{\mathscr{E}_0}) \right|.
\end{eqnarray*}
Then
\[
\sum_{e \in \mathscr{E}_0} \inner{|A|e}{e} \leq \norm{\Phi} \norm{C_{\mathscr{E}_0}} \leq \norm{\Phi}.
\]
This is true for any finite subset $\mathscr{E}_0$ of $\mathscr{E}$, and it follows that
\begin{equation}
\norm{A}_1 \leq \norm{\Phi},
\label{norminequality}
\end{equation}
and thus $A \in \mathscr{B}_1(H)$.

Let $C \in \mathscr{B}_{00}(H)$. Then there are some $g_1,\ldots,g_n$, $h_1,\ldots,h_n \in H$ such that (cf. Theorem \ref{finiteranksum})
\[
C = \sum_{k=1}^n g_k \otimes h_k.
\]
We have
\begin{eqnarray*}
\Phi(C)&=&\sum_{k=1}^n \Phi( g_k \otimes h_k)\\
&=&\sum_{k=1}^n B(g_k,h_k)\\
&=&\sum_{k=1}^n \inner{Ag_k}{h_k}\\
&=&\sum_{k=1}^n \tr (A(g_k \otimes h_k))\\
&=&\tr(AC)\\
&=&\Phi_A(C).
\end{eqnarray*}
Since the bounded linear functionals $\Phi$ and $\Phi_A$ agree on $\mathscr{B}_{00}(H)$,  a dense subset of $\mathscr{B}_1(H)$ with the
trace norm, they are equal. Therefore, $\rho$ is surjective. Moreover, we showed in \eqref{norminequality} that
$\norm{A}_1 \leq \norm{\Phi}$, so $\norm{\Phi_A} \geq \norm{A}_1$, so  $\rho$ is an isometry, which  completes the proof.
\end{proof}




We have shown that the $\mathscr{B}_1(H)$ is the dual of $\mathscr{B}_0(H)$. It can further be shown that
$\mathscr{B}(H)$ is the dual of $\mathscr{B}_1(H)$. For 
$T \in \mathscr{B}(H)$ we define $\Psi_T:\mathscr{B}_1(H) \to \mathbb{C}$ by 
$\Psi_T(A)=\tr(TA)$, $A \in \mathscr{B}_1(H)$.
Then the map $T \mapsto \Psi_T$ is an isometric isomorphism $\mathscr{B}(H) \to \mathscr{B}_1(H)^*$.\footnote{John B. Conway, {\em A Course in Operator Theory}, p.~94, Theorem 19.2.} 






\section{The Hilbert-Schmidt inner product}
If $A,B \in \mathscr{B}_2(H)$, we define
\[
\inner{A}{B} = \tr(B^*A),
\]
which makes sense because, by Theorem \ref{traceTFAE}, $AB \in \mathscr{B}_1(H)$.
As $\tr:\mathscr{B}_1(H) \to \mathbb{C}$ is a positive definite linear functional, we get that $\inner{\cdot}{\cdot}:\mathscr{B}_2(H) \times \mathscr{B}_2(H) \to \mathbb{C}$
is an inner product. 
We call this the {\em Hilbert-Schmidt inner product}. Check that $\inner{A}{A}=\norm{A}_2^2$.
It is a fact that $\mathscr{B}_2(H)$ is a complete metric space with the Hilbert-Schmidt norm,\footnote{Gert K. Pedersen,
{\em Analysis Now}, revised printing, p.~119, Theorem 3.4.9.} and hence $\mathscr{B}_2(H)$ with the Hilbert-Schmidt inner product
is a Hilbert space.



\begin{theorem}
$\mathscr{B}_2(H)$ with the Hilbert-Schmidt inner product is a Hilbert space.
\end{theorem}





\end{document}
