\documentclass{article}
\usepackage{amsmath,amssymb,mathrsfs,amsthm}
%\usepackage{tikz-cd}
%\usepackage{hyperref}
\newcommand{\inner}[2]{\left\langle #1, #2 \right\rangle}
\newcommand{\tr}{\ensuremath\mathrm{tr}\,} 
\newcommand{\Span}{\ensuremath\mathrm{span}} 
\def\Re{\ensuremath{\mathrm{Re}}\,}
\def\Im{\ensuremath{\mathrm{Im}}\,}
\newcommand{\id}{\ensuremath\mathrm{id}} 
\newcommand{\diam}{\ensuremath\mathrm{diam}} 
\newcommand{\lcm}{\ensuremath\mathrm{lcm}} 
\newcommand{\supp}{\ensuremath\mathrm{supp}\,}
\newcommand{\dom}{\ensuremath\mathrm{dom}\,}
\newcommand{\upto}{\nearrow}
\newcommand{\downto}{\searrow}
\newcommand{\norm}[1]{\left\Vert #1 \right\Vert}
\newtheorem{theorem}{Theorem}
\newtheorem{lemma}[theorem]{Lemma}
\newtheorem{proposition}[theorem]{Proposition}
\newtheorem{corollary}[theorem]{Corollary}
\theoremstyle{definition}
\newtheorem{definition}[theorem]{Definition}
\newtheorem{example}[theorem]{Example}
\begin{document}
\title{The Glivenko-Cantelli theorem}
\author{Jordan Bell}
\date{April 12, 2015}

\maketitle

\section{Narrow topology}
Let $X$ be a metrizable space and let $C_b(X)$ be the Banach space of bounded continuous functions $X \to \mathbb{R}$, with the norm
$\norm{f} = \sup_{x \in X} |f(x)|$. 
If $X$ is metrizable with the metric $d$, let $U_d(X)$ be the collection of bounded $d$-uniformly continuous functions
$X \to \mathbb{R}$. This is a vector space and is a closed subset of $C_b(X)$, thus is itself a Banach space. 


Let $X$ be a  metrizable space and
denote by $\mathscr{P}(X)$ the collection of Borel probability measures on $X$.
The \textbf{narrow topology on $\mathscr{P}(X)$} is the
coarsest topology on $\mathscr{P}(X)$ such that for every $f \in C_b(X)$,  the mapping
$\mu \mapsto \int_X fd\mu$ is continuous $\mathscr{P}(X) \to \mathbb{R}$.
It can be proved that if $X$ is metrizable with a metric $d$ and $D$ is a dense subset of $U_d(X)$, then
the narrow topology is equal to the coarsest topology such that for each $f \in U_d(X)$, the mapping
$\mu \mapsto \int_X fd\mu$ is continuous $\mathscr{P}(X) \to \mathbb{R}$.\footnote{Charalambos D. 
Aliprantis and Kim C. Border, {\em Infinite Dimensional Analysis: A Hitchhiker's Guide}, third ed., p.~507,
Theorem 15.2.} 

If $X$ is a separable metrizable space, then it is metrizable by a metric $d$ such that the metric space $(X,d)$
is totally bounded. It is a fact that if $(X,d)$ is a totally bounded metric space, then
$U_d(X)$ is separable.\footnote{Daniel W. Stroock, {\em Probability Theory: An Analytic View},
p.~371, Lemma 9.1.4.}

\begin{theorem}
If $X$ is a separable metrizable space, then $X$ is metrizable by a metric $d$ for which there
is a countable dense subset $D$ of $U_d(X)$ such that $\mu_n$ converges narrowly to $\mu$ if and only if
\[
\int_X f d\mu_n \to \int_X f d\mu, \qquad f \in D.
\]
\label{Ud}
\end{theorem}



\section{Independent and identically distributed random variables}
Let $(\Omega,S,P)$ be a probability space and let $X$ be a separable metric space, with the Borel $\sigma$-algebra $\mathscr{B}_X$.
We say that a finite collection measurable functions $\xi_i: \Omega \to X$, $1 \leq i \leq n$, is \textbf{independent} if
\[
P\left( \bigcap_{i=1}^n \xi_i^{-1}(A_i) \right) = \prod_{i=1}^n P(\xi_i^{-1}(A_i)),
\qquad A_1,\ldots,A_n \in \mathscr{B}_X,
\]
i.e.
\[
P(\xi_1 \in A_1, \ldots, \xi_n \in A_n) = P(\xi_1 \in A_1) \cdots P(\xi_n \in A_n),
\qquad A_1,\ldots,A_n \in \mathscr{B}_X.
\]
We say that a family of measurable functions is independent if every finite subset of it is independent.

We say that two measurable functions $f,g:\Omega \to X$ are \textbf{identically distributed} if the pushforward $f_* P$ of $P$ by $f$
is equal to the pushforward $g_* P$ of $P$ by $g$, i.e. $P(f^{-1}(A)) = P(g^{-1}(A))$ for every $A \in \mathscr{B}_X$. We say that a family of
measurable functions is identically distributed if any two of them are identically distributed.




\section{Strong law of large numbers}
If $\zeta \in L^1(P)$,  
the \textbf{expectation of $\zeta$} is
\[ 
E(\zeta) = \int_\Omega \zeta dP,
\]
and by the change of variables theorem,
\[
\int_\Omega \zeta(\omega) dP(\omega) = \int_{\mathbb{R}} x d(\zeta_* P)(x).
\]
The \textbf{strong law of large numbers}\footnote{M. Lo\`eve, {\em Probability Theory I}, 4th ed.,
p.~251, 17.B.} states that if $\zeta_1,\zeta_2, \ldots \in L^1(P)$ 
are independent and identically distributed, with common expectation $E_0$, then
\[
P\left( \left\{ \omega \in \Omega: \sum_{i=1}^n \frac{\zeta_i(\omega)}{n} \to E_0 \right\} \right) = 1.
\]




\section{Sample distributions}
Let $X$ be a separable metrizable space and let $\xi_1,\xi_2,\ldots$ be independent and identically distributed measurable
functions $\Omega \to X$.
For $\omega \in \Omega$, define $\mu_n^\omega$ on $\mathscr{B}_X$ by
\[
\mu_n^\omega = \sum_{i=1}^n \frac{1}{n} \delta_{\xi_i(\omega)},
\]
which is a probability measure. We call the sequence $\mu_n^\omega$ the \textbf{sample distribution of $\omega$}.

The following is the \textbf{Glivenko-Cantelli theorem}, which shows that
the sample distributions of a sequence of independent and identically
distributed measurable functions converge narrowly almost everywhere
to the common pushforward measure.\footnote{K. R. Parthasarathy,
{\em Probability Measures on Metric Spaces}, p.~53, Theorem 7.1.}

\begin{theorem}[Glivenko-Cantelli theorem]
Let $(\Omega,S,P)$ be a probability space,
let $X$ be a separable metrizable space and 
let  $\xi_1,\xi_2,\ldots$ be independent and identically distributed measurable functions
$\Omega \to X$, with common pushforward measure $\mu$. Then
\[
P\left( \left\{ \omega \in \Omega: \textnormal{$\mu_n^\omega \to \mu$ narrowly} \right\} \right) = 1.
\]
\end{theorem}
\begin{proof}
For $g \in C_b(X)$, $g \circ \xi_i:\Omega \to \mathbb{R}$ is measurable bounded, hence belongs to $L^1(P)$. Also, 
$(g \circ \xi_i)_* P = g_* \mu$, so the sequence $g \circ \xi_i$ are identically distributed. We now check that the sequence is independent.
Let $A_1,\ldots,A_n \in \mathscr{B}_{\mathbb{R}}$. Then $g^{-1}(A_1),\ldots,g^{-1}(A_n) \in \mathscr{B}_X$, and because
$\xi_1,\xi_2,\ldots$ are independent,
\[
P\left( \bigcap_{i=1}^n \xi_i^{-1}(g^{-1}(A_i)) \right) = \prod_{i=1}^n P(\xi_i^{-1}(g^{-1}(A_i))),
\]
i.e.,
\[
P\left( \bigcap_{i=1}^n (g \circ \xi_i)^{-1}(A_i) \right) = \prod_{i=1}^n P((g \circ \xi_i)^{-1}(A_i)),
\]
showing that  $(g \circ \xi_1),(g \circ \xi_2),\ldots$ are independent. 
For any $i$, by the change of variables theorem
\[
E( g \circ \xi_i) = \int_\Omega g \circ \xi_i dP = \int_X g d((\xi_i)_* P) = 
\int_X g d\mu,
\]
so the strong law of large numbers tells us that there is a set $N_g \in S$ with $P(N_g)=0$ such that
for all $\omega \in \Omega \setminus N_g$, 
\[
\sum_{i=1}^n \frac{(g \circ \xi_i)(\omega)}{n} \to \int_X g d\mu.
\]
But
\[
\sum_{i=1}^n \frac{(g \circ \xi_i)(\omega)}{n} = 
\sum_{i=1}^n \frac{1}{n} \int_X g d\delta_{\xi_i(\omega)}
=\int_X g d\mu_n^\omega,
\]
so for all $\omega \in \Omega \setminus N_g$,
\[
\int_X g d\mu_n^\omega \to \int_X g d\mu.
\]

Because $X$ is separable,
Theorem \ref{Ud} tells us that there is a metric $d$ that induces the topology of $X$ and some countable dense subset $G$
of $U_d(X)$ such that a sequence $\nu_n$ in $\mathscr{P}(X)$ converges narrowly to $\nu$  if and only if 
\[
\int_X g d\nu_n \to \int_X g d\nu, \qquad g \in G.
\]
Now let $N=\bigcup_{g \in G} N_g$, which satisfies $P(N)=0$, and if $\omega \in \Omega \setminus N$ then
for each $g \in G$,
\[
\int_X g d\mu_n^\omega \to \int_X g d\mu.
\]
This implies that for all $\mu_n^\omega$ converges narrowly to $\mu$. That is,
there is a set $N \in S$ with $P(N)=0$ such that for all $\omega \in \Omega \setminus N$, 
the sample distribution $\mu_n^\omega$ converges narrowly to the common pushforward measure 
$\mu$, proving the claim.
\end{proof}



\end{document}