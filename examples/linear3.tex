\documentclass[14pt,letterpaper]{article}
\usepackage[lmargin=0.75in,rmargin=0.75in,tmargin=0.75in,bmargin=0.5in]{geometry}

% -------------------
% Packages
% -------------------
\usepackage{
	amsmath,			% Math Environments
	amssymb,			% Extended Symbols
	enumerate,		    % Enumerate Environments
	graphicx,			% Include Images
	lastpage,			% Reference Lastpage
	multicol,			% Use Multi-columns
	multirow			% Use Multi-rows
}
\usepackage[framemethod=TikZ]{mdframed}
\usepackage{tikz, tabularx}
\usepackage{graphics}
\newcolumntype{W}{>{\centering\arraybackslash}X}%Para agilizar las columnas.
% -------------------
% Font
% -------------------
\usepackage[T1]{fontenc}
\usepackage{charter}


% -------------------
% Commands
% -------------------
\newcommand{\prob}{\noindent\textbf{Problema. }}
\newcounter{problema}
\newcommand{\problem}{
	\stepcounter{problema}%
	\noindent \textbf{Problema \theproblem. }%
}
\newcommand{\pointproblem}[1]{
	\stepcounter{problema}%
	\noindent \textbf{Problema \theproblem.} (#1 points)\,%
}
\newcommand{\pspace}{\par\vspace{\baselineskip}}
\newcommand{\ds}{\displaystyle}


% -------------------
% Theorem Environment
% -------------------
\mdfdefinestyle{theoremstyle}{%
	frametitlerule=true,
	roundcorner=5pt,
	linecolor=black,
	outerlinewidth=0.5pt,
	middlelinewidth=0.5pt
}
\mdtheorem[style=theoremstyle]{exercise}{\textbf{Problem}}


% -------------------
% Header & Footer
% -------------------
\usepackage{fancyhdr}

\fancypagestyle{pages}{
	%Headers
	\fancyhead[L]{}
	\fancyhead[C]{}
	\fancyhead[R]{}
\renewcommand{\headrulewidth}{0pt}
	%Footers
	\fancyfoot[L]{}
	\fancyfoot[C]{}
	\fancyfoot[R]{\thepage \,de \pageref{LastPage}}
\renewcommand{\footrulewidth}{0.0pt}
}
\headheight=0pt
\footskip=14pt




% -------------------
% Content
% -------------------
\begin{document}

\begin{exercise}
Solve the system of linear equations
\begin{align*}
x+2y&=2x-5\\
x-y&=3
\end{align*}
\hline
\textbf{Step 1}
\begin{align*}
2y&=\framebox(40,20){}-5\\
x-y&=3\\
\end{align*}
\hline
\textbf{Step 2}
\begin{align*}
x-5&=2y\\
x-y&=3
\end{align*}
\hline
\textbf{Step 3}
\begin{align*}
x&=\framebox(60,20){}\\
x-y&=3
\end{align*}
\hline
\textbf{Step 4}
\begin{align*}
x&=2y+5\\
\framebox(60,20){}-y&=3
\end{align*}
\hline
\textbf{Step 5}
\begin{align*}
x&=2y+5\\
(2y+5)-y&=3
\end{align*}
\hline
\textbf{Step 6}
\begin{align*}
x&=2y+5\\
y+\framebox(40,20){}&=3
\end{align*}
\hline
\textbf{Step 7}
\begin{align*}
x&=2y+5\\
y&=-2
\end{align*}
\hline
\textbf{Step 8}
\begin{align*}
x&=2\left(\framebox(40,20){}\right)+5\\
y&=-2
\end{align*}
\hline
\textbf{Step 9}
\begin{align*}
x&=1\\
y&=-2
\end{align*}
\end{exercise}


\end{document}