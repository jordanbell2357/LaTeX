% !TEX TS-program = pdflatex
% !TEX encoding = UTF-8 Unicode

% Example of the Memoir class, an alternative to the default LaTeX classes such as article and book, with many added features built into the class itself.

%\documentclass[12pt,a4paper]{memoir} % for a long document
\documentclass[12pt,a4paper,article]{memoir} % for a short document

\usepackage[utf8]{inputenc} % set input encoding to utf8

\usepackage{hyperref} % links
\usepackage{amsmath}

% Don't forget to read the Memoir manual: memman.pdf

%%% Examples of Memoir customization
%%% enable, disable or adjust these as desired

%%% PAGE DIMENSIONS
% Set up the paper to be as close as possible to both A4 & letter:
\settrimmedsize{11in}{210mm}{*} % letter = 11in tall; a4 = 210mm wide
\setlength{\trimtop}{0pt}
\setlength{\trimedge}{\stockwidth}
\addtolength{\trimedge}{-\paperwidth}
\settypeblocksize{*}{\lxvchars}{1.618} % we want to the text block to have golden proportionals
\setulmargins{50pt}{*}{*} % 50pt upper margins
\setlrmargins{*}{*}{1.618} % golden ratio again for left/right margins
\setheaderspaces{*}{*}{1.618}
\checkandfixthelayout 
% This is from memman.pdf

%%% \maketitle CUSTOMISATION
% For more than trivial changes, you may as well do it yourself in a titlepage environment
\pretitle{\begin{center}\sffamily\huge\MakeUppercase}
\posttitle{\par\end{center}\vskip 0.5em}

%%% ToC (table of contents) APPEARANCE
\maxtocdepth{subsection} % include subsections
\renewcommand{\cftchapterpagefont}{}
\renewcommand{\cftchapterfont}{}     % no bold!

%%% HEADERS & FOOTERS
\pagestyle{ruled} % try also: empty , plain , headings , ruled , Ruled , companion

%%% CHAPTERS
\chapterstyle{hangnum} % try also: default , section , hangnum , companion , article, demo

\renewcommand{\chaptitlefont}{\Huge\sffamily\raggedright} % set sans serif chapter title font
\renewcommand{\chapnumfont}{\Huge\sffamily\raggedright} % set sans serif chapter number font

%%% SECTIONS
\hangsecnum % hang the section numbers into the margin to match \chapterstyle{hangnum}
\maxsecnumdepth{subsection} % number subsections

\setsecheadstyle{\Large\sffamily\raggedright} % set sans serif section font
\setsubsecheadstyle{\large\sffamily\raggedright} % set sans serif subsection font

%% END Memoir customization

\title{Microeconomics}
\author{\href{https://jordanbell.info/}{jordanbell.info}}
\date{} % Delete this line to display the current date

%%% BEGIN DOCUMENT
\begin{document}

\maketitle

\chapter*{Chapter 10}

Firms hire and organize factors of production to
produce and sell goods and services

Each firm is an insti-
tution that hires factors of production and organizes
those factors to produce and sell goods and services

Economic profit
is equal to total revenue minus total cost, with total
cost measured as the opportunity cost of production .

The opportunity cost of any action is the highest-
valued alternative forgone. The opportunity cost of pro-
duction is the value of the best alternative use of the
resources that a firm uses in production.

Economic depreciation is the fall in the
market value of a firm’s capital over a given period.
It equals the market price of the capital at the begin-
ning of the period minus the market price of the
capital at the end of the period.

ntrepreneurship The factor of production that organ-
izes a firm and makes its decisions might be supplied
by the firm’s owner or by a hired entrepreneur. The
return to entrepreneurship is profit, and the profit
that an entrepreneur earns on average is called normal
profit . Normal profit is the cost of entrepreneurship
and is an opportunity cost of production.
If Cindy supplies entrepreneurial services herself,
and if the normal profit she can earn on these services
is $45,000 a year, this amount is an opportunity cost
of production at Campus Sweaters.

In addition to supplying
entrepreneurship, the owner of a firm might supply
labour but not take a wage. The opportunity cost of
the owner’s labour is the wage income forgone by not
taking the best alternative job.

Technological efficiency occurs
when the firm produces a given output by using the
least amount of inputs. Economic efficiency occurs
when the firm produces a given output at the least
cost.

Economic efficiency depends on the relative costs
of resources. The economically efficient method
is the one that uses a smaller amount of the more
expensive resource and a larger amount of the less
expensive resource.

The principal–agent problem is the problem of devis-
ing compensation rules that induce an agent to act
in the best interest of a principal .

A principal must create incentives that induce each
agent to work in the interests of the principal. Three
ways of coping with the principal–agent problem are:
■ Ownership
■ Incentive pay
■ Long-term contracts

The three main types of business organization are:
■ Sole proprietorship
■ Partnership
■ Corporation

A sole proprietorship is a firm
with a single owner—a proprietor—who has
unlimited liability. Unlimited liability is the legal
esponsibility for all the debts of a firm up to an
amount equal to the entire personal wealth of the
owner. Farmers, computer programmers, and artists
often operate as sole proprietorships.
The proprietor makes management decisions,
receives the firm’s profits, and is responsible for its
losses. Profits from a sole proprietorship are taxed at
the same rate as other sources of the proprietor’s per-
sonal income.

A partnership is a firm with two or more
owners who have unlimited liability. Partners must
agree on an appropriate management structure and
on how to divide the firm’s profits among them-
selves. The profits of a partnership are taxed as the
personal income of the owners, but each partner
is legally liable for all the debts of the partnership
(limited only by the wealth of that individual part-
ner). Liability for the full debts of the partnership
is called joint unlimited liability . Most law firms are
partnerships.

Corporations’ profits are taxed independently
of shareholders’ incomes. Shareholders pay a capi-
tal gains tax on the profit they earn when they sell
a stock for a higher price than they paid for it.
Corporate stocks generate capital gains when a
corporation retains some of its profit and reinvests
it in profitable activities. So retained earnings are
taxed twice because the capital gains they generate
are taxed. Dividend payments are also taxed but at a
lower rate than other sources of income.

Most firms in Canada today, regardless of size, are
corporations. Many of the very small firms that employ
fewer than 5 people find it advantageous to incorporate.

Monopolistic competition is a market structure in
which a large number of firms compete by making sim-
ilar but slightly different products. Making a product
slightly different from the product of a competing firm
is called product differentiation . Product differentiation
gives a firm in monopolistic competition an element of
market power. The firm is the sole producer of the par-
ticular version of the good in question. For example,
in the market for pizzas, hundreds of firms make their
own version of the perfect pizza. Each of these firms is
the sole producer of a particular brand. Differentiated
products are not necessarily different products. What
matters is that consumers perceive them to be differ-
ent. For example, different brands of potato chips and
ketchup might be almost identical but be perceived by
consumers to be different.

Oligopoly is a market structure in which a small
number of firms compete. Computer software,
airplane manufacture, and international air trans-
portation are examples of oligopolistic industries.
Oligopolies might produce almost identical products,
such as the colas produced by Coke and Pepsi. Or they
might produce differentiated products, such as Boeing
and Airbus aircraft.

Economists use two measures of concentration:
■ The four-firm concentration ratio
■ The Herfindahl-Hirschman Index

Also, a market with only a few firms might be
competitive because of potential entry . The few firms
in a market face competition from the many poten-
tial firms that will enter the market if economic profit
opportunities arise.

Firms coordinate economic activity when a task
can be performed more efficiently by a firm than by
markets. In such a situation, it is profitable to set up
a firm. Firms are often more efficient than markets as
coordinators of economic activity because they can
achieve:
■ Lower transactions costs
■ Economies of scale
■ Economies of scope
■ Economies of team production

\chapter{11: Output and Costs

Decisions about the quantity to produce and
the price to charge depend on the type of market
in which the firm operates. Perfect competition,
monopolistic competition, oligopoly, and monop-
oly all confront the firm with different problems.
Decisions about how to produce a given output do
not depend on the type of market in which the firm
operates. All types of firms in all types of markets
make similar decisions about how to produce.}

The long run is a time frame in which the quantities of
all factors of production can be varied. That is, the long
run is a period in which the firm can change its plant .

The product curves are graphs of the relationships
between employment and the three product concepts
you’ve just studied. They show how total product,
marginal product, and average product change as
employment changes. They also show the relation-
ships among the three concepts.

The law of diminishing returns states that
As a fi rm uses more of a variable factor of
production with a given quantity of the fi xed
factor of production, the marginal product of
the variable factor eventually diminishes.

Notice also that average product is largest when
average product and marginal product are equal.
That is, the marginal product curve cuts the aver-
age product curve at the point of maximum average
product. For the number of workers at which mar-
ginal product exceeds average product, average prod-
uct is increasing . For the number of workers at which
marginal product is less than average product, average
product is decreasing .

\end{document}
