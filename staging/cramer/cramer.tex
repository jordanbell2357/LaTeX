\documentclass{article}
\usepackage{amsmath,amssymb,mathrsfs,amsthm}
%\usepackage{tikz-cd}
\usepackage[draft]{hyperref}
\newcommand{\inner}[2]{\left\langle #1, #2 \right\rangle}
\newcommand{\tr}{\ensuremath\mathrm{tr}\,} 
\newcommand{\Span}{\ensuremath\mathrm{span}} 
\def\Re{\ensuremath{\mathrm{Re}}\,}
\def\Im{\ensuremath{\mathrm{Im}}\,}
\newcommand{\id}{\ensuremath\mathrm{id}} 
\newcommand{\var}{\ensuremath\mathrm{var}} 
\newcommand{\Lip}{\ensuremath\mathrm{Lip}} 
\newcommand{\Sh}{\ensuremath\mathrm{Sh}} 
\newcommand{\GL}{\ensuremath\mathrm{GL}} 
\newcommand{\diam}{\ensuremath\mathrm{diam}} 
\newcommand{\sgn}{\ensuremath\mathrm{sgn}}
\newcommand{\Var}{\ensuremath\mathrm{Var}} 
\newcommand{\lcm}{\ensuremath\mathrm{lcm}} 
\newcommand{\supp}{\ensuremath\mathrm{supp}\,}
\newcommand{\dom}{\ensuremath\mathrm{dom}\,}
\newcommand{\arctanh}{\ensuremath\mathrm{arctanh}\,}
\newcommand{\upto}{\nearrow}
\newcommand{\downto}{\searrow}
\newcommand{\norm}[1]{\left\Vert #1 \right\Vert}
\newtheorem{theorem}{Theorem}
\newtheorem{lemma}[theorem]{Lemma}
\newtheorem{proposition}[theorem]{Proposition}
\newtheorem{corollary}[theorem]{Corollary}
\theoremstyle{definition}
\newtheorem{definition}[theorem]{Definition}
\newtheorem{example}[theorem]{Example}
\begin{document}
\title{Subgaussian random variables, Hoeffding's inequality, and Cram\'er's large deviation theorem}
\author{Jordan Bell}
\date{June 4, 2014}

\maketitle


\section{Subgaussian random variables}
For a random variable $X$, let $\Lambda_X(t)=\log E(e^{tX})$, the 
\textbf{cumulant generating function of $X$}.
A \textbf{$b$-subgaussian random variable}, $b>0$, is a random variable $X$ such that
\[
\Lambda_X(t) \leq \frac{b^2 t^2}{2}, \qquad t \in \mathbb{R}.
\]
We remark that for $\gamma_{a,\sigma^2}$ a \textbf{Gaussian measure}, whose density with respect to Lebesgue
measure on $\mathbb{R}$ is
\[
p(x,a,\sigma^2)= \frac{1}{\sqrt{2\pi \sigma^2}} e^{-\frac{(x-a)^2}{2\sigma^2}},
\]
we have
\[
\int_\mathbb{R} e^{tx} d\gamma_{0,b^2}(x)
=\int_\mathbb{R} e^{bty} \frac{1}{\sqrt{2\pi}} e^{-\frac{y^2}{2}} dy
=\int_\mathbb{R} e^{\frac{b^2 t^2}{2}} \frac{1}{\sqrt{2\pi}} e^{-\frac{(y-bt)^2}{2}} dy
=e^{\frac{b^2t^2}{2}}.
\]

We prove that a $b$-subgaussian random variable is centered and has variance $\leq b^2$.\footnote{Karl  R. Stromberg,
{\em Probability for Analysts}, p.~293, Proposition 9.8.}


\begin{theorem}
If $X$ is $b$-subgaussian then $E(X)=0$ and $\Var(X) \leq b^2$. 
\end{theorem}
\begin{proof}
For each $\omega \in \Omega$,
$\sum_{k=0}^n \frac{t^k X(\omega)^k}{k!} \to e^{tX(\omega)}$, and by the dominated convergence theorem,
\[
\sum_{k=0}^n \frac{t^k E(X)^k}{k!} \to E(e^{tX}) \leq e^{\frac{b^2t^2}{2}}
=\sum_{k=0}^\infty \left(\frac{b^2 t^2}{2}\right)^k \frac{1}{k!}.
\]
Therefore
\[
1+tE(X)+t^2E(X^2) + O(t^3) \leq 1+\frac{b^2t^2}{2}+O(t^4),
\]
whence
\[
tE(X)+t^2E(X^2) \leq  \frac{b^2t^2}{2} + o(t^2),
\]
and so, for $t>0$,
\[
E(X)+tE(X^2) \leq \frac{b^2 t}{2} + o(t).
\]
First, this yields $E(X) = o(t)$, which means that $E(X)=0$. Second,
since $E(X)=0$, 
\[
tE(X^2) \leq \frac{b^2t}{2}+o(t),
\]
and then
\[
E(X^2) \leq \frac{b^2}{2}+o(1),
\]
which measn that $E(X^2) \leq \frac{b^2}{2}$.
\end{proof}


Stromberg attributes the following theorem to Saeki;
further, it is proved in Stromberg that if for some $t$ the inequality in the theorem is an equality then
the random variable has the Rademacher distribution.\footnote{Karl  R. Stromberg,
{\em Probability for Analysts}, p.~293, Proposition 9.9;
Omar Rivasplata, {\em Subgaussian random variables:
An expository note},
\url{http://www.math.ualberta.ca/~orivasplata/publications/subgaussians.pdf}}

\begin{theorem}
If $X$ is a random variable satisfying $E(X)=0$ and $P(X \in [-1,1])=1$, then
\[
E(e^{tX}) \leq \cosh t, \qquad t \in \mathbb{R}.
\]
\end{theorem}
\begin{proof}
Define $f:\mathbb{R} \to \mathbb{R}$ by
\[
f(t)  =e^t \left(\cosh t - E(e^{tX})\right) = \frac{e^{2t}}{2} + \frac{1}{2} - e^t  E(e^{tX}).
\]
Then
\[
f'(t) = e^{2t} - e^t E(e^{tX})  - e^t E(Xe^{tX});
\]
the derivative of $E(e^{tX})$ with respect to $t$ is obtained using the dominated convergence theorem.
Let  $Y=1+X$, with which
\[
f'(t) = e^{2t} -E(e^{tY}) - E(Xe^{tY})
=e^{2t}-E(e^{tY}) - E((Y-1)e^{tY})
=e^{2t}-E(Ye^{tY}).
\]
$E(X)=0$, so $E(Y)=1$, hence
\[
f'(t) = E(e^{2t}Y)-E(Ye^{tY})=E(Y(e^{2t}-e^{tY})).
\]
Because $P(Y \in [0,2])=1$, for $t \geq 0$, we have almost surely $e^{2t}-e^{tY} \geq 0$, and therefore almost surely
$Y(e^{2t}-e^{tY}) \geq 0$. 
Therefore, for $t \geq 0$,
\[
f'(t)=E(Y(e^{2t}-e^{tY})) \geq 0,
\]
which tells us that for $t \geq 0$,
\[
f(0) \leq f(t).
\]
As $f(0)=0$,  for $t \geq 0$,
\[
0 \leq e^t \left(\cosh t - E(e^{tX})\right),
\]
and so
\[
E(e^{tX}) \leq \cosh t.
\]
\end{proof}

\begin{corollary}
If a random variable $X$ satisfies $E(X)=0$ and
$P(|X| \leq b)=1$, then $X$ is $b$-subgaussian.
\end{corollary}


\section{Hoeffding's inequality}
We first prove \textbf{Hoeffding's lemma}.\footnote{St\'ephane Boucheron, G\'abor Lugosi, and
Pascal Massart, {\em Concentration Inequalities: A Nonasymptotic Theory of Independence}, p.~27, Lemma 2.2.}


\begin{lemma}[Hoeffding's lemma]
If a random variable $X$ satisfies $E(X)=0$ and $P(X \in [a,b])=1$, then $X$ is $\frac{b-a}{2}$-subgaussian.
\end{lemma}
\begin{proof}
Because $P(X \in [a,b]) =1$, it follows that
\[
\Var(X) \leq \frac{(b-a)^2}{4},
\]
not using that $P(X)=0$. (Namely, Popoviciu's inequality.)

Write $\mu = X_*P$
and for $\lambda \in \mathbb{R}$ define
\[
d\nu_\lambda(t) = \frac{e^{\lambda t}}{e^{\Lambda(\lambda)}} d\mu(t).
\]
We check
\[
\int_\mathbb{R} d\nu_\lambda(t) =\frac{1}{e^{\Lambda(\lambda)}} \int_\mathbb{R} 
e^{\lambda t} d(X_*P)(t)
=\frac{1}{e^{\Lambda(\lambda)}} \int_\Omega e^{\lambda X} dP
=1.
\]
There is a random variable $X_\lambda:(\Omega_\lambda,\mathscr{F}_\lambda,P_\lambda) \to \mathbb{R}$ for which ${X_\lambda}_*P_\lambda=\nu_\lambda$. 
$X_\lambda$ satisfies $P_\lambda(X_\lambda \in [a,b])=1$, and so 
\[
\Var(X_\lambda) \leq \frac{(b-a)^2}{4}.
\]
We calculate
\[
\Lambda_X'(t) = \frac{E(Xe^{tX})}{E(e^{tX})}
\]
and
\[
\Lambda_X''(t) = \frac{E(X^2 e^{tX}) E(e^{tX}) - E(Xe^{tX})E(Xe^{tX})}{E(e^{tX})^2}.
\]
But
\[
E(X_\lambda) =\int_\mathbb{R} t d\nu_\lambda(t) 
=\int_\mathbb{R} t \frac{e^{\lambda t}}{e^{\Lambda(\lambda)}} d\mu(t)
=\frac{1}{e^{\Lambda(\lambda)}} E(Xe^{\lambda X})
\]
and
\[
E(X_\lambda^2) = \int_\mathbb{R} t^2 d\nu_\lambda(t) = \frac{1}{e^{\Lambda(\lambda)}} E(X^2 e^{\lambda X}),
\]
and so
\begin{align*}
\Var(X_\lambda)& =E(X_\lambda^2)-E(X_\lambda)^2\\
&= \frac{E(X^2 e^{\lambda X})}{e^{\Lambda(\lambda)}}  - \frac{E(Xe^{\lambda X})^2}{e^{2\Lambda(\lambda)}}\\
&=\Lambda_X''(\lambda).
\end{align*}

For $\lambda \in \mathbb{R}$, Taylor's theorem tells us that there is some $\theta$ between $0$ and $\lambda$ such that
\[
\Lambda_X(\lambda) = \Lambda_X(0)+\lambda \Lambda_X'(0)
+\frac{\lambda^2}{2} \Lambda_X''(\theta)
=\frac{\lambda^2}{2} \Lambda_X''(\theta);
\]
here we have used that $E(X)=0$.
But from what we have shown, $\Var(X_\theta)=\Lambda_X''(\theta)$ and $\Var(X_\theta) \leq \frac{(b-a)^2}{4}$, so
\[
\Lambda_X(\lambda)  = \frac{\lambda^2}{2} \Var(X_\theta) \leq 
\frac{\lambda^2}{2} \cdot \frac{(b-a)^2}{4},
\] 
which shows that $X$ is $\frac{b-a}{2}$-subgaussian.
\end{proof}


We now  prove \textbf{Hoeffding's inequality}.\footnote{St\'ephane Boucheron, G\'abor Lugosi, and
Pascal Massart, {\em Concentration Inequalities: A Nonasymptotic Theory of Independence}, p.~34, Theorem 2.8.}

\begin{theorem}[Hoeffding's inequality]
Let $X_1,\ldots,X_n$ be independent random variables such that for each $1 \leq k \leq n$, 
$P(X_k \in [a_k,b_k])=1$, and write
$S_n=\sum_{k=1}^n X_k$.
For any $a>0$,
\[
P(S_n-E(S_n) \geq a) \leq \exp\left(-\frac{2a^2}{\sum_{k=1}^n (b_k-a_k)^2} \right).
\]
\end{theorem}
\begin{proof}
For $\lambda>0$ 
and  $\phi(t)=e^{\lambda t}$, 
because $\phi$ is nonnegative and nondecreasing, for $X$ a random variable we have
\[
1_{X \geq a} \phi(a) \leq \phi(X),
\]
and so
$E(1_{X \geq a} \phi(a)) \leq E(\phi(X))$, i.e.
\[
P(X \geq a) \leq \frac{E(e^{\lambda X})}{e^{\lambda a}}.
\]
Using this with $X=S_n-E(S_n)$ and because the $X_k$ are independent,
\[
P(S_n-E(S_n) \geq a) \leq \frac{1}{e^{\lambda a}} E(e^{\lambda(S_n-E(S_n))})
=e^{-\lambda a} \prod_{k=1}^n E(e^{\lambda(X_k-E(X_k))}).
\]
Because $P(X_k \in [a_k,b_k])=1$, we have $P(X_k-E(X_k) \in [a_k-E(X_k),b_k-E(X_k)])=1$, and  as
$(b_k-E(X_k))-(a_k-E(X_k))=b_k-a_k$,  Hoeffding's lemma tells us
\[
\log E(e^{\lambda(X_k-E(X_k))}) \leq \frac{(b_k-a_k)^2 \lambda^2}{8},
\]
and thus
\begin{align*}
P(S_n-E(S_n) \geq a)& \leq e^{-\lambda a} \exp\left( \sum_{k=1}^n \frac{(b_k-a_k)^2 \lambda^2}{8} \right)\\
&=\exp\left(-\lambda a + \frac{\lambda^2}{8} \sum_{k=1}^n (b_k-a_k)^2\right).
\end{align*}
We remark that
 $\lambda$ does not appear in the left-hand side. Define
 \[
 g(\lambda) = -\lambda a + \frac{\lambda^2}{8} \sum_{k=1}^n (b_k-a_k)^2,
 \]
 for which 
 \[
 g'(\lambda)=-a + \frac{\lambda}{4} \sum_{k=1}^n(b_k-a_k)^2.
 \]
 Then $g'(\lambda)=0$ if and only if
 \[
\lambda = \frac{4a}{ \sum_{k=1}^n(b_k-a_k)^2},
 \]
 at which $g$ assumes its infimum. Then
 \begin{align*}
 P(S_n-E(S_n) \geq a) &\leq \exp\left( -\frac{4a^2}{\sum_{k=1}^n(b_k-a_k)^2} + \frac{16a^2}{8} \frac{1}{\sum_{k=1}^n(b_k-a_k)^2} \right)\\
 &=\exp\left(-\frac{2a}{\sum_{k=1}^n(b_k-a_k)^2}\right),
 \end{align*}
proving the claim.
\end{proof}



\section{Cram\'er's large deviation theorem}

The following is \textbf{Cram\'er's large deviation theorem}.\footnote{Achim Klenke, {\em Probability Theory: A Comprehensive Course},
p.~508, Theorem 23.3.}


\begin{theorem}[Cram\'er's  large deviation theorem]
Suppose that $X_n:(\Omega,\mathscr{F},P) \to \mathbb{R}$, $n \geq 1$, are independent identically
distributed random variables such that for all $t \in \mathbb{R}$,
\[
\Lambda(t) = \log E(e^{tX_1})<\infty.
\]
For $x \in \mathbb{R}$ define
\[
\Lambda^*(x) = \sup_{t \in \mathbb{R}} (tx-\Lambda(t)).
\]
If $a>E(X_1)$, then
\[
\lim_{n \to \infty} \frac{1}{n} \log P(S_n \geq an) = -\Lambda^*(a),
\]
where $S_n = \sum_{k=1}^n X_k$. 
\end{theorem}
\begin{proof}
For $a>E(X_1)$,
let $Y_n=X_n-a$, let
\[
L(t) = \log E(e^{tY_1}) = \log E(e^{tX_1}e^{-ta})=-ta+\Lambda(t)
\]
and let
\[
L^*(x) = \sup_{t \in \mathbb{R}} (tx-L(t)) = \sup_{t \in \mathbb{R}} (t(x+a)-\Lambda(t))
=\Lambda^*(x+a).
\]
Lastly, let $T_n=\sum_{k=1}^n Y_k=S_n - na$, with which
\[
P(T_n \geq bn) = P(S_n \geq (b+a)n).
\]
Thus, if we have
\[
\lim_{n \to \infty} \frac{1}{n} \log P(T_n \geq 0) = -L^*(0),
\]
then
\[
\lim_{n \to \infty} \frac{1}{n} \log P(S_n \geq an)=-L^*(0)
=-\Lambda^*(a).
\]
Therefore it suffices to prove the theorem for when $E(X_1)<0$ and
$a=0$. 



Define
\[
\phi(t) = e^{\Lambda(t)}=E(e^{tX_1}) = \int_\Omega e^{tX_1} dP = \int_\mathbb{R} e^{tx} d({X_1}_*P)(x), \qquad t \in \mathbb{R},
\]
the moment generating function of $X_1$,
and define 
\[
\rho = e^{-\Lambda^*(0)} = \exp\left( -\sup_{t \in \mathbb{R}}(-\Lambda(t))\right)
=\exp\left(\inf_{t \in \mathbb{R}} \Lambda(t) \right)
=\inf_{t \in \mathbb{R}} \phi(t),
\]
using that $x \mapsto e^x$ is increasing. 


Using the dominated convergence theorem, for $k \geq 0$ we obtain
\[
\phi^{(k)}(t) = \int_\mathbb{R} x^k e^{tx} d({X_1}_*P)(x)
=E(X_1^k e^{tX_1}).
\]
In particular, $\phi'(t)=E(X_1e^{tX_1})$, for which
$\phi'(0)=E(X_1)<0$, and $\phi''(t) = E(X_1^2 e^{tX_1})>0$ for all $t$ (either the expectation is $0$ or positive, and if it is $0$ then
$X_1^2 e^{tX_1}$ is $0$ almost everywhere, which contradicts $E(X_1)<0$).

Either $P(X_1 \leq 0)=1$ or $P(X_1 \leq 0)<1$. In the first case,
\[
\phi'(t) = \int_\Omega X_1 e^{tX_1} dP
=\int_{X_1 \leq 0} X_1 e^{tX_1} dP
\leq 0,
\]
so, using the dominated convergence theorem,
\[
\rho = \inf_{t \in \mathbb{R}} \phi(t) = \lim_{t \to \infty} \phi(t) = 
\int_{X_1 \leq 0} \left( \lim_{t \to \infty} e^{tX_1}\right) dP
=\int_{X_1=0} dP=P(X_1=0).
\]
Then 
\[
P(S_n \geq 0) = P(X_1=0,\ldots,X_n=0) = P(X_1=0) \cdots P(X_n=0)
=\rho^n.
\]
That is, as $a=0$,
\[
P(S_n \geq a)   = e^{-n\Lambda^*(a)},
\]
and the claim is immediate in this case.

In the second case, $P(X_1 \leq 0)<1$.  As $\phi''(t)>0$ for all $t$, there is some $\tau \in \mathbb{R}$ at
which $\phi(\tau)<\phi(t)$ for all $t \neq \tau$ (namely, a unique global minimum). Thus,
\[
\phi(\tau) = \rho, \qquad \phi'(\tau)=0.
\]
And $\phi'(0)=E(X_1)<0$, which with the above yields $\tau>0$.  
Because $\tau>0$, $S_n(\omega) \geq 0$ if and only if
$\tau S_n(\omega) \geq 0$ if and only if $e^{\tau S_n(\omega)} \geq 1$.  Applying Chebyshev's inequality, and because $X_n$ are independent,
\[
P(S_n \geq 0) = P(e^{\tau S_n}\geq 1) \leq E(e^{\tau S_n}) = E(e^{\tau X_1}) \cdots E(e^{\tau X_n})
=\phi(\tau)^n = \rho^n,
\]
thus $\log P(S_n \geq 0) \leq n \log \rho$ and then 
\[
\limsup_{n \to \infty} \frac{1}{n} \log P(S_n \geq 0) \leq \limsup_{n \to \infty} \log \rho = \log \rho
=\log e^{-\Lambda^*(0)}=-\Lambda^*(0).
\]

To prove the claim, it now suffices to prove that, in the case $P(X_1 \leq 0)<1$, 
\begin{equation}
\liminf_{n \to \infty} \frac{1}{n} \log P(S_n \geq 0) \geq \log \rho.
\label{suffices1}
\end{equation}
Let $\mu = {X_1}_*P$, and 
let
\[
d\nu(x) = \frac{e^{\tau x}}{\rho} d\mu(x).
\]
$\nu$ is a Borel probability measure: it is apparent that it is a Borel measure, and
\[
\nu(\mathbb{R}) = \int_\mathbb{R} d\nu(x) = \int_\mathbb{R}  \frac{e^{\tau x}}{\rho} d\mu(x)
=\frac{1}{\rho} \int_\mathbb{R} e^{\tau x} d\mu(x)
=\frac{\phi(\tau)}{\rho} =1.
\]
There are independent identically distributed random variables $Y_n$, $n \geq 1$,
each with ${Y_n}_*Q = \nu$.
\footnote{Gerald B. Folland, {\em Real Analysis: Modern Techniques and
Their Applications}, p.~329, Corollary 10.19.}
Define
\[
\psi(t) = E(e^{tY_1}) = \int_\mathbb{R} e^{tx} d\nu(x)
=\int_\mathbb{R} e^{tx} \frac{e^{\tau x}}{\rho} d\mu(x)
=\frac{1}{\rho} \int_\mathbb{R} e^{(t+\tau)x} d\mu(x)
=\frac{\phi(t+\tau)}{\rho},
\]
the moment generating function of $Y_1$.
As $\phi'(\tau)=0$,
\[
E(Y_1) = \psi'(0) = \frac{\phi'(\tau)}{\rho} = 0.
\]
As $\rho>0$ and $\phi''(t)>0$ for all $t$,
\[
\Var(Y_1) = E(Y_1^2) = \psi''(0) = \frac{\phi''(\tau)}{\rho} \in (0,\infty).
\]
For $T_n=\sum_{k=1}^n Y_k$,
using that the $X_n$ are independent and that the $Y_n$ are independent,
\begin{align*}
P(S_n \geq 0) &= \int_{x_1 + \cdots +x_n \geq 0} d({S_n}_*P)(x)\\
&=\int_{x_1+\cdots+x_n \geq 0} d\mu(x_1) \cdots d\mu(x_n)\\
&=\int_{x_1+\cdots+x_n \geq 0} \left(\frac{\rho}{e^{\tau x_1}} d\nu(x_1) \right) 
\cdots \left( \frac{\rho}{e^{\tau x_n}} d\nu(x_n) \right)\\
&=\rho^n \int_{x_1+\cdots+x_n \geq 0} e^{-\tau(x_1+\cdots+x_n)} d({T_n}_*Q).
\end{align*}
But
\begin{align*}
\int_{x_1+\cdots+x_n \geq 0} e^{-\tau(x_1+\cdots+x_n)} d({T_n}_*Q)&= 
\int_{T_n \geq 0} e^{-\tau T_n} dQ\\
&=E(1_{\{T_n \geq 0\}} \cdot e^{-\tau T_n}),
\end{align*}
hence
\[
P(S_n \geq 0) = \rho^n E(1_{\{T_n \geq 0\}} \cdot e^{-\tau T_n}).
\]
Thus,   \eqref{suffices1} is equivalent to
\[
\liminf_{n \to \infty} \frac{1}{n} \log \left(  \rho^n E(1_{\{T_n \geq 0\}} \cdot e^{-\tau T_n}) \right) \geq \log \rho,
\]
so, to prove the claim it suffices to prove that
\[
\liminf_{n \to \infty} \frac{1}{n} \log\left( E(1_{\{T_n \geq 0\}} \cdot e^{-\tau T_n})\right) \geq 0.
\]
For any $c>0$, 
\begin{align*}
\log\left( E(1_{\{T_n \geq 0\}} \cdot e^{-\tau T_n})\right) & \geq \log E\left(1_{\{0 \leq T_n \leq c \sqrt{n}\}} \cdot  e^{-\tau T_n} \right)\\
&\geq \log \left( e^{-\tau c\sqrt{n}} \cdot Q\left( 0\leq T_n \leq c\sqrt{n}\right) \right)\\
&=-\tau c \sqrt{n} + \log Q \left( \frac{T_n}{\sqrt{n}} \in [0,c]\right). 
\end{align*}
Because the $Y_n$ are independent  identically distributed $L^2$ random variables
with mean $0$ and variance $\sigma^2=\Var(Y_1)=\frac{\phi''(\tau)}{\rho}$,
the central limit theorem tells
us that  as $n \to \infty$,
\[
Q\left( \frac{T_n}{\sqrt{n}} \in [0,c] \right) \to \gamma_{0,\sigma^2}([0,c]), 
\]
where $\gamma_{a,\sigma^2}$ is the Gaussian measure, whose density with respect to Lebesgue measure on $\mathbb{R}$
is 
\[
p(t,a,\sigma^2)= \frac{1}{\sigma \sqrt{2\pi}} e^{-\frac{(t-a)^2}{2\sigma^2}}.
\]
Thus, because for $c>0$ we have  $\gamma_{0,\sigma^2}([0,c])>0$,
\begin{align*}
\liminf_{n \to \infty} \frac{1}{n} \log\left( E(1_{\{T_n \geq 0\}} \cdot e^{-\tau T_n})\right)
&\geq \liminf_{n \to \infty} \left(\frac{-\tau c}{\sqrt{n}} + \frac{1}{n} \log Q \left( \frac{T_n}{\sqrt{n}} \in [0,c]\right) \right)\\
&=\lim_{n \to \infty} -\frac{\tau c}{\sqrt{n}} + \lim_{n \to \infty} \frac{1}{n} \log Q \left( \frac{T_n}{\sqrt{n}} \in [0,c]\right)\\
&=\lim_{n \to \infty} \frac{1}{n} \log  \gamma_{0,\sigma^2}([0,c])\\
&=0,
\end{align*}
which completes the proof.
\end{proof}


For example, say that $X_n$ are independent identically distributed random variables with
${X_1}_*P=\gamma_{0,1}$. 
We calculate that the cumulant generating function $\Lambda(t)=\log E(e^{tX_1})$ is
\begin{align*}
\Lambda(t)&=\log \left( \int_\mathbb{R} e^{tx} d\gamma_{0,1}(x) \right)\\
&=\log \left(  \int_\mathbb{R} e^{tx} \frac{e^{-\frac{x^2}{2}}}{\sqrt{2\pi}} dx \right)\\
&=\log \left( \int_\mathbb{R} \frac{e^{-\frac{1}{2}(x-t)^2}}{\sqrt{2\pi}} e^{\frac{t^2}{2}}  dx \right)\\
&=\log e^{t^2}{2}\\
&=\frac{t^2}{2},
\end{align*}
thus $\Lambda(t)<\infty$ for all $t$. 
Then
\[
\Lambda^*(x) = \sup_{t \in \mathbb{R}} (tx - \Lambda(t))
=\sup_{t \in \mathbb{R}} \left(tx-\frac{t^2}{2} \right)
=\frac{x^2}{2}.
\]
Now applying Cram\'er's theorem we get that for $a>E(X_1)=0$,
for $S_n=\sum_{k=1}^n X_k$ we have
\[
\lim_{n \to \infty} \frac{1}{n} \log P(S_n \geq an)
=-\frac{a^2}{2}.
\]

Another example: If $X_n$ are independent identically distributed random
variables with the \textbf{Rademacher distribution}:
\[
{X_n}_*P = \frac{1}{2} \delta_{-1}+\frac{1}{2}\delta_1. 
\]
Then 
\[
E(e^{tX_1}) = \int_\mathbb{R} e^{tx} d\left(\frac{1}{2}\delta_{-1}+\frac{1}{2}\delta_1\right)(x)
=\frac{1}{2}e^{-t}+\frac{1}{2}e^t
=\cosh t,
\]
so the cumulant generating function of $X_1$ is
\[
\Lambda(t) = \log \cosh t,
\]
and indeed $\Lambda(t)<\infty$ for all $t \in \mathbb{R}$. 
Then, as $\frac{d}{dt}(tx-\log \cosh t) = x - \tanh t$,
\[
\Lambda^*(x) = \sup_{t \in \mathbb{R}} \left(tx - \log \cosh t\right)
=\arctanh x \cdot x- \log \cosh \arctanh x.
\]
For $x \in (-1,1)$, 
\[
\arctanh x = \frac{1}{2} \log \frac{1+x}{1-x}.
\]
Then
\[
\cosh \arctanh x = \frac{1}{2}\left(e^{\arctanh x} + e^{-\arctanh x}\right)
=\frac{1}{2} \sqrt{\frac{1+x}{1-x}} + \frac{1}{2} \sqrt{\frac{1-x}{1+x}} 
=\frac{1}{\sqrt{1-x^2}}.
\]
With these identities,
\begin{align*}
\Lambda^*(t)& = \frac{x}{2} \log \frac{1+x}{1-x} + \frac{1}{2} \log(1-x^2)\\
&=\frac{x}{2} \log (1+x) - \frac{x}{2} \log(1-x) + \frac{1}{2} \log(1+x) + \frac{1}{2} \log(1-x)\\
&=\frac{1+x}{2} \log(1+x) + \frac{1-x}{2} \log(1-x).
\end{align*}
With $S_n=\sum_{k=1}^n X_k$,
applying Cram\'er's theorem, we get that for 
any $a>E(X_1)=0$, 
\[
\lim_{n \to \infty} \frac{1}{n} \log P(S_n \geq an) = -\frac{1+x}{2} \log(1+x) - \frac{1-x}{2} \log(1-x).
\]

For a Borel probability measure $\mu$ on $\mathbb{R}$,
we define its \textbf{Laplace transform} $\check{\mu}:\mathbb{R} \to (0,\infty]$ by
\[
\check{\mu}(t) = \int_\mathbb{R} e^{ty} d\mu(y).
\]
Suppose that $\int_\mathbb{R} |y| d\mu(y)<\infty$ and let
 $M_1 = \int_\mathbb{R} y d\mu(y)$, the first moment of $\mu$.
For any $t$ the function $x \mapsto e^{tx}$ is convex, so by  Jensen's inequality,
\[
e^{tM_1} \leq \int_\mathbb{R} e^{ty} d\mu(y) = \check{\mu}(t).
\]
Thus for all $t \in \mathbb{R}$,
\[
t M_1 - \log \check{\mu}(t) \leq 0.
\]

For a Borel probability measure $\mu$ with finite first moment, we define
 its \textbf{Cram\'er transform}
$I_\mu:\mathbb{R} \to [0,\infty]$ by\footnote{Heinz Bauer, {\em Probability Theory},
pp.~89--90,  \S 12.}
\[
I_\mu(x) = \sup_{t \in \mathbb{R}} (tx - \log \check{\mu}(t)).
\]
For $t=0$, $tx-\log \check{\mu}(t)=-\log \check{\mu}(0)=-\log(1)=0$, which shows that
indeed $0 \leq I_\mu(x) \leq \infty$ for all $x \in \mathbb{R}$. 
But $tM_1 - \log \check{\mu}(t) \leq 0$ for all $t$ yields 
\[
I_\mu(M_1)  = 0.
\]





\end{document}