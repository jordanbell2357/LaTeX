\documentclass{article}
\usepackage{amsmath,amssymb,graphicx,subfig,mathrsfs,amsthm}
%\usepackage{tikz-cd}
\newcommand{\inner}[2]{\left\langle #1, #2 \right\rangle}
\newcommand{\tr}{\textrm{tr}} 
\newcommand{\im}{\textrm{im\,}} 
\newcommand{\Span}{\textrm{span}} 
\newcommand{\point}{\sigma_{\textrm{point}}}
\newcommand{\ap}{\sigma_{\textrm{ap}}}
\newcommand{\cont}{\sigma_{\textrm{cont}}}
\newcommand{\residual}{\sigma_{\textrm{res}}}
\newcommand{\id}{\textrm{id}} 
\newcommand{\norm}[1]{\left\Vert #1 \right\Vert}
\newtheorem{theorem}{Theorem}
\newtheorem{lemma}[theorem]{Lemma}
\newtheorem{corollary}[theorem]{Corollary}
\begin{document}
\title{Decomposition of the spectrum of a bounded linear operator}
\author{Jordan Bell\\ \texttt{jordan.bell@gmail.com}\\Department of Mathematics, University of Toronto}
\date{\today}


\maketitle

\section{Definitions}
Let $H$ be a complex Hilbert space. If $\lambda \in \mathbb{C}$, we also write $\lambda$ to denote $\lambda \cdot \id_H \in B(H)$.

For $T \in B(H)$, the {\em spectrum} $\sigma(T)$ of $T$ is the set of those $\lambda \in \mathbb{C}$ such that the map
$T-\lambda$ is not bijective.\footnote{$\sigma(T)$
is defined to be the set of $\lambda \in \mathbb{C}$ such that $v \mapsto Tv-\lambda v$ is not a bijection. It is a fact that
if $T \in B(H)$ and $v \mapsto Tv- \lambda v$ is a bijection then it is an element of $B(H)$. That it is linear
can be proved quickly. The fact that it is bounded is proved using the {\em open-mapping theorem}, which states
that a surjective bounded linear map from one Banach space to another is an open map, from which it follows
that a bijective bounded linear map from one Banach space to another has a bounded inverse.}  It happens that it is  useful for some purposes to write $\sigma(T)$ as a union of three particular
 disjoint subsets of itself.
\begin{itemize}
\item The {\em point spectrum} $\point(T)$ is the set of those $\lambda \in \mathbb{C}$
such that $T-\lambda$ is not injective. Equivalently, $\lambda \in \point(T)$ if $\lambda$ is an {\em eigenvalue} of $T$.\footnote{The point spectrum is often called the {\em discrete spectrum}. From the definition by itself it is not apparent what $\point(T)$ has to do either with points or discreteness.}
\item The {\em continuous spectrum} $\cont(T)$ is the set of those $\lambda \in \mathbb{C}$ such that
$T-\lambda$ is injective, has dense image, and is not surjective.
\item The {\em residual spectrum} $\residual(T)$ is the set of those $\lambda \in \mathbb{C}$ such that
$T-\lambda$ is injective and does not have dense image.
\end{itemize}
It is apparent that the sets $\point(T)$, $\cont(T)$, and $\residual(T)$ are disjoint, and that
\[
\sigma(T)=\point(T) \cup \cont(T) \cup \residual(T).
\] 
If $T \in B(H)$ then $\sigma(T) \neq \emptyset$, but any  of the above three sets may
be empty; they merely can't all be empty for a given operator.

\section{Properties}
\label{properties}
\subsection{Residual spectrum}
If $T \in B(H)$ is a normal operator then $\residual(T) = \emptyset$.\footnote{$T \in B(H)$ is {\em normal} if
$T^*T=TT^*$; in particular, a self-adjoint operator is normal. Equivalently, $T \in B(H)$ is normal if and only if
$\norm{Tv}=\norm{T^*v}$ for all $v \in H$.} We prove this. Suppose that
$T-\lambda$ is injective. We have to show that $\im (T-\lambda)$ is dense in $H$, and thus that $\lambda \not \in
\residual(T)$. ($\lambda$ might be in $\cont(T)$ or might not be in $\sigma(T)$;  we merely want to show that it is not
in $\residual(T)$.) We have
\[
H=\overline{ \im(T-\lambda)} \oplus (\im(T-\lambda))^\perp.
\]
Let $w \in (\im(T-\lambda))^\perp$; we have to show that $w=0$. For all $v \in H$,
\[
\inner{(T-\lambda)v}{w}=0,
\]
so for all $v \in H$ we have $\inner{v}{(T-\lambda)^*w}=0$ and therefore
$(T-\lambda)^*w=0$, so $w \in \ker (T-\lambda)^*=\ker(T-\lambda)$.\footnote{If $S$ is normal then
$\ker S=\ker S^*$. Proof: If $v \in \ker S$ then $\inner{Sv}{Sv}=0$, hence $\inner{S^*Sv}{v}=0$, hence
$\inner{SS^*v}{v}=0$, hence $\inner{S^*v}{S^*v}=0$, hence $S^*v=0$, hence $v \in \ker S^*$.}
 As $T-\lambda$ is injective, $w=0$, completing the proof.

\subsection{Point spectrum}
If $A \in B(H)$ is normal then it is straightforward to show
that $\ker A=\ker A^*$. Also, if $T \in B(H)$ is normal then for any $z \in \mathbb{C}$, $T-z$ is normal.
Thus, $\ker (T-z) = \{0\}$ if and only if $\ker ((T-z)^*)=\{0\}$. That is, $\lambda \in \point(T)$ if and only
if $\overline{\lambda} \in \point(T^*)$. 
For $X \subseteq \mathbb{C}$ we define
$X^*=\{\overline{z}:z \in X\}$. We have shown that if $T \in B(H)$ is normal then
\[
\point(T)^* = \point(T^*).
\]


\subsection{Continuous spectrum}
If $\lambda \in \cont(T)$, then $\im(T-\lambda)$ is dense in $H$. Also,
\[
(T-\lambda)^{-1}:\im(T-\lambda) \to H
\]
is a surjective linear map (the inverse of a linear map is itself a linear map) that is not continuous. For $\im(T-\lambda)$ is dense in $H$,
so if $(T-\lambda)^{-1}$ were continuous then it would
have a unique extension to a continuous, hence bounded, map $H \to H$. Using this and the fact that
$T-\lambda$ is not surjective will give a contradiction.



\subsection{Approximate point spectrum}
Let $\lambda \in \mathbb{C} \setminus (\point(T) \cup \residual(T))$. I claim that $\lambda \in \cont(T)$  if and only if $\lambda$ is in the {\em approximate
point spectrum} of $T$, the set of those $\lambda \in \mathbb{C}$ such that there is
is a sequence
$v_n \in H$ with  $\norm{v_n}=1$ and $\norm{(T-\lambda)v_n} \to 0$ as $n \to \infty$.
It is a fact that for $T \in B(H)$,
$T$ is invertible if and only if $T(H)$ is dense in $H$ and there is some $\alpha>0$ such that $\norm{Tv} \geq \alpha \norm{v}$ for all $v \in H$.\footnote{In words,
$T \in B(H)$ is invertible if and only if it has dense image and is {\em bounded below}. This result is proved
in Paul Halmos, {\em Introduction to Hilbert Space and the Theory
of Spectral Multiplicity}, p.~38, \S 21, Theorem 3.} If $\lambda 
\in \cont(T)$, then $T-\lambda$ is not invertible but the image of $T-\lambda$ is dense in $H$, then it must therefore be that
$T-\lambda$ is not bounded below. That is, there is no $\alpha>0$ such that for every $w \in H$ we have $\norm{(T-\lambda)w} \geq \alpha \norm{w}$.
Then for each $n$ there is some $w \in H$ such that $\norm{(T-\lambda)w_n} < \frac{1}{n} \norm{w_n}$. Let $v_n=\frac{w_n}{\norm{w_n}}$. We  have
$\norm{v_n}=1$ and $\norm{(T-\lambda)v_n}<\frac{1}{n}$, showing that $\lambda \in \ap(T)$.

On the other hand, if $\lambda \in \ap(T)$ then  there is a sequence $v_n \in H$ such that $\norm{(T-\lambda)v_n} \to 0$ as $n \to \infty$. Then for any $\alpha>0$, there is some $v_n$ such that
$\norm{(T-\lambda)v_n}<\alpha=\alpha \norm{v_n}$, so $T-\lambda$ is not invertible and hence $\lambda \in \sigma(T)$. Since we assumed that
$\lambda \in \mathbb{C} \setminus (\point(T) \cup \residual(T))$, we then have $\lambda \in \cont(T)$.

Halmos shows in Problem 62 of his {\em Hilbert Space Problem Book} that
$\ap(T)$ is a closed subset of $\mathbb{C}$, and proves in Problem 63 that $\partial \sigma(T) \subseteq \ap(T)$: the boundary of the spectrum of $T$
is contained in the approximate point spectrum of $T$.

\section{Normal operators}
We showed earlier that if $T \in B(H)$ is normal then $\point(T)^*=\point(T^*)$, where, for
$X \subseteq \mathbb{C}$, $X^*=\{\overline{z}:z \in X\}$. This is one reason why it can be helpful to know that an operator
is normal. Using this we can show something more about normal operators. 
Let $T \in B(H)$ be normal and suppose that $\lambda,\mu \in \point(T)$ are distinct. Then there are nonzero
$v,w \in H$ with $Tv=\lambda v$ and $Tw=\mu w$. Using $T^*w=\overline{\mu}w$ we get 
\begin{eqnarray*}
\lambda \inner{v}{w}&=&\inner{\lambda v}{w}\\
&=&\inner{Tv}{w}\\
&=&\inner{v}{T^*w}\\
&=&\inner{v}{\overline{\mu}w}\\
&=&\mu \inner{v}{w}.
\end{eqnarray*}
As $\lambda \neq \mu$, this means that $\inner{v}{w}=0$. In words, if $T \in B(H)$ is normal, then its
eigenspaces are mutually orthogonal.


\section{Compact operators}
Let $K(H)$ be the closure in $B(H)$
of the set of finite-rank operators. We call the elements of $K(H)$ {\em compact operators}. The following are equivalent ways to state that an operator
is compact.
\begin{itemize}
\item $T \in B(H)$ is compact if and only if
for every bounded subset $S$ of $H$, the closure of the image $T(S)$ is compact.
\item $T \in B(H)$ is compact if and only if for every sequence $v_n \in H$ with $\norm{v_n}=1$, 
$T(v_n)$ has a convergent subsequence.
\item Let $B$ be the closed unit ball in $H$, and let $B$ be a topological space with the {\em weak topology}: a net $v_\alpha \in B$ converges weakly to
$v \in B$ if for all $w \in H$ we have $\inner{v_\alpha}{w} \to \inner{v}{w}$. If $H$ is separable, then the weak topology on $B$ is metrizable and thus can be characterized
using merely sequences instead of nets.\footnote{This is proved
in Paul Halmos, {\em Hilbert Space Problem Book}, Problem 18.} $T \in B(H)$ is compact if and only if the restriction of $T$ to $B$ is continuous
$B \to H$, where $B$ has the weak topology and $H$ has the norm topology.
\end{itemize}

If $K \in K(H)$  and $V$ is a closed subspace of $V$ such that $K(V) \subseteq V$, then the restriction of $K$ to $V$ is an element of $K(V)$.

$B(H)$ is a $C^*$-algebra, and $K(H)$ is a $C^*$-subalgebra of $B(H)$. 
If $T \in K(H)$ and $S \in
B(H)$, then
\[
ST, TS \in K(H).
\]
Hence $K(H)$ is an {\em ideal} of the $C^*$-algebra $B(H)$.\footnote{The $C^*$-algebra
$B(H)/K(H)$ is called the {\em Calkin algebra} of $H$. $T \in B(H)$ is called a {\em Fredholm operator}
if $T+K(H)$ is an invertible element of the Calkin algebra.
 In particular, if $T \in K(H)$ then $\id_H-T$ is a Fredholm operator.
 
 $T \in B(H)$ is a Fredholm operator if and only if the following three conditions holds: $\im T$ is closed in $H$, 
 $\ker T$ is finite dimensional, and $\ker T^*$ is finite dimensional. This equivalence is called {\em Atkinson's theorem}.
The {\em index} of a Fredholm operator is $\dim \ker T - \dim \ker T^*$. If $T \in K(H)$, then $\id_H -T$ has index $0$.}

Useful facts about compact operators are proved in
Yuri A. Abramovich and Charalambos D. Aliprantis,
{\em An Invitation to Operator Theory}, 
p.~272, \S 7.1.

\section{Fredholm alternative}
The {\em Fredholm alternative} states that
if $K \in K(H)$, $\lambda \neq 0$, and  $\ker(K-\lambda) = \{0\}$ then $(K-\lambda)^{-1} \in B(H)$.\footnote{We are following Paul Halmos, {\em Hilbert Space Problem Book}, 
p.~293, Problem 140.} Equivalently, 
if $K \in K(H)$, $\lambda \neq 0$, and $\lambda \not \in \point(K)$ then
$\lambda \not \in \sigma(K)$.
Equivalently, if
$K \in K(H)$, then
\[
\sigma(K) \subseteq \point(K) \cup \{0\}.
\]
The above forms are the ones that we want to use. The following is the one that we want to prove, which is equivalent because
a nonzero multiple of a compact operator is compact: If $K \in K(H)$ and $\ker(\id_H - K)=\{0\}$ then $(\id_H-K)^{-1} \in B(H)$.

We prove two standalone lemmas that we then use to prove the Fredholm alternative.
\begin{itemize}
\item  Let $K \in K(H)$ let $A=\id_H - K \in B(H)$,
and suppose that $A(H) = H$. Define $K_n=\ker(A^n)$. We have $K_1 \subseteq K_2 \subseteq \cdots$. Assume by contradiction that $K_1 \neq \{0\}$.  Then there is some nonzero $f_1 \in K_1$.
As $A(H) = H$, there is some $f_2 \in H$ with $Af_2=f_1$; but $A^2 f_2=Af_1=0$ and $Af_2=f_1 \neq 0$, so $f_2 \in K_2 \setminus K_1$. Let $f_{n+1} \in K_{n+1} \setminus 
K_n$ with $Af_{n+1}=f_n$. Therefore $K_1, K_2, \ldots$ are a strictly increasing sequence of subspaces of $H$. Using Gram-Schmidt, there is an orthonormal
sequence $e_1,e_2,\ldots$ with $e_n \in K_n$ for all $n$; we caution that we do not necessarily
have $Ae_{n+1}=e_n$. As $Ae_{n+1} \in K_n$, $\inner{Ae_{n+1}}{e_{n+1}}$, giving 
\[
\norm{Ke_{n+1}}^2=\norm{e_{n+1}-Ae_{n+1}}^2=\norm{e_{n+1}}^2+\norm{Ae_{n+1}}^2 \geq 1\norm{e_{n+1}}^2=1.
\]
Each $e_n$ is an element of the closed unit ball $B$, and $e_n \to 0$ weakly (this is the case for any orthonormal sequence in $H$, basis or not, and is proved using Bessel's inequality). 
Since $K$ is compact, it is continuous $B \to H$ where $B$ has the weak topology and $H$ has the norm topology; but $e_n \to 0$, $K(0)=0$,
and $\norm{Ke_n} \geq 1$, so $Ke_n$ does not converge to $0$ in $H$, a contradiction. Therefore $K_1 = \{0\}$, that is, $\ker A=\{0\}$.

\item Let $K \in K(H)$ and let $A=\id_H - K \in B(H)$.  Suppose by contradiction that $A$ is not bounded below on
$(\ker A)^\perp$. So for every $\alpha>0$ there is some $w \in (\ker A)^\perp$ such that $\norm{Aw} \geq \alpha \norm{w}$. Then for
all $n$ there is some $w_n \in (\ker A)^\perp$ with $\norm{Aw_n} < \frac{1}{n} \norm{w_n}$. Let $v_n = \frac{w_n}{\norm{w_n}} \in (\ker A)^\perp$. Then
\[
\norm{Av_n} < \frac{1}{n}.
\]
As $K$ is compact, there is
some subsequence $v_{a(n)}$ such that $Kv_{a(n)}$ converges to some
$v \in H$. 
$Av_n \to Av$ and $Av_n \to 0$, so $v \in \ker A$.
On the other hand,
because $v_n=Av_n+Kv_n \to v$ and $(\ker A)^\perp$ is closed, we have $v \in (\ker A)^\perp$, so
$v \in \ker A \cap (\ker A)^\perp = \{0\}$.
But as $\norm{v_n}=1$ for each $n$, we have $\norm{v}=1$, a contradiction.
Therefore, $A$ is bounded below on $(\ker A)^\perp$.
\end{itemize}

If $K \in K(H)$ and $A=\id_H - K$, then
by the second of the two lemmas, we have that $A$ is bounded below on $(\ker A)^\perp$: there is some $\alpha>0$ such that $\norm{Av} \geq \alpha \norm{v}$ for all
$v \in (\ker A)^\perp$. If
\[
w_n \in A(H) = A(\ker A \oplus (\ker A)^\perp) = A((\ker A)^\perp)
\]
with $w_n \to w \in H$, then there are $v_n \in (\ker A)^\perp$ such that $Av_n=w_n$. As $w_n$ converges, for all $\epsilon>0$ there is some $N$ such that
if $n,m \geq N$ then $\norm{w_n-w_m} \leq \epsilon$, so $\norm{A(v_n-v_m)}=\norm{Av_n-Av_m} \leq \epsilon$. But
\[
\norm{A(v_n-v_m)} \geq \alpha \norm{v_n-v_m},
\]
so
\[
\norm{v_n-v_m} \leq \frac{\epsilon}{\alpha},
\]
so $v_n$ is a Cauchy sequence and hence converges, say to $v$. $v_n \in (\ker A)^\perp$, which is closed, so $v \in (\ker A)^\perp$. Then $w_n = Av_n \to Av \in A(H)$. Therefore, if $K \in K(H)$ and $A=\id_H -K$, then $A(H)$ is closed in $H$.

Let $K \in K(H)$ and $A=\id_H - K$, and suppose that $\ker A = \{0\}$. By the above paragraph, $A(H)$ is closed in $H$. $K^* \in K(H)$ and $A^*=\id_H - K^*$,\footnote{The
adjoint of a compact operator is itself a compact operator. This is true even for Banach spaces, and a proof of this is given by
Paul Garrett in his note {\em Compact operators on Banach spaces: Fredholm-Riesz}.
A bounded linear map $K:X  \to Y$, where $X$ and $Y$ are Banach spaces, is said to be {\em compact}
if for every bounded subset $S$ of $X$, the closure of $K(S)$ in $Y$ is  compact.}
so by the above
paragraph we also get that $A^*(H)$ is closed in $H$. It is a fact that if $T \in B(H)$ then $\ker T^* = (T(H))^\perp$, so using this with $T=A^*$ we get
\[
\ker A = (A^*(H))^\perp;
\]
 taking orthogonal complements and using the fact that the double orthogonal complement of a subspace is its closure and that
 $A^*(H)$ is closed, we obtain
\[
(\ker A)^\perp = A^*(H).
\]
Since $\ker A=\{0\}$, we have $A^*(H)=H$. Then we can apply the first of the two lemmas: as $A^*=\id_H - K^*$, $K^* \in K(H)$, and $A^*(H)=H$, we have
$\ker A^*=\{0\}$. 
We now apply the second of the two lemmas: $A^*$ is bounded below on $(\ker A^*)^\perp=H$. Using the fact that if $T \in B(H)$ is bounded below and has dense image then
$T^{-1} \in B(H)$, we get $(A^*)^{-1} \in B(H)$ ($A^*(H)=H$ so $A^*$ certainly has dense image). Taking adjoints commutes with taking inverses,
so $A^{-1} \in B(H)$. This completes the proof of the Fredholm alternative.



\section{Compact self-adjoint operators}
It is a fact that
if $T \in B(H)$ is self-adjoint then\footnote{This is proved in Anthony W. Knapp, {\em Advanced Real Analysis},
p. ~37, Proposition 2.2.}
\[
\norm{T}=\sup_{\norm{v} \leq 1} |\inner{Tv}{v}| = \sup_{\norm{v}=1} |\inner{Tv}{v}|.
\]

Let $T \in B(H)$ be compact and self-adjoint and $T \neq 0$.
Since $T$ is self-adjoint, $\inner{Tv}{v} \in \mathbb{R}$, so either $\norm{T}=\sup_{\norm{v}=1} \inner{Tv}{v}$ or
$\norm{T}=-\inf_{\norm{v}=1} \inner{Tv}{v}$. Say the first is the case. Let $\norm{v_n}=1$ and $\inner{Tv_n}{v_n} \to \norm{T}$ as
$n \to \infty$. Then, as $T=T^*$,
\begin{eqnarray*}
\inner{Tv_n-\norm{T}v_n}{Tv_n-\norm{T}v_n}&=&\inner{Tv_n}{Tv_n}-\inner{Tv_n}{\norm{T}v_n}-\inner{\norm{T}v_n}{Tv_n}\\
&&+\inner{\norm{T}v_n}{\norm{T}v_n}\\
&=&\norm{Tv_n}^2-2\norm{T}\inner{Tv_n}{v_n}+\norm{T}^2 \norm{v_n}^2\\
&\leq&\norm{T}^2 \norm{v_n}^2-2\norm{T}\inner{Tv_n}{v_n}+\norm{T}^2 \norm{v_n}^2\\
&=&2\norm{T}^2-2\norm{T}\inner{Tv_n}{v_n}.
\end{eqnarray*}
Thus $\norm{Tv_n-\norm{T}v_n} \to 0$ as $n \to \infty$, that is, $Tv_n - \norm{T}v_n \to 0$. On the other hand, as $\norm{v_n}=1$ for each $n$, there is some subsequence
$v_{a(n)}$ such that $Tv_{a(n)}$ converges, say to $v$. Together with $Tv_n -\norm{T}v_n \to 0$ this gives
$\norm{T}v_{a(n)} \to v$ as $n \to \infty$, from which we get $\norm{v}=\frac{1}{\norm{T}}>0$. Thus $v \neq 0$. And
\[
(T-\norm{T})v=(T-\norm{T})\lim_{n \to \infty} v_{a(n)} = \lim_{n \to \infty} (T-\norm{T}) v_{a(n)} = 0,
\]
which means that $\norm{T} \in \point(T)$. Likewise, in the case $\norm{T}=-\inf_{\norm{v}=1} \inner{Tv}{v}$ we get
$-\norm{T} \in \point(T)$.

\section{Multiplication operators}
\subsection{The Hilbert space $L^2(X)$}
Let $(X,\Sigma,\mu)$ be a $\sigma$-finite measure space. $L^2(X)$ is a Hilbert space\footnote{Whether
 $L^2(X)$ is separable depends on the measure space $(X,\mu)$. Let $\mathscr{S}$ be the set of all measurable subsets
 of $X$ with finite measure, and let $\rho(A,B)=\mu(A \cup B \setminus A \cap B)$, the measure
 of the symmetric difference of $A$ and $B$. One shows that $\mathscr{S}$ is a pseudometric space with pseudometric $\rho$. It is a fact
 that $L^2(X)$ is separable if and only if $\mathscr{S}$ is separable; cf. Paul Halmos, {\em Measure Theory},
 p. 177, \S 42. For this to be the case, it suffices that $X$ is $\sigma$-finite and that its $\sigma$-algebra
 is countably generated.} with inner product
\[
\inner{f}{g}=\int f g^* d\mu,
\]
where $g^*(x)=\overline{g(x)}$.

\subsection{The $C^*$-algebra homomorphism $\phi \mapsto M_\phi$}
A {\em multiplication operator} on $L^2(X)$ is an operator $M_\phi:L^2(X) \to L^2(X)$, $\phi \in L^\infty(X)$, of the form
 \[
 (M_\phi f)(x)=\phi(x) f(x).
 \]
 As 
 \[
 \norm{M_\phi f}^2=\int_X \phi(x) f(x) \overline{\phi(x) f(x)} d\mu(x) \leq \norm{\phi}_\infty^2 \norm{f}^2,
 \]
 where $\norm{\phi}_\infty$ is the essential supremum of $\phi(x)$ for $x \in X$, we have $\norm{M_\phi} \leq \norm{\phi}$ and so $M_\phi \in B(H)$. $L^\infty(X)$ is a $C^*$-algebra, and so is
 $B(L^2(X))$. If $X$ is $\sigma$-finite then I claim that
 \[
 \phi \mapsto M_\phi, \qquad L^\infty(X) \to B(L^2(X))
 \]
 is an injective homomorphism of $C^*$-algebras. It is straightforward to show that this map is a homomorphism of $C^*$-algebras, and this 
 does not use the assumption that $X$ is $\sigma$-finite.
For our benefit, we shall show that
 $\phi \mapsto M_\phi$ is injective. 
 If $\phi \neq 0$, then
$\norm{\phi}_\infty>0$, so for  
\[
E=\{x \in X:|\phi(x)| \geq \frac{1}{2} \norm{\phi}_\infty\}
\]
 we have $0 < \mu(E) \leq \infty$. Because $(X,\mu)$ is $\sigma$-finite, there is
some subset $F$ of $E$ with $0<\mu(F)<\infty$. As $f=\phi^* \cdot \chi_F \in L^2(X)$, we have
\begin{eqnarray*}
M_\phi f &=&\int_F \phi(x) \overline{\phi(x)} d\mu(x)\\
&\geq&\int_F \frac{1}{4} \norm{\phi}_\infty^2  d\mu(x)\\
&=& \frac{1}{4} \norm{\phi}_\infty^2 \cdot \mu(F)\\
&>&0,
\end{eqnarray*}
so $M_\phi \neq 0$. 
 Generally, an injective homomorphism of $C^*$-algebras is an isometry, so $\norm{M_\phi}=\norm{\phi}_\infty$. 

As $M_{\phi \phi^*}=M_\phi M_{\phi^*}=M_\phi M_\phi^*$ and $M_{\phi \phi^*}=M_{\phi^* \phi}$, we have $M_\phi M_\phi^*=M_\phi^* M_\phi$, namely,
 a multiplication operator is a normal operator.
Since residual spectrum of a normal operator is empty, the 
residual spectrum of a multiplication operator is empty.

\subsection{Essential range}
For $\phi \in L^\infty(X)$, we define the {\em essential range} of $\phi$ to be the set 
\[
\Big\{z \in \mathbb{C}:\textrm{if $\epsilon>0$ then $\mu(\{x \in X:|f(x)-z|<\epsilon\})>0$}\Big\}.
\]
Equivalently, the essential range of $\phi$ is the set of those $z \in \mathbb{C}$ such that for all $\epsilon>0$,
\[
\mu(\phi^{-1}(D_\epsilon(z)))>0,
\]
in words, those $z \in \mathbb{C}$ such that the inverse image of every $\epsilon$-disc about $z$ has positive measure.
Equivalently, the essential range of $\phi$ is the intersection of all closed subsets $K$ of $\mathbb{C}$ such that for almost all $x \in X$, $\phi(x) \in K$.
It is a fact that if $\phi \in L^\infty(X)$ then the essential range of $\phi$ is a compact subset of $\mathbb{C}$.

Let $\phi \in L^\infty(X)$.
If $\lambda$ is not in the essential range of $\phi$, then there is some $\epsilon>0$ such that $\mu(\phi^{-1}(D_\epsilon(\lambda)))=0$,
which means that for almost all $x \in X$ we have $|\phi(x)-\lambda| \geq \epsilon$.
Define $\psi(x)=\frac{1}{\phi(x)-\lambda}$. For almost all $x \in X$,
\[
|\psi(x)|=\frac{1}{|\phi(x)-\lambda|} \leq \frac{1}{\epsilon},
\]
hence $\psi \in L^\infty(X)$. Then
\[
M_{\psi} M_{\phi-\lambda} = M_{\phi-\lambda} M_{\psi}=M_{(\phi-\lambda)\cdot \psi}=M_{1}=\id_{L^2(X)},
\]
so $M_{\phi-\lambda}$ is invertible. But $M_{\phi-\lambda}=M_\phi - \lambda$, so $M_\phi-\lambda$ is invertible and hence $\lambda \not \in
\sigma(M_\phi)$.

If $\lambda$ is in the essential range of $\phi$, then for each $n$ we have $0 < \mu(\phi^{-1}(D_{1/n}(\lambda))) \leq \infty$; 
since $\Sigma$ is $\sigma$-finite, for each $n$ there is a subset $E_n$ of 
$\phi^{-1}(D_{1/n}(\lambda))$ with $0 < \mu(E_n) < \infty$, and so $\chi_{E_n} \in L^2(X)$.
We have, since $|\phi(x)-\lambda|<\frac{1}{n}$ for $x \in E_n$,
\begin{eqnarray*}
\norm{(M_\phi-\lambda) \chi_{E_n}}^2&=&\int_X |(\phi(x)-\lambda)\chi_{E_n}(x)|^2 d\mu(x)\\
&=&\int_{E_n} |\phi(x)-\lambda|^2 d\mu(x)\\
&\leq&\frac{1}{n^2} \int_{E_n} d\mu(x)\\
&=&\frac{1}{n^2} \int_X \chi_{E_n} d\mu(x)\\
&=&\frac{1}{n^2} \norm{\chi_{E_n}}^2,
\end{eqnarray*}
so for each $n$,
\[
\norm{(M_\phi-\lambda) \chi_{E_n}} \leq \frac{1}{n} \norm{\chi_{E_n}}.
\]
It follows that $M_\phi-\lambda$ is not invertible, as it is not bounded below. Therefore $\lambda \in \sigma(M_\phi)$.
Therefore the essential range of $\phi \in L^\infty(X)$ is equal to the spectrum of $M_\phi \in B(L^2(X))$.

We say that $\phi \in L^\infty(X)$ is {\em invertible} if there is some $\psi \in L^\infty(X)$ such that $\phi(x)\psi(x)=1$ for almost all $x \in X$. (It would not make sense to demand
that $\phi(x)\psi(x)=1$ for all $x \in X$.) For $\phi \in L^\infty(X)$ to be invertible, it is necessary and sufficient that there is some $\alpha>0$ such that
$|\phi(x)| \geq \alpha$ for almost all $x \in X$ (lest its inverse not have an essential supremum). 


\subsection{Isolated elements of the essential range}
If  $\lambda$ is not just an element of the essential range $\phi$ 
 but is an isolated element of the essential range, then we can say more
than just that $\lambda \in \sigma(M_\phi)$. In this case, there is some $\epsilon>0$ such that the intersection
of
$D_\epsilon(\lambda)$  and the essential range of $\phi$ is equal to the singleton $\{\lambda\}$. Let $E$ be a subset of $\phi^{-1}(D_\epsilon(\lambda))$ with
$0<\mu(E)<\infty$.  For almost all $x \in X$, $\phi(x)$ is an element of the essential range of $\phi$, hence for almost all  $x \in E$ we have $\phi(x)=\lambda$. Therefore,
 for almost all  $x \in X$ we have
\[
(M_\phi \chi_E)(x)=\phi(x) \chi_E(x)=\lambda \chi_E(x).
\]
Hence $(M_\phi - \lambda)\chi_E=0$, and as $\mu(E)>0$ we have $\chi_E \neq 0$. Therefore, if $\lambda$ is an isolated element of the essential range of $\phi$ then
$\lambda \in \point(M_\phi)$.\footnote{If $\lambda$ is an isolated element of the essential range of $\phi$ then 
one finds that the inverse image of the singleton
$\{\lambda\}$ 
has positive measure.
I would be surprised if this were not the origin of the term {\em point spectrum}. Being isolated corresponds to being discrete.}






\section{Functional calculus}
Let $T \in B(H)$ be self-adjoint. The spectrum $\sigma(T)$ is a compact
subset of $\mathbb{R}$, and one checks that the set $C(\sigma(T))$ 
of continuous functions $\sigma(T) \to \mathbb{C}$ is a $C^*$-algebra, with norm
$\norm{f}=\sup_{\lambda \in \sigma(T)} |f(\lambda)|$. 

Let $\mathbb{C}[x]$ be the set of polynomials with complex coefficients. For
$T \in B(H)$ self-adjoint and $p(x)=\sum_{k=0}^n a_k x^k \in \mathbb{C}[x]$, we define 
\[
p(T) \in B(H)
\]
by
\[
p(T)=\sum_{k=0}^n a_k T^k.
\]
$T \mapsto p(T)$ is a homomorphism of $C^*$-algebras, where\footnote{If we had not stipulated that $T$ be self-adjoint then
we would have to define the conjugation of polynomials as conjugation of polynomial functions: for $p$ defined by
$p(z)=\sum_{k=0}^n a_k z^k$, then
$p^*$ is defined by $p^*(z)=\sum_{k=0}^n \overline{a_k} (\overline{z})^k$.}
\[
\left(\sum_{k=0}^n a_k x^k \right)^*=\sum_{k=0}^n \overline{a_k} x^k.
\]
It is a fact that\footnote{See Paul Halmos, {\em Hilbert Space Problem Book}, 
p. 62, Problem 97.}
\[
\sigma(p(T))=p(\sigma(T)),
\]
 where for $M \subseteq \mathbb{C}$
we define
$p(M)=\{p(z):z \in M\}$. It is also a fact that 
\[
\norm{p(T)}=\norm{p}=\sup_{\lambda \in \sigma(T)} |p(\lambda)|;
\]
this is proved using the result that the norm of a normal operator $T$ is equal to its {\em spectral radius}, which is given by the two
following expressions that one proves are equal:
\[
r(T)=\lim_{n \to \infty} \norm{T^n}^{1/n}=\max_{\lambda \in \sigma(T)} |\lambda|.
\]

The above is used to define $f(T)$ for any continuous function $f:\sigma(T) \to \mathbb{C}$. This map
$C(\sigma(T)) \to B(H)$
is called the {\em continuous functional calculus}. It is an isometric homomorphism of $C^*$-algebras.
The continuous functional calculus can be 
used to prove things about the spectrum of self-adjoint operators that do not obviously have
to do with making sense of continuous functions applied to these operators.

Let $T \in B(H)$ be
self-adjoint and let $\lambda \in \sigma(T)$ be an isolated point in $\sigma(T)$. I will show that that $\lambda 
\in \point(T)$. Since $\lambda$ is isolated in $\sigma(T)$, the function $f:\sigma(T) \to \mathbb{C}$
defined by 
\[
f(z)=\begin{cases}
1&z=\lambda,\\
0&z \neq \lambda,
\end{cases}
\]
is continuous. Since $f$ is continuous, $f(T) \in B(H)$, and because $f=f^*$, $f(T)$ is self-adjoint. Let $P=f(T)$. As $\norm{P}=\norm{f}=1$,
$P \neq 0$. Define $g \in C(\sigma(T))$ by $g(x)=(x-\lambda)f(x)$. Then $g(x)=0$ for all $x \in \sigma(T)$, so
$g(T)=0$, i.e.
\[
(T-\lambda)P=0.
\]
Hence $\im P \subseteq \ker(T-\lambda)$. As $P \neq 0$, there is some $v \in H$ with $Pv \neq 0$.
Then $(T-\lambda)Pv=0$, $Pv \neq 0$, so
$\lambda \in \point(T)$.


\section{Spectral measures}
It is a fact that if $f \geq 0$ then $f(T) \geq 0$, where, for a self-adjoint operator $T$, $T \geq 0$ means $\inner{Tv}{v} \geq 0$ for all $v \in H$.
For $T \in B(H)$ self-adjoint and $v \in H$, using the continuous functional calculus talked about in the previous section we
define $\phi:C(\sigma(T)) \to \mathbb{C}$ by
\[
\phi(f)=\inner{v}{f(T)v}.
\]
$\phi$ is a  {\em positive linear functional}: if $f$ is real-valued and $f(x) \geq 0$ for all $x \in \sigma(T)$, then 
$\phi(f) \geq 0$. $\sigma(T)$ is indeed a locally compact Hausdorff space
and since $\sigma(T)$ is compact the continuous functions of compact support on $\sigma(T)$ are precisely
the continuous functions on $\sigma(T)$, so we satisfy the conditions
of  the Riesz-Markov theorem. Thus there exists a 
unique
regular Borel measure $\mu$ on the Borel $\sigma$-algebra of $\sigma(T)$ such that, for all 
$f \in C(\sigma(T))$, 
\[
\inner{v}{f(T)v}=\phi(f)=\int_{\sigma(T)} f(x) d\mu(x).
\]
 {\em Lebesgue's decomposition theorem} states that
\[
\mu=\mu_{\textrm{ac}}+\mu_{\textrm{sing}}+\mu_{\textrm{pp}},
\]
where
\begin{itemize}
\item $\mu_{\textrm{ac}}$ is absolutely continuous with respect to Lebesuge measure: if $A$ is a measurable
subset of $\sigma(T)$ and its Lebesgue measure is $0$, then $\mu_{\textrm{ac}}(A)=0$.
\item $\mu_{\textrm{sing}}$ and Lebesgue measure are mutually
singular,\footnote{There are disjoint measurable sets $A$ and $B$
with $A \cup B = \sigma(T)$ such that $\mu_{\textrm{sing}}(A)=0$ and the Lebesgue measure of $B$ is $0$.} and if $\lambda \in \sigma(T)$ then $\mu_{\textrm{sing}}(\{\lambda\})=0$.
\item  There is a countable subset $J$ of $\sigma(T)$ such that
\[
\mu_{\textrm{pp}}=\sum_{\lambda \in J} a_\lambda \delta_\lambda, \qquad a_\lambda \in \mathbb{C}.
\]
\end{itemize}

We define $H_{\textrm{ac}}$ to be the set of those $v \in H$ such that $\mu$ is equal to the absolutely
continuous part of its Lebesgue decomposition, i.e. the other two parts are 0, and we define 
 $H_{\textrm{sing}}$ and $H_{\textrm{pp}}$ likewise. (Note that we first took $v \in H$ and then defined $\mu$ using $v$.) One proves that 
 $H_{\textrm{ac}}$, $H_{\textrm{sing}}$ and $H_{\textrm{pp}}$
  are closed subspaces of $H$ and that
 they are invariant under $T$, and defines
the {\em absolutely continuous spectrum} of $T$ to be the spectrum of the restriction of $T$ to $H_{\textrm{ac}}$;
the {\em singular spectrum} of $T$ to be the spectrum of the restriction of $T$ to $H_{\textrm{sing}}$;
and the {\em pure point spectrum} of $T$ to be the spectrum of the restriction of $T$ to
$H_{\textrm{pp}}$. It is a fact that\footnote{See Reed and Simon, {\em Methods of Modern Mathematical Physics. I: Functional Analysis}, revised and enlarged ed.,
p.~231, \S VII.2.}
\[
\sigma(T) =\sigma_{\textrm{ac}}(T) \cup  \sigma_{\textrm{sing}}(T) \cup \overline{\sigma_{\textrm{pp}}(T)},
\]
but these three sets might not be disjoint.


\end{document}