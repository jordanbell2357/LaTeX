\documentclass{article}
\usepackage{amsmath,amssymb,graphicx,subfig,mathrsfs,amsthm}
\usepackage{tikz-cd}
\usepackage[draft]{hyperref}
\newcommand{\inner}[2]{\left\langle #1, #2 \right\rangle}
\newcommand{\Tr}{\textrm{Tr}\,} 
\newcommand{\Span}{\textrm{span}} 
\newcommand{\SA}{B_{\textrm{sa}}(H)} 
\newcommand{\positive}{B_{\textrm{+}}(H)} 
\newcommand{\id}{\textrm{id}} 
\newcommand{\norm}[1]{\left\Vert #1 \right\Vert}
\newtheorem{theorem}{Theorem}
\newtheorem{lemma}[theorem]{Lemma}
\newtheorem{corollary}[theorem]{Corollary}
\begin{document}
\title{Complexification, complex structures, and linear ordinary differential equations}
\author{Jordan Bell}
\date{April 3, 2014}
\maketitle

\section{Motivation}
The solution of the initial value problem 
\[
x'(t)=Ax(t), \qquad x(0)=x_0 \in \mathbb{R}^n,
\]
where $A$ is an $n \times n$ matrix over $\mathbb{R}$,
is $x(t)=\exp(At)x_0$. If we want to compute the solution and if $A$ is diagonalizable, say $A=PDP^{-1}$, we use
\[
\exp(At)=\exp((PDP^{-1})t)=P\exp(Dt)P^{-1}.
\]
Thus if the matrix $A$ has complex eigenvalues, then although $\exp(At)x_0 \in \mathbb{R}^n$, it may not be the case that $P^{-1}x_0 \in \mathbb{R}^n$. 
For example, if $A=\begin{pmatrix}0&-1\\1&0\end{pmatrix}$, then
\[
D=\begin{pmatrix}-i&0\\0&i\end{pmatrix}, \quad
P=\begin{pmatrix}-i&i\\1&1\end{pmatrix}, \quad
P^{-1}=\frac{1}{2}\begin{pmatrix}i&1\\-i&1\end{pmatrix}.
\]
For $x_0=\begin{pmatrix}1\\0\end{pmatrix}$, 
\[
P^{-1}x_0=\frac{1}{2}\begin{pmatrix}i&1\\-i&1\end{pmatrix} \begin{pmatrix}1\\0\end{pmatrix}
=\frac{1}{2} \begin{pmatrix}i\\-i\end{pmatrix}.
\]
This is similar to how Cardano's formula, which expresses the roots of a real cubic polynomial in terms of its coefficients,
involves complex numbers and yet the final result may still be real.

In the following, unless I specify the dimension of a vector space, any statement about real vector spaces is about real vector spaces of finite or infinite dimension, and any
statement about complex vector spaces is about complex vector spaces of finite or infinite dimension. 


\section{Direct sums}
If $V$ is a real vector space, a {\em complex structure} for $V$ is an $\mathbb{R}$-linear map $J:V \to V$ such that $J^2=-\id_V$.

If $V$ is a real vector space and $J:V \to V$ is a complex structure,  define a complex vector space $V_J$ in the following way: let the set of elements of $V_J$ be $V$,
let addition in $V_J$ be addition in $V$, and define scalar multiplication in $V_J$ by 
\[
(a+ib)v=av+bJ(v).
\]
One checks that for $\alpha,\beta \in \mathbb{C}$ and $v \in V_J$ we have $(\alpha \beta)v=\alpha (\beta v)$, and thus
that $V_J$ is indeed a complex vector space with this definition of scalar multiplication.\footnote{One should also
verify that distributivity holds with this definition of scalar product; the other properties of a vector space are satisfied because $V_J$ has
the same addition as the real vector space $V$.}

 Let $V$ be a real vector space, and define the $\mathbb{R}$-linear map 
$J:V \oplus V \to V \oplus V$ by
\[
J(v,w)=(-w,v).
\]
$J^2=-\id_{V \oplus V}$. $J$ is a complex structure on the real vector space $V \oplus V$. 
The {\em complexification} of $V$
is the complex vector space $V^\mathbb{C}=(V \oplus V)_J$. 
Thus, $V^\mathbb{C}$ has the same set of elements as $V \oplus V$, the same addition as $V \oplus V$, and scalar multiplication
\[
(a+ib) (v,w)=a(v,w)+bJ(v,w),
\]
which gives
\[
(a+ib)(v,w)=a(v,w)+b(-w,v)=(av,aw)+(-bw,bv)=(av-bw,aw+bv).
\]

If the real vector space $V$ has dimension $n$ and
if $\{e_1,\ldots,e_n\}$ is a basis for $V$, then
\[
\{(e_1,0),\ldots,(e_n,0),(0,e_1),\ldots,(0,e_n)\}
\]
is a basis for the real vector space $V \oplus V$. 
Let $v \in V^\mathbb{C}$. Using the basis for the real vector space $V \oplus V$, there exist 
\[
a_1,\ldots,a_n,b_1,\ldots,b_n \in \mathbb{R}
\]
such that
\begin{eqnarray*}
v&=&a_1(e_1,0)+\cdots a_n(e_n,0)+b_1(0,e_1)+\cdots+b_n(0,e_n)\\
&=&a_1(e_1,0)+\cdots+a_n(e_n,0)+b_1J(e_1,0)+\cdots+b_n J(e_n,0)\\
&=&(a_1+ib_1)(e_1,0)+\cdots+(a_n+ib_n)(e_n,0),
\end{eqnarray*}
where in the last line we used the definition of scalar multiplication in $V^\mathbb{C}$. One checks that
the set $\{(e_1,0),\ldots,(e_n,0)\}$ is linearly independent over $\mathbb{C}$, and therefore it is a basis for $V^\mathbb{C}$. 
Hence
\[
\dim_\mathbb{C} V^\mathbb{C} = \dim_\mathbb{R} V.
\]

\section{Complexification is a functor}
If $V,W$ are real vector spaces and $T:V \to W$ is an $\mathbb{R}$-linear map, we define 
\[
T^\mathbb{C}:V^\mathbb{C} \to W^\mathbb{C}
\]
by 
\[
T^\mathbb{C}(v_1,v_2)=(Tv_1,Tv_2);
\]
this is a $\mathbb{C}$-linear map.
Setting $\iota_V(v_1,v_2)=(v_1,0)$ and $\iota_W(w_1,w_2)=(w_1,0)$, $T^\mathbb{C}:V^\mathbb{C} \to W^\mathbb{C}$ is
the unique $\mathbb{C}$-linear map
such that the following diagram commutes\footnote{See Keith Conrad's note: \url{https://kconrad.math.uconn.edu/blurbs/linmultialg/complexification.pdf}}
% http://www.math.uconn.edu/$\sim$kconrad/blurbs/linmultialg/complexification.pdf}
\begin{center}
\begin{tikzcd}
V^\mathbb{C} \arrow{r}{T^\mathbb{C}} & W^\mathbb{C} \\
V \arrow{r}{T}\arrow{u}{\iota_V}  & W \arrow{u}{\iota_W}
\end{tikzcd}
\end{center}
Complexification is a {\em  functor} from the category of real vector spaces to the category of complex vector spaces: 
\[
(\id_V)^\mathbb{C}(v_1,v_2)=(\id_V v_1, \id_V v_2)=(v_1,v_2)=\id_{V^\mathbb{C}}(v_1,v_2),
\]
so $(\id_V)^\mathbb{C}=\id_{V^\mathbb{C}}$, and if $S:U \to V$ and $T:V \to S$ are $\mathbb{R}$-linear maps, then 
\[
(T \circ S)^\mathbb{C}(v_1,v_2)=(T(Sv_1),T(Sv_2))=
T^\mathbb{C}(Sv_1,Sv_2)=T^\mathbb{C}(S^\mathbb{C}(v_1,v_2)),
\]
so $(T \circ S)^\mathbb{C} =  T^\mathbb{C} \circ
S^\mathbb{C}$.




\section{Complexifying a complex structure}
If $V$ is a real vector space and $J:V \to V$ is a complex structure, then
\begin{eqnarray*}
(J^\mathbb{C})^2(v_1,v_2)&=&J^\mathbb{C}(Jv_1,Jv_2)\\
&=&(J^2v_1,J^2v_2)\\
&=&(-v_1,-v_2)\\
&=&-(v_1,v_2),
\end{eqnarray*}
so $(J^\mathbb{C})^2=-\id_{V^\mathbb{C}}$. 
Let
\[
E_i=\{w \in V^\mathbb{C}:J^\mathbb{C}w=iw\}, \qquad E_{-i}=\{w \in V^\mathbb{C}:
J^\mathbb{C}w=-iw\}.
\]
If $w \in V^\mathbb{C}$, then one checks that
\[
w-iJ^\mathbb{C} w \in E_i, \qquad w+iJ^\mathbb{C}w \in E_{-i},
\]
and 
\[
w=\frac{1}{2}\left(w-iJ^\mathbb{C} w\right)+\frac{1}{2}\left(w+iJ^\mathbb{C}w\right).
\]
It follows that
\[
V^\mathbb{C}=E_i \oplus E_{-i}.
\]


\section{Complex structures, inner products, and symplectic forms}
If $V$ is a real vector space of odd dimension, then one can show that there is no linear map $J:V \to V$ satisfying $J^2=-\id_V$, i.e.
there does not exist a complex structure for it. On the other hand, if $V$ has even dimension,
let
\[
\{e_1,\ldots,e_n,f_1,\ldots,f_n\}
\]
be a basis for the real vector space $V$, and define $J:V \to V$ by
\[
Je_j = f_j, \qquad Jf_j=-e_j.
\]
Then $J:V \to V$ is a complex structure.

If $V$ is a real vector space of dimension $2n$ with a complex structure $J$, let $e_1 \neq 0$. Check that $J e_1 \not \in \Span\{e_1\}$. If $n > 1$, let
\[
e_2 \not \in \Span\{e_1,Je_1\}.
\]
Check that the set $\{e_1,e_2,Je_1,Je_2\}$ is linearly independent. If $n > 2$, let 
\[
e_3 \not \in \Span\{e_1,e_2,Je_1,Je_2\}.
\]
Check that the set $\{e_1,e_2,e_3,Je_1,Je_2,Je_3\}$ is linearly independent. If $2i<2n$ then there is some
\[
e_{i+1} \not \in \Span\{e_1,\ldots,e_i,Je_1,\ldots,Je_i\}.
\]
I assert that 
\[
\{e_1,\ldots,e_n,Je_1,\ldots,Je_n\}
\]
is a basis for $V$. 

Using the above basis $\{e_1,\ldots,e_n,Je_1,\ldots,Je_n\}$ for $V$, let $f_i=Je_i$, and define $\inner{\cdot}{\cdot}:V \times V \to \mathbb{R}$ by
\[
\inner{e_i}{e_j}=\delta_{i,j}, \qquad \inner{f_i}{f_j}=\delta_{i,j},\qquad \inner{e_i}{f_j}=0, \qquad
\inner{f_i}{e_j}=0.
\]
Check that this is an inner product on the real vector space $V$. Moreover, 
\[
\inner{Je_i}{Je_j}=\inner{f_i}{f_j}=\delta_{i,j}=\inner{e_i}{e_j},
\]
and
\[
\inner{Jf_i}{Jf_j}=\inner{J^2 e_i}{J^2 e_j}=\inner{-e_i}{-e_j}=\inner{e_i}{e_j}=\delta_{i,j}=\inner{f_i}{f_j},
\]
and
\[
\inner{Je_i}{Jf_j}=\inner{f_i}{-e_j}=-\inner{f_i}{e_j}=0=\inner{e_i}{f_j},
\]
and
\[
\inner{Jf_i}{Je_j}=\inner{-e_i}{f_j}=-\inner{e_i}{f_j}=0=\inner{f_i}{e_j}.
\]
Hence for any $v,w \in V$,
\[
\inner{Jv}{Jw}=\inner{v}{w}.
\]
We say that the complex structure $J$ is {\em compatible} with the inner product $\inner{\cdot}{\cdot}$, i.e.
$J:(V,\inner{\cdot}{\cdot}) \to (V,\inner{\cdot}{\cdot})$ is an {\em orthogonal transformation}.

A {\em symplectic form} on a real vector space $V$ is a bilinear form $\omega:V \times V \to \mathbb{R}$
such that $\omega(v,w)=-\omega(w,v)$, and such that if $\omega(v,w)=0$ for all $w$ then
$v=0$; we say respectively that $\omega$ is {\em skew-symmetric} and {\em non-degenerate}.
If a real vector space $V$ has a complex structure $J$, and $\inner{\cdot}{\cdot}$ is an inner product on $V$ that is compatible with $J$,
 define $\omega$ by
\[
\omega(v,w)=\inner{v}{J^{-1}w}=\inner{v}{-Jw}=-\inner{v}{Jw},
\]
which is equivalent to
\[
\omega(v,Jw)=\inner{v}{w}.
\]
Using that the inner product is compatible with $J$ and that it is symmetric,
\[
\omega(v,w)=-\inner{v}{Jw}=-\inner{Jv}{J^2 w}=-\inner{Jv}{-w}=\inner{w}{Jv}=-\omega(w,v),
\]
so $\omega$ is skew-symmetric. If $w \in V$ and $\omega(v,w)=0$ for all $v \in V$, then
\[
-\inner{v}{Jw}=0
\]
for all $v \in V$, and thus $Jw=0$. Since $J$ is invertible, $w=0$. Thus $\omega$ is nondegenerate. Therefore
$\omega$ is a symplectic form on $V$.\footnote{Using the basis
$\{e_1,\ldots,e_n,f_1,\ldots,f_n\}$ for $V$, $f_i=Je_i$, we have 
\[
\omega(e_i,f_j)=-\inner{e_i}{Jf_j}=-\inner{e_i}{J^2e_j}=-\inner{e_i}{-e_j}=\inner{e_i}{e_j}=\delta_{i,j},
\]
and
\[
\omega(e_i,e_j)=-\inner{e_i}{Je_j}=-\inner{e_i}{f_j}=0, \qquad \omega(f_i,f_j)=0.
\]
A basis $\{e_1,\ldots,e_n,f_1,\ldots,f_n\}$ for a symplectic vector space that satisfies these three conditions
is called a {\em Darboux basis}.}
We have
\[
\omega(Jv,Jw)=-\inner{Jv}{J^2w}=-\inner{J^2v}{J^3w}=-\inner{-v}{-Jw}=-\inner{v}{Jw}=\omega(v,w).
\]
We say that $J$ is {\em compatible} with the sympletic form $\omega$, namely, $J:(V,\omega) \to (V,\omega)$ is
a {\em symplectic transformation}.

On the other hand,
if $V$ is a real vector space with symplectic form $\omega$ and $J$ is a compatible complex structure,
then $\inner{\cdot}{\cdot}: V \times V \to \mathbb{R}$ defined by
\[
\inner{v}{w}=\omega(v,Jw)
\]
is an inner product on $V$ that is compatible with the complex structure $J$. 

Suppose $V$ is a real vector space with complex structure $J:V \to V$ and that $h:V_J \times V_J \to \mathbb{C}$ 
is an inner product on the complex vector space $V_J$. Define $g:V \times V \to \mathbb{R}$ by\footnote{The
letter $h$ refers to a Hermitian form, i.e. an inner product on a complex vector space, and the letter $g$ refers
to the usual notation for a metric on a Riemannian manifold.}
\[
g(v_1,v_2)=\frac{1}{2}\left(h(v_1,v_2)+\overline{h(v_1,v_2)}\right)=
\frac{1}{2}\left(h(v_1,v_2)+h(v_2,v_1)\right).
\]
It is straightforward to check that $g$ is an inner product on the real vector space $V$. 
Similarly, define $\omega:V \times V \to \mathbb{R}$ by
\[
\omega(v_1,v_2)=\frac{i}{2}\left(h(v_1,v_2)-\overline{h(v_1,v_2)}\right)=\frac{i}{2}\left(h(v_1,v_2)-h(v_2,v_1)\right).
\]
It is apparent that $\omega$ is skew-symmetric. If $\omega(v_1,v_2)=0$ for all $v_1$, then in particular
$\omega(iv_1,v_1)=0$, and so
\[
h(iv_1,v_1)-h(v_1,iv_1)=0.
\]
As $h$ is a complex inner product,
\[
ih(v_1,v_1)-\overline{i}h(v_1,v_1)=0,
\]
i.e.
\[
2ih(v_1,v_1)=0,
\]
and thus $h(v_1,v_1)=0$, which implies that $v_1=0$. Therefore $\omega$ is nondegenerate, and thus
$\omega$ is a symplectic form on the real vector space $V$. 
With these definitions of $g$ and $\omega$, for $v_1,v_2 \in V_J$ we have
\[
h(v_1,v_2)=g(v_1,v_2)-i\omega(v_1,v_2),
\]
which writes the inner product on the complex vector space $V_J$ using an inner product on the real
vector space $V$ and a symplectic form on the real vector space $V$;
note that $V_J$ has the same set of elements as $V$. 
Moreover, for $v_1,v_2 \in V$ we have
\begin{eqnarray*}
\omega(v_1,Jv_2)&=&\frac{i}{2}\left(h(v_1,Jv_2)-h(Jv_2,v_1)\right)\\
&=&\frac{i}{2}\left(h(v_1,iv_2)-h(iv_2,v_1)\right)\\
&=&\frac{i}{2}\left(-ih(v_1,v_2)-ih(v_2,v_1)\right)\\
&=&g(v_1,v_2).
\end{eqnarray*}


\subsection{Tensor products}
Here we give another presentation of the complexification of a real vector space, this time using tensor products
of real vector spaces. If you were satisfied by the first definition you don't need to read this one; read this either if you are curious
about another way to define complexification, if you want to see a pleasant application of tensor products, or if you didn't like the first
definition.
Let $V$ be a real vector space of dimension $n$. $\mathbb{C}$ is a real vector space of dimension $2$, and
\[
V \otimes_{\mathbb{R}} \mathbb{C}
\]
is a real vector space of dimension $2n$. If $V$ has basis $\{e_1,\ldots,e_n\}$, then $V \otimes_{\mathbb{R}} \mathbb{C}$
has basis $\{e_1 \otimes 1, \ldots, e_n \otimes 1, e_1 \otimes i, \ldots, e_n \otimes i\}$.
Since every element of $V \otimes_{\mathbb{R}} \mathbb{C}$ can be written uniquely in the form
\[
v_1 \otimes 1 + v_2 \otimes i, \qquad v_1, v_2 \in V,
\]
one often writes
\[
V \otimes_{\mathbb{R}} \mathbb{C} \cong V \oplus iV;
\]
here $iV$ is a real vector space that is isomorphic to $V$. 


The {\em complexification} of $V$ is the
complex vector space $V^\mathbb{C}$ whose set of elements is $V \otimes_{\mathbb{R}} \mathbb{C}$, with the same addition
as the real vector space $V \otimes_{\mathbb{R}} \mathbb{C}$, and with scalar multiplication defined by
\[
\alpha(v \otimes \beta)=v \otimes (\alpha \beta), \qquad v \in V, \alpha,\beta \in \mathbb{C}.
\]
Let $v \in V^\mathbb{C}$. Using the basis of the real vector space $V \otimes_{\mathbb{R}} \mathbb{C}$, there exist some 
\[
a_1,\ldots,a_n,b_1,\ldots,b_n \in \mathbb{R}
\]
 such that
\begin{eqnarray*}
v&=&a_1 e_1 \otimes 1 + \cdots + a_n e_n \otimes 1 + b_1 e_1 \otimes i + \cdots + b_n e_n \otimes i\\
&=&e_1 \otimes (a_1+ib_1) + \cdots +e_n \otimes (a_n + ib_n)\\
&=&(a_1+ib_1) e_1 \otimes 1 + \cdots + (a_n+ib_n) e_n \otimes 1,
\end{eqnarray*}
where in the last line we used the definition of scalar multiplication in $V^\mathbb{C}$. One checks
that the 
$\{e_1 \otimes 1,\ldots, e_n \otimes 1\}$ is linearly independent over $\mathbb{C}$, and hence that it is a basis for the complex vector space $V^\mathbb{C}$,
so $V^\mathbb{C}$ has dimension $n$ over $\mathbb{C}$.

If $V$ and $W$ are real vector spaces and $T:V \to W$ is a linear map, define $T^\mathbb{C}:V^\mathbb{C} \to W^\mathbb{C}$ by
\[
T^\mathbb{C}(v \otimes z)=(Tv) \otimes z.
\]
With this definition of $T^\mathbb{C}$, one can check that complexification is a functor from the category of real vector
spaces to the category of complex vector spaces.


\section{Decomplexification}
If $V$ is a complex vector space, let $V^\mathbb{R}$ be the real vector space whose set of elements is $V$, in which addition is the same as addition
in $V$, and in which scalar multiplication is defined by
\[
av=(a+0i)v, \qquad a \in \mathbb{R}.
\]
Because $V$ is a complex vector space, it is apparent that $V^\mathbb{R}$ is  a real vector space with this scalar multiplication.
We call $V^\mathbb{R}$ the {\em decomplexification} of the complex vector space $V$.

If $V$ has basis $\{e_1,\ldots,e_n\}$ and $v \in V$,
then there are $a_1+ib_1,\ldots,a_n+ib_n \in \mathbb{C}$ such that
\[
v=(a_1+ib_1)e_1+\cdots+(a_n+ib_n)e_n=a_1e_1+\cdots+a_ne_n+b_1(ie_1)+\cdots+
b_n(ie_n).
\]
One checks that 
\[
e_1,\ldots,e_n,ie_1,\ldots,ie_n
\]
are linearly independent over $\mathbb{R}$, and hence are a basis for the real vector space $V^\mathbb{R}$. Thus, 
\[
\dim_\mathbb{R} V^\mathbb{R} = 2\dim_\mathbb{C} V.
\]

If $V$ is a complex vector space and $T:V \to V$ is a $\mathbb{C}$-linear map, define $T^\mathbb{R}:V^\mathbb{R} \to V^\mathbb{R}$ by
\[
T^\mathbb{R}v=Tv.
\]
Because $T$ is $\mathbb{C}$-linear it follows that $T^\mathbb{R}$ is $\mathbb{R}$-linear.
Decomplexification is a functor from the category of complex vector spaces to the category of real vector spaces. Since 
decomplexification is defined simply by ignoring the fact that $V$ is closed
under multiplication by complex scalars and only using real scalars,
decomplexification is called a forgetful functor

\section{Complex conjugation in complexified vector spaces}
If $V$ is a real vector space, define $\sigma:V^\mathbb{C} \to V^\mathbb{C}$ by
\[
\sigma(v_1,v_2)=(v_1,-v_2).
\]
We call $\sigma$ {\em complex conjugation} in $V^\mathbb{C}$. We have $\sigma \circ \sigma = \id_{V^\mathbb{C}}$.
If $T:V^\mathbb{C} \to V^\mathbb{C}$ is a $\mathbb{C}$-linear map, define $T^\sigma:V^\mathbb{C} \to V^\mathbb{C}$
by
\[
T^\sigma(w)=\sigma(T \sigma (w)).
\]
$T^\sigma$ is a $\mathbb{C}$-linear map.
It is a fact that if $T:V^\mathbb{C} \to V^\mathbb{C}$ is $\mathbb{C}$-linear, then $T^\sigma=T$ if and only if there
is some $\mathbb{R}$-linear $S:V \to V$ such that $T=S^\mathbb{C}$. In words, a linear map on the complexification
of a real vector space is equal to its own conjugate if and only if it is the complexification of a linear map
on the real vector space.

The following are true statements:\footnote{These are exercises from V. I. Arnold's {\em Ordinary differential equations},
p.~122, \S 18.4, in Richard A. Silverman's translation.} ($\mathbb{C}^n=(\mathbb{R}^n)^\mathbb{C}$)

\begin{itemize}
\item If $T:\mathbb{C}^n \to \mathbb{C}^n$ is a linear map, then
\[
\exp(T)^\mathbb{R}=\exp(T^\mathbb{R}),
\]
and 
\[
\exp(T)^\sigma=\exp(T^\sigma).
\]
\item If $T:\mathbb{R}^n \to \mathbb{R}^n$ is a linear map, then
\[
\exp(T)^\mathbb{C}=\exp(T^\mathbb{C}).
\]
\item If $T:\mathbb{C}^n \to \mathbb{C}^n$ is a linear map, then
\[
\det T^\mathbb{R}=|\det T|^2,
\]
and 
\[
\det T^\sigma=\overline{\det T}.
\]
\item If $T:\mathbb{R}^n \to \mathbb{R}^n$ is a linear map, then
\[
\det T^\mathbb{C}=\det T.
\]
\item If $T:\mathbb{C}^n \to \mathbb{C}^n$ is a linear map, then
\[
\Tr (T^\mathbb{R})=\Tr T+\Tr T^\sigma,
\]
and 
\[
\Tr T^\sigma=\overline{\Tr T}.
\]
\item If $T:\mathbb{R}^n \to \mathbb{R}^n$ is a linear map, then
\[
\Tr T^\mathbb{C}=\Tr T.
\]
\end{itemize}


\section{Linear ordinary differential equations over $\mathbb{C}$}
Let $A$ be an $n \times n$ matrix over $\mathbb{C}$. The solution of the initial value problem
\[
z'(t)=Az(t), \qquad z(0)=z_0 \in \mathbb{C}^n,
\]
is $z(t)=\exp(At)$. 

If $A$ has $n$ distinct eigenvalues $\lambda_1,\ldots,\lambda_n \in \mathbb{C}$, then, with
\[
E_\lambda=\{z \in \mathbb{C}^n: Az=\lambda z\},
\]
 we have
\[
\mathbb{C}^n = E_{\lambda_1} \oplus + \cdots + \oplus E_{\lambda_n},
\]
where each $E_{\lambda_k}$ has dimension 1. For $z \in E_{\lambda_k}$, 
\[
\exp(At)z=\sum_{m=0}^\infty t^m \frac{A^m z}{m!} = \sum_{m=0}^\infty t^m \frac{\lambda_k^m z}{m!}=e^{\lambda_k t}z.
\]
Let $\xi_k \in E_{\lambda_k}$ be nonzero, $1 \leq k \leq n$. They are a basis for $\mathbb{C}^n$, so there are $c_k \in \mathbb{C}$ such that
\[
z_0=\sum_{k=1}^n c_k \xi_k.
\]
Then
\[
z(t)=\exp(At)z_0=\exp(At)\sum_{k=1}^n c_k \xi_k=\sum_{k=1}^n c_k \exp(At) \xi_k=
\sum_{k=1}^n c_k e^{\lambda_k t} \xi_k.
\]

Suppose that $A$ is an $n \times n$ matrix over $\mathbb{C}$, that $z_0 \in \mathbb{C}^n$, that
$A^\sigma=A$ and that $\sigma(z_0)=z_0$. The solution of the initial value problem 
\[
z'(t)=Az(t), \qquad z(0)=z_0,
\]
is $z(t)=\exp(At)z_0$. We have, as $\exp(At)^\sigma=\exp((At)^\sigma)=\exp(At)$,
\[
\sigma(z(t))=\sigma(\exp(At)z_0)=\sigma(\exp(At)\sigma(z_0))=\exp(At)^\sigma z_0=\exp(At)z_0=z(t).
\]
Therefore, if $A^\sigma=A$ and $\sigma(z_0)=z_0$, then $\sigma(z(t))=z(t)$ for all $t$.

\section{Linear ordinary differential equations over $\mathbb{R}$}
Let $A$ be an $n \times n$ matrix over $\mathbb{R}$ and let $x_0 \in \mathbb{R}^n$. Let $z_0=(x_0,0) \in \mathbb{C}^n=(\mathbb{R}^n)^\mathbb{C}$, and let
$z(t)=(z_1(t),z_2(t))$ be the solution of the initial value problem
\[
z'(t)=A^\mathbb{C} z(t), \qquad z(0)=z_0 \in \mathbb{C}^n.
\]
As $A^\mathbb{C}$ is the complexification of a real linear map, $(A^\mathbb{C})^\sigma=A^\mathbb{C}$, and
\[
\sigma(z_0)=\sigma(x_0,0)=(x_0,-0)=(x_0,0)=z_0,
\]
so $\sigma(z(t))=z(t)$, i.e. $(z_1(t),z_2(t))=(z_1(t),-z_2(t))$, so $z_2(t)=0$ for all $t$.
But $z'(t)=(z_1'(t),z_2'(t))$ and $A^\mathbb{C} z(t)=(Az_1(t),Az_2(t))$,
so
\[
z_1'(t)=Az_1(t)
\]
for all $t$. Also, $z(0)=(z_1(0),z_2(0))$ and $z(0)=z_0=(x_0,0)$, so $z_1(0)=x_0$. Therefore, $x(t)=z_1(t)$ is the solution
of the initial value problem
\[
x'(t)=Ax(t), \qquad x(0)=x_0.
\]
Thus, to solve an initial value problem in $\mathbb{R}^n$ we can complexify it, solve the initial value problem in $\mathbb{C}^n$, and take
the first entry of the solution of the complex initial value problem.

If $A$ is an $n \times n$ matrix over $\mathbb{R}$, let 
\[
\det(A-xI)=\sum_{k=0}^n a_k x^k, \qquad a_k \in \mathbb{R},
\]
its characteristic polynomial. The Cayley-Hamilton theorem states that
\[
\sum_{k=0}^n a_k A^k=0.
\]
Taking the complexification of this gives
\[
\sum_{k=0}^n a_k (A^\mathbb{C})^k=0.
\]
It follows that the roots of $\det(A-xI)$ are the same as the roots of $\det(A^\mathbb{C}-xI^\mathbb{C})$. A complex root 
of $\det(A-xI)$ is not an eigenvalue of $A:\mathbb{R}^2 \to \mathbb{R}^2$, but is indeed an eigenvalue of $A^\mathbb{C}:\mathbb{C}^2
\to \mathbb{C}^2$, so the roots of the characteristic polynomial
of $A$ are the eigenvalues of $A^\mathbb{C}$.

\section{Linear ordinary differential equations in $\mathbb{R}^2$}
Let $A$ be a $2 \times 2$ matrix over $\mathbb{R}$.\footnote{This section follows Arnold, p.~132, \S 20.3.} Suppose that the roots of
the characteristic polynomial
\[
\det(A-x I)=\det A-x \Tr A + x^2
\]
are $\lambda,\overline{\lambda}$, i.e. that the roots of the characteristic polynomial are
complex conjugate. Let $\lambda=\alpha+i\omega$, $\omega \neq 0$.\footnote{Define $J: \mathbb{R}^2 \to \mathbb{R}^2$ by
\[
J=\frac{1}{\omega}(A-\alpha I ).
\]
We have $J^2=\frac{1}{\omega^2}(A^2-2\alpha A+\alpha^2 I)$. By the Cayley-Hamilton theorem, 
\[
I\det A - A\Tr A + A^2=0,
\]
so
\[
I \lambda \overline{\lambda}-A(\lambda+\overline{\lambda})+A^2=0,
\]
and written using $\lambda=\alpha+i\omega$ this is
\[
I(\alpha^2+\omega^2)-2\alpha A+A^2=0.
\]
Hence
\[
J^2=-I,
\]
so $J=\frac{1}{\omega}(A-\alpha I )$ is a complex structure on $\mathbb{R}^2$.}
$\lambda$ is an eigenvalue for $A^\mathbb{C}$, so
let $A^\mathbb{C}(v_1,v_2)=\lambda(v_1,v_2)$, $(v_1,v_2) \neq 0$. 
Furthermore,
\[
\sigma(A^\mathbb{C}(v_1,v_2))=\sigma(\lambda(v_1,v_2)),
\]
so
\[
(A^\mathbb{C})^\sigma \sigma(v_1,v_2)=\overline{\lambda} \sigma(v_1,v_2),
\]
hence, as $(A^\mathbb{C})^\sigma=A^\mathbb{C}$,
\[
A^\mathbb{C}(v_1,-v_2)=\overline{\lambda} (v_1,-v_2).
\]
Therefore $(v_1,-v_2)$ is an eigenvector of $A^\mathbb{C}$ with eigenvalue $\overline{\lambda} \neq \lambda$, so
$(v_1,-v_2)$ and $(v_1,v_2)$ are linearly independent over $\mathbb{C}$. If $a_1v_1+a_2v_2=0$, $a_1,a_2 \in \mathbb{R}$, then
\[
\left(\frac{a_1}{2}-i\frac{a_2}{2}\right)(v_1,v_2)+\left(\frac{a_1}{2}+i\frac{a_2}{2}\right)(v_1,-v_2)=0,
\]
from which it follows that $a_1,a_2=0$. Therefore $v_1,v_2 \in \mathbb{R}^2$ are linearly independent over $\mathbb{R}$. 


We have
\[
(\alpha+i\omega)(v_1,v_2)=(\alpha v_1-\omega v_2,\alpha v_1+\omega v_2),
\]
and
\[
A^\mathbb{C}(v_1,v_2)=(Av_1,Av_2),
\]
so
\[
Av_1=\alpha v_1-\omega v_2, \qquad Av_2=\alpha v_1+\omega v_2,
\]
and
hence
\[
A\begin{pmatrix}v_1&v_2\end{pmatrix}=\begin{pmatrix}\alpha v_1-\omega v_2&\alpha v_1+\omega v_2\end{pmatrix}
=\begin{pmatrix}v_1&v_2\end{pmatrix}
\begin{pmatrix}\alpha&\omega \\ -\omega&\alpha\end{pmatrix}.
\]
Therefore
\[
A=\begin{pmatrix}v_1&v_2\end{pmatrix}
\begin{pmatrix}\alpha&\omega \\ -\omega&\alpha\end{pmatrix}\begin{pmatrix}v_1&v_2\end{pmatrix}^{-1}.
\]


\end{document}