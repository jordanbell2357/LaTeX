%%%%%%%%%%%%%%%%%%%%%%%%%%%%%%%%%%%%%%%%%
% Medium Length Graduate Curriculum Vitae
% LaTeX Template
% Version 1.1 (9/12/12)
%
% This template has been downloaded from:
% http://www.LaTeXTemplates.com
%
% Original author:
% Rensselaer Polytechnic Institute (http://www.rpi.edu/dept/arc/training/latex/resumes/)
%
% Important note:
% This template requires the res.cls file to be in the same directory as the
% .tex file. The res.cls file provides the resume style used for structuring the
% document.
%
%%%%%%%%%%%%%%%%%%%%%%%%%%%%%%%%%%%%%%%%%

%----------------------------------------------------------------------------------------
%	PACKAGES AND OTHER DOCUMENT CONFIGURATIONS
%----------------------------------------------------------------------------------------

\documentclass[margin, 10pt]{res} % Use the res.cls style, the font size can be changed to 11pt or 12pt here

\usepackage{helvet} % Default font is the helvetica postscript font
%\usepackage{newcent} % To change the default font to the new century schoolbook postscript font uncomment this line and comment the one above

\usepackage{hyperref}

\setlength{\textwidth}{5.1in} % Text width of the document

\begin{document}

%----------------------------------------------------------------------------------------
%	NAME AND ADDRESS SECTION
%----------------------------------------------------------------------------------------

\moveleft.5\hoffset\centerline{\large\bf Jordan Bell} % Your name at the top
 
\moveleft\hoffset\vbox{\hrule width\resumewidth height 1pt}\smallskip % Horizontal line after name; adjust line thickness by changing the '1pt'

\href{https://jordanbell.info/}{\texttt{jordanbell.info}}\\
\texttt{jordan.bell@gmail.com}\\
545 Danforth Ave., Apt. 2\\
Toronto ON M4K1P7\\
416-528-3258

%----------------------------------------------------------------------------------------

\begin{resume}

%----------------------------------------------------------------------------------------
%	OBJECTIVE SECTION
%----------------------------------------------------------------------------------------
 
%\section{OBJECTIVE}  

%Improving my teaching mastery by working in new virtual teaching environments, both from working with international students whose curriculum and expectations I must learn, and by working with tutoring companies with specific methods for me to assimilate and follow.


%----------------------------------------------------------------------------------------
%	PROFESSIONAL EXPERIENCE SECTION
%----------------------------------------------------------------------------------------
 
\section{WORK HISTORY}

{\sl Tutor, Content Creator} \hfill January 2021 - current \\
Jordan Bell Mathematics Tutoring Toronto (\href{https://jordanbell.info/}{\texttt{jordanbell.info}})

\begin{itemize} \itemsep -2pt % Reduce space between items
\item Tutoring high school and university mathematics, and occassional sessions for
probability, statistics, micro and macro economics, physics, and accounting at secondary and post-secondary level.

\item Comprehensive experience with Ontario secondary, IB, and AP curriculum for mathematics, physics, accounting, and economics, and with University of Toronto undergraduate mathematics courses from first year to fourth year, and substantial experience with first and second year probability, statistics and micro and macro economics courses.
\end{itemize}

{\sl Mathematics tutor} \hfill March 2018 - December 2020 \\
Toronto Elite Tutorial Services

\begin{itemize} \itemsep -2pt % Reduce space between items
\item Tutoring high school mathematics, all grades, occasional postsecondary clients for statistics. Made plans for full school year (private schools) and semester with students and parents for regular tutoring agendas to reliably and noticeably improve marks and sense of mastery. 
\end{itemize}

{\sl Course Instructor} \hfill Apr 2013 - Apr 2017 \\
Department of Mathematics, University of Toronto

\begin{itemize} \itemsep -2pt % Reduce space between items
\item Course instructor for undergraduate mathematics courses at all three campuses of the University of Toronto.

\item Experience as sole instructor of a one section course (differential equations): setting syllabus according to university calendar and past courses, delivered lectures, and made assignments, tests and final exam.

\item Experience as part of teaching teams for multiple section courses, both when there is a designated senior instructor and when there is a consensus system without a senior instructor.

\end{itemize}
 
{\sl Teaching Assistant} \hfill September 2009 - April 2013 \\
Department of Mathematics, University of Toronto

\begin{itemize} 
\item Experience with all formats of tutorials: working out examples, answering questions, explaining topics, administering quizzes, group assignments.
\item Experience evaluating student work (quizzes, tests, midterms, assignments, essays)
\item Experience as teaching assistant for majority of University of Toronto undergraduate mathematics courses, up to fourth year
\end{itemize} 

%----------------------------------------------------------------------------------------
%	EDUCATION SECTION
%----------------------------------------------------------------------------------------

\section{EDUCATION}

{\sl Graduate Certificate}, Analytics for Business Decision Making \\
George Brown College, Toronto, May 2019

Canada Graduate Scholarships – Doctoral (CGS D), University of Toronto, Department of Mathematics

{\sl Master of Science,} Mathematics \\
University of Toronto, Toronto, June 2009\\
Canada Graduate Scholarships – Master's (CGS M)

{\sl Bachelor of Mathematics,} Mathematics \\
Carleton University, Ottawa, June 2007\\
University Medal in Mathematics



\section{PUBLICATIONS}



\begin{itemize}
\item Andrews, George E., and Jordan Bell. “Euler’s Pentagonal Number Theorem and the Rogers-Fine Identity.” {\em Annals of Combinatorics} 16, no. 3 (2012): 411–20. \url{https://doi.org/10.1007/s00026-012-0139-4}

\item Bell, Jordan. “A New Method for Constructing Nonlinear Modular *n*-Queens Solutions.” {\em Ars Combinatoria} 78 (2006): 151–55.

\item ———. “A Summary of Euler’s Work on the Pentagonal Number Theorem.” {\em Archive for History of Exact Sciences} 64, no. 3 (2010): 301–73. \url{https://doi.org/10.1007/s00407-010-0057-y}

\item ———. “Cyclotomic Orthomorphisms of Finite Fields.” {\em Discrete Applied Mathematics} 161, no. 1–2 (2013): 294–300. \url{https://doi.org/10.1016/j.dam.2012.08.013}

\item ———. “Estimates for the Norms of Products of Sines and Cosines.” {\em Journal of Mathematical Analysis and Applications} 405, no. 2 (2013): 530–45. \url{https://doi.org/10.1016/j.jmaa.2013.04.010}

\item ———. “Nonlinear Modular Latin Queen Squares.” {\em Utilitas Mathematica} 74 (2007): 71–75.

\item ———. “Polynomial Modular *n*-Queens Solutions.” {\em Acta Arithmetica} 129, no. 4 (2007): 335–39. \url{https://doi.org/10.4064/aa129-4-4>}

\item Bell, Jordan, and Viktor Blåsjö. “Pietro Mengoli’s 1650 Proof that the Harmonic Series Diverges.” {\em Mathematics Magazine} 91, no. 5 (2018): 341–47. \url{https://doi.org/10.1080/0025570X.2018.1506656}

\item Bell, Jordan, and Brett Stevens. “A Survey of Known Results and Research Areas for *n*-Queens.” {\em Discrete Mathematics} 309, no. 1 (2009): 1–31. \url{https://doi.org/10.1016/j.disc.2007.12.043}

\item ———. “Constructing Orthogonal Pandiagonal Latin Squares and Panmagic Squares from Modular $n$-Queens Solutions.” {\em Journal of Combinatorial Designs} 15, no. 3 (2007): 221–34. \url{https://doi.org/10.1002/jcd.20143}

\item ———. “Results for the *n*-Queens Problem on the Möbius Board.” {\em The Australasian Journal of Combinatorics} 42 (2008): 21–34. \url{https://ajc.maths.uq.edu.au}

\item Bell, Jordan, and Qiang Wang. “Results on Permutations with Distinct Difference Property.” {\em Contributions to Discrete Mathematics} 4, no. 1 (2009): 107–11. \url{https://cdm.ucalgary.ca/}
\end{itemize}


2019 recipient of Carl B. Allendoerfer Award for expository mathematical writing, Mathematical Association of America (MAA) for Jordan Bell and Viktor Blåsjö, {\em Pietro Mengoli’s 1650 Proof that the Harmonic Series Diverges}, Mathematics Magazine, Vol. 91, no. 5, December 2018, pp. 341-347.

%----------------------------------------------------------------------------------------
%	COMPUTER SKILLS SECTION
%----------------------------------------------------------------------------------------

\section{COMPUTER \\ SKILLS} 

{\sl Languages:} 
Python, Excel/Google Sheets, SQL, R, SAS, \LaTeX{}, Markdown, HTML, CSS, JavaScript \\

{\sl Software:}
Git, Jupyter, Visual Studio Code, ImageMagick, OpenShot, QGIS (accessing and transforming geospatial datasets from Statistics Canada, and making raster and vector visualizations with QGIS)\\

{\sl Knowledge areas:}
Time series analysis, differential equations, financial accounting, business KPI

\end{resume}
\end{document}