\documentclass{article}
\usepackage{amsmath,amssymb,mathrsfs,amsthm}
%\usepackage{tikz-cd}
%\usepackage{hyperref}
\newcommand{\inner}[2]{\left\langle #1, #2 \right\rangle}
\newcommand{\tr}{\ensuremath\mathrm{tr}\,} 
\newcommand{\Span}{\ensuremath\mathrm{span}} 
\def\Re{\ensuremath{\mathrm{Re}}\,}
\def\Im{\ensuremath{\mathrm{Im}}\,}
\newcommand{\id}{\ensuremath\mathrm{id}} 
\newcommand{\var}{\ensuremath\mathrm{var}} 
\newcommand{\Lip}{\ensuremath\mathrm{Lip}} 
\newcommand{\GL}{\ensuremath\mathrm{GL}} 
\newcommand{\diam}{\ensuremath\mathrm{diam}} 
\newcommand{\sgn}{\ensuremath\mathrm{sgn}\,} 
\newcommand{\lcm}{\ensuremath\mathrm{lcm}} 
\newcommand{\supp}{\ensuremath\mathrm{supp}\,}
\newcommand{\dom}{\ensuremath\mathrm{dom}\,}
\newcommand{\upto}{\nearrow}
\newcommand{\downto}{\searrow}
\newcommand{\norm}[1]{\left\Vert #1 \right\Vert}
\newtheorem{theorem}{Theorem}
\newtheorem{lemma}[theorem]{Lemma}
\newtheorem{proposition}[theorem]{Proposition}
\newtheorem{corollary}[theorem]{Corollary}
\theoremstyle{definition}
\newtheorem{definition}[theorem]{Definition}
\newtheorem{example}[theorem]{Example}
\begin{document}
\title{The Bernstein and Nikolsky inequalities for trigonometric polynomials}
\author{Jordan Bell}
\date{January 28, 2015}

\maketitle

\section{Introduction}
Let $\mathbb{T}=\mathbb{R}/2\pi \mathbb{Z}$. For a function $f:\mathbb{T} \to \mathbb{C}$ and $\tau \in \mathbb{T}$, we define
$f_\tau:\mathbb{T} \to \mathbb{C}$ by $f_\tau(t)=f(t-\tau)$.
For measurable $f:\mathbb{T} \to \mathbb{C}$ and $0<r<\infty$, write
\[
\norm{f}_r = \left(\frac{1}{2\pi} \int_{\mathbb{T}} |f(t)|^r dt\right)^{1/r}.
\]
For $f,g \in L^1(\mathbb{T})$, write
\[
(f*g)(x) = \frac{1}{2\pi} \int_{\mathbb{T}} f(t)g(x-t) dt, \qquad x \in \mathbb{T},
\]
and for $f \in L^1(\mathbb{T})$, write
\[
\hat{f}(k)=\frac{1}{2\pi} \int_{\mathbb{T}} f(t) e^{-ikt} dt, \qquad k \in \mathbb{Z}.
\]
This note works out proofs of some inequalities involving the support of  $\hat{f}$ for
$f \in L^1(\mathbb{T})$.



Let $\mathscr{T}_n$ be the set of trigonometric polynomials of degree $\leq n$. 
We define the \textbf{Dirichlet kernel} $D_n:\mathbb{T} \to \mathbb{C}$ by
\[
D_n(t) = \sum_{|j| \leq n} e^{ijt}, \qquad t \in \mathbb{T}.
\]
It is straightforward to check that if $T \in \mathscr{T}_n$ then 
\[
D_n * T = T.
\]



\section{Bernstein's inequality for trigonometric polynomials}
DeVore and Lorentz attribute the following inequality to Szeg\"o.\footnote{Ronald A. DeVore and George G. Lorentz,
{\em Constructive Approximation}, p.~97, Theorem 1.1.}


\begin{theorem}
If $T \in \mathscr{T}_n$ and $T$ is real valued, then for all $x \in \mathbb{T}$,
\[
T'(x)^2+n^2 T(x)^2 \leq n^2 \norm{T}_\infty^2.
\]
\label{szego}
\end{theorem}
\begin{proof}
If $T=0$ the result is immediate. Otherwise, take $x \in \mathbb{T}$,
and for real $c>1$ define 
\[
P_c(t)=\frac{T(t+x) \sgn T'(x)}{c \norm{T}_\infty}, \qquad t \in \mathbb{T}.
\]
$P_c \in \mathscr{T}_n$, and satisfies
\[
P_c'(0) = \frac{T'(x) \sgn T'(x)}{c \norm{T}_\infty} \geq 0
\]
and $\norm{P_c}_\infty \leq \frac{1}{c}<1$.
Since $\norm{P_c}_\infty < 1$, in particular $|P_c(0)|<1$ and so there is some
$\alpha$, $|\alpha|<\frac{\pi}{2n}$, such that $\sin n\alpha = P_c(0)$. 
We define $S \in \mathscr{T}_n$ by
\[
S(t) = \sin n(t+\alpha)-P_c(t), \qquad t \in \mathbb{T},
\]
which satisfies $S(0)=\sin n\alpha-P_c(0)=0$.
For $k=-n,\ldots,n$, let $t_k=-\alpha+\frac{(2k-1)\pi}{2n}$, for which we have
\[
\sin n(t_k+\alpha) = \sin \frac{(2k-1)\pi}{2} = (-1)^{k+1}.
\]
Because $\norm{P_c}_\infty<1$, 
\[
\sgn S(t_k) =  (-1)^{k+1},
\]
so by the intermediate value theorem, for each $k=-n,\ldots,n-1$ there is some $c_k \in (t_k,t_{k+1})$ such that
$S(c_k)=0$.
Because
\[
t_n-t_{-n} = \frac{(2n-1)\pi}{2n}-\frac{(-2n-1)\pi}{2n}=
2\pi,
\]
it follows that if $j \neq k$ then $c_j$ and $c_k$ are distinct in $\mathbb{T}$.
It is a fact that a trigonometric polynomial of degree $n$ has $\leq 2n$ distinct roots in $\mathbb{T}$, so if
$t \in (t_k,t_{k+1})$ and $S(t)=0$, then $t=c_k$. 
It is the case that $t_1=-\alpha+\frac{\pi}{2n}>0$ and $t_0=-\alpha-\frac{\pi}{2}<0$, so
$0 \in (t_0,t_1)$. But $S(0)=0$, so $c_0=0$. 
Using $S(t_1)=1>0$ and the fact that $S$ has no zeros in $(0,t_1)$ we get a contradiction from $S'(0) < 0$, so $S'(0) \geq 0$. 
This gives
\[
0 \leq P_c'(0) = n\cos n\alpha-S'(0) \leq n\cos n\alpha=n\sqrt{1-\sin^2 n\alpha} = n\sqrt{1-P_c(0)^2}.
\]
Thus
\[
P_c'(0) \leq n\sqrt{1-P_c(0)^2},
\]
or
\[
n^2 P_c(0)+P_c'(0)^2 \leq n^2.
\]
Because
\[
P_c(0)^2=\frac{T(x)^2}{c^2 \norm{T}_\infty^2}, \qquad
P_c'(0)^2 = \frac{T'(x)^2}{c^2\norm{T}_\infty^2}
\]
we get
\[
n^2 T(x)^2 + T'(x)^2 \leq c^2 n^2 \norm{T}_\infty^2.
\]
Because this is true for all $c>1$,
\[
n^2 T(x)^2 + T'(x)^2 \leq n^2 \norm{T}_\infty^2,
\]
completing the proof.
\end{proof}

Using the above we now prove Bernstein's inequality.\footnote{Ronald A. DeVore and George G. Lorentz,
{\em Constructive Approximation}, p.~98.}

\begin{theorem}[Bernstein's inequality]
If $T \in \mathscr{T}_n$, then
\[
\norm{T'}_\infty \leq n \norm{T}_\infty.
\]
\label{bernstein}
\end{theorem}
\begin{proof}
There is some $x_0 \in \mathbb{T}$ such that $|T'(x_0)| = \norm{T'}_\infty$. 
Let $\alpha \in \mathbb{R}$ be such that
$e^{i\alpha} T'(x_0) = \norm{T'}_\infty$. 
Define $S(x)=\Re(e^{i\alpha} T(x))$ for $x \in \mathbb{T}$, which
satisfies $S'(x)=\Re(e^{i\alpha} T'(x))$ and in particular
\[
S'(x_0)=\Re(e^{i\alpha} T'(x_0))=
e^{i\alpha} T'(x_0)=\norm{T'}_\infty.
\]
Because $S \in \mathscr{T}_n$ and $S$ is real valued,  
Theorem \ref{szego} yields
\[
S'(x_0)^2+n^2 S(x_0)^2 \leq n^2 \norm{S}_\infty^2.
\]
A fortiori,
\[
S'(x_0)^2 \leq n^2 \norm{S}_\infty^2,
\]
giving, because $S'(x_0)=\norm{T'}_\infty$ and
$\norm{S}_\infty \leq \norm{T}_\infty$,
\[
\norm{T'}_\infty^2 \leq n^2  \norm{T}_\infty^2,
\]
proving the claim.
\end{proof}



The following is a version of Bernstein's inequality.\footnote{Ronald A. DeVore and George G. Lorentz,
{\em Constructive Approximation}, p.~101, Theorem 2.4.} 

\begin{theorem}
If $T \in \mathscr{T}_n$
and $A \subset \mathbb{T}$ is a Borel set, there is some $x_0 \in \mathbb{T}$ such that
\[
\int_A |T'(t)| dt \leq n \int_{A-x_0} |T(t)| dt.
\]
\end{theorem}
\begin{proof}
Let $A \subset \mathbb{T}$ be a Borel set with indicator function $\chi_A$.
Define
$Q:\mathbb{T} \to \mathbb{C}$ by
\[
Q(x) = \int_{\mathbb{T}} \chi_A(t) T(t+x) \sgn T'(t) dt, \qquad x \in
\mathbb{T},
\]
which we can write as
\begin{align*}
Q(x)&=\int_{\mathbb{T}} \chi_A(t) \sum_j \widehat{T}(j) e^{ij(t+x)} \sgn T'(t) dt\\
&=\sum_j \widehat{T}(j) \left( \int_{\mathbb{T}} \chi_A(t) e^{ijt} \sgn T'(t) dt \right)   e^{ijx},
\end{align*}
showing that $Q \in \mathscr{T}_n$.  Also,
\[
Q'(x) = \int_{\mathbb{T}} \chi_A(t) T'(t+x) \sgn T'(t) dt, \qquad x \in \mathbb{T}.
\]
 Let
$x_0 \in \mathbb{T}$ with $|Q(x_0)| = \norm{Q}_\infty$. Applying
Theorem \ref{bernstein} we get
\[
\norm{Q'}_\infty \leq n \norm{Q}_\infty.
\]
Using
\[
Q'(0) = \int_{\mathbb{T}} \chi_A(t) T'(t) \sgn T'(t) dt
=\int_{\mathbb{T}} \chi_A(t) |T'(t)| dt,
\]
this gives
\begin{align*}
\int_{\mathbb{T}} \chi_A(t) |T'(t)| dt&\leq n \norm{Q}_\infty\\
&=n|Q(t_0)|\\
&=n\left| \int_{\mathbb{T}} \chi_A(t) T(t+x_0) \sgn T'(t) dt \right|\\
&\leq n \int_{\mathbb{T}} \chi_A(t) |T(t+x_0)| dt\\
&=n \int_{\mathbb{T}} \chi_{A-x_0}(t) |T(t)| dt.
\end{align*}
\end{proof}

Applying the above with $A=\mathbb{T}$ gives the following version of Bernstein's inequality, for
the $L^1$ norm.

\begin{theorem}[$L^1$ Bernstein's inequality]
If $T \in \mathscr{T}_n$, then 
\[
\norm{T'}_1 \leq n \norm{T}_1.
\]
\end{theorem}


\section{Nikolsky's inequality for trigonometric polynomials}
DeVore and Lorentz attribute the following inequality to Sergey Nikolsky.\footnote{Ronald A. DeVore and George G. Lorentz,
{\em Constructive Approximation}, p.~102, Theorem 2.6.}

\begin{theorem}[Nikolsky's inequality]
If $T \in \mathscr{T}_n$ and $0<q \leq p \leq \infty$, then for $r \geq \frac{q}{2}$ an integer,
\[
\norm{T}_p \leq (2nr+1)^{\frac{1}{q}-\frac{1}{p}} \norm{T}_q.
\]
\end{theorem}
\begin{proof}
Let $m=nr$. Then $T^r \in \mathscr{T}_m$, so $T^r * D_m = T^r$, and using this and the Cauchy-Schwarz inequality we have,
for $x \in \mathbb{T}$,
\begin{align*}
|T(x)^r|&= \left| \frac{1}{2\pi} \int_{\mathbb{T}} T(t)^r D_m(x-t)\right|\\
&\leq  \frac{1}{2\pi} \int_{\mathbb{T}} |T(t)|^r |D_m(x-t)| dt\\
&\leq  \norm{T}_\infty^{r-\frac{q}{2}}  \cdot \frac{1}{2\pi} \int_{\mathbb{T}}
|T(t)|^{\frac{q}{2}} |D_m(x-t)| dt\\
&\leq   \norm{T}_\infty^{r-\frac{q}{2}} \norm{ |T|^{q/2}}_2 \norm{D_m}_2\\
&= \norm{T}_\infty^{r-\frac{q}{2}} \norm{T}_q^{\frac{q}{2}} \norm{\widehat{D_m}}_{\ell^2(\mathbb{Z})}\\
&=\sqrt{2m+1}  \norm{T}_\infty^{r-\frac{q}{2}} \norm{T}_q^{\frac{q}{2}}.
\end{align*}
Hence
\[
\norm{T}_\infty^r \leq \sqrt{2m+1}  \norm{T}_\infty^{r-\frac{q}{2}} \norm{T}_q^{\frac{q}{2}},
\]
thus
\[
\norm{T}_\infty \leq  (2m+1)^{\frac{1}{q}}  \norm{T}_q.
\]
Then, using
$\norm{T}_p \leq \norm{T}_\infty^{1-\frac{q}{p}} \norm{T}_q^{\frac{q}{p}}$,
we have
\[
\norm{T}_p \leq (2m+1)^{\frac{1}{q}-\frac{1}{p}} \norm{T}_q^{1-\frac{q}{p}}
\norm{T}_q^{\frac{q}{p}}
=(2m+1)^{\frac{1}{q}-\frac{1}{p}} \norm{T}_q.
\]
\end{proof}


\section{The complementary Bernstein inequality}
We define a \textbf{homogeneous Banach space} to be a linear subspace $B$
of $L^1(\mathbb{T})$ with a norm $\norm{f}_{L^1(\mathbb{T})} \leq
\norm{f}_B$ with which $B$ is a Banach space,
such that if $f \in B$ and $\tau \in \mathbb{T}$ then $f_\tau \in B$ and $\norm{f_\tau}_B = \norm{f}_B$,
and such that if $f \in B$ then $f_\tau \to f$ in $B$ as $\tau \to 0$. 

 \textbf{Fej\'er's kernel} is, for $n \geq 0$,
\[
K_n(t) = \sum_{|j| \leq n} \left(1-\frac{|j|}{n+1} \right) e^{ijt}=
\sum_{j \in \mathbb{Z}} \chi_n(j) \left(1-\frac{|j|}{n+1} \right) e^{ijt} \qquad t \in \mathbb{T}.
\]
One calculates that, for $t \not \in 4\pi \mathbb{Z}$,
\[
K_n(t) = \frac{1}{n+1} \left( \frac{\sin \frac{n+1}{2}t}{\sin \frac{1}{2}t} \right)^2.
\]

Bernstein's inequality is a statement about functions whose Fourier transform is supported only on low frequencies. The following is a statement
about functions whose Fourier transform is supported only on high frequencies.\footnote{Yitzhak Katznelson, {\em An Introduction to Harmonic
Analysis}, third ed., p.~55, Theorem 8.4.} In particular, for $1 \leq p < \infty$, $L^p(\mathbb{T})$ is a homogeneous Banach space, and
so is $C(\mathbb{T})$ with the supremum norm.

\begin{theorem}
Let $B$ be a homogeneous Banach space and let $m$ be a positive integer.
Define $C_m$ as $C_m=m+1$ if  $m$ is even and $C_m=12m$ if $m$ is odd.
If
\[
f(t)=\sum_{|j| \geq n} a_j e^{ijt}, \qquad t \in \mathbb{T},
\]
is $m$ times differentiable and $f^{(m)} \in B$, then
$f \in B$ and 
\[
\norm{f}_B \leq C_m n^{-m} \norm{f^{(m)}}_B.
\]
\end{theorem}
\begin{proof}
Suppose that $m$ is even.
It is a fact that if $a_j,  j\in \mathbb{Z}$, is an even sequence of nonnegative real numbers such that $a_j \to 0$ as $|j| \to \infty$ and 
such that for each $j>0$,
\[
a_{j-1}+a_{j+1}-2a_j \geq 0,
\]
then there is a nonnegative function $f \in L^1(\mathbb{T})$ such that
$\hat{f}(j)=a_j$ for all $j \in \mathbb{Z}$.\footnote{Yitzhak Katznelson, {\em An Introduction to Harmonic
Analysis}, third ed., p.~24, Theorem 4.1.}
Define 
\[
a_j = \begin{cases}
j^{-m}&|j| \geq n\\
n^{-m} + (n-|j|)(n^{-m}-(n+1)^{-m})&|j| \leq n-1.
\end{cases}
\]
It is apparent that $a_j$ is even and tends to $0$ as $|j| \to \infty$. For 
$1 \leq j \leq n-2$,
\[
a_{j-1}+a_{j+1}-2a_j = 0.
\]
For $j=n-1$,
\begin{align*}
a_{j-1}+a_{j+1}-2a_j &= n^{-m}+(n-(n-2))(n^{-m}-(n+1)^{-m})+n^{-m}\\
&-2\left(n^{-m} + (n-(n-1))(n^{-m}-(n+1)^{-m})\right)\\
&=0.
\end{align*}
The function $j \mapsto j^{-m}$ is convex on $\{n,n+1,\ldots\}$, as $m \geq 1$, so
for $j \geq n$ we have $a_{j-1}+a_{j+1}-2a_j \geq 0$. Therefore,
there is some nonnegative $\phi_{m,n} \in L^1(\mathbb{T})$ such that
\[
\widehat{\phi_{m,n}}(j) = a_j, \qquad j \in \mathbb{Z}.
\]
Because $\phi_{m,n}$ is nonnegative, and using
$n^{-m}-(n+1)^{-m} < \frac{m}{n} n^{-m}$,
\[
\norm{\phi_{m,n}}_1 = \widehat{\phi_{m,n}}(0) = n^{-m}+n(n^{-m}-(n+1)^{-m})
<(m+1)n^{-m}.
\]
Define $d\mu_{m,n}(t) =\frac{1}{2\pi} \phi_{m,n}(t) dt$. 
For
$|j| \geq n$,
\begin{align*}
\widehat{f^{(m)}*\mu_{m,n}}(j)&=\widehat{f^{(m)}}(j) \widehat{\mu_{m,n}}(j)\\
&=(ij)^m \hat{f}(j) \widehat{\phi_{m,n}}(j)\\
&=(ij)^m \hat{f}(j)  \cdot |j|^{-m}\\
&=i^m \hat{f}(j).
\end{align*}
For $|j|<n$, since $\hat{f}(j)=0$ we have
\[
\widehat{f^{(m)}*\mu_{m,n}}(j)=(ij)^m \hat{f}(j) \widehat{\phi_{m,n}}(j)=0=i^m \hat{f}(j),
\]
so for all $j \in \mathbb{Z}$,
\[
\widehat{f^{(m)}*\mu_{m,n}}(j)=i^m \hat{f}(j).
\]
This implies that $f^{(m)}*\mu_{m,n}=i^m f$, which in particular tells us that $f \in B$.
Then,
\begin{align*}
\norm{f}_B &= \norm{i^m f}_B\\
&=\norm{f^{(m)}*\mu_{m,n}}_B\\
&\leq \norm{f^{(m)}}_B \norm{\mu_{m,n}}_{M(\mathbb{T})}\\
&=\norm{\phi_{m,n}}_1 \norm{f^{(m)}}_B\\
&\leq (m+1)n^{-m}  \norm{f^{(m)}}_B.
\end{align*}
This shows what we want in the case that $m$ is even, with $C_m=m+1$.

Suppose that $m$ is odd. For $l$ a positive integer, define $\psi_l:\mathbb{T} \to \mathbb{C}$ by
\[
\psi_l(t) = \left(e^{2lit}+\frac{1}{2}e^{3lit} \right) K_{l-1}(t), \qquad  t \in \mathbb{T}.
\]
There is a unique  $l_n$ such that $n \in \{2l_n,2l_n+1\}$. For $k \geq 0$ an integer, define
$\Psi_{n,k}:\mathbb{T} \to \mathbb{C}$ by
\[
\Psi_{n,k}(t) = \psi_{l_n 2^k}(t), \qquad t \in \mathbb{T}.
\]
$\Psi_{n,k}$ satisfies
\[
\norm{\Psi_{n,k}}_1 \leq \frac{3}{2} \norm{K_{k-1}}_1 
=\frac{3}{2} \cdot \frac{1}{2\pi} \int_{\mathbb{T}} |K_{k-1}(t)| dt
=\frac{3}{2} \cdot \frac{1}{2\pi}  \int_{\mathbb{T}} K_{k-1}(t) dt=\frac{3}{2}.
\]
On the one hand,
for $j \leq 0$,
from the definition of $\psi_l$ we have $\widehat{\Psi_{n,k}}(j)=0$, hence
$\sum_{k=0}^\infty \widehat{\Psi_{n,k}}(j)=0$.
On the other hand, for $j \geq n$ we assert that
\[
\sum_{k=0}^\infty \widehat{\Psi_{n,k}}(j)=1.
\]
We define $\Phi_n: \mathbb{T} \to \mathbb{C}$ by
\[
\Phi_n(t) =\sum_{k=0}^\infty (\Psi_{n,k}*\phi_{1,n2^k})(t), \qquad t \in \mathbb{T}.
\]
We calculate the Fourier coefficients of $\Phi_n$. For $j \geq n$, 
\[
\widehat{\Phi_n}(j) = \sum_{k=0}^\infty \widehat{\Phi_{n,k}}(j) \widehat{\phi_{1,n2^k}}(j)
=\frac{1}{j} \sum_{k=0}^\infty   \widehat{\Phi_{n,k}}(j) 
=\frac{1}{j}.
\]
As well,
\[
\norm{\Phi_n}_1 \leq \sum_{k=0}^\infty \norm{\Psi_{n,k}*\phi_{1,n2^k}}_1
\leq \sum_{k=0}^\infty \norm{\Psi_{n,k}}_1 \norm{\phi_{1,n2^k}}_1
\leq \frac{3}{2}  \sum_{k=0}^\infty 2 (n2^k)^{-1}
=\frac{6}{n}
\]
We now define
\[
d\mu_{1,n}(t) = \frac{1}{2\pi} (\Phi_n(t)-\Phi_n(-t)) dt,
\]
which satisfies for $|j| \geq n$,
\[
\widehat{\mu_{1,n}}(j)=\widehat{\Phi_n}(j)-\widehat{\Phi_n}(-j)=\frac{1}{j}
\]
and hence
\[
\widehat{f'*\mu_{1,n}}(j)=\widehat{f'}(j) \widehat{\mu_{1,n}}(j)=ij \hat{f}(j) \cdot \frac{1}{j}=i \hat{f}(j).
\]
Because $\hat{f}(j)=0$ for $|j|<n$, 
$\widehat{f'*\mu_{1,n}}(j) =0$ for $|j|<n$, it follows that for any $j \in \mathbb{Z}$,
\[
\widehat{f'*\mu_{1,n}}(j) = i \hat{f}(j),
\]
and therefore,
\[
f'*\mu_{1,n} = if.
\]
Then
\[
\norm{f}_B = \norm{if}_B = \norm{f'*\mu_{1,n}}_B \leq 
\norm{\mu_{1,n}}_{M(\mathbb{T})} \norm{f'}_B
\leq 2 \norm{\Phi_n}_1 \norm{f'}_B
\leq \frac{12}{n} \norm{f'}_B.
\]
That is, with $C_1=12$ we have
\[
\norm{f}_B \leq 12 n^{-1} \norm{f'}_B.
\]
For $m=2\nu+1$, we define 
\[
\mu_{m,n} = \mu_{1,n}*\mu_{2\nu,n},
\]
for which we have, for $|j| \geq n$,
\[
\widehat{f^{(m)} *\mu_{m,n}}(j) = 
(ij)^m \hat{f}(j) \widehat{\mu_{1,n}}(j) \widehat{\mu_{2\nu,n}}(j)
=(ij)^m \hat{f}(j) \cdot \frac{1}{j} \cdot j^{-2\nu}
=i^m \hat{f}(j).
\]
It follows that 
\[
f^{(m)}*\mu_{m,n} = i^m f,
\]
whence
\begin{align*}
\norm{f}_B &= \norm{i^m f}_B \\
&= \norm{f^{(m)}*\mu_{m,n}}_B\\
&\leq \norm{\mu_{m,n}}_{M(\mathbb{T})} \norm{f^{(m)}}_B\\
&\leq \norm{\mu_{1,n}}_{M(\mathbb{T})}
\norm{\mu_{2\nu,n}}_{M(\mathbb{T})} 
\norm{f^{(m)}}_B\\
&\leq \frac{12}{n} \cdot (2\nu+1)n^{-2\nu} \norm{f^{(m)}}_B\\
&=12mn^{-m}  \norm{f^{(m)}}_B.
\end{align*}
That is, with $C_m=12m$, we have
\[
\norm{f}_B \leq C_m n^{-m} \norm{f^{(m)}}_B,
\]
completing the proof.
\end{proof}




\end{document}