\documentclass{article}
\usepackage{amsmath,amssymb,graphicx,subfig,mathrsfs,amsthm}
%\usepackage{tikz-cd}
\newcommand{\innerL}[2]{\langle #1, #2 \rangle_{L^2}}
\newcommand{\inner}[2]{\left\langle #1, #2 \right\rangle}
\newcommand{\HSinner}[2]{\left\langle #1, #2 \right\rangle_{\ensuremath\mathrm{HS}}}
\newcommand{\tr}{\ensuremath\mathrm{tr}\,} 
\newcommand{\Span}{\ensuremath\mathrm{span}} 
\def\Re{\ensuremath{\mathrm{Re}}\,}
\def\Im{\ensuremath{\mathrm{Im}}\,}
\newcommand{\id}{\ensuremath\mathrm{id}} 
\newcommand{\GL}{\ensuremath\mathrm{GL}} 
\newcommand{\rank}{\ensuremath\mathrm{rank\,}} 
\newcommand{\co}{\ensuremath\mathrm{co}}
\newcommand{\ext}{\ensuremath\mathrm{ext}\,}
\newcommand{\Ind}{\ensuremath\mathrm{Ind}}
\newcommand{\cco}{\overline{\ensuremath\mathrm{co}}}
\newcommand{\point}{\ensuremath\sigma_{\mathrm{point}}} 
\newcommand{\supp}{\ensuremath\mathrm{supp}}
\newcommand{\Hom}{\ensuremath\mathrm{Hom}}
\newcommand{\norm}[1]{\left\Vert #1 \right\Vert}
\newtheorem{theorem}{Theorem}
\newtheorem{lemma}[theorem]{Lemma}
\newtheorem{proposition}[theorem]{Proposition}
\newtheorem{corollary}[theorem]{Corollary}
\theoremstyle{definition}
\newtheorem{definition}[theorem]{Definition}
\newtheorem{remark}[theorem]{Remark}
\begin{document}
\title{Banach algebras}
\author{Jordan Bell}
\date{April 3, 2014}

\maketitle

\section{Introduction}
This note is a collection of results on Banach algebras whose proofs do not require the machinery of integrating functions that take values in Banach spaces, and that
do not require the algebra to be commutative.

\section{Banach algebras and ideals}
Unless we say otherwise, every vector  space we talk about is taken to be over $\mathbb{C}$. A   {\em Banach algebra}  is a  Banach space $\mathfrak{A}$ that is also an algebra satisfying
 $\norm{AB} \leq \norm{A} \norm{B}$ for $A,B \in \mathfrak{A}$. We say that $\mathfrak{A}$ is {\em unital} if there is 
 a nonzero element $I \in \mathfrak{A}$ such that $AI=A$ and $IA=A$ for all $A \in \mathfrak{A}$, called a {\em identity element}. 
If $X$ is a Banach space,
let $\mathscr{B}(X)$ denote the set of bounded linear operators $X \to X$, and let $\mathscr{B}_0(X)$ denote the set of compact linear operators
$X \to  X$ (to be compact means that the image of each bounded set is precompact). $\mathscr{B}(X)$ is a unital Banach algebra. $\mathscr{B}_0(X)$ is a Banach algebra,
but it is unital if and only if $X$ is finite dimensional.
 
An {\em ideal} of a Banach algebra $\mathfrak{A}$ is a vector subspace $\mathfrak{I}$  such that
\[
\mathfrak{I}\mathfrak{A} \subseteq \mathfrak{I}, \qquad \mathfrak{A} \mathfrak{I} \subseteq \mathfrak{I}.
\]
We define the algebra quotient $\mathfrak{A} / \mathfrak{I}$ by 
\[
\mathfrak{A} / \mathfrak{I} = \left\{ A+\mathfrak{I}: A \in \mathfrak{A}\right\},
\]
and define
\[
(A+\mathfrak{I})(B+\mathfrak{I}) = AB+\mathfrak{I};
\]
this makes sense because $\mathfrak{I}$ is an ideal.
We define a seminorm on $\mathfrak{A}/\mathfrak{I}$ by 
\[
\norm{A+\mathfrak{I}} = \inf_{S \in \mathfrak{I}} \norm{A-S};
\]
we call this the {\em quotient seminorm}. In the following theorem we show that if the ideal is closed then this seminorm is a norm.

\begin{theorem}
If $\mathfrak{I}$ is a closed ideal in a Banach algebra $\mathfrak{A}$, then
$\mathfrak{A} / \mathfrak{I}$ is a Banach algebra with the quotient norm.
\end{theorem}
\begin{proof}
Suppose that $\norm{A+\mathfrak{I}}=0$. This means
\[
\inf_{S \in \mathfrak{I}} \norm{A-S} = 0.
\]
Let $S_n \in \mathfrak{I}$ with $\norm{A-S_n} \to 0$. That is, $S_n \to A$. Since $\mathfrak{I}$ is closed we get $A \in \mathfrak{I}$, and hence
$A+\mathfrak{I} = 0 + \mathfrak{I}$, showing that the seminorm on the quotient is in fact a norm.

Let $z_n \in \mathfrak{A} / \mathfrak{I}$ be a Cauchy sequence and let $z_{a(n)}$ be a subsequence with $\norm{z_{a(n+1)}-z_{a(n)}}<2^{-n}$ for all $n \in \mathbb{N}$.
We shall use the fact that $\mathfrak{A}$ is complete to prove that the sequence $z_n$ converges
and therefore that $\mathfrak{A}/\mathfrak{I}$ is complete.
We define a sequence $A_n$ in $\mathfrak{A}$ inductively as follows.
Let $A_1$ be any element of $z_{a(1)}$. 
Suppose that $z_{a(n)}=A_n+\mathfrak{I}$, and let $T$ be any element of $z_{a(n+1)}$. We have
\[
\norm{z_{a(n+1)}-z_{a(n)}} = \inf_{S \in \mathfrak{I}} \norm{T-A_n-S},
\]
hence
\[
 \inf_{S \in \mathfrak{I}} \norm{T-A_n-S}<2^{-n}.
\]
As this is an infimum and the inequality is strict, there is some $S \in \mathfrak{I}$ for which $\norm{T-A_n-S}< 2^{-n}$, and we define $A_{n+1}=T-S$. Thus defined, the sequence $A_n$ satisfies
$\norm{A_{n+1}-A_n}<2^{-n}$ for all $n$ and hence is a Cauchy sequence (as the consecutive differences are summable). Thus $A_n$ converges to some $A \in
\mathfrak{A}$, as $\mathfrak{A}$ is a Banach space. We have
\[
\norm{z_{a(n)}-(A+\mathfrak{I})} = \inf_{S \in \mathfrak{I}} \norm{A_n-A-S} \leq \inf_{S \in \mathfrak{I}} \norm{A_n-A} + \inf_{S \in \mathfrak{I}} \norm{S}=
\norm{A_n-A},
\]
which tends to $0$ as $n \to \infty$. Therefore $z_{a(n)} \to A+\mathfrak{I}$. As $z_n$ is a Cauchy sequence a subsequence of which converges
to $A+\mathfrak{I}$, it follows that $z_n \to A+\mathfrak{I}$. We have shown that each Cauchy sequence in $\mathfrak{A}/\mathfrak{I}$ converges, and hence
$\mathfrak{A} / \mathfrak{I}$ is complete.
\end{proof}

If $X$ is a Banach space, then
 $\mathscr{B}_0(X)$ is a closed ideal of the Banach algebra $\mathscr{B}(X)$.
Hence with the quotient norm, the  quotient algebra $\mathscr{B}(X) / \mathscr{B}_0(X)$ is a Banach algebra.
Let $\mathscr{B}_{00}(X)$ denote the set of bounded finite rank linear operators $X \to X$ (to be finite rank means to have a finite dimensional image).
$\mathscr{B}_{00}(X)$ is an ideal of $\mathscr{B}(X)$, but it is closed if and only if $X$ is finite dimensional. The closure of $\mathscr{B}_{00}(X)$ is contained in 
$\mathscr{B}_0(X)$. (If the closure is equal to $\mathscr{B}_0(X)$ we say that the Banach space $X$ has the {\em approximation property}.)


\section{Left regular representation}
A Banach algebra $\mathfrak{A}$ is in particular a Banach space, and thus $\mathscr{B}(\mathfrak{A})$
is itself a Banach algebra with the operator norm. The {\em left regular representation} is the map $L:\mathfrak{A} \to \mathscr{B}(\mathfrak{A})$ defined by
\[
L(A)B=AB, \qquad A,B \in \mathfrak{A}.
\]
It is apparent that $L$ is an algebra homomorphism, and for $A,B \in \mathfrak{A}$,
\[
\norm{L(A)B} = \norm{AB} \leq \norm{A} \norm{B},
\]
showing that $\norm{L(A)} \leq \norm{A}$. If $\mathfrak{A}$ has identity element $I$, then
$\norm{I}=\norm{I\cdot I} \leq \norm{I} \norm{I}$, and as $\norm{I} \neq 0$ we get $\norm{I} \geq 1$.
We'd like the norm on $\mathfrak{A}$ to satisfy $\norm{I}=1$, and we define 
\[
\norm{A}_0=\norm{L(A)}=\sup_{\norm{B} \leq 1} \norm{L(A)B} \leq \norm{A}.
\]
Check that $L$ is injective, from which it follows that $\norm{\cdot}_0$ is nondegenerate and is thus a norm on $\mathfrak{A}$. On the other hand,
\[
\norm{A} = \norm{L(A)I} \leq \norm{L(A)} \norm{I}=\norm{A}_0 \norm{I}.
\]
Therefore,
\[
\norm{A}_0 \leq \norm{A} \leq \norm{I} \norm{A}_0,
\]
so $\norm{\cdot}$ and $\norm{\cdot}_0$ are equivalent norms on $\mathfrak{A}$, and thus for most purposes anything we say using one can be said just as well using the other.
We decree that if we are talking about a unital Banach algebra then its norm is such that $\norm{I}=1$. Then, $L$ is an isometry.


\section{Invertible elements}
If $\mathfrak{A}$ is a unital Banach algebra, we say that $A \in \mathfrak{A}$ is {\em invertible} if there is some $B \in \mathfrak{A}$ such that
$AB=I$ and $BA=I$, and we call $A^{-1}=B$ the {\em inverse} of $A$. We denote by
 $\GL(\mathfrak{A})$ the set of invertible elements of $\mathfrak{A}$, and we call $\GL(\mathfrak{A})$ the {\em multiplicative group} of the Banach algebra.
We now prove that a   perturbation of norm $<1$ of the identity element in a unital Banach algebra remains in the multiplicative group.\footnote{Walter Rudin, {\em Functional Analysis}, second ed., p.~249, Theorem 10.7.}
 
 \begin{theorem}
 If $\mathfrak{A}$ is a unital Banach algebra, $A \in \mathfrak{A}$, and $\norm{A} < 1$, then $I-A \in \GL(\mathfrak{A})$, and
 \[
 \norm{(I-A)^{-1}-I-A} \leq \frac{\norm{A}^2}{1-\norm{A}}.
 \]
 \label{neumann}
 \end{theorem}
 \begin{proof}
 Define $S_n=\sum_{k=0}^n A^k$. For $n > m$ we have
 \[
 \norm{S_n-S_m} \leq \sum_{k=m+1}^n \norm{A}^k = \norm{A}^{m+1} \frac{1-\norm{A}^{n-m}}{1-\norm{A}} \leq \frac{\norm{A}^{m+1}}{1-\norm{A}}.
 \]
 Thus $S_n$ is a Cauchy sequence and so has a limit $S \in \mathfrak{A}$. Because
 $ S_n (I-A) = I-A^{n+1}$,
 we have
 \[
 \norm{S_n(I-A)-I} = \norm{A^{n+1}} \leq \norm{A}^{n+1} \to 0,
 \]
 so $S_n(I-A) \to I$. But
 \[
 \norm{S_n(I-A) -S(I-A)} \leq \norm{S_n-S}\norm{I-A} \to 0,
 \]
 so $S_n(I-A) \to S(I-A)$. Therefore $S(I-A)=I$. One similarly shows that $(I-A)S=I$, and hence that $I-A \in \GL(\mathfrak{A})$,
 with $(I-A)^{-1}=S$. 
Furthermore,
\[
\norm{S-I-A} = \lim_{n \to \infty} \norm{S_n-I-A} = \lim_{n \to \infty} \norm{\sum_{k=2}^n A^k}
\leq \lim_{n \to \infty} \sum_{k=2}^n \norm{A}^k
=\frac{\norm{A}^2}{1-\norm{A}}.
\]
 \end{proof}
 
 In the above  theorem we found that if $\norm{A}<1$ then $(I-A)^{-1} = \sum_{n=0}^\infty A^n$. This is an analog of the geometric series, and is called
 a {\em Neumann series}. 
 
 If $\mathfrak{A}$ is a Banach algebra and $\phi:\mathfrak{A} \to \mathbb{C}$ is a linear map satisfying
 \[
 \phi(AB)=\phi(A)\phi(B),\qquad A,B \in \mathfrak{A},
 \]
 then $\phi$ is called a {\em complex homomorphism on $\mathfrak{A}$}.
 That is, a complex homomorphism on $\mathfrak{A}$ is an algebra homomorphism $\mathfrak{A} \to \mathbb{C}$. It is straightforward to prove that if $\mathfrak{A}$ is a unital
 Banach algebra and $\phi$ is a complex homomorphism on $\mathfrak{A}$, then $\phi(I)=1$ and $\phi(A) \neq 0$ for all $A \in \GL(\mathfrak{A})$. A linear functional
 on a Banach algebra need not be continuous, but in the following  we prove that a nonzero complex homomorphism has operator norm 1, and in particular is continuous.
 
  \begin{corollary}
 If $\mathfrak{A}$ is a unital Banach algebra and $\phi$ is a nonzero complex homomorphism on $\mathfrak{A}$, then $\norm{\phi}=1$.
 \end{corollary}
 \begin{proof}
 Let $A \in \mathfrak{A}$ with $\norm{A}<1$. If $\lambda \in \mathbb{C}$ with $|\lambda| \geq 1$ then $\norm{\lambda^{-1}A}=|\lambda|^{-1} \norm{A}<1$, and
by Theorem \ref{neumann} we have $I-\lambda^{-1}A \in \GL(\mathfrak{A})$. Then,
\[
1-\lambda^{-1} \phi(A) = \phi(I)-\phi(\lambda^{-1}A) =\phi(I-\lambda^{-1}A) \neq 0,
\]
hence $1 \neq \lambda^{-1} \phi(A)$, i.e. $\phi(A) \neq \lambda$. This shows that $|\phi(A)| < 1$. 
 \end{proof}

  
 The {\em Gleason-Kahane-Zelazko theorem}\footnote{Walter Rudin, {\em Functional Analysis}, second ed., p.~251, Theorem 10.9}
 states that if $\mathfrak{A}$ is a unital Banach algebra and $\phi$ is a linear functional on $\mathfrak{A}$ satisfying
 $\phi(I)=1$ and $\phi(A) \neq 0$ for $A \in \GL(\mathfrak{A})$, then $\phi$ is a complex homomorphism.

 
 

 
 In Theorem \ref{neumann} we proved that a  perturbation of the identity element remains in the multiplicative group. We
 now prove that for any element of the multiplicative group, a sufficiently small perturbation of this element remains in the multiplicative
 group, i.e. that the multiplicative group is an open set.\footnote{Walter Rudin, {\em Functional Analysis}, second ed., p.~253, Theorem 10.11.} 
 
 \begin{theorem}
 If $\mathfrak{A}$ is a unital Banach algebra, $A \in \GL(\mathfrak{A})$, $h \in \mathfrak{A}$, and $\norm{h}<\frac{1}{2} \norm{A^{-1}}^{-1}$,
 then $A+h \in \GL(\mathfrak{A})$, and
 \[
 \norm{(A+h)^{-1}-A^{-1}+A^{-1}hA^{-1}} < 2 \norm{A^{-1}}^3 \norm{h}^2.
 \]
 \label{GLA}
 \end{theorem}
 \begin{proof}
 $\norm{A^{-1}h} \leq \norm{A^{-1}} \norm{h} < \frac{1}{2}$, so by Theorem \ref{neumann} we get $I+A^{-1}h \in \GL(\mathfrak{A})$. Therefore
\[
A+h = A(I+A^{-1}h) \in \GL(\mathfrak{A}).
\]
Furthermore,
\[
(A+h)^{-1}-A^{-1}+A^{-1}hA^{-1} = \big((I+A^{-1}h)^{-1}-I+A^{-1}h\big)A^{-1},
\]
and  by Theorem \ref{neumann},
\begin{align*}
\norm{\big((I+A^{-1}h)^{-1}-I+A^{-1}h\big)A^{-1}} &\leq \norm{(I+A^{-1}h)^{-1}-I+A^{-1}h} \norm{A^{-1}}\\
&\leq \frac{\norm{A^{-1}h}^2}{1-\norm{A^{-1}h}} \norm{A^{-1}}\\
&\leq \frac{\norm{A^{-1}}^3 \norm{h}^2}{1-\norm{A^{-1}h}}\\
&< 2 \norm{A^{-1}}^3 \norm{h}^2;
\end{align*}
the final inequality is because $\norm{A^{-1}h} \leq \norm{A^{-1}}\norm{h} < \frac{1}{2}$ and hence 
$1-\norm{A^{-1}h} > \frac{1}{2}$.
 \end{proof}



 
 
 \section{Spectrum}
 If $\mathfrak{A}$ is a unital Banach algebra and $A \in \mathfrak{A}$, the {\em spectrum} of $A$ is the set
 \[
 \sigma(A) = \{\lambda \in \mathbb{C}: \lambda I -A\not \in \GL(\mathfrak{A})\},
 \]
 and the {\em spectral radius} of $A$ is
 \[
 r(A)=\sup_{\lambda \in \sigma(A)} |\lambda|.
 \]
 If $|\lambda| > \norm{A}$ then  $\norm{\frac{A}{\lambda}}<1$ and so by Theorem \ref{neumann},
 \[
 \lambda I-A = \lambda\left(I-\frac{A}{\lambda}\right) \in \GL(\mathfrak{A}),
 \]
 so $\lambda \not \in \sigma(A)$. Therefore, 
 \[
 r(A) \leq \norm{A}.
 \]
 
 
For $\lambda \not \in \sigma(A)$, we define the {\em resolvent of $A$} by 
\[
R(A,\lambda)=(A-\lambda I)^{-1} \in \GL(\mathfrak{A}).
\]
The following is the {\em resolvent identity}.

\begin{theorem}[Resolvent identity]
If $\mathfrak{A}$ is a unital Banach algebra, $A \in \mathfrak{A}$, and $\lambda,\mu \not \in \sigma(A)$, then
\[
R(A,\lambda)-R(A,\mu) = (\lambda-\mu)R(A,\lambda)R(A,\mu).
\]
\label{resolvent}
\end{theorem}
\begin{proof}
We have
\begin{eqnarray*}
(A-\lambda I)(R(A,\lambda)-R(A,\mu))(A-\mu I)&=&(A-\lambda I)R(A,\lambda)(A-\mu I)\\
&&-(A-\lambda I)R(A,\mu)(A-\mu I)\\
&=&(A-\mu I)-(A-\lambda I)\\
&=&(\lambda -\mu)I.
\end{eqnarray*}
\end{proof}

The {\em resolvent set} of $A$ is $\rho(A)=\mathbb{C} \setminus \sigma(A)$.
The following theorem implies that the resolvent set of $A$ is  open, and therefore that the spectrum
of $A$ is  closed. Because $r(A) \leq \norm{A}$, it follows that $\sigma(A)$ is a compact set.

\begin{theorem}
If $\mathfrak{A}$ is a unital Banach algebra, $A \in \mathfrak{A}$, and $\lambda \in \rho(A)$, then $|\mu-\lambda| < \norm{R(A,\lambda)}^{-1}$
implies that $\mu \in \rho(A)$.
\label{open}
\end{theorem}


Moreover, $\sigma(A)$ is nonempty, and although $r(A)$ may be strictly less than $\norm{A}$, the spectral
radius  is equal to a limit of
norms.\footnote{Gert K. Pedersen, {\em Analysis Now}, revised printing, p.~131, Theorem 4.1.13. This proof does not use the holomorphic functional
calculus, while Walter Rudin, {\em Functional Analysis}, second ed., p.~253, Theorem 10.13 does. That is, to read Pedersen's proof one does
not have to make sense of   integrals of functions taking values in a Banach space, while to read Rudin's one does.}

\begin{theorem}
If $\mathfrak{A}$ is a unital Banach algebra and $A \in \mathfrak{A}$, then
 $\sigma(A)$ is  nonempty, and
\[
r(A) = \lim_{n \to \infty} \norm{A^n}^{1/n}.
\]
\end{theorem}

Using the fact that the spectrum of any element of a unital Banach algebra is nonempty, one can prove the
 {\em Gelfand-Mazur theorem}, which states that if $\mathfrak{A}$ is a unital Banach algebra for which every nonzero element is invertible, then there is an isometric
algebra isomorphism $\mathfrak{A} \to \mathbb{C}$.\footnote{Walter Rudin, {\em Functional Analysis}, second ed., p.~255, Theorem 10.14.}



The following theorem shows
that $\lambda \mapsto R(A,\lambda)$ is a continuous function $\rho(A) \to \mathfrak{A}$. From this and the resolvent identity it follows that
for any $\lambda \in \rho(A)$,
\[
\lim_{\mu \to \lambda} \frac{R(A,\mu)-R(A,\lambda)}{\mu-\lambda} = R(A,\lambda)^2,
\]
i.e.
\[
R'(A,\lambda)=R(A,\lambda)^2.
\]

\begin{theorem}
If $\mathfrak{A}$ is a unital Banach algebra and $A \in \mathfrak{A}$, then $\lambda \mapsto R(A,\lambda)$ is a continuous function
$\rho(A) \to \mathfrak{A}$.
\label{resolventcont}
\end{theorem}
\begin{proof}
From Theorem \ref{resolvent},
if $\lambda,\mu  \in \rho(A)$ then
\begin{equation}
\norm{R_\mu-R_\lambda} \leq |\mu-\lambda| \norm{R_\lambda} \norm{R_\mu},
\label{resolventeq}
\end{equation}
where we write $R_\lambda=R(A,\lambda)$.
As
\[
\norm{R_\mu-R_\lambda} \geq \norm{R_\mu}-\norm{R_\lambda},
\]
we get
\[
\norm{R_\mu} \leq |\lambda-\mu| \norm{R_\lambda} \norm{R_\mu} + \norm{R_\lambda},
\]
which is
\begin{equation}
\norm{R_\mu}(1-|\lambda-\mu| \norm{R_\lambda}) \leq \norm{R_\lambda}.
\label{Rmu}
\end{equation}

If $\lambda \in \rho(A)$ and 
$|\mu-\lambda| \leq \frac{1}{2}\cdot \norm{R_\lambda}^{-1}$, then
from Theorem \ref{open} we get $\mu \in \rho(A)$, and  combined with \eqref{Rmu} this gives 
\[
\norm{R_\mu} \leq 2 \norm{R_\lambda}.
\]
Applying this to \eqref{resolventeq} we have that if $\lambda \in \rho(A)$ and $|\mu-\lambda| \leq \frac{1}{2}\cdot \norm{R_\lambda}^{-1}$ then
\[
\norm{R_\mu-R_\lambda} \leq 2|\mu-\lambda| \norm{R_\lambda}^2,
\]
from which it follows that 
 $\lambda \mapsto R_\lambda$ is a continuous function $\rho(A) \to \mathfrak{A}$.
\end{proof}


The following theorem tells us that if the spectrum of an element of a unital Banach algebra is contained in an open set,
 then the spectrum of a sufficiently small perturbation of that element is contained in the same open set.\footnote{Walter Rudin, {\em Functional Analysis}, second ed., p.~257, Theorem 10.20.} 
 
 \begin{theorem}
 If $\mathfrak{A}$ is a unital Banach algebra, $A \in \mathfrak{A}$, and $\Omega$ is an open subset of $\mathbb{C}$ containing
 $\sigma(A)$, then there is some $\delta>0$ such that $\norm{h}<\delta$ implies that $\sigma(A+h) \subset \Omega$.
 \end{theorem}
 \begin{proof}
 If $|\lambda|>\norm{A}$, then
\[
R(A,\lambda)=\left(A-\lambda I \right)^{-1}= \frac{1}{\lambda} \cdot \left(\frac{A}{\lambda}- I\right)^{-1}=-\frac{1}{\lambda} \cdot \sum_{n=0}^\infty \left( \frac{A}{\lambda} \right)^n,
\]
hence
\[
\norm{R(A,\lambda)} \leq \frac{1}{|\lambda|} \sum_{n=0}^\infty \left( \frac{\norm{A}}{|\lambda|} \right)^n=\frac{1}{|\lambda|} \cdot \frac{1}{1-\frac{\norm{A}}{|\lambda|}}
=\frac{1}{|\lambda|-\norm{A}}.
\]
Hence $\lim_{|\lambda| \to \infty} \norm{R(A,\lambda)}=0$.
By Theorem \ref{resolventcont}, $\lambda \mapsto \norm{R(A,\lambda)}$ is a continuous function $\rho(A) \to \mathbb{R}$. 
It follows that there is some $M>0$ such that $\lambda \not \in \Omega$ implies that $\norm{R(A,\lambda)}<M$.
If $\norm{h}<\frac{1}{M}$ and $\lambda \not \in \Omega$,
then
\[
\norm{R(A,\lambda)h} \leq \norm{R(A,\lambda)} \norm{h} < 1,
\]
which implies that $R(A,\lambda)h+I \in \GL(\mathfrak{A})$. 
From this we obtain that
\[
A+h-\lambda I=(A-\lambda I)(R(A,\lambda)h+I) \in \GL(\mathfrak{A}),
\]
which means that $\lambda \not \in \sigma(A+h)$. We have shown that if $\norm{h}<\frac{1}{M}$ then
$\sigma(A+h) \subset \Omega$.
 \end{proof}
 
  
The following theorem gives conditions under which we can evaluate a holomorphic function at an element of a unital Banach algebra, and shows
that the image of the spectrum is contained in the spectrum of the image.

\begin{theorem}
If $\mathfrak{A}$ is a unital Banach algebra, $A \in \mathfrak{A}$, $\norm{A} \leq r$, and $f(z)=\sum_{n=0}^\infty \alpha_n z^n$ is holomorphic on a domain that contains the closed disc $\overline{B_r(0)}$,
then
\[
f(A)=\sum_{n=0}^\infty \alpha_n A^n \in \mathfrak{A},
\]
and if $\lambda \in \sigma(A)$ then $f(\lambda) \in \sigma(f(A))$.
\label{mapping}
\end{theorem}
\begin{proof}
For $N>M$,
\begin{eqnarray*}
\norm{\sum_{n=0}^N \alpha_n A^n -\sum_{n=0}^M \alpha_n A^n} &=&\norm{ \sum_{n=M+1}^N \alpha_n A^n}\\
&\leq& \sum_{n=M+1}^N |\alpha_n| \norm{A^n}\\
&\leq&\sum_{n=M+1}^N |\alpha_n| \norm{A}^n\\
&\leq&\sum_{n=M+1}^N |\alpha_n| r^n.
\end{eqnarray*}
Because the radius of convergence of $\sum_{n=0}^\infty \alpha_n z^n$ is $>r$, it follows that the above sums tend to $0$ as $M \to \infty$, showing that
$f(A) \in \mathfrak{A}$.
Define
\[
P_n(A,\lambda)=\sum_{k=0}^n \lambda^k A^{n-k},
\]
giving
\[
(A-\lambda I)P_n(A,\lambda)=\sum_{k=0}^n\Big( \lambda^k A^{n-k+1} -  \lambda^{k+1} A^{n-k}\Big)
=A^{n+1} - \lambda^{n+1} I.
\]
We have
\[
f(A)-f(\lambda)I = \sum_{n=1}^\infty \alpha_n(A^n - \lambda^n I) =
(A-\lambda I) \sum_{n=1}^\infty \alpha_n P_{n-1}(A,\lambda).
\]
But  if $|\lambda| \leq r$ then
\[
\norm{P_n(A,\lambda)} \leq \sum_{k=0}^n |\lambda|^k \norm{A^{n-k}}
\leq \sum_{k=0}^n |\lambda|^k \norm{A}^{n-k} \leq \sum_{k=0}^n r^n=(n+1)r^n,
\]
and if $N>M$ then
\begin{eqnarray*}
\norm{\sum_{n=1}^N \alpha_n P_{n-1}(A,\lambda) - \sum_{n=1}^M \alpha_n P_{n-1}(A,\lambda)}&=&
\norm{\sum_{n=M+1}^N \alpha_n P_{n-1}(A,\lambda)}\\
&\leq&\sum_{n=M+1}^N |\alpha_n| \norm{P_{n-1}(A,\lambda)}\\
&\leq&\sum_{n=M+1}^N |\alpha_n| n r^{n-1},
\end{eqnarray*}
and as $\sum_{n=0}^\infty \alpha_n z^n$ has radius of convergence $>r$ this tends to $0$ as $M \to \infty$.
Therefore
\[
\sum_{n=1}^\infty \alpha_n P_{n-1}(A,\lambda) \in \mathfrak{A},
\]
and hence
\[
f(A)-f(\lambda)I = (A-\lambda I) \sum_{n=1}^\infty \alpha_n P_{n-1}(A,\lambda) \in \mathfrak{A}.
\]
If $\lambda \in \sigma(A)$ then $A-\lambda I \not \in \GL(\mathfrak{A})$, and as we have written $f(A)-f(\lambda)I$ as a product of $A-\lambda I$ and another element of $\mathfrak{A}$,
it follows that $f(A)-f(\lambda)I \not \in \GL(\mathfrak{A})$, and thus $f(\lambda) \in \sigma(f(A))$.
\end{proof}

\begin{theorem}
 If $\mathfrak{A}$ is a unital Banach algebra, $A \in \mathfrak{A}$, and $z_0 \in \mathbb{C}$, then
 \[
 \sigma(A-z_0I) = \sigma(A)-z_0.
 \]
 \label{translation}
 \end{theorem}
 \begin{proof}
Let $f(z)=z-z_0$. If $\lambda \in \sigma(A)$ then by Theorem \ref{mapping},
\[
\lambda - z_0=f(\lambda) \in \sigma(f(A))= \sigma(A-z_0 I),
\]
so
\[
\sigma(A)-z_0 \subseteq \sigma(A-z_0 I).
\]
Let $B=A-z_0 I$ and $g(z)=z+z_0$.
If $\lambda \in \sigma(B)$ then by Theorem \ref{mapping},
\[
\lambda + z_0=g(\lambda) \in \sigma(g(B)) = \sigma(A),
\]
so
\[
\sigma(B)+z_0 \subseteq \sigma(A),
\]
i.e.
\[
\sigma(A-z_0) \subseteq \sigma(A)-z_0.
\]
 \end{proof} 




\end{document}