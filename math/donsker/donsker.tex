\documentclass{article}
\usepackage{amsmath,amssymb,mathrsfs,amsthm}
%\usepackage{tikz-cd}
%\usepackage{hyperref}
\newcommand{\inner}[2]{\left\langle #1, #2 \right\rangle}
\newcommand{\tr}{\ensuremath\mathrm{tr}\,} 
\newcommand{\Span}{\ensuremath\mathrm{span}} 
\def\Re{\ensuremath{\mathrm{Re}}\,}
\def\Im{\ensuremath{\mathrm{Im}}\,}
\newcommand{\id}{\ensuremath\mathrm{id}} 
\newcommand{\var}{\ensuremath\mathrm{var}} 
\newcommand{\Lip}{\ensuremath\mathrm{Lip}} 
\newcommand{\GL}{\ensuremath\mathrm{GL}} 
\newcommand{\diam}{\ensuremath\mathrm{diam}} 
\newcommand{\sgn}{\ensuremath\mathrm{sgn}\,} 
\newcommand{\lcm}{\ensuremath\mathrm{lcm}} 
\newcommand{\supp}{\ensuremath\mathrm{supp}\,}
\newcommand{\dom}{\ensuremath\mathrm{dom}\,}
\newcommand{\upto}{\nearrow}
\newcommand{\downto}{\searrow}
\newcommand{\norm}[1]{\left\Vert #1 \right\Vert}
\newcommand{\HS}[1]{\left\Vert #1 \right\Vert_{\mathrm{HS}}}
\newtheorem{theorem}{Theorem}
\newtheorem{lemma}[theorem]{Lemma}
\newtheorem{proposition}[theorem]{Proposition}
\newtheorem{corollary}[theorem]{Corollary}
\theoremstyle{definition}
\newtheorem{definition}[theorem]{Definition}
\newtheorem{example}[theorem]{Example}
\begin{document}
\title{Wiener measure and Donsker's theorem}
\author{Jordan Bell\\ \texttt{jordan.bell@gmail.com}\\Department of Mathematics, University of Toronto}
\date{September 4, 2015}

\maketitle

\section{Relatively compact sets of Borel probability measures on {\em C[0,1]}}
Let $E=C[0,1]$, let $\mathscr{B}_E$ be the Borel $\sigma$-algebra of $E$, and let $\mathscr{P}_E$ be the collection of Borel probability measures on $E$. We assign $\mathscr{P}$ the \textbf{narrow topology}, the coarsest topology
on $\mathscr{P}_E$ such that for each $F \in C_b(E)$ the map $\mu \mapsto \int_E F d\mu$ is continuous.



For $f \in E$ and $\delta>0$ we define
\[
\omega_f(\delta) = \sup_{s,t \in [0,1], |s-t| \leq \delta} |f(s)-f(t)|.
\]
For $f \in E$, $\omega_f(\delta) \downarrow 0$ as $\delta \downarrow 0$, and for $\delta>0$,
$f \mapsto \omega_f(\delta)$ is continuous.
We shall use the following characterization of a relatively compact subset $A$ of $E$, which is proved using the Arzel\`a-Ascoli theorem.

\begin{lemma}
Let $A$ be a  subset of $E$. $\overline{A}$ is compact if and only if
\[
\sup_{f \in A} |f(0)| < \infty
\]
and
\[
\sup_{f \in A} \omega_f(\delta) \downarrow 0,\qquad \delta \downarrow 0.
\]
\label{modulus}
\end{lemma}

We shall use \textbf{Prokhorov's theorem}:\footnote{K. R.
Parthasarathy, {\em Probability Measures on Metric Spaces}, p.~47, Chapter II, Theorem 6.7.}
for $X$ a Polish space and for $\Gamma \subset \mathscr{P}_X$, 
$\overline{\Gamma}$ is compact if and only if for each $\epsilon>0$ there is a compact
subset $K_\epsilon$ of $X$ such that $\mu(K_\epsilon) \geq 1-\epsilon$ for all $\mu \in \Gamma$. 
Namely, a subset of $\mathscr{P}_X$ is relatively compact if and only if it is \textbf{tight}.
We use Prokhorov's theorem to   prove a characterization of relatively compact subsets of $\mathscr{P}_E$, which we then use 
to prove the characterization  in Theorem \ref{theorem22}.\footnote{K. R.
Parthasarathy, {\em Probability  Measures on Metric Spaces}, p.~213, Chapter VII, Lemma 2.2.}

\begin{lemma}
Let $\Gamma$ be a subset of $\mathscr{P}_E$. $\overline{\Gamma}$ is compact if and only if for each $\epsilon>0$ there is some $M_\epsilon<\infty$ and a function
$\delta \mapsto \omega_\epsilon(\delta)$ satisfying $\omega_\epsilon(\delta) \downarrow 0$ as $\delta \downarrow 0$ and such that for all $\mu \in \Gamma$,
\[
\mu(A_\epsilon) \geq 1-\frac{\epsilon}{2},\qquad
\mu(B_\epsilon) \geq 1-\frac{\epsilon}{2},
\]
where
\[
A_\epsilon = \{f \in E : |f(0)| \leq M_\epsilon\},\qquad B_\epsilon = \{f \in E: \textrm{$\omega_f(\delta) \leq \omega_\epsilon(\delta)$ for all $\delta>0$}\}.
\]
\label{compactlemma}
\end{lemma}
\begin{proof}
Suppose that $\Gamma$ satisfies the above conditions. Because $f \mapsto |f(0)|$ is continuous, 
$A_\epsilon$ is closed. For $\delta>0$, suppose that $f_n$ is a sequence in $B_\epsilon$ tending
to some $f \in E$. Because $g \mapsto \omega_g(\delta)$ is continuous, $\omega_{f_n}(\delta)
\to \omega_f(\delta)$, and because $\omega_{f_n}(\delta) \leq \omega_\epsilon(\delta)$ for each
$n$, we get $\omega_f(\delta) \leq \omega_\epsilon(\delta)$ and hence $f \in B_\epsilon$, showing that
$B_\epsilon$ is closed. Therefore $K_\epsilon = A_\epsilon \cap B_\epsilon$ is closed, i.e.
$K_\epsilon = \overline{K_\epsilon}$. The set $K_\epsilon$ satisfies
\[
\sup_{f \in K_\epsilon} |f(0)| \leq M_\epsilon
\]
and
\[
\limsup_{\delta \downarrow 0} \sup_{f \in K_\epsilon} \omega_f(\delta)
\leq \limsup_{\delta \downarrow 0} \omega_\epsilon(\delta)
= 0,
\]
thus by Lemma \ref{modulus}, $K_\epsilon$ is compact.
For $\mu \in \Gamma$,
\[
\mu(K_\epsilon) \geq 1-\frac{\epsilon}{2},
\]
and because $K_\epsilon$ is compact, this means that $\Gamma$ is tight, so by
Prokhorov's theorem, $\Gamma$ is relatively compact. 

Now suppose that $\Gamma$ is relatively compact and let $\epsilon>0$. By Prokhorov's theorem,
there is a compact set $K_\epsilon$ in $E$ such that $\mu(K_\epsilon) \geq 1-\frac{\epsilon}{2}$ for all
$\mu \in \Gamma$. 
Define
\[
M_\epsilon = \sup_{f \in K_\epsilon} |f(0)|,
\qquad \omega_\epsilon(\delta) = \sup_{f \in K_\epsilon} \omega_f(\delta),\qquad \delta>0.
\]
Because $K_\epsilon$ is compact, by Lemma \ref{modulus} we get that 
$M_\epsilon<\infty$ and $\omega_\epsilon(\delta) \downarrow 0$ as $\delta \downarrow 0$. 
For $\mu \in \Gamma$, 
\[
\mu(A_\epsilon) \geq \mu(K_\epsilon) \geq 1-\frac{\epsilon}{2},
\qquad \mu(B_\epsilon) \geq \mu(K_\epsilon) \geq 1-\frac{\epsilon}{2},
\]
showing that $\Gamma$ satisfies the conditions of the theorem.
\end{proof}



We now prove the characterization of relatively compact subsets of $\mathscr{P}_E$ that we shall use in our proof of Donsker's theorem.\footnote{K. R.
Parthasarathy, {\em Probability Measures on Metric Spaces}, p.~214, Chapter VII, Theorem 2.2.}

\begin{theorem}[Relatively compact sets in $\mathscr{P}$]
Let $\Gamma$ be a subset of $\mathscr{P}_E$. $\overline{\Gamma}$ is compact if and only if the following conditions are satisfied:
\begin{enumerate}
\item For each $\epsilon>0$ there is some $M_\epsilon<\infty$ such that 
\[
\mu(f:|f(0)| \leq M_\epsilon) \geq 1-\frac{\epsilon}{2},\qquad \mu \in \Gamma.
\]
\item For each $\epsilon>0$ and $\delta>0$ there is some $\eta = \eta(\epsilon,\delta)>0$ such that 
\[
\mu(f : \omega_f(\eta) \leq \delta) \geq 1-\frac{\epsilon}{2},\qquad \mu \in \Gamma.
\]
\end{enumerate}
\label{theorem22}
\end{theorem}
\begin{proof}
Suppose that $\overline{\Gamma}$ is compact and let $\epsilon>0$.
By Lemma \ref{compactlemma}, there is some $M_\epsilon<\infty$ and a function $\eta \mapsto \omega_\epsilon(\eta)$
satisfying $\omega_\epsilon(\eta) \downarrow 0$ as $\eta \downarrow 0$ and
\[
\mu(A_\epsilon) \geq 1-\frac{\epsilon}{2},\qquad \mu(B_\epsilon) \geq 1-\frac{\epsilon}{2},
\qquad \mu \in \Gamma.
\]
For $\delta>0$, there is some $\eta = \eta(\epsilon,\delta)$ with $\omega_\epsilon(\eta) \leq \delta$. Then for
$\mu \in \Gamma$,
\[
\mu(f : \omega_f(\eta) \leq \delta) \geq 
\mu(f : \omega_f(\eta) \leq \omega_\epsilon(\eta))
\geq \mu(B_\epsilon)
\geq 1-\frac{\epsilon}{2}.
\]

Now suppose that the conditions of the theorem hold. For  each $\epsilon>0$ and $n \geq 1$ there is some
$\eta_{\epsilon,n}>0$ such that 
\[
\mu(F_{\epsilon,n}) \geq 1-\frac{\epsilon}{2^{n+1}},\qquad \mu \in \Gamma,
\]
where
\[
F_{\epsilon,n} = \left\{f : \omega_f(\eta_{\epsilon,n}) \leq \frac{1}{n}\right\}.
\]
Let
\[
K_\epsilon = \{f : |f(0)| \leq M_\epsilon\} \cap \bigcap_{n=1}^\infty F_{\epsilon,n},
\]
for which
\[
\mu(K_\epsilon) \geq \mu(f : |f(0)| \leq M_\epsilon) \geq 1-\frac{\epsilon}{2},\qquad \mu \in \Gamma.
\]
For $f \in K_\epsilon$, then for each $n \geq 1$ we have $f \in F_{\epsilon,n}$, which means that
$\omega_f(\eta_{\epsilon,n}) \leq \frac{1}{n}$, and therefore
\[
\sup_{f \in K_\epsilon} \omega_f(\eta_{\epsilon,n}) \leq \frac{1}{n}.
\]
Thus for $n \geq 1$, if $0<\eta \leq \eta_{\epsilon,n}$ then
\[
\sup_{f \in K_\epsilon} \omega_f(\eta) \leq \frac{1}{n},
\]
which shows $\sup_{f \in K_\epsilon} \omega_f(\eta) \downarrow 0$ as $\eta \downarrow 0$. 
Then because
\[
\sup_{f \in K_\epsilon} |f(0)| \leq M_\epsilon,
\]
applying Lemma \ref{modulus} we get that $\overline{K_\epsilon}$ is compact. The map
$f \mapsto \omega_f(\eta_{\epsilon,n})$ is continuous, so the set $F_{\epsilon,n}$ is closed, and therefore
the set $K_\epsilon$ is closed.  
Because $K_\epsilon$ is compact and $\mu(K_\epsilon) \geq 1-\frac{\epsilon}{2}$ for all
$\mu \in \Gamma$, it follows from
by Prokhorov's theorem that  $\Gamma$ is relatively compact. 
\end{proof}



\section{Wiener measure}
For $t_1,\ldots,t_d \in [0,1]$, $t_1<\cdots<t_d$, 
define $\pi_{t_1,\ldots,t_d}:E \to \mathbb{R}^d$ by 
\[
\pi_{t_1,\ldots,t_d}(f) = (f(t_1),\ldots,f(t_d)),\qquad f \in E,
\]
which is continuous. We state the following results, which we will use later.\footnote{K. R.
Parthasarathy, {\em Probability Measures on Metric Spaces}, p.~212, Chapter VII, Theorem 2.1.}


\begin{theorem}[The Borel $\sigma$-algebra of $E$]
$\mathscr{B}_E$ is equal to the $\sigma$-algebra generated by $\{\pi_t: t \in [0,1]\}$.

Two elements $\mu$ and $\nu$ of $\mathscr{P}_E$ are equal if and only if for any $d$ and any $t_1<\cdots<t_d$, the pushforward measures
\[
\mu_{t_1,\ldots,t_d} = (\pi_{t_1,\ldots,t_d})_* \mu,\qquad
\nu_{t_1,\ldots,t_d} =  (\pi_{t_1,\ldots,t_d})_* \nu
\]
are equal.
\label{theorem21}
\end{theorem}



Let $(\xi_t)_{t \in [0,1]}$ be a stochastic process with state space $\mathbb{R}$ and sample space
$(\Omega,\mathscr{F},P)$. 
For $t_1<\cdots<t_d$,
let $\xi_{t_1,\ldots,t_d} = \xi_{t_1} \otimes \cdots \otimes \xi_{t_d}$ and
let $P_{t_1,\ldots,t_d} = (\xi_{t_1,\ldots,t_d})_*P$: for $B \in \mathscr{B}_{\mathbb{R}}^d$,  
\[
P_{t_1,\ldots,t_d}(B) = ((\xi_{t_1,\ldots,t_d})_*P)(B) = P(\xi_{t_1,\ldots,t_d}^{-1}(B))
=P((\xi_{t_1},\ldots,\xi_{t_d}) \in B).
\]
$P_{t_1,\ldots,t_d}$ is a Borel probability measure on $\mathbb{R}^d$ and is called a \textbf{finite-dimensional distribution
of the stochastic process}.

The \textbf{Kolmogorov continuity theorem}\footnote{K. R.
Parthasarathy, {\em Probability Measures on Metric Spaces}, p.~216, Chapter VII, Theorem 3.1}
tells us that if there are $\alpha,\beta,K>0$ 
such that  for all $s,t \in [0,1]$,
\[
E |\xi_t - \xi_s|^\alpha \leq K |t-s|^{1+\beta},
\]
then there is a unique $\mu \in \mathscr{P}_E$ such that for all $k$ and for all $t_1<\cdots<t_d$,
\[
\mu_{t_1,\ldots,t_d} = P_{t_1,\ldots,t_d}.
\]

We now define and prove the existence of \textbf{Wiener measure}.\footnote{K. R.
Parthasarathy, {\em Probability Measures on Metric Spaces}, p.~218, Chapter VII, Theorem 3.2.}

\begin{theorem}[Wiener measure]
There is a unique  Borel probability measure $W$ on $E$ satisfying:
\begin{enumerate}
\item $W(f \in E: f(0)=0) = 1$.
\item For $0 \leq t_0 < t_1<\cdots<t_d \leq 1$ the random variables 
\[
\pi_{t_1}-\pi_{t_0}, \quad \pi_{t_2}-\pi_{t_1}, \quad \pi_{t_3}-\pi_{t_2},\quad
\pi_{t_d}-\pi_{t_{d-1}}
\]
are independent $(E,\mathscr{B}_E,W) \to (\mathbb{R},\mathscr{B}_{\mathbb{R}})$.
\item If $0 \leq s < t \leq 1$, the random variable $\pi_t-\pi_s:(E,\mathscr{B}_E,W) \to (\mathbb{R},\mathscr{B}_{\mathbb{R}})$ is normal with mean $0$ and variance $t-s$.
\end{enumerate} 
\end{theorem}
\begin{proof}
There is a stochastic process $(\xi_t)_{t \in [0,1]}$ with state space $\mathbb{R}$ and some sample space $(\Omega,\mathscr{F},P)$, 
such that (i) $P(\xi_0=0)=1$, (ii)  $(\xi_t)_{t \in [0,1]}$ has independent increments, and (iii) for $s < t$, $\xi_t-\xi_s$ is a normal
random variable with mean $0$ and variance $t-s$.
(Namely,
\textbf{Brownian motion with starting point $0$}.)
Because $\xi_t-\xi_s$ has mean $0$ and variance $t-s$, we calculate (cf. Isserlis's theorem)
\[
E|\xi_t-\xi_s|^4 = 3|t-s|^2.
\]
Thus using the Kolmogorov continuity theorem with $\alpha=4$, $\beta=1$, $K=3$, 
there is a unique $W \in \mathscr{P}_E$ such that for all $t_1<\cdots<t_d$,
\[
W_{t_1,\ldots,t_d} = P_{t_1,\ldots,t_d},
\]
i.e. for $B \in \mathscr{B}_{\mathbb{R}}^d$,
\[
W(\pi_{t_1} \otimes \cdots \otimes \pi_{t_d} \in B)
= P(\xi_{t_1} \otimes  \cdots \otimes \xi_{t_d} \in B).
\]

For $t_1<\cdots<t_d$ and $B \in \mathscr{B}_{\mathbb{R}}^d$, with $T:\mathbb{R}^d \to \mathbb{R}^d$
defined by $T(x_1,\ldots,x_d) = (x_1,x_2-x_1,\ldots,x_d-x_{d-1})$,
\[
\begin{split}
&W( \pi_{t_1} \otimes (\pi_{t_2}-\pi_{t_1}) \otimes \cdots \otimes (\pi_{t_d}-\pi_{t_{d-1}}) \in B)\\
=&W(T \circ (\pi_{t_1} \otimes \pi_{t_2} \otimes \cdots \otimes \pi_{t_d}) \in B)\\
=&W(\pi_{t_1} \otimes \pi_{t_2} \otimes \cdots \otimes \pi_{t_d} \in T^{-1}(B))\\
=&P(\xi_{t_1} \otimes \xi_{t_2} \otimes \cdots \otimes \xi_{t_d} \in T^{-1}(B))\\
=&P(T \circ (\xi_{t_1} \otimes \xi_{t_2} \otimes \cdots \otimes \xi_{t_d}) \in B)\\
=&P(\xi_{t_1} \otimes (\xi_{t_2}-\xi_{t_1}) \otimes \cdots \otimes (\xi_{t_d}-\xi_{t_{d-1}}) \in B).
\end{split}
\]
Hence, because $\xi_{t_1},\xi_{t_2}-\xi_{t_1},\ldots,\xi_{t_d}-\xi_{t_{d-1}}$ are independent,
\[
\begin{split}
&(\pi_{t_1} \otimes (\pi_{t_2}-\pi_{t_1}) \otimes \cdots \otimes (\pi_{t_d}-\pi_{t_{d-1}}))_* W\\
=&(\xi_{t_1} \otimes (\xi_{t_2}-\xi_{t_1}) \otimes \cdots \otimes (\xi_{t_d}-\xi_{t_{d-1}}))_*P\\
=&(\xi_{t_1})_*P \otimes (\xi_{t_2}-\xi_{t_1})_*P \otimes \cdots \otimes (\xi_{t_d}-\xi_{t_{d-1}})_*P\\
=&(\pi_{t_1})_*W \otimes (\pi_{t_2}-\pi_{t_1})_*W \otimes \cdots \otimes (\pi_{t_d}-\pi_{t_{d-1}})_*W,
\end{split}
\]
which means that the random variables $\pi_{t_1},\pi_{t_2}-\pi_{t_1},\ldots,\pi_{t_d}-\pi_{t_{d-1}}$ are independent. 

If $s < t$ and $B_1,B_2 \in \mathscr{B}_{\mathbb{R}}$, and for $T:\mathbb{R}^2 \to \mathbb{R}^2$ defined by
$T(x,y)=(x,y-x)$,
\begin{align*}
W( (\pi_s, \pi_t-\pi_s) \in (B_1,B_2))&=W(T \circ (\pi_s,\pi_t) \in (B_1,B_2))\\
&=P((\xi_s,\xi_t) \in T^{-1}(B_1,B_2))\\
&=P((\xi_s,\xi_t-\xi_s) \in (B_1,B_2)),
\end{align*}
which implies that $(\pi_t-\pi_s)_*W = (\xi_t-\xi_s)_*P$, and because
$\xi_t-\xi_s$ is a normal random variable with mean $0$ and variance $t-s$, so is $\pi_t-\pi_s$. 

Finally, 
\[
W(f : f(0) = 0)=W(\pi_0 = 0)
=P(\xi_0 = 0)=1.
\]
\end{proof}

$(E,\mathscr{B}_E,W)$ is a probability space, and the stochastic process
$(\pi_t)_{t \in [0,1]}$ is a Brownian motion. 





\section{Interpolation and continuous stochastic processes}
Let $(\xi_t)_{t \in [0,1]}$ be a \textbf{continuous stochastic process} with state space $\mathbb{R}$ and sample space
$(\Omega,\mathscr{F},P)$. To say that the stochastic process is continuous means that
for each $\omega \in \Omega$ the map $t \mapsto \xi_t(\omega)$ is continuous $[0,1] \to \mathbb{R}$. 
Define $\xi:\Omega \to E$ by
\[
\xi(\omega) = (t \mapsto \xi_t(\omega)),\qquad \omega \in \Omega.
\]
For $t \in [0,1]$ and $B$ a Borel set in $\mathbb{R}$,
\[
\xi^{-1} \pi_t^{-1} B = \{\omega \in \Omega : \xi_t(\omega) \in B\} = \xi_t^{-1} B,
\]
and because $\xi_t:(\Omega,\mathscr{F}) \to (\mathbb{R},\mathscr{B}_{\mathbb{R}})$ is measurable this belongs to $\mathscr{F}$. 
But by Theorem \ref{theorem21}, $\mathscr{B}_E$ is generated by the collection $\{\pi_t^{-1} B: t \in [0,1], B \in \mathscr{B}_{\mathbb{R}}\}$. 
Now, for $f:X \to Y$ and for a nonempty collection $\mathscr{F}$ of subsets of $Y$,\footnote{Charalambos D. Aliprantis
and Kim C. Border, {\em Infinite Dimensional Analysis: A Hitchhiker's Guide}, third ed., p.~140, Lemma 4.23.}
\[
\sigma(f^{-1}(\mathscr{F})) = f^{-1}(\sigma(\mathscr{F})).
\]
Therefore $\xi^{-1}(\mathscr{B}_E) \subset \mathscr{F}$, which means that
$\xi:(\Omega,\mathscr{F}) \to (E,\mathscr{B}_E)$ is measurable. This means that a continuous stochastic proess with index set $[0,1]$ induces a random variable with state space
$E$.
Then the pushforward measure of $P$ by $\xi$ is a Borel probability measure on $E$. 
We shall end up constructing a sequence of pushforward measures from a sequence of continuous stochastic processes,
that converge in $\mathscr{P}_E$ to Wiener measure $W$. 


Let $(X_n)_{n \geq 1}$ be a sequence of independent identically distributed random variables on a sample space
$(\Omega,\mathscr{F},P)$ with $E(X_n)=0$ and $V(X_n)=1$, and let $S_0=0$ and
\[
S_k = \sum_{i=1}^k X_i.
\]
Then $E(S_k)=0$ and $V(S_k) = k$. 
For $t \geq 0$ let 
\[
Y_t = S_{[t]} + (t-[t]) X_{[t]+1}.
\]
Thus, for $k \geq 0$ and $k \leq t \leq k+1$,
\begin{align*}
Y_t &= S_k + (t-k) X_{k+1}\\
& = S_k + (t-k) (S_{k+1}-S_k)\\
& = (1-t+k)S_k +(t-k)S_{k+1}.
\end{align*}
For each $\omega \in \Omega$, the map $t \mapsto Y_t(\omega)$ is piecewise linear, equal to $S_k(\omega)$ when $t=k$, and in particular
it is continuous. 
For $n \geq 1$, define
\begin{equation}
X_t^{(n)} = n^{-1/2} Y_{nt} = n^{-1/2} S_{[nt]} + n^{-1/2}(nt-[nt])X_{[nt]+1}, \qquad t \in [0,1].
\label{Xtn}
\end{equation}
For $0 \leq k \leq n$,
\[
X_{k/n}^{(n)} = n^{-1/2} S_k.
\]
For each $n \geq 1$, $(X^{(n)}_t)_{t \in [0,1]}$ is a continuous stochastic process on the sample space $(\Omega,\mathscr{F},P)$, and we denote
by $P_n \in \mathscr{P}_E$ the pushforward measure of $P$ by $X^{(n)}$. 




\section{Donsker's theorem}
\begin{lemma}
If $Z_n$ and $U_n$ are random variables with state space $\mathbb{R}^d$ such that
$Z_n \to Z$ in distribution and $U_n \to 0$ in distribution, then $Z_n + U_n \to 0$ in distribution.

If $Z_n$ are random variables  with state space $\mathbb{R}$ that converge
in distribution to some random variable $Z$ and $c_n$ are real numbers that converge to some real number $c$, then
$c_n Z_n \to cZ$ in distribution. 
\label{191}
\end{lemma}

For $\sigma \geq 0$, let $\nu_{\sigma^2}$ be the Gaussian measure on $\mathbb{R}$ with mean $0$ and
variance $\sigma^2$.
The \textbf{characteristic function} of $\nu_{\sigma^2}$ is, for $\sigma > 0$,
\[
\widetilde{\nu}_{\sigma^2}(\xi) = \int_{\mathbb{R}} e^{i\xi x} d\nu_{\sigma^2}(x) = 
\int_{\mathbb{R}} e^{i\xi x} \frac{1}{\sigma \sqrt{2\pi}} e^{-\frac{x^2}{2\sigma^2}} dx
=e^{-\frac{1}{2}\sigma^2 \xi^2},
\]
and $\widetilde{\nu}_0(\xi) = 1$. One checks that $c_* \nu_1 = \nu_{c^2}$ for $c \geq 0$. 

In following theorem and in what follows, $X^{(n)}$ is the
piecewise linear stochastic process defined in \eqref{Xtn}. We prove that a sequence of finite-dimensional distributions
converge to a Gaussian measure.\footnote{Bert Fristedt and Lawrence Gray,
{\em A Modern Approach to Probability Theory}, 
p.~368, \S 19.1, Lemma 1.} 

\begin{theorem}
For $0 \leq t_0 < t_1 < t_1 < \cdots < t_d \leq 1$, the random vectors 
\[
(X^{(n)}_{t_1}-X^{(n)}_{t_0},\ldots,X^{(n)}_{t_d}-X^{(n)}_{t_{d-1}}),
\qquad (\Omega,\mathscr{F},P) \to
(\mathbb{R}^d,\mathscr{B}_{\mathbb{R}}^d),
\]
converge in distribution to $\nu_{t_1-t_0} \otimes \cdots \otimes \nu_{t_d-t_{d-1}}$ as $n \to \infty$. 
\label{jointtheorem}
\end{theorem}
\begin{proof}
For $0<j \leq d$ and $n \geq 1$ let
\[
r_{j,n} = \frac{[nt_j]}{n},\qquad U_{j,n} = X^{(n)}_{t_j} - X^{(n)}_{r_{j,n}},
\]
and for $0 \leq j < d$ and $n \geq 1$ let
\[
s_{j,n} = \frac{\lceil nt_j \rceil}{n},\qquad V_{j,n} = X^{(n)}_{s_{j,n}} - X^{(n)}_{t_j},
\]
with which
\begin{align*}
(X^{(n)}_{t_1} - X^{(n)}_{t_{0}},
\ldots,X^{(n)}_{t_d} - X^{(n)}_{t_{d-1}})&=(X^{(n)}_{r_{1,n}}-X^{(n)}_{s_{0,n}},
\ldots,X^{(n)}_{r_{d,n}}-X^{(n)}_{s_{d-1,n}})\\
&+(U_{1,n},\ldots,U_{d,n})+(V_{0,n},\ldots,V_{d-1,n}).
\end{align*}
Because $E(X^{(n)}_t)=0$, 
\[
E(U_{j,n}) = 0,\qquad E(V_{j,n})=0.
\]
Furthermore,
\[
\begin{split}
&V(U_{j,n})\\
=&V(X^{(n)}_{t_j} - X^{(n)}_{r_{j,n}})\\
=&n^{-1} V(S_{[nt_j]} +(nt_j-[nt_j]) X_{[nt_j]+1} - S_{[nr_{j,n}]} - (nr_{j,n}-[nr_{j,n}]) X_{[nr_{j,n}]+1})\\
=&n^{-1} V(S_{[nt_j]} +(nt_j-[nt_j]) X_{[nt_j]+1} - S_{[nt_j]} - ([nt_j]-[nt_j]) X_{[nr_{j,n}]+1})\\
=&n^{-1} (nt_j-[nt_j])^2 V(X_{[nt_j]+1})\\
=&n^{-1} (nt_j-[nt_j])^2,
\end{split}
\]
and because $0 \leq nt_j-[nt_j] < 1$ this tends to $0$ as $n \to \infty$. Likewise,
$V(V_{j,n}) \to 0$ as $n \to \infty$. 


For $1 \leq j \leq d$, 
\begin{align*}
X^{(n)}_{r_{j,n}} - X^{(n)}_{s_{j-1,n}} &= n^{-1/2} S_{[nr_{j,n}]} + n^{-1/2} (nr_{j,n} - [nr_{j,n}]) X_{[nr_{j,n}]+1}\\
&-n^{-1/2} S_{[ns_{j-1,n}]} - n^{-1/2} (ns_{j-1,n} - [ns_{j-1,n}]) X_{[ns_{j-1,n}]+1}\\
&=n^{-1/2} S_{[nt_j]} - n^{-1/2} S_{\lceil nt_{j-1} \rceil}\\
&=n^{-1/2} \frac{([nt_j] - \lceil nt_{j-1} \rceil - 1)^{1/2}}{([nt_j] - \lceil nt_{j-1} \rceil - 1)^{1/2}} \sum_{i=\lceil nt_{j-1} \rceil+1}^{[nt_j]} X_i.
\end{align*}
By the central limit theorem,
\[
([nt_j] - \lceil nt_{j-1} \rceil - 1)^{1/2}  \sum_{i=\lceil nt_{j-1} \rceil+1}^{[nt_j]} X_i \to \nu_1
\]
in distribution as $n \to \infty$.
But
\[
n^{-1/2} ([nt_j] - \lceil nt_{j-1} \rceil - 1)^{1/2} \to (t_j-t_{j-1})^{1/2}
\]
as $n \to \infty$, and $(t_j-t_{j-1})^{1/2}_* \nu_1 = \nu_{t_j-t_{j-1}}$, 
so by Lemma \ref{191},
\[
X^{(n)}_{r_{j,n}} - X^{(n)}_{s_{j-1,n}}  \to  \nu_{t_j-t_{j-1}}
\]
in distribution as $n \to \infty$. 


For sufficiently large $n$, depending on $t_0,\ldots,t_d$, 
\[
t_0 \leq s_{0,n} < r_{1,n} \leq t_1 \leq s_{1,n} < r_{2,n} \leq \cdots \leq t_{d-1} \leq s_{d-1,n}
<r_{d,n} \leq t_d.
\]
Check that $(U_{1,n},\ldots,U_{d,n}) \to 0$ in probability and that $(V_{0,n},\ldots,V_{d-1,n}) \to 0$ in probability, and hence
these random vectors converge  to $0$ in distribution as $n \to \infty$. 
The random variables $X^{(n)}_{r_{1,n}}-X^{(n)}_{s_{0,n}},
\ldots,X^{(n)}_{r_{d,n}}-X^{(n)}_{s_{d-1,n}}$ are independent,  and therefore their joint distribution is equal to the product of their
distributions. 
Now, if $\mu_n = \mu_n^1 \otimes \cdots \otimes \mu_n^d$ and $\mu_n^j \to \mu^j$ as $n \to \infty$, $1 \leq j \leq d$, then for $\xi \in \mathbb{R}^d$,
\begin{align*}
\widetilde{\mu}_n(\xi)&=\widetilde{\mu}_n^1(\xi_1) \cdots \widetilde{\mu}_n^d(\xi_d)\\
&\to \widetilde{\mu}^1(\xi_1) \cdots \widetilde{\mu}^d(\xi_d)\\
&=(\mu^1 \otimes \cdots \otimes \mu^d)^{\widetilde{\;}}(\xi)
\end{align*}
as $n \to \infty$, and therefore by \textbf{L\'evy's continuity theorem},
$\mu_n \to \mu^1 \otimes \cdots \otimes \mu^d$ as
$n \to \infty$.
This means that the joint distribution of $X^{(n)}_{r_{1,n}}-X^{(n)}_{s_{0,n}},
\ldots,X^{(n)}_{r_{d,n}}-X^{(n)}_{s_{d-1,n}}$ converges  to
\[
\nu_{t_1-t_0} \otimes \cdots \otimes \nu_{t_d-t_{d-1}}
\]
as $n \to \infty$. 
Because $(U_{1,n},\ldots,U_{d,n}) \to 0$ in distribution as $n \to \infty$ and $(V_{0,n},\ldots,V_{d-1,n}) \to 0$ in distribution as $n \to \infty$,
applying Lemma \ref{191} we get that 
\[
(X^{(n)}_{t_1} - X^{(n)}_{t_{0}},
\ldots,X^{(n)}_{t_d} - X^{(n)}_{t_{d-1}}) \to \nu_{t_1-t_0} \otimes \cdots \otimes \nu_{t_d-t_{d-1}}
\]
in distribution as $n \to \infty$, completing the proof.
\end{proof}

Let $t_0=0$ and let $0<t_1<\cdots<t_d \leq 1$. As $X^{(n)}_0=0$,
the above lemma tells us that
\[
(X^{(n)}_{t_1},X^{(n)}_{t_2}-X^{(n)}_{t_1},\ldots,X^{(n)}_{t_d}-X^{(n)}_{t_{d-1}}) \to \nu_{t_1} \otimes \nu_{t_2-t_1} \otimes \cdots \otimes \nu_{t_d-t_{d-1}}
\]
in distribution as $n \to \infty$. 
Define $g:\mathbb{R}^d \to \mathbb{R}^d$ by
\[
g(x_1,x_2,\ldots,x_d) = (x_1,x_1+x_2,\ldots,x_1+x_2+\cdots+x_d).
\]
The function $g$  is continuous and satisfies
\[
g \circ (X^{(n)}_{t_1} - X^{(n)}_{t_{0}},
\ldots,X^{(n)}_{t_d} - X^{(n)}_{t_{d-1}})
=(X^{(n)}_{t_1},X^{(n)}_{t_2},\ldots,X^{(n)}_{t_d}).
\]
Then by the \textbf{continuous mapping theorem},
\begin{equation}
(X^{(n)}_{t_1},X^{(n)}_{t_2},\ldots,X^{(n)}_{t_d}) \to g_* (\nu_{t_1} \otimes \nu_{t_2-t_1} \otimes \cdots \otimes \nu_{t_d-t_{d-1}})
\label{continuousmapping}
\end{equation}
in distribution as $n \to \infty$.\footnote{Allan Gut, {\em Probability: A Graduate Course}, second ed., p.~245, Chapter 5, Theorem 10.4.}




We prove a result that we use to prove the next lemma, and that lemma is used in the proof of Donsker's theorem.\footnote{Ioannis Karatzas and Steven E. Shreve,
{\em Brownian Motion and Stochastic Calculus}, second ed., p.~68, Lemma 4.18.}

\begin{lemma}
For $\epsilon>0$,
\[
\lim_{\delta \downarrow 0} \limsup_{n \to \infty} \frac{1}{\delta} P\left(\max_{1 \leq j \leq [n\delta]+1} |S_j| > \epsilon n^{1/2} \right)=0.
\]
\label{lemma418}
\end{lemma}
\begin{proof}
For each $\delta>0$, by the central limit theorem,
\[
([n\delta]+1)^{-1/2} S_{[n\delta]+1} \to Z
\]
in distribution as $n \to \infty$, where $Z_*P=\nu_1$.  Because $\frac{([n\delta]+1)^{1/2}}{(n\delta)^{1/2}} \to 1$ as $n \to \infty$, by Lemma \ref{191} we then
get that
\[
(n\delta)^{-1/2} S_{[n\delta]+1} \to Z
\]
in distribution as $n \to \infty$.
Now let $\lambda>0$, and there is a sequence $\phi_k$ in $C_b(\mathbb{R})$ such that
$\phi_k \downarrow 1_{(-\infty,-\lambda] \cup [\lambda,\infty)}=\chi_\lambda$ pointwise as $k \to \infty$. 
For each $k$, writing $X=S_{[n\delta]+1}$, using the change of variables formula,
\begin{align*}
P(|X| \geq \lambda (n\delta)^{1/2})
&=
\int_\Omega 
 \chi_{\lambda (n\delta)^{1/2}}(X(\omega)) dP(\omega)\\
 & = 
\int_\Omega \chi_{\lambda}((n\delta)^{-1/2} X(\omega)) dP(\omega)\\
&\leq \int_\Omega \phi_k((n\delta)^{-1/2} X(\omega)) dP(\omega)\\
&=E(\phi_k((n\delta)^{-1/2}X)).
\end{align*}
Therefore, by the continuous mapping theorem,
\begin{align*}
\limsup_{n \to \infty} P(|S_{[n\delta]+1}| \geq \lambda (n\delta)^{1/2})&\leq \lim_{n \to \infty} E(\phi_k((n\delta)^{-1/2}S_{[n\delta]+1}))\\
&=E(\phi_k \circ Z).
\end{align*}
Because $\phi_k \downarrow \chi_\lambda$ pointwise as $k \to \infty$, using the monotone convergence theorem and then using Chebyshev's inequality,
\[
E(\phi_k \circ Z) \to E(\chi_\lambda \circ Z) =  P(|Z| \geq \lambda) \leq \lambda^{-3} E|Z|^3.
\]
We have established that for each $\lambda>0$,
\begin{equation}
\limsup_{n \to \infty} P(|S_{[n\delta]+1}| \geq \lambda (n\delta)^{1/2}) \leq  \lambda^{-3} E|Z|^3.
\label{412}
\end{equation}

Define
\[
\tau = \min\{j \geq 1: |S_j| > n^{1/2} \epsilon\}.
\]
For $0<\delta<\epsilon^2/2$, it is a fact that
\[
\begin{split}
&P\left( \max_{0 \leq j \leq [n\delta]+1} |S_j| > n^{1/2}\epsilon \right)\\
\leq & P(|S_{[n\delta]+1}| \geq n^{1/2}  (\epsilon - (2\delta)^{1/2}))\\
&+\sum_{j=1}^{[n\delta]} P(|S_{[n\delta]+1}| < n^{1/2}(\epsilon - (2\delta)^{1/2}) | \tau = j) P(\tau = j).
\end{split}
\]
If $\tau(\omega) = j$ and $|S_{[n\delta]+1}(\omega)| < n^{1/2}(\epsilon-(2\delta)^{1/2})$ then
\[
|S_j(\omega) - S_{[n\delta]+1}(\omega)| \geq |S_j(\omega)| - |S_{[n\delta]+1}(\omega)|
>n^{1/2} \epsilon  - n^{1/2}(\epsilon-(2\delta)^{1/2})
=(2n\delta)^{1/2}.
\]
But by Chebyshev's inequality  and the fact that the random variables $X_1,X_2,\ldots$ are independent with mean $0$ and variance $1$,
\[
P( |S_j  -  S_{[n\delta]+1}| > (2n\delta)^{1/2}) \leq \frac{1}{2n\delta} E( (S_j  -  S_{[n\delta]+1})^2)
= \frac{1}{2n\delta} ([n\delta]-j) \leq \frac{1}{2},
\]
so
\[
P(|S_{[n\delta]+1}(\omega)| < n^{1/2}(\epsilon-(2\delta)^{1/2}) | \tau = j) \leq \frac{1}{2}.
\]
Therefore,
\[
\begin{split}
&P\left( \max_{0 \leq j \leq [n\delta]+1} |S_j| > n^{1/2}\epsilon \right)\\
\leq & P(|S_{[n\delta]+1}| \geq n^{1/2}  (\epsilon - (2\delta)^{1/2})) + \sum_{j=1}^{[n\delta]} \frac{1}{2} \cdot P(\tau = j)\\
=&P(|S_{[n\delta]+1}| \geq n^{1/2}  (\epsilon - (2\delta)^{1/2}))  + \frac{1}{2} P(\tau \leq [n\delta])\\
=&P(|S_{[n\delta]+1}| \geq n^{1/2}  (\epsilon - (2\delta)^{1/2}))  + \frac{1}{2} P\left(\max_{0 \leq j \leq [n\delta]+1} |S_j| > n^{1/2} \epsilon \right),
\end{split}
\]
so
\[
P\left( \max_{0 \leq j \leq [n\delta]+1} |S_j| > n^{1/2}\epsilon \right) \leq 2 P(|S_{[n\delta]+1}| \geq n^{1/2}  (\epsilon - (2\delta)^{1/2})).
\]
Now using \eqref{412} with $\lambda =  (\epsilon - (2\delta)^{1/2}) \delta^{-1/2}$, 
\[
\limsup_{n \to \infty} P(|S_{[n\delta]+1}| \geq  (\epsilon - (2\delta)^{1/2}) \delta^{-1/2} (n\delta)^{1/2}) \leq   (\epsilon - (2\delta)^{1/2})^{-3} \delta^{3/2} E|Z|^3,
\]
hence
\[
\limsup_{n \to \infty} P\left( \max_{0 \leq j \leq [n\delta]+1} |S_j| > n^{1/2}\epsilon \right)
\leq 2 (\epsilon - (2\delta)^{1/2})^{-3} \delta^{3/2} E|Z|^3.
\]
Dividing both sides by $\delta$ and then taking $\delta \downarrow 0$ we obtain the claim.
\end{proof}




We prove one more result that we use to prove Donsker's theorem.\footnote{Ioannis Karatzas and Steven E. Shreve,
{\em Brownian Motion and Stochastic Calculus}, second ed., p.~69, Lemma 4.19.}

\begin{lemma}
For $T>0$ and $\epsilon>0$,
\[
\lim_{\delta \downarrow 0} \limsup_{n \to \infty} P\left( \max_{0 \leq k \leq [nT]+1}
\max_{1 \leq j \leq [n\delta]+1} |S_{j+k}-S_k| > n^{1/2} \epsilon \right) = 0.
\]
\label{lemma419}
\end{lemma}
\begin{proof}
For $0<\delta \leq T$, let $m = \lceil T/\delta \rceil$, so $T/m < \delta \leq T/(m-1)$. 
Then
\[
\lim_{n \to \infty} \frac{[nT]+1}{[n\delta]+1} = \frac{T}{\delta}<m,
\]
so for all $n \geq n_\delta$ it is the case that $[nT]+1 < ([n\delta]+1)m$. 
Suppose that $\omega \in \Omega$ is such that there are
$1 \leq j \leq [n\delta]+1$
and $0 \leq k \leq [nT]+1$ satisfying
\[
|S_{j+k}(\omega) - S_k(\omega)| > n^{1/2} \epsilon,
\]
 and  then let  $p=[k/([n\delta]+1)]$,
which satisfies
$0 \leq p \leq m-1$ and
\[
([n\delta]+1)p \leq k < ([n\delta]+1)(p+1).
\]
Because $1 \leq j \leq [n\delta]+1$, either
\[
([n\delta]+1)p < k+j \leq ([n\delta]+1)(p+1)
\]
or
\[
([n\delta]+1)(p+1) < k+j < ([n\delta]+1)(p+2).
\]
We separate the first case into the cases
\[
|S_k(\omega)-S_{([n\delta]+1)p}(\omega)| > \frac{1}{2}n^{1/2} \epsilon
\]
and
 \[
 |S_{j+k}(\omega) - S_{([n\delta]+1)p}(\omega)| > \frac{1}{2}n^{1/2}\epsilon,
 \]
and we separate the second case into the cases 
\[
|S_k-S_{([n\delta]+1)p}(\omega)| > \frac{1}{3}n^{1/2} \epsilon,
\]
and
\[ 
|S_{([n\delta]+1)p}(\omega) - S_{([n\delta]+1)(p+1)}(\omega)| > \frac{1}{3}n^{1/2} \epsilon,
\]
 and
\[
|S_{([n\delta]+1)(p+1)}(\omega) - S_{([n+\delta]+1)(p+2)}(\omega)| > \frac{1}{3}n^{1/2} \epsilon.
\] 
It follows that\footnote{This should be worked out more carefully. In Karatzas and Shreve, there is $m+1$ where I have $m$.}
\[
\begin{split}
&\left\{ \max_{1 \leq j \leq [n\delta]+1} \max_{0 \leq k \leq [nT]+1} |S_{j+k}-S_k| > n^{1/2} \epsilon\right\}\\
\subset&\bigcup_{p=0}^{m-1} \left\{ \max_{1 \leq j \leq [n\delta]+1} |S_{j+([n\delta]+1)p}-S_{([n\delta]+1)p}| > \frac{1}{3}n^{1/2}\epsilon\right\}.
\end{split}
\]
For $0 \leq p \leq m-1$,
\[
\begin{split}
&P\left( \max_{1 \leq j \leq [n\delta]+1} |S_{j+([n\delta]+1)p} - S_{([n\delta]+1)p}| > \frac{1}{3}n^{1/2}\epsilon\right)\\
\leq&P\left(\max_{1 \leq j \leq [n\delta]+1} |S_j| > \frac{1}{3}n^{1/2}\epsilon\right),
\end{split}
\]
so
\[
\begin{split}
&P\left\{ \max_{1 \leq j \leq [n\delta]+1} \max_{0 \leq k \leq [nT]+1} |S_{j+k}-S_k| > n^{1/2} \epsilon\right\}\\
\leq&\sum_{p=0}^{m-1} P\left(\max_{1 \leq j \leq [n\delta]+1} |S_j| > \frac{1}{3}n^{1/2}\epsilon\right)\\
=&m P\left(\max_{1 \leq j \leq [n\delta]+1} |S_j| > \frac{1}{3}n^{1/2}\epsilon\right).
\end{split}
\]
 Lemma \ref{lemma418} tells us
 \[
\lim_{\delta \downarrow 0} \limsup_{n \to \infty} \frac{1}{\delta} P\left(\max_{1 \leq j \leq [n\delta]+1} |S_j| > \frac{1}{3}n^{1/2}\epsilon\right) = 0,
 \]
 and because $m \leq \frac{T}{\delta}+1 = \frac{T+\delta}{\delta}$, 
 \[
\lim_{\delta \downarrow 0} \limsup_{n \to \infty} P\left\{ \max_{1 \leq j \leq [n\delta]+1} \max_{0 \leq k \leq [nT]+1} |S_{j+k}-S_k| > n^{1/2} \epsilon\right\} = 0,
 \]
 proving the claim.
\end{proof}




In the following, $P_n \in \mathscr{P}_E$ denotes the pushforward measure of $P$ by $X^{(n)}$, for $X^{(n)}$ defined in \eqref{Xtn}.
We now prove \textbf{Donsker's theorem}.\footnote{Ioannis Karatzas and Steven E. Shreve,
{\em Brownian Motion and Stochastic Calculus}, second ed., p.~70, Theorem 4.20.}

\begin{theorem}[Donsker's theorem]
$P_n \to W$.
\end{theorem}
\begin{proof}
We shall use Theorem \ref{theorem22} to prove that $\Gamma = \{P_n : n \geq 1\}$ is relatively compact in $\mathscr{P}_E$. 
For $n \geq 1$, 
\[
P_n( f \in E: |f(0)| = 0) = P(\omega \in \Omega : |X_0^{(n)}(\omega)|=0) = 1,
\]
thus the first condition of Theorem \ref{theorem22} is satisfied with $M_\epsilon=0$.
For the second condition of Theorem \ref{theorem22} to be satisfied it suffices that for each $\epsilon>0$,
\[
\lim_{\delta \downarrow 0} \limsup_{n \to \infty} P\left( \sup_{0 \leq s,t \leq 1, |s-t| \leq \delta} |X^{(n)}(s)-X^{(n)}(t)| > \epsilon\right) = 0.
\]
Now,
\[
P\left( \sup_{0 \leq s,t \leq 1, |s-t| \leq \delta} |X^{(n)}_s-X^{(n)}_t| > \epsilon\right) = P\left(\sup_{0 \leq s,t \leq n, |s-t| \leq n\delta} |Y_s-Y_t| > n^{1/2} \epsilon\right).
\]
Also,
\begin{align*}
\sup_{0 \leq s,t \leq n, |s-t| \leq n\delta} |Y_s-Y_t|& \leq \sup_{0 \leq s,t \leq n, |s-t| \leq n\delta} |Y-s-Y_t| \\
&\leq \max_{1 \leq j \leq [n\delta]+1} \max_{0 \leq k \leq n+1} |S_{j+k}-S_k|,
\end{align*}
so 
applying Lemma \ref{lemma419},
\[
\begin{split}
&\lim_{\delta \downarrow 0} \limsup_{n \to \infty} 
P\left( \sup_{0 \leq s,t \leq 1, |s-t| \leq \delta} |X^{(n)}_s-X^{(n)}_t| > \epsilon\right) \\
\leq&\lim_{\delta \downarrow 0} \limsup_{n \to \infty} P\left( \max_{1 \leq j \leq [n\delta]+1} \max_{0 \leq k \leq n+1} |S_{j+k}-S_k| > n^{1/2} \epsilon\right)\\
\to&0,
\end{split}
\]
from which we get that $\Gamma$ is tight in $\mathscr{P}_E$. 
\end{proof}



\end{document}