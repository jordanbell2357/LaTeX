\documentclass{article}
\begin{document}
\title{Bibliography for the history of induction in mathematics}
\author{Jordan Bell}
\date{January 15, 2015}


\maketitle

Mathematical induction (=``complete induction'') often is worked out as a generalizable example (because once we have our hands on something fixed it is easier to do things),
and the idea of a generalizable example is contained in the general idea of induction as used in philosophy.

In the Euler-Goldbach correspondence no.~85--86

Wallis \cite[p.~474]{wallisIV}

Euler used incomplete induction as
an instrument of scientific research.
Ju{\v s}kevi{\v c} \cite{eulerdsb} writes
the following:
``It is frequently said that Euler saw no intrinsic impossibility in
the deduction of mathematical laws from a very limited basis in
observation; and naturally he employed methods of induction to
make empirical use of the results he had arrived at through analysis
of concrete numerical material. But he himself warned many times that an
incomplete induction serves only as a heuristic device, and he never
passed off as finally proved truths the suppositions arrived at by
such methods''; also cf. Weil \cite[Chapter II, \S III]{weil}
and
Cajori \cite{cajori}.

Bernoulli \cite[p.~29]{bernoulli2}

\nocite{*}

\bibliographystyle{amsplain}
\bibliography{induction}

\end{document}