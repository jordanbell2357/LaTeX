\documentclass{article}
\usepackage[polutonikogreek,english]{babel}
\newcommand{\Gk}[1]{\selectlanguage{polutonikogreek}#1\selectlanguage{english}}
\begin{document}
\title{Ancient Greek weights and measures}
\author{Jordan Bell}
\date{March 4, 2017}
\maketitle

Rostovtzeff \cite[I, p.~259]{SEHHW}:

\begin{quote}
Moreover, the comparative study of prices in Egypt and even
in parts of Egypt (a nome, a toparchy, a village) is rendered
very difficult by many peculiar circumstances. This is true, for
example, of the prices of such a basic commodity as grain. In
the statistical lists compiled by modern scholars the price of
grain is shown per {\em artaba}, as if the {\em artaba} were a constant
measure. In fact, as has been proved by Wilcken and Tarn,
many {\em artabae} of various sizes were in use at the same time and
in the same region of Egypt.
\end{quote}

Rostovtzeff \cite[I, p.~279]{SEHHW}:

\begin{quote}
In addition to the rent, the peasant paid a countless number
of different taxes for the privilege of cultivating his land. The
list of these taxes, incomplete as it is, is imposing. The number
of {\em artabae} paid as rent for each {\em aroura} and the quantity of grain
paid in respect of the various taxes is often mentioned in the
documents at our disposal. But we are reduced to uncertain
guesses when we try to figure out what part of the crop the
royal peasants paid to the king. It was certainly not less, and
perhaps more, than half.
\end{quote}

Rostovtzeff \cite[I, p.~451]{SEHHW}:

\begin{quote}
The Babylonian tablets of Hellenistic
date suggest that there was no change as regards the weights
and measures used in Babylonia in the Hellenistic period: the
traditional weights and measures were still in exclusive use. 
But two bronze weights of the Parthian period (56/5 B.C. and
A.D. 72/3), one found near Babylon bearing the name of the
city {\em agoranomos}, the other in Seleucia with the monogram of
the chief of the city police (the {\em paraphylax}), seem to testify to the
use of the Attic standard in Parthian Babylonia, along with
the Babylonian one.
\end{quote}

Syll.\textsuperscript{3} 374, 283/2 BC, Athens Honours Philippides \cite[p.~28]{hellenistic2004}:

\begin{quote}
Whereas Philippides has continued in every circumstance to
show his goodwill towards the {\em demos} and, having journeyed abroad to King
Lysimachus and having first spoken with the king he obtained as a gift for
the {\em demos} 10,000 Attic medimnoi of grain (\Gk{pur~wn med'imnous >Attiko`us}) which were distributed to all the
Athenians in the archonship of Euktemon, \dots
\end{quote}

{\em Milet} I 3 138, 283/2 BC, Knidian Loans to Miletos \cite[p.~31]{hellenistic2004}:

\begin{quote}
The following Knidians made loans to the Milesians in Rhodian silver:\\
Stiphos (son) of Akroteles and Timodamas (son) of Lachartos: 6,000
drachmas.\\
Philophron (son) of Philistas and Archippos (son) of Timaithios: 3,000
drachmas.\\
Diotimos and Mellinos, (sons) of Agathoboulos and Timas (son) of Timas:
6,000 drachmas.\\
Kleisilochos (son) of Anaxippidas: 2,000 drachmas.\\
Antigonos (son) of Epigonos: 6,000 drachmas.\\
Thessalakon (son) of Kallippos: 3,000 drachmas.\\
Stipholaidas (son) of Akrotatos: 2,000 drachmas.\\
Antikrates and Philokrates, (sons) of Epikrates: 3,000 drachmas.\\
Menippos (son) of Apollodoros: 3,000 drachmas.\\
Euphragoras, Kleumenes and Kleumbrotos, (sons) of Philistas: 3,000
drachmas.\\
Kallikles (son) of Athenokritos, Halikarnassian: 6,000 drachmas.\\
Athenodoros (son) of Theodoros, Cyrenean: 12,000 drachmas.\\
These made loans for three years; the loan begins in the month Artemision
in (the stephanephorate) of Alexippos; interest is three obols per mina per
month.\\
The following made loans without interest for a year:\\
Athenagoras (son) of Kleon: 6,000 drachmas.\\
Boularchidas (son) of Archipolis: 6,000 drachmas.\\
Epikydes (son) of Theanos: 4,000 drachmas.\\
Nikandros (son) of Symmachos, Halikarnassian: 2,000 drachmas.\\
The total (of the loans): twelve talents and ten minas of Rhodian (silver).
\end{quote}

{\em Ilion} 33, {\em RC} 11, ca.~274 BC, A Gift of Land by Antiochus I \cite[p.~37]{hellenistic2004}:

\begin{quote}
King Antiochus to Meleager, greeting. Aristodikides has come to us, saying
that, because it had been assigned to Athenaios the commander of the naval
base, he has not even yet received the place Petra and the land belonging to
it, which we previously wrote giving it to him, and he has asked that there
be conveyed to him instead of the land of Petra the same number of {\em plethra}
elsewhere, and that there be granted to him two thousand {\em plethra} besides,
which he may join to any of the cities in our alliance he wishes, just as we
wrote before. 
\end{quote}

OGIS 266, 263--241 BC, Eumenes I and his Mercenaries \cite[p.~46]{hellenistic2004}:

\begin{quote}
Requests which Eumenes (son) of Philetairos granted to [the] soldiers [in]
Philetaireia and to those in Attaleia. To pay as the cash value of the grain
(allowance) four drachmas the {\em medimnos}, and of the wine (allowance) four
drachmas the {\em metretes}. 
\end{quote}

P. Tebt. I 5, 118 BC, Decree of Amnesty and Regulation \cite[p.~98]{hellenistic2004}:

\begin{quote}
And since it sometimes happens that the sitologoi and antigrapheis use larger
measures than the correct bronze measures appointed in each nome . . . in
estimating dues to the Crown, and in consequence the cultivators are
not charged the (correct number of) choinikes, they have decreed that
the strategoi and the overseers of the revenues and the royal scribes shall
test the measures in the most thorough manner possible in the presence
of those concerned with the revenues, of the farmers, and the priests
and the cleruchs and other holders of conceded land . . . , and the measures
must not exceed (the government measure) by more than the two...
allowed for errors. Those who disobey this decree are punishable with
death.
\end{quote}

SB XX 14708, 151 BC, A Komarch's Extortion Racket \cite[p.~159]{hellenistic2004}:

\begin{quote}
Moreover, in the 29th year he compelled us to take charge of the same {\em paralogeia},
for a half-artaba and 50 dr. of bronze for each aroura, which was
exacted illegally through us and Thothoes son of Papentpos by the six-choinix
measure of the village, and the wheat was stored in the houses of Limnaios
and Leontomenes the 80-aroura settler, and the bronze money was turned
over to Mesthasytmis himself. 
\end{quote}

 {\em paralogeia}: ``A term used for collections beyond (para) the legal amount, with the
usual sense of being collected for the benefit of the collector, not to enrich the treasury.'' \cite[p.~160]{hellenistic2004}.

P.Cair.Zen. V 59823, 253 BC, Letter to Zenon about Wax \cite[p.~161]{hellenistic2004}:

\begin{quote}
Promethion to Zenon, greeting. You wrote to me about the wax to say that
the cost per talent, including the toll at Memphis, comes to 44 drachmas,
whereas you are told that with us it costs 40 drachmas. Now do not listen to
the babblers; for it is selling here at 48 drachmas. Please therefore send me
as much as you can. Following your instructions I have given your agent
Aigyptos 500 drachmas of silver towards the price of the wax, and the
remainder, whatever it may be, I will pay immediately to whomever you tell
me to. And of honey also let 5 metretai be procured for me. I appreciate the
kindness and willingness which you always show to me, and if you yourself
have need of anything here, do not hesitate to write. Farewell. Year 33,
Pharmouthi 19. (Address) To Zenon.
\end{quote}

P. Lille 1, 259 BC, Plans for Reclamation Work \cite[pp.~170--171]{hellenistic2004}:

\begin{quote}
The perimeter of the ten thousand arouras is 400 schoinia, there are 4 dikes,
and in the middle (running from south to north) will be 3 dikes at a distance
of 25 schoinia from each other, and 9 others running cross-ways from East to
West, ten schoinia distant from each other; in the 10,000 arouras (there are)
40 dike-enclosed basins of 250 arouras each, whose measurements are 25 by
10 (schoinia); the total number of dikes is 16, each 100 schoinia in length, for
a total of 1,600, which need to be excavated.

The width of the ditch (is to be) 4 cubits, the depth 2, for we suppose that
a ditch of this size will give dikes of the stated size; the total per schoinion is
thus 86 naubia, and for the 1,600 (schoinia) 137,600 naubia. And it is necessary
for another four water-channels to be made in addition to the four existing
ones, at 100 schoinia each, a total of 400 schoinia, at 86 (naubia) a total
of 34,000, or a grand total of 172,000 (naubia).
\end{quote}

P. Hib. I 85, 261 BC, Receipt for Seed-grain \cite[p.~173]{hellenistic2004}:

\begin{quote}
In the reign of Ptolemy son of Ptolemy and his son Ptolemy, year 24, the
priest of Alexander and the Brother and Sister Gods being Aristonikos son of
Perilaos, the canephore of Arsinoe Philadelphos being Charea daughter of
Apios, in the month of Mesore. Pasis, son of ..., priest, has received from
Paris son of Sisybaios, agent of Harimouthes the nomarch from the lower
toparchy, as seed for the 25th year, being included in the lists of receipts
and expenditures, for the royal holding of Philoxenos of the (troop) of
Telestes 40 artabas of wheat, $38 \frac{1}{3}$
of barley which are equivalent to 23 of
wheat, and $67 \frac{1}{2}$ of olyra which are equivalent to 27 of wheat, making a total
of 90 artabas of wheat, in grain pure and unadulterated in any way, according
to just measurement by the 29-choinix measure on the bronze standard.
\end{quote}

P. Rev., 259 BC, ``Revenue Laws'' of Ptolemy Philadelphos \cite[p.~188]{hellenistic2004}:

\begin{quote}
They shall seal the oil in the country at the rate of 48 drachmas in copper
for a metretes of sesame oil or cnecus oil containing 12 choes, and at the rate
of 30 drachmas for a metretes of castor oil, colocynth oil, or lamp oil. (Altered
to . . . both sesame and cnecus oil and castor oil, colocynth oil and lamp oil at
the rate of 48 drachmas in copper for a metretes of 12 choes and 2 obols for
a cotyla.)
\end{quote}

P. Tebt. I 35, 267 BC, Regulating the Price of Myrrh \cite[p.~197]{hellenistic2004}:

\begin{quote}
Apollonios to the epistatai in the division of Polemon and the other officials,
greeting. For the myrrh distributed in the villages no one shall exact more
than 40 drachmas of silver for a mina-weight, or in copper 3 talents 2,000
drachmas, and 200 drachmas on the talent for carriage; which sum shall be
paid not later than Pharmouthi 3 to the collector sent for this purpose. 
\end{quote}





SEG 46, 2120:

\begin{quote}
Rectangular local pink sandstone stele with rounded top; found in Bir 'Iayyan (Eastern Desert), an unfortified Ptolemaic station supplying water to passersby on the road between the gold-mining center at Barramiya and the Nile River centers of Edfou and el-Kab; it is located ca. 97 km. east of Apollinopolis Magna (Edfou).
\end{quote}

\cite[p.~322]{bagnall1996}

\begin{quote}
From the river to this point, four hundred sixty-one stadioi (\Gk{st'adioi tetrak'osioi <ex'hkonta e~<is}). In the
reign of Ptolemy son of Ptolemy Soter, year 28, month of Epeiph, Rhodon
son of Lysimachos, from Ptolemais, toparch of the three ?, set up (this
stone).
\end{quote}

\cite[p.~322]{bagnall1996}

\begin{quote}
The Greek stade was of notoriously variable length. If, as one
would expect, it means here the most common Ptolemaic stade of about
212 meters, the distance of 461 stades here would be about 97.7 km, very
nearly exactly the actual measured distance (cf. section 1). Minor deviations
in the course of the modern road from the ancient would suffice to
explain this difference.
\end{quote}





SEG 38, 619 \cite[p.~367]{thonemann}:

\begin{quote}
When Timesios was priest of Lysimachos. King Lysimachos has given these estates to
Limnaios son of Harpalos as a patrimonial possession: the estate in the (former) territory
of Sermylia, 1200 \Gk{pl'ejra} of land planted with trees, contiguous with the properties
of Agathokles of Lysimachos and Bithys son of Kleon; the estate in the
(former) territory of Olynthos at Trapezous, 360 \Gk{pl'ejra} of land planted with trees,
contiguous with thc properties of Menon son of Sosikles and Pylon son of Epiteles;
and the estate in the (former) territory of Strepsa, 900 \Gk{pl'ejra} of land planted with
trees and 20 \Gk{pl'ejra} of vineyards, contiguous with the properties of Gouras son of Annythes,
Chionidcs, and Eualkes son of Demetrios. (He has granted these estates) to
him and his descendants with the full right to possess and sell and exchange and give
them to whomsoever they wish.
\end{quote}

SIG 3, 302 \cite[p.~371]{thonemann}:

\begin{quote}
God. Good fortune. When Alexander was king, in the eleventh year when Menandros was satrap and Isagoras was prytanis.

Krateuas gave to Aristomenes a plot (\Gk{>agr'os}) of arable land (\Gk{g`h yil'h}) on which
to settle (\Gk{>epoik'isai}), in addition to the nursery (\Gk{fut'on}) planted under Krateuas.

The perimeter of the land (\Gk{g'h}) is 170 \Gk{k'uproi} of seed, and
building-plots (\Gk{o`ik'opeda}) and a garden (\Gk{k~hpos}).
The tribute (\Gk{f'oros}) payable on the garden is one gold stater (\Gk{qrouso~us})
per year.
\end{quote}






Athenian Agora \cite[p.~296]{grace1934}:

\begin{quote}
 Latin and late Greek literatures are full of evidence in the form of comments, anecdotes, and
 quotations from earlier writers, as to the esteem in which the wine of Chios was held;
 it is the famous wine of antiquity.
 There was a not unnatural excitement, therefore,
 over some broken jars found last year in an ancient well when it was realized that
 they were stamped with the arms of Chios. One of these jars is illustrated on p.~202
 (Fig. 1, 1). Of the five reconstructed, the shape and dimensions are so constant that
 evidently a uniform capacity was intended. This is approximately 22 litres, or six
 gallons.
 An anecdote of Plutarch informs us that in the time of Socrates an amphora
 of Chian wine sold in Athens for one mina, which is a hundred drachmas, or a hundred
 day's wages to the skilled workman of that period. Our jars are dated by the deposit
 in which they were found in the third quarter of the fifth century B.C., probably about
 430.
\end{quote}

Athenian Agora \cite[pp.~303--304]{grace1934}:

\begin{quote}
Chian amphora: third quarter of 5th cent. B.C.

SS 1838. Ht. 0.702 m. One of a series of nearly complete jars found in Section I
 in an ancient well the filling of which can be dated closely by the red-figured and
 black-glazed stamped ware included. Of the large jars, seven bear stamps: five, of which
 this is one, have the Chian coin type illustrated Pl. I, 1; these are all similar in shape
  to a sixth, the stamp of which shows simply a kantharos;
 the seventh is no.~2 of this figure. For Chian clay, see the
 note on fabric (p.~201). For the profile of the handle, see
 Pl. II, 1. The capacity of each of the six similar jars is roughly twenty-two litres
\end{quote}



\begin{quote}
 Similar complete Chian amphorae of a somewhat later time hold seven Athenian choes.
 Despite the evidence which makes the twelve-chous metretes the large liquid unit, the ordinary
 amphora of Greek as well as Roman times is more likely to hold eight choes.i This fact combines
 with the 7: 8 ratio of Chian and Athenian coins to persuade us that the Chian standard chous also
 stood in a 7: 8 ratio with the Athenian chous. These jars would then hold eight Chian, or seven
 Athenian, choes. On this pot the informal indication of capacity would then be the work of the
 Athenian purchaser, probably for a re-use of the jar.
 \end{quote}





SIG 334

\begin{quote}
This was 80 certified talents of Alexandrian silver and 18,000 gold pieces, making a total of 140 talents. 
\end{quote}


P. Lond. V, 1718 \cite[pp.~154--165]{PLondV}


Plutarch, Moralia 416 B, ``often the measures and the things measured are called by the same name,
as, for example, the kotyle, choinix, amphoreus and medimnos.''

Greek liquid measures

\begin{tabular}{ll}
kotyle&\\
chous&12 kotylai\\
metretes&12 choes
\end{tabular}

Greek dry measures

\begin{tabular}{ll}
kotyle&\\
choinix&4 kotylai\\
medimnos&48 choinikes
\end{tabular}

\begin{tabular}{ll}
stadion&\\
cubit&
\end{tabular}

\begin{tabular}{ll}
parasang&
\end{tabular}

Herodotus 2.6.3, Strabo 11.11.5

Olympics, Julius Africanus, {\em Chronographiae}, 199

\begin{quote}
33rd. Gylis of Laconia\qquad stadion race.\\
A pancration was added, and the victor was Lygdamis of Syracuse, an enormous
man who measured out the stadion with his feet in only 600 paces.\\
A horse race was added, and the victor was Craxilas of Thessaly.
\end{quote}

Rhodes and Osborne \cite[p.~121]{rhodes}:

\begin{quote}
The buyer will weigh out the wheat at a weight of a talent for five {\em hekteis}, and the barley at a weight of a talent for a {\em medimnos}, dry and clean of
darnel, arranging the standard weight on the balance, just as the other merchants.
\end{quote}



Engels \cite{engels}

Walbank \cite{walbankII}



P. Oxy I.9 verso \cite[pp.~77--79]{POxyI}, Oxyrhynchus, third or early fourth century AD,
List of Weights and Measures:

\begin{quote}
A copper drachma has 6 obols, and an obol 8 chalki, so that the copper drachma
consist of 48 chalki. A drachma has seven, 7, obols, and an obol has 8 chalki, so that
the drachma consists of 56 chalki. The talent has 60 minae, and the mina 25 staters or
100 drachmae, and the stater has 4 drachmae, so that the talent consists of 1500 staters
or 6000 drachmae, or forty-two thousand obols. An artaba has 10 measures, and the
measure has 4 choenices, so that the artaba consists of 40 choenices. A medimnus has 12
hemihekta and the hemihekton four choenices, so that the medimnus consists of forty-eight
choenices. The ell has 6 palms, and the palm 4 digits, so that the ell consists of 24
digits. The metretes has 12 cho\"es, and the chous 12 cotylae, so that the metretes consists
of 144 cotylae. The mina-weight has sixteen, 16 quarters, and a quarter has ....
\end{quote}

P. Oxy. IV.669 \cite[pp.~116--121]{POxyIV}, Oxyrhynchus, end of the third century AD, Metrological Work, ll.~1--20:

\begin{quote}
The schoenium [\Gk{sqoin'ion}] used in land-survey has 8 eighths [\Gk{>'wgdoa h}], and the eighth [\Gk{>'ogdoon}] has 12 cubits [\Gk{p'hqis ib}],
so that the schoenium used in land-survey has 96 cubits [\Gk{phq~wn {\textqoppa}{\textstigma}}], while the \dots schoenium has
100 cubits [\Gk{phq~wn r}]. The linear cubit [\Gk{e>ujumetri]\d{k}`os p~hqe'is}] is that which is measured by length alone, the plane
cubit [\Gk{>embadik`os}] is that which is measured by length and breadth; the solid cubit [\Gk{stere`os}] is that which
is measured by length and breadth and depth or height. The \dots building cubit [\Gk{$\langle{}$o$\rangle{}$>ikopedik`os p~hqis}] contains
100 plane cubits. \Gk{Na'ubia} are measured by the \Gk{x'ulon}; the royal \Gk{x'ulon} contains 3 cubits,
18 \Gk{palaista'i}, 72 \Gk{d'aktuloi}, while the \dots \Gk{x'ulon} contains $2 \frac{2}{3}$ cubits [\Gk{b{b'}}], 16 \Gk{palaista'i} and
64 \Gk{d'aktuloi}; so that the schoenium used in land-survey contains 32 royal \Gk{x'ula} and 36 \dots \Gk{x'ula}.
\end{quote}

ll.~31--41:

\begin{quote}
2  \Gk{palaista'i} make a \Gk{liq'as}, 3 \Gk{palaista'i} a \Gk{spijam'h}, 4 \Gk{palaista'i} an (Egyptian?)
foot [\Gk{po`us}], 5 a cloth-weaver's cubit [\Gk{p~hqus lino\={u}fik`os}] \dots, 6  \Gk{palaista'i} a public and a carpenter's cubit, 7 \Gk{palaista'i}
a Nilometric cubit, 8 \Gk{palaista'i} a \dots cubit, 10 \Gk{palaista'i} a \Gk{b~hma}, which is the distance
of the outstretched feet. 3 cubits [\Gk{g p'hq[eis]}] make a public \Gk{x'ulon}, 4 cubits an \Gk{>orgui'a}, which is the
distance of the outstretched hands. . . cubits make a \Gk{k'alamos}, $6 \frac{2}{3}$ [\Gk{{\textstigma}{b'}}] an \Gk{>'akaina}.
\end{quote}

P. Ryl. II.64 \cite[pp.~4--5]{PRyl2}

P.Oxy.XIII 1609, 1669 v, P.Oxy.XLIX 3455, P.Oxy.XLIX, P.Oxy.XLIX 3457, P.Oxy.XLIX 3458, 
P.Oxy.XLIX 3459, P.Oxy.XLIX 3460

Hultsch \cite{metrologie}

{\em Tabula Heronianae} \cite[p.~182]{MSRI}

Segr\`e \cite{segre}

Pernice \cite{pernice}

Stilbach \cite{schilbach}

Smith, {\em A Dictionary of Greek and Roman Antiquities}, 1891, s.v. logistica.

Ptolemy, {\em Geography} \cite{berggren}



\bibliographystyle{plain}
\bibliography{greekweights}

\end{document}