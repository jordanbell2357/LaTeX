\documentclass{article}
\usepackage{amsmath,amssymb,graphicx,subfig,mathrsfs,amsthm}
\usepackage{tikz-cd}
\newcommand{\inner}[2]{\langle #1, #2 \rangle}
\newcommand{\Res}{\textrm{Res}} 
\newcommand{\SO}{\textrm{SO}} 
\newcommand{\Diff}{\textrm{Diff}} 
\newcommand{\Sympl}{\textrm{Sympl}} 
\newcommand{\Lie}{\textrm{Lie}} 
\newcommand{\id}{\textrm{id}} 
\newcommand{\Ad}{\textrm{Ad}} 
\newcommand{\norm}[1]{\left\Vert #1 \right\Vert}
\newtheorem{theorem}{Theorem}
\newtheorem{lemma}[theorem]{Lemma}
\newtheorem{corollary}[theorem]{Corollary}
\begin{document}
\title{Summable series and the Riemann rearrangement theorem}
\author{Jordan Bell\\ \texttt{jordan.bell@gmail.com}\\Department of Mathematics, University of Toronto}
\date{\today}

\maketitle

\section{Introduction}
Let $\mathbb{N}$ be the set of positive integers. A function from $\mathbb{N}$ to a set is called a {\em sequence}. If $X$ is a topological space 
and $x \in X$, a sequence $a:\mathbb{N} \to X$ is said to {\em converge to $x$} if for every open neighborhood $U$ of $x$ there is some
$N_U$ such that $n \geq N_U$ implies that $a_n \in U$. 
If there is no $x \in X$ for which  $a$ converges to $x$, we say that $a$ {\em diverges}.

Let $a:\mathbb{N} \to \mathbb{R}$.
We define $s(a):\mathbb{N} \to \mathbb{R}$ by $s_n(a)=\sum_{k=1}^n a_k$. We call $s_n(a)$  the {\em $n$th partial sum of the sequence $a$}, and we call
the sequence $s(a)$ a {\em series}. 
If there is some $\sigma \in \mathbb{R}$ such that $s(a)$ converges to $\sigma$, we write
\[
\sum_{k=1}^\infty a_k = \sigma.
\]

\section{Goldbach}
Euler \cite[\S 110]{E212}:
``If, as is commonly the case, we take the sum of a series to be the aggregate
of all of its terms, actually taken together, then there is no doubt that only
infinite series that converge continually closer to some value, the more
terms we actually add, can have sums''.


Euler Goldbach correspondence nos. 55, 161, 162.



\section{Dirichlet}
In 1837 Dirichlet proved that one can rearrange terms in an absolutely convergent series and not change the sum,
and gave examples to show that this was not the case for conditionally
convergent series.

If $a$ is a sequence and the series $s(|a|)$ converges, we say that the series $s(a)$ is {\em absolutely convergent}.
Because $\mathbb{R}$ is a complete metric space, a series being absolutely convergent implies that it is convergent.

Dirichlet \cite{dirichlet} and \cite[p.~176, \S 101]{dedekind}


Elstrodt \cite{elstrodt} and \cite{MR1613384}

In the following theorem we prove that if a series converges absolutely, then every rearrangement of it converges to the same
value. Our proof follows Landau \cite[p.~157, Theorem 216]{landau}.

\begin{theorem}
If $a$ is a sequence for which $s(a)$ converges absolutely and 
\[
\sum_{n=1}^\infty a_n = \sigma,
\]
then for any bijection $\lambda:\mathbb{N} \to \mathbb{N}$, the series $s(a \circ \lambda)$ converges to
$\sigma$.
\end{theorem}
\begin{proof}
Let $\epsilon>0$, and let $M$ be large enough so that
\[
\sum_{n=M}^\infty |a_n|<\epsilon.
\]
Let $r$ be large enough so that
\[
\{n: 1 \leq n <M\} \subseteq \{\lambda_n: 1 \leq n \leq r\}.
\]
Fix $m \geq r$, and
let $h:\mathbb{N} \to \mathbb{N}$ be the sequence whose terms are the elements of
\[
\mathbb{N} \setminus \{\lambda_n: 1 \leq n \leq m\}
\]
arranged in ascending order. If $t+m \geq \max_{1 \leq n \leq m} \lambda_n$ then 
\[
\{\lambda_n: 1 \leq n \leq m\} \cup \{h_n: 1 \leq n \leq t\} = \{n: 1 \leq n \leq t+m\},
\]
and hence
\[
\sum_{n=1}^m a_{\lambda_n} + \sum_{n=1}^t a_{h_n} = \sum_{n=1}^{t+m} a_n.
\]
Taking $t \to \infty$, we get
\[
\sum_{n=1}^m a_{\lambda_n} + \sum_{n=1}^\infty a_{h_n} = \sigma;
\]
the series $s(a \circ h)$ converges because for sufficiently large $n$, $h_n=n$.
Hence, for every $m \geq r$,
\[
\left| \sum_{n=1}^m a_{\lambda_n} - \sigma \right| = \left| \sum_{n=1}^\infty a_{h_n} \right|
\leq \sum_{n=1}^\infty |a_{h_n}| \leq \sum_{n=M}^\infty |a_n| < \delta,
\]
which shows that $s(a \circ \lambda)$ converges to $\sigma$.
\end{proof}


\section{Riemann rearrangement theorem}
If $a:\mathbb{N} \to \mathbb{R}$ and $\lambda:\mathbb{N} \to \mathbb{N}$ is a bijection,
we call the sequence $a \circ \lambda:\mathbb{N} \to \mathbb{R}$ a {\em rearrangement} of the sequence $a$.

Because $\mathbb{N}$ is a well-ordered set, if there are at least $n$ elements in the set $\{k \in \mathbb{N}: a_k \geq 0\}$ then it makes
sense to talk about the $n$th nonnegative term in the sequence $a$. If $a$ were not a function from $\mathbb{N}$ to $\mathbb{R}$ but merely
a function from a countable set to $\mathbb{R}$, it would not make sense to talk about the $n$th nonnegative term in $a$ or the $n$th negative
term in $a$.


Riemann \cite[pp.~96-97]{riemann}

Our proof follows Landau \cite[p.~158, Theorem 217]{landau}.



\begin{theorem}[Riemann rearrangement theorem]
If $a:\mathbb{N} \to \mathbb{R}$ and $s(a)$ converges but $s(|a|)$ diverges, then
 for any nonnegative real number $\sigma$ there is some rearrangement $b$ of $a$ such that $s(b) \to \sigma$.
\end{theorem}
\begin{proof}
Define $p,q:\mathbb{N} \to \mathbb{R}$ by
\[
p_n = \frac{|a_n|+a_n}{2}, \qquad q_n = \frac{|a_n|-a_n}{2}.
\]
$p_n$ and $q_n$ are nonnegative, and satisfy $p_n-q_n = a_n$, $p_n+q_n=|a_n|$. 
If one of $s(p)$ or $s(q)$ converges and the other diverges, we obtain a contradiction from
\[
s_n(a)=\sum_{k=1}^n a_k = \sum_{k=1}^n (p_k-q_k) = \sum_{k=1}^n p_k - \sum_{k=1}^n q_k=s_n(p)-s_n(q)
\]
and the fact that $s(a)$ converges.
If both $s(p)$ and $s(q)$ converge, then we obtain a contradiction from
\[
s_n(|a|)=\sum_{k=1}^n |a_k| = \sum_{k=1}^n (p_k+q_k) = \sum_{k=1}^n p_k + \sum_{k=1}^n q_k = s_n(p)+s_n(q)
\] 
and the fact that $s(|a|)$ diverges. Therefore, both $s(p)$ and $s(q)$ diverge.

Because $s(a)$ converges and $s(|a|)$ diverges, there are infinitely many $n$ with $a_n>0$ and there are infinitely many $n$ with
$a_n<0$. Let $P_n$ be the $n$th nonnegative term in the sequence $a$, and let $Q_n$ be the absolute value of the $n$th negative term
in the sequence $a$. 
The fact that $s(p)$ diverges  implies that  $s(P)$ diverges, and the fact that $s(q)$ diverges implies that $s(Q)$ diverges.

Let $\sigma\geq 0$. We define sequences $\mu,\nu:\mathbb{N} \to \mathbb{N}$ by induction
as follows.
Let $\mu_1$ be  the least element of $\mathbb{N}$ such that
\[
s_{\mu_1}(P) >\sigma,
\]
and with $\mu_1$ chosen, let $\nu_1$ be the least element of $\mathbb{N}$ such that 
\[
s_{\mu_1}(P)-s_{\nu_1}(Q)<\sigma.
\]
Let $m_2$ be the least element of $\mathbb{N}$ such that
\[
s_{\mu_2}(P)-s_{\nu_1}(Q)>\sigma,
\]
and with $\mu_2$ chosen, let $\nu_2$ be the least element of $\mathbb{N}$ such that
\[
s_{\mu_2}(P)-s_{\nu_2}(Q)<\sigma.
\]
It is straightforward to check that $\mu_2>\mu_1$ and $\nu_2>\nu_1$.

Suppose that $\mu_1,\ldots,\mu_n$ and $\nu_1,\ldots,\nu_n$ have been chosen,
that $\mu_n$ is the least element of $\mathbb{N}$ such that
\[
s_{\mu_n}(P)-s_{\nu_{n-1}}(Q)>\sigma,
\]
that $\nu_n$ it the least element of $\mathbb{N}$ such that
\[
s_{\mu_n}(P)-s_{\nu_n}(Q)<\sigma
\]
and that
$\mu_n>\mu_{n-1}$ and $\nu_n>\nu_{n-1}$. 
Let $\mu_{n+1}$ be the least element of $\mathbb{N}$ such that
\[
s_{\mu_{n+1}}(P)-s_{\nu_n}(Q)>\sigma,
\]
and with $\mu_{n+1}$ chosen, let $\nu_{n+1}$ be the least element of $\mathbb{N}$ such that
\[
s_{\mu_{n+1}}(P)-s_{\nu_{n+1}}(Q)<\sigma.
\]
It is straightforward to check that $\mu_{n+1} > \mu_n$ and
$\nu_{n+1}>\nu_n$.

Define $b:\mathbb{N} \to \mathbb{R}$ by taking $b_n$ to be the $n$th term in 
\[
P_1,\ldots,P_{\mu_1},-Q_1,\ldots,-Q_{\nu_1},P_{\mu_1+1},\ldots,P_{\mu_2},-Q_{\nu_1+1},\ldots,-Q_{\nu_2},\ldots,
\]
which, because the sequences $\mu$ and $\nu$ are strictly increasing, is a rearrangement of the sequence $a$.


\end{proof}


\section{Symmetry}
Don't use order where it is accidental.


\section{Nets}
A {\em directed set} is a set $D$ and a binary relation $\preceq$ satisfying
\begin{itemize}
\item if $m,n,p \in D$, $m \preceq n$, and $n\preceq p$, then $m \preceq p$
\item if $m \in D$, then $m \preceq m$
\item if $m,n \in D$, then there is some $p \in D$ such that $m \preceq p$ and $n \preceq p$.
\end{itemize}
For example, let $A$ be a set, let $D$  be the set of all subsets of $A$, and say that  $F \preceq G$ when $F \subseteq G$. Check that 
$(D,\preceq)$ is a directed set: for $F,G \in D$, we have $F \cup G \in D$,
and $F \cup G$  is an upper bound for both $F$ and $G$. 

A {\em net} is a function from a directed set $(D,\preceq)$ to a set $X$. 
Let $(X,\tau)$ be a topological space, let $S:(D,\preceq) \to (X,\tau)$ be a net, and let $x \in X$. We say that $S$ {\em converges to $x$} if for every $U \in \tau$ with $x \in U$
 there is some $N_U \in D$ such that $N_U \preceq i$ implies that $S(i) \in U$. One proves that a topological space is Hausdorff if and only if every net in this space converges to at most one
point  \cite[p.~67, Theorem 3]{kelley}.

A net $S:(D,\preceq) \to \mathbb{R}$ is said to be {\em increasing} if $m \preceq n$ implies that $S(m) \leq S(n)$.

\begin{lemma}
If $S:(D,\preceq) \to \mathbb{R}$ is an increasing net and the range $R$ of $S$ has an upper bound, then $S$ converges to the supremum  of $R$.
\label{monotone}
\end{lemma}
\begin{proof}
Because $R$ is a subset of $\mathbb{R}$ that has an upper bound, it has a supremum, call it $\sigma$. To say that $\sigma$ is the supremum of $R$ means that
for all $r \in R$ we have $r \leq \sigma$ ($\sigma$ is an upper bound) and that for all $\epsilon>0$ there is some $r_\epsilon \in R$ with $\sigma-\epsilon<r_\epsilon$ (nothing less than 
$\sigma$ is an upper bound).
Take $\epsilon>0$. There is some $r_\epsilon \in R$ with $\sigma-\epsilon<r_\epsilon$.  As $r_\epsilon \in R$,
there is some $n_\epsilon \in D$ with $S(n_\epsilon) = r_\epsilon$. If $n_\epsilon \preceq n$, then because $S$ is increasing,
$S(n_\epsilon) \leq S(n)$, and hence
\[
\sigma-\epsilon < r_\epsilon = S(n_\epsilon) \leq S(n).
\]
But $S(n) \in R$, so $S(n) \leq \sigma$. Hence $n_\epsilon \preceq n$ implies that $|S(n)-\sigma|<\epsilon$, showing that $S$
converges to $\sigma$.
\end{proof}


\section{Unordered sums}
Let $A$ be a set, and let $\mathscr{P}_0(A)$ be the set of all finite subsets of $A$. 
Check that $(\mathscr{P}_0(A),\subseteq)$ is a directed set:
if $F,G \in \mathscr{P}_0(A)$  then
$F \cup G \in \mathscr{P}_0(A)$ and $F \cup G$ is an upper bound for both $F$ and $G$. Let $f:A \to \mathbb{R}$ be a function, and define $S_f:\mathscr{P}_0(A) \to \mathbb{R}$ by
\[
S_f(F) =\sum_{a \in F} f(a), \qquad F \in \mathscr{P}_0(A).
\]
If the net $S_f$ converges, we say that the function $f$ is {\em summable}, and we call the element of $\mathbb{R}$ to which $S_f$ converges the {\em unordered sum of $f$},
denoted by
\[
\sum_{a \in A} f(a).
\]
If $B$ is a subset of $A$, we say that $f$ is {\em summable over $B$} if the restriction of $f$ to $B$ is summable. If $f_B$ is the restriction of $f$ to $B$
and $f$ is summable over $B$ (i.e. $f_B$ is summable), by 
\[
\sum_{a \in B} f(a) 
\]
we mean
\[
\sum_{a \in B} f_B(a).
\]


\begin{lemma}
Suppose that $f,g:A \to \mathbb{R}$ are functions and  $\alpha,\beta \in \mathbb{R}$. If $f$ and $g$ are summable,
then $\alpha f+  \beta g$ is summable and
\[
\sum_{a \in A} (\alpha f(a)+ g(a)) = \alpha \sum_{a \in A} f(a) + \beta  \sum_{a \in A} g(a).
\]
\label{linearity}
\end{lemma}
\begin{proof}
Let $\sigma_1=\sum_{a \in A} f(a)$ and $\sigma_2=\sum_{a \in A} g(a)$, and set $h = \alpha f+ \beta g$.
For $\epsilon>0$, there is some $F_\epsilon \in \mathscr{P}_0(A)$ such that $F_\epsilon \subseteq F \in \mathscr{P}_0(A)$ implies
that $|S_f(F)-\sigma_1|<\epsilon$, and there is some $G_\epsilon \in \mathscr{P}_0(A)$ such that
$G_\epsilon \subseteq G \in \mathscr{P}_0(A)$ implies that $|S_g(G)-\sigma_2|<\epsilon$. Let $H_\epsilon = F_\epsilon \cup G_\epsilon
 \in \mathscr{P}_0(A)$. If $H_\epsilon \subseteq H \in \mathscr{P}_0(A)$, then, as $F_\epsilon \subseteq H$ and $G_\epsilon \subseteq H$,
\begin{eqnarray*}
|S_h(H)- (\alpha \sigma_1 + \beta \sigma_2)|&=&\left|\sum_{a \in H} (\alpha f(a)+\beta g(a)) - \alpha \sigma_1 - \beta \sigma_2 \right|\\
&=&| \alpha S_f(H)+\beta S_g(H) - \alpha \sigma_1 -\beta  \sigma_2|\\
&\leq&|\alpha| |S_f(H) - \sigma_1| +|\beta| |S_g(H)-\sigma_2|\\
&\leq&|\alpha| \epsilon +|\beta| \epsilon;
\end{eqnarray*}
we write $\leq$ rather than $<$ in the last inequality to cover the case where $\alpha=\beta=0$. 
It follows that $S_h$ converges to $\alpha \sigma_1 + \beta \sigma_2$. 
\end{proof}

The following lemma is simple to prove and ought to be true, but should not to be called obvious. 
For example, the Ces\`aro sum of the sequence $1,-1,1,-1,\ldots$ is $\frac{1}{2}$, while the Ces\`aro sum of the sequence
$1,-1,0,1,-1,0,\ldots$ is $\frac{1}{3}$.

\begin{lemma}
If $f:A \to \mathbb{R}$ is summable, then for any set $C$ that contains $A$, the function $g:C \to \mathbb{R}$ defined by
\[
g(c) = \begin{cases}
f(c)&c \in A\\
0&\textrm{otherwise}
\end{cases}
\]
is summable, and
\[
\sum_{a \in A} f(a) = \sum_{c \in C} g(c).
\]
\label{nullsets}
\end{lemma}
\begin{proof}
Let $\sigma=\sum_{a \in A} f(a)$. For $\epsilon>0$, there is some $F_\epsilon \in \mathscr{P}_0(A)$ such that 
$F_\epsilon \subseteq F \in \mathscr{P}_0(A)$ implies that $|S_f(F)-\sigma|<\epsilon$. If $F_\epsilon \subseteq H \in \mathscr{P}_0(C)$,
then, as $F_\epsilon \subseteq H \cap A \in \mathscr{P}_0(A)$,
\begin{eqnarray*}
|S_g(H)-\sigma|&=&\left| \sum_{c \in H} g(c) - \sigma\right|\\
& =& \left| \sum_{c \in H \cap A} g(c) + \sum_{c \in H \setminus A} g(c) - \sigma\right|\\
&=&\left| \sum_{a \in H \cap A} f(a) + \sum_{c \in H \setminus A} 0 - \sigma\right|\\
&=&|S_f(H \cap A) - \sigma|\\
&<&\epsilon.
\end{eqnarray*}
This shows that $S_g$ converges to $\sigma$.
\end{proof}

The previous two lemmas are useful, and also convince us that unordered summation works similarly to finite sums. 
We now establish conditions under which a function is summable.

\begin{lemma}
If $f:A \to \mathbb{R}$ is nonnegative and there is some $M \in \mathbb{R}$ such that
$F \in \mathscr{P}_0(A)$ for all
$S_f(F) \leq M$, then $f$ is summable. If $f:A \to \mathbb{R}$ is nonnegative and  summable, then $S_f(F) \leq \sum_{a \in A} f(a)$ for all $F \in \mathscr{P}_0(A)$.
\label{bounded}
\end{lemma}
\begin{proof}
Suppose there is some $M \in \mathbb{R}$ such that if $F \in \mathscr{P}_0(A)$ then $S_f(F) \leq M$. That is, $M$ is an upper bound for the range of $S_f$.
Because $f$ is nonnegative, the net $S_f$ is increasing. We apply Lemma \ref{monotone}, which tells us that $S_f$ converges to the supremum of its
range. That $S_f$ converges means that $f$ is summable.

Suppose that $f$ is summable, and let $\sigma=\sum_{a \in A} f(a)$. Suppose by contradiction that there is some $F_0 \in \mathscr{P}_0(A)$ such that
$S_f(F_0)>\sigma$, and let $\epsilon = S_f(F_0) - \sigma$. Then there is some $F_\epsilon \in \mathscr{P}_0(A)$ such that
$F_\epsilon \subseteq F \in \mathscr{P}_0(A)$ implies that $|S_f(F) -\sigma| <\epsilon$. As $F_\epsilon \subseteq F_0 \cup F_\epsilon
\in\mathscr{P}_0(A)$, we have 
$|S_f(F_0 \cup F_\epsilon) - \sigma|<\epsilon$, and hence
\[
S_f(F_0 \cup F_\epsilon) < \sigma+\epsilon =  S_f(F_0).
\]
But $F_0$ is contained in $F_0 \cup F_\epsilon$ and $f$ is nonnegative, so
\[
S_f(F_0) \leq S_f(F_0 \cup F_\epsilon),
\]
which gives $S_f(F_0) < S_f(F_0)$, a contradiction. Therefore, there is no $F \in \mathscr{P}_0(A)$ for which $S_f(F_0) > \sigma$.
\end{proof}






\begin{lemma}
Suppose that $f:A \to \mathbb{R}$ is a function and that $A_+ = \{a \in A: f(a) \geq 0\}$ and $A_-=\{a \in A: f(a) \leq 0\}$. Then,
$f$ is summable if and only if $f$ is summable over both $A_+$ and $A_-$. If $f$ is summable, then
\[
\sum_{a \in A} f(a) = \sum_{a \in A_+} f(a) + \sum_{a \in A_-} f(a).
\]
\label{positivenegative}
\end{lemma}
\begin{proof}
Suppose that $f$ is summable.
Because $f$ is summable, there is  some $E \in \mathscr{P}_0(A)$ such that $E \subseteq F \in \mathscr{P}_0(A)$ implies that
$|S_f(F)-\sigma|<1$.
Define
\[
E_+ = \{a \in E: f(a) \geq 0\} \in \mathscr{P}_0(A_+), \qquad E_-=\{a \in E: f(a) \leq 0\} \in \mathscr{P}_0(A_-).
\]
 If $G \in \mathscr{P}_0(A_+)$ then
$E \subseteq G \cup E  \in \mathscr{P}_0(A)$, and 
hence 
$|S_f(G \cup E) - \sigma|<1$.
We have
\[
S_{f_+}(G) = \sum_{a \in G} f(a) \leq \sum_{a \in G \cup E_+} f(a) = \sum_{a\in G \cup E} f(a) - \sum_{a \in E_-} f(a),
\]
and hence
\[
S_{f_+}(G) \leq S_f(G \cup E) - S_f(E_-) < \sigma + 1 - S_f(E_-).
\]
That is, $\sigma+1-S_f(E_-)$ is an upper bound for the range of $S_{f_+}$. The net $S_{f_+}$ is increasing, hence applying Lemma \ref{monotone} we get that
$S_{f_+}$ converges. That is, $f_+$ is summable. 
If $H \in \mathscr{P}_0(A_-)$, then $E \subseteq H \cup E \in \mathscr{P}_0(A)$, and hence $|S_f(H \cup E)-\sigma|<1$. We have
\[
S_{f_-}(H) = \sum_{a \in H} f(a) \geq \sum_{a \in H \cup E_-} f(a) = \sum_{a \in H \cup E} f(a) - \sum_{a \in E_+} f(a),
\]
and then
\[
S_{f_-}(H) \geq S_f(H \cup E) - S_f(E_+) > \sigma-1 - S_f(E_+),
\]
showing that $-\sigma+1+S_f(E_+)$ is an upper bound for the net $-S_{f_-}$. As $-S_{f_-}$ is increasing, by Lemma \ref{monotone} it
converges, and it follows that $S_{f_-}$ converges. That is, $f_-$ is summable.

Suppose that $f$ is summable over both $A_+$ and $A_-$.
Let $f_+$ be the restriction of $f$ to $A_+$ and let $f_+$ be the restriction of $f$ to $A_+$, and define
$g_+,g_-:A \to \mathbb{R}$ by
\[
g_+(a)=
\begin{cases}
f(a)&a \in A_+\\
0&a \in A_-,
\end{cases}
\qquad
g_-(a)=\begin{cases}
0&a \in A_+\\
f(a)&a \in A_-.
\end{cases}
\]
By Lemma \ref{nullsets}, $f_+$ being summable implies that $g_+$ is summable, with
\[
\sum_{a \in A_+} f_+(a) = \sum_{a \in A} g_+(a),
\]
 and $f_-$ being summable implies that $g_-$ is summable, with
 \[
 \sum_{a \in A_-} f_-(a) = \sum_{a \in A} g_-(a).
 \]
But $f=g_++g_-$, so by Lemma \ref{linearity} we get that $f$ is summable, with
\[
\sum_{a \in A} f(a) = \sum_{a \in A} g_+(a) + \sum_{a \in A} g_-(a) = \sum_{a \in A_+} f_+(a)  + \sum_{a \in A_-} f_-(a).
\]
\end{proof}

If $f:A \to \mathbb{R}$ is a function, we define $|f|:A \to \mathbb{R}$ by $|f|(a)=|f(a)|$.

\begin{theorem}
If $f:A \to \mathbb{R}$ is a function, then $f$ is summable if and only if $|f|$ is summable.
\label{absolute}
\end{theorem}
\begin{proof}
Let $A_+=\{a \in A: f(a) \geq 0\}$ and $A_-=\{a \in A: f(a) \leq 0\}$, and let $f_+$ and $f_-$ be the restrictions of $f$ to $A_+$ and $A_-$
respectively. 
Suppose that $f$ is summable. Then by Lemma \ref{positivenegative} we get that $f_+$ is summable and $f_-$ is summable. 
Let $F \in \mathscr{P}_0(A)$ and write $F_+=\{a \in F: f(a) \geq 0\}$, $F_-=\{a \in F: f(a) \leq 0\}$. We have
\[
S_{|f|}(F) = \sum_{a \in F} |f(a)| = \sum_{a \in F_+} f(a) - \sum_{a \in F_-} f(a) = S_{f_+}(F_+) - S_{f_-}(F_-).
\]
But
by  Lemma \ref{bounded}, because the net $S_{f_+}$ is increasing we have
$S_{f_+}(F_+) \leq \sum_{a \in A_+} f_+(a)$, and because the net $-S_{f_-}$ is increasing we have
$-S_{f_-}(F_-) \leq - \sum_{a \in A_-}f_-(a)$.
Therefore, $ \sum_{a \in A_+} f_+(a)- \sum_{a \in A_-}f_-(a)$ is an upper bound for the range of $S_{|f|}$. 
Moreover, $S_{|f|}$ is increasing, so by Lemma \ref{bounded} it follows that $S_{|f|}$ converges, i.e. that $|f|$ is summable.

Suppose that $|f|$ is summable. By Lemma \ref{bounded}, for any $F \in \mathscr{P}_0(A_+)$ we have
\[
S_{f_+}(F) = S_{|f|}(F) \leq \sum_{a \in A} |f|(a),
\]
i.e., $\sum_{a \in A} |f|(a)$ is an upper bound for the range of $S_{f_+}$. As $S_{f_+}$ is increasing, by Lemma \ref{bounded} it follows
that $S_{f_+}$ converges, i.e., that $f_+$ is summable. Because $-S_{f_-}$ is increasing, we likewise get that $-S_{f_-}$ converges and hence
that $S_{f_-}$ converges, i.e. that $f_-$ is summable. Now applying Lemma \ref{positivenegative}, we get that $f$ is summable.
\end{proof}



\begin{theorem}
If $f:A \to \mathbb{R}$ is summable,
then $\{a \in A: f(a) \neq 0\}$ is countable.
\end{theorem}
\begin{proof}
Suppose by contradiction that $\{a \in A: f(a) \neq 0\}$ is uncountable. We have
\[
\{a \in A: f(a) \neq 0\}= \{a \in A: |f(a)|>0\} = \bigcup_{n \in \mathbb{N}} \left\{a \in A: |f(a)| \geq \frac{1}{n} \right\}.
\]
Since this is a countable union, there is some $n \in \mathbb{N}$ such that $\left\{ a \in A: |f(a)| \geq \frac{1}{n} \right\}$ is uncountable; in particular,
this set is infinite.
Because $f$ is summable, by Theorem \ref{absolute} we have that $|f|$ is summable, with unordered sum $\sigma$. Hence, there is 
some $F_1 \in \mathscr{P}_0(A)$ such that $F_1 \subseteq F \in \mathscr{P}_0(A)$ implies that $|S_{|f|}(F)-\sigma|<1$.
Let $F$ be a finite subset of $\left\{a \in A: |f(a)| \geq \frac{1}{n} \right\}$ with at least $n(\sigma+1)$ elements. Then
\[
S_{|f|}(F \cup F_1) = \sum_{a \in F \cup F_1} |f(a)| \geq \sum_{a \in F} |f(a)| \geq n(\sigma+1) \cdot \frac{1}{n}=\sigma+1.
\]
But $F_1 \subseteq F \cup F_1 \in \mathscr{P}_0(A)$, so $S_{|f|}(F \cup F_1) < \sigma+1$,  a contradiction. Therefore, 
$\{a \in A: f(a) \neq 0\}$ is countable. 
\end{proof}




\section{References}
McArthur \cite{MR0235336}

Schaefer \cite[p.~120]{schaefer}

Roytvarf \cite[p.~282]{roytvarf}

McShane \cite{mcshane}

Diestel, Jarchow and Tonge \cite{diestel}

Remmert \cite[p.~29]{remmert}

Sorenson \cite{MR2640720}

Lattice sums \cite{MR2807930}

Kadets and Kadets \cite{kadets}

Manning \cite{manning}

Bottazini \cite{bottazini}

Boyer \cite{boyer}

Weil \cite{weil}

Smithies \cite{MR863341}

Dugac \cite{dugac}

Grattan-Guinness \cite{MR0497690} and \cite{MR1263061} and \cite{gg} and \cite{bolzano}

Whiteside \cite{whiteside}

Schaefer \cite{schaefer1986}

Cauchy \cite{cours}

Polya \cite{polyaI}

Lakatos \cite{lakatos}

Krantz \cite{MR2026311}

Cunha \cite{cunha}

Youschkevitch \cite{youschkevitch}

Tweddle \cite{tweddle} and \cite{MR1996414}

Bromwich \cite[p.~74, Art. 28]{bromwich}

Laugwitz \cite{laugwitz1989}, \cite{laugwitz2000}, \cite{laugwitz2008}

Tucciarone \cite{tucciarone}

Fraser \cite{fraser}

Cowen \cite{MR602844}

Spence \cite{spence}

Jahnke \cite{jahnke}

Epple \cite{epple}

Mascr\'e \cite{mascre}

Rosenthal \cite{MR883287}

Freniche \cite{MR2663251}

Goursat \cite[p.~348]{goursat}

\section{Probability}
Baker \cite{baker}

Nathan \cite{nathan}

Nover and Harris \cite{nover}

Colyvan \cite{colyvan}

Liouville \cite[pp.~74--75]{lutzen}

Chrystal \cite[p.~118]{chrystal}

Jordan \cite[p.~277, Theorem 291]{jordan}

Cayley \cite[a]{cayley}

Harkness and Morley \cite[p.~66]{harkness}

Hofmann \cite{hofmann}

Ferreir\'os \cite{ferreiros}

Ferraro \cite{MR2368303} and \cite{ferraro} and \cite{leibniz}

Brouncker \cite{brouncker}

Roy \cite{roy}

Wallis \cite{wallis}

Bourbaki \cite[p.~261, chapter III, \S 5.1]{bourbaki}

Pringsheim \cite{pringsheim}

Dutka \cite{dutka}

Gr\"unbaum \cite{grunbaum}

Hinton and Martin \cite{achilles}

Gersonides \cite{gersonides}

Watling \cite{watling}

Moore \cite{moore}

Wojtaszczyk \cite[Chapter 7]{wojtaszczyk}

\bibliographystyle{amsplain}
\bibliography{summable}

\end{document}
