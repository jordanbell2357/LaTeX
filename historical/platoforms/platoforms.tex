\documentclass{article}
\usepackage{amsmath,amssymb,graphicx,subfig,mathrsfs,amsthm,enumitem}
\newtheorem{theorem}{Theorem}
\newtheorem{lemma}[theorem]{Lemma}
\newtheorem{proposition}[theorem]{Proposition}
\newtheorem{corollary}[theorem]{Corollary}
\theoremstyle{definition}
\newtheorem{definition}[theorem]{Definition}
\newtheorem{example}[theorem]{Example}
\begin{document}
\title{Plato's theory of forms and the axiom of foundation}
\author{Jordan Bell}

\date{January 27, 2017}

\maketitle


\section{Axiom of foundation}
By $y \subset x$ we mean that $z \in y$ implies that $z \in x$. 


If $x$ is a set, we say that an element $y$ of $x$ is \textbf{epsilon-minimal} if
$y \cap x = \emptyset$. 
The \textbf{axiom of foundation} states that if $x \neq \emptyset$ then $x$ has an epsilon-minimal element. 
See Jech  \cite[p.~63, Chapter 6]{jech}.

Zermelo \cite{zermelo1930}. Translated and glossed in \cite[pp.~1208--1233]{ewaldII}. Kanamori \cite{kanamori}

Ebbinghaus: \cite{ebbinghaus} 

Forster \cite{forster}

Von Neumann universe, cumulative hierarchy, $\epsilon$-induction, well-founded induction.

The following theorem shows in particular that $x \not \in x$. 


\begin{theorem}
For all  $x_1,\ldots,x_n$,
\[
\neg ((x_1 \in x_2) \wedge (x_2 \in x_3) \wedge \cdots \wedge (x_{n-1} \in x_n) \wedge (x_n \in x_1)).
\] 
\end{theorem}
\begin{proof}
Suppose that
\begin{equation}
(x_1 \in x_2) \wedge (x_2 \in x_3) \wedge \cdots \wedge (x_{n-1} \in x_n) \wedge (x_n \in  x_1).
\label{cyclic}
\end{equation}
Let $x=\{x_1,\ldots,x_n\}$. The axiom of foundation asserts that $x$ has an epsilon-minimal element, say $x_i$, for which
$x_i \cap x = \emptyset$. This means that for each $j=1,\ldots,n$ we have $x_j \not \in x_i$. If $1<i \leq n$, then with
$j=i-1$ we have $x_{i-1} \not \in x_i$, contradicting \eqref{cyclic}. If $i=1$, then with $j=n$ we have $x_n \not \in x_1$, contradicting
\eqref{cyclic}. Therefore \eqref{cyclic} is false, proving the claim.
\end{proof}



\begin{theorem}
There is no function $f$ with domain $\omega$ such that for each $i \in \omega$, 
$f(i+1) \in f(i)$.
\end{theorem}
\begin{proof}
Suppose there is such a function $f$. 
Let $x = \{f(i): i \in \omega\}$. 
The axiom of foundation asserts that $x$ has an epsilon-minimal element, say $f(i)$, for which $f(i) \cap x = \emptyset$. 
This means that for each $j \in \omega$ we have $f(j) \not \in f(i)$. In particular, with $j=i+1$ we have
$f(i+1) \not \in f(i)$, a contradiction. 
\end{proof}

Kunen \cite{kunen}


\section{Ordinal numbers}
Suppose that $<$ is a linear order on a set $X$. A subset $I$ of $X$ is said to be an \textbf{initial segment of $X$}
if $t \in I$, $x \in X$ and $x<t$ imply that $x \in I$. For $x \in X$ we define
\[
s_X(x) = \{t \in X: t < x\}.
\] 
If $t \in s_X(x)$, $y \in X$ and $y<t$, then from $t<x$ and $y<t$ we have $y<x$ and hence
$y \in s_X(x)$. Therefore $s_X(x)$ is an initial segment of $X$.

\begin{theorem}
If $<$ is a well-order on a set $X$ and $I$ is an initial segment of $X$ with $I \neq X$, then there is some $x \in X$ such that
$I=s_X(x)$.
\end{theorem}
\begin{proof}
Because $I \neq X$,
$X \setminus I$ is nonempty, and because $<$ is a well-order, there is a minimum element $x$ of $X \setminus I$. On the one hand, suppose that
$t \in I$. If $x \leq t$ then because $I$ is an initial segment and $t \in I$ we get $x \in I$, contradicting that $x \in X \setminus I$.
Therefore $t<x$, i.e. $t \in s_X(x)$, showing that $I \subset s_X(x)$. 
On the other hand, suppose that $t \in s_X(x)$. That is, $t < x$.
If $t \in X \setminus I$ then because $x$ is a minimum element of $X \setminus I$ we have 
$x \leq t$, contradicting $t<x$. Therefore $t \in I$, showing that $s_X(x) \subset I$.
\end{proof}

A set $x$ is said to be \textbf{transitive} if $y \in x$ and $z \in y$ imply that $z \in x$. 
A set $\alpha$ is called an \textbf{ordinal number} if $\alpha$ is transitive and $\in$ is a well-order on $\alpha$. 
We denote the class of all ordinal numbers by 
$\mathbf{Ord}$. 

Jech \cite[p.~19, Lemma 2.11]{jech}

\begin{theorem}
\begin{enumerate}
\item $\emptyset \in \mathbf{Ord}$.
\item If $\alpha \in \mathbf{Ord}$ and $\beta \in \alpha$ then $\beta \in \mathbf{Ord}$.
\item If $\alpha,\beta \in \mathbf{Ord}$, $\alpha \subset \beta$, and $\alpha \neq \beta$, then
$\alpha \in \beta$.
\item If $\alpha,\beta \in \mathbf{Ord}$ then $\alpha \subset \beta$ or $\beta \subset \alpha$.
\end{enumerate}
\end{theorem}
\begin{proof}
It is immediate that $\emptyset \in \mathbf{Ord}$. Suppose that $\alpha \in \mathbf{Ord}$ and $\beta \in \alpha$. 
Suppose that $\gamma \in \beta$ and $\delta \in \gamma$. Because $\alpha$ is transitive, $\beta \in \alpha$ and
$\gamma \in \beta$ yield $\gamma \in \alpha$. Again, because $\alpha$ is transitive, $\delta \in \gamma$ and
$\gamma \in \alpha$ yield $\delta \in \alpha$. 
Let $x=\{\beta,\delta\}$. Because $\in$ is a well-order on $\alpha$ and $x \subset \alpha$, $x$ has a minimum element:
(i) $\beta \in \delta$, (ii) $\beta=\delta$, or (iii) $\delta \in \beta$. 
For (i), $\beta \in \delta \in \gamma \in \beta$, which by Theorem
\ref{cyclic} contradicts the axiom of foundation. For (ii), $\gamma \in \beta = \delta \in \gamma$, 
and likewise by Theorem
\ref{cyclic} this contradicts the axiom of foundation. Therefore (iii) $\delta \in \beta$, which shows that $\beta$ is transitive.
Because $\beta$ is a subset of $\alpha$ and $\epsilon$ is a well-order on $\alpha$, it is immediate that $\in$ is a well-order on $\beta$. Therefore
$\beta \in \mathbf{Ord}$.

Suppose that $\alpha,\beta \in \mathbf{Ord}$, $\beta \subset \alpha$, and $\beta \neq \alpha$. 
The subset $\alpha \setminus \beta$ of $\alpha$ is nonempty, and because $\epsilon$ is a well-order on $\alpha$
there is a minimum element $\gamma$ of $\alpha \setminus \beta$. Because $\alpha$ is transitive and
$\gamma$ is an element of $\alpha$ we have $\gamma \subset \alpha$. 
Also, because $\gamma$ is a minimum element of $\alpha \setminus \beta$, for any $\delta \in \alpha \setminus \beta$
we have $\delta \not \in \gamma$. Hence $(\alpha \setminus \beta) \cap \gamma = \emptyset$, which
is equivalent to $(\alpha \cap \gamma) \setminus \beta = \emptyset$, and because $\gamma \subset \alpha$ this implies
$\gamma \setminus \beta = \emptyset$. And $\gamma \setminus \beta = \emptyset$ means
that
$\gamma \subset \beta$. 
\end{proof}


\section{Plato}
{\em First Alcibiades}

{\em Laches}: ``What is courage?'' 191e: ``So what actually is each of these attributes, courage and cowardice?
That's what I wanted to find out. Let's take courage first, then, and could you please try again to tell me what it is
that's the same in all these situations.'' \cite[p.~53]{waterfieldmeno}

{\em Charmides}: ``What is self-control?''
Critias says at 166e, ``I maintain that it's the only kind of knowledge which knows itself and all other cases of knowledge.'' \cite[p.~21]{waterfieldmeno}
Socrates soon responds in 167c, ``But we're committing ourselves here to an extraordinary assertion, my friend. If you try to find the same phenomenon elsewhere,
you'll see how impossible it is, I think.'' For example, 167d--e, ``Is there a kind of hearing that doesn't hear any sound, but hears itself and all other cases of hearing
and not hearing?'', and ``And would you say that there was a kind of love which is such that it isn't love of anything beautiful, but of itself and all other cases of loving?''


{\em Protagoras}, 330--331 \cite[pp.~27--28]{taylorprotagoras}:
\begin{quote}
``So then,'' I said, ``none of the other parts of excellence is like knowledge, none is like justice, none like courage, none like soundness of mind, and none like
holiness.''

``No.''

``Well now,'' I said, ``let's consider together what sort of thing each one
is. Here's the first question: is justice something, or not a thing at all? It seems to me that it is something; what do you think?''.

``I think so too.''

``Well, then, suppose someone asked us, `Tell me, is that thing that you have just mentioned,
justice, itself just or unjust?' I should reply that it is just. How would you cast your vote? The same as mine, or different?''

``The same.''

``So my reply to the question would be that justice is such as to be just; would you give the same answer?''

``Yes.''

``Suppose he went on to ask us, `Do you think that there is also such a thing as holiness?' we whould, I think, say that we do.''

``Yes.''

`` `And do you say that that too is something?' We should say so, don't you agree?''

``I agree there too.''

`` `And do you say that this thing is itself such as to be unholy, or such as to be holy?' I should be annoyed at the question, and say, `Watch what you say,
sir; how could anything else be holy, if holiness itself is not to be holy?' What about you? Wouldn't you give the same answer?''

``Certainly,'' he said.

``Suppose he carried on with his questioning: `Well, what was it that you were saying a moment ago. Didn't I hear you correctly? You seemed to me to be saying that the 
parts of excellence are related to one another in such a way that none of them is like any other.' I should say, `Yes, you heard the rest correctly, but you must have misheard if you think
that I said that. It was Protagoras who said it in answer to a question of mine.' Suppose he said, `Is that right, Protagoras? Do you say that none of the parts of excellence
is like any of the others? Is that your opinion?' What would you say?''

``I think have to agree, Socrates,'' he said.

``Well, once we've agreed to that, Protagoras, how shall we deal with his next question? `So holiness is not such as to be something just, nor justice such as to be holy,
but rather such as to be not holy; and holiness such as to be not just, and so unjust, and justice unholy?' What shall we reply? For my own part I should say both that
justice is holy and holiness just; and, if you let me, I should give the same answer on your behalf too, that justness is either the same thing as holiness or very similar,
and above all that justice is like holiness and holiness like justice. Is that your view too, or had you rather that I didn't give that answer?''
\end{quote}

332c \cite[p.~30]{taylorprotagoras}:
\begin{quote}
``Well now,'' I said, ``is there such a thing as the beautiful?''

``Yes, there is.''

``And does it have any opposite except the ugly?''

``No, none.''

``Is there such a thing as the good?''

``There is.''

``Does it have any opposite apart from the bad?''

``No.''

``Is there such a thing as the high-pitched in sound?''

``Yes.''

``And that has no opposite apart from the low-pitched?''

``None.''

``So,'' I said, ``each member of an opposition has only one opposite, not many.''
\end{quote}

349b \cite[p.~50]{taylorprotagoras}:
\begin{quote}
``The question, I think, was this: are `wisdom', `soundness of mind', `courage', `justice', and `holiness' five names for the one thing, or does there correspond to
each of these names some separate thing or entity with its own particular power, unlike any of the others?''
\end{quote}


{\em Hippias Major}.  292e, 300a--b, 303: two different beautiful things have something identical which makes them beautiful and this
common thing is present in them.

{\em Gorgias}: good, 497e.

{\em Meno}. In this  dialogue Socrates asks ``what is excellence?''  72: Socrates: ``Suppose I asked what it is to be a bee and you said that there were many
bees, of many varieties. What answer would you give me if I then asked: `Are you saying that these bees are many,
and of many different varieties, in that they are bees? Or do they not differ from one another at all in that they are bees, but differ from
one another in some other respect -- in beauty, for example, or size or something like that?' Tell me: what answer would you give to this
question?'' Meno: ``I'd say that they don't differ from another at all, in so far as they are bees.'' Socrates: ``So what if I went
on to say: `Here's the crucial question, then, Meno: what, in your opinion, is it that makes them all no different from one another, but 
the same?' I imagine you'd be able to tell me, wouldn't you?'' Meno: ``Yes, I would.'' Socrates: ``So do the same
for excellence as well, please. Even if there are many aspects of that excellence, of different kinds, they all share a single
characteristic, thanks to which they are aspects of excellence, and it's this single characteristic which a person should look to when he's replying
to someone who has asked him to explain what excellence actually is. But perhaps you're not following what I'm saying.'' Meno: ``No,
I think I understand, but I'm not quite as clear as I'd like to be about the point of the present inquiry.''
Socrates: ``Well, Meno, is it only excellence that seems to you to be like that -- to be different for a man and for a woman and so on
-- or does the same go, in your opinion, for health and height and strength? Do you think that health is different in a man and in a woman? Or is
it the same characteristic wherever there's health, whether it's in a man or a woman or anything else?'' Meno: ``As far as health is
concerned, I think it's the same for a man and for a woman.'' Socrates: ``And height and strength, too? If a woman is strong, will she
be strong thanks to the same characteristic, the same strength? By `the same' I mean that, in so far as it's strength, strength is no different
whether it's in a man or in a woman. Or do you think there's a difference?''
Socrates then gets Meno to agree in 73 that everyone's excellence is the same. \cite[pp.~101--103]{waterfieldmeno}



{\em Euthyphro} \cite{euthyphro} 5--7: ``Isn't the holy itself the same as itself in every action? And conversely, isn't the unholy the exact opposite of the holy, in itself
similar to itself, or possessed of a single character, in anything at all that is going to be unholy?'' Maybe it makes sense to read this as saying that the only predicate satisfied by
``the holy'' is ``$x$ is holy'', in other words, ``the holy'' does not belong to any class of things except ``the holy''. ``And do you recall that I wasn't urging you to teach me about one or two
of those many things that are holy, but rather about the form itself whereby all holy things are holy? Because you said, I think, that it was by virtue of a single character that unholy
things are unholy, and holy things are holy. Don't you remember?'' ``Then teach me about that character, about what it might be, so that by fixing my eye upon it and using it as a model, I may call
holy any action of yours or another's, which conforms to it, and may den to be holy whatever does not.'' 12, 13: shame is contained in fear, the even numbers are contained
in the numbers.

{\em Symposium} 210--212. 210e: beauty is beautiful

{\em Phaedrus}. 247: When the gods go to one of their banquets, they ``journey skyward to the rim of the heavenly vault.'' Then,
``When the souls we call `immortal' reach the rim, they make their way to the outside and stand on the outer edge of heaven,
and as they stand there the revolution carries them around, while they gaze outward from the heaven. The region beyond heaven has never
yet been adequately described in any of our earthly poets' compositions, nor will it ever be. But since one has to make a courageous
attempt to speak the truth, especially when it is truth that one is speaking about, here is a description. This region is filled with true being.
True being has no colour or form; it is intangible, and visible only to intelligence, the soul's guide. True being is the province
of everything that counts as true knowledge. So since the mind of god is nourished by intelligence and pure knowledge
(as is the mind of every soul which is concerned to receive its proper food), it is pleased to be at last in a position to see true being,
and in gazing on the truth it is fed and feels comfortable, until the revolution carries it around to the same place again. In the course of its
circuit it observes justice as it really is, self-control, knowledge -- not the kind of knowledge that is involved with change and differs
according to  which of the various existing things (to use the term `existence' in its everyday sense) it makes its object, but the kind of knowledge
whose object is things as they really are. And once it has feasted its gaze in the same way on everything else that really is, it sinks
back into the inside of heaven and returns home. Once back home, the soul's charioteer reins in his horses by their manger,
throws them ambrosia to eat, and gives them nectar to wash the ambrosia down.'' \cite[pp.~29--30]{waterfieldphaedrus}

In Socrates' palinode, 249: ``For a soul which has never seen the truth cannot enter into human form, because a man must understand the impressions he receives
by reference to classes: he draws on the plurality of perceptions to combine them by reasoning into a single class. This is recollection of the things which our souls once
saw during their journey as companions to a god, when they saw beyond the things we now say `exist' and poked their heads up into true reality.'' \cite[p.~32]{waterfieldphaedrus}


{\em Republic}:
``That since beauty and ugliness are opposite, they are two things; and consequently each of them is one. The same holds
of justice and injustice, good and bad, and all the essential Forms: each in itself is one; but they manifest themselves in a great variety
of combinations, with actions, with material things, and with one another, and so each seems to be
many.''\cite[p.~183, 475--476]{cornfordrepublic}

``On these assumptions, then, I shall call for an answer from our friend who denies the existence of Beauty itself
or of anything that can be called an essential Form of Beauty remaining unchangeably in the same state for ever, though he does
recognize the existence of beautiful things as a plurality-- that lover of things seen who will not listen to anyone who says that Beauty
is one, Justice is one, and so on. I shall say to him, Be so good as to tell us: of all these many beautiful things is there one which will
not appear ugly? Or of these many  just or righteous actions, is there one that will not appear unjust or unrighteous?'' ``No, replied
Glaucon, they must inevitably appear to be in some way both beautiful and ugly; and so with all the other terms your question
refers to.'' \cite[p.~187, 478--479]{cornfordrepublic} This seems to deny that it is possible for a Form to participate in itself,
unless it is not counted as a ``thing''.

 ``Let me remind you of the distinction we drew earlier and have often drawn on other
occasions, between the multiplicity of things that we call good or beautiful or whatever it may be and, on the other hand,
Goodness itself or Beauty itself and so on. Corresponding to each of these sets of many things, we postulate a single Form
or real essence, as we call it.'' ``Further, the many things, we say, can be seen, but are not objects of rational thought; whereas
the Forms are objects of thought, but invisible.'' \cite[pp.~217--218, 506--507]{cornfordrepublic}

Knowledge is about Forms: ``never making use of any sensible object, but only of Forms, moving through
Forms from one to another, and ending with Forms.'' \cite[p.~226, 511--512]{cornfordrepublic}.

The Form of Bed is more real than any given bed: ``If so, what he makes is not the reality, but only something that resembles it. It would
not be right to call the work of a carpenter or of any other handicraftsman a perfectly real thing, would it?'' \cite[p.~326, 596--597]{cornfordrepublic}

{\em Phaedo} 65d: there is a thing that is ``the just itself'' and there is a thing that is ``the beautiful itself'' and there is a thing that is ``the good itself'', and these are
known most closely by thought alone. 78. 100c: beauty is beautiful. 73--80, 109--111.

{\em Cratylus}: 389--390: a craftsman uses an archetype. 439--440, knowing forms. The Form of a weaver's shuttle. 439c, there is a ``beautiful itself'' and a ``good itself''. 

{\em Euthydemus} 301b

{\em Parmenides} \cite{cornfordparmenides} 129--135. Alexander of Aphrodisias see \cite[p.~88]{cornfordparmenides} on ``man walks''.

{\em Theaetetus} \cite[p.~162]{cornfordtheaetetus}: 184--186. 208: ``image of thought in speech'', go through element by element,
``being able to state some mark by which the thing one is asked for differs from everything else''

{\em Sophist} \cite{cornfordtheaetetus} 246--248, 251--259.

{\em Philebus} 14--18. 

{\em Timaeus} \cite{cornfordtimaeus} 27d-28a  \cite[pp.~21--22]{cornfordtimaeus}, 51a--52a \cite[pp.~188--193]{cornfordtimaeus}

{\em Lysis} 217d.

{\em Seventh Letter} 341--345








\section{Testimonia}
Antisthenes: ``Plato, I see a horse; I don't see horseness.'' Plato: ``That's because you have the eye
that sees horses, but you haven't acquired the one that's needed to think about horseness.'' \cite[p.~296, 11.6]{socratics}.

Diogenes the Cynic: ``When Plato was once talking about hs Ideas and used the terms `tableness' and `cupness', Diogenes remarked, `I can see a table and a cup, but in no way
this tableness and cupness.' `Of course not,' replied Plato, `because you have the eyes that are needed to see a cup and table, but lack the intellect through which tableness
and cupness can alone be beheld.'' \cite[pp.~32--33, no. 121]{cynic}.


Plutarch, {\em Platonic Questions}, Question III, 1001E--1002A \cite[pp.~37--39]{LCL427}:

\begin{quote}
Moreover, just as each of the perceptibles themselves
has a multiplicity of semblances and shadows
and images and as generally both in nature and in
art it is possible for numerous copies to come from a
single pattern, so the things of this world must surpass
in number the things of that world according to
Plato's supposition that the intelligibles are patterns,
that is ideas, of which the perceptibles are as semblances
or reflections. Moreover, the ideas are the
objects of intellection {$\langle$}; and intellection{$\rangle$} he introduces
as a result of abstraction or lopping away
of body when in the order of studies he leads down
from arithmetic to geometry and then after this to
astronomy and crowns all with the theory of harmony,
for the objects of geometry are the result
when quantity has taken on extension, the solids
when extension has taken on depth, the objects of
astronomy when solid body has taken on motion, 
and the objects of harmonics when sound has been
added to the body in motion. Hence by abstracting
sound from the things in motion and motion from the
solids and depth from the planes and extension from
the quantities we shall arrive at intelligible ideas
themselves, which do not differ from one another
at all when conceived in respect of their singularity 
and unity.
\end{quote}

Eusebius, {\em Praeparatio Evangelica}, 11.23.3--4, ``On the Ideas in Plato''.

Aristotle, {\em Metaphysics}

Aristotle, {\em Eudemian Ethics} I.8


\section{Philosophy}
Klein \cite{klein}

Coxon \cite{coxon}

Moore \cite{moore}

Kneale and Kneale \cite{kneale}

Graeser \cite{graeser}

Cherniss \cite{cherniss}

Stillwell \cite{stillwell}

Grube \cite{grube}

Zalta \cite{zalta}

Edwards \cite{edwards}

Heinaman \cite{heinaman}

Barnes \cite{barnes} and \cite{barnes2009}

Clegg \cite{clegg}

Meinwald \cite{meinwald}

Ellerman \cite{ellerman}

Horky \cite{horky}

Allen \cite{allen1960}

Black \cite{black}

Ross \cite{ross}

Cohen \cite{cohen}

Dauben \cite{dauben}

Malcolm \cite{malcolm}

Nehamas \cite{nehamas} and \cite{nehamas1982}

Plato self-predication: Vlastos \cite{vlastos1981}

Rickless \cite{rickless}

Sedley \cite{sedley}

Sayre \cite{sayre}

Mates \cite{mates}

Harte \cite{harte}

Taylor \cite{taylor}

Russell \cite{russell}

Kr\"amer \cite{kramer}


\bibliographystyle{plain}
\bibliography{platoforms}

\end{document}