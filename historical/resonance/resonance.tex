\documentclass{amsart}
\usepackage{amsmath,amssymb,graphicx,subfig,mathrsfs}
\usepackage{hyperref}
\newcommand{\norm}[1]{\left\Vert #1 \right\Vert}
\newtheorem{theorem}{Theorem}
\newtheorem{lemma}[theorem]{Lemma}
\newtheorem{corollary}[theorem]{Corollary}
\begin{document}
\title{Bibliography for the history of resonance}
\author{Jordan Bell}
\email{jordan.bell@gmail.com}
\address{Department of Mathematics, University of Toronto, Toronto, Ontario, Canada}
\date{\today}

\maketitle

Truesdell on Leonardo \cite[pp.~18--20]{truesdell1968I}: Leonardo
first to use a ``light rider'' to demonstrate sympathetic vibration 
(MS A 22v.)

Truesdell \cite[p.~108]{truesdell1968II}

Zubov \cite[p.~88]{zubov}: Paris Manuscript A, 22v; MacCurdy 267

Philoponus \cite[pp.~46, 135]{lacey}

Plotinus \cite[p.~155]{sorabji}

Grosseteste's commentary on Aristotle's {\em Posterior Analytics}, book 2, chapter 4.

Chinese \cite[pp.~14,15,1814]{selin}

Bibliotheca mathematica 3. Folge vol. 4, p. 378; 3. Folge vol 6, pp.~32, 48, 42,
22, 33, 50; vol. 7, pp. 148, 152; 3. Folge vol. 9, p. 349; 3. Folge vol. 12,
p. 240

Grendler \cite[p.~11]{grendler}

Thorndike \cite[p.~600]{thorndike}

Pohl and Deans \cite[p.~259]{pohl}

Chapman \cite[Chapter 10]{chapman}

Whewell \cite[p.~297]{whewell}

Finlay-Fruendlich \cite[pp.~95, 117]{fruendlich}

Barker \cite[p.~116]{barker}

Folta \cite[p.~103]{folta}

Handbuch der Physik, Festk\"orpermechanik I, Volume 1, p.~156

Francis Bacon \cite[pp.~141--152]{bacon}

Commercium p.~243

Bibliotheca Mathematica, p.~240, 1912

Euler on tides E57 \cite[pp.~300--304]{E57}

Courant \cite[p.~514]{courant}

Hargreave \cite[p.~102]{hargreave}

Olenick \cite[p.~400]{olenick}

Sambursky \cite[pp.~9, 41--42]{sambursky1959}, and on Philoponus and
Theon of Smyrna \cite[pp.~100--104]{sambursky1962}

Newton's notebooks \cite[p.~310]{newton}

Truesdell \cite[pp.~22, 170--178]{rational}

Kassler \cite[pp.~53, 57]{kassler}

Whiteside \cite[p.~335]{newtonVI}

Commercium \cite[pp.~54, 58, 304, 305, 695]{commerciumII}

Commentationes mechanicae ad theoriam corporum fluidorum pertinentes 2nd part, p. LXII

Louise Diehl Patterson, {\em Hooke's Analysis of Simple Harmonic Motion}

Zeidler \cite[\S 5.9]{zeidler}

Lynn White \cite[pp.~126--127]{white}

Schaffer \cite[p.~157]{schaffer}

Greenberg \cite[p.~548]{greenberg}

A history of science and technology, Volume 2 p. 368, Robert James Forbes, Eduard Jan Dijksterhuis

Resonance in watches, p. 325 vol. 146 No. 9 September 2004, Horological Journal

The application of the pendulum to timekeeping (Huygens, 1656-57) gave us for the first time an oscillating controller with its own natural frequency. (The verge-and-foliot mechanism of the early clocks oscillated at a frequency that was in large part a function of the driving force, which has implications for perturbation and irregularity.)

Mahoney \cite[p.~303]{mahoney}

Cross \cite[p.~227]{cross}

G. W. Krafft, {\em Observatio eclipseos solaris d. 25 Iulii 1748 Tubingae facta}, Novi Commentarii, tom. I, among his instruments was a horologium portatile Londinense

Commercium \cite[p.~77]{commerciumVI}

R. 2642, Letter 122, Teplov

Euler, Opera omnia, Vol. II, p. 54, 58

Euler to Lambert letter, R. 1408, p. 243 of Index

Hund \cite[p.~170]{hund}

Todhunter \cite[p.~39]{todhunter}

Mach \cite[p.~272]{mach}

Sommerfeld \cite{sommerfeld}

Truesdell \cite[p.~309]{truesdell1984} writing about the Euler-Daniel Bernoulli correspondence states that it is unclear from the summaries of the letters whether Bernoulli
understood Euler's discovery of resonance. Truesdell \cite[p.~323]{truesdell1984}  in his review of Opera omnia II.10-11, states that E126 contains the first analysis of a single
harmonically driven oscillator.

Truesdell on moment of momentum \cite[pp.~239--271]{truesdell1968V}, ``Whence the law of moment of momentum?''

Steele \cite[p.~349]{steele}

Euler and modern science, p. 228, 226, 171

Newton Principia, Section VII, Book II, Proposition XXXVIII, Theorem XII

Die Werke Von Johann I Und Nicolaus II Bernoulli, p. 8

Proc\'es-verbaux des s\'eances de l'Acad\'emie imp\'eriale des sciences depuis sa fondation jusqu'\`a 1803, Tome I, p. 522, 554

Pesic \cite[p.~22]{pesic2014}

Kaye \cite[p.~287]{kaye} on Jean de Jandun's {\em Tractatus de laudibus Parisius}

\nocite{*}

\bibliographystyle{amsplain}
\bibliography{resonance}

\end{document}
