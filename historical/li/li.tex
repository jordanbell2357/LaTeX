\documentclass{article}
\usepackage{amsmath,amssymb}
\newcommand{\li}{\mathrm{li}}
\newcommand{\Li}{\mathrm{Li}}
\newcommand{\Ei}{\mathrm{Ei}}
\begin{document}
\title{The logarithmic integral}
\author{Jordan Bell}
\date{August 17, 2016}
\maketitle

It is an important mathematical object in the theory of
prime numbers and its use in number theory seems to first arise with Gauss.
But it is also
one of the first transcendental functions one runs into after the
trigonometric and logarithmic functions: having classified the
trigonometric and logarithmic functions as known, we then take
integrals involving them and want to know whether those can be
expressed as a ``closed expression'' involving just them. If we take the
integral of $\log(t)$ from 1 to $x$ we find that it is equal to
$x\log(x)-x$,
while if we take the integral of $1/\log(t)$ say from 0 to $x$ we are not
able to find any expression for it, and we may be led to call it
$\li(x)$.

There is no paper in the literature that gives the history of the introduction
of the logarithmic integral to analysis. Indeed it's well known that Gauss
conjectured the prime number theorem which is stated in terms of the logarithmic
integral, but what were the first publications in which the logarithmic integral
appeared? What it a known object of analysis when Gauss made his conjecture?
When I was reading on the history of the prime number theorem this is a
question to which I couldn't find a single paper that gave a reliable answer.


The logarithmic integral is defined as
\[
\li(x)=\lim_{\epsilon \to 0} \Big( \int_0^{1-\epsilon} \frac{dt}{\log t}
+\int_{1+\epsilon}^x \frac{dt}{\log t}\Big).
\]

The exponential integral is defined as
\[
\Ei(x)=\lim_{\epsilon \to 0} \Big( \int_{-\infty}^{-\epsilon}
\frac{e^{-t}}{t} dt + \int_{\epsilon}^x \frac{e^{-t}}{t} dt\Big).
\]

\section{Why is $\sin x$ an elementary function and $\li x$ isn't?}
N. N. Lebedev, {\em Special functions and their applications}, 1972.

R\"udiger Thiele, {\em What is a function?}

D. T. Whiteside, {\em Patterns of Mathematical Thought 
in the later Seventeenth Century}

Computation of values of functions in tables. History of mathematical tables.

Cajori on notations for functions.

{\em Companion encyclopedia of the history and philosophy of the Mathematical Sciences}, volume 1, sect. 4.4.

{\em Encyclopaedia Britannica}, Thomas Spencer Baynes, p. 39, ``function sui generis''. I don't remember which edition.

Cayley's review of J. W. L. Glaisher's {\em Tables of the Numerical Values of the 
Sine-integral, Cosine-integral, and Exponential Integral}, p. 262 in the
Proceedings of the Royal Society of London, From June 17, 1869 to June 16, 1870,
vol. XVIII, 1870.

Are some functions more transcendental than others? For example, is
some unclassified power series more transcendental than the power
series for $\sin(x)$? What about Bessel functions?

\section{Euler and his contemporaries}
E421, E464, E475, E500, E521, E583, E620, E621, E629, E662.

Institutiones calculi integralis

Pietro Ferroni,
{\em Magnitudinum exponentialium Logarithmorum},
1782.

P. Mako S. J., 1768,
Calculi differentialis et integralis institutio,
p. 149

Lorenzo Mascheroni,
{\em Adnotationes ad calculum integralem Euleri},
1790, pp. 42ff.

P. Mako, S.J.,
{\em Calculi differentialis et integralis institutio}, 1768,
p. 149.

Silvestre Fran{\c c}ois Lacroix,
{\em An elementary treatise on the differential and integral calculus},
(translated from the French),
1816, p. 239.

Other books that may discuss the logarithmic integral:
Disqvisitiones analyticae maxime ad calcvlvm integralem et doctrinam ...
 By Johann Friedrich Pfaff, 
Principiorum calculi differentialis et integralis expositio elementaris
 By Simon Antoine J. L'Huilier, 

\section{Gauss and the prime number theorem}
{\em Gauss Werke}, Band 8, pp. 90--92.

See Ingham, {\em The distribution of prime numbers}

Charles James Hargreave,
{\em Analytical Researches concerning Numbers},
The London, Edinburgh and Dublin Philosophical Magazine and Journal of Science,
Third Series, Vol. 35, No. 233, July 1849, 
p. 45.

Factor table for the sixth million: containing the least factor of every ... By James Glaisher, Introduction, p. 93

Augustus De Morgan,
Library of Useful Knowledge,
{\em The Differential and Integral Calculus},
Society for the Diffusion of Useful Knowledge, Baldwin and Cradock,
London, 1842, p. 660.

Edmund Landau,
{\em Der Integgrallogarithmus und die Zahlentheorie},
Rend. Circ. Matem. Palermo, t. XXIII, 1907, p. 126

\section{Later authors in the mid 19th century century}
Charles Brooke,
{\em A synopsis of the principal formulae and results of pure mathematics},
1829,
p. 224. Cites LCD 427-37, Tr. L. 181-91, Hrsch, Int. Tab.
LCD=Lacroix, Traitè du Calcul Differentiel,
Tr. L.=Translation of Lacroix,
Hirsch=Meyer Hirsch Integral tables.

T. G. Hall,
{\em Treatise on the Differential and Integral Calculus},
1837, p. 338.

{\em Ausz\"uge aus einigen Briefen an der Professor Gilbert, aus mehreren Schreiben
des Prof. Soldner zu M\"unchen},
Annalen der Physik, Neue Folge, Neunter Band, 1811, (old series
Neun und Dreissigster Band), 
p. 239.

Johann Georg von Soldner,
{\em Theorie et tables d'une nouvelle fonction transcendante},
1809, Lindauer, M\"unchen, p. 6.

Andreas von Ettingshausen, {\em Vorlesungen \"uber die h\"ohere Mathematik},
Erster Band,
Carl Gerold, Wien, 1827, p. 365.

J. J. Littrow,
{\em Anleitung zur h\"oheren Mathematik},
Carl Gerold, Wien, 1836, p. 301.

R. Beez,
{\em Beitr\"age zur Theorie des Integrallogarithmus},
pp. 419--441, Archiv der Mathematik und Physik, Neunzehnter Theil, 1852.

Rudolf Engelmann (ed.),
{\em Abhandlungen von Friedrich Wilhelm Bessel},
Zweiter Band, Wilhelm Engelmann, Leipzig, 1876.
Several contributions, starting p. 326.

Johann August Grunert, {\em Mathematisches W\"orterbuch oder Erkl\"arung der
Begriffe,
Lehrst\"atze, Aufgaben und Methoden der Mathematik}, Erste Abtheilung,
F\"unfter Theil, Erster Band, E. B. Schwickert, Leipzig, 1831, p. 138.

Johann August Grunert,
{\em Elemente der Differential- und Integralrechnung zum Gebrauche
bei Vorlesungen}, Zweiter Theil, E. B. Schwickert, Leipzig, 1837,
p. 126

Oskar Schl\"omilch, {\em Beitr\"age zur Theorie bestimmter Integrale},
Friedrich Frommann, Jena, 1843, p. 70.

Oskar Schl\"omilch, {\em Zur Theorie des Integrallogarithmus},
Archiv der Mathematik und Physik,
Neunter Theil, 1847, p. 5 and 307.

Hardy, {\em Divergent series}, p. 40.

Ferdinand Minding (ed.), {\em Handbuch der differential- und Integralrechnung
und ihrer Anwendungen auf Geometrie zun\"achst zum Gebrauche in Vorlesungen},
F. D\"ummler, Berlin, 1836, p. 100.

Car. Ant. Bretschneider, {\em Theoriae logarithmi integralis lineamenta nova},
p. 257,
Journal f\"ur die reine und angewandte Mathematik, 
Siebenzehnter Band, 1837.

\section{Liouville's theorem on integration in terms of elementary functions}
Liouville's theorem in differential algebra.

Manuel Bronstein, {\em Symbolic integration I: transcendental functions}.

Brian Conrad, {\em Impossibility theorems for elementary integration}

\section{Later history}
Niels Nielsen, {\em Theorie des Integrallogarithmus und verwandter
Transzendenten}, B. G. Teubner, Leipzig, 1906.

Detlef Laugwitz, {\em Bernhard Riemann 1826-1866: turning points in the
conception of mathematics}.

Jos. E. Hofmann, {\em Gesichte der Mathematik}, p. 59.

Friedrich L. Bauer, {\em Why Legendre made a wrong guess about $\pi(x)$, and
how Laguerre's continued fraction for the logarithmic integral improved it},
Math. Intelligencer, volume 25, number 3, 2003, pp. 7-11.

Julian Havil, {\em Gamma: exploring Euler's constant}, p. 106.

Bromwich,
{\em An introduction to the theory of infinite series}
p. 334, Ch. XXI, sect. 109.

Whittaker and Watson, {\em A course of modern analysis}, p. 341.

G. H. Hardy,
{\em The integration of functions of a single variable},
2nd ed., Cambridge University Press, 1928.

\end{document}
