% trylinearb.tex    Linear B font
\documentclass{article}
%%\documentclass[12pt]{article}

\title{Ancient balance scales}
\author{Jordan Bell}
\date{October 5, 2016}

\renewcommand{\baselinestretch}{1.2}
\begin{document}
\maketitle

Agade period, No.~42, Plate VI \cite[p.~46]{OIP47}:

\begin{quote}
Marble. Cylinder seal. $20 \times 13$ mm. Mrs. Moore's No.~108. Akkadian.

Shamash, seated before an altar, holds the saw; from his shoulders issue the usual 
sunrays. Before him stands a person holding in his raised left hand a pair of scales. This 
person is followed by another, who carries a goat as an offering. A smaller person behind
the others holds a knife(?) over an altar(?), perhaps waiting to perform a sacrifice. At
the end of the scene are two badly worn cuneiform signs.

To the best of the author's knowledge, this is the only representation of the balance with scalepans in
Mesopotamian glyptic.
\end{quote}

R. M. Boehmer 1965, Die Entwicklung der Glyptik w\"ahrend der
Akkad-Zeit, de Gruyter, n. 1105

``The Ada Small Moore Collection of Ancient Near Eastern Seals'', Sotheyby's December 12, 1991

Doyen, Charles. ``An Athenian Decree Revisited.'' CHS Research Bulletin 4, no.1 (2015).

A History of Trust in Ancient Greece, Johnstone, p.~55

Isaiah 40:12--15.

shaduf=kelon, keloneion=telo, tolleno. 

shaduf, mosaic floor in House of the Laberii at Oudna in Tunisia, Gauckler 1896

Markets And Marketing in Roman Palestine, pp.~76--77

Roman steelyard circa 79 CE, found in Pompeii (Naples, National Archaeological Museum. Inv. No. 5569).

E.D. Moutsoulas, ed., Epiphanius of Salamis, Concerning Weights and Measures, Introduction, Commentary, Text, and Notes, Theologia, 44 (1973) 157--198.

Next prepare round pills weighing one drachma, precisely limiting their weight with a scale (plastinx) Nicander, Theriaca 954--955

Weighing devices have turned up at several surgical sites, including a scale and a steelyard in a grave at Rheims (Kunzl [1983a] 63.32-64.33; Fig. 93), a scale--probably in a grave--at Colophon (Caton [1914] 118; Fig. 92) and several scales (bilancie) in Pompeian houses: the Casa del Medico Nuovo (I), the Casa del Medico Nuovo (ii), and the Casa del Centauro (where there were two); see Bliquez, Jackson (1994) 83, 84, 91.

Ernst Kunzl, Thomas Weber, Das sp\"atantike Grab eines Zahnarztes zu Gadara in der Dekapolis, Damaszener Mitteilungen, 5: 81-118, plus 7 plates.

Richard Caton, Notes on a Group of Medical and Surgical Instruments found near Kolophon, Journal of Hellenic Studies, 34: 114-118.

Bliquez, Jackson (1994): Lawrence J. Bliquez, Roman Surgical Instruments and Minor Objects in the National Archaeological Museum of Naples, with a Catalogue of the Surgical Instruments in the Antiquarium at Pompeii by Ralph Jackson (Mainz, Philip von Zabern).

Ducros, Hippolyte. Etude sur les balances Egyptiennes, Ann. du Service des Antiquities de l'Egypt, 9, 32--53, 1908;
Deuxieme etude sur les balances Egyptiennes, 10, 240--253, 1910

Epiphanias, James Elmer Dean, Epiphanias' Treatise on Weights and Measures. The Syrian Version. Chicago 1935

Glanville, Weights and balances in ancient Egypt, Proc Roy Inst of Great Britain, 29, 10--40, 1937. Weights and Balances in Ancient Egypt, Nature, 137, 890--892, 1936

Griffith, Notes on Egyptian weights and measures, Proc Soc Biblical Archaeol., 14, 403--440, 1892; 15, 301--15, 1893

Illahun, Kahun and Gurob. Medum, Petrie, p.~30, steelyard

Yassi Ada: A Seventh-Century Byzantine Shipwreck, p.~223, Bass, 1982

Daremberg Saglio, sv libra and machina

Ventris and Chadwick, 55

Barker GMW II

[EGYPT 29589] 'Libra and Virgo on astronomical ceiling at Dendera.' Scales and a woman holding an ear of corn portray the zodiac signs Libra and Virgo on the astronomical ceiling in the outer hypostyle hall of the Hathor Temple at Dendera. 

Balance and stone weights from Naqada, Science Museum, South Kensington, London


Theognis 157--8

Bach. 4.13,17.24--9, Aes. Pers. 345, 

Aeschylus, {\em Suppliants}, ll.~402 ff., ``holding the balance nicely poised, as ready to incline this
way as that'', Tucker, 1889

Homeric {\em Hymn 4 to Hermes} ll.~320--324, Evelyn-White:

\begin{quote}
Soon they came, these lovely children of Zeus, to the top of fragrant Olympus, to their father, the Son of Cronos; for there were the scales of judgement set for them both.
\end{quote}


Libra, Akkadian zibanitu, zi-ba-an-na, Sumerian RIN, ``scales''.



LG/23, No. 4 Straight St. outside SE wall of Room 5, Larnax A, p.~196,
two copper scale-pans and weights.

LG/45, No. 1 B Baker's Square, Chapel, Larnax A.

LG/170 On town wall, Corbel-vaulted tomb

U.16307, U.16769

A Companion to the Archaeology of the Ancient Near East, Volume 1

Nineveh palace walls shaduf

Beazley, ABV, 174,1

Lazzarini, M. 1948, Le bilance romane del Museo Nazionale e dell' Antiquarium Comunale
di Roma, Atti della Accademia Naz dei Lincei, 8 Ser, 3, 221--254.

Machabey, Armand J, Memoire sur l'histoire de la balance et de la balancerie, Paris, 1949

Petire, Ancient Weights and Measures, London 1926; Measures and Weights, London 1934

Viedebantt, Oscar. Zur Metrologie des Altertums, Leipzig 1917; Antike Gewichtsnormen und Munzfusse, Berlin 1923

shaduf, Tomb of Ipy, Thebes, ca.~1250 BC, N. de G. Davies, Two Ramesside Tombs at Thebes, Plate XXIX, 1927.
Tomb of Neferhotep, ca.~1340 BC, Plate XLVI.

EGYPTIAN TOMB PAINTING. Shadoof used for irrigation. Tomb painting, 19th Dynasty. From Tomb of Ipuy at Der-el-Medineh, c. 1275 B.C.


\begin{itemize}
\item Mastaba of Mereruka, Necropolis of Saqqara. Sixth Dynasty. Chamber A 3, East Wall, Scene 2, Plate 32, detail
of Plate 29, A.
 ``Metal-workers weighing and smelting ore, pouring molten metal, beating out gold
foil, and making collars and pectorals.'' \cite[Plate 29]{OIP31}
\item Mohenjodaro, Harappa. ``Mizan''.
\end{itemize}

Temple of Kom Ombo, outer corridor, depiction of scales on relief to the right of Marcus Aurelius.
Inner aspect of the northern part of the outer enclosure wall of the Temple of Kom Ombo,
Gift of Ritual and/or Surgical Instruments from the Roman Emperor Trajan.

Museo archeologico nazionale di Napoli. Sommer, Giorgio. n. 11144.

Pompeii, House of the Centenarian; Chemist's scales

udj Egyptian, judge, judgment

Minoan, Evans IV.2 656, balance sign on libation bowl from Knossos

Mohenjo-Daro, 2500 B.C., DK-80/2604 and DK I-355/2605

Yale 1938.2976

Yale 1938.2979

Yale 1938.2980

Cyrene, Egyptian scales on plate

Cyrene, Arcesilas cup, ca. 550 BC, Cabinet de M\'edailles, Paris, bibl nationale


K11.14 HERMES PSYKHOSTASIA  MEMNON
Museum Collection: British Museum, London, United Kingdom 
Catalogue Number: London B639 
Beazley Archive Number: 456
Ware: Attic Black Figure, White Ground 
Shape: Lekythos
Painter: Attributed to the Sappho Painter or Little-Lion Class 
Date: --
Period: Archaic

Scene in a cloth shop, weighing rolls of cloth. Black-figured jug (oenochoe), terracotta (mid 6th BCE), Attic Height: 22.5 cm Inv. IV 1105 Kunsthistorisches Museum, Antikensammlung, Vienna, Austria

Ancient scenes of weighing are virtually nonexistent in Mesopotamia but are much more common in Egypt. This depiction is part of a larger harbor scene from the tomb of Kenamun, an 18th Dynasty official from Thebes. Drawing adapted from N. de Garis Davies and R. O. Faulkner ``A Syrian Trading Venture to Egypt.'' Journal of Egyptian Archaeology 33 (1947):40-46.




Met, 30.4.103, Dynasty 18, Tomb of Nebamun and Ipuky

Met, 30.3.31, Dynasty 21, The Singer of Amun Nany's Funerary Papyrus

Met, 33.8.21, Dynasty 19, Anubis Weighing the Heart, Tomb of Nakhtamun

Met, 2008.355a-d, Byzantine

Met, 31.11.10, Amasis Painter

Met, 47.11.5, Taleides Painter

Met, 25.3.34, Dyntasy 21, Funerary Papyrus Belonging to the Singer Tiye

Berlin, Inv. Nr.: Fr. 892, Waagen und Gewichte 

Berlin, Inv. Nr.: Misc. 10030, Waagen und Gewichte 

Merchant carrying a pair of scales for weighing metals. Funeral stele from Marash, Northern Syria. Neo-Hittite, 8th BCE Basalt, 49 x 31,5 cm AO 19221 

Louvre, Departement des Antiquites Orientales, Paris, France


Wooden box for ``Ushebtis'', on the side God Anubis weighing souls. Box decorated with small funerary figurines and servants for the Hereafter. Wood/stucco/paint, Middle Kingdom (2060-1785 BCE) N 4124 

Louvre, Departement des Antiquites Egyptiennes, Paris, France

The Voyage to Punt: Cattle weighed against gold. Coloured limestone relief from the temple of Queen Hatshepsut (Maat Ka-Re) (1495-1475 BCE), south side of the west terrace, Deir el-Bahri. 18th Dynasty (1554 BC--1305 BC), New Kingdom 

Deir el-Bahri, Luxor-Thebes, Egypt

Weight, dedicated to the god Nanna, with an inscription of king Shulgi. From Mesopotamia. Dark diorite, 6,2 x 4,5 cm AO 22187 

Louvre, Departement des Antiquites Orientales, Paris, France

The Rassam obelisk from Nimrud, Mesopotamia, northern Iraq. Neo-Assyrian, 883-859 BCE. This fragment of a stone relief formed part of an obelisk discovered by archaeologist Hormuzd Rassam. The obelisk decorated one of the central squares in Nimrud, the site where King Ashurbanipal II chose to build his new administrative centre of the Assyrian Empire. This panel shows the king watching treasure being weighed on a pair of scales. BM ANE, 118800,
136897, 136898 

British Museum, London, Great Britain

One of the oldest surviving images of an equal-arm balance. Egyptian tomb, VI Dynasty, after ca. 2300 BC. (OIP 31, plate 30, Copyright The Oriental Institute of The University of Chicago).

To the modern observer, this is no longer something to marvel at. The operation of the unequal-arm balance can be described by the law of the lever. Equilibrium is obtained when the arms of the lever are in inverse proportion to the weights that are applied to them. Should it therefore be assumed that insight into the law of the lever was what made it possible for such balances to be built? Research by Department I has shown that the opposite is more likely to be the case. The first mention of an unequal-arm balance is found in a comedy from the year 421 BCE by the Athenian playwright Aristophanes. The first known formulation of the law of the lever, however, dates to the next century. There is in fact much to suggest that it was the existence of unequal-arm balances--which so clearly embody the law of the lever--that first stimulated the formulation of the law.

Book of the Dead with 125 chapters, judgement in the Otherworld. The defunct, the Lady Nefer-is, is led into the courtroom: her heart is weighed against the truth; she sacrifices to the Gods Osiris, Isis, and Nephtys. Papyrus (350 BCE), Late Period, Egypt - Inv. 10477 

Staatl. Museen, Aegyptisches Museum, Berlin, Germany

Stone weights from Megiddo. Israelite, Iron Age II 

Israel Museum, Jerusalem, Iron Age II, 7th--6th century BC, bronze weight pans, 
71.090.0305, 71.090.0306

Bronze balance with one pan and a weight shaped like a bust. 

Musee des Antiquites Nationales, St-Germain-en-Laye, France

Roman scales. From the Foret de Compiegne, France. 

Musee des Antiquites Nationales, St-Germain-en-Laye, France

A set of Hematite weights from Ur, southern Iraq, 1900-1600 BCE. Hematite was consistently used in Mesopotamia for weights from the late 3rd milennium BC; it is a hard stone which wears well and it would be obvious if it had been tampered with. A system of weights and measures was adopted, so that payments to workers could be reckoned, and also in order to calculate the value of precious objects. ANE 117891, ANE 117. 

British Museum, London, Great Britain

British Museum, London, Great Britain

Moorey \cite[p.~4]{moorey}:

\begin{quote}
Evolving irrigation networks are likely to have
been a powerful stimulus to innovation in related technology.
As the necessary technical devices were made
of organic materials and very rarely appear in art, their
development is largely a matter of conjecture. Apart
from the screw, which is a development of the Neo-Assyrian
(?) period, it is known that the four basic
technical devices for the redirection of muscular
effort-the lever, the wedge, the windlass, and the pulley-were
inherited by the Greek world from the East.
In all likelihood the lever and the
wedge had emerged there in remote prehistory; but
it is possible that a particular application in irrigation
popularized the lever. It is vital to the shadoof, an
extremely versatile water-lifting device easily constructed
from locally available timber.
Although this machine is not illustrated until it appears
on a cylinder seal in the third quarter of the third
millennium BC, it is a device
so necessary to irrigation in southern Mesopotamia that
its invention has been assumed to be much older. The
shadoof was generally made of a long wooden pole
secured at a fulcrum to a horizontal beam of wood,
supported at each end by a timber pole or by a mudbrick
column. The short end of the lever was counterweighted
with a stone or a lump of clay, with the bucket
attached by a rope to the other end, The best surviving
illustrations are on a relief of the reign of Sennacherib
in the seventh century BC.

Evolving irrigation networks are likely to have
been a powerful stimulus to innovation in related technology.
As the necessary technical devices were made 
of organic materials and very rarely appear in art, their
development is largely a matter of conjecture. Apart 
from the screw, which is a development of the Neo-Assyrian (?) period,
it is known that the four basic technical devices for the redirection of muscular
effort--the lever, the wedge, the windlass, and the pulley--
were inherited by the Greek world from the East. 
In all likelihood the lever and the
wedge had emerged there in remote prehistory; but
it is possible that a particular application in irrigation
popularized the lever. It is vital to the shadoof, an 
extremely versatile water-lifting device easily constructed
from locally available timber. Although this machine is not illustrated
until it appears on a cylinder seal in the third quarter of the third
millenium BC (Boehmer 1965: no.~716), it is a device
so necessary to irrigation in southern Mesopotamia that
its invention has been assumed to be much older. The
shadoof was generally made of a long wooden pole
secured at a fulcrum to a horizontal beam of wood,
supported at each end by a timber pole or by a 
mud-brick column. The short end of the lever was
counter-weighted with a stone or a lump of clay, with the bucket
attached by a rope to the other end. The best surviving
illustrations are on a relief of the reign of Sennacherib
in the seventh century BC (BM 124820).
\end{quote}

Hermes - psychostasia - Paris. Mus\'ee du Louvre G 399 

Peintre de Nicon, Amphore \`a figures rouges,
Hermès  pesant les âmes sur une balance (psychostasie),
Vers 470 avant J.-C.,
Provenance : Italie,
Athens. Louvre CA 2243.

Micah 6:11, Proverbs 11:1, 16:11--12, Leviticus 19:35-36,
Ezekiel 45:10-11, Amos 8:5-6, Job 31:5-6,
Hosea 12:7, 


Roman Moneta

Vapheio Tholos tomb, Lakonia, Athens Museum

Aeschylus, {\em Persians},
l.~346, ``. Do you think that we were simply outnumbered in this contest? No, it was some divine power that tipped the scale of fortune with unequal weight and thus destroyed our host.''
p.~179, Garvie

Iliad, viii, 69; xx, 209; xvi, 658; xix, 223

200468, Rome, Mus. Naz. Etrusco di Villa Giulia, Rome, Mus. Naz. Etrusco di Villa Giulia, 57684

Rijksmuseum van Oudheden, e 1897/12.1, 50--100 AD, bronze balance scales.

BM EA10554,80. Book of the Dead of Nestanebetisheru; sheet 80.

BM EA10554,63. Book of the Dead of Nestanebetisheru; sheet 63.

BM EA10479,6, Book of the Dead, Papyrus of Hor (sheet 6), 300 BC, Akhmim.

BM 1873,0820.300, Attica, 490 BC--480 BC.

BM 1836,0224.172, 350 BC--320 BC.

BM 1914,0219.1. Bronze bust of Silenus, companion of Bacchus. Originally an appliqué, the bust was re-used in late antiquity as a steelyard weight.The bronze steelyard has two suspension hooks on a chain.

BM EA10558,18, 305 BC--30 BC, Book of the Dead of Ankhwahibra (sheet 18).

BM EA10008,3, Funerary papyrus of Tameni; sheet 3

Sargonid seal, Louvre A. 156, Forbes 2, p.~17

BM EA10470,3, 1250 BC, Tomb of Ani, Book of the Dead, Papyrus of Ani (sheet 3)

BM EA9901,3. Book of the Dead: The papyrus of Hunefer.

Mesembria coin, Nikola Moushmov  4019, 4024

Calciati 17. Katane, Sicily. After 212 BC. AE 21mm. KATANAIWN, laureate head of Zeus Ammon right / Dikaiosyne (Aequitas), standing left, holding scales and cornucopiae, monogram left, two monograms right. Calciati 17; BMC 82; SNG Cop 203.

Ref Scipio denarius, RSC Caecilia 49, Syd 1048, Cr460/3

Varbanov  1913 Nemesis-Aequitas standing with scales and cornucopiae; wheel below.

Aristophanes


Plate XXIX, Davies, Two Ramesside tombs at Thebes.l
Apy's House and Garden, detail from Plate XXVII.
Painted by Davies.

Petrie \cite[p.~2]{petrie1926}:

\begin{quote}
Looking at the conditions of the ancient world, of a large number of communities 
each developing a strongly individual civilisation, the presumption is that there would be as many
standards as there were languages. The vision of our reducing all to one original standard is as
hopeless as the old idea of one primitive universal language.
\end{quote}

Metropolitan Museum of Art, Acc. No. G.R. 355.
445--446, Greek, Etruscan and Roman bronzes
By G.M. Richter

The Walters Art Museum, 54.197 

Roman Steelyard Balance Scale from Pompeii Bronze 1st century CE

Roman scales. From the Foret de Compiegne, France. Musee des Antiquites Nationales, St-Germain-en-Laye, France

Bronze balance with weight shaped like a Medusa's head. from Pompeii. 

Museo Archeologico Nazionale, Naples, Italy

Balance with one pan and bust of Hermes as a weight. Bronze, from Pompeii. Museo Archeologico Nazionale, Naples, Italy

ROMAN SCALES, WEIGHTS 1ST-3RD CE Bronze balance with one pan and a weight shaped like a bust. Musee des Antiquites Nationales, St-Germain-en-Laye, France

MFA 04.1792, Hu, Egypt, steelyard

MFA 2001.546.1-4

Harvard 2007.104.3.A-C

The Insula of the Menander at Pompeii: Volume III: The Finds, a Contextual Study

BM 1980,0602.2

Roman steelyard circa 79 CE, found in
Pompeii (Naples, National Archaeological
Museum. Inv. No. 5569).

Schnellwaage, Waagen und Gewichte, 2.Jh. n.Chr. Bronze, Ident.Nr. 31921

Boscoreale, villa of
P. Fannius Synistor, artifacts
from rooms D, 24 and 13: 
Bronze steelyard, photo
after Sammlung A. Ruesch
1936, no 111, pl. 32. 
Dumbarton Oaks, Acc. no. 40.55, p.~40 Catalogue of Greek and Roman Antiquities, Plate VI, B

Mus\'ee de Frontignan (34), inv. 84-176-08.0024.00220, balance compl\`ete avec curseur et crochets

Scale/Balance: Johns Hopkins Archaeological Museum, Inv. Buckler 2: L. of beam 30 cm. 1st-2nd cent. ce.

Steelyard: Gaius Firmius Severus Instrumentarium, Rheims: Mus\'ee d'Arch\'eologie nationale, Saint-Germain-en-Laye. L. of beam 27.5 cm. Late 2nd-early 3rd century ce.

\bibliographystyle{plain}
\bibliography{balance-scales}

\end{document}