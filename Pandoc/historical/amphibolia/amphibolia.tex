\documentclass{article}
\usepackage{amsmath,amssymb,mathrsfs,amsthm,xfrac}
\usepackage[LGR,T1]{fontenc}
\newcommand{\textgreek}[1]{\begingroup\fontencoding{LGR}\selectfont#1\endgroup}
%\usepackage{tikz-cd}
%\usepackage{hyperref}
\newcommand{\inner}[2]{\left\langle #1, #2 \right\rangle}
\newcommand{\tr}{\ensuremath\mathrm{tr}\,} 
\newcommand{\Span}{\ensuremath\mathrm{span}} 
\def\Re{\ensuremath{\mathrm{Re}}\,}
\def\Im{\ensuremath{\mathrm{Im}}\,}
\newcommand{\id}{\ensuremath\mathrm{id}} 
\newcommand{\gcm}{\ensuremath\mathrm{gcm}} 
\newcommand{\diam}{\ensuremath\mathrm{diam}} 
\newcommand{\sgn}{\ensuremath\mathrm{sgn}\,} 
\newcommand{\lcm}{\ensuremath\mathrm{lcm}} 
\newcommand{\supp}{\ensuremath\mathrm{supp}\,}
\newcommand{\dom}{\ensuremath\mathrm{dom}\,}
\newcommand{\norm}[1]{\left\Vert #1 \right\Vert}
\newcommand*\rfrac[2]{{}^{#1}\!/_{#2}}
\newtheorem{theorem}{Theorem}
\newtheorem{lemma}[theorem]{Lemma}
\newtheorem{proposition}[theorem]{Proposition}
\newtheorem{corollary}[theorem]{Corollary}
\begin{document}
\title{Amphibolia}
\author{Jordan Bell\\ \texttt{jordan.bell@gmail.com}\\Department of Mathematics, University of Toronto}
\date{\today}

\maketitle



Kneale and Kneale \cite[p.~224]{kneale}:

\begin{quote}
But in literary history,
as contrasted with archaeology, the forays of enthusiasts do not
destroy the evidence. On the contrary, they may provide the stimulus
to research by which their own errors can be corrected. And so
iit has proved in this case. For it is now established that apart
from the printed texts there exist scores of manuscript treatises
on logic from the twelfth and later centuries. Until these have all
been edited, or at least analysed and described, by experts in
medieval studies, there can be no hope of determining correctly
the various strata of the great deposit.
\end{quote}


\section{Plato}
Plural predication in Plato, {\em Hippias Major} 300d--e \cite[p.~259]{hippias}:

\begin{quote}
Hippias: You'll have  finer knowledge than anyone whether or not I'm
playing games, Socrates, when you try to describe these notions of yours
and are shown to be talking nonsense. It's quite impossible -- you'll never
find an attribute which neither you nor I have, but which both of us have.

Socrates: Are you sure, Hippias? I suppose you've got a point, but
{\em I} don't understand. Let me explain more clearly what I'm getting at: it
seems to me that both of us together may possess as an attribute something
which neither I have as an attribute nor am (and neither are you); and, to
put it the other way round, that neither of us, as individuals, may be
something which both of us together have as an attribute.
\end{quote}

Subsequently, Socrates ironically says the following, 301d--e \cite[p.~260]{hippias}:

\begin{quote}
You see, before you spoke, my friend,
we were so inane as to believe that {\em each} of us -- you and I -- is one, but that
both of us together, being two not one, are not what each individual is.
See how stupid we were! But now we know better: you've explained that
if both together are two, then each individual must be two as well; and if
each individual is one, both must be one as well.
\end{quote}

Hippias eventually agrees with the following statement of Socrates, 302b \cite[p.~261]{hippias}:

\begin{quote}
There {\em can}, then, contrary to your recent assertion, be no
absolute necessity that each of us is what both are, and vice versa. 
\end{quote}

Alcinous, {\em Didaskalikos} 6.9 \cite[p.~12]{alcinous}:

\begin{quote}
As for the practice of sophisms, we will find that sketched out
by Plato in the {\em Euthydemus}, if we pay close attention to that book. We
will find indicated in it, then, which sophisms are dependent upon
words, and which are dependent upon facts, and the solutions to
them.
\end{quote}

sophisms: \textgreek{sof'ismata}

{\em Euthydemus} 298e \cite{euthydemus}.

Called {\em sophism} by Aristotle in {\em Topics} VIII.11, 162a14

{\em Theaetetus} 154c \cite[p.~41]{theaetetus}: Socrates says, ``When you compare six dice with four, we say that the six are more than the four
or half as many again; while if you compare them with twelve, the six are fewer--only half as many--and one cannot say anything else. Or do you think you can?''
The point here is that without changing, the six dice are both more and fewer.

Aesopica 421, the sailor and his son

{\em Dissoi Logoi} V.3--4 \cite[p.~163]{dissoilogoi}:

\begin{quote}
And, furthermore, the same thing is larger and smaller and more and less and heavier and lighter. Thus all things are the same.
A talent is heavier than a mina and lighter than two talents; therefore the same thing is both heavier and lighter.
\end{quote}

V.15 \cite[pp.~163--165]{dissoilogoi}:

\begin{quote}
With respect to the assertion that the same man
both is and is not, I put the following question: ``Does he exist with respect to some particular thing, or just in general?''
Then if someone denies that the man exists, he is mistaken, because he is treating {$\langle$}the particular and${\rangle}$ universal senses as being the same.
Because everything exists in {\em some} sense.
\end{quote}

Plato, {\em Parmenides} 156c--157a

Aulus Gellius, {\em Attic Nights} 18, 13.7--8 \cite[p.~32]{diogenes}:

\begin{quote}
It is pleasant to record how wittily Diogenes responded to
a sophism of the kind that I mentioned above, when a dialectician
from Plato's school put it forward in the hope of making fun of him.
For when the dialectician had asked, `That which I am, you are not?'
and Diogenes had assented, and he had then added, `Now I am a 
human being', and Diogenes had assented to that too, he concluded:
`So it follows that you are not a human being.' `Now that', replied
Diogenes, `is false, but if you want it to become true, start off with
me.'
\end{quote}

















\section{Aristotle}
{\em kategorein} is taken by Aristotle from law-courts, meaning ``to accuse someone of something'', ``to impute something
to someone''. Imputation is in accusative, imputee in genitive.

{\em Categories} 2, 1a16 \cite[p.~3]{ackrill}:

\begin{quote}
Of things that are said, some involve combination
while others are said without combination. Examples of
those involving combination are `man runs', `man wins';
and of those without combination `man', `ox', `runs', 
`wins'.
\end{quote}

{\em Categories} 3, 1b10 \cite[p.~4]{ackrill}:

\begin{quote}
Whenever one thing is predicated of another as of a subject, all things said of what is predicated will be said
of the subject also. For example, man is predicated of the individual man, and animal of man; so animal will be
predicated of the individual man also -- for the individual man is both a man and an animal.
\end{quote}

{\em Categories} 4, 1b25 \cite[p.~5]{ackrill}:

\begin{quote}
Of things said without any combination, each signifies either substance or qualification or a relative or where
or when or being-in-a-position or having or doing or being-affected.
\end{quote}

{\em Categories} 5, 2a11 \cite[pp.~5--6]{ackrill}:

\begin{quote}
A {\em substance} -- that which is called a substance most strictly, primarily, and most of all -- is that
which is neither said of a subject nor in a subject, e.g. the individual man or the individual horse. The species in which
the things primarily called substance are, are called {\em secondary substances}, as also are the genera of these species.
For example, the individual man belongs in a species, man, and animal is a genus of the species; so these -- both man and animal --
are called secondary substances.
\end{quote}

{\em Categories} 5, 2b29  \cite[pp.~7--8]{ackrill}:

\begin{quote}
For if one is to say of the individual man what he is, it will be in place to give the species or the genus
(though more informative to give man than animal); but to give any of the other things would be out of place -- for example,
to say `white' or `runs' or anything like that. So it is reasonable that these should be the only things called substances.
\end{quote}











{\em On Interpretation} 7, 17a38 \cite[pp.~47--48]{ackrill}:

\begin{quote}
Now of actual things some are universal, others
particular (I call universal that which is by its nature
predicated of a number of things, and particular that which
is not; man, for instance, is a universal, Callias a particular.)
So it must sometimes be of a universal that one
states that something holds or does not, sometimes of a 
particular. Now if one states universally of a universal that
something holds or does not, there will be contrary statements
(examples of what I mean by `stating universally
of a universal' are `every man is white' and `no man is
white'). But when one states something of a universal but
not universally, the statements are not contrary (though
what is being revealed may be contrary). Examples of
what I mean by `stating of a universal not universally'
are `a man is white' and `a man is not white';  man is
a universal but it is not used universally in the statement
(for `every' does not signify the universal but that it is
taken universally). It is not true to predicate a universal
universally of a subject, for there cannot be an affirmation
in which a universal is predicated universally of a subject,
for instance `every man is every animal'.
\end{quote}



10, 20a16: ``Is Socrates wise? No. Then Socrates is not-wise.'' vs. ``Is every man wise? No. Then
every man is not-wise.'' \cite[p.~56]{ackrill}












Aristotle, {\em De sophisticis elenchis} IV, 165b \cite{pickard-cambridge}:

\begin{quote}
There are two styles of refutation: for some depend on the language used, while some are independent of language. Those ways of producing the false appearance of an argument which depend on language are six in number: they are ambiguity, amphiboly, combination, division of words, accent, form of expression.
\end{quote}

Boethius \cite[p.~8]{ALVI1to3}:

\begin{quote}
Modi autem sunt arguendi quidem quo; nam alii quidem
sunt secundum dictionem, alii vero extra dictionem. Sunt autem
ea quidem quae secundum dictionem faciunt fantasiam sex
numero; sunt autem haec aequivocatio, amphibolia, compositio,
divisio, accentus, figura dictionis.
\end{quote}

IV, 165b \cite{pickard-cambridge}:

\begin{quote}
Arguments such as the following depend upon ambiguity. `Those learn who know: for it is those who know their letters who learn the letters dictated to them'. For to `learn' is ambiguous; it signifies both `to understand' by the use of knowledge, and also `to acquire knowledge'.
\end{quote}

IV, 166a \cite{pickard-cambridge}:

\begin{quote}
Examples such as the following depend upon amphiboly: `I wish that you the enemy may capture'. Also the thesis, `There must be knowledge of what one knows': for it is possible by this phrase to mean that knowledge belongs to both the knower and the known. Also, `There must be sight of what one sees: one sees the pillar: {\em ergo} the pillar has sight'. Also, `What you profess to-be, that you profess to-be: you profess a stone to-be: {\em ergo} you profess-to-be a stone'. Also, `Speaking of the silent is possible': for `speaking of the silent' also has a double meaning: it may mean that the speaker is silent or that the things of which he speaks are so.\footnote{cf. {\em Euthydemus} 300b--c.} There are three varieties of these ambiguities and amphibolies: (1) When either the expression or the name has strictly more than one meaning, e.g. \textgreek{>aet'os} and the `dog'; (2) when by custom we use them so; (3) when words that have a simple sense taken alone have more than one meaning in combination; e.g. `knowing letters'. For each word, both `knowing' and `letters', possibly has a single meaning: but both together have more than one--either that the letters themselves have knowledge or that someone else has it of them. 
\end{quote}

Boethius \cite[p.~9]{ALVI1to3}:

\begin{quote}
Secundum amphiboliam autem
tales: velle accipere me pugnantes, et `putas quod quis scit
hoc scit?' Nam et scientem et scitum contingit ut scientem
significari hac oratione. Et `putas quod videt quis, hoc videt?
Videt autem columnam, quare videt columna'. Et `putas quod
tu dicis esse, hoc tu dicis esse? Dicis autem lapidem esse; tu
ergo dicis lapis esse'. Et `putas est tacentem dicere?' Duplex
enim est tacentem dicere, et hunc dicentem tacere et quare dicuntur.
Sunt autem tres modi secundum aequivocationem et
amphiboliam: unus quidem quando vel oratio vel nomen principaliter
significat plura, ut piscis et canis; alius autem quando
soliti sumus sic dicere; tertius vero quando compositum plura
significat, separatum vero simpliciter, ut `scit saeculum'; nam
utrumque, si contingit, unum quid significat, et scire et saeculum;
ambo autem plura, aut saeculum ipsum scientiam habere
aut saeculi alium.
\end{quote}

IV, 166a \cite{pickard-cambridge}:

\begin{quote}
Upon the combination of words there depend instances such as the following: `A man can walk while sitting, and can write while not writing'. For the meaning is not the same if one divides the words and if one combines them in saying that `it is possible to walk-while-sitting' [and write while not writing]. The same applies to the latter phrase, too, if one combines the words `to write-while-not-writing': for then it means that he has the power to write and not to write at once; whereas if one does not combine them, it means that when he is not writing he has the power to write.
\end{quote}

combine: a man can walk-while-sitting, separately: a man while sitting has the power to walk. 

Boethius \cite[p.~9]{ALVI1to3}:

\begin{quote}
secundum compositionem autem huiusmodi, ut posse sedentem
ambulare et non scribentem scribere. Non enim idem significat
si dividens quis dicat et componens, quoniam possibile `sedentem
ambulare' et `non scribentem scribere'. Et hoc similiter
si quis componat `non scribentem scribere'; significat enim
quoniam habet potestatem ut non scribens scribat, si autem
non componat, quoniam habet potestatem, quando non scribit,
ut scribat.
\end{quote}

IV, 166a \cite{pickard-cambridge}:

\begin{quote}
Upon division depend the propositions that 5 is 2 and 3, and odd, and that the greater is equal: for it is that amount and more besides. For the same phrase would not be thought always to have the same meaning when divided and when combined, e.g. `I made thee a slave once a free man', and `God-like Achilles left fifty a hundred men'. 
\end{quote}

Boethius \cite[p.~10]{ALVI1to3}:

\begin{quote}
Secundum divisionem vero, quoniam quinque sunt et duo et
tria, et paria et imparia, et maius aequale; tantundem enim
maius et adhuc amplius. Nam eadem oratio divisa et composita
non idem semper significare videbitur, ut `ego te posui servum
entem liberum' et `quinquaginta virorum centum reliquit divus
Achilles'.
\end{quote}

IV, 166b \cite{pickard-cambridge}:
 
 \begin{quote}
Of fallacies, on the other hand, that are independent of language there are seven kinds: 
(1) that which depends upon Accident:
(2) the use of an expression absolutely or not absolutely but with some qualification of respect or place, or time, or relation: 
(3) that which depends upon ignorance of what `refutation' is: 
(4) that which depends upon the consequent: 
(5) that which depends upon assuming the original conclusion: 
(6) stating as cause what is not the cause: 
(7) the making of more than one question into one. 
 \end{quote}
 
 Boethius, \cite[pp.~10--11]{ALVI1to3}:
 
 \begin{quote}
 Ergo secundum dictionem redargutiones ex his locis sunt;
 eorum vero qui extra dictionem sunt paralogismorum species
 sunt septem; una quidem secundum accidens, secunda autem
 in eo quod simpliciter vel non simpliciter sed quo aut ubi aut
 quando aut ad aliquid dicitur, tertia autem secundum elenchi
 ignorantiam, quarta vero secundum consequens, quinta autem
 secundum quod in principio est sumere, sexta quod non causa
 est ut causam ponere, septima vero plures interrogationes
 unam facere.
 \end{quote}

V, 166b \cite{pickard-cambridge}:

\begin{quote}
Fallacies, then, that depend on Accident occur whenever any attribute is claimed to belong in like manner to a thing and to its accident. For since the same thing has many accidents there is no necessity that all the same attributes should belong to all of a thing's predicates and to their subject as well. Thus (e.g.), `If Coriscus be different from ``man'', he is different from himself: for he is a man': or `If he be different from Socrates, and Socrates be a man, then', they say, `he has admitted that Coriscus is different from a man, because it so happens
({\em accidit}) that the person from whom he said that he (Coriscus) is different is a man'. 
\end{quote}

V, 166b--167a \cite{pickard-cambridge}:

\begin{quote}
Those that depend on whether an expression is used absolutely or in a certain respect and not strictly, occur whenever an expression used in a particular sense is taken as though it were used absolutely, e.g. in the argument `If what is not is the object of an opinion, then what is not is': for it is not the same thing `to be $x$' and `to be'' absolutely.
Or again, `What is, is not, if it is not a particular kind of being, e.g. if it is not a man.' For it is not the same thing `not to be $x$' and `not to be' at all: it looks as if it were, because of the closeness of the expression, i.e. because `to be $x$' is but little different from `to be', and `not to be $x$' from `not to be'. Likewise also with any argument that turns upon the point whether an expression is used in a certain respect or used absolutely. Thus e.g. `Suppose an Indian to be black all over, but white in respect of his teeth; then he is both white and not white.' Or if both characters belong in a particular respect, then, they say, `contrary attributes belong at the same time'. This kind of thing is in some cases easily seen by any one, e.g. suppose a man were to secure the statement that the Ethiopian is black, and were then to ask whether he is white in respect of his teeth; and then, if he be white in that respect, were to suppose at the conclusion of his questions that therefore he had proved dialectically that he was both white and not white. But in some cases it often passes undetected, viz. in all cases where, whenever a statement is made of something in a certain respect, it would be generally thought that the absolute statement follows as well; and also in all cases where it is not easy to see which of the attributes ought to be rendered strictly. A situation of this kind arises, where both the opposite attributes belong alike: for then there is general support for the view that one must agree absolutely to the assertion of both, or of neither: e.g. if a thing is half white and half black, is it white or black? 
\end{quote}












{\em Prior Analytics} \cite{strikerI}: ``that which depends upon the consequent''. 















\section{Stoics}
Atherton \cite{atherton}

Diogenes Laertius 7.62, verbal ambiguity

Galen, {\em De sophismatis} \cite[pp.~582--598]{galen20}. {\em De sophismatis} 4 \cite[p.~228]{LS1}:

\begin{quote}
We must take up their actual divisions of the so-called `ambiguities'. The more refined (Stoics) list eight. \dots.
Seventh is the one which fails to indicate what signifying element is construed with what, as in `Fifty men having a hundred did Achilles
leave.'
\end{quote}

Galen, {\em De Captionibus} 1 \cite[p.~89*]{captionibus}:

\begin{quote}
We have a case of Amphiboly when there is ambiguity in the
sentence on account of the sentence as such: for example, in
`\textgreek{g'enoito katalabe{\~i}n t`on \~<un >em'e}' (`May it happen that I catch the
wild boar,' `May it happen that the wild boar catches me'). There none
of the words is ambiguous. The sentence as such by itself, however,
signifies both the catching and the being caught by [the wild board].
\end{quote}














\section{Rhetoric}
{\em Rhetorica ad Herennium} 2.11.16 \cite[pp.~85--87]{LCL403}:

\begin{quote}
If a text is regarded as ambiguous, because it can be interpreted in two or more meanings, the treatment is as follows: first we must examine whether it is indeed ambiguous; then we must show how it would have been written if the writer had wished it to have the meaning which our adversaries give to it; next, that our interpretation is practicable, and practicable in conformity with the Honourable and the Right, with Statute Law, Legal Custom, the Law of Nature, or Equity; of our adversaries' interpretation the opposite is true; and the text is not ambiguous since one well understands which is the true sense. There are some who think that for the development of this kind of cause a knowledge of amphibolies as taught by the dialecticians is highly useful. I, however, believe that this knowledge is of no help at all, and is, I may even say, a most serious hindrance. In fact these writers are on the lookout for all amphibolies, even for such as yield no sense at all in one of the two interpretations. Accordingly, when some one else speaks, they are his annoying hecklers, and when he writes, they are his boring and also misty interpreters. And when they themselves speak, wishing to do so cautiously and deftly, they prove to be utterly inarticulate. Thus, in their fear to utter some ambiguity while speaking, they cannot even pronounce their own names.
\end{quote}

Cicero, {\em Topica} 28 \cite[p.~129]{reinhardt}:

\begin{quote}
Further, some definitions consist of partitions, others of 
divisions; of partitions when the subject at issue is, as it were,
dismembered into its parts, e.g. if one were to say the civil law
was that which consists of laws, decrees of the senate, previous
decisions, the authority of the jurisconsults, the edicts of the
magistrates, custom, and equity. A definition based on division,
on the other hand, comprises all species which are subordinate to
the species which is being defined, in the following way: A legal
transfer of property is either transfer with legal obligation or
cession at law, of things which can be bought, between parties
who can do this in accordance with the civil law.
\end{quote}

partitions: {\em partitionum}, division: {\em divisionum}, dismembered: {\em discerpitur}, parts: {\em membra},
consists: {\em consistat}, species: {\em forma},
comprises: {\em complectitur}.

Cicero, {\em Topica} 31 \cite[p.~131]{reinhardt}:

\begin{quote}
They define genus and species in the following way: A genus
is a notion applying to several different things; a species is a notion
whose difference can be referred back to the genus as its source,
as it were. I call notion what the Greeks sometimes call {\em ennoia}
and sometimes {\em prolepsis}. This is an ingrained grasp of something,
developed through previous perceptions, which requires articulation.
Thus species are those things into which a genus may
be divided without leaving out anything, e.g. if one were to
divide `the law' into the sum of all legal statutes, custom,
and equity. 
\end{quote}

genus: {\em genus}, species: {\em formam}, notion={\em notio}, divided: {\em dividitur}.

Cicero, {\em Topica} 33 \cite[p.~133]{reinhardt}:

\begin{quote}
Partition is to be used sometimes in such a way that you do
not leave out any part; e.g. if you want to make a partition of
tutelage, you would act ignorantly if you left out any part. But if
you make a partition of stipulations and legal formulae, it is not a
fault to leave out something in such a boundless area. In a division
on the other hand the same thing is a fault. For there is a definite
number of species which are subordinate to each genus; whereas
the spread of parts is often more indefinite, in the way several
streams come from a single source.
\end{quote}

Quintilian, {\em Institutio Oratoria} 7.9.4 \cite[p.~283]{LCL126}:

\begin{quote}
Groups of words give more scope for ambiguity. It occurs 
(a) because of cases:

\begin{quote}
I say that you, O child of Aeacus, the Romans can defeat;
\end{quote}

(b) because of the arrangement, when it is doubtful what refers to
what, especially when there is a word in the middle of a sentence
which may be taken either with what precedes or with what
follows. Thus in Vergil's line about Troilus: {\em lora tenens tamen}, it
may be disputed whether he
\end{quote}

Ennius, {\em Annals}

{\em Aeneid} I.477.

{\em Institutio Oratorio} 7.9.6, 7.9.10, 8.2.16, accusative and infinite.












Aelius Theon, {\em Progymnasmata} 5, ``On narrative'', 81--82 \cite[p.~31]{progymnasmata}:

\begin{quote}
What is called an ``amphiboly'' by the dialecticians makes the expression obscure because of the confusion between an undivided and divided word, as in the
phrase ``Let an {\em aul\^{e}tris} (``flute-girl'')
that has fallen be `public'.'' It means one thing when the word {\em aul\^{e}tris} is taken as a whole and undivided, another when
divided: ``Let an {\em aul\^{e} tris} (``a hall thrice'') fallen be public property.''
Furthermore, (the expression is ambiguous) when it is unclear what some part of a word belongs to; for example, Heracles fights {\em oukentaurois}. This has two meanings, that
Heracles does not at all ({\em oukhi}) fight with centaurs or that he fights not among ({\em ouk en}) bulls. Similarly, an expression becomes unclear when it is not evident to what some signifying portion refers; for example ({\em Iliad} 2.270),
``And they though distressed at him sweetly laughed.'' For it is ambiguous whether they are distressed at Thersites, which is false, or at the launching of the ships. And again
({\em Iliad} 2.547--48), ``The people of great Erechtheus, whom once Athene / Nurtured, Zeus' daughter, and the grain-giving land bore.'' It is unclear whether he is saying the people or Erechtheus were nurtured by Athene and born from the land.
\end{quote}

`public': regarded as a prostitute

Aelius Theon, {\em Progymnasmata} 12, ``On law'', 129 \cite[p.~62]{progymnasmata}:

\begin{quote}
Since our discussion now is about refutation and confirmation of a law, and especially laws that are being introduced, we must describe this. When laws are being introduced we either speak against them and rebut them or we speak for them and supply supporting evidence. 
After the prooemion we rebut from the following topics: from what is unclear, impossible, unnecessary, contradictory, unjust, unworthy, inexpedient, shameful.

Some lack of clarity occurs from pronunciation, which certain authorities call ``from prosody,'' some from the meaning of a word, some from homonymy, some from polyonymy,
which others call synonymy, some from syntax, some from compounding and dividing words, some from pleonasm, some from ellipsis, some from inconsistency.
\end{quote}

Aelius Theon, {\em Progymnasmata} 12, 130 \cite[p.~63]{progymnasmata}:

\begin{quote}
And in syntax; for example, when Pittacus said ``to share, father and mother, equally''; for the statement is ambiguous as to whether the children are to share the possessions of the parents or the parents those of the children. And further, when a word in the middle of a sentence creates a different meaning when taken with what precedes or what follows it; for example, ``Let a general victorious in war dedicate a statue of Ares, golden with a spear.'' Is a golden statue or golden spear meant?

Concerning combining and dividing--or as some say concerning confusion between the divided and the undivided--; for example, the law ordering brothers and children to
come to the settlement (of an estate). If this is taken as ``divided'' it means that first the brother, and if he is not alive, then the children are to be summoned, but it can be
combined to mean they are to be called at the same time. Or again, ``The false witness taken three times 1000 (drachmas) let him give'';
for either it means that one detected thrice in giving false evidence should pay a 1000 or that if detected once he should give 3000 drachmas.
\end{quote}















\section{Grammarians}
{\em Tekhne Grammatike} XI \cite[p.~176]{tekhne}:

\begin{quote}
A word is the smallest part of a properly constructed sentence.

A sentence is a combination of words in prose [and poetry] conveying a meaning which is complete in itself.

There are eight parts of the sentence: noun, verb, participle, article,
pronoun, preposition, adverb, conjunction. For the appellative is a subspecies
of the noun. 
\end{quote}

{\em Tekhne Grammatike} XII \cite[p.~176]{tekhne}:

\begin{quote}
A noun is a part of the sentence which is subject to case inflection, and
signifies something corporeal or non-corporeal; by corporeal I mean something
like `a stone', and by non-corporeal something like `education'; it can
be used in a general way, as in `man', `horse', and in a specific way, as in
`Socrates'. 
\end{quote}

{\em Tekhne Grammatike} XII \cite[pp.~178--179]{tekhne}:

\begin{quote}
There are the following subtypes of the noun (these also are referred to as `species'): proper, appellative, attached, relative, quasi-relative, homonymous, synonymous, dionymous, eponymous, ethnic, interrogative, indefinite, anaphoric (also referred to by the names `similative', `demonstrative', and `correlative'), collective, distributive, inclusive, onomatopoeic, generic, specific, ordinal, absolute, participatory.

1. A proper noun signifies substance that is individual, such as `Homer, Socrates'.

2. An appellative noun signifies substance that is shared, such as `man, horse'.

3. An attached noun is placed next to proper or appellative nouns alike, and conveys praise or blame. It is understood in three ways -- as referring to the soul, to the body, or to externals; to the soul as in `temperate, licentious'; to the body as in `fast, slow'; to externals as in `rich, poor'.

4. A relative noun is exemplified by `father, son, friend, right' [opp. to `left'].

5. A quasi-relative noun is exemplified by `night, day, death, life'.

\dots

12. An interrogative noun, also called `questioning', is so called because it makes an enquiry, for example `who? what sort of? how much? how great?'

13. An indefinite noun is one which conveys an opposite sense to that of the interrogative, for instance `whoever, whatever sort, however much, however great'.

14. An anaphoric noun, also called `similative', `demonstrative', and `correlative', is one which signifies a likeness; for instance `of such a kind, so much, so great'.

15. A collective noun is a noun in the singular number which signifies a plurality, such as `people, chorus, crowd'.

16. A distributive is a noun which, when two or more things are involved, refers to one of them, such as {\em hekateros} (each -- of two), {\em hekastos} (each -- of more than two).
\dots
\end{quote}

Marius Plotius Sacerdos, {\em Artes} I.12, ``De ceteris vitiis'', GLK 6.455:

\begin{quote}
De amphibolia. Amphibolia est dictio ambigua dubiam faciens sensuum sententiam, ut `aut mixta
rubent ubi lilia multa alba rosa'.
\end{quote}

{\em Aeneid} XII.68: ``or when white lilies blush mingled with many a rose''.

Luhtala \cite[p.~73]{luhtala}:

\begin{quote}
Ambiguity is a expression which renders an ambiguous sense, e.g., mixta rubent ubi
lilia multa alba rosa.
\end{quote}

Servius, {\em In Vergilii Aeneidos Libros} \cite[p.~582]{serviusII}:

\begin{quote}
MIXTA RVBENT VBI LILIA MVLTA ALBA ROSA aut ubi multa alba lilia permixta rubent rosa, id est rosae coniunctione: naturaliter enim omnis candor vicinum in se trahit ruborem.
\end{quote}

Charisius, {\em Ars} 4.1.13, ``De vitiis ceteris'', GLK 1.271 \cite[p.~271]{GLKI}:

\begin{quote}
De amphibolia. Amphibolia est dictio sententiave dubiae significationis:
dictio, ut vadatur Cato; sententia, ut
\begin{quote}
`aio te, Aeacida, Romanos vincere posse'.
\end{quote}
ambigua enim sors fuit ante eventum utrum Pyrrus a Romanis an
Romani a Pyrro vinci possent. fit aliquando et in uno verbo * ut siquis se dicat
hominem occidisse, cum appareat eum qui loquitur occisum non esse.
 \end{quote}

Ennius, {\em Annals} VI.186.

Diomedes, {\em Ars} II, ``De vitiis orationis'', GLK 1.450:

\begin{quote}
De amphibolia. Amphibolia est vitio compositionis in ambiguo posita
sententia, ut
\begin{quote}
aio te, Aeacida, Romanos vincere posse;
\end{quote}
item `certum est Antonium praecedere eloquentia Crassum'.
hi enim duo sensus vitio ambiguitatis carent proprietate, cum sit incertum
ab Aeacida Romanos vinci posse an a Romanis Aeacidam; similiter
ab Antonio vinci Crassum eloquentia an {\em a} Crasso Antonium.
\end{quote}

is the son of Aeacus able to win the Romans or vice-versa, similarly
does Antonius excel Crassus in eloquence or vice-versa.

Aelius Donatus, {\em Ars maior} 3.3, ``De ceteriis vitiis'', GLK 4.395:

\begin{quote}
Amphibolia est ambiguitas dictionis, quae fit aut per casum accusativum, ut siquis dicat `audio secutorem retiarium superasse'; aut per commune verbum, ut siquis dicat `criminatur Cato', `vadatur Tullius' nec addat quem vel a quo; aut per distinctionem, ut `vidi statuam auream hastam tenentem'. Fit et per homonyma, ut siquis `aciem' dicat et non addat oculorum aut exercitus aut ferri. Fit praeterea pluribus modis, quos percensere omnes, ne nimis longum sit, non oportet.
\end{quote}




Priscian, {\em Institutiones grammaticae} 
















Cassiodorus, {\em Expositio in Psalterium} XX.12:

\begin{quote}
Quod schema dicitur amphibologia, id est dictio ambigua, dubiam faciens pendere sententiam.
\end{quote}

Isidore of Seville, {\em Etymologies} 1.34.13--14:

\begin{quote}
amphibolia, ambigua dictio, quae fit aut per casum
accusativum, ut illud responsum Apollinis ad Pyrrhum: `aio te
Aeacida, Romanos vincere posse'. in quo non est certum, quem
in ipso versu monstraverit esse victorem.
\end{quote}

\cite[p.~57]{isidore}:

\begin{quote}
13. Amphibolia is ambiguous speech that occurs with the accusative case, as in this answer of Apollo to Pyrrhus (Ennius, Annals 179):
\begin{quote}
{\em Aio \underline{te}, Aeacida, \underline{Romanos} vincere posse}\\
(I say that you, scion of Aeacus, can conquer the Romans -- or -- I say that the Romans can conquer you, scion of Aeacus).
\end{quote}
In this verse it is not clear whom he has designated as the victor. 14. It can also occur due to a distinction that is not clear, as (Vergil, {\em Aen}. 1.263):
\begin{quote}
{\em Bellum ingens geret Italia}\\
(Italy will wage an immense war -- or -- Immense Italy will wage war).
\end{quote}
The distinction is unclear, whether it is `immense war' or `immense Italy.'
\end{quote}

















\section{Commentators}
Porphyry, {\em Isagoge} \cite{isagoge}: explains the five words genus, difference, species, property, accident.

Porphyry, {\em in Cat} 80.32--81.11 \cite[p.~145]{barnes}, on {\em Cat} 1b10--15:

\begin{quote}
But how can that be true? After all, man is said of Socrates as of a subject, and of man is predicated not only animal but also species (for man is a species). But they won�t all be predicated of Socrates; rather, animal will be predicated of him but species will not--for Socrates isn't a species.
\end{quote}













Boethius, {\em De divisione}

Boethius, {\em De topicis differentiis} II, 1189A--B \cite[p.~52]{topicis}:

\begin{quote}
One can take parts and whole into account not only in substances
but also in mode, times, quantities, and place.
What we call {\em always} is a temporal whole; what we call {\em sometimes}
is a temporal part. Again, if something is put forth without
qualification, it is a whole with regard to mode; if something
is put forth with a qualification, it is a part with
regard to mode. Similarly, if we talk about {\em everything}, we talk about a
whole with regard to quantity; if we pick out {\em something} from
quantity, we present a part of quantity. In the same way also
with regard to place: that which is {\em everywhere} is a whole; what
is {\em somewhere}, a part.
\end{quote}

cf. Aristotle, {\em Metaphysics} 1023b26--36.



















\section{Scholastic writers}
Alexander de Villa Dei, {\em Doctrinale}, pars IV, capitulum XI, ll. 2399--2403 \cite[p.~159]{doctrinale}:

\begin{quote}
amphibologia [est] constructio non manifestans sensum perfecte:
puto te socium superare. hoc fit multotiens, quia non determino plene
affectum mentis defectu praepediente, sive duplex sensus ex verbis possit
haberi.
\end{quote}









Adam of Balsham, {\em Ars Disserendi} \cite[pp.~64--65]{modernorum1}: 
A: {\em principia sophistica sine complexione}, B: {\em principia sophistica secundum complexionem}








Richard the Sophister, {\em Abstractiones}. no. 1: {\em Omnis homo est omnis homo}










Peter of Spain, {\em Summule Logicales} 7.47 \cite[p.~287]{copenhaver}:

\begin{quote}
The third species of amphiboly arises from the fact that some phrase signifies several
things while any one of its parts used by itself signifies only one, like `knows the age.'
Now this phrase is ambiguous: it signifies both that someone knows the age and
that the age has knowledge of someone. And the reason for this is that the word `age'
can be the object of this verb `knows' or its subject. This case is similar: `what someone
sees, sees this,' where the word `this' can be the subject of the verb `sees' in its second
occurence or its object. This is also similar: `what someone knows, knows this,'
where the word `this' can be the subject or object of the verb `knows' in its second
occurrence. Also this: `I would like me to take the enemy.' This is ambiguous because
the accusative `me' can be the subject of the verb `take' or its object, and the accusative
`enemy' can also be the subject or object of the same verb.
\end{quote}

`knows the age': {\em scit seculum}, `what someone sees, sees this': {\em quod quis videt, hoc videt}, 
`what someone knows, know this': {\em quod quis scit, hoc scit},
`I would like me to take the enemy': {\em vellem me accipere pugnantes}.







Thomas Aquinas, {\em De fallaciis}.















The anonymous {\em Summa sophisticorum elencorum}  \cite{modernorum1}, of the Abbey of St. Victor, Paris.

{\em Fallacie Vindobonenses}  \cite{modernorum1}

{\em Fallacie Parvipontane}  \cite{modernorum1}

{\em Dialectica Monacensis} \cite[pp.~565--567]{modernorum22}

Peter the Chanter, {\em De Tropis Loquendi} \cite{modernorum22}

{\em Fallacie Magistri Willelmi} \cite{modernorum22}













Walter Burley, {\em De puritate}.

Lambert of Auxerre, {\em Logica} \cite[pp.~102--162]{CTMPT}

{\em Syncategoremata Monacenia} \cite[pp.~163--173]{CTMPT}

Nicholas of Paris, {\em Syncategoremata} \cite[pp.~174--215]{CTMPT}

William of Sherwood, {\em Introduction to Logic}

William of Sherwood, {\em Syncategoremata} \cite{syncategorematic}

Richard Kilvington \cite{kilvington}








Peter of Spain, {\em Syncategoreumata}, chapter 4 \cite{syncategoreumata}:

\begin{quote}
{\em Omne animal preter hominem est irrationale.}

Every animal but man is irrational.
\end{quote}













\bibliographystyle{amsplain}
\bibliography{amphibolia}

\end{document}