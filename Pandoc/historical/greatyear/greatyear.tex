\documentclass{article}
\usepackage[T1]{fontenc}
\begin{document}
\title{The great year, calendars, and the incommensurability of celestial rotations}
\author{Jordan Bell}
\date{November 25, 2016}

\maketitle

Zoroastrianism: p.~20, Mary Boyce, {\em Textual Sources for the Study of Zoroastrianism}

Dicks \cite[p.~26]{dicks} writes:

\begin{quote}
It is a mistaken emphasis to regard the Ptolemaic planetary theory as the crowning achievement of
Greek astronomy, as is commonly done; rather is it an ingenious appendage to the primary
aim, which was the establishment of a scientific treatment
of solar, lunar, and stellar phenomena to facilitate the measurement
of time (for which the planets are useless in naked-eye astronomy).
\end{quote}






\section{Words and phenomena}
Day: day and night

noon

meridian: {\em mesembria}




\section{Various {\em loci classici}}
{\em Iliad} 18.483--89, Pleiades; 2.551; 23.833

{\em Odyssey} 5.271--77, Pleiades and Arcturus in Bo\"otes; east and west, 10.190--92; 5.295--296 winds; 
10.467; 11.295--95

Hesiod 383--87, Pleiades heliacal rising and cosmical setting

Aeschylus, {\em Prometheus Bound} 443--483, ``I taught them to determine when stars rise or set -- A difficult art.'', p.~34

Hippocratic corpus, {\em Airs, Waters, and Places} 

Herodotus, 2.29; 2.34; 4.42

Thucydides vii.16, vi.39, ii.78

Anaximander, sun lights moon, DKA76

Heraclitus, Fragment 94, Plutarch, {\em De exilio} 11, 604A \cite[p.~284]{heraclitus}: ``And yet each of the planets revolves in one orbit as though
in an island, and preserves its regularity: for Sun will not overstep his measures, says Heraclitus; if he does, the Erinyes, the minions of Justice, will
find him out.'' Dodds, p.~7;
Fragment 120, Strabo 1.3, p.~289: ``Heraclitus is better and more Homeric when he likewise uses the name `the Bear' for `the Arctic circle':
The limits of dawn and evening are the Bear, and, opposite the Bear, the boundary of bright Zeus. For the Arctic circle, and not
the Bear, is the northern boundary of rising and setting.''
Fragment 100, Plutarch, {\em Quaest. Plat.} 8, 1007D, p.~294: ``Time is movement in an order that has measure and limits
and periods. Of these periods the sun is overseer and guardian, for the 
defining and arbitrarting and revealing and illuminating of changes and Seasons which bring all things as Heraclitus says -- 
not of unimportant and small periods but of the greatest and most influential; and so the 
sun becomes a fellow-worker with the highest and chief god.''

Orphic Hymns, {\em Orphica}

Oenopides, Aelian: 10.7, p.~319; Heath \cite[p.~130]{aristarchus}

{\em Laws} 809c--d, 817e--818a, 945e

{\em Epinomis} 977a--979a

{\em Timaeus} 37d--e, 97d--98a

{\em Gorgias} 451c

{\em Statesman} 269

{\em Republic} 516c, 530b--c, 809c--d

{\em Phaedo} 108b--110b

{\em Cratylus} 409a--b

Xenophon, {\em Hellenica},  I.6.1, II.3.2

Aristotle, {\em De caelo} ii.14, 297a8, 285b25--27; Simplicius, {\em in De caelo}, ii.12, 221a.

Aristotle, {\em Physics} iv, 223b

Aristotle, {\em Metaphysics} XII.8, 1073b17

Aristotle, {\em Meteorology} 3.5; 1.6, 343a

Eratosthenes, Fraser, I.414

Polybius IX.12--20, Walbank \cite[p.~141]{walbankII}

Lucretius, V.621

Diogenes Laertius ii.9, eclipse

Blass, {\em Ars Eudoxi} 

Vitruvius, 1.6.9; 9.3 vernal equinox

Pliny, 2.64.160; 6.96--100

Solinus

Seneca, {\em De Beneficiis} 5.6.2, LCL 310; {\em Natural Questions}

Ausonius, {\em Eclogues}, LCL 96, pp.~187, 193

Pomponius Mela

Columella, {\em On Agriculture}, LCL 408, pp.~98--99, 148--149

Palladius, {\em Opus Agriculturae}

Aulus Gellius, {\em Attic Nights} II.22; LCL 195, pp.~268--269; LCL 200, pp.~172--175

Plutarch, {\em De facie quae in orbe lunae apparet}, XII, LCL 406

Diodorus Siculus, 12.36.2; 4.27.5; 3.60; 1.98

Philo, {\em On the Creation of the World}, LCL 226, pp.~90--91; {\em On Providence}, LCL 363, p.~493

Varro, {\em On Agriculture}, LCL 283, pp.~416-417, 466--467

Ovid, {\em Fasti}

Propertius

Nonnus, {\em Dionysiaca} XXXVIII.19--46, p.~95

P.Oxy.LIII 3710, col. II, 36-43, Haslam, 1986

Pythagoreans, Diogenes Laertius 8.48, 9.21 

Geminus 16.6--9

Cleomedes 1.7

Caesar, {\em Gallic War} 5.13

Strabo 1.4.1, 2.6

Cicero, {\em De natura deorum}

Apollonius of Rhodes, {\em Argonautica} I.661--704, ``As the seasons revolve, will they gather the harvest as soon as 
it is ripe?'', p.~19; III, p.~88

Firmicus Maternus, {\em Matheseos libri VIII}

Vettius Valens, {\em Anthologiarum libri novem}

Martianus Capella, {\em Marriage of Philology and Mercury} 6.590--8

Proclus, {\em Hypotyposis}

John Lydus, {\em De Mensibus}

Sallustius



\section{Macrobius}
Macrobius, {\em Saturnalia}, Book I: Greek and Roman calendars.




\section{Egyptian astronomy}
Sothic year



\section{Mesopotamian astronomy}


{\em Greek Anthology} V, pp.~97 ff., no. 132 ff.

Firmicus Maternus, {\em Matheseos} \cite{firmicus}

Averroes, On Aristotle's De generatione et corruptione: middle commentary and epitome, Volume 4, Parts 1-2, p.~138

{\em Enneads} III,7,11: measure enters into time through the celestial motions which, although they do not produce time, do make it manifest.


Al-Ghazli, {\em Tahafut} p. 31: and al-Nazzam. if the celestial bodies have different speeds then the numbers of their revolutions will necessarily have a numerical proportion,
and if the celestial motions are infinite then there will be division of the infinite into parts.

Genesis Rabbah 10,4: sidereal periods of the planets. 

{\em Talmud  Berachot} 59b

Book of Enoch

Basil the Great, {\em Hexaemeron}, Homily VI.

Hippolytus, {\em Refutatio Omnium Haeresium}, book IV.

Philo of Alexandria, {\em De aeternitate mundi}, 76--7

Origen, {\em Contra Celsum}, V.21.

Strobel \cite{strobel}

Tacitus, {\em Dialogus de oratoribus}, 16.7 \cite{tacitus}:

\begin{quote}
Hence we see that not much more than four hundred years has intervened between our own era and that of Demosthenes. If you measure this space of time by the frailty of human life, it perhaps seems long; if by the course of ages and by the thought of this boundless universe, it is extremely short and is very near us. For indeed, if, as Cicero says in his Hortensius, the great and the true year is that in which the position of the heavens and of the stars at any particular moment recurs, and if that year embraces twelve thousand nine hundred and ninety four of what we call years, then your Demosthenes, whom you represent as so old and ancient, began his existence not only in the same year, but almost in the same month as ourselves.
\end{quote}

De Muris, {\em De arte mensurandi} \cite{mensurandi}

Thomas Aquinas, {\em Summa Theologica}, part III, supplement,
question 91, article 2, reply to objection 8:

\begin{quote}
Although a heavenly body, so far as regards its nature, is equally inclined to every situation that it can possibly occupy, nevertheless in comparison with things outside it, it is not
equally inclined to every situation: but in respect of one situation it has a more noble disposition in comparison with certain things than in respect of another situation; thus in our regard the sun has a more noble disposition at daytime than at night-time. Hence it is probable, since the entire renewal of the world is directed to man, that the heaven will have in this renewal the most noble situation possible in relation to our dwelling there. Or, according to some, the heaven will rest in that situation wherein it was made, else one of its revolutions would remain incomplete. But this argument seems improbable, for since a revolution of the heaven takes no less than 36,000 years to complete, it would follow that the world must last that length of time, which does not seem probable. Moreover according to this it would be possible to know when the world will come to an end. For we may conclude with probability from astronomers in what position the heavenly bodies were made, by taking into consideration the number of years that have elapsed since the beginning of the world: and in the same way it would be possible to know the exact number of years it would take them to return to a like position: whereas the time of the world's end is stated to be unknown.
\end{quote}

Lucian, vol V.

Lucan

Fulgentius, {\em Mythologies}, I.18.

Sextus, vol. IV.

Seneca, Quaest. nat., 3.2.1

Hesiod, {\em Theogony}, 799

Herodotus 2.123.

Heraclitus, Fragment 100 \cite[pp.~294--305]{heraclitus}. 

Plato, {\em Phaedrus} 248e

Plato, {\em Politicus} 272d--e

Plato, {\em Timaeus} 37c--38e, 39b--e, 47a \cite[pp.~97--106, 115--117, 157]{timaeus}



\begin{quote}
The disciples of Pythagoras, too, and of Plato, although they appear to hold the incorruptibility of the world, yet fall into similar errors. For as the planets, after certain definite cycles, assume the same positions, and hold the same relations to one another, all things on earth will, they assert, be like what they were at the time when the same state of planetary relations existed in the world. From this view it necessarily follows, that when, after the lapse of a lengthened cycle, the planets come to occupy towards each other the same relations which they occupied in the time of Socrates, Socrates will again be born of the same parents, and suffer the same treatment, being accused by Anytus and Melitus, and condemned by the Council of Areopagus! The learned among the Egyptians, moreover, hold similar views, and yet they are treated with respect, and do not incur the ridicule of Celsus and such as he; while we, who maintain that all things are administered by God in proportion to the relation of the free-will of each individual, and are ever being brought into a better condition, so far as they admit of being so, and who know that the nature of our free-will admits of the occurrence of contingent events (for it is incapable of receiving the wholly unchangeable character of God), yet do not appear to say anything worthy of a testing examination.
\end{quote}

Philolaus of Croton \cite[p.~276]{philolaus}, Testimonium A22.

Manilius, {\em Astronomica} 

Phoenix, Van Den Broek \cite{phoenix}

Hermann of Carinthia, {\em De essentiis} \cite{essentiis}

Ptolemy, {\em Tetrabiblos}, I.2 \cite[pp.~15--17]{tetrabiblos}:

\begin{quote}
For in general, besides the fact that every science that deals with the quality of its subject-matter is conjectural and not to be absolutely affirmed, particularly one which is composed of many unlike elements, it is furthermore true that the ancient configurations of the planets, upon the basis of which we attach to similar aspects of our own day the effects observed by the ancients in theirs, can be more or less similar to the modern aspects, and that, too, at long intervals, but not identical, since the exact return of all the heavenly bodies and the earth to the same positions, unless one holds vain opinions of his ability to comprehend and know the incomprehensible, either takes place not at all or at least not in the period of time that falls within the experience of man; so that for this reason predictions sometimes fail, because of the disparity of the examples on which they are based.
\end{quote}

See also Ptolemy, {\em Almagest}, III.1, VII.2, VII.3.

Alcinous, {\em The Handbook of Platonism}, 14.6 \cite[pp.~24--25]{alcinous}:

\begin{quote}
The moon is regarded as being in second place as regards potency, and the rest of the planets follow each in proportion to its particular character. The moon creates the measure
of a month, by completing her own orbit and overtaking the sun in this space of time. The sun gives measure to the year; for in making the circuit of the zodiac it completes
the seasons of the year. The other planets each have their own revolutions, which are not accessible to the casual observer, but only to the experts. All these revolutions combine
to produce the perfect number and time, when all the planets come round to the same point and in an order such that, if one imagines a straight line dropped perpendicularly
from the sphere of the fixed stars to the earth, it would pass through the centre of each of them.
\end{quote}


Pliny, {\em Natural History} \cite[p.~193]{pliny}, II.VI.38--40: 

\begin{quote}
It [Venus] completes the circuit of the zodiac every 348 days, and according to Timaeus is never more than 46 degrees
distant from the sun. The star next to Venus is Mercury, by some called Apollo; it has a similar orbit, but is by no means similar in magnitude or
power. It travels in a lower circle, with a revolution nine days quicker, shining sometimes before sunrise and sometimes after sunset, but according to Cidenas
and Sosigenes never more than 22 degrees away from the sun. Consequently the course of these stars also is peculiar, and not shared by those above-mentioned [Saturn, Jupiter,
Mars, the sun]:
those are often observed to be a quarter or a third of the heaven away from the sun and travelling against the sun, and they all have other larger circuits of full revolution,
the specification of which belongs to the theory of the Great Year.
\end{quote}

Aulus Gellius, {\em Attic Nights}, XIV.I.18--19 \cite[pp.~9-11]{gellius}:

\begin{quote}
For he said that it was agreed among astrologers that those stars which they call ``wandering,'' which are supposed to determine the fate of all things,
beginning their course together, return to the same place from which they set out only after an innumerable and almost infinite number of years, so that there could be no continuity
of observation, and no literary record could endure for so long an epoch.
\end{quote}

Neugebauer \cite[p.~618, 749]{neugebauer}


Nemesius, {\em De Natura Hominis}, SVF 2.625 \cite[p.~309]{longsedleyI}:

\begin{quote}
The Stoics says that when the planets return to the same celestial sign, in length and breadth, where each was originally when the world was first formed,
at sets periods of time they cause conflagration and destruction of existing things. Once again the world returns anew to the same condition as before; and when
the stars are moving again in the same way, each thing which occurred in the previous period will come to pass indiscernibly [from its previous occurrence]. For
once again there will be Socrates and Plato and each one of mankind with the same friends and fellow citizens; they will suffer the same things and they will encounter
the same things, and put their hand to the same thing, and every city and village and piece of land return in the same way. The periodic return of everything occurs not
once but many times; or rather, the same things return infinitely and without end. The gods who are not subject to destruction, from their knowledge of this single
period, know from it everything that is going to be in the next periods. For there will be nothing strange in comparison with what occurred previously, but everything
will be just the same and indescrnible down to the smallest details.
\end{quote}
 
Chalcidius, cxviii, cxlviii.

Empedocles, \cite{obrien}, Minar \cite{minar}

Augustine, {\em The City of God}, XII.13, and see XII.12--14.

Isidore, {\em Etymologies}, \cite{isidore}, V.xxxvi.3: 

\begin{quote}
There are three kinds of years. The lunar year is of thirty days; the solstitial year, which contains twleve months;
or the great year, when all the heavenly bodies have returned to their original places, which happens after
very many solstitial years.
\end{quote}

Aratus, {\em Phaenomena} 454ff. \cite[pp.~149--150]{constellations}:

\begin{quote}
But mixed in among them are five other stars of a quite different nature, which circulate here and there through the twelve figures of the zodiac. In their case
it is no longer possible for you to work out their position by looking at other stars, because all of them constantly change their position. Long are the periods of their orbits,
and far distant from one another the signs of their renewed conjunction, and no longer do I have confidence in myself when it comes to them.
\end{quote}

Ker\'enyi \cite{kerenyi} XIII.4: ``At the end of Aeschylus's lost tragedy, {\em Promethus the Bringer of Fire}, it was stated that the Titan was bound for thirty thousand years. In those days this meant the world's longest period.'' Cf. scholion to {\em Prometheus Bound} 94, and 
Hyginus, {\em Astronomy} 2.15 \cite[pp.~54--55]{constellations}.

Bede, {\em De temporum ratione}, chapter 36 \cite[p.~104]{bede}: ``The year of the wandering stars is that in which each of them lights up the circuit of the zodiac, of which we spoke above. The Great Year is when all the planets return at one and the same time to the very places where they once simultaneously were.''

Macrobius, {\em Commentary on the Dream of Scipio} \cite[pp.~219--222]{macrobius}. Book two, chapter XI, 10--12:

\begin{quote}
A world-year will therefore be completed when all stars and constellations in the celestial sphere have gone from a definite place and returned to it, so that
not a single star is out of the position it previously held at the beginning of the world-year, and when the sun and moon and the five other planets are in the same positions
and quarters that they held at the start of the world-year. This, philosophers tell us, occurs every 15,000 years. Thus the lunar year is a month, the solar year twelve
months, the world-year is estimated to be 15,000 of the years we reckon by at the present. That must truly be called the revolving year which we measure
not by the return of a single star, the sun, but by the return of all stars in every quarter of the sky to their original positions, with all
the same configurations over all the sky; hence it is called the world-year, for it is proper to refer to the sky as the world.
\end{quote}

This is commentary on chapter VII of the {\em Somnium scipionis} of Cicero:

\begin{quote}
What difference does it make whether you will be remembered by those who came after you when there was no mention made of you by men
before your time? They were just as numerous and were certainly better men. Indeed, among those who can possibly hear of the name of Rome,
there is not one who is able to gain a reputation that will endure a single year. Men commonly reckon a year solely by the return of the sun, which is just one star;
but in truth when all the stars have returned to the same places from which they started out and have retored the same configurations over the great distances
of the whole sky, then alone can the returning cycle truly be called a year; how many generations of men are contained in a great year I scarcely dare say.
\end{quote}

Aelian, {\em Varia Historia}, 10.7 \cite[p.~319]{aelian}:

\begin{quote}
The astronomer Oenopides of Chios dedicated at Olympia
the famous bronze tablet on which he had inscribed the movements
of the stars for fifty-nine years, what he called the Great Year.
Note that the astronomer Meton of the deme Leuconoe set up
pillars and recorded on them the solstices. He claimed to have discovered
the Great Year and said it was nineteen years.
\end{quote}

Jean de Meung, {\em Roman de la Rose}, ll. 16801ff. \cite[pp.~282--283]{dahlberg}:

\begin{quote}
``I should not complain of heaven; it turns forever without hesitation and carries with it in its polished circle all the twinkling stars, powerful over all precious
stones. It goes along diverting the world; beginning its westward journey, it sets out from the east and does not stop turning backward, carrying all the wheels
that ascend against it to retard its movement. But they cannot hold it back enough that it ever, on their account, runs so slowly that it does not take
36,000 years to come exactly, with an entire circle completed, to the point where God first created it. It follows the extent of the path of the zodiac
with the great heavenly circle, which turns on it as on a form....
\end{quote}

Chaucer, {\em The Parliament of Fowls}, ll. 64--70: \cite[p.~386]{chaucer}:994

\begin{quote}
Than bad he hym, syn erthe was so lyte,\\
And dissevable and ful of harde grace,\\
That he ne shulde hym in the world delyte.\\
Thanne tolde he hym, in certeyn yeres space\\
That every sterre shulde come into his place\\
Ther it was first, and al shulde out of mynde\\
That in this world is don of al mankinde.
\end{quote}


Censorinus, {\em De die natali liber}, chapter 18 \cite{censorinus}.

{\em Timaeus} 39D, {\em Republic} VIII.546.

Thorndike \cite[pp.~398--423, Chapter XXV]{magicIII} discusses Oresme's writings against astrology.
See volume 2, pp.~203, 370, 418, 589, 710, 745, 895

Cicero, {\em De natura deorum} 2.20, 11.51; {\em De finibus} 11.102.

Kaye \cite[pp.~430--442]{kaye}

Oresme {\em Questiones super geometriam Euclidis}, q. 9.

Grant \cite[pp.~288--313]{grant1971}

Pseudo-Plutarch, {\em On fate} 569 \cite[p.~317]{moraliaVII}:

\begin{quote}
Although events are infinite, extending infinitely into the past and future, fate, which encloses them all in a cycle, is nevertheless not infinite but finite, as neither a law nor a formula
nor anything divine can be infinite. Further, you would understand what is meant if you should apprehend the entire revolution and the complete sum of time, ``when,''
as Timaeus says, ``the speeds of the eight revolutions, completing their courses relatively to one another, are measured by the circuit of the Same and Uniformly moving and come to a head.''
For
in this time, which is definite and knowable, everything in the heavens
and everything on earth whose production is necessary and due to 
celestial influences, will once again be restored to the same state
and once more be produced anew in the same way and manner.
\end{quote}


Proclus, in Dodds \cite[pp.~173, 302]{dodds}, Proposition 198.

Cassiodorus \cite{cassiodorus}

Oresme, {\em Medieval ratio theory vs compound medicines}.

\nocite{*}

\bibliographystyle{plain}
\bibliography{greatyear}

\end{document}