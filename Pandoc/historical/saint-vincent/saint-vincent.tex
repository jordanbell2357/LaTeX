\documentclass{amsart}
\usepackage{amsmath,amssymb,graphicx,subfig,mathrsfs,amsthm,enumitem}
\usepackage[T1]{fontenc}
\newtheorem{theorem}{Theorem}
\newtheorem{lemma}[theorem]{Lemma}
\newtheorem{proposition}[theorem]{Proposition}
\newtheorem{corollary}[theorem]{Corollary}
\theoremstyle{definition}
\newtheorem{definition}[theorem]{Definition}
\newtheorem{example}[theorem]{Example}
\begin{document}
\title{Gregory of Saint-Vincent and Zeno's paradoxes}
\author{Jordan Bell}
\email{jordan.bell@gmail.com}
\address{Department of Mathematics, University of Toronto, Toronto, Ontario, Canada}
\date{\today}

\maketitle

H. Bosmans, ``Saint-Vincent (Gr\'egoire de)'', Biographie nationale belge XXI, 141--171; H. Bosmans, ``Documents in\'edits sur Gr\'egoire de Saint-Vincent'', Annales de la soci\'et\'e scientifique de Bruxelles 27 (1903), 43--44; H. Bosmans, ``Lettre in\'edite de Christophe Grienberger sur Gr\'egoire de Saint-Vincent'', Annales de la Soci\'et\'e d'Emulation pour l'Etude de l'Histoire et des Antiquit\'es de la Flandre 1 (1913), 41--50; E. Sauvenier-Goffin, ``Note au sujet des manuscrits de H Bosmans relatifs \`a Gr�goire de Saint-Vincent'', Bull. Soc. Roy. Sci. Li\`ege 21 (1952), 301--302; E. Sauvenier-Goffin, ``Les manuscrits de Gr\'egoire de Saint-Vincent'', Bull. Soc. Roy. Sci. Li\`ege 20 (1951), 413-426; 427-436; 563-590; 711-732; 733-737; Charles Naux, ``L'Opus geometricum de Gr\'egoire de Saint-Vincent'', Rev. Histoire Sci. Appl. 15 (1962), 93-104; Charles Naux, ``Gr\'egoire de St. Vincent et les propri\'et\'es logarithmiques de l'hyperbole \'equilat\`ere'', Rev. Questions Sci. 143 (2) (1972), 209-221; H. van Looy, ``Chronologie et analyse des manuscrits math\'ematiques de Gregoire de Saint Vincent (1584-1667)'', Archivum Historium Societatis Jesu 49 (1980), 279-303; H. van Looy, ``Chronology and analysis of the mathematical manuscripts of Gregory of Saint Vincent (1584-1667)'', Historia Mathematica 11 (1984), 57-75; J.-P. Le Goff, ``De la m\'ethode dite d'exhaustion : Gr\'egoire de Saint-Vincent'', in La d\'emonstration
math\'ematique dans l'histoire, \'ed. IREM, Besan\c{c}on, 1989, 197-219; O. Van de Wyver, ``L'\'ecole de math�matiques des j\'esuites de la Province Flandro-Belge au XVIIe si\`ecle'', Archivum Historicum Societatis Jesu 97 (1990), 265-278; Jean Dhombres, ``Las progesiones infinitas: el papel del discreto y del continuo en el siglo XVII'', Llull 16/30 (1993); A. Meskens, ``Gregory of Saint Vincent : A Pioneer of the Calculus'', The Mathematical Gazette 78 (1994), 315-319.

Math\'ematique universelle, tome 2, Louis Bertrand Castel, p.~25, 47, 1758

G.W. Leibniz, Interrelations between Mathematics and Philosophy Archimedes Volume 41, 2015, pp 111-134 Date: 19 Apr 2015 Leibniz as Reader and Second Inventor: The Cases of Barrow and Mengoli Siegmund Probst

Evangelista Torricelli and the �Common Bond of Truth� in Greek Mathematics
Andrew Leahy
Mathematics Magazine
Vol. 87, No. 3 (June 2014), pp. 174-184

1673�1676. Arithmetische Kreisquadratur, Volume 6
 By Gottfried Wilhelm Leibniz

The Problem of the Earth's Shape from Newton to Clairaut: The Rise of,
John L. Greenberg, p.~242, 261.

John Napier: Life, Logarithms, and Legacy By Julian Havil

Newton et les origines de l'analyse: 1664-1666 - Page 305, Marco Panza

Torricelli

Before Newton: The Life and Times of Isaac Barrow,
Mordechai Feingold

The Oxford Handbook of the History of Mathematics
 edited by Eleanor Robson, Jacqueline Stedall
 
 Clavius and Mathematics in the Collegio Romano Agust�n Ud�as

The 16th century Iberian calculatores CP CALDERON - Revista de la Union, 1989 - inmabb.criba.edu.ar

{\em Institutions physiques}, p.~193, chapter IX, 1742, tome I; 1740, p.~184.

{\em Lettre in\'edite de Christophe Grienberger sur Gr\'egoire de Saint-Vincent'',
Annales de la Societe d'Emulation, I, 1913, p.~50.

Sets and integration An outline of the development 1972, pp 75-154 The integral from Riemann to Bourbaki Henri Lebesgue

Correspondence of James Jurin, 1684-1750, Editions Rodopi, p.~465.

Vieta, {\em Varia Responsa}, 1593, Opera, 347--435: infinite geometric sum. 

Las progresiones infinitas: el papel del discreto y del continuo en el siglo XVII J Dhombres - Llull: Revista de la Sociedad Espa�ola de Historia �, 1993 - dialnet.unirioja.es

Guarino Guarini and Universal Mathematics CS Roero - Nexus Network Journal, 2009 - Springer

Geometric progressions B Burn - BSHM Bulletin, 2007

The History of the Calculus and Its Conceptual, Boyer

Descartes geometric series: letter to Clerselier, June 1646, Descartes 1972, vol. 4, p.~442.

Costabel, P., {\em Descartes et la math\'ematique de l'infini, Historia Scientiarium, 26,
37--49, 1985.

I point out that Torricelli gave a geometric proof of the sum of a geometric series in his De dimensione Parabolae [1644]. For Torricelli�s proof, I refer to Panza [1992, 307--308].

1684-1691, D. T. Whiteside, p.~33

Huygens letters.

Westfall \cite[p.~33]{westfall}

The Arithmetic of Infinitesimals, John Wallis, p.~xx.

Gregory Saint-Vincent \cite[p.~102]{opus}, book II, part I, scholion to proposition LXXXVII. See
Whiteside \cite[p.~300]{newtonVIII}.

{\em Encyclop\'edie}, ``mouvement'', p.~464, ``le premier qui en''

Journal de Tr\'evoux, 1736, p.~1113

Leibniz: Letter to Foucher 1693; letter to Mr. de Bauval.

Chastellet \cite[p.~179, chap. IX]{chatelet}

Vincent Leotaud, {\em Examen circuli quadraturae}, 1654

Pierre Bayle, {\em Dictionnair\'e historique et critique}, 1696, ``Zenon d'El\'ee''

{\em Analise des infiniment petits}, Edmund Stone, p.~xlvii

Wilson \cite[pp.~74, 204]{wilson}

Thomas De Quincey, {\em The Collected Writings of Thomas De Quincey}, vol. 5, p.~349.

Duhem \cite[p.~56]{ariew1985}

Nicholas of Cusa \cite[p.~593]{lohr}

Mancosu \cite{mancosu}

\nocite{*}

\bibliographystyle{amsplain}
\bibliography{saint-vincent}

\end{document}