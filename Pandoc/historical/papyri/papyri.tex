\documentclass{article}
\usepackage{amsmath,amssymb,graphicx,subfig,mathrsfs,amsthm,enumitem,xfrac,flexisym}
\usepackage[polutonikogreek,english]{babel}
\newcommand{\Gk}[1]{\selectlanguage{polutonikogreek}#1\selectlanguage{english}}
\newcommand{\textoverline}[1]{$\overline{\mbox{#1}}$}
\begin{document}
\title{Numbers and fractions in Greek papyri}
\author{Jordan Bell}
\date{March 4, 2017}
\maketitle


 
\section{Hieratic papyri}
Rhind Mathematical Papyrus (RMP), BM 10057 and 10058. The translations I have consulted are 
Chace \cite{chace} and Clagett \cite{egyptian3}.

RMP $\sfrac{2}{n}$ entry for 
$\sfrac{2}{7}$ is the following \cite[p.~122]{egyptian3}:

\begin{quote}
\textbf{1/4} [of 7 is] 1 1/2 1/4, \textbf{1/28} [of 7 is] 1/4.

\begin{tabular}{llllll}
&1&7&&&\\
&1/2&3 1/2&&1&7\\
\textbackslash&1/4&1 1/2 1/4&&2&14\\
\textbackslash&4&28&1/4&4&28.
\end{tabular}
\end{quote}

$\sfrac{1}{4}$ of 7  is  $1 \; \overline{2} \; \overline{4}$.
$1 + \overline{2} + \overline{4} + R = 2$. $R=\overline{4}$.
$7\cdot 4 = 28$ so $\sfrac{1}{28}$ of $7$ is $\overline{4}$.
\[
2 = \sfrac{1}{4} \cdot 7 + R = \sfrac{1}{4} \cdot 7 + \sfrac{1}{28} \cdot 7.
\]
Then
\[
\sfrac{2}{7} = \overline{4} \; \overline{28}.
\]
Cf. Chace \cite[pp.~14--15]{chace}.

Chace \cite[pp.~5--6]{chace}:

\begin{quote}
Egyptian division might be described as a second kind of multiplication,
where the multiplicand and product were given to find the
multiplier. In the first kind of multiplication, the multiplier, being
given, can be made up as a combination of the multipliers that were
generally used, and the corresponding combination of products would
be the required product. When it was the product that was given along
with the multiplicand, various multipliers would be tried, 2, 10, and
combinations of these numbers, or combinations of the fractions
$\sfrac{2}{3}$, $\sfrac{1}{2}$, and $\sfrac{1}{10}$, and from the products thus obtained the Egyptians
would endeavor to make up the entire given product. When they succeeded in doing this the corresponding combination of multipliers
would be the required multiplier. But they were not always able to
get the given product at once in this way, and in such cases the complete
solution of the problem involved three steps: (a) multiplications
from which selected products would make a sum less than the required product but nearly equal to it; (b) determination of the remainder that must be added to this sum to make the complete product;
and (c) determination of the multiplier or multipliers necessary to produce this remainder. The multipliers used in the first and third steps made up the required multiplier. The second step was called completion and will be explained below. For the third step they had
a definite process which they generally used. The remainder, being a small number, would consist of one or more reciprocal numbers.
For one of these numbers the third step may be expressed by the
rule: To get the multiplier that will produce the reciprocal of a given
whole number as a product multiply the multiplicand by the number
itself and take the reciprocal of the result of this multiplication. If, for example, we wish to multiply 17 so as to get $\sfrac{1}{3}$, we take 3 times 17,
which is 51, and then we can say that $\sfrac{1}{51}$ of 17 equals $\sfrac{1}{3}$.
\end{quote}






\begin{tabular}{l p{2cm} l l}
&1&&7\\
\textbackslash&\textoverline{2}&\textbackslash&3 \textoverline{2}\\
&\textoverline{\textoverline{3}}&&4 \textoverline{\textoverline{3}}\\
\textbackslash&\textoverline{3}&\textbackslash&2 \textoverline{3}\\
&\textoverline{7}&&1\\
&\textoverline{14}&&\textoverline{2}\\
\textbackslash&\textoverline{42}&\textbackslash&\textoverline{6}\\
\hline
&&&\\
&\textoverline{2} \textoverline{3} \textoverline{42}&&3 \textoverline{2} 2 \textoverline{3} \textoverline{6} = 3 \textoverline{2} 2 \textoverline{2} = 6
\end{tabular}

Therefore $\sfrac{6}{7}=$ \textoverline{2} \textoverline{3} \textoverline{42}.








Egyptian Mathematical Leather Roll (EMLR), BM 10250.

RMP Problem 69, $1120:80=14$.

RMP 24: $19:8$, RMP 25 $16:3$, RMP 21 $4:15$.

Gillings \cite{gillings}.

Parker, DMP Problem 56 \cite{DMP}.









\section{Greek inscriptions}
Rhodes and Osborne \cite[p.~xiii]{rhodes}:

\begin{quote}
Temporary notices--lists of candidates for office, proposals for new legislation and so on--were written on whitewashed boards, and have not survived for us to read; for permanent
publication bronze or wood was sometimes used, but the normal medium was stone. For example, texts of a city's religious calendars, of its laws and decrees, and of its alliances with
other cities; schedules of work on a public building project, and accounts of public expenditure on the project; inventories of precious objects in the temple treasuries or of ships in the
dockyards; epigrams commemorating a famous victory; honours voted to a native or foreign benefactor; lists of office-holders and benefactors--all these and comparable documents
might be inscribed on stone for members of the public to see.
\end{quote}

{\em Corpus des inscriptions de Delphes} (CID) I.9 \cite[p.~3]{rhodes}, fifth/fourth century BC, Face A, ll.~19--23:

\begin{quote}
Resolved by the Labyadai. On the tenth of the month Boukatios [\Gk{Boukat'iou mhn`os dek['a]tai}],
in the archonship of Kampos, at the Assembly, by
182 votes [\Gk{s`um p'a[f]ois <ekat`on >ogdo'hkont[a] duo~in}].
\end{quote}

{\em CID} I.9 \cite[p.~5]{rhodes}, Face B, ll.~30--34:

\begin{quote}
Anyone who does not swear may not be a {\em tagos}. If someone serves as a {\em tagos} without swearing he is to pay a fine of 50 drachmas.
\end{quote}

\begin{quote}
If the seller is a slave-man or a slave-woman, he shall be beaten with fifty lashes with the whip [\Gk{plhg`as t~hi}] by the {\em archontes} commissioned in each matter.
\end{quote}

\cite[p.~141]{rhodes}, B(a), ll.~1--10:

\begin{quote}
Of the cities these failed to pay the interest that they should have paid during our magistracy and did not pay during the four years:
the people of Ceos 4,127 dr., $1 \sfrac{1}{2}$ obols; the people of Myconos
420 dr.; the people of Syros 4,900 dr.; the people of
Siphnos 2,089 dr. 2 obols; the people of Tenos 2,400 dr.; the Thermaians from Icaros 400 dr.;
the people of Paros 4 talents 1,830 dr.; the Oinaians from Icaros 1 talent 80 dr. Of the cities these did not pay the interest during the
four years of our magistracy during the archonships of Galleas, Gharisander, Hippodamas and Socratides at Athens and of Epigenes,
Galaios, Hippias, and Pyrraethus on Delos: the people of Naxos, 1 talent 3,600 dr.; the people of Andros 2 talents; the people of
Carystus 1 talent 2,400 dr.
\end{quote}

Stephen Lambert, P. J. Rhodes: Payments from the treasury of Athena, 410/409 BC, {\em IG} I\textsuperscript{3} 375:

\begin{quote}
In the seventh prytany, of Antiochis X, on the fifth of the prytany, was handed over to Dionysios of Kydathenaion and his fellow officials, for the two-obol grant, 1 tal; on the seventh
of the prytany, to the hellenotamiai Thrason of Boutadai and his fellow officials, for the two-obol grant, 1 tal. 1,232 dr. $3 \sfrac{1}{4}$ ob.; on the same day, to the hellenotamiai
Phalanthos of Alopeke and his fellow officials, fodder for the horses, 4 tal. (?); on the sixteenth of the prytany, to the hellenotamiai Proxenos 25 of Aphidna and his fellow officials,
1,534 dr. 3 ob.; on the twenty-fourth of the prytany, to the hellenotamiai Eupolis of Aphidna and his fellow officials, 5,400 dr.; on the twenty-seventh of the prytany, to the
hellenotamiai Kallias of Euonymon and his fellow officials, 1 tal. 2,565 dr. $4 \sfrac{1}{2}$ ob.
\end{quote}

Tod \cite{tod1912} and \cite{tod1937} on the acrophonic numerals.










\section{Greek papyri}
Hultsch \cite[p.~170]{MSRI}, symbols for numbers and fractions in Greek manuscripts.

Marrou \cite{marrou}

Smyly \cite[pp.~516--517]{smyly}:

\begin{quote}
The letters of the ordinary Greek alphabet, together with
\Gk{\textstigma}  \Gk{\qoppa} and \Gk{\textsampi}, were arranged in four horizontal rows
each of which contained nine symbols:
\begin{center}
\begin{tabular}{lllllllll}
\Gk{a}&\Gk{b}&\Gk{g}&\Gk{d}&\Gk{e}&\Gk{\textstigma}&\Gk{z}&\Gk{h}&\Gk{j}\\
\Gk{i}&\Gk{k}&\Gk{l}&\Gk{m}&\Gk{n}&\Gk{x}&\Gk{o}&\Gk{p}&\Gk{\textqoppa}\\
\Gk{r}&\Gk{\textsigma}&\Gk{t}&\Gk{u}&\Gk{f}&\Gk{q}&\Gk{y}&\Gk{w}&\Gk{\textsampi}\\
\Gk{>A}&\Gk{>B}&\Gk{>G}&\Gk{>D}&\Gk{>E}&\Gk{>\textStigma}&\Gk{>Z}&\Gk{>H}&\Gk{>J}
\end{tabular}
\end{center}

The symbols of the first row represent units (\Gk{mon'ades})
from 1 to 9; those of the second row tens (\Gk{dek'ades}) from 10
to 90; those of the third hundreds (\Gk{<ekatont'ades}) from 100 to
900 and those of the fourth thousands (\Gk{xili'ades}) from 1000
to 9000. The fourth row is a repitition of the first, the
symbols being differentiated by a large curved flourish at
the top, which makes them very prominent in papyrus
documents; this prominence is best attained in modern
printing by the employment of capital letters. Owing to
the loss of all Greek treatises on elementary Logistic we
are ignorant of the Greek names of these rows, but we
learn from Martianus Capella VII, 745 that in Latin they
were called {\em versus}: -- primus igitur versus est a monade
usque ad enneadem, secundus a decade usque ad nonaginta,
tertius vero ab hecatontade usque ad nongentos,
quartus qui et ultimus a mille usque ad novem milia, licet
nonnulli Graeci etiam \Gk{m'uria} adiecisse videantur. The word
is also found in Favonius Eulogius, {\em in somnium Scipionis},
p.~22: -- primi versus absolutio novenario numero
continetur.
\end{quote}

Martianus Capella, {\em The Marriage of Philology and Mercury} \cite[p.~286]{martianusII}:

\begin{quote}
The first series [{\em versus}] runs from the monad to the ennead; the second
from the decad to ninety; the third from one hundred to nine
hundred; the fourth and last from one thousand to nine thousand;
although some Greek writers appear to have included the myriad [\Gk{m'uria}].
\end{quote}

Smyly \cite[pp.~519--520]{smyly}:

\begin{quote}
When numbers were written in connexion with words
they were often distinguished from the latter by a horizontal
line drawn above them, \textoverline{\Gk{pkd}} = 124. In the course
of time this line was written with an upward inflection,
so that it came to present the appearance of an accent in
the form generally given in modern texts. This careless
method of writing is inconvenient in the extreme, as it
leads to great confusion between integers and fractions.
In purely arithmetical operations the distinguishing line
was unnecessary and accordingly was generally omitted.

The space at my disposal does not permit me to discuss
fully the treatment of fractions; but since they occur in
some of the examples which I shall have to quote, I shall
briefly indicate the various methods of writing them. Of
these there were three: -- 1) The fractions most often
employed were those with unity for numerator; the denominator
only was written down and they were distinguished
from integers by accents: thus \Gk{d'}=$\frac{1}{4}$, \Gk{{i'}{b'}}=$\frac{1}{12}$ etc.;
\Gk{b'} is an exception and denoted $\frac{2}{3}$, not $\frac{1}{2}$ for which a special
sign $\angle$ was employed. These fractions were convenient
in commercial transactions since the denominators
chosen generally corresponded with the natural
divisions of the measures or weights. 2) Occasionally
vulgar fractions, such as are used now, were employed,
the numerator being written on the line and the denominator
either above it, or twice repeated after it; e.g. $\frac{12}{13}$
would have been written \Gk{ib}\textsuperscript{\Gk{ig}} or
\Gk{ib{i'}{g'}{i'}{g'}}. 3) In astronomical
calculations sexegesimal fractions were employed:
they correspond to modern decimal fractions, but since
the scale of notation was 60 instead of 10, they had the
great advantage of being divisible by 3. From this system
are derived the minutes and seconds which are still employed
as fractions of an hour or a degree. The fraction 
$\frac{11}{12}$ might have been represented by 1) $\angle$ \Gk{{g'}{i'}{b'}} = 
$\frac{1}{2}+\frac{1}{3}+\frac{1}{12}$ or 2) \Gk{ia{i'}{b'}{i'}{b'}} or
3) \Gk{ne'} = $\frac{55}{60}$.
\end{quote}

Smyly \cite[p.~522]{smyly}:

\begin{quote}
A symbol $\complement$ was employed
whenever it was necessary, for any reason, to separate one
number from another; in subtraction it divided the 
subtrahend from the minuend; the remainder was introduced
by $\cap$, a cursive corruption of \Gk{L} the initial letter of
\Gk{loip'on} [adjective, ``remaining over'']. Thus \Gk{>Bfz} $\complement$ \Gk{>Aqpj} $\cap$ \Gk{wih} 
is the Greek equivalent of $2507-1689=818$.
\end{quote}

Smyly \cite[p.~525]{smyly}:

\begin{quote}
Suppose that it is required to divide 889
by 24, the process would, in my opinion, have been written 
down thus: -- \Gk{wpj} $\complement$ \Gk{kd >ep`i l / yk} $\cap$ \Gk{rxj} $\complement$ \Gk{>ep`i z / rxh} 
$\cap$ \Gk{a / lz >el'assw a}.

It is obvious that the first multiple of 24 must be 30;
then $24 \times 30=720$, subtract from 889, the remainder
is 169; the next figure of the quotient is seen to be 7,
$24 \times 7=168$ $\therefore$ the result is 37 and there is a remainder 1.
\end{quote}

Tod \cite[p.~128]{tod1950}:

\begin{quote}
 In numbers from 1000 to 9999 the thousands, placed first in order, are
 represented by the same signs as the units, but their value is normally indicated
 by the addition to the letter of some mark of differentiation, most frequently
 a slanting stroke prefixed to the numeral; thus the number 1754 is written
 as \Gk{/AUND}.
 But 10,000, 20,000 and higher multiples of 10,000 are never
 represented by \Gk{/I, /K}, etc.; instead the alphabetic system calls to its aid an
 acrophonic element, \Gk{M}, standing for \Gk{muri'as} or \Gk{m'urioi}. In order to avoid
 confusion with \Gk{M} = 40, the 
 single myriad may take the form of a monogram
 of \Gk{MU}, or it may have a small \Gk{A} placed immediately above it, denoting \Gk{m'ia muri'as}.
 Multiples of a myriad are similarly represented by \Gk{M} with small
  letters superposed, showing the number of myriads represented; 
 {\em e.g.} $\overset{\textrm{\Gk{B}}}{\textrm{\Gk{M}}}$ = 20,000,
 $\overset{\textrm{\Gk{R}}}{\textrm{\Gk{M}}}$ = 1,000,000.
\end{quote}

Tod \cite[p.~129]{tod1950}:

\begin{quote}
The numeral signs used to represent any given
 number are normally arranged, as was invariably the case with acrophonic
 numerals, in the descending order of value, {\em e.g.} \Gk{ria'} = 111. This rule is, I
 believe, everywhere and always observed in numbers exceeding 1000, though
 in some parts of the Greek world examples of a reversed or mixed order in the
 representation of numbers below 1000 are very common.
\end{quote}

Mithradates VI Eupator. No.~586 \cite{ward}: \Gk{BASILEWS JS MIJRADATOU EUPATOROS}.











P. Brem. 36 \cite[pp.~83--86]{pbrem} = Chrest. Wilck. 352 \cite[pp.~414--415]{wilcken}, \cite[p.~114]{naphtali}, 28 December AD 117:

\begin{quote}
To Apollonios, strategos of the Seven-Village Apollinopolite nome,
from Paphis son of Hon\={e}s and his brothers, of the village of Terythis.
Near the said village district there are registered in the name of our
father Hon\={e}s royal land as follows: taxed at [the rate of] $3 \frac{1}{2}$ artabas
[per aroura], $3 \frac{1}{12}$ arouras [\Gk{g \textoverline{ib} (>'arourai) g$\angle$}], and another $1 \frac{1}{2}$ arouras; at 2 artabas, $\frac{43}{64}$
aroura [\Gk{>'al(lai) dmo'iws (>'arourai) a$\angle$, ka`i >an(`a) b $\angle$\textoverline{hl[b]xd}}]; at $5 \frac{1}{12}$ artabas, $\frac{1}{16}$ aroura [\Gk{>an(`a) e \textoverline{ib} \textoverline{i\textstigma}}];
at $4 \frac{1}{12}$ artabas, 1 aroura [\Gk{>an(`a) d \textoverline{ib} (>'aroura) a}];
total, $6 \frac{47}{64}$
arouras [\Gk{g'i(nontai) (>'arourai) {\textstigma}$\angle$\textoverline{hi{\textstigma}lbxd}}].
\end{quote}









P. Mon. Epiph. II.619 \cite[p.~136]{epiphaniusII}, Monastery of Epiphanius, Thebes, uncertainly dated to sixth or seventh century AD, 

\begin{tabular}{lll}
\Gk{prwth}&\Gk{ogdoh}&\Gk{pempte}\\
\Gk{deuterh}&\Gk{ennath}&\Gk{kai deka-}\\
\Gk{trith}&\Gk{dekath}&\Gk{th}\\
\Gk{tetarth}&\Gk{endekath}&\\
\Gk{pempth}&\Gk{d[w]dekath}&\\
\Gk{ekth}&\Gk{[tr]is kai dekath}&\\
\Gk{ebdomh}&\Gk{te[ssa]res k[a]i}&\\
&\Gk{[dekat]h}&
\end{tabular}












Smith, {\em A Dictionary of Greek and Roman Antiquities}, 1891, s.v. logistica.

Vogel \cite{vogel}

Knorr \cite{knorr1982}

Benoit, Chemla and Ritter \cite{benoit}








Lahun Mathematical Papyri IV.2 \cite{griffith1898}, UC 32159, recovered in Kahun,
dated about 1800 BC. 






P. Sarga, Wadi Sarga, about 15 miles south of Asyut \cite{wadisarga}







The Akhmim Mathematical Papyrus, P. Cairo Inv.~10758 \cite{baillet}






Zalateo \cite{zalateo}








Chester Beatty codex Ac. 1390 \cite[pp.~33--56]{beatty13}, Dishna, ca. fourth century AD, Mathematical School Exercises in Greek.
Extant Page Three, ll.~19--23 \cite[pp.~54--55]{beatty13}:

\begin{quote}
$3' \, 14' \, 42'$ (is) 200 talents [\Gk{t`o {g\textprime\textprime} \textoverline{idm}b (t'alanta) S}].
How much is the $42'$ part [\Gk{p'osou t`o \textoverline{m}b m'eros}]? Do it thus.
In what \Gk{y~hfos} does $3' \, 14' \, 42'$ [\Gk{{g\textprime\textprime} \textoverline{idm}b}] go? (It is) from (the table of) 7ths. $3/7$ is
$3' \, 14' \, 42'$ [\Gk{>ap`o z t~w(n) g t`o \textoverline{z} {g\textprime\textprime} \textoverline{idm}b}]. Take the last part, $42'$ [\Gk{\textoverline{mb}}].
$3' \, 14' \, 42'$  [\Gk{{g\textprime\textprime} \textoverline{idm}b}] (of 42 is) 18 [\Gk{\textoverline{ih}}]. (Dividing) the
aforementioned (200) talents by 18 [\Gk{\textoverline{ih}}] equals $11 \,\, 9'$ talents [\Gk{(t'alanta) ia{$\langle$}{j'}{$\rangle$}}]. Number (of talents) 466 $2/3$ [\Gk{qx{\textstigma}b'}].
\end{quote}

Brashear \cite[p.~53]{beatty13} remarks that there is not a satisfactory definition of the term \Gk{y~hfos} (pebble, in particular a pebble used in reckoning), and that it has been translated by other editors 
variously as ``calcul'' (calculation, reckoning) and Tabelle (table).

$\overline{3}$ of 42 is 14. $\overline{14}$ of 42 is 3. Then
$\overline{3} \; \overline{14} \; \overline{42}$ of 42 is
$14+3+1=18$.
On the one hand $\overline{3} \; \overline{14} \; \overline{42}$ of the sought quantity $x$ is 
200 talents, and on the other hand $\overline{3} \; \overline{14} \; \overline{42}$ of 42 is 
18. Therefore the proportion 200 talents : 18 is the same as the proportion
$x$ : 42. 
$466 \; \overline{\overline{3}}$, \Gk{ux{\textstigma}b'}, in Brashear's translation is right,
$666 \; \overline{\overline{3}}$, \Gk{qx{\textstigma}b'}, in the Greek text is wrong.




Crawford \cite{crawford}




Louvre \cite{louvre1973}, \cite{louvre1977}, \cite{louvre1983}, \cite{louvre1984}








T. Varie 7 = P. Vat. gr. 55 A \cite[pp.~40--43]{Tvarie}, Oxyrhynchites?, seventh century AD. Table of seventeenths (Col. I) and table of nineteenths (Col. II). The table of seventeenths is the following:

\begin{tabular}{lll}
\multicolumn{3}{l}{\Gk{t`o iz [>en] y'hfwn tnb} L \Gk{\textoverline{g iz l}d na t~hs mi~as t`o iz iz}}\\
\multicolumn{3}{l}{\Gk{(o<'utws);}}\\
\Gk{t~wn b}&\Gk{ib na xh}&\Gk{ib a g' \textoverline{ib} na t`o g' xh t`o} d\textsuperscript{$-$}\\
\Gk{t~wn g}&\Gk{ib iz na xh}&\Gk{ib a g' \textoverline{ib} iz a na t`o g' xh t`o} d\\
\Gk{t~wn d}&\Gk{ib ie iz xh pe'}&\Gk{ib a g' \textoverline{ib} ie a i l iz a xh t`o} d \Gk{pe t`o e}\\
\Gk{t~wn e}&d \Gk{ld xh}&d \Gk{d} d \Gk{ld t`o} L \Gk{xh t`o} d\textsuperscript{$-$}\\
\Gk{t~wn \textstigma}&\Gk{g' na}&\Gk{g' e} w \Gk{na t`o g'}\\
\Gk{t~wn z}&\Gk{g' iz na}&\Gk{g' e} w \Gk{iz a na t`o g'}\\
\Gk{t~wn h}&\Gk{g' ie iz pe'}&\Gk{g' e} w \Gk{ie a i l iz a pe t`o e\textsuperscript{$-$}}\\
\Gk{t~wn j}&L \Gk{ld'}&L \Gk{h} L \Gk{ld t`o} L\textprime\\
\Gk{t~wn i}&L \Gk{iz ld'}&L \Gk{h} L \Gk{iz a ld t`o} L\\
\Gk{t~wn ia}&L \Gk{ib ld' na xh\textsuperscript{$-$}}&L \Gk{h} L \Gk{ib ag' \textoverline{ib} ld t`o} L \Gk{na t`o g' xh t`o} d\\
\Gk{t~wn ib}&L \Gk{ib iz ld' na xh\textsuperscript{$-$}}&L \Gk{h} L \Gk{ib ag' \textoverline{ib} iz a ld t`o} L \Gk{na t`o g' xh t`o} d\\
\Gk{t~wn ig}&L d\textsuperscript{$-$} \Gk{xh\textsuperscript{$-$}}&L \Gk{h} L d\textsuperscript{$-$} \Gk{d} d\textsuperscript{$-$} \Gk{xh t`o} d\textsuperscript{$-$}\\
\Gk{t~wn id}&L d\textsuperscript{$-$} \Gk{iz xh\textsuperscript{$-$}}&L \Gk{h} L d\textsuperscript{$-$} \Gk{d} d\textsuperscript{$-$} \Gk{iz a xh t`o} d\textsuperscript{$-$}\\
\Gk{t~wn ie}&L \Gk{g'} \textoverline{\Gk{ld' na}}&L \Gk{h} L \Gk{g' e} w \Gk{ld t`o} L \Gk{na t`o g'}\\
\Gk{t~wn i\textstigma}&L \Gk{g' \textoverline{iz ld'} na}&L \Gk{h} L \Gk{g' e} w \Gk{iz a ld t`o} L \Gk{n[a t`o g']}\\
\Gk{t~wn iz}&\Gk{a}&\Gk{a iz}
\end{tabular}

$\sfrac{6000}{17}$: $352 \; \overline{2} \; \overline{3} \; \overline{17} \; \overline{34} \; \overline{51}$.

$\sfrac{1}{17}$: $\overline{17}$.

$\sfrac{2}{17}$: $\overline{12} \; \overline{51} \; \overline{68}$.
$\overline{12} \cdot 17 = 1 \; \overline{3} \; \overline{12}$.
$\overline{51} \cdot 17 = \overline{3}$.
$\overline{68} \cdot 17 = \overline{4}$.

$\sfrac{3}{17}$: $\overline{12} \; \overline{17} \; \overline{51} \; \overline{68}$. 
$\overline{12} \cdot 17 = 1 \; \overline{3} \; \overline{12}$.
$\overline{17} \cdot 17 = 1$.
$\overline{51} \cdot 17 = \overline{3}$.
$\overline{68} \cdot 17 = \overline{4}$.

$\sfrac{4}{17}$: $\overline{12} \; \overline{15} \; \overline{17} \; \overline{68} \; \overline{85}$.
$\overline{12} \cdot 17 = 1 \; \overline{3} \; \overline{12}$.
$\overline{15} \cdot 17 = 1 \; \overline{10} \; \overline{30}$.
$\overline{17} \cdot 17 = 1$.
$\overline{68} \cdot 17 = \overline{4}$.
$\overline{85} \cdot 17 = \overline{5}$.

$\sfrac{5}{17}$: $\overline{4} \; \overline{34} \overline{68}$.
$\overline{4} \cdot 17 = 4 \; \overline{4}$.
$\overline{34} \cdot 17 = \overline{2}$.
$\overline{68} \cdot 17 = \overline{4}$.

$\sfrac{6}{17}$: $\overline{3} \; \overline{51}$.
$\overline{3} \cdot 17 = 5 \; \overline{\overline{3}}$.
$\overline{51} \cdot 17 = \overline{3}$.

$\sfrac{7}{17}$: $\overline{3} \; \overline{17} \; \overline{51}$.
$\overline{3} \cdot 17 = 5 \; \overline{\overline{3}}$.
$\overline{17} \cdot 17 =1$.
$\overline{51} \cdot 17 = \overline{3}$.

$\sfrac{8}{17}$: $\overline{3} \; \overline{15} \; \overline{17} \; \overline{85}$.
$\overline{3} \cdot 17 = 5 \; \overline{\overline{3}}$.
$\overline{15} \cdot 17 = 1 \; \overline{10} \; \overline{30}$.
$\overline{17} \cdot 17 = 1$.
$\overline{85} \cdot 17 = \overline{5}$.

$\sfrac{9}{17}$: $\overline{2} \; \overline{34}$.
$\overline{2} \cdot 17 = 8 \; \overline{2}$.
$\overline{34} \cdot 17 = \overline{2}$.

$\sfrac{10}{17}$: $\overline{2} \; \overline{17} \; \overline{34}$.
$\overline{2} \cdot 17 = 8 \; \overline{2}$.
$\overline{34} \cdot 17 = \overline{2}$.

$\sfrac{11}{17}$: $\overline{2} \; \overline{12} \; \overline{34} \; \overline{51} \; \overline{68}$.
$\overline{2} \cdot 17 = 8 \; \overline{2}$.
$\overline{12} \cdot 17 = 1 \; \overline{3} \; \overline{12}$.
$\overline{34} \cdot 17 = \overline{2}$.
$\overline{51} \cdot 17 = \overline{3}$.
$\overline{68} \cdot 17 = \overline{4}$.

$\sfrac{12}{17}$: $\overline{2} \; \overline{12} \; \overline{17} \; \overline{34} \; \overline{51} \; \overline{68}$.
$\overline{2} \cdot 17 = 8 \; \overline{2}$.
$\overline{12} \cdot 17 = 1 \; \overline{3} \; \overline{12}$.
$\overline{17} \cdot 17 = 1$.
$\overline{34} \cdot 17 = \overline{2}$.
$\overline{51} \cdot 17 = \overline{3}$.
$\overline{68} \cdot 17 = \overline{4}$.

$\sfrac{13}{17}$: $\overline{2} \; \overline{4} \; \overline{68}$.
$\overline{2} \cdot 17 = 8 \; \overline{2}$.
$\overline{4} \cdot 17 = 4 \; \overline{4}$.
$\overline{68} \cdot 17 = \overline{4}$.

$\sfrac{14}{17}$: $\overline{2} \; \overline{4} \; \overline{17} \; \overline{68}$.
$\overline{2} \cdot 17 = 8 \; \overline{2}$.
$\overline{4} \cdot 17 = 4 \; \overline{4}$.
$\overline{17} \cdot 17 = 1$.
$\overline{68} \cdot 17 = \overline{4}$.

$\sfrac{15}{17}$: $\overline{2} \; \overline{3} \; \overline{34} \; \overline{51}$.
$\overline{2} \cdot 17 = 8 \; \overline{2}$.
$\overline{3} \cdot 17 = 5 \; \overline{\overline{3}}$.
$\overline{34} \cdot 17 = \overline{2}$.
$\overline{51} \cdot 17 = \overline{3}$.

$\sfrac{16}{17}$: $\overline{2} \; \overline{3} \; \overline{17} \; \overline{34} \; \overline{51}$.
$\overline{2} \cdot 17 = 8 \; \overline{2}$.
$\overline{3} \cdot 17 = 5 \; \overline{\overline{3}}$.
$\overline{17} \cdot 17 = 1$.
$\overline{34} \cdot 17 = \overline{2}$.
$\overline{51} \cdot 17 = \overline{3}$.

$\sfrac{17}{17}$: 1.
$1 \cdot 17= 17$.









P. K\"oln VII.325, Inv. 20839 C 2-21 v \cite[pp.~166--174]{koln7}, Heracleopolites, end of sixth century/seventh century, nine arithmetic exercises.
First example, ll.~1--4:

\begin{quote}
\Gk{>ap`o t~wn g}w \Gk{t`o xe g'i(netai) m'or(ai) nb o pd {\textqoppa}a.
>'allh m'ejwtos;
>ap`o t(~wn) g}w \Gk{>ep`i g g'i(netai) ia, >ap`o t(~wn) xe >ep`i g g'i(netai)
r{\textqoppa}e, >ap`o t(~wn) ia
>ep`i z g'i(netai) oz,
>ap`o t(~wn) r{\textqoppa}e >ep`i z g'i(netai)} \textsubscript{,}{\Gk{a}}\Gk{txe}.
\Gk{tout'estin oz e>is} \textsubscript{,}{\Gk{a}}\Gk{txe, g'i(netai) m'or(ia) \textoverline{n}\textoverline{b}}
$\downharpoonright$ \Gk{k{\textstigma}} d{\textprime} \Gk{o} $\downharpoonright$ \Gk{ij} $\angle$ \Gk{pd}
${\langle}\downharpoonright{\rangle}$ \Gk{i{\textstigma}} d{\textprime} \Gk{\textoverline{\textqoppa}\textoverline{a}}
$\downharpoonright$ \Gk{[ie]}

$3 \; \overline{\overline{2}}: 65$ makes \textoverline{52} \textoverline{70} \textoverline{84} \textoverline{91}.
Another method: 
$3 \; \overline{\overline{2}} \cdot 3$ makes $11$,
$65 \cdot 3$ makes 195,
$11 \cdot 7$ makes 77,
$195 \cdot 7$ makes 1365.
Result, $77:1365$ makes \textoverline{52} $\downharpoonright$
26 \textoverline{4} \textoverline{70}
$\downharpoonright$
19 \textoverline{2} \textoverline{84}
$\downharpoonright$
16 \textoverline{4}, \textoverline{91} $\downharpoonright$ 15.
\end{quote}

w stands for a glyph used in the edition that looks like the Coptic letter shay, and denotes
two thirds. $\downharpoonright$ stands for a glyph used in the edition.




Second example, ll.~5--6:

\begin{quote}
\Gk{t(~wn)} \textsubscript{,}{\Gk{a}} \Gk{t`o} \textsubscript{,}{\Gk{a}}\Gk{a} \Gk{g'i(netai) m'or(ia)} $\angle$ \Gk{g ig kb l[g] oz [$\angle$ f$\angle$]\textoverline{g}
tlg}w \Gk{\textoverline{ig} oz \textoverline{k}\textoverline{b} \quad me}$\angle$
\qquad
\Gk{\textoverline{l}\textoverline{g} l {g'} \textoverline{o}\textoverline{z} ig}
\end{quote}



P. Oxy. IV.669 \cite[pp.~116--121]{POxyIV}, Oxyrhynchus, end of the third century AD, Metrological Work, ll.~1--20:

\begin{quote}
The schoenium used in land-survey has 8 eighths [\Gk{>'wgdoa h}], and the eighth [\Gk{>'ogdoon}] has 12 cubits [\Gk{p'hqis ib}],
so that the schoenium used in land-survey has 96 cubits [\Gk{phq~wn {\textqoppa}{\textstigma}}], while the \dots schoenium has
100 cubits [\Gk{phq~wn r}]. The linear cubit is that which is measured by length alone, the plane
cubit is that which is measured by length and breadth; the solid cubit is that which
is measured by length and breadth and depth or height. The \dots building cubit contains
100 plane cubits [\Gk{[>embadiko`us p'h]qis r}]. \Gk{Na'ubia} are measured by the \Gk{x'ulon}; the royal \Gk{x'ulon} contains 3 [\Gk{g}] cubits,
18 [\Gk{ih}] \Gk{palaista'i}, 72 [\Gk{ob}] \Gk{d'aktuloi}, while the \dots \Gk{x'ulon} contains $2 \frac{2}{3}$ cubits [\Gk{b{b'}}], 16 [\Gk{i{\textstigma}}] \Gk{palaista'i} and
64 [\Gk{xd}] \Gk{d'aktuloi}; so that the schoenium used in land-survey contains 32 [\Gk{lb}] royal \Gk{x'ula} and 36 [\Gk{l{\textstigma}}]
\dots \Gk{x'ula}.
\end{quote}

ll.~31--41:

\begin{quote}
2 [\Gk{b}] \Gk{palaista'i} make a \Gk{liq'as}, 3 [\Gk{g}] \Gk{palaista'i} a \Gk{spijam'h}, 4 [\Gk{d}] \Gk{palaista'i} an (Egyptian?)
foot [\Gk{po`us}], 5 [\Gk{[e]}] a cloth-weaver's cubit [\Gk{p~hqus lino\={u}fik`os}] \dots, 6 [\Gk{\textstigma}] \Gk{palaista'i} a public and a carpenter's cubit, 7 [\Gk{[z]}] \Gk{palaista'i}
a Nilometric cubit, 8 [\Gk{h}] \Gk{palaista'i} a \dots cubit, 10 [\Gk{i}] \Gk{palaista'i} a \Gk{b~hma}, which is the distance
of the outstretched feet. 3 cubits [\Gk{g p'hq[eis]}] make a public \Gk{x'ulon}, 4 [\Gk{d}] cubits an \Gk{>orgui'a}, which is the
distance of the outstretched hands. . . cubits make a \Gk{k'alamos}, $6 \frac{2}{3}$ [\Gk{{\textstigma}{b'}}] an \Gk{>'akaina}.
\end{quote}










Hunt and Edgar \cite[pp.~2--3]{LCL266}, No.~1, P. Eleph.~1, Marriage Contract, 311 BC, ll.~1--2:

\begin{quote}
\Gk{>Alex'androu to~u >Alex'androu basile'uontos >'etei
<ebd'omwi, Ptolema'iou satrape'uontos >'etei tesareskaidek'atwi, mhn`os D'iou.}

In the 7th year of the reign of Alexander son
of Alexander, the 14th year of the satrapship of
Ptolemy, in the month Dius.
\end{quote}

Hunt and Edgar \cite[pp.~4--7]{LCL266}, No.~2, P. Tebt. 104, Marriage Contract. 
Hunt and Edgar divide this into a summary, ll.~1--4, and text of the contract, written in a second hand.
Summary:

\begin{quote}
Year 22, Mecheir 11 [\Gk{(>'Etous) kb Meq(e`ir) ia}]. Philiscus son of
Apollonius, Persian of the Epigone, acknowledges to
Apollonia also called Kellauthis, daughter of Heraclides,
Persian, having with her as guardian her
brother Apollonius, that he has received from her
in copper money 2 talents 4000 drachmae [\Gk{(t'alanta) b ka`i (draqm`as)
>D}], the
dowry for herself, Apollonia, agreed upon with him \dots Keeper of the contract: Dionysius.
\end{quote}

In the text of the contract, ll.~5--13:

\begin{quote}
In the 22nd year [\Gk{>'etous deut'erou ka`i e>ikosto~u}] of the reign
of Ptolemy also called Alexander, the god Philometor,
the priest of Alexander and the other priests
being as written in Alexandria, the 11th of the
month Xandicus, which is the 11th of Mecheir [\Gk{mhn`os Xandik[o]~u <endek'athi M[eqe`i]r <endek'athi}], at
Kerkeosiris in the division of Polemon of the
Arsinoite nome. Philiscus son of Apollonius, Persian
of the Epigone, acknowledges to Apollonia, also
called Kellauthis, daughter of Heraclides, Persian,
having with her as guardian her brother Apollonius, 
that he has received from her in copper money 2
talents 4000 drachmae [\Gk{t'alanta d'uo ka`i draqm`as tetrakisqil'ia[s]}], the dowry for herself,
Apollonia, agreed upon with him.
\end{quote}

Hunt and Edgar \cite[pp.~10--11]{LCL266}, No.~3, B.G.U. 1052, 13 BC, Marriage Contract, l.~34:

\begin{quote}
\Gk{(>'etous) iz Ka'isaros Farmo~uji \textoverline{k}.}

The 17th year of Caesar, Pharmouthi 20.
\end{quote}

Hunt and Edgar \cite[pp.~96--99]{LCL266}, No.~32, P. Oxy 95, AD 129, Sale of a Slave, ll.~1--3:

\begin{quote}
The 13th year [\Gk{>'Etous triskaidek'atou}] of the Emperor Caesar Trajanus
Hadrianus Augustus, Pauni 29 [\Gk{Pa~uni \textoverline{kj}}], at Oxyrhynchus in
the Thebaid.
\end{quote}

Hunt and Edgar \cite[pp.~156--157]{LCL266}, No.~52, P. Ryl. 157, AD 135, Division of Property Held on Lease, ll.~3--11:

\begin{quote}
We acknowledge that we have divided
between each other at this present time the domain-land
vineyard which we hold on lease in the village
Thrage in the toparchy of the Upper Suburb, being
part of the holding of Xenon \dots whatever the 
extent of its acreage is, containing an orchard, and
that Soeris also called Souerous has been allotted
the southern portion, having forthwith paid to Eudaemonis
for the choice 210 silver drachmae [\Gk{>argur'iou draqm`as diakos'ias d'eka}]. Its measurements
are \dots beginning from south to north
inside the wall of the plot in the second stade following
the western wall 1 schoenion, from the western
wall eastward for a certain distance $1\frac{55}{64}$ schoenia [\Gk{sqoin'ion <`en <'h[mi\textsigma]u t'etarton <ekkaid'ekaton
duotriakost`on [tetrakaiexhk]ost'on}],
and
from this latter boundary turning off to the north
$\frac{23}{32}$ of a schoenion [\Gk{sqoin'iou <'hmisu >'ogdo[o]n [<e]kkaid'ekaton duotria[kost`on]}], and from this boundary eastward
up to the eastern wall which is the boundary of the
whole plot $\frac{13}{23}$ of a schoenion [\Gk{sq[o]in'iou t'etarton >'ogdoon duotriakost'on}].
\end{quote}

Hunt and Edgar \cite[pp.~172--173]{LCL266}, No.~57, P. Oxy 270, AD 94, Indemnification of a Surety, ll.~17--27:

\begin{quote}
in the area of Seruphis from the holding of
Demetrius the Milesian $3\frac{1}{2}$ arurae [\Gk{>aro'urais tris`i <hm'isei}] of catoecic and
purchased land, and from the same holding out of 12
arurae [\Gk{>arour~wn d'eka d'uo}] of catoecic and purchased land the 5 arurae [\Gk{>aroura~is p'ente}]
remaining after the 7 [\Gk{>aro'uras <ept`a}] which she mortgaged to Taaphunehis daughter of Thonion, and from the holding of Callias a third share [\Gk{tr'it~w| m'erei}] of 8 arurae [\Gk{>arour~wn >okt'o}] of catoecic and purchased land, making
$2\frac{2}{3}$ arurae [\Gk{>'arourai d'uo d'imoiron}], and in the area of Syron Kome from the holding of Heracleides together with that of Alexander
$6\frac{3}{4}$ arurae [\Gk{>aro'urais <`ex <hm'isei tet'artw|}] of catoecic land, and from the holding of Alexander and others $6\frac{1}{2}$ arurae [\Gk{>aro'urais <`ex <hm'isei}] of catoecic and purchased land, making a total of $24 \frac{5}{12}$ arurae [\Gk{>aro'urais e>'ikosi t'essaroi tr'itw| dwdekatw|}] of catoecic land and land purchased for conversion into catoecic
\end{quote}

Hunt and Edgar \cite[pp.~186--189]{LCL266}, No.~63, P. Ryl. 177, AD 246, Loan on Mortgage, ll.~4--5:

\begin{quote}
We acknowledge that we have received from you by hand out of
your house a loan at interest \dots of one thousand
nine hundred and twenty silver drachmae, total
1920 silver dr. [\Gk{>argur'iou draqm`as qeil'ias >ennakos'ias e>'ikosi, g('inontai)
>arg(ur'iou) (draqma`i) >A{\textsampi}k}]
\end{quote}

Hunt and Edgar \cite[pp.~270--273]{LCL266}, No.~89, P.S.I. 333, 256 BC, From Promethion to Zenon:

\begin{quote}
Promethion to Zenon greeting. I suffered
anxiety when I heard of your long protracted illness,
but now I am delighted to hear that you are convalescent
and already on the point of recovery.
I myself am well. I previously gave your agent
Heraclides 150 drachmae in silver [\Gk{>argur'iou (draqm`as) \textoverline{rn}}]
 from your account,
 as you wrote to me to do, and he is bringing you now
 10 {\em hins} [\Gk{<'inia \textoverline{i}}]
  of perfume in 21 vases [\Gk{[>al]ab'astrois \textoverline{ka}}] which have been
  sealed with my finger-ring. For though Apollonius
  wrote to me to buy and give him also
  300 [\Gk{t}] wild
  pomegranate wreaths, I did not manage to give
  him these at the same time, as they were not ready,
  but Pa \dots will bring them to him at Naucratis;
  for they will be finished before the 30th [\Gk{l}]. I have
  paid the price both of these and of the perfume from
  your account, as Apollonius wrote. I have also paid
  a charge of 10 drachmae
  [\Gk{(draqm`as i}] in copper for the boat
  in which he is sailing up. And 400 drachmae in
 silver [\Gk{>argur'iou (draqma`i) u}] have been paid to Iatrocles for the papyrus
 rolls which are being manufactured in Tanis for
 Apollonius. Take note then that these affairs have
 been settled thus. And please write yourself if
 ever you need anything here. Goodbye. Year 29,
 Choiach 28 [\Gk{(>e'tous) kj Qo'iaq k\textoverline{h}}]. (Addressed) To Zenon. (Docketed)
 Year 29, Peritius 3 [\Gk{(>e'tous) kj Perit'iou g}]. Promethion about what he has paid.
\end{quote}

Hunt and Edgar \cite[pp.~406--409]{LCL284}, No.~346, P.S.I. 488, 257 BC, Tender for Repairing Embankments:

\begin{quote}
To Apollonius the dioecetes greeting from Harmais.
At the city of Memphis the various embankments
measure 100 schoenia [\Gk{sqoin'iwn r}], being as follows: those of the
Syro-Persian quarter 12 schoenia [\Gk{sqoin'iwn \textoverline{ib}}], of Paasu 7 [\Gk{\textoverline{z}}], those
above the quay of Hephaestus and those below 4 [\Gk{\textoverline{d}}],
those about the city together with the palace 23 [\Gk{\textoverline{kg}}],
those of the Carian quarter \dots, of the Hellenion 3 [\Gk{\textoverline{g}}],
beyond Memphis those on the west of the royal
garden 20 [\Gk{\textoverline{k}}] and on the east \dots and on the north
5 schoenia 30 cubits [\Gk{\textoverline{e} (phq~wn) \textoverline{l}}]. For the heaping up of these
embankments the sum given in the 28th year [\Gk{\textoverline{kh} (>'etei)}] was
1 talent 5500 drachmae [\Gk{(t'alanton) a (draqma`i) >Ef}], when the rise of the river
was 10 cubits 3 palms $1\frac{1}{6}$ finger-breadths [\Gk{ph(q~wn) i pa(laist~wn) g da(kt'ulou) a{\textstigma'}}], and in the
27th year [\Gk{kz (>'etei)}] the sum given was 1 talent 1300 drachmae [\Gk{(t'alanton) a (draqma`i) >At}],
when the river rose 10 cubits 6 palms $2\frac{2}{3}$ finger-breadths [\Gk{p'h(qeis) i pa(laist`as) \textstigma
da(kt'ulous) bb'}].
I now undertake to heap up the same
embankments beginning from their bases to the
height of a rise of 12 cubits [\Gk{ph(q~wn) ib}], to the satisfaction of the
oeconomus and the chief engineer, if I receive 1 talent [\Gk{(t'alanton) a}]
from the Treasury. And according to the usual
practice we shall be furnished with mattocks, which
we will return. Farewell.
\end{quote}

Hunt and Edgar \cite[pp.~424--425]{LCL284}, No.~354, P. Giess. 4, AD 118, Offer to Lease State Lands at a Reduced Rate:

\begin{quote}
As our lord Hadrianus Caesar among his other indulgences has ordained that Crown land, public land, and domain land shall be cultivated
at rents corresponding to their various values and not in accordance with the old order, and as we have been overburdened for a long time with public dues on Crown
land in the area of the metropolis, Pseathuris the younger paying on $8\frac{1}{2}$ [\Gk{h$\angle$}] arurae at the rate of
$2\frac{1}{12}$ artabae [\Gk{b {i'}{b'}}] for each and on $\frac{7}{32}$ [\Gk{h' {i'}{\textstigma'} {l'}{b'}}] of an arura at the rate of $3\frac{1}{12}$ [\Gk{g {i'}{b'}}],
and Senpachompsais daughter of Pseathuris on $1\frac{11}{16}$ [\Gk{a$\angle$ h' {i'}{\textstigma'}}] arurae at the rate of $4\frac{1}{12}$ [\Gk{d {i'}{b'}}] artabae, total $10\frac{3}{8}$ [\Gk{i d' h'}] arurae,
having just now obtained the indulgence mentioned we present this application, undertaking to cultivate the aforesaid $10\frac{3}{8}$ [\Gk{i d' h'}] arurae at the
rate of $1\frac{1}{24}$ [\Gk{a {k'}{d'}}] artabae of wheat for each arura, unirrigated land and half of the artificially irrigated land being exempted according to custom.
\end{quote}

P. Hibeh 87 \cite[p.~250]{phibehI}, Hibeh, 256/5 BC, Advance of Seed-Corn:

\begin{quote}
\dots son of Heraclides and Her \dots son of Meniscus and Ze \dots son of \dots,
holders of 25 arourae [\Gk{(e>ikosipent'arouroi}) >'eqein], acknowledge that we have received from \dots, sitologus, for the
holdings which we possess at the village of the Pastophori, as seed for the 30th year [\Gk{l (>'etos)}]
$79 \frac{3}{4}$ artabae of wheat and $33 \frac{1}{4}$ artabae of barley 
[\Gk{pur[o]~u <ebdom'hk[on]ta >enn'ea <'hmusu t'etarton ka`i krij~hs tri'akonta tre~is t'etarton}],
in pure corn measured by the receiving
measures, and we make no complaint.
\end{quote}

Hibeh \cite{hibeh}

P. Yale I.75, Inv. 297 \cite[p.~239]{pyaleI}, Tebtunis, AD 176, Two Customs House Receipts From Tebtunis:

\begin{quote}
Paid through the gate of Tebtunis, the 1/100 and 1/50 [\Gk{p' ka`i n'}], by Petesouchos,
importing a donkey, female, black, having shed its teeth, one. Year 16 [\Gk{(>'etous) i\textstigma}], Payni
twenty-one, 21 [\Gk{Pa~uni m'ia ka`i e>ik'adi ka}]. Seal: Year 16 [\Gk{(>'etous) i\textstigma}] of Aurelius Antoninus Caesar the lord, the gate
of Tebtunis.
\end{quote}






Galen, {\em De antidotis}, book I, chapter V \cite[p.~142]{GMPI}, \cite[pp.~31--32]{galeniXIV}: 

\begin{quote}
Books lying in the libraries that have signs for numbers are easily
distorted, with the five changing into nine, and also the seventy,
or the 13, through the addition of one letter, just as also
through the subtraction of another. As a result I follow the
practice of Menecrates author of the work entitled \Gk{A>utokr'atwr <ologr'ammatos},
in which the 7's were written out with two syllables,
not \Gk{z} by itself; the 20's with three syllables, not \Gk{k} by itself;
and the 30's with four syllables, not \Gk{l} by itself -- and the rest
similarly, as I myself shall do as well. I also praise Andromachus who
wrote his {\em Theriaka} in verse, as did some others. Damocrates, too, did rightly
by writing recipes in verse, for then the rascals are least of all able to
distort them.
\end{quote}








P. Princ. Inv. GD 9556  \cite[p.~246]{sijpesteijn} = SB XX.15071 \cite[pp.~637--638]{SB20},
provenance unknown, third or fourth century AD. 








P. Petrie, Flinders Petri papyri \cite{ppetrIII}

P. Petrie 3.76 \cite[p.~521]{smyly}:

\begin{center}
\Gk{{>\textStigma}yxhb'  >Brx\textstigma{$\angle$}g'  >D{\textsigma}{\qoppa}dh'  >Zyk{\textstigma}h' >Atxjb'
{>\textStigma}qied' >Hwnhd' >Awp$\angle$}

$\overset{\textrm{\Gk{a}}}{\textrm{\Gk{M}}}$\Gk{>Gskgb' >Arpd$\angle$g'  >D{\qoppa}{\textstigma}$\angle$g'
>Aqoed'} / $\overset{\textrm{\Gk{e}}}{\textrm{\Gk{M}}}$ \Gk{>Jwk}
\end{center}

\begin{tabular}{ll}
&\Gk{b'+$\angle$+g'+h'+h'+b'+d'+d'+$\angle$+b'+$\angle$+g'+$\angle$+g'+d'}\\
=&\Gk{\textstigma}\\
\hline
&\Gk{h+\textstigma+d+\textstigma+j+e+h+g+d+\textstigma+e+\textstigma}\\
=&\Gk{o}\\
\hline
&\Gk{x+k+\qoppa+k+x+i+n+p+k+p+\qoppa+o+o}\\
=&\Gk{yk}\\
\hline
&\Gk{y+r+\textsigma+y+t+q+w+w+\textsigma+r+q+y}\\
=&\Gk{>Ew}\\
\hline
&\Gk{>\textStigma+>B+>D+>Z+>A+>\textStigma+>H+>A
+>G+>A+>D+>A+>E}\\
=&$\overset{\textrm{\Gk{d}}}{\textrm{\Gk{M}}}$\Gk{>J}\\
\hline
&$\overset{\textrm{\Gk{a}}}{\textrm{\Gk{M}}}$+$\overset{\textrm{\Gk{d}}}{\textrm{\Gk{M}}}$\\
=&$\overset{\textrm{\Gk{e}}}{\textrm{\Gk{M}}}$\\
\hline
&$\overset{\textrm{\Gk{e}}}{\textrm{\Gk{M}}}$\Gk{>Jwk}
\end{tabular}













Leonardo Pisano, {\em Liber abaci}, chapter 5 \cite[p.~49]{abaci}, \cite[p.~24]{boncompagni}:

\begin{quote}
If over any number will be made a fraction line, and over the same line
will be written another number, the upper number means the number of parts
determined by the lower number; the lower is called the denominator and the
upper is called the numerator. And if over the number two will be made a 
fraction line, and over the fraction line the number one is written, then one of
the two parts of the whole is meant, that is, one half, thus $\frac{1}{2}$, and if over the number
three the same one is put, thus $\frac{1}{3}$, it denotes one third; and if over seven, thus
$\frac{1}{7}$, one seventh; and if over 10, one tenth; and if over 19, a nineteenth part of
the whole is meant, and so on successively. Also if two over three will be shown,
thus $\frac{2}{3}$, two of three parts of the whole is meant, that is two thirds. And if
over 7, then two sevenths, thus $\frac{2}{7}$, and if over 23, then two twenty-thirds will
be denoted, and so on successively. Also if seven is put over nine, thus $\frac{7}{9}$, seven
ninths of the whole is meant; and if 7 is put over 97, seven ninety-sevenths will
be denoted. Also 13 put over 29 means thirteen twenty-ninths. And if 13 is put
over 347, thirteen three hundred forty-sevenths will be indicated, and thus it  is
understood for the remaining numbers.

Cum super quemlibet numerum quedam virgula protracta fuerit, et super ipsam
quilibet alius numerus descriptus fuerit, superior numerus partem vel partes inferioris
numeri affirmat; nam inferior denominatus, et superior denominans appellatur. Ut si super
binarium protracta fuerit virgula, et super ipsam unitas descripta sit, ipsa unitas unam
partem de duabus partibus unius integri affirmat, hoc est medietatem sic $\frac{1}{2}$ et super
ternarium ipsa unitas posita fuerit sic $\frac{1}{3}$, denotat tertium: et si super septenarium sic
$\frac{1}{7}$ septimam; et si super 10 decimam; et si super 19, nonamdecimam partem unius integri
affirmat, et sic deinceps. Item si binarius super ternarium extiterit sic $\frac{2}{3}$, duas partes
de tribus partibus unius integri affirmat, hoc est duas tertias. Et si super 7 super septimas
sic $\frac{2}{7}$ et si super 23 duas vigesimas tertias denotabunt, et sic deinceps. Item si
septenarius super novenarium positus fuerit sic $\frac{7}{9}$ septem, novenas unius integri affirmant;
et si 7 super 97, septem nonagesimas septimas denotabunt. Item 13 posita super 29, tredecim
vigesimas nonas affirmant. Et si 13 sunt super 347, tredecim trecentesimas quadragesimas
septimas indicabunt, et sic de reliquis numeris est intelligendum.
\end{quote}

Leonardo Pisano, {\em Liber abaci}, chapter 5 \cite[p.~50]{abaci}:

\begin{quote}
If under a certain fraction line one puts 2 and 7, and over the
2 is 1, and over the 7 is 4, as here is displayed, $\frac{1 \enspace 4}{2 \enspace 7}$, four sevenths plus one half
of one seventh are denoted. However if over the 7 is the zephir [{\em zephyrum}], thus $\frac{1 \enspace 0}{2 \enspace 7}$, one
half of one seventh is denoted. Also under another fraction line are 2, 6, and 10;
and over the 2 is 1, and over the 6 is 5, and over the 10 is 7, as is here displayed,
$\frac{1 \enspace 5 \enspace 7}{2 \enspace 6 \enspace 10}$,
the seven that is over the 10 at the head of the fraction line represents
seven tenths, and the 5 that is over the 6 denotes five sixths of one tenth, and
the 1 which is over the 2 denotes one half of one sixth of one tenth, and thus
singly, one at a time, they are understood; \ldots
\end{quote}

Leonardo Pisano, {\em Liber abaci}, chapter 7, part 6, first distinction \cite[p.~119]{abaci}:

\begin{quote}
In the first and second part of this chapter we taught how to add together several fractions into a single fraction.
In this part truly we teach how to
separate fractions with several parts into the sum of unit fractions, and seeing
the parts of any fraction, to know the values of the part or parts of the
integer one. This work is indeed divided into seven distinctions, the first of
which is when the greater number which is below the fraction is divisible by
the lesser, namely by that which is over the fraction line. The rule for the first
distinction is that you divide the greater by the lesser, and you will have the
part that the lesser is of the greater. For example, we wish to know what
part $\frac{3}{12}$ is of the integer one. The 12 is indeed divided by the 3; this yields 4 for
which you say $\frac{1}{4}$, and such is the part $\frac{3}{12}$ is of the integer one.
\end{quote}

$\frac{k}{kl} = \frac{1}{l}$

Leonardo Pisano, {\em Liber abaci}, chapter 7, part 6, second distinction \cite[p.~119]{abaci}:

\begin{quote}
The second distinction is when the greater number is not divisible by the
lesser, but of the lesser can be made such parts which will divide integrally into
the greater; in the rule for this distinction you make parts of the lesser by which
you can divide the greater; and the greater is divided by each of the parts, and
you will have unit fractions that the lesser makes from the greater.
\end{quote}

$\frac{k+l}{klm} = \frac{1}{lm}+\frac{1}{km}$

Leonardo Pisano, {\em Liber abaci}, chapter 7, part 6, third distinction \cite[pp.~121--122]{abaci}:

\begin{quote}
The third distinction indeed is when one more than the greater number
is divisible by the lesser; the rule for this distinction is, you divide the number
that is one more by the lesser, and the quotient of the division will be the part of
the integer one, and will be less than the greater, and to this you add the same
part of the part that is the greater number. For example, we wish to make unit
fractions of $\frac{2}{11}$; that is from this distinction because one plus the 11, namely
12, is divisible by the 2 that is over the fraction; from this division comes the
quotient 6 which yields $\frac{1}{6}$, and to this is added a sixth of an eleventh, namely
$\frac{1 \enspace 0}{6 \enspace 11}$, for the unit fraction parts of $\frac{2}{11}$; using the same rule for $\frac{3}{11}$
you will have a
quarter and $\frac{1 \enspace 0}{4 \enspace 11}$ ,that is $\frac{1}{44} \; \frac{1}{4}$. And for $\frac{4}{11}$
you will have a third and $\frac{1 \enspace 0}{3 \enspace 11}$, that is
$\frac{1}{33} \; \frac{1}{3}$; and so for the $\frac{6}{11}$ you will have half and $\frac{1 \enspace 0}{2 \enspace 11}$,
that is $\frac{1}{22} \; \frac{1}{2}$; and similarly for
the $\frac{5}{19}$, as the 5 that is over the 19 is $\frac{1}{4}$ of 20, that is 1 plus the 19, you will have
$\frac{1 \enspace 0}{4 \enspace 19}$, that is $\frac{1}{76} \; \frac{1}{4}$;
still by the third distinction there are those that are composed
a second time, as $\frac{2 \enspace 0}{3 \enspace 7}$, that is $\frac{1 \enspace 0}{2 \enspace 7}$ and $\frac{1 \enspace 0}{6 \enspace 7}$;
as $\frac{2}{3}$ of $\frac{1}{6} \; \frac{1}{2}$; similarly $\frac{4 \enspace 0}{7 \enspace 9}$ is $\frac{1 \enspace 0}{2 \enspace 9}$
and $\frac{1 \enspace 0}{14 \enspace 9}$, because
$\frac{4}{7}$ is $\frac{1}{14} \; \frac{1}{2}$; therefore the $\frac{3 \enspace 0}{11 \enspace 7}$ is $\frac{1 \enspace 0}{4 \enspace 7}$
plus $\frac{1 \enspace 0}{44 \enspace 7}$; similarly, the $\frac{3 \enspace 0}{7 \enspace 8}$ is
reversed to $\frac{3 \enspace 0}{8 \enspace 7}$, that is from two composed distinctions, namely from the second
and from the third.
\end{quote}

$\frac{k}{kl-1} = \frac{1}{l} + \frac{1}{kl-1}$

Leonardo Pisano, {\em Liber abaci}, chapter 7, part 6, fourth distinction \cite[pp.~122--123]{abaci}:

\begin{quote}
The fourth distinction is when the greater is a prime number, and the greater plus one is divisible by the lesser minus 1, as
$\frac{5}{11}$ and $\frac{7}{11}$; this distinction rule is, you subtract 1 from the lesser, from which you make a unit fraction, namely with whatever is the number which is under the
fraction, and then there will
remain for you the parts using the third distinction; if you will subtract
$\frac{1}{11}$ from $\frac{5}{11}$, then there will remain $\frac{4}{11}$,
for which $\frac{4}{11}$ you will have the unit fractions $\frac{1}{33} \; \frac{1}{3}$
by the third distinction, and with the abovewritten $\frac{1}{11}$ added this will yield
$\frac{1}{33} \; \frac{1}{11} \; \frac{1}{3}$; and by the same rule for $\frac{7}{11}$ you will have
$\frac{1}{22} \; \overline{1}{11} \; \overline{1}{2}$, and for $\frac{3}{7}$ you will have 
$\frac{1}{28} \; \frac{1}{7} \; \frac{1}{4}$, for $\frac{6}{19}$ you will have $\frac{1}{76} \; \frac{1}{19} \; \frac{1}{4}$,
and for $\frac{7}{29}$ you will have $\frac{1}{5} \; \frac{1}{29} \; \frac{1}{145}$;
$\frac{1}{145} \; \frac{1}{29} \; \frac{1}{5}$ that is.
\end{quote}








Pack \cite{pack}, nos. 2306--2325










MPER XV.144 = P. Vindob. G 26011o \cite[p.~134]{MPERXV}, Arsinoites/Heracleopolites, seventh century AD:

\begin{tabular}{ll}
\Gk{a b g d e {\textstigma} z h j i ia ib}&1 2 3 4 5 6 7 8 9 10 11 12\\
\Gk{ig id ie i{\textstigma} iz ih ij k ka kb kg}&13 14 15 16 17 18 19 20 21 22 23\\
\Gk{kd ke k{\textstigma} kz [k]h [k]j [l]}&24 25 26 27 28 29 30\\
\Gk{l[a] lb lg [ld le l]{\textstigma} lz [lh]}&31 32 33 34 35 36 37 38\\
\Gk{lj m ma [m]b [m]g md [me m{\textstigma}]}&39 40 41 42 43 44 45 46\\
\Gk{mz mh mj n [na n]b [n]g nd ne [n{\textstigma}]}&47 48 49 50 51 52 53 54 55 56\\
\Gk{nz nh nj x xa x[b x]g xd xe}&57 58 59 60 61 62 63 64 65\\
\Gk{x{\textstigma} xz xh xj o oa ob}&66 67 68 69 70 71 72\\
\Gk{og [o]d oe o{\textstigma} oz oh oj p p[a]}&73 74 75 76 77 78 79 80 81\\
\Gk{pb pg pd pe p{\textstigma} p[z]}&82 83 84 85 86 87\\
\Gk{ph pj {\textqoppa} {\textqoppa}[a] {\textqoppa}b {\textqoppa}g {\textqoppa}d {\textqoppa}e}&88 89 90 91 92 93 94 95\\
\Gk{{\textqoppa}{\textstigma} {\textqoppa}z {\textqoppa}h {\textqoppa}j r s t u f q}&96 97 98 99 100 200 300 400 500 600\\
\Gk{y w {\textsampi}} \textsubscript{,}{\Gk{a}} \textsubscript{,}{\Gk{b}} \textsubscript{,}{\Gk{g}} \textsubscript{,}{\Gk{d}} 
\textsubscript{,}{\Gk{e}} \textsubscript{,}{\Gk{\textstigma}} \textsubscript{,}{\Gk{z}} \textsubscript{,}{\Gk{h}}
\textsubscript{,}{\Gk{j}} $\overset{\textrm{\Gk{a}}}{\textrm{\Gk{m}}}$&700 800 900 1000 2000 3000 4000 5000\\
&6000 7000 8000 9000 10000
\end{tabular}










P. Mich. XV.686, Inv. 5663a \cite[pp.~2--7]{pmichXV}, Karanis, second or third century AD. 
Fragment B, Col. I:

\begin{tabular}{ll}
\Gk{t`o m' t[~wn] n a}d\textprime&the $\sfrac{1}{40}$ of the 50, $1 \; \sfrac{1}{4}$\\
\Gk{t`o n t~wn x ae'}&the $\sfrac{1}{50}$ of the 60, $1 \; \sfrac{1}{5}$\\
\Gk{t`o x' t~wn o a\textstigma'}&the $\sfrac{1}{60}$ of the 70, $1 \; \sfrac{1}{6}$\\
\Gk{t`o o' t~wn p az'}&the $\sfrac{1}{70}$ of the 80, $1 \; \sfrac{1}{7}$\\
\Gk{t`o p' t~wn {\textqoppa} ah'}&the $\sfrac{1}{80}$ of the 90, $1 \; \sfrac{1}{8}$\\
\Gk{t`o {\textqoppa} t~wn r aj'}&the $\sfrac{1}{90}$ of the 100, $1 \; \sfrac{1}{9}$\\
\Gk{t`o r' t~wn ri ai'}&the $\sfrac{1}{100}$ of the 110, $1 \; \sfrac{1}{10}$
\end{tabular}

Fragment B, Col. II, multiples of \textoverline{30}:

\begin{tabular}{ll}
\Gk{t`a g i'}&3, \textoverline{10}\\
\Gk{t`a g$\angle$ {i'}{x'}}&3 \textoverline{2}, \textoverline{10} \textoverline{60}\\
\Gk{t`a d {i'}{l'}}&4, \textoverline{10} \textoverline{30}\\
\Gk{[t`a] d$\angle$ {i'}{k'}}&4 \textoverline{2}, \textoverline{10} \textoverline{20}
\end{tabular}

Fragment C, Col. I, multiples of \textoverline{30}:

\begin{tabular}{ll}
\Gk{[t]`a e \textstigma'}&5, \textoverline{6}\\
\Gk{t`a e$\angle$ {\textstigma'}{x'}}&5 \textoverline{2}, \textoverline{6} \textoverline{60}\\
\Gk{t`a {\textstigma} e'}&6, \textoverline{5}\\
\Gk{t`a {\textstigma}$\angle$ {e'}{x'}}&6 \textoverline{2}, \textoverline{5} \textoverline{60}\\
\Gk{[t`a] z {e'}{l'}}&7, \textoverline{5} \textoverline{30}\\
\Gk{[t`a] z$\angle$ $\angle$e'[{l'}{x'}]}&7 \textoverline{2}, \textoverline{2} [!] \textoverline{5} \textoverline{30} \textoverline{60}\\
\Gk{t`a h [{e'} i{e'}]}&8, \textoverline{5} \textoverline{15}\\
\Gk{t`a h$\angle$ {e'}[{i'}{x'}]}&8 \textoverline{2}, \textoverline{5} \textoverline{10} [!] \textoverline{60}\\
\Gk{t`a j {e'}[i']}&of 9, \textoverline{5} \textoverline{10}\\
\Gk{t`a j$\angle$ {e'}[{i'}{x'}]}&of 9 \textoverline{2}, \textoverline{5} \textoverline{10} \textoverline{60}\\
\Gk{t`a i g'}&of 10, \textoverline{3}
\end{tabular}

Fragment C, Col. II, multiples of \textoverline{30}:

\begin{tabular}{ll}
\Gk{t`a ij [$\angle${i'}{l'}]}&19, \textoverline{2} \textoverline{10} \textoverline{30}\\
\Gk{t`a ij$\angle$ $\angle${i'}{k'}}&19 \textoverline{2}, \textoverline{2} \textoverline{10} \textoverline{20}\\
\Gk{t`a k b'}&20, $\overline{\overline{3}}$\\
\Gk{t`a k$\angle$ {b'}{x'}}&20 \textoverline{2}, \textoverline{\textoverline{3}} \textoverline{60}\\
\Gk{t`a ka {b'}{l'}}&21, \textoverline{\textoverline{3}} \textoverline{30}\\
\Gk{t`a ka$\angle$ {b'}{k'}}&21 \textoverline{2}, \textoverline{\textoverline{3}} \textoverline{20}\\
\Gk{[t`a] kb {b'}{ie'}}&22, \textoverline{\textoverline{3}} \textoverline{15}\\
\Gk{[t`a] kb$\angle$ {b'}{ib'}}&22 \textoverline{2}, \textoverline{\textoverline{3}} \textoverline{12}\\
\Gk{[t`a] kg {b'}{i'}}&23, \textoverline{\textoverline{3}} \textoverline{10}\\
\Gk{[t]`a kg$\angle$ {b'}{i'}{x'}}&23 \textoverline{2}, \textoverline{\textoverline{3}} \textoverline{10} \textoverline{60}\\
\Gk{[t]`a kd {b'}{i'}{l'}}&24, \textoverline{\textoverline{3}} \textoverline{10} \textoverline{30}\\
\Gk{t`a kd$\angle$ [{b'}{l'}]}&24 \textoverline{2}, \textoverline{\textoverline{3}} \textoverline{30} [!]\\
\Gk{t`a ke [{b'}{\textstigma'}]}&25, \textoverline{\textoverline{3}} \textoverline{6}\\
\Gk{t`a ke$\angle$ [{b'}{\textstigma'}{x'}]}&25 \textoverline{2}, \textoverline{\textoverline{3}} \textoverline{6} \textoverline{60}\\
\Gk{t`a k{\textstigma} {b'}[{e'}]}&26, \textoverline{\textoverline{3}} \textoverline{5}\\
\Gk{t`a k{\textstigma}$\angle$ {b'}[{e'}{x'}]}&26 \textoverline{2}, \textoverline{\textoverline{3}} \textoverline{5} \textoverline{60}
\end{tabular}









P. Mich. III.145, Inv. 4966, second century AD, unknown origin \cite[pp.~34--52]{pmichIII}. Robins describes the papyrus 
as composed of 19 pieces grouped into ten fragments.
Fragment I has five columns but only Col. i--ii are intact. 
P. Mich. III.145, I,i \cite[p.~36]{pmichIII}:

\begin{tabular}{ll}
\Gk{ths a k'g'}&of 1, \textoverline{23}\\
\Gk{[twn b] {i'}{b'} {\textsigma}{o'}{\textstigma'}}&of 2, \textoverline{12} \textoverline{276}\\
\Gk{[twn g] i' {m'}{\textstigma'} {r'}{i'}{e'}}&of 3, \textoverline{10} \textoverline{46} \textoverline{115}\\
\Gk{[twn d \textstigma]' {r'}{l'}{h'}}&of 4, \textoverline{6} \textoverline{138}\\
\Gk{[twn e \textstigma' {k'}g]' r{l'}[{h'}]}&of 5, \textoverline{6} \textoverline{23} \textoverline{138}
\end{tabular}

P. Mich. III.145, I,ii \cite[p.~36]{pmichIII}:

\begin{tabular}{ll}
\Gk{twn ib} d \Gk{h' kj' {\textsigma}{l'}b'}&of 12, \textoverline{4} \textoverline{8} \textoverline{29} \textoverline{232}\\
\Gk{[twn] ig g' {i'}{e'} [k'j' p']z' u[{l'}{e'}]}&of 13, \textoverline{3} \textoverline{15} \textoverline{29} \textoverline{87} \textoverline{435}\\
\Gk{[twn] id} d \Gk{e' [n']h' ric' p{m'}{e'}}&of 14, \textoverline{4} \textoverline{5} \textoverline{58} \textoverline{116} \textoverline{145}\\
\Gk{[twn] ie} \, $\angle$ \, \d{\Gk{n'}} \d{\Gk{h'}}&of 15, \textoverline{2} \textoverline{58}\\
\Gk{[twn i]} \d{\Gk{c}} \quad [$\angle$ \Gk{k'}] \d{\Gk{j'}} \, \Gk{{n'}{h'}}&of 16, \textoverline{29} \textoverline{58}\\
\Gk{[twn iz} \, $\angle$ \, \Gk{i'b'] {t'}{m'}{h'}}&of 17, \textoverline{2} \textoverline{12} \textoverline{348}
\end{tabular}











P. Mich. III.146, Inv. 621, fourth century AD, from the Fayum \cite[pp.~52--58]{pmichIII}, edited and described previously
by Robbins \cite{mich621}.
Robbins \cite[p.~328]{mich621}:

\begin{quote}
The papyrus numbered 621 in the recently acquired collection of the University of Michigan, since it contains tables of fractions, adds another to the rather brief list of documents
bearing upon logistic, which, as the science of calculating, was clearly distinguished by the ancients from arithmetic, the science of numbers as such. It is one of the longest rolls of
the collection, and of peculiar shape, since it is almost exactly 3 feet 6 inches long and, on the average, only $3\frac{5}{8}$ inches (92 mm.) high. It came originally from the Fayum, and dates from approximately the fourth century A.D.; there are none but paleographic clues to its age. It is clearly written, in a uniform hand, on the recto only, the verso remaining blank.
There are 19 narrow columns, of which the first is mutilated and the last, save for the heading, left blank. It may be inferred that the beginning of the tables has been lost, and that the scribe, who undoubtedly was copying, not actually calculating, left off without completing his task.

Papyrus 621 contains a list of the fractions of numbers, beginning with the last part of the sevenths and continuing with eighths, ninths, and so on, through the eighteenths; the
heading ``nineteenths'' appears, but nothing is written under it. Through the tenths, the fractions for the numbers 1, 2, 3, 4, \dots. 10, 20, 30, 40 \dots. 100, 200, 300, 400 \dots. 1,000, 2,000, 3,000, 4,000 \dots. 10,000 are given, but thereafter only for the successive numbers up to the denominator of the fraction itself. The results are invariably expressed as sums of fractions with unit numerators, with the single exception of $\frac{2}{3}$, which in both Egyptian and Greek logistic seems to have ranked with the unit fractions and had a special sign. In
each table the second entry consists of the denominiator of the fraction followed by the number which multiplied thereby gives 6,000; this was included, doubtless, because 6,000 drachmas make a talent.
\end{quote}

A talent is a unit of weight, typically used with silver; a talent of silver has the weight of around 26 kg. 

Sevenths: Col. i;
eighths: Col. i--iv;
ninths: Col. iv--vii;
tenths: Col. vii--ix;
elevenths: Col. x;
twelfths: Col. xi;
thirteenths: Col. xii;
fourteenths: Col. xiii;
fifteenths: Col. xiv--xv;
sixteenths: Col. xv--xvi;
seventeenths: Col. xvi--xvii;
eighteenths: Col. xvii--xviii;
nineteenths: Col. xix, intact but blank except for the heading \Gk{enneakaidekata}.

P. Mich. III.146, sevenths, Col. i \cite[p.~54]{pmichIII}:

\begin{tabular}{ll}
\Gk{[twn A] {r'}{m'}{b'}}$\angle'$\Gk{{g'}{m'}{b'}}&of 1000, 142 \textoverline{2} \textoverline{3} \textoverline{42}\\
\Gk{[twn B] {\textsigma'}{p'}{e'}{G'}{k'}{a'}}&of 2000, 285 \textoverline{\textoverline{3}} \textoverline{21}\\
\Gk{[twn G] {u'}{k'}{h'}}$\angle'$\Gk{{i'}{d'}}&of 3000, 428 \textoverline{2} \textoverline{14}\\
\Gk{[twn D] {f'}{o'}{a'}{g'}{i'}{e'}{l'}{e'}}&of 4000, 571 \textoverline{3} \textoverline{15} \textoverline{35}\\
\Gk{[twn E] {y'}{i'}{d'}{d'}{k'}{h'}}&of 5000, 714 \textoverline{4} \textoverline{28}\\
\Gk{[twn \textStigma] {w'}{n'}{z'}{z'}}&of 6000, 857 \textoverline{7}\\
\Gk{[twn Z] A'}&of 7000, 1000\\
\Gk{[twn H] {A'}{r'}{m'}{b'}}$\angle'$\Gk{{g'}{m'}{b'}}&of 8000, 1142 \textoverline{2} \textoverline{3} \textoverline{42}\\
\Gk{[twn J] {A'}{\textsigma'}{p'}{e'}{G'}{k'}{a'}}&of 9000, 1285 \textoverline{\textoverline{3}} \textoverline{21}\\
\Gk{[twn} $\overset{\textrm{\Gk{a}}}{\textrm{\Gk{m}}}]$ \Gk{{A'}{u'}{k'}{h'}}$\angle'$\Gk{{i'}{d'}}&of 10000, 1428 \textoverline{2}
\textoverline{14}
\end{tabular}

MPER XV.167 = P. Vindob. G 24550, Soknopaiou Nesos, Fayum, second century AD \cite[pp.~160--161]{MPERXV}, table of sevenths, Col. I--II:
the table has further entries for 1000, 2000, 3000, 4000, 5000, 6000, 7000, 8000, 9000, 10000. These entries except for 4000 are the same as the entries in
P. Mich. III.146; the entry for
4000 in P. Mich. III.146 is \Gk{{f'}{o'}{a'}{g'}{i'}{e'}{l'}{e'}}, 571 \textoverline{3} \textoverline{15} \textoverline{35}, which is right, and the entry
in MPER XV.167 is \Gk{[fo]a{g'}iale}, 571 \textoverline{3} \textoverline{11} \textoverline{35}, which is wrong.
However, the editors transcribe the entry in
MPER XV.167 as \Gk{[fo]\d{a}{g'}i\d{a}le}. \Gk{\d{a}} indicates that the reading \Gk{a} is uncertain.


\begin{tabular}{ll}
\Gk{[<'eb]dom[on]}&\\
\Gk{[t~hs] a [m]i~as t`o \textoverline{z} \textoverline{z}}&of 1, \textoverline{7}\\
\Gk{t~wn b} d\Gk{kh}&of 2, \textoverline{4} \textoverline{28}\\
\Gk{t~wn g {g'}i{a'}le}&of 3, \textoverline{3} \textoverline{11} \textoverline{35} [!]\\
\Gk{[t~wn d] $\angle$id}&of 4, \textoverline{2} \textoverline{14}\\
\Gk{[t~wn e] Gka}&of 5, \textoverline{\textoverline{3}} \textoverline{21}\\
\Gk{[t~wn] {\textstigma} $\angle${g'}{mb}}&of 6, \textoverline{2} \textoverline{3} \textoverline{42}\\
\Gk{t~wn z a}&of 7, 1\\
\Gk{[t]~w[n] h az[']}&of 8, 1 \textoverline{7}\\
\Gk{t~wn j a}{d\textprime}\Gk{kh}&of 9, 1 \textoverline{4} \textoverline{28}\\
\Gk{t~wn i [ag']iale}&of 10, 1 \textoverline{3} \textoverline{11} \textoverline{35} [!]\\
\Gk{[t~w]n k b$\angle${g'}mb}&of 20, 2 \textoverline{2} \textoverline{3} \textoverline{42}\\
\Gk{t~wn m e[G]'\textoverline{k} \textoverline{a}}&of 40, 5 \textoverline{\textoverline{3}} \textoverline{21}\\
\Gk{t[~w]n [n zz']}&of 50, 7 \textoverline{7}\\
\Gk{[t~wn x h$\angle$i]d}&of 60, 8 \textoverline{2} \textoverline{14}\\
\Gk{[t~wn o i]}&of 70, 10\\
\Gk{[t~wn p iai]{a'}le}&of 80, 11 \textoverline{11} \textoverline{35} [!]\\
\Gk{[t~wn] {\textqoppa} [ib$\angle$g']mb}&of 90, 12 \textoverline{2} \textoverline{3} \textoverline{42}\\
\Gk{[t~wn r id}{d\textprime}\Gk{kh]'}&of 100, 14 \textoverline{4} \textoverline{28}\\
\Gk{[t~wn {\textsigma} kh$\angle$i]d'}&of 200, 28 \textoverline{2} \textoverline{14}\\
\Gk{[t~wn t mb$\angle$g{]'}mb}&of 300, 42 \textoverline{2} \textoverline{3} \textoverline{42}\\
\Gk{[t~wn u nzz]'}&of 400, 57 \textoverline{7}\\
\Gk{[t~wn f oa{g'}ia]le}&of 500, 71 \textoverline{3} \textoverline{11} \textoverline{35} [!]\\
\Gk{[t~wn q peG]ka}&of 600, 85 \textoverline{\textoverline{3}} \textoverline{21}\\
\Gk{[t~wn y r]}&of 700, 100\\
\Gk{[t~wn w r[id]}d\Gk{kh}&of 800, 114 \textoverline{4} \textoverline{28}\\
\Gk{[t~wn] {\textsampi} rkh$\angle$id}&of 900, 128 \textoverline{2} \textoverline{14}
\end{tabular}

\textbf{\Gk{t~wn g}}: $\sfrac{3}{7} = \overline{3} \; \overline{11} \; \overline{231}$ and
$\sfrac{3}{7} = \overline{3} \; \overline{15} \; \overline{235}$.

\textbf{\Gk{t~wn i}}: $\sfrac{10}{7} = 1 + \sfrac{3}{7}$.

\textbf{\Gk{[t~wn p}}: $\sfrac{80}{7} = \sfrac{70}{7}+\sfrac{10}{7} = 10 + \sfrac{10}{7}$.

\textbf{\Gk{[t~wn f}}: 







\textbf{\Gk{[twn A]}}: $7\cdot 100=700$, $7\cdot 40=280$, $7\cdot 2 = 14$.
$7\cdot 142 = 994$. $7 \cdot (142+x)=1000$. $994+7x=1000$.
$7x=6$. 
\[
x=\sfrac{6}{7}=\overline{2} \; \overline{3} \; \overline{42}.
\]
\[
\sfrac{1000}{7} = 142+x = 142 \; \overline{2} \; \overline{3} \; \overline{42}.
\]
Thus the seventh part of \Gk{A} is \Gk{{r'}{m'}{b'}} $\angle'$ \Gk{{g'}{m'}{b'}}.


\textbf{\Gk{[twn B]}}:
\[
\sfrac{2000}{7} = 284 \; 1 \; \overline{\overline{3}} \; \overline{21} = 285 \; \overline{\overline{3}} \; \overline{21}.
\]
Thus the seventh part of \Gk{B} is \Gk{{\textsigma'}{p'}{e'}} \Gk{{G'}{k'}{a'}}.


\textbf{\Gk{[twn G]}}:
$7\cdot 400 = 2800$, $7\cdot  20=140$, $7\cdot 8=56$.
$7\cdot 428=2996$. $7\cdot (428+x)=3000$.
$7x=4$. 
\[
x=\sfrac{4}{7}=\sfrac{8}{14}=\sfrac{1}{14}+\sfrac{7}{14} = \overline{2} \; \overline{14}.
\]
\[
\sfrac{3000}{7} = 428+x = 428 \; \overline{2} \; \overline{14}.
\]
Thus the seventh part of \Gk{G} is \Gk{{u'}{k'}{h'}}$\angle'$\Gk{{i'}{d'}}. 





\textbf{\Gk{[twn D]}}:
Using $\sfrac{2000}{7}$ and $\sfrac{2}{21} = \overline{14} \; \overline{42}$ (RMP Recto):
\[
\sfrac{4000}{7} = 570 \; 1 \; \overline{3} \; \overline{14} \; \overline{42} = 571 \; \overline{3} \; \overline{14} \; \overline{42}. 
\]
Thus the seventh part of \Gk{D} is \Gk{{f'}{o'}{a'}{g'}{i'}{d'}{m'}{b'}}. 

On the one hand,
the entry in col. i is \Gk{{f'}{o'}{a'}{g'}{i'}{e'}{l'}{e'}}, 571 \textoverline{3} \textoverline{15} \textoverline{35}, different than 
the expression just caclculated;
on the other hand,
the entry in the table of sevenths of the
Akhmim Mathematical Papyrus, col. 5 \cite[p.~27]{baillet} is \Gk{FOA{g'}{id'}{mb'}}.
Using
\textoverline{14} = \textoverline{15} \textoverline{210} and
\textoverline{35} = \textoverline{42} \textoverline{210},
\begin{align*}
\sfrac{4000}{7}  &= 571 \; \overline{3} \; \overline{14} \; \overline{42}\\
&=571 \; \overline{3} \; (\overline{15} \; \overline{210}) \; \overline{42}\\
&=571 \; \overline{3} \; \overline{15} \; (\overline{42} \; \overline{210})\\
&=571 \; \overline{3} \; \overline{15} \; \overline{35}.
\end{align*}
Thus the seventh part of \Gk{D} is \Gk{{f'}{o'}{a'}{g'}{i'}{e'}{l'}{e'}}.


\textbf{\Gk{[twn E]}}:
$7\cdot 700=4900$, $7\cdot 10=70$, $7\cdot 4=28$. $7 \cdot 714=4998$.
$7\cdot (714+x) = 5000$. $7x=2$.
\[
x = \sfrac{2}{7} = \overline{4} \; \overline{28}.
\]
\[
\sfrac{5000}{7} = 714 \; \overline{4} \; \overline{28}.
\]
Thus the seventh part of \Gk{E} is \Gk{{y'}{i'}{d'}{d'}{k'}{h'}}.




\textbf{\Gk{[twn \textStigma]}}:
Using $\sfrac{3000}{7}$,
\[
\sfrac{6000}{7} = 856 \; 1 \; \overline{7} = 857 \; \overline{7}.
\]
Thus the seventh part of \Gk{\textStigma} is 
\Gk{{w'}{n'}{z'}{z'}}.




\textbf{\Gk{[twn Z]}}:
The seventh part of \Gk{Z} is \Gk{A'}.




\textbf{\Gk{[twn H]}}:
Using the expression worked out for $\sfrac{4000}{7}$ and \textoverline{7} \textoverline{21} = \textoverline{6} \textoverline{42},
\[
\sfrac{8000}{7} = 1142 \; \overline{\overline{3}} \; \overline{7} \; \overline{21} = 1142 \; \overline{\overline{3}} \; \overline{6} \; \overline{42}
= 1142 \; \overline{2} \; \overline{3} \; \overline{42}.
\]
Thus the seventh part of \Gk{H} is \Gk{{A'}{r'}{m'}{b'}}$\angle'$\Gk{{g'}{m'}{b'}}.




\textbf{\Gk{[twn J]}}:
Using $\sfrac{3000}{7}$ = 428 \textoverline{2} \textoverline{14} and $\sfrac{3}{14}$ = \textoverline{5} \textoverline{70},
\[
\sfrac{9000}{7} = 1284 \; 1 \; \overline{2} \; \overline{5} \; \overline{70} = 1285 \; \overline{2} \;
\overline{5} \; \overline{70}.
\]
Thus the seventh part of \Gk{J} is \Gk{{A'}{\textsigma'}{p'}{e'}}$\angle'$\Gk{{e'}{o'}}. Calculated differently,
$7 \cdot 1000=7000$, $7\cdot 200=1400$, $7\cdot 80=560$, $7\cdot 5=35$. 
$7\cdot 1285=8995$. $7\cdot (1285+x)=9000$.
$7x = 5$. 
\[
x=\sfrac{5}{7} = \overline{2} \; \overline{5} \; \overline{70}.
\]
\[
\sfrac{9000}{7} = 1285+x = 1285 \; \overline{2} \; \overline{5} \; \overline{70},
\]
which is the same expression found above for $\sfrac{9000}{7}$. 
On the other hand, the entry in the table is \Gk{{A'}{\textsigma'}{p'}{e'}{G'}{k'}{a'}}, i.e.
1285 \textoverline{\textoverline{3}} \textoverline{21}, which is also
the entry in the table of sevenths of the Akhmim Mathematical Papyrus, col. 5 \cite[p.~27]{baillet}.
Using
\textoverline{2} \textoverline{5} = \textoverline{\textoverline{3}} \textoverline{30},
\[
\overline{2} \; \overline{5} \; \overline{70}
=\overline{\overline{3}} \; \overline{30} \; \overline{70}.
\]
\textoverline{6} \textoverline{30} = \textoverline{5}.
\textoverline{84} \textoverline{420} = \textoverline{70}.
\[
\overline{\overline{3}} \; \overline{30} \; \overline{70}
=\overline{\overline{3}} \; \overline{30} \;  \overline{84} \; \overline{420}
=\overline{\overline{3}} \;   \overline{84} \; (\overline{30} \; \overline{420})
=\overline{\overline{3}} \; \overline{84} \; \overline{28}
=\overline{\overline{3}} \; \overline{28} \; \overline{84}
=\overline{\overline{3}} \; \overline{21}.
\]
Then
\[
\sfrac{9000}{7} = 1285 \; \overline{\overline{3}} \; \overline{21}.
\]
Thus the seventh part of \Gk{J} is \Gk{{A'}{\textsigma'}{p'}{e'}{G'}{k'}{a'}}.




\textbf{\Gk{[twn} $\overset{\textrm{\Gk{a}}}{\textrm{\Gk{m}}}]$}:
$7\cdot 1000=7000$, $7\cdot 400=2800$, $7\cdot 20=140$, $7\cdot 8=56$.
$7\cdot 1428=9996$.
$7\cdot (1428+x) = 10000$.
$7x = 4$.
\[
x = \sfrac{4}{7} = \overline{2} \; \overline{14}.
\]
\[
\sfrac{10000}{7} = 1428+x = 1428 \; \overline{2} \; \overline{14}.
\]
Thus the seventh part of $\overset{\textrm{\Gk{a}}}{\textrm{\Gk{m}}}$ is
\Gk{{A'}{u'}{k'}{h'}}$\angle'$\Gk{{i'}{d'}}.






















P. Mich. III.146, seventeenths, \Gk{eptakaidekata}, Col. xvi--xvii  \cite[p.~57]{pmichIII}:

\begin{tabular}{ll}
\Gk{ths a to iziz}&of 1, \textoverline{17}\\
\Gk{to [i]z {t'}{n'}{b'}}$\angle'$\Gk{{g'}{i'}{z'}{l'}{d'}{n'}{a'}}&of 6000, 352 \textoverline{2} \textoverline{3} \textoverline{17} \textoverline{34} \textoverline{51}\\
\Gk{tw[n] b {i'}{b'}{n'}{a'}{x'}{h'}}&of 2, \textoverline{12} \textoverline{51} \textoverline{68}\\
\Gk{tw[n g] {i'}{b'}{i'}{z'}{n'}{a'}{x'}{h'}}&of 3, \textoverline{12} \textoverline{17} \textoverline{51} \textoverline{68}\\
\Gk{tw[n d] {i'}{b'}{i'}{e'}{i'}{z'}{l'}{d'}{n'}{a'}}&of 4, \textoverline{12} \textoverline{15} \textoverline{17} \textoverline{34} \textoverline{51}\\
\Gk{t[wn] e {d'}{l'}{d'}{x'}{h'}}&of 5, \textoverline{4} \textoverline{34} \textoverline{68}\\
\Gk{twn \textstigma' {g'}{n'}{a'}}&of 6, \textoverline{3} \textoverline{51}\\
\Gk{twn z {g'}{i'}{z'}{n'}{a'}}&of 7, \textoverline{3} \textoverline{17} \textoverline{51}\\
\Gk{twn h {g'}{i'}{e'}{i'}{z'}{p'}{e'}}&of 8, \textoverline{3} \textoverline{15} \textoverline{17} \textoverline{85}\\
\Gk{twn j} $\angle'$\Gk{{l'}{d'}}&of 9, \textoverline{2} \textoverline{34}\\
\Gk{twn i} $\angle'$\Gk{{i'}{z'}{l'}{d'}}&of 10, \textoverline{2} \textoverline{17} \textoverline{34}\\
\Gk{twn ia} $\angle'$\Gk{{i'}{b'}{i'}{z'}{l'}{d'}{n'}{a'}}&of 11, \textoverline{2} \textoverline{12} \textoverline{17} \textoverline{34} \textoverline{51} [!]\\
\Gk{twn ib} $\angle'$\Gk{{i'}{b'}{i'}{z'}{l'}{d'}{n'}{a'}{x'}{h'}}&of 12, \textoverline{2} \textoverline{12} \textoverline{17} \textoverline{34} \textoverline{51} \textoverline{68}\\
\Gk{twn ig} $\angle'$\Gk{{d'}{x'}{h'}}&of 13, \textoverline{2} \textoverline{4} \textoverline{68}\\
\Gk{twn id} $\angle'$\Gk{{d'}{i'}{z'}{x'}{h'}}&of 14, \textoverline{2} \textoverline{4} \textoverline{17} \textoverline{68}\\
\Gk{twn ie} $\angle'$\Gk{{d'}{i'}{z'}{l'}{d'}{x'}{h'}}&of 15, \textoverline{2} \textoverline{17} \textoverline{34} \textoverline{68} [!]\\
\Gk{twn i\textstigma} $\angle'$\Gk{{g'}{i'}{z'}{l'}{d'}{n'}{a'}}&of 16, \textoverline{2} \textoverline{3} \textoverline{17} \textoverline{34} \textoverline{51}\\
\Gk{twn iz a}&of 17, 1
\end{tabular}

\textbf{\Gk{ths a}}: The seventeenth part of 1 is \textoverline{17}, i.e.
$a$ is \Gk{iz}. The entry in the table is \Gk{iziz}. Robbins \cite[p.~329]{mich621}:

\begin{quote}
There is no way of distinguishing fractions from integers; for example, \Gk{g'} can mean either 3 or $\frac{1}{3}$,
and \Gk{{g'}{g'}} occurs in the sense $3\frac{1}{3}$.
The only exception is that in the first entry in each table the doubling of the letter shows that it denotes a fraction.
\end{quote}

\textbf{\Gk{to [i]z}}: $17 \cdot 300=5100$, $17 \cdot 50=850$, $17 \cdot 2 = 34$.
$17 \cdot 352 = 5984$. $17 \cdot (352+x) = 6000$.
$17x = 16$. $x=\sfrac{16}{17}$.

RMP $\sfrac{2}{n}$ table, entry for 
$\sfrac{2}{17}$ is the following \cite[pp.~123--124]{egyptian3}.

\begin{quote}
\textbf{Call 2} out of 17 [i.e., Get 2 by operating on 17].\\
\textbf{1/12} [of 17 is] 1 1/3 1/12, \textbf{1/51} [of 17 is] 1/3, \textbf{1/68} [of 17 is] 1/4.\\
\textbf{Procedure:}

\begin{tabular}{lllll}
&1&17&&\\
&2/3&11 1/3&&\\
&1/3&5 2/3&\textbackslash&1 17\\
&1/6&2 1/2 1/3&\textbackslash&2 34\\
\textbackslash&1/12&1 1/4 1/6 [Total:]&&3 51 1/3
\end{tabular}

Remainder 1/3 1/4
\end{quote}

This entry is explained by Clagett \cite[p.~34]{egyptian3}; cf. Chace \cite[pp.~16--117]{chace}.
$\frac{1}{12} \cdot 17 = 1 \; \overline{4} \; \overline{6}$.
$\frac{1}{12} \cdot 17+R = 2$.
$\sfrac{1}{12}+R \cdot \sfrac{1}{17} = \sfrac{2}{17}$.
\[
R = 2-\frac{1}{12} \cdot 17 = 2- (1 \; \overline{4} \; \overline{6}) = 
1-(\overline{4} \; \overline{6})
=(\overline{2} - \overline{4}) + (\overline{2} - \overline{6})
=\overline{4} + \overline{3}.
\]
$R=\overline{3} \; \overline{4}$ is the Remainder (\underline{d}3t).
Therefore 
\[
\sfrac{2}{17} = \sfrac{1}{12}+R \cdot \sfrac{1}{17} = \overline{12} + \overline{3} \cdot \overline{17} + \overline{4} \cdot \overline{17}
=\overline{12} \; \overline{51} \; \overline{68}.
\]

Using $\sfrac{2}{17}=\overline{12} \; \overline{51} \; \overline{68}$ and
$\sfrac{2}{51} = \overline{34} \; \overline{102}$ from the RMP $\sfrac{2}{n}$ table \cite[p.~128]{egyptian3},
\[
\sfrac{4}{17} = \overline{6} + \sfrac{2}{51} + \overline{34}
=\overline{6} \;  (\overline{34} \; \overline{102}) \; \overline{34}
=\overline{6} \; (\overline{34} \; \overline{34}) \; \overline{102}
=\overline{6} \; \overline{17} \; \overline{102}.
\]

Using $\sfrac{4}{17} = \overline{6} \; \overline{17} \; \overline{102}$
 $\sfrac{2}{17}=\overline{12} \; \overline{51} \; \overline{68}$,
 and $\sfrac{2}{51}= \overline{34} \; \overline{102}$,
\begin{align*}
\sfrac{8}{17} &= \overline{3} + \sfrac{2}{17} + \overline{51}\\
&=\overline{3} \; (\overline{12} \; \overline{51} \; \overline{68}) \; \overline{51}\\
&=\overline{3} \; \overline{12} \; (\overline{51} \; \overline{51}) \; \overline{68}\\
&=\overline{3} + \overline{12} + \sfrac{2}{51} + \overline{68}\\
&=\overline{3} + \overline{12} + (\overline{34} + \overline{102}) + \overline{68}\\
&=\overline{3} \; \overline{12} \; \overline{34} \; \overline{68} \; \overline{102}.
\end{align*}

Using $\sfrac{8}{17} = \overline{3} \; \overline{12} \; \overline{34} \; \overline{68} \; \overline{102}$
and $\sfrac{2}{3} = \overline{2} \; \overline{6}$,
\[
\sfrac{16}{17} =  (\overline{2} \; \overline{6}) \; \overline{6} \; \overline{17} \; \overline{34} \; \overline{51}
=\overline{2} \; \overline{3} \; \overline{17} \; \overline{34} \; \overline{51}.
\]

Therefore, with $17 \cdot (352+x) = 6000$, for which $x = \sfrac{16}{17}$,
\[
\sfrac{6000}{17} = 352 + x = 352 + \sfrac{16}{17} = 352 \; \overline{2} \; \overline{3} \; \overline{17} \; \overline{34} \; \overline{51}.
\]
Therefore the seventeenth part of \Gk{>\textStigma} is \Gk{{t'}{n'}{b'}}$\angle'$\Gk{{g'}{i'}{z'}{l'}{d'}{n'}{a'}}.


\textbf{\Gk{tw[n] b}}: 
Using $\sfrac{2}{17}=\overline{12} \; \overline{51} \; \overline{68}$ from the RMP $\sfrac{2}{n}$ table, 
the seventeenth part of \Gk{b'} is \Gk{{i'}{b'}{n'}{a'}{x'}{h'}}.

On the other hand,
using
\[
\frac{a}{bc} = \frac{1}{c \cdot \frac{b+c}{a}} + \frac{1}{b \cdot \frac{b+c}{a}}
\]
with $a=2, b=17, c=1$,
\[
\sfrac{2}{17} = \sfrac{1}{1 \cdot 9} + \sfrac{1}{17 \cdot 9} = \overline{9} \; \overline{153}.
\]


\textbf{\Gk{tw[n g]}}: Using $\sfrac{2}{17}=\overline{12} \; \overline{51} \; \overline{68}$,
\[
\sfrac{3}{17} = \overline{17} + \sfrac{2}{17}
= \overline{17} \; (\overline{12} \; \overline{51} \; \overline{68})
= \overline{12} \; \overline{17} \; \overline{51} \; \overline{68}. 
\]
Therefore the seventeenth part of \Gk{g} is 
\Gk{{i'}{b'}{i'}{z'}{n'}{a'}{x'}{h'}}.

On the other hand,
using
\[
\sfrac{a}{bc} = \sfrac{1}{c \cdot \sfrac{b+c}{a}} + \sfrac{1}{b \cdot \sfrac{b+c}{a}}
\]
with $a=3, b=17, c=1$,
\[
\sfrac{3}{17} = \sfrac{1}{1 \cdot 6} + \sfrac{17 \cdot 6} = \overline{6} \; \overline{102}.
\]


\textbf{\Gk{tw[n d]}}: Using $\sfrac{3}{17}= \overline{6} \; \overline{102}$,
\[
\sfrac{4}{17} =\sfrac{3}{17}+\sfrac{1}{17} =  (\overline{6} \;  \overline{102}) \; \overline{17} = \overline{6} \; \overline{17} \; \overline{102}.
\]
Thus
the seventeenth part of \Gk{d} is \Gk{{\textstigma'}{i'}{z'}{r'}{b'}}.
The entry in the table is 
\Gk{{i'}{b'}{i'}{e'}{i'}{z'}{l'}{d'}{n'}{a'}}, \textoverline{12} \textoverline{15} \textoverline{17} \textoverline{34} \textoverline{51}, which
is wrong; the difference of this and $\sfrac{4}{17}$ is $\sfrac{23}{1020}$ not $0$.
However in the table of seventeenths in the Akhmim Mathematical Papyrus, col. 11 \cite[p.~30]{baillet},
the entry for the seventeenth part of \Gk{D} 
is \Gk{{ib'}{ie'}{iz'}{xh'}{pe'}}, \textoverline{12} \textoverline{15} \textoverline{17} \textoverline{68} \textoverline{85}.

We calculate $\sfrac{2}{17}$ differently. 
Rather than using $\sfrac{2}{17} = \overline{12} \; \overline{51} \; \overline{68}$ from the RMP $\sfrac{2}{n}$ table,
we use $\sfrac{2}{17} = \overline{17} \; \overline{24} \; \overline{102} \; \overline{136}$.
Using $\sfrac{2}{85} = \overline{51} \; \overline{255}$ from the RMP $\sfrac{2}{n}$ table \cite[p.~131]{egyptian3},
\begin{align*}
\sfrac{4}{17} &= \sfrac{2}{17} + \overline{12} + \overline{51} + \overline{68}\\
&=(\overline{17} \; \overline{24} \; \overline{102} \; \overline{136})
\; \overline{12} \; \overline{51} \; \overline{68}\\
&=\overline{12} \; \overline{17} \; \overline{24} \; (\overline{51} \;\overline{102}) \; \overline{68} \; \overline{136}\\
&=\overline{12} \; \overline{17}  \; \overline{24} \;  \overline{34} \; \overline{68} \; \overline{136}\\
&=\overline{12} \; \overline{17}  \; \overline{24} \;  \overline{68} \;  (\overline{34} \; \overline{136})\\
&=\overline{12} \; \overline{17}  \; \overline{24} \;  \overline{68} \;  (\overline{40} \; \overline{85})\\
&=\overline{12} \; \overline{17}  \; (\overline{24} \; \overline{40})  \; \overline{68} \; \overline{85}\\
&=\overline{12} \; \overline{15}  \; \overline{17}  \; \overline{68} \;\overline{85}.
\end{align*}
Therefore, the seventeenth part of \Gk{d} is
\Gk{ib ie iz xh pe}, which is the entry in the Akhmim Mathematical Papyrus, col. 11 \cite[p.~30]{baillet}.



\textbf{\Gk{t[wn] e}}: Using
$\sfrac{2}{17}=\overline{12} \; \overline{51} \; \overline{68}$ and
$\sfrac{3}{17} = \overline{12} \; \overline{17} \; \overline{51} \; \overline{68}$,
and $\sfrac{2}{17} = \overline{12} \; \overline{51} \; \overline{68}$ and
$\sfrac{2}{51} = \overline{34} \; \overline{102}$ from the RMP $\sfrac{2}{n}$ table \cite[pp.~123, 128]{egyptian3},
\begin{align*}
\sfrac{5}{17} &=(\overline{12} \;  \overline{12}) \; \overline{17} \; (\overline{51} \;  \overline{51}) \; (\overline{68} \;  \overline{68})\\
&=\overline{6} \; \overline{17} \; \sfrac{2}{51} \; \overline{34}\\
&=\overline{6} \; \overline{17} \; (\overline{34} \; \overline{102}) \; \overline{34}\\
&=\overline{6} \; \overline{17} \; \overline{17} \; \overline{102}\\
&=\overline{6} \; \sfrac{2}{17} \; \overline{102}\\
&=\overline{6} \; (\overline{12} \; \overline{51} \; \overline{68}) \; \overline{102}\\
&=(\overline{6} \; \overline{12}) \; (\overline{51} \; \overline{102}) \; \overline{68}\\
&=\overline{4} \; \overline{34} \; \overline{68}.
\end{align*}
Therefore, the seventeenth part of \Gk{e} is
\Gk{{d'}{l'}{d'}{x'}{h'}}.



\textbf{\Gk{twn \textstigma'}}: Using
$\sfrac{3}{17} = \overline{12} \; \overline{17} \; \overline{51} \; \overline{68}$,
\begin{align*}
\sfrac{6}{17}&=\overline{6} \; \sfrac{2}{17} \; \sfrac{2}{51} \; \overline{34}\\
&=\overline{6} \; (\overline{12} \; \overline{51} \; \overline{68}) \; (\overline{34} \; \overline{102}) \; \overline{34}\\
&=\overline{6} \; \overline{12} \; \overline{51} \; (\overline{17} \;  \overline{68} \; \overline{102})\\
&=\overline{6} \; \overline{12} \; \overline{51} \; \overline{12}\\
&=\overline{6} \; (\overline{12} \; \overline{12}) \; \overline{51}\\
&=\overline{6} \; \overline{6} \; \overline{51}\\
&=\overline{3} \; \overline{51}.
\end{align*}

On the other hand,
using
\[
\frac{a}{bc} = \frac{1}{c \cdot \frac{b+c}{a}} + \frac{1}{b \cdot \frac{b+c}{a}}
\]
with $a=6, b=17, c=1$,
\[
\frac{6}{17} = \frac{1}{1\cdot \frac{18}{6}} + \frac{1}{17\cdot \frac{18}{6}}
=\frac{1}{3}+\frac{1}{51}.
\]
Thus,
\[
\sfrac{6}{17} = \overline{3} \; \overline{51}.
\]
Therefore the seventeenth part of \Gk{\textstigma} is \Gk{{g'}{n'}{a'}}.



\textbf{\Gk{twn z}}: Using
$\sfrac{6}{17} = \overline{3} \; \overline{51}$,
\[
\sfrac{7}{17} = \sfrac{6}{17} + \sfrac{1}{17} = (\overline{3} ; \overline{51}) \; \overline{17} = 
\overline{3} \; \overline{17} \; \overline{51}.
\]
Therefore the seventeenth part of \Gk{z} is \Gk{{g'}{i'}{z'}{n'}{a'}}.



\textbf{\Gk{twn h}}: 
Using 
$\sfrac{7}{17}=\overline{3} \; \overline{17} \; \overline{51}$,
and $\overline{17} \; \overline{51} = \overline{15} \; \overline{85}$,
\begin{align*}
\sfrac{8}{17} &= \sfrac{7}{17} + \sfrac{1}{17} = (\overline{3} \; \overline{17} \; \overline{51}) \overline{17}\\
&=\overline{3} \; (\overline{17} \; \overline{51}) \; \overline{17}\\
&=\overline{3} \; (\overline{15} \; \overline{85}) \; \overline{17}\\
&=\overline{3} \; \overline{15} \; \overline{17} \; \overline{85}.
\end{align*}
Therefore the seventeenth part of \Gk{h} is \Gk{{g'}{i'}{e'}{i'}{z'}{p'}{e'}}.



\textbf{\Gk{twn j}}: Using the expressions worked out for
$\sfrac{3}{17}$ and $\sfrac{6}{17}$,
\begin{align*}
\sfrac{9}{17}&=\sfrac{6}{17}+\sfrac{3}{17}\\
&=(\overline{3} \; \overline{51}) \; (\overline{6} \; \overline{102})\\
&=(\overline{3} \; \overline{6}) \; (\overline{51} \; \overline{102})\\
&=\overline{2} \; \overline{34}.
\end{align*}
Therefore the seventeenth part of \Gk{j} is $\angle'$\Gk{{l'}{d'}}.



\textbf{\Gk{twn i}}: 
Using the expression worked out for $\sfrac{2}{17}$,
\[
\sfrac{10}{17}=\sfrac{9}{17}+\sfrac{1}{17}=(\overline{2} \; \overline{34}) \; \overline{17}
=\overline{2} \; \overline{17} \; \overline{34}.
\]
Therefore the seventeenth part of \Gk{i} is $\angle'$\Gk{{i'}{z'}{l'}{d'}}




\textbf{\Gk{twn ia}}: Using the expression worked out for $\sfrac{10}{17}$,
and $\sfrac{2}{17}= \overline{12} \; \overline{51} \; \overline{68}$ from the RMP $\sfrac{2}{n}$ table \cite[p.~128]{egyptian3},
\begin{align*}
\sfrac{11}{17}&=\sfrac{10}{17}+\sfrac{1}{17}\\
&=(\overline{2} \; \overline{17} \; \overline{34}) \; \overline{17}\\
&=\overline{2} \; (\overline{17} \; \overline{17}) \; \overline{34}\\
&=\overline{2} \; (\overline{12} \; \overline{51} \; \overline{68}) \; \overline{34}\\
&=\overline{2} \; \overline{12} \; \overline{34} \; \overline{51} \; \overline{68}.
\end{align*}
Therefore, the seventeenth part of \Gk{ia} is $<$\Gk{{ib'}{ld'}{na'}{xh'}}.

The entry in the table is
$\angle'$\Gk{{i'}{b'}{i'}{z'}{l'}{d'}{n'}{a'}}, $\overline{2} \; \overline{12} \; \overline{17} \; \overline{34} \; \overline{51}$, which is wrong.
On the other hand, in the table of seventeenths in the Akhmim Mathematical Papyrus, col. 11 \cite[p.~30]{baillet},
the entry for the seventeenth part of \Gk{IA} is 
$<$\Gk{{ib'}{ld'}{na'}{xh'}},
$\overline{2} \; \overline{12} \; \overline{34} \; \overline{51}\; \overline{68}$.



\textbf{\Gk{twn ib}}: Using the expression worked out for $\sfrac{11}{17}$,
\[
\sfrac{12}{17} = \sfrac{11}{17}+\sfrac{1}{17} = (\overline{2} \; \overline{12} \; \overline{34} \; \overline{51} \; \overline{68}) \; \overline{17}
=\overline{2} \; \overline{12} \; \overline{17} \; \overline{34} \; \overline{51} \; \overline{68}.
\]
Therefore the seventeenth part of \Gk{ib} is 
$\angle'$\Gk{{i'}{b'}{i'}{z'}{l'}{d'}{n'}{a'}{x'}{h'}}.


\textbf{\Gk{twn ig}}: Using the expression worked out for $\sfrac{12}{17}$,
the equality
$\overline{17} \; \overline{34} \; \overline{102} = \overline{12} \; \overline{68}$,
and  $\sfrac{2}{17} = \overline{12} \; \overline{51} \; \overline{68}$ and
$\sfrac{2}{51} = \overline{34} \; \overline{102}$ from the RMP $\sfrac{2}{n}$ table \cite[pp.~123, 128]{egyptian3},
\begin{align*}
\sfrac{13}{17}&=\sfrac{12}{17}+\sfrac{1}{17}\\
&=(\overline{2} \; \overline{12} \; \overline{17} \; \overline{34} \; \overline{51} \; \overline{68}) \; \overline{17}\\
&=\overline{2} \; \overline{12} \; (\overline{17} \; \overline{17}) \; \overline{34} \; \overline{51} \; \overline{68}\\
&=\overline{2} \; \overline{12} \; (\overline{12} \; \overline{51} \; \overline{68}) \; \overline{34} \; \overline{51} \; \overline{68}\\
&=\overline{2} \; (\overline{12} \; \overline{12}) \; (\overline{51} \; \overline{51}) \; \overline{34} \; (\overline{68} \; \overline{68})\\
&=\overline{2} \; \overline{6} \; (\overline{34} \; \overline{102}) \; \overline{34} \; \overline{34}\\
&=\overline{2} \; \overline{6} \; (\overline{34} \; \overline{102}) \; \overline{17}\\
&=\overline{2} \; \overline{6} \; (\overline{17} \; \overline{34} \; \overline{102})\\
&=\overline{2} \; \overline{6} \; (\overline{12} \; \overline{68})\\
&=\overline{2} \; (\overline{6} \; \overline{12}) \; \overline{68}\\
&=\overline{2} \; \overline{4} \; \overline{68}.
\end{align*}
Therefore the seventeenth part of \Gk{ig} is $\angle'$\Gk{{d'}{x'}{h'}}
 

\textbf{\Gk{twn id}}: Using the expression worked out for $\sfrac{13}{17}$,
\[
\sfrac{14}{17} = \sfrac{13}{17} + \sfrac{1}{17} = (\overline{2} \; \overline{4} \; \overline{68}) \; \overline{17}
=\overline{2} \; \overline{4} \; \overline{17} \; \overline{68}.
\]
Therefore the seventeenth part of \Gk{id} is $\angle'$\Gk{{d'}{i'}{z'}{x'}{h'}}.



\textbf{\Gk{twn ie}}: Using the expression worked out for $\sfrac{14}{17}$,
and $\sfrac{2}{17}= \overline{12} \; \overline{51} \; \overline{68}$ from the RMP $\sfrac{2}{n}$ \cite[p.~128]{egyptian3},
\begin{align*}
\sfrac{15}{17}&=\sfrac{14}{17}+\sfrac{1}{17}\\
&=(\overline{2} \; \overline{4} \; \overline{17} \; \overline{68}) \; \overline{17}\\
&=\overline{2} \; \overline{4} \; (\overline{17} \; \overline{17}) \; \overline{68}\\
&=\overline{2} \; \overline{4} \; ( \overline{12} \; \overline{51} \; \overline{68}) \; \overline{68}\\
&=\overline{2} \; \overline{4} \; \overline{12} \; \overline{51} \; \overline{34}\\
&=\overline{2} \; (\overline{4} \; \overline{12}) \; \overline{34} \; \overline{51}\\
&=\overline{2} \; \overline{3} \; \overline{34} \; \overline{51}.
\end{align*}
Therefore the seventeenth part of \Gk{ie} is $<$\Gk{{g'}{ld'}{na'}}.

The entry in the table is $\angle'$\Gk{{d'}{i'}{z'}{l'}{d'}{x'}{h'}}, 
\textoverline{2} \textoverline{4} \textoverline{17} \textoverline{34} \textoverline{68}, which is wrong.
On the other hand, in the table of seventeenths in the Akhmim Mathematical Papyrus, col. 11 \cite[p.~30]{baillet},
the entry for the seventeenth part of \Gk{IE} is 
$<$\Gk{{g'}{ld'}{na'}}.



\textbf{\Gk{twn i\textstigma}}: Using the expression worked out for
$\sfrac{15}{17}$,
\[
\sfrac{16}{17} = \sfrac{15}{17}+\frac{1}{17} = (\overline{2} \; \overline{3} \; \overline{34} \; \overline{51}) \; \overline{17}
=\overline{2} \; \overline{3} \; \overline{17} \; \overline{34} \; \overline{51}.
\]
Therefore the seventeenth part of \Gk{i\textstigma} is 
$\angle'$\Gk{{g'}{i'}{z'}{l'}{d'}{n'}{a'}}.


\textbf{\Gk{twn iz a}}: The seventeenth part of \Gk{iz} is \Gk{a}.










\bibliographystyle{plain}
\bibliography{papyri}

\end{document}