\documentclass{amsart}
\usepackage{amsmath,amssymb,graphicx,subfig,mathrsfs}
\usepackage{hyperref}
\newcommand{\norm}[1]{\left\Vert #1 \right\Vert}
\newtheorem{theorem}{Theorem}
\newtheorem{lemma}[theorem]{Lemma}
\newtheorem{corollary}[theorem]{Corollary}
\begin{document}
\title{The Euler-Maclaurin summation formula}
\author{Jordan Bell}
\email{jordan.bell@gmail.com}
\address{Department of Mathematics, University of Toronto, Toronto, Ontario, Canada}
\date{\today}

\maketitle

\section{Proof}



\section{References}
Whiteside \cite[pp.~44, 257]{whitesideVIII}

Newton and Collins \cite[pp.~186, 199]{bourbaki}

Domingues \cite[p.~44]{domingues}

Todhunter \cite[p.~192]{todhunter}

Estrada and Kanwal \cite[p.~36]{estrada}

Bourbaki \cite[Chapter VI]{bourbaki}

\cite[pp.~45, 160, 337, 475, 531]{commerciumII}

\cite[pp.~XL--XLIX]{I9}

Stirling \cite[p.~274]{tweddle2003}

Euler correspondence R. 1998, 236; p. 53, 113, 137, 433

Institutiones calculi differentialis, E212

E19, E20, E25, E43, E47, E55, E125, E130, E247, E352, E368, E393, E432, E642, E746

\nocite{*}

\bibliographystyle{amsplain}
\bibliography{eulermaclaurin}

\end{document}
