\documentclass{article}
\usepackage{amsmath,amssymb,mathrsfs,amsthm}
%\usepackage{tikz-cd}
%\usepackage{hyperref}
\newcommand{\inner}[2]{\left\langle #1, #2 \right\rangle}
\newcommand{\tr}{\ensuremath\mathrm{tr}\,} 
\newcommand{\Span}{\ensuremath\mathrm{span}} 
\def\Re{\ensuremath{\mathrm{Re}}\,}
\def\Im{\ensuremath{\mathrm{Im}}\,}
\newcommand{\id}{\ensuremath\mathrm{id}} 
\newcommand{\var}{\ensuremath\mathrm{var}} 
\newcommand{\Lip}{\ensuremath\mathrm{Lip}} 
\newcommand{\GL}{\ensuremath\mathrm{GL}} 
\newcommand{\diam}{\ensuremath\mathrm{diam}} 
\newcommand{\sgn}{\ensuremath\mathrm{sgn}\,} 
\newcommand{\lcm}{\ensuremath\mathrm{lcm}} 
\newcommand{\supp}{\ensuremath\mathrm{supp}\,}
\newcommand{\dom}{\ensuremath\mathrm{dom}\,}
\newcommand{\upto}{\nearrow}
\newcommand{\downto}{\searrow}
\newcommand{\norm}[1]{\left\Vert #1 \right\Vert}
\newtheorem{theorem}{Theorem}
\newtheorem{lemma}[theorem]{Lemma}
\newtheorem{proposition}[theorem]{Proposition}
\newtheorem{corollary}[theorem]{Corollary}
\theoremstyle{definition}
\newtheorem{definition}[theorem]{Definition}
\newtheorem{example}[theorem]{Example}
\begin{document}
\title{Diophantine numbers}
\author{Jordan Bell\\ \texttt{jordan.bell@gmail.com}\\Department of Mathematics, University of Toronto}
\date{\today}

\maketitle

\section{Baire category}
For real $\tau,\gamma>0$,
let $\mathcal{D}(\tau,\gamma)$ be the set of those $\xi \in \mathbb{R}$ such that for all $q \in \mathbb{Z}_{\geq 1}$
and $p \in \mathbb{Z}$,
\[
\left|\xi-\frac{p}{q}\right| \geq \gamma q^{-\tau}.
\]
In other words, 
\[
\mathcal{D}(\tau,\gamma) =  \bigcap_{q \in \mathbb{Z}_{\geq 1}, p \in \mathbb{Z}} \left\{\xi \in \mathbb{R}:
\left|\xi-\frac{p}{q}\right| \geq \gamma q^{-\tau}\right\}
= \bigcap_{q \in \mathbb{Z}_{\geq 1}, p \in \mathbb{Z}} \mathcal{D}(\tau,\gamma,q,p).
\]
Each set $\mathcal{D}(\tau,\gamma,q,p)$ is closed, so $\mathcal{D}(\tau,\gamma)$ is closed.
Let
\[
\mathcal{D}(\tau) = \bigcup_{\gamma>0} \mathcal{D}(\tau,\gamma).
\]
Now, if $\gamma_1 \geq \gamma_2$ and $\xi \in \mathcal{D}(\tau,\gamma_1)$, take $q \in \mathbb{Z}_{\geq 1}$ and
$p \in \mathbb{Z}$, and then
\[
\left|\xi-\frac{p}{q}\right| \geq \gamma_1 q^{-\tau} \geq  \gamma_2 q^{-\tau}.
\]
Hence
\[
\mathcal{D}(\tau,\gamma_1) \subset \mathcal{D}(\tau,\gamma_2),\qquad \gamma_1 \geq \gamma_2,
\]
and thus
$\mathcal{D}(\tau) = \bigcup_{n \geq 1} \mathcal{D}(\tau,n^{-1})$,
which means that $\mathcal{D}(\tau)$ is an $F_\sigma$ set. 
Let
\[
\mathcal{D} = \bigcup_{\tau>0} \mathcal{D}(\tau)
=\bigcup_{\tau>0} \bigcup_{\gamma>0} \bigcap_{q \in \mathbb{Z}_{\geq 1}, p \in \mathbb{Z}} \mathcal{D}(\tau,
\gamma,q,p),
\]
whose elements are called \textbf{Diophantine numbers}.
If $\tau_1 \leq \tau_2$ and $\xi \in \mathcal{D}(\tau_1)$, then 
for some $\gamma>0$, $\xi \in \mathcal{D}(\tau_1,\gamma)$. 
Take $q \in \mathbb{Z}_{\geq 1}$ and
$p \in \mathbb{Z}$. Then
\[
\left|\xi-\frac{p}{q}\right| \geq \gamma q^{-\tau_1} \geq \gamma q^{-\tau_2},
\]
so $\xi \in \mathcal{D}(\tau_2,\gamma)$, i.e.
\[
\mathcal{D}(\tau_1) \subset \mathcal{D}(\tau_2),\qquad \tau_1 \leq \tau_2.
\]
Thus
$\mathcal{D} = \bigcup_{m \geq 1} \mathcal{D}(m) = \lim_{m \to \infty} \mathcal{D}(m)$,
which means that $\mathcal{D}$ is an $F_{\sigma}$ set.


Now,
\[
\mathcal{D}^c = \bigcap_{m \geq 1} \bigcap_{n \geq 1} \bigcup_{q \in \mathbb{Z}_{\geq 1},
p \in \mathbb{Z}} \mathcal{D}(m,n^{-1},q,p)^c,
\]
and
\[
\mathcal{D}(m,n^{-1},q,p)^c = \left\{\xi \in \mathbb{R}:
\left| \xi - \frac{p}{q}\right| < n^{-1} q^{-m}\right\}.
\]
For $\xi = \frac{p}{q} \in \mathbb{Q}$ with $q \geq 1$,
$\xi \in \mathcal{D}(m,n^{-1},q,p)^c$, so for each $m,n \geq 1$, 
\[
\mathbb{Q} \subset  \bigcup_{q \in \mathbb{Z}_{\geq 1},
p \in \mathbb{Z}} \mathcal{D}(m,n^{-1},q,p)^c.
\]
This means that each set $\bigcup_{q \in \mathbb{Z}_{\geq 1},
p \in \mathbb{Z}} \mathcal{D}(m,n^{-1},q,p)^c$ is dense. It is a union of open sets hence open, so it
is an open dense set. Therefore $\mathcal{D}^c$ is a comeager set and $\mathcal{D}$ is a meager set. 
Because $\mathbb{R}$ is a complete metric space, by the Baire category theorem $\mathcal{D}^c$ is dense. 





\section{Algebraic numbers}
Suppose that $\xi \in \mathbb{R} \setminus \mathbb{Q}$ and that $f(\xi)=0$ for a nonzero irreducible  polynomial
$f \in \mathbb{Z}[x]$ of degree $d$. 
Let
\[
M = \sup \{|f'(x)|: x \in [\xi-1,\xi+1]\},\qquad A = \min(1,M^{-1}),
\]
and suppose
by contradiction that there are $q \in \mathbb{Z}_{\geq 1}$ and $p \in \mathbb{Z}$ such that
\[
\left|\xi - \frac{p}{q} \right| < Aq^{-d}.
\]
By the mean-value theorem there is some $c$ between $\xi$ and $\frac{p}{q}$
such that
\[
f(\xi)-f(p/q) = f'(c) \left(\xi-\frac{p}{q}\right),
\]
and as $f(\xi)=0$,
\[
|f'(c)| \left| \xi - \frac{p}{q} \right| = |f(p/q)|.
\]
As $f$ is irreducible, $f(p/q) \neq 0$. But then $f(p/q)$ is a rational number which in lowest
terms has denominator $\leq q^d$, 
therefore $|f(p/q)| \geq q^{-d}$, giving
\[
\left| \xi - \frac{p}{q} \right| \geq \frac{1}{|f'(c)|} q^{-d}.
\]
Furthermore, $\left| \xi - \frac{p}{q} \right| < A \leq 1$ so
$\frac{p}{q} \in [\xi-1,\xi+1]$, and as $c$ is between $\xi$ and $\frac{p}{q}$
it holds that $c \in [\xi-1,\xi+1]$ and thus $|f'(c)| \leq M$. Therefore
\[
\left| \xi - \frac{p}{q} \right| \geq M^{-1} q^{-d} \geq Aq^{-d},
\] 
a contradiction. Therefore for all $q \in \mathbb{Z}_{\geq 1}$ and
$p \in \mathbb{Z}$,
\[
\left|\xi - \frac{p}{q} \right| \geq Aq^{-d}.
\]
This means that $\xi \in \mathcal{D}(d,A)$.

\begin{theorem}[Liouville approximation theorem]
If $\xi \in \mathbb{R} \setminus \mathbb{Q}$ and  $f(\xi)=0$ for a nonzero irreducible polynomial
$f \in \mathbb{Z}[x]$ of degree $d$, then $\xi \in \mathcal{D}(d)$.
\end{theorem}

We remark for the above theorem that because $\xi$ is irrational, $d \geq 2$. Also, the above theorem shows that
the set of  real irrational algebraic numbers is contained in $\mathcal{D}$, which implies that if
$\xi \in \mathbb{R} \setminus \mathbb{Q}$ and $\xi \in \mathcal{D}^c$ then $\xi$ is transcendental. 
Elements of $(\mathbb{R}  \setminus \mathbb{Q}) \cap \mathcal{D}^c$ are called \textbf{Liouville numbers}.
\textbf{Roth's theorem} states that the  set of real irrational algebraic numbers is contained in
\[
\mathcal{D}(2+) = \bigcap_{\tau>2} \mathcal{D}(\tau).
\]





\section{Measure theory}
Take $\tau>2$ and $\gamma>0$.
\[
\mathcal{D}(\tau,\gamma)^c = \bigcup_{q \in \mathbb{Z}_{\geq 1}} \bigcup_{p \in \mathbb{Z}} \left\{\xi \in \mathbb{R}:
\left| \xi - \frac{p}{q} \right| < \gamma q^{-\tau}\right\}.
\]
For $m \in \mathbb{Z}_{\geq 1}$, 
\[
\mathcal{D}(\tau,\gamma)^c  \cap (-m,m)
=\bigcup_{q \in \mathbb{Z}_{\geq 1}} \bigcup_{|p| \leq mq+\gamma q^{-\tau+1}} \left\{ -m<\xi<m: \left| \xi - \frac{p}{q} \right| < \gamma q^{-\tau} \right\}.
\]
Hence
\begin{align*}
\lambda(\mathcal{D}(\tau,\gamma)^c  \cap (-m,m))&\leq \sum_{q \geq 1} (2mq+2\gamma q^{-\tau+1}+1) \cdot 2\gamma q^{-\tau}\\
&=\sum_{q \geq 1} 4m\gamma q^{-\tau+1} + \sum_{q \geq 1} 4\gamma^2 q^{-2\tau+1} + \sum_{q \geq 1} 2\gamma q^{-\tau}\\
&=4m\gamma \zeta(\tau-1) + 4\gamma^2 \zeta(2\tau-1) + 2\gamma \zeta(\tau).
\end{align*}
Therefore for each $\gamma>0$,
\[
\lambda(\mathcal{D}(\tau)^c \cap (-m,m)) \leq 4m\gamma \zeta(\tau-1) + 4\gamma^2 \zeta(2\tau-1) + 2\gamma \zeta(\tau),
\]
hence $\lambda(\mathcal{D}(\tau)^c \cap (-m,m))=0$. This is true for each $m \in \mathbb{Z}_{\geq 1}$, therefore
$\lambda(\mathcal{D}(\tau)^c)=0$. 



\section{Exponentials}
Let $\xi \in \mathbb{R} \setminus \mathbb{Q}$ and $\lambda = e^{2\pi i\xi}$. For $q \in \mathbb{Z}_{\geq 1}$ and 
$p_q \in \mathbb{Z}$ closest to $q\xi$, 
\[
|\lambda^q-1| = |e^{2\pi iq\xi}-e^{2\pi ip_q}| \leq |2\pi q\xi - 2\pi p_q| = 2\pi |q\xi-p_q|,
\]
and, as $|q\xi-p_q| \leq \frac{1}{2}$ and since
$|\sin x| \geq \frac{2}{\pi}|x|$ for $|x| \leq \frac{\pi}{2}$,
\begin{align*}
|\lambda^q-1| &= |e^{2\pi i(q\xi-p_q)}-1|\\
&= |e^{\pi i(q\xi-p_q)}-e^{-\pi i(q\xi-p_q)}|\\
&= 2 |\sin \pi (q\xi-p_q)|\\
&\geq 2 \cdot \frac{2}{\pi} |\pi(q\xi-p_q)|\\
&=4|q\xi-p_q|.
\end{align*}
Thus
\[
4|q\xi-p_q| \leq |\lambda^q-1| \leq 2\pi |q\xi-p_q|.
\]
If $\xi \in \mathcal{D}(\tau,\gamma)$, then, for $q \in \mathbb{Z}_{\geq 1}$,
\[
|\lambda^q-1|^{-1} \leq  \frac{1}{4|q\xi-p_q|} 
=\frac{1}{4q} \left|\xi-\frac{p_q}{q}\right|^{-1}
\leq \frac{1}{4q} \gamma^{-1} q^\tau
=\frac{\gamma^{-1}}{4} q^{\tau-1}.
\]

Siegel's linearization theorem states that for
$\xi \in \mathbb{R} \setminus \mathbb{Q}$, $\lambda=e^{2\pi i\xi}$, 
if $|\lambda^q-1|$ is majorized by a power of $q$, then
if $f$ is holomorphic on a neighborhood of $0$ and $f'(0)=\lambda$ then
$f$ is linearizable at $0$.\footnote{John Milnor, {\em Dynamics in One Complex Variable},
third ed., p.~127, Theorem 11.4.}
Thus if $\xi \in \mathcal{D}$ and $f'(0)=\lambda=e^{2\pi i\xi}$, then $f$ is linearizable at $0$.




\end{document}