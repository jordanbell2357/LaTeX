\documentclass{article}
\usepackage{amsmath,amssymb,mathrsfs,amsthm}
%\usepackage{tikz-cd}
\usepackage{hyperref}
\newcommand{\inner}[2]{\left\langle #1, #2 \right\rangle}
\newcommand{\tr}{\ensuremath\mathrm{tr}\,} 
\newcommand{\Span}{\ensuremath\mathrm{span}} 
\def\Re{\ensuremath{\mathrm{Re}}\,}
\def\Im{\ensuremath{\mathrm{Im}}\,}
\newcommand{\id}{\ensuremath\mathrm{id}} 
\newcommand{\Var}{\ensuremath\mathrm{Var}} 
\newcommand{\diam}{\ensuremath\mathrm{diam}} 
\newcommand{\lcm}{\ensuremath\mathrm{lcm}} 
\newcommand{\supp}{\ensuremath\mathrm{supp}\,}
\newcommand{\dom}{\ensuremath\mathrm{dom}\,}
\newcommand{\upto}{\nearrow}
\newcommand{\downto}{\searrow}
\newcommand{\norm}[1]{\left\Vert #1 \right\Vert}
\newtheorem{theorem}{Theorem}
\newtheorem{lemma}[theorem]{Lemma}
\newtheorem{proposition}[theorem]{Proposition}
\newtheorem{corollary}[theorem]{Corollary}
\theoremstyle{definition}
\newtheorem{definition}[theorem]{Definition}
\newtheorem{example}[theorem]{Example}
\begin{document}
\title{Total variation, absolute continuity, and the Borel $\sigma$-algebra of $C(I)$}
\author{Jordan Bell\\ \texttt{jordan.bell@gmail.com}\\Department of Mathematics, University of Toronto}
\date{\today}

\maketitle

\section{Total variation}
Let $a < b$.
A \textbf{partition} of $[a,b]$ is a sequence $t_0,t_1,\ldots,t_n$ such that
\[
a=t_0 <  t_1 < \cdots < t_n = b.
\]
The \textbf{total variation} of a function $f:[a,b] \to \mathbb{C}$ is
\[
\Var_f[a,b] = \sup\left\{ \sum_{i=1}^n |f(t_i)-f(t_{i-1})|: \textrm{$t_0,t_1,\ldots,t_n$ is a partition of $[a,b]$}\right\}.
\]
If $\Var_f[a,b]<\infty$ then we say that $f$ has \textbf{bounded variation}.


\begin{lemma}
If $a \leq c < e < d \leq b$, then
\[
\Var_f[c,d] = \Var_f[c,e]+\Var_f[e,d].
\]
\end{lemma}

The following theorem establishes properties of functions of bounded variation.\footnote{Charalambos D. Aliprantis and Owen Burkinshaw, {\em Principles of Real Analysis}, third ed.,
p.~377, Theorem 39.10.}


\begin{theorem}
Suppose that $f:[a,b] \to \mathbb{R}$ is of bounded variation and define
\[
F(x) = \Var_f[a,x], \qquad x \in [a,b].
\]
Then:
\begin{enumerate}
\item $|f(y)-f(x)| \leq F(y)-F(x)$ for all $a \leq x < y \leq b$.
\item $F$ is a nondecreasing function.
\item $F-f$ and $F+f$ are nondecreasing functions.
\item For $x_0 \in [a,b]$, $f$ is continuous at $x_0$ if and only if $F$ is continuous at $x_0$.
\end{enumerate}
\label{nondecreasing}
\end{theorem}
\begin{proof}
If $t_0,\ldots,t_n$ is a partition of $[a,x]$ then
$t_0,\ldots,t_n,y$ is a partition of $[a,y]$, so
\[
\sum_{i=1}^n  |f(t_i)-f(t_{i-1})| +|f(y)-f(x)| \leq F(y).
\]
Since this is true for any partition $t_0,\ldots,t_n$ of $[a,x]$,
\[
F(x)+|f(y)-f(x)| \leq F(y).
\]
This shows in particular that $F(x) \leq F(y)$, and thus that $F$ is nondecreasing.

For $a \leq x < y \leq b$, 
\[
f(y)-f(x) \leq |f(y)-f(x)| \leq F(y)-F(x),
\]
thus
\[
F(x)-f(x) \leq F(y)-f(y),
\]
showing that $x \mapsto F(x)-f(x)$ is nondecreasing. Likewise,
\[
f(x)-f(y) \leq |f(y)-f(x)| \leq F(y)-F(x),
\]
thus
\[
f(x)+F(x) \leq f(y)+F(y),
\]
showing that $x \mapsto F(x)+f(x)$ is nondecreasing.

Suppose that $F$ is continuous at $x_0$ and let $\epsilon>0$. There is some $\delta>0$ such that
$|x-x_0|<\delta$ implies that $|F(x)-F(x_0)|<\epsilon$. If $|x-x_0|<\delta$, then
\[
|f(x)-f(x_0)| \leq |F(x)-F(x_0)|<\epsilon,
\]
showing that $f$ is continuous at $x_0$.

Suppose that $f$ is continuous at $x_0$ and let $\epsilon>0$. 
Then there is some $\delta>0$ such that
$|x-x_0|<\delta$ implies that
$|f(x)-f(x_0)|<\epsilon$, and such that $x_0-\delta>a$. Let 
$x_0-\delta<s<x_0$, and  let
$t_0,\ldots,t_n$ be a partition of $[s,b]$ such that
\[
\Var_f[s,b] <\sum_{i=1}^n |f(t_i)-f(t_{i-1})| + \epsilon
\]
and such that none of $t_0,\ldots,t_n$ is equal to $x_0$.
Say that $t_k < x_0 < t_{k+1}$. Then
\[
t_0,\ldots,t_k,x_0,t_{k+1},\ldots,t_n
\]
 is a partition of
$[s,b]$. For $t_k< x< x_0$ we have $|x-x_0| < \delta$ and therefore
\begin{align*}
\Var_f[s,x]+\Var_f[x,b]&=\Var_f[s,b]\\
&<\sum_{i=1}^n |f(t_i)-f(t_{i-1})| + \epsilon\\
&\leq \sum_{i=1}^k  |f(t_i)-f(t_{i-1})| + |f(x)-f(t_k)|\\
&+|f(x_0)-f(x)|\\
&+|f(t_{k+1})-f(x_0)|
+\sum_{i=k+2}^n |f(t_i)-f(t_{i-1})|+\epsilon\\
&\leq \Var_f[s,x]+|f(x)-f(x_0)|+\Var_f[x_0,b]+\epsilon\\
&<\Var_f[s,x]+\Var_f[x_0,b]+2\epsilon,
\end{align*}
giving
\[
\Var_f[x,b]-\Var_f[x_0,b]<2\epsilon.
\]
As  $\Var_f[a,b]=\Var_f[a,x]+\Var_f[x,b]$ and
also
$\Var_f[a,b]=\Var_f[a,x_0]+\Var_f[x_0,b]$, we have
$F(x)+\Var_f[x,b]=F(x_0)+\Var_f[x_0,b]$, and therefore
\[
F(x_0)-F(x)<2\epsilon.
\]
Thus, if $t_k<x<x_0$ then $|F(x_0)-F(x)|<2\epsilon$, showing that 
$F$ is left-continuous at $x_0$. It is straightforward to show in the same way that
$F$ is right-continuous at $x_0$, and thus continuous at $x_0$.
\end{proof}

If $f:[a,b] \to \mathbb{R}$ is of bounded variation, then Theorem \ref{nondecreasing}
tells us that $F$ and $F+f$ are nondecreasing functions. A monotone function is
differentiable almost everywhere,\footnote{Charalambos D. Aliprantis and Owen Burkinshaw, {\em Principles of Real Analysis}, third ed.,
p.~375, Theorem 39.9.} and it follows that $f=(F+f)-F$ is differentiable almost everywhere.




\section{Absolute continuity}
Let $a<b$ and let $I=[a,b]$. A function $f:I \to \mathbb{C}$ is said to be \textbf{absolutely continuous} if
for any $\epsilon>0$ there is some $\delta>0$ such that for any $n$ and any
collection of pairwise disjoint intervals $(\alpha_1,\beta_1),\ldots,(\alpha_n,\beta_n)$
satisfying
\[
\sum_{i=1}^n (\beta_i-\alpha_i) < \delta,
\]
we have
\[
\sum_{i=1}^n |f(\beta_i)-f(\alpha_i)|<\epsilon.
\]
It is immediate that if $f$ is absolutely continuous then $f$ is uniformly continuous.

\begin{lemma}
If $f:[a,b] \to \mathbb{C}$ is absolutely continuous then $f$ has bounded variation.
\label{BV}
\end{lemma}
\begin{proof}
Because $f$ is absolutely continuous, there is some $\delta>0$ such that if $(\alpha_1,\beta_1),\ldots,(\alpha_n,\beta_n)$
are pairwise disjoint and
\[
\sum_{i=1}^n (\beta_i-\alpha_i) < \delta,
\]
then
\[
\sum_{i=1}^n |f(\beta_i)-f(\alpha_i)|<1.
\]
Let $N$ be an integer that is $>\frac{b-a}{\delta}$ and let
$a=x_0<\cdots<x_N=b$ such that
$x_i-x_{i-1}<\frac{b-a}{N}$ for each $i=1,\ldots,N$. Then
\[
\Var_f[a,b] = \sum_{i=1}^N \Var_f[x_{i-1},x_i]
\leq N,
\]
showing that $f$ has bounded variation.
\end{proof}

Let $\lambda$ be Lebesgue measure on $\mathbb{R}$ and let $\mathfrak{M}$ be the collection of Lebesgue measurable
subsets of $\mathbb{R}$. 

The following theorem establishes connections between absolute continuity of a function and Lebesgue measure.\footnote{Walter
Rudin, {\em Real and Complex Analysis}, third ed., p.~146, Theorem 7.18.} In the following theorem, we extend $f:[a,b] \to \mathbb{R}$
to $\mathbb{R} \to \mathbb{R}$ by defining $f(x)=f(b)$ for $x>b$ and $f(x)=f(a)$ for $x<a$. In particular, for any $x>b$,
$f'(x)$ exists and is equal to $0$, and for any $x<a$, $f'(x)$ exists and is equal to $0$.

\begin{theorem}
Suppose that $I=[a,b]$ and that $f:I \to \mathbb{R}$ is continuous and nondecreasing. Then the following statements are equivalent.
\begin{enumerate}
\item $f$ is absolutely continuous.
\item If $E \subset I$ and $\lambda(E)=0$ then $\lambda(f(E))=0$. (In words: $f$ has the \textbf{Luzin property}.)
\item $f$ is differentiable $\lambda$-almost everywhere on $I$, $f' \in L^1(\lambda)$, and 
\[
f(x)-f(a)=\int_a^x f'(t) d\lambda(t), \qquad a \leq x \leq b.
\]
\end{enumerate}
\label{TFAE}
\end{theorem}
\begin{proof}
Assume that $f$ is absolutely continuous and let $E \subset I$ with $\lambda(E)=0$. Let $E_0=E \setminus \{a,b\}$; to prove that
$\lambda(f(E))=0$ it suffices to prove that $\lambda(f(E_0))=0$. 
Let $\epsilon>0$.
As $f$ is absolutely continuous, there is some $\delta>0$ such that
for any $n$ and any
collection of pairwise disjoint intervals $(\alpha_1,\beta_1),\ldots,(\alpha_n,\beta_n)$
satisfying
\[
\sum_{i=1}^n (\beta_i-\alpha_i) < \delta,
\]
we have
\[
\sum_{i=1}^n |f(\beta_i)-f(\alpha_i)|<\epsilon.
\]
There is an open set $V$ such that $E_0 \subset V \subset I$ and such that $\lambda(V)<\delta$. (Lebesgue measure is outer regular.)
There are countably many pairwise disjoint intervals $(\alpha_i,\beta_i)$ such that
$V=\bigcup_i (\alpha_i,\beta_i)$. Then
\[
\sum_i (\beta_i-\alpha_i) = \lambda(V) < \delta,
\]
 so
for any $n$,
\[
\sum_{i=1}^n (\beta_i-\alpha_i) < \delta,
\]
and
because $f$ is absolutely continuous it follows that
\[
\sum_{i=1}^n |f(\beta_i)-f(\alpha_i)|<\epsilon.
\]
This is true for all $n$, so
\[
\sum_i  |f(\beta_i)-f(\alpha_i)| \leq \epsilon.
\]
Because $f$ is continuous and nondecreasing, $f(\alpha_i,\beta_i) = (f(\alpha_i),f(\beta_i))$ for each $i$. Therefore
\[
f(V) = f\left(\bigcup_i (\alpha_i,\beta_i)\right)=\bigcup_i f(\alpha_i,\beta_i)
= \bigcup_i  (f(\alpha_i),f(\beta_i)),
\]
which gives 
\[
\lambda(f(V)) = \sum_i (f(\beta_i)-f(\alpha_i)) = \sum_i |f(\beta_i)-f(\alpha_i)| \leq \epsilon.
\]
This is true for all $\epsilon>0$, so 
$\lambda(f(V))=0$. Because $f(E_0)\subset f(V)$, it follows that $f(E_0) \in \mathfrak{M}$ (Lebesgue measure is complete) and that
$\lambda(f(E_0))=0$.

Assume that for all $E \subset I$ with $\lambda(E)=0$, $\lambda(f(E))=0$. Define $g:I \to \mathbb{R}$ by
\[
g(x)=x+f(x), \qquad x \in I.
\]
Because $f$ is continuous and nondecreasing, $g$ is continuous and strictly increasing. Thus if $(\alpha,\beta) \subset I$ then
$g(\alpha,\beta)=(g(\alpha),g(\beta))$ and so
\[
\lambda(g(\alpha,\beta))=g(\beta)-g(\alpha) = \beta+f(\beta)-(\alpha+f(\alpha))
=\beta-\alpha+f(\beta)-f(\alpha),
\]
showing that if $J \subset I$ is an interval then $\lambda(g(J)) = \lambda(J)+\lambda(f(J))$.
Suppose that $E \subset I$ and $\lambda(E)=0$, and let $\epsilon>0$. There are countably many pairwise disjoint intervals
$(\alpha_i,\beta_i)$ such that $E \subset \bigcup_i (\alpha_i,\beta_i)$ and $\sum_i (\beta_i-\alpha_i)<\epsilon$,
and because $\lambda(f(E))=0$, there are countably many pairwise disjoint intervals
$(\gamma_i,\delta_i)$ such that $f(E) \subset \bigcup_i (\gamma_i,\delta_i)$ and 
$\sum_i (\delta_i-\gamma_i)<\epsilon$. Let
\[
N=f^{-1}\left( \bigcup_i (\gamma_i,\delta_i) \right) \cap \bigcup_i (\alpha_i,\beta_i)
=\bigcup_{i,j} (f^{-1}(\gamma_i,\delta_i) \cap (\alpha_i,\beta_i)) \in \mathfrak{M}.
\]
We check that 
\[
\lambda(g(N)) = \lambda(N)+\lambda(f(N)),
\]
and because
\[
 \lambda(N)+\lambda(f(N)) \leq \sum_i (\beta_i-\alpha_i)+\sum_i (\delta_i-\gamma_i)<2\epsilon
\]
we have
\[
\lambda(g(N)) < 2\epsilon.
\]
Finally, $E \subset N$ so $g(E) \subset g(N)$.
Therefore, for every $\epsilon>0$ there is some $N \in \mathfrak{M}$ with 
$g(E) \subset g(N)$ and $\lambda(g(N))<\epsilon$, from which it follows that $\lambda(g(E))=0$.

Suppose that $E \subset I$ belongs to $\mathfrak{M}$. Because
$E \in \mathfrak{M}$, there are  $E_0,E_1 \in \mathfrak{M}$ such that
$E=E_0 \cup E_1$, $\lambda(E_0)=0$, and $E_1$ is a countable union of closed sets (namely, an $F_\sigma$-set).
On the one hand, as $E_1 \subset I$, $E_1$ is a countable union of compact sets, and because
$g$ is continuous, $g(E_1)$ is a countable union of compact sets, and in particular
belongs to $\mathfrak{M}$.
On the other hand, because $\lambda(E_0)=0$, $g(E_0) \in \mathfrak{M}$. Therefore
$g(E) = g(E_0) \cup g(E_1) \in \mathfrak{M}$.
Define
$\mu:\mathfrak{M} \to [0,\infty)$ by
\[
\mu(E) = \lambda(g(E \cap I)), \qquad E \in \mathfrak{M}.
\]
If $E_i$ are countably many pairwise disjoint elements of $\mathfrak{M}$, then
$g(E_i \cap I)$ are pairwise disjoint elements of $\mathfrak{M}$, hence
\begin{align*}
\mu\left(\bigcup_i E_i \right)& = \lambda\left( g\left( \left(\bigcup_i E_i\right) \cap I \right)\right)\\
&= \lambda\left(  \bigcup_i g(E_i \cap I) \right)\\
&= \sum_i \lambda(g(E_i \cap I))\\
&=\sum_i \mu(E_i),
\end{align*}
showing that $\mu$ is a measure. If $\lambda(E)=0$, then 
$\lambda(E \cap I)=0$ so $\lambda(g(E \cap I))=0$, i.e. 
$\mu(E)=0$. This shows that $\mu$ is absolutely continuous with respect to $\lambda$. 
Therefore by the Radon-Nikodym theorem\footnote{Walter Rudin,
{\em Real and Complex Analysis}, third ed., p.~121, Theorem 6.10.} there is a unique
$h \in L^1(\lambda)$ such that
\[
\mu(E) = \int_E h d\lambda, \qquad E \in \mathfrak{M}.
\]
$h(x) \geq 0$ for $\lambda$-almost all $x \in \mathbb{R}$. 

Suppose that $x \in \mathbb{R}$ and let $E=[a,x]$. Then
$g(E)=[g(a),g(x)]$, and 
\[
\mu(E) = \int_E h(t) d\lambda(t) = \int_a^x h(t) d\lambda(t).
\]
On the other hand,
\[
\mu(E) = \lambda(g(E)) = \lambda([g(a),g(x)]) = g(x)-g(a) = x+f(x)-(a+f(a)).
\]
Hence
\[
f(x)-f(a) = \int_a^x h(t) d\lambda(t)-(x-a),
\]
i.e.,
\[
f(x)-f(a) = \int_a^x (h(t)-1) d\lambda(t).
\]
By the Lebesgue differentiation theorem,\footnote{Walter Rudin, {\em Real and Complex Analysis},
third ed., p.~141, Theorem 7.11.} 
$f'(x) = h(x)-1$ for $\lambda$-almost all $x \in \mathbb{R}$, and it follows that $f' \in L^1(\lambda)$ and
\[
f(x)-f(a) = \int_a^x f'(t) d\lambda(t), \qquad  x \in I.
\]

Assume that $f$ is differentiable $\lambda$-almost everywhere in $I$, $f' \in L^1(\lambda)$, and
\[
f(x)-f(a) = \int_a^x f'(t) d\lambda(t), \qquad x \in I.
\]
Let $\epsilon>0$ and let $(\alpha_1,\beta_1),\ldots,(\alpha_n,\beta_n)$ be pairwise disjoint intervals satisfying
\[
\sum_{i=1}^n (\beta_i-\alpha_i)<\delta.
\]
Because $f$ is nondecreasing, for $\lambda$-almost all $x \in I$, $f'(x) \geq 0$, and hence
the measure $\mu$ defined by $d\mu = f' d\lambda$ is absolutely continuous with respect to $\lambda$.
It follows\footnote{Walter Rudin, {\em Real and Complex Analysis}, third ed., p.~124, Theorem 6.11.}
that there is some $\delta>0$ such that for $E \in \mathfrak{M}$,
$\lambda(E)<\delta$ implies that $\mu(E)<\epsilon$. This gives us
\[
\mu\left(\bigcup_{i=1}^n (\alpha_i,\beta_i) \right)
<\epsilon,
\]
and as
\[
\mu(\alpha_i,\beta_i)=\int_{\alpha_i}^{\beta_i} f'(t) d\lambda(t)
=f(\beta_i)-f(\alpha_i),
\]
we get
\[
\sum_{i=1}^n f(\beta_i)-f(\alpha_i)<\epsilon.
\]
This shows that $f$ is absolutely continuous, completing the proof.
\end{proof}


The following lemma establishes properties of the total variation of absolutely continuous functions.\footnote{Walter Rudin,
{\em Real and Complex Analysis}, third ed., p.~147, Theorem 7.19.}

\begin{lemma}
Suppose that $I=[a,b]$ and that $f:I \to \mathbb{R}$ is absolutely continuous. Then the function $F:I \to \mathbb{R}$ defined
by
\[
F(x)=\Var_f[a,x], \qquad x \in I
\]
is absolutely continuous.
\label{Fvariation}
\end{lemma}
\begin{proof}
Let $\epsilon>0$. Because $f$ is absolutely continuous,
there is some $\delta>0$ such that if
 $(a_1,b_1),\ldots,(a_m,b_m)$ are disjoint intervals
with $\sum_{k=1}^m (b_k-a_k)<\delta$, then
\[
\sum_{k=1}^m |f(b_k)-f(a_k)|<\epsilon.
\]
Suppose that $(\alpha_1,\beta_1),\ldots,(\alpha_n,\beta_n)$ are disjoint intervals with
$\sum_{i=1}^n (\beta_i-\alpha_i)<\delta$. If
 $\alpha_i=t_{i,0}<\cdots<t_{i,m_i}=\beta_i$ for $i=1,\ldots,n$, then
 $(t_{i,j-1},t_{i,j})$, $1 \leq i \leq n$, $1 \leq j \leq m_i$, are disjoint intervals
whose total length is $<\delta$, hence 
\[
\sum_{i=1}^n \sum_{j=1}^{m_i} |f(t_{i,j})-f(t_{i,j-1})|<\epsilon.
\]
It follows that
\[
\sum_{i=1}^n |F(\beta_i)-F(\alpha_i)|
=\sum_{i=1}^n \Var_f[\alpha_i,\beta_i]
\leq \epsilon,
\]
which shows that $F$ is absolutely continuous.
\end{proof}


We now prove the \textbf{fundamental theorem of calculus} for absolutely continuous functions.\footnote{Walter Rudin, {\em Real and Complex Analysis}, third ed., p.~148, Theorem 7.20.}

\begin{theorem}
Suppose that $I=[a,b]$ and that $f:I \to \mathbb{R}$ is absolutely continuous. Then
$f$ is differentiable at almost all $x$ in $I$, $f' \in L^1(\lambda)$, and
\[
f(x)-f(a) = \int_a^x f'(t) d\lambda(t), \qquad x \in I.
\]
\label{FTOC}
\end{theorem}
\begin{proof}
Define
$F:I \to \mathbb{R}$  by
\[
F(x)=\Var_f[a,x], \qquad x \in I.
\]
By Lemma \ref{BV}, $f$ has bounded variation, and then using Theorem \ref{nondecreasing}, $F-f$ and $F+f$ are nondecreasing. 
Furthermore, by Lemma \ref{Fvariation}, $F$ is absolutely continuous, so $F-f$ and $F+f$ are absolutely continuous.
Let
\[
f_1 = \frac{F+f}{2}, \qquad f_2 = \frac{F-f}{2},
\]
which are thus nondecreasing and absolutely continuous.
Applying Theorem \ref{TFAE}, we get that $f_1,f_2$
are differentiable at almost all $x \in I$, $f_1',f_2' \in L^1(\lambda)$, and 
\[
f_1(x)-f_1(a)=\int_a^x f_1'(t) d\lambda(t), \qquad a \leq x \leq b
\]
and
\[
f_2(x)-f_2(a)=\int_a^x f_2'(t) d\lambda(t), \qquad a \leq x \leq b.
\]
Because $f=f_1-f_2$, $f$ is differentiable at almost all $x \in I$, $f'=f_1'-f_2' \in L^1(\lambda)$,
and
\[
f(x) - f(a) = \int_a^x f'(t) d\lambda(t), \qquad a \leq x \leq b,
\]
proving the claim.
\end{proof}


\section{Borel sets}
Let $I=[a,b]$.
Denote by $C(I)$ the set of continuous functions $I \to \mathbb{C}$, which with the norm
\[
\norm{f}_{C(I)} = \sup_{x \in I} |f(x)|, \qquad f \in C(I),
\]
is a Banach space. 
Denote by  $AC(I)$ the set of absolutely continuous functions $I \to \mathbb{C}$. 
Let $\mathscr{B}_{C(I)}$ be the Borel $\sigma$-algebra of $C(I)$. We have $AC(I) \subset C(I)$, and in the following
theorem we prove that $AC(I)$ is a Borel set in $C(I)$. 

\begin{theorem}
$AC(I) \in \mathscr{B}_{C(I)}$.
\end{theorem}
\begin{proof}
If $X,Y$ are Polish spaces, $f:X \to Y$ is continuous, $A \in \mathscr{B}_X$, and $f|A$ is injective, then
$f(A) \in \mathscr{B}_Y$.\footnote{Alexander Kechris, {\em Classical Descriptive Set Theory}, p.~89, Theorem 15.1.}
Let $X=\mathbb{C} \times L^1(I)$,
which is a Banach space with the norm
\[
\norm{(A,g)}_X = |A|+\int_a^b |g| d\lambda, \qquad (A,g) \in X.
\]
Furthermore, $\mathbb{C}$ and $L^1(I)$ are separable and thus so is $X$, so $X$ is indeed a Polish space. The Banach space $C(I)$ is separable and thus is a Polish
space. 
Define $\Phi:X \to C(I)$ by
\[
\Phi(A,g)(x) = A + \int_a^x g(t) d\lambda(t), \qquad (A,g) \in X,  \qquad x \in I.
\]

For $(A_1,g_1),(A_2,g_2) \in X$,
\begin{align*}
\norm{\Phi(A_1,g_1)-\Phi(A_2,g_2)}_{C(I)} &= \norm{(A_1-A_2)+\int_a^x (g_1(t)-g_2(t)) d\lambda(t)}_{C(I)}\\
&=|A_1-A_2|+\sup_{x \in I} \left| \int_a^x (g_1(t)-g_2(t)) d\lambda(t)\right| \\
&\leq |A_1-A_2|+\int_a^b |g_1(t)-g_2(t)| d\lambda(t)\\
&=\norm{(A_1,g_1)-(A_2,g_2)}_X,
\end{align*}
which shows that $\Phi:X \to C(I)$ is continuous.

Let $(A,g) \in X$ and $\epsilon>0$. Because $g \in L^1(I)$, there is some
$\delta>0$ such that
if $\lambda(E)<\delta$ then $\int_E |g| d\lambda<\epsilon$.\footnote{Walter Rudin, {\em Real and Complex Analysis},
third ed., p.~32, exercise 1.12.}
If $(\alpha_1,\beta_1),\ldots,(\alpha_n,\beta_n)$ are disjoint intervals whose total length is $<\delta$,
then, with $E=\bigcup_{i=1}^n (\alpha_i,\beta_i)$,
\begin{align*}
\sum_{i=1}^n |\Phi(A,g)(\beta_i)-\Phi(A,g)(\alpha_i)| &= 
\sum_{i=1}^n \left| \int_{\alpha_i}^{\beta_i} g(t) d\lambda(t) \right|\\
&\leq \sum_{i=1}^n \int_{\alpha_i}^{\beta_i} |g(t)| d\lambda(t)\\
&=\int_E |g| d\lambda\\
&<\epsilon,
\end{align*}
showing that $\Phi(A,g)$ is absolutely continuous. 
On the other hand, let $f \in AC(I)$. From Theorem \ref{FTOC}, $f$ is differentiable at almost all $x \in I$,
$f' \in L^1(I)$, and  
\[
f(x)-f(a) = \int_a^x f'(t) d\lambda(t), \qquad x \in I.
\]
Then $(f(a),f') \in X$, and the above gives us, for all $x \in I$,
\[
\Phi(f(a),f')(x) =  f(a)+\int_a^x f'(t) d\lambda(t) = f(x),
\]
thus $\Phi(f(a),f')=f$. 
Therefore
\[
\Phi(X) = AC(I).
\]

If $\Phi(A_1,g_1)=\Phi(A_2,g_2)$, then $\Phi(A_1,g_1)(a)=\Phi(A_2,g_2)(a)$ gives $A_1=A_2$. Using this, and defining
$G:I \to \mathbb{C}$ by
$G=\int_a^x (g_1(t)-g_2(t)) d\lambda(t)$, we have
$G(x)=0$   for all $x \in I$. Then $G'(x)=0$ for all $x \in I$, and by the Lebesgue differentiation theorem\footnote{Walter Rudin,
{\em Real and Complex Analysis}, third ed., p.~141, Theorem 7.11.} we have $G'(x)=g_1(x)-g_2(x)$ for almost all $x \in I$. That is,
$g_1(x)=g_2(x)$ for almost all $x \in I$, and thus in $L^1(I)$ we have $g_1=g_2$. 
Therefore $\Phi:X \to C(I)$ is injective. 

Therefore $\Phi(X) \in \mathscr{B}_{C(I)}$.
\end{proof}

\end{document}