\documentclass{article}
\usepackage{amsmath,amssymb,mathrsfs,amsthm}
%\usepackage{tikz-cd}
\usepackage{hyperref}
\newcommand{\inner}[2]{\left\langle #1, #2 \right\rangle}
\newcommand{\tr}{\ensuremath\mathrm{tr}\,} 
\newcommand{\Span}{\ensuremath\mathrm{span}} 
\def\Re{\ensuremath{\mathrm{Re}}\,}
\def\Im{\ensuremath{\mathrm{Im}}\,}
\newcommand{\id}{\ensuremath\mathrm{id}} 
\newcommand{\var}{\ensuremath\mathrm{var}} 
\newcommand{\Lip}{\ensuremath\mathrm{Lip}} 
\newcommand{\Sh}{\ensuremath\mathrm{Sh}} 
\newcommand{\GL}{\ensuremath\mathrm{GL}} 
\newcommand{\diam}{\ensuremath\mathrm{diam}} 
\newcommand{\sgn}{\ensuremath\mathrm{sgn}}
\newcommand{\Var}{\ensuremath\mathrm{Var}} 
\newcommand{\lcm}{\ensuremath\mathrm{lcm}} 
\newcommand{\supp}{\ensuremath\mathrm{supp}\,}
\newcommand{\dom}{\ensuremath\mathrm{dom}\,}
\newcommand{\upto}{\nearrow}
\newcommand{\downto}{\searrow}
\newcommand{\norm}[1]{\left\Vert #1 \right\Vert}
\newtheorem{theorem}{Theorem}
\newtheorem{lemma}[theorem]{Lemma}
\newtheorem{proposition}[theorem]{Proposition}
\newtheorem{corollary}[theorem]{Corollary}
\theoremstyle{definition}
\newtheorem{definition}[theorem]{Definition}
\newtheorem{example}[theorem]{Example}
\begin{document}
\title{The weak and strong laws of large numbers}
\author{Jordan Bell\\ \texttt{jordan.bell@gmail.com}\\Department of Mathematics, University of Toronto}
\date{\today}

\maketitle

\section{Introduction}
Using Egorov's theorem, one proves that if
a sequence of random variables $X_n$ \textbf{converges almost surely} to a random variable $X$ then
$X_n$ \textbf{converges in probability} to $X$.\footnote{
V. I. Bogachev, {\em Measure Theory}, volume I, p.~111, Theorem 2.2.3;
\url{http://individual.utoronto.ca/jordanbell/notes/L0.pdf},
p.~3, Theorem 3.}

Let $X_n$ be a sequence of $L^1$ random variables, $n \geq 1$. 
\textbf{A weak law of large numbers} is a statement 
that
\begin{equation}
\frac{1}{n} \sum_{k=1}^n(X_k-E(X_k))
\label{empirical}
\end{equation}
 converges in probability to $0$.
\textbf{A strong law of large numbers} is a statement
that
\eqref{empirical}
 converges almost surely to $0$. 
Thus, if the hypotheses assumed on the sequence of random variables
are the same, a strong law implies a weak law.

We shall
prove the weak law of large numbers for a sequence of
 independent identically distributed $L^1$ random variables, and the strong law of large
for the same hypotheses. We give separate proofs for these theorems as an occasion  to inspect different machinery,
although to establish the weak law it thus suffices
to prove the strong law. One reason to distinguish these laws is for cases when we impose different hypotheses.\footnote{cf.
Jordan M. Stoyanov, {\em Counterexamples in Probability}, third ed., p.~163, \S 15.3;
Dieter Landers and Lothar Rogge, 
{\em Identically distributed uncorrelated random variables not fulfilling the WLLN},
Bull. Korean Math. Soc. \textbf{38} (2001), no. 3, 605--610.}

We also prove \textbf{Markov's weak law of large numbers}, which 
states that if $X_n$ is a sequence of $L^2$ random variables that are pairwise uncorrelated and
\[
\frac{1}{n^2} \sum_{n=k}^n \Var(X_k) \to 0,
\]
then $\frac{1}{n} \sum_{k=1}^n (X_k-E(X_k))$ converges to $0$ in $L^2$, from which it follows using Chebyshev's inequality that
it converges in probability to $0$. (We remark that a sequence of $L^2$ random variables converging in $L^2$ to $0$ does not
imply that it converges almost surely to $0$, although there is indeed a subsequence that converges
almost surely to $0$.\footnote{\url{http://individual.utoronto.ca/jordanbell/notes/L0.pdf}, p.~4, Theorem 5.})


If $(\Omega,\mathscr{F},P)$ is a probability space,
 $(Y,\mathscr{A})$ is a measurable space, and
$T:(\Omega,\mathscr{F}) \to (Y,\mathscr{A})$ is measurable,
the \textbf{pushforward measure of $P$ by $T$} is 
\[
(T_*P)(A) = P(T^{-1}(A)), \qquad A \in \mathscr{A}.
\]
Then $(Y,\mathscr{A},T_*P)$ is a probability space.
We remind ourselves of the \textbf{change of variables theorem}, which we shall use.\footnote{Charalambos D. Aliprantis
and Kim C. Border, {\em Infinite Dimensional Analysis: A Hitchhiker�s Guide}, third ed., p.~485, Theorem 13.46.}
Let $f:Y \to \mathbb{R}$ be a function. On the one hand, if $f \in L^1(T_*P)$ then
 $f \circ T \in L^1(P)$ and
\begin{equation}
\int_Y f d(T_*P) = \int_\Omega f \circ T dP.
\label{COV}
\end{equation}
On the other hand, if $f$ is $T_*P$-measurable and $f \circ T \in L^1(P)$, then 
$f \in L^1(T_*P)$ and
\[
\int_Y f d(T_*P) = \int_\Omega f \circ T dP.
\]


\section{The weak law of large numbers}
Suppose that
$X_n:(\Omega,\mathscr{F},P) \to \mathbb{R}$, $n \geq 1$, are independent
identically distributed $L^1$ random variables, and write
\[
S_n = \sum_{k=1}^n X_k,
\]
for which $E(S_n) = \sum_{k=1}^n E(X_k) = nE(X_1)$.

\begin{lemma}
If $X_n$ are $L^2$, then for any $\lambda>0$ and for any
$n \geq 1$,
\[
P\left(\left| \frac{S_n}{n} - E(X_1)\right| \geq \lambda\right) \leq \frac{E(|X_1-E(X_1)|^2)}{n \lambda^2}.
\]
\label{secondmoment}
\end{lemma}
\begin{proof}
Using
\[
S_n^2 = \sum_{k=1}^n X_k^2 + \sum_{j \neq k} X_j X_k,
\]
we have
\begin{align*}
E\left( \left| \frac{S_n}{n} - E(X_1) \right|^2 \right)&=
E\left( \sum_{k=1}^n \frac{X_k^2}{n^2} + \sum_{j \neq k} \frac{X_j X_k}{n^2}
-2E(X_1) \frac{S_n}{n} + E(X_1)^2 \right)\\
&=\sum_{k=1}^n \frac{E(X_k^2)}{n^2} + 
\sum_{j \neq k} \frac{E(X_j X_k)}{n^2}
-2E(X_1)^2
+E(X_1)^2\\
&=\sum_{k=1}^n \frac{E(X_1^2)}{n^2} + \sum_{j \neq k} \frac{E(X_j) E(X_k)}{n^2}-E(X_1)^2\\
&=\sum_{k=1}^n \frac{E(X_1^2)}{n^2}
+\sum_{j \neq k} \frac{E(X_1)^2}{n^2}
+E(X_1)^2\\
&=\frac{E(X_1^2)}{n}+\frac{E(X_1)^2}{n}.
\end{align*}
On the other hand,
\[
E(|X_1-E(X_1)|^2) = E(X_1^2-2E(X_1)X_1+E(X_1)^2)
=E(X_1^2)+E(X_1)^2.
\]
So
\[
E\left( \left| \frac{S_n}{n} - E(X_1) \right|^2 \right) = \frac{1}{n} E(|X_1-E(X_1)|^2).
\]
Using this and Chebyshev's inequality,
\begin{align*}
P\left(\left| \frac{S_n}{n} - E(X_1)\right| \geq \lambda\right)&\leq 
\frac{1}{\lambda^2} E\left(\left| \frac{S_n}{n} - E(X_1)\right|^2 \right)\\
&=\frac{1}{n\lambda^2}E(|X_1-E(X_1)|^2),
\end{align*}
proving the claim.
\end{proof}



We now prove the \textbf{weak law of large numbers}, which states that if
$X_n$ are independent identically distributed $L^1$ random variables each with mean $0$, then
$\frac{S_n}{n}$ converges in probability to $E(X_1)$.
$Z_n=X_n-E(X_n)=X_n-E(X_1)$ are independent and identically distributed $L^1$ random variables
with mean $0$, and if $Z_n$ converges in probability to $0$ then $X_n=Z_n+E(X_1)$ converges in probability
to $E(X_1)$, showing that it suffices to prove the theorem when $E(X_1)=0$.\footnote{Allan Gut, {\em Probability: A Graduate Course}, second ed.,
p.~270, Theorem 6.3.1, and p.~121, Theorem 3.1.5.}

\begin{theorem}[Weak law of large numbers]
Suppose that $E(X_1)=0$.
For each $\epsilon>0$,
\[
P\left( \left| \frac{S_n}{n} \right| \geq \epsilon \right) \to 0, \qquad n \to \infty.
\]
\label{weaklaw}
\end{theorem}
\begin{proof}
For $n \geq 1$ and $1 \leq k \leq n$, let
\[
Y_{n,k} = 1_{\{|X_k| \leq n \epsilon^3\}} X_k = f \circ X_k,
\]
where $f(x)=1_{[-n\epsilon^3,n\epsilon^3]}(x) \cdot x$. 
Because $X_1,\ldots,X_k$ are independent and identically distributed, so are 
$Y_{n,1},\ldots,Y_{n,n}$.\footnote{Gerald B. Folland, {\em Real Analysis: Modern Techniques and Their Applications},
second ed., p.~316, Proposition 10.2.}
Moreover, $E(Y_{n,k}^2) \leq (n\epsilon^3)^2$, so each $Y_{n,k}$ belongs to $L^2$. 
Let
\[
T_n = \sum_{k=1}^n Y_{n,k},
\]
for which $E(T_n) = n E(Y_{n,1})$. 


If $\omega \in \bigcap_{k=1}^n \{|X_k| \leq n \epsilon^3\}$, then
\[
T_n(\omega) = \sum_{k=1}^n Y_{n,k}(\omega) = \sum_{k=1}^n X_k(\omega) = S_n(\omega),
\]
and using this and Lemma \ref{secondmoment},
for $t>0$ we have
\begin{align*}
P(|S_n-E(T_n)| \geq t)&=P\left( \{|S_n - E(T_n)| \geq t\} \cap \bigcap_{k=1}^n \{|X_k| \leq n \epsilon^3\} \right)\\
&+P\left( \{|S_n - E(T_n)| \geq t\} \cap \bigcup_{k=1}^n \{|X_k| > n \epsilon^3\} \right)\\
&\leq P(|T_n-E(T_n)| \geq t)
+P\left(  \bigcup_{k=1}^n \{|X_k| > n \epsilon^3\} \right)\\
&=P\left(\left|\frac{T_n}{n} -  E(Y_{n,1})\right| \geq \frac{t}{n}\right) + P\left(  \bigcup_{k=1}^n \{|X_k| > n \epsilon^3\} \right)\\
&\leq \frac{E(|Y_{n,1}-E(Y_{n,1})|^2)}{n \cdot \left(\frac{t}{n}\right)^2}
+\sum_{k=1}^n P(|X_k| > n\epsilon^3)\\
&=\frac{n}{t^2} \cdot (E(Y_{n,1}^2)-E(Y_{n,1})^2)+ nP(|X_1|>n\epsilon^3)\\
&\leq \frac{n}{t^2}  \cdot E(Y_{n,1}^2) + nP(|X_1|>n\epsilon^3).
\end{align*}
For $t=n\epsilon$ this is
\begin{align*}
P(|S_n-E(T_n)| \geq n\epsilon)&\leq \frac{1}{n \epsilon^2} \cdot E(Y_{n,1}^2)
+nP(|X_1|>n\epsilon^3)\\
&=\frac{1}{n \epsilon^2} \cdot E(1_{\{|X_1| \leq n \epsilon^3\}} X_1^2)+ nP(|X_1|>n\epsilon^3)\\
&\leq \frac{1}{n\epsilon^2} \cdot E(1_{\{|X_1| \leq n \epsilon^3\}} n\epsilon^3 |X_1|) + nP(|X_1|>n\epsilon^3)\\
&\leq \epsilon E(|X_1|) + nP(|X_1|>n\epsilon^3).
\end{align*}
Now, 
\[
n\epsilon^3 P(|X_1|>n\epsilon^3)
=n\epsilon^3 \int_{(n\epsilon^3,\infty)} d({X_1}_*P)(y)
\leq   \int_{(n\epsilon^3,\infty)} y d({X_1}_*P)(y),
\]
which tends to $0$ as $n \to \infty$ because
\[
\int_{\mathbb{R}} |y| d({X_1}_*P)(y) = E(|X_1|) < \infty.
\]
Therefore,
\[
\limsup_n P(|S_n-E(T_n)| \geq n\epsilon) \leq \epsilon E(|X_1|),
\]
that is, 
\[
\limsup_n P\left(\left|\frac{S_n-E(T_n)}{n} \right| \geq \epsilon\right) \leq \epsilon E(|X_1|),
\]
which implies that $\frac{S_n-E(T_n)}{n}$ converges in probability to $0$. 

Because $E(X_1)=0$,
\[
E(1_{\{|X_1| \leq n \epsilon^3\}} X_1) + E(1_{\{|X_1| > n \epsilon^3\}} X_1) = 0,
\]
and hence
\[
|E(T_n)|=|nE(Y_{n,1})|
=n|E(1_{\{|X_1| \leq n \epsilon^3\}} X_1)|
= n | E(1_{\{|X_1| > n\epsilon^3\}} X_1)|,
\]
thus
\[
\frac{|E(T_n)|}{n} = | E(1_{\{|X_1| > n\epsilon^3\}} X_1)| 
\leq E(1_{\{|X_1|>n\epsilon^3\}} |X_1|),
\]
which tends to $0$ as $n \to \infty$, because $E(|X_1|)<\infty$. 
Thus $\frac{E(T_n)}{n}$ converges in probability to $0$, and therefore
$\frac{S_n}{n}$ converges in probability to $0$, completing the proof.
\end{proof}



\begin{lemma}[Bienaym\'e's formula]
If $X_n$, $n \geq 1$, are $L^2$ random variables that are pairwise uncorrelated, then
\[
\Var\left(\sum_{k=1}^n X_k \right) = \sum_{k=1}^n \Var(X_k).
\]
\label{bienayme}
\end{lemma}
\begin{proof}
Let $Y_k=X_k-E(X_k)$. 
Using that the $X_k$ are pairwise uncorrelated, for $j \neq k$ we have
\begin{align*}
E(Y_j Y_k)&=E(X_j X_k -X_jE(X_k)-X_kE(X_j)+E(X_j)E(X_k))\\
&=E(X_j) E(X_k) - E(X_j)E(X_k)-E(X_k)E(X_j)+E(X_j)E(X_k)\\
&=0,
\end{align*}
showing that the $Y_k$ are pairwise uncorrelated. Then, because $E(S_n)=\sum_{k=1}^n E(X_k)$,
\begin{align*}
\Var(S_n)&=E\left(\left(S_n - \sum_{k=1}^n E(X_k)\right)^2\right)\\
&=E\left(\left(\sum_{k=1}^n Y_k\right)^2\right)\\
&=E\left(\sum_{k=1}^n Y_k^2 + \sum_{j \neq k} Y_j Y_k\right)\\
&=\sum_{k=1}^n E(Y_k^2) + \sum_{j \neq k} E(Y_j) E(Y_k)\\
&=\sum_{k=1}^n E(Y_k^2),
\end{align*}
and as $E(Y_k^2) = \Var(X_k)$, we have
\[
\Var(S_n) = \sum_{k=1}^n \Var(X_k).
\]
\end{proof}

We now prove a weak law of large numbers, that is sometimes attributed to Markov,
that neither supposes that the random variables are independent nor
supposes that they are identically distributed. We remind ourselves that if a sequence $Y_n$ of $L^2$ random variables converges
to $0$ in $L^2$ then it converges in probability to $0$;
this is proved using Chebyshev's inequality.

\begin{theorem}[Markov's weak law of large numbers]
If $X_n$, $n \geq 1$, are $L^2$ random variables that are pairwise uncorrelated and  which satisfy
\[
\frac{1}{n^2} \sum_{n=k}^n \Var(X_k) \to 0,
\]
then $\frac{1}{n} \sum_{k=1}^n (X_k-E(X_k))$ converges to $0$ in $L^2$ and thus in probability.
\end{theorem}
\begin{proof}
Because $E\left( \sum_{k=1}^n X_k\right) =  \sum_{k=1}^n E(X_k)$
and $\Var(X)=E((X-E(X))^2)$,
using Bienaym\'e's formula we get
\begin{align*}
E\left( \left( \frac{1}{n} \sum_{k=1}^n (X_k-E(X_k)) \right)^2 \right)&=
\frac{1}{n^2} E\left(\left(\sum_{k=1}^n X_k - \sum_{k=1}^n E(X_k)\right)^2 \right)\\
&=\frac{1}{n^2} \Var\left( \sum_{k=1}^n X_k \right)\\
&=\frac{1}{n^2} \sum_{k=1}^n \Var(X_k).
\end{align*}
Thus
\[
E\left( \left( \frac{1}{n} \sum_{k=1}^n (X_k-E(X_k)) \right)^2 \right) = \frac{1}{n^2} \sum_{k=1}^n \Var(X_k) \to 0
\]
as $n \to \infty$, namely, $ \frac{1}{n} \sum_{k=1}^n (X_k-E(X_k))$ converges to $0$ in $L^2$ as $n \to \infty$, proving the claim.
\end{proof}



\section{Ergodic theory}
We here assemble 
machinery and results that we will use to prove the strong law of large numbers. For a probability
space $(\Omega,\mathscr{F},P)$, a function $T:\Omega \to \Omega$ is said to be a \textbf{measure-preserving transformation} if
(i) it is measurable, and (ii) $T_*P=P$. To say that $T_*P=P$ means that for any $A \in \mathscr{F}$,
$(T_*P)(A)=P(A)$, i.e. $P(T^{-1}(A))=P(A)$. 

A collection $\mathscr{A}$ of subsets of $\Omega$ is called an \textbf{algebra of sets} if
(i) $\emptyset \in \mathscr{A}$, (ii) $\Omega \in \mathscr{A}$, (iii) if $A,B \in \mathscr{A}$ then
$A \cup B, A \setminus B \in \mathscr{A}$. If $\mathscr{G}$ is a nonempty collection of subsets of $\Omega$, it is a fact that there is a unique
algebra of sets $A(\mathscr{G})$ that (i) contains $\mathscr{G}$ and (ii) is contained in any algebra of sets that contains $\mathscr{G}$. We call
$A(\mathscr{G})$ the \textbf{algebra of sets generated by $\mathscr{G}$}.


A collection $\mathscr{S}$ of subsets of $\Omega$ is called a \textbf{semialgebra of sets} if
(i) $\emptyset \in \mathscr{S}$, (ii) $\Omega \in \mathscr{S}$, (iii) if $A,B \in \mathscr{S}$ then $A \cap B \in \mathscr{S}$,
(iv) if $A,B \in \mathscr{S}$ then there are pairwise disjoint $E_1,\ldots,E_n \in \mathscr{S}$ such that
\[
A \setminus B = \bigcup_{k=1}^n E_k.
\] 
It is a fact that $A(\mathscr{S})$ is equal to the collection of all unions of finitely many pairwise disjoint
elements of $\mathscr{S}$.\footnote{V. I. Bogachev, {\em Measure Theory}, volume I, p.~8,
Lemma 1.2.14.}


A nonempty collection $\mathscr{M}$ of subsets of $\Omega$ is called a
\textbf{monotone class} if whenever $A_n \in \mathscr{M}$, $A_n \subset A_{n+1}$ (an \textbf{increasing sequence
of sets}), implies
that $\bigcup_n A_n \in \mathscr{M}$ and
$A_n \in \mathscr{M}$, $A_{n+1} \subset A_n$ (a \textbf{decreasing sequence of sets}),
implies that $\bigcap_n A_n \in \mathscr{M}$.
In other words, a monotone class is a nonempty collection of subsets of $\Omega$ such that
if $A_n$ is a monotone sequence in $\mathscr{M}$ then $\lim_{n \to \infty} A_n \in \mathscr{M}$.
If $\mathscr{G}$ is a nonempty collection of subsets of $\Omega$, it is a fact that there is a unique
monotone class $M(\mathscr{G})$ that (i) contains $\mathscr{G}$ and (ii) is contained in any monotone class that contains $\mathscr{G}$. We call
$M(\mathscr{G})$ the \textbf{monotone class generated by $\mathscr{G}$}.
The \textbf{monotone class theorem} states that if $\mathscr{A}$ is an algebra of sets then 
$\sigma(\mathscr{A})=M(\mathscr{A})$.\footnote{V. I. Bogachev, {\em Measure Theory}, volume I, p.~33,
Theorem 1.9.3.}


The following lemma gives conditions under which we can establish that a function is measure-preserving.\footnote{Peter Walters, {\em An Introduction to Ergodic Theory}, p.~20, Theorem 1.1.}

\begin{lemma}
Let $T:\Omega \to \Omega$ be a function and suppose that $\mathscr{S}$ is a semialgebra of sets for which
$\mathscr{F}$ is the $\sigma$-algebra generated by $\mathscr{S}$. If
(i) $T^{-1}(A) \in \mathscr{F}$ for each $A \in \mathscr{S}$ and
(ii) $P(T^{-1}(A))=P(A)$ for each $A \in \mathscr{S}$, then $T$ is measure-preserving.
\label{semialgebra}
\end{lemma}
\begin{proof}
Let
\[
\mathscr{C} = \{A \in \mathscr{F}: T^{-1}(A) \in \mathscr{F}, P(T^{-1}(A))=P(A)\};
\]
we wish to prove that $\mathscr{C}=\mathscr{F}$. 

If $A_n \in \mathscr{C}$ is an increasing sequence, let $A = \bigcup_{n=1}^\infty A_n$. 
Then, as $T^{-1}(A_n) \in \mathscr{F}$ for each $n$,
\[
T^{-1}(A) = \bigcup_{n=1}^\infty T^{-1}(A_n) \in \mathscr{F}
\]
and as (i) $T^{-1}(A_n)$ is an increasing sequence,
(ii) $P$ is continuous from below,\footnote{V. I. Bogachev, {\em Measure Theory}, volume I, p.~9, Proposition 1.3.3.}
and (iii) $P(T^{-1}(A_n))=P(A_n)$,
\begin{align*}
P(T^{-1}(A))&=P\left(\bigcup_{n=1}^\infty T^{-1}(A_n) \right)\\
&=\lim_{n \to \infty} P(T^{-1}(A_n))\\
&=\lim_{n \to \infty} P(A_n)\\
&=P\left( \bigcup_{n=1}^\infty A_n \right)\\
&=P(A),
\end{align*}
and hence $A \in \mathscr{C}$.
If $A_n \in \mathscr{C}$ is a decreasing sequence, let $A = \bigcap_{n=1}^\infty A_n$. 
Because $T^{-1}(A_n) \in \mathscr{F}$,
\[
T^{-1}(A) = \bigcap_{n=1}^\infty T^{-1}(A_n) \in \mathscr{F},
\]
and as (i) $T^{-1}(A_n)$ is a decreasing sequence, 
(ii) $P$ is continuous from above, and (iii) $P(T^{-1}(A_n))=P(A_n)$,
\begin{align*}
P(T^{-1}(A))&=P\left(\bigcap_{n=1}^\infty T^{-1}(A_n) \right)\\
&=\lim_{n \to \infty} P(T^{-1}(A_n))\\
&=\lim_{n \to \infty} P(A_n)\\
&=P\left(\bigcap_{n=1}^\infty A_n \right)\\
&=P(A),
\end{align*}
and hence $A \in \mathscr{C}$. 
Therefore, $\mathscr{C}$ is a monotone class.

$\mathscr{S} \subset \mathscr{C}$. If $A \in A(\mathscr{S})$, then there are pairwise disjoint
$A_1,\ldots,A_n \in \mathscr{S}$ with $A = \bigcup_{k=1}^n A_k$. 
As $T^{-1}(A_k) \in \mathscr{F}$, 
\[
T^{-1}(A) = \bigcup_{k=1}^n T^{-1}(A_k) \in \mathscr{F}.
\]
As the $A_k$ are pairwise disjoint, so are the sets $T^{-1}(A_k)$, so, because $P(T^{-1}(A_k))=P(A_k)$,
\begin{align*}
P(T^{-1}(A)) &= P\left( \bigcup_{k=1}^n T^{-1}(A_k) \right)\\
&=\sum_{k=1}^n P(T^{-1}(A_k))\\
& = \sum_{k=1}^n P(A_k)\\
&=P\left(\bigcup_{k=1}^n A_k \right)\\
&=P(A).
\end{align*}
Therefore $A \in \mathscr{C}$. This shows that $A(\mathscr{S}) \subset \mathscr{C}$. 

The monotone class theorem tells us
$\sigma(A(\mathscr{S})) = M(A(\mathscr{S}))$.
 On the one hand, 
$\mathscr{F}=\sigma(\mathscr{S})$ and hence
$\mathscr{F}=\sigma(A(\mathscr{S}))$. 
On the other hand,
$A(\mathscr{S}) \subset \mathscr{C}$ and the fact that $\mathscr{C}$ is a monotone class  yield
\[
M(A(\mathscr{S})) \subset M(\mathscr{C})=\mathscr{C}.
\]
Therefore
\[
\mathscr{F} \subset \mathscr{C}.
\]
Of course $\mathscr{C} \subset \mathscr{F}$, so $\mathscr{C} = \mathscr{F}$, proving the claim.
\end{proof}


Let $(\Omega,\mathscr{F},P)$ be a probability space.  A measure-preserving transformation
$T:\Omega \to \Omega$ is called \textbf{ergodic} if $A \in \mathscr{F}$ and $T^{-1}(A)=A$ implies that
$P(A)=0$ or $P(A)=1$.
It is proved\footnote{Peter Walters,
{\em An Introduction to Ergodic Theory}, p.~37, Corollary 1.14.2.}
 using the \textbf{Birkhoff ergodic theorem} that for a measure-preserving transformation
$T:\Omega \to \Omega$, $T$ is ergodic if and only if for all $A,B \in \mathscr{F}$,
\[
\frac{1}{n} \sum_{k=0}^{n-1} P(T^{-1}(A) \cap B) \to P(A)P(B).
\]

A measure-preserving transformation $T:\Omega \to \Omega$ is called
\textbf{weak-mixing} if for all $A,B \in \mathscr{F}$,
\[
\frac{1}{n} \sum_{k=0}^{n-1} |P(T^{-k}(A) \cap B) - P(A) P(B)| \to 0.
\]
It is immediate that a weak-mixing transformation is ergodic.

A measure-preserving transformation $T:\Omega \to \Omega$ is called
\textbf{strong-mixing} if for all $A, B \in \mathscr{F}$,
\[
P(T^{-n}(A) \cap B) \to P(A) P(B).
\]
If a sequence of real numbers $a_n$ tends to $0$, then
\[
\frac{1}{n} \sum_{k=0}^{n-1} |a_k| \to 0,
\]
and using this we check that a strong-mixing transformation is weak-mixing.


The following statement gives conditions under which a measure-preserving transformation
is ergodic, weak-mixing, or strong-mixing.\footnote{Peter Walters,
{\em An Introduction to Ergodic Theory}, p.~41, Theorem 1.17.}

\begin{theorem}
Let $(\Omega,\mathscr{F},P)$ be a probability space, let $\mathscr{S}$ be a semialgebra that generates
$\mathscr{F}$, and let $T:\Omega \to \Omega$ be a measure-preserving transformation.
\begin{enumerate}
\item $T$ is ergodic if and only if
for all $A,B \in \mathscr{S}$,
\[
\frac{1}{n} \sum_{k=0}^{n-1} P(T^{-k}(A) \cap B) \to P(A)P(B).
\]
\item $T$ is weak-mixing if and only if for all $A,B \in \mathscr{S}$,
\[
\frac{1}{n} \sum_{k=0}^{n-1} |P(T^{-k}(A) \cap B) - P(A)P(B)| \to 0.
\]
\item $T$ is strong-mixing if and only if for all $A,B \in \mathscr{S}$,
\[
P(T^{-n}(A) \cap B) \to P(A)P(B).
\]
\label{ergodic}
\end{enumerate}
\end{theorem}



\section{The strong law of large numbers}
Let $\mu$ be a Borel probability measure on $\mathbb{R}$ with \textbf{finite first moment}:
\[
\int_{\mathbb{R}} |x| dm(x) < \infty.
\]
We shall specify when we use the hypothesis that $\mu$ has finite first moment; until we say so, what we say merely
supposes that it is a Borel probability measure on $\mathbb{R}$.

For $n \geq 0$, let $\Omega_n = \mathbb{R}$, let
$\mathscr{B}_n=\mathscr{B}_{\mathbb{R}}$, the Borel $\sigma$-algebra of $\mathbb{R}$, and let $\mu_n=\mu$,
for which  $(\Omega_n,\mathscr{B}_n,\mu_n)$ is a probability space. 
Let $\Omega=\prod_{n=0}^\infty \Omega_n$. A \textbf{cylinder set} is a subset of $\Omega$ of the form
\[
\prod_{n=0}^\infty A_n,
\]
where $A_n \in \mathscr{B}_n$ for each $n$ and where $\{n \geq 0: A_n \neq \Omega_n\}$ is finite. 
We denote by $\mathscr{C}$ the collection of all cylinder sets. It is a fact that
$\mathscr{C}$ is a semialgebra of sets.\footnote{S. J. Taylor, {\em Introduction to Measure Theory and Integration},
p.~136, \S 6.1;
\url{http://individual.utoronto.ca/jordanbell/notes/productmeasure.pdf}}

The \textbf{product $\sigma$-algebra}
is $\mathscr{F}=\sigma(\mathscr{C})$, the $\sigma$-algebra generated by the collection of all cylinder sets.
The \textbf{product measure},\footnote{\url{http://individual.utoronto.ca/jordanbell/notes/productmeasure.pdf}} which we denote by $P$, is the unique probability measure
on $\mathscr{F}$ such that for any cylinder set $\prod_{n=0}^\infty A_n$,
\[
P(\prod_{n=0}^\infty A_n) = \prod_{n=0}^\infty \mu_n(A_n).
\]

Let $\pi_n:\Omega \to \Omega_n$, $n \geq 0$, be the projection map. Define $\tau:\Omega \to \Omega$ by
\[
\tau(\omega_0,\omega_1,\ldots) = (\omega_1,\omega_2,\ldots),
\]
the \textbf{left shift map}.
In other words, for $n \geq 0$, $\tau$ satisfies $\pi_n \circ \tau = \pi_{n+1}$, and so
$\pi_n  = \pi_0 \circ (\tau^n)$. 

\begin{lemma}
$\tau$ is measure-preserving.
\end{lemma}
\begin{proof}
For $A=\prod_{n=0}^\infty A_n \in \mathscr{C}$, 
\[
\tau^{-1}(A) = \prod_{n=0}^\infty B_n=B,
\]
where $B_0=\Omega_0$ and for $n \geq 1$, $B_n=A_{n-1}$. $B$ is a cylinder set so a fortiori belongs to $\mathscr{F}$, and
\[
P(B) = \prod_{n=0}^\infty \mu_n(B_n) = \mu_0(\Omega_0) \cdot \prod_{n=1}^\infty \mu_n(B_n)
=\prod_{n=1}^\infty \mu_n(A_{n-1})
=\prod_{n=1}^\infty \mu_{n-1}(A_{n-1}),
\]
so
\[
P(\tau^{-1}(A)) = \prod_{n=0}^\infty \mu_n(A_n) = P(A).
\]
Therefore by Lemma \ref{semialgebra}, because
$\mathscr{C}$ is a semialgebra that generates $\mathscr{F}$, it follows that
$\tau$ is measure-preserving.
\end{proof}


\begin{lemma}
$\tau$ is strong-mixing.
\end{lemma}
\begin{proof}
Let $A=\prod_{n=0}^\infty A_n$ and $B=\prod_{n=0}^\infty B_n$ be cylinder sets.
For $n \geq 0$,
\[
\tau^{-n}(A) = \prod_{m=0}^\infty C_m,
\]
where $C_m=\Omega_m$ for $0 \leq m \leq n-1$ and 
$C_m=A_{m-n}$ for $m \geq n$. Because $A$ and $B$ are cylinder sets, there is some $N$ such that when
$m \geq N$, $A_m = \Omega_m$ and $B_m=\Omega_m$. Thus for $n \geq N$,
\begin{align*}
\tau^{-n}(A) \cap B) &= \prod_{m=0}^\infty C_m \cap \prod_{m=0}^\infty B_m\\
&=\prod_{m=0}^\infty (C_m \cap B_m)\\
&= \prod_{m=0}^{n-1} (C_m \cap B_m) \times \prod_{m=n}^\infty (C_m \cap B_m)\\
&=\prod_{m=0}^{n-1} (\Omega_m \cap B_m) \times \prod_{m=n}^\infty (A_{m-n} \cap \Omega_m)\\
&=\prod_{m=0}^{n-1} B_m \times \prod_{m=n}^\infty A_{m-n}.
\end{align*}
Hence
\begin{align*}
P(\tau^{-n}(A) \cap B) &= \prod_{m=0}^{n-1} \mu_m(B_m) \cdot \prod_{m=n}^\infty \mu_m(A_{m-n})\\
&=\prod_{m=0}^{n-1} \mu_m(B_m) \cdot \prod_{m=0}^\infty \mu_{m+n}(A_m)\\
&=\prod_{m=0}^{n-1} \mu_m(B_m) \cdot \prod_{m=0}^\infty \mu_m(A_m)\\
&=P(B) \cdot P(A).
\end{align*}
That is, there is some $N$ such that when $n \geq N$,
\[
P(\tau^{-n}(A) \cap B)  = P(A) P(B),
\]
and  so a fortiori,
\[
\lim_{n \to \infty} P(\tau^{-n}(A) \cap B) = P(A) P(B).
\]
Therefore, because the cylinder sets generate the $\sigma$-algebra $\mathscr{F}$, by Theorem \ref{ergodic}
we get that $\tau$ is strong-mixing.
\end{proof}



We now use the hypothesis that $\mu$ has finite first moment.\footnote{Elias M. Stein and Rami Shakarchi,
{\em Functional Analysis}, Princeton Lectures in Analysis, volume IV, p.~208, chapter 5, \S 2.1.}

\begin{lemma}
$\pi_0 \in L^1(P)$.
\end{lemma}
\begin{proof}
$T=\pi_0:\Omega \to \Omega_0$ is measurable,
and $T_*P=\mu_0$. 
The statement that $\mu=\mu_0$ has finite first moment means that
$f:\Omega_0 \to \mathbb{R}$ defined by
$f(x)=|x|$ belongs to $L^1(\mu_0)$. 
Therefore by the change of variables theorem \eqref{COV},
we have $f \circ T \in L^1(P)$. 
\end{proof}

\begin{lemma}
For almost all $\omega \in \Omega$,
\[
\frac{1}{n} \sum_{k=0}^{n-1} \pi_k(\omega) \to \int_{\mathbb{R}} t d\mu(t).
\]
\label{pi0}
\end{lemma}
\begin{proof}
Because 
 $\tau:\Omega \to \Omega$ is ergodic and
$\pi_0 \in L^1(P)$,
the \textbf{Birkhoff ergodic theorem}\footnote{Peter Walters,
{\em An Introduction to Ergodic Theory}, p.~34, Theorem 1.14.}
tells us that
for almost all $\omega \in \Omega$,
\[
\frac{1}{n} \sum_{k=0}^{n-1} \pi_0(\tau^k(\omega)) \to \int_\Omega \pi_0(\omega) dP(\omega),
\]
i.e.,
\[
\frac{1}{n} \sum_{k=0}^{n-1} \pi_k(\omega) \to \int_{\Omega_0} \omega_0 d\mu_0(\omega_0),
\]
proving the claim.
\end{proof}


We will use Lemma \ref{pi0} to prove the strong law of large numbers. First
we prove two lemmas about \textbf{joint distributions}.\footnote{Elias M. Stein and Rami Shakarchi,
{\em Functional Analysis}, Princeton Lectures in Analysis, volume IV, p.~208, Lemma 2.2, Lemma 2.3.}

\begin{lemma}
If $X_1,\ldots,X_n:\Omega \to \mathbb{R}$ and $Y_1,\ldots,Y_n:\Omega' \to \mathbb{R}$ are random variables
with the same joint distribution and for each $1 \leq k \leq n$,
 $\Phi_k:\mathbb{R}^n \to \mathbb{R}$ is a continuous function,
then
for $U_k=\Phi_k(X_1,\ldots,X_n)$ and
$V_k=\Phi_k(Y_1,\ldots,Y_n)$,
the random variables
$U_1,\ldots,U_n$ and $V_1,\ldots,V_n$ 
have the same joint distribution. 
\label{joint}
\end{lemma}
\begin{proof}
Write 
$X=(X_1,\ldots,X_n)$, $Y=(Y_1,\ldots,Y_n)$,
$U=(U_1,\ldots,U_n)$, and $V=(V_1,\ldots,V_n)$, which are Borel measurable.
Let $\Phi=(\Phi_1,\ldots,\Phi_n)$, which is continuous $\mathbb{R}^n \to \mathbb{R}^n$,
and for which
\[
U = \Phi(X), \qquad V = \Phi(Y).
\]

To show that $U_1,\ldots,U_n$ and $V_1,\ldots,V_n$ have the same joint distribution
means to show that $U_*P=V_*P'$. Let
$A \in \mathscr{B}_{\mathbb{R}^n}$, for which
$\Phi^{-1}(A) \in \mathscr{B}_{\mathbb{R}^n}$ and
\begin{align*}
(U_*P)(A)&=P(U^{-1}(A))\\
&=P(X^{-1}(\Phi^{-1}(A)))\\
&=P'(Y^{-1}(\Phi^{-1}(A)))\\
&=P'(V^{-1}(A))\\
&=(V_* P')(A),
\end{align*}
where, because $X_1,\ldots,X_n$ and $Y_1,\ldots,Y_n$ have the same joint distribution,
\[
P(X^{-1}(\Phi^{-1}(A)))=(X_*P)(\Phi^{-1}(A))
=(Y_*P')(\Phi^{-1}(A))
=P'(Y^{-1}(\Phi^{-1}(A))).
\]
\end{proof}


\begin{lemma}
If sequences of random variables $X_n:(\Omega,\mathscr{F},P) \to \mathbb{R}$ and $Y_n:(\Omega',\mathscr{F}',P') \to \mathbb{R}$, $n \geq 1$, have the same \textbf{finite-dimensional distributions} and there is some
$a \in \mathbb{R}$ such that
$X_n$ converges to $a$ almost surely, then $Y_n$ converges to $a$ almost surely.
\label{finitedimensional}
\end{lemma}
\begin{proof}
That the sequences $X_n$ and $Y_n$ have the same finite-dimensional distributions means that
for each $n \geq 1$,
\[
(X_1,\ldots,X_n)_*P = (Y_1,\ldots,Y_n)_*P',
\]
i.e., for each $n \geq 1$ and for each $A \in \mathscr{B}_{\mathbb{R}^n}$,
\[
P((X_1,\ldots,X_n) \in A) = P'((Y_1,\ldots,Y_n) \in A).
\]

Define
\begin{align*}
E_{k,N,n} &= \left\{ \omega \in \Omega: |X_n(\omega)-a| \leq \frac{1}{k} \right\}\\
F_{k,N,n} &= \left\{ \omega \in \Omega': |Y_n(\omega)-a| \leq \frac{1}{k} \right\}.
\end{align*}
Then define
\begin{align*}
E&=\bigcap_{k=1}^\infty \bigcup_{N=1}^\infty \bigcap_{n=N}^\infty E_{k,N,n}=
\bigcap_{k=1}^\infty \bigcup_{N=1}^\infty E_{k,N}
=\bigcap_{k=1}^\infty E_k
\\
F&=\bigcap_{k=1}^\infty \bigcup_{N=1}^\infty \bigcap_{n=N}^\infty F_{k,N,n}
=
\bigcap_{k=1}^\infty \bigcup_{N=1}^\infty F_{k,N}
=\bigcap_{k=1}^\infty F_k,
\end{align*}
and 
\begin{align*}
G_{k,N,n}&=\bigcap_{m=N}^n E_{k,N,m}\\
H_{k,N,n}&=\bigcap_{m=N}^n F_{k,N,m}.
\end{align*}
Because $(X_1,\ldots,X_n)$ and $(Y_1,\ldots,Y_n)$ have the same distribution,
\[
P(G_{k,N,n}) = P'(H_{k,N,n}).
\]
But
\[
E_{k,N} = \bigcap_{n=N}^\infty E_{k,N,n}
=\bigcap_{n=N}^\infty G_{k,N,n}
=\lim_{n \to \infty} G_{k,N,n} 
\]
and
\[
F_{k,N} = \lim_{n \to \infty} H_{k,N,n},
\]
so 
\[
P(E_{k,N}) = P'(F_{k,N}).
\]
Then 
\[
P(E_k) = \lim_{N \to \infty} P(E_{k,N}) =  \lim_{N \to \infty} P'(F_{k,N})=P'(F_k).
\]
Then
\[
P(E) = \lim_{k \to \infty} P(E_k) = \lim_{k \to \infty} P'(F_k) = P'(F).
\]
That the sequence $X_n$ converges almost surely means that $P(E)=1$, and therefore
$P'(F)=1$, i.e. the sequence $Y_n$  converges almost surely.
\end{proof}


We now use what we have established to prove the \textbf{strong law of large numbers}.\footnote{Elias M. Stein and Rami Shakarchi,
{\em Functional Analysis}, Princeton Lectures in Analysis, volume IV, p.~206, Theorem 2.1.}
(We write $(\Omega',\mathscr{F}',P')$ for a probability space because  $(\Omega,\mathscr{F},P)$
denotes the product probability space constructed already.)

\begin{theorem}
If $X_n:(\Omega',\mathscr{F}',P') \to \mathbb{R}$, $n \geq 1$, are independent identically distributed $L^1$ random variables, then
for almost all $\omega \in \Omega'$,
\[
\frac{1}{n} \sum_{k=1}^n X_k(\omega) \to E(X_1).
\]
\end{theorem}
\begin{proof}
For $n \geq 0$, let $Y_n = X_{n+1}$; we do this to 
make the index set the same as for $\Omega_n$. 
Let $\mu_n={Y_n}_*P'$ and set $\mu=\mu_0$. Because the $Y_n$ are  identically distributed,
$\mu_n=\mu_0$ for each $n$.

Because $Y_0$ is $L^1$, $\mu$ has finite first moment. 
As
$\mu_0={Y_0}_*P'$, applying
the change of variables theorem \eqref{COV},
\[
\int_{\mathbb{R}} t d\mu(t) = \int_{\mathbb{R}} t d({Y_0}_*P')(t)
=\int_{\Omega'} Y_0(\omega) dP'(\omega) = E(Y_0).
\]
With this, Lemma \ref{pi0} says that for almost all $\omega \in \Omega$,
\begin{equation}
\frac{1}{n} \sum_{k=0}^{n-1} \pi_k(\omega) \to E(Y_0).
\label{EY0}
\end{equation}


For each $n$, 
\[
(\pi_0,\ldots,\pi_n)_*P = \mu_0 \times \cdots \times \mu_n,
\]
and because the $Y_n$ are independent,
\[
(Y_0,\ldots,Y_n)_* P' = {Y_0}_*P' \times \cdots \times {Y_n}_*P' = 
\mu_0 \times \cdots \times \mu_n,
\]
so the sequences $\pi_n$ and $Y_n$ have the same finite-dimensional distributions. 
For $n \geq 1$, define $\Phi_n:\mathbb{R}^n \to \mathbb{R}$ by
\[
\Phi_n(t_1,\ldots,t_n) = \frac{1}{n} \sum_{k=1}^n t_k.
\]
Lemma \ref{joint} then tells us that the sequences 
\[
U_n=\Phi_{n}(Y_0,\ldots,Y_{n-1}) = \frac{1}{n} \sum_{k=0}^{n-1} Y_k
\]
 and
 \[
 V_n=\Phi_{n}(\pi_0,\ldots,\pi_{n-1}) = \frac{1}{n} \sum_{k=0}^{n-1} \pi_k,
 \] 
$n \geq 1$, have the same finite-dimensional distributions. 

Now, \eqref{EY0} says that $V_n$ converges to $E(Y_0)$ almost surely, and thus applying
Lemma \ref{finitedimensional} we get that $U_n$ converges to $E(Y_0)$ almost surely.
That is, 
\[
\frac{1}{n} \sum_{k=0}^{n-1} Y_k \to E(Y_0)
\]
almost surely,
and because $Y_k=X_{k+1}$,
\[
\frac{1}{n} \sum_{k=1}^n X_k \to E(X_1)
\]
almost surely,
completing the proof.
\end{proof}


\end{document}