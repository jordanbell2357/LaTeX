\documentclass{article}
\usepackage{amsmath,amssymb,graphicx,subfig,mathrsfs,amsthm,siunitx}
%\usepackage{tikz-cd}
%\usepackage{hyperref}
\newcommand{\inner}[2]{\left\langle #1, #2 \right\rangle}
\newcommand{\tr}{\ensuremath\mathrm{tr}\,} 
\newcommand{\Span}{\ensuremath\mathrm{span}} 
\def\Re{\ensuremath{\mathrm{Re}}\,}
\def\Im{\ensuremath{\mathrm{Im}}\,}
\newcommand{\id}{\ensuremath\mathrm{id}} 
\newcommand{\rank}{\ensuremath\mathrm{rank\,}} 
\newcommand{\co}{\ensuremath\mathrm{co}\,} 
\newcommand{\cco}{\ensuremath\overline{\mathrm{co}}\,}
\newcommand{\supp}{\ensuremath\mathrm{supp}\,}
\newcommand{\epi}{\ensuremath\mathrm{epi}\,}
\newcommand{\lsc}{\ensuremath\mathrm{lsc}\,}
\newcommand{\ext}{\ensuremath\mathrm{ext}\,}
\newcommand{\cl}{\ensuremath\mathrm{cl}\,}
\newcommand{\dom}{\ensuremath\mathrm{dom}\,}
\newcommand{\LSC}{\ensuremath\mathrm{LSC}}
\newcommand{\USC}{\ensuremath\mathrm{USC}}
\newcommand{\extreals}{\overline{\mathbb{R}}}
\newcommand{\upto}{\nearrow}
\newcommand{\downto}{\searrow}
\newcommand{\norm}[1]{\left\Vert #1 \right\Vert}
\newtheorem{theorem}{Theorem}
\newtheorem{lemma}[theorem]{Lemma}
\newtheorem{proposition}[theorem]{Proposition}
\newtheorem{corollary}[theorem]{Corollary}
\theoremstyle{definition}
\newtheorem{definition}[theorem]{Definition}
\newtheorem{example}[theorem]{Example}
\begin{document}
\title{Gaussian measures and Bochner's theorem}
\author{Jordan Bell\\ \texttt{jordan.bell@gmail.com}\\Department of Mathematics, University of Toronto}
\date{\today}

\maketitle

\section{Fourier transforms of measures}
Let $m_n$ be {\em normalized Lebesgue measure} on $\mathbb{R}^n$: $dm_n(x) = (2\pi)^{-n/2} dx$. 
If $\mu$ is a finite  positive Borel  measure on $\mathbb{R}^n$, the {\em Fourier transform of $\mu$} is the function
$\hat{\mu}:\mathbb{R}^n \to \mathbb{C}$ defined by
\[
\hat{\mu}(\xi) = \int_{\mathbb{R}^n} e^{-i\xi \cdot x} d\mu(x), \qquad \xi \in \mathbb{R}^n.
\]
One proves using the dominated convergence theorem that $\hat{\mu}$ is continuous.
If $f \in L^1(\mathbb{R}^n)$, the Fourier transform of $f$ is the function $\hat{f}:\mathbb{R}^n \to \mathbb{C}$ defined by
\[
\hat{f}(\xi) = \int_{\mathbb{R}^n} e^{-i\xi\cdot x} f(x) dm_n(x), \qquad \xi \in \mathbb{R}^n.
\]
Likewise, using the dominated convergence theorem, $\hat{f}$ is continuous.
One proves that if $f \in L^1(\mathbb{R}^n)$ and $\hat{f} \in L^1(\mathbb{R}^n)$ then, for almost all $x \in \mathbb{R}^n$,
\[
f(x) = \int_{\mathbb{R}^n} e^{ix\cdot \xi} \hat{f}(\xi) dm_n(\xi).
\]

As
\[
\hat{\mu}(0)=\int_{\mathbb{R}^n} d\mu(x) = \mu(\mathbb{R}^n),
\]
$\mu$ is a probability measure if and only if $\hat{\mu}(0)=1$. (By a probability measure we mean a positive measure with mass $1$.)

If $\phi \in L^1(\mathbb{R}^n)$ and $\hat{\phi} \in L^1(\mathbb{R}^n)$, then, inverting the Fourier transform,
\begin{eqnarray*}
\inner{\phi}{\mu}&=&\int_{\mathbb{R}^n} \phi(x) d\mu(x)\\
&=&\int_{\mathbb{R}^n} \left(\int_{\mathbb{R}^n} \hat{\phi}(\xi) e^{ix\cdot \xi} dm_n(\xi) \right) d\mu(x)\\
&=& \int_{\mathbb{R}^n} \hat{\phi}(\xi) \int_{\mathbb{R}^n} e^{i\xi \cdot x} d\mu(x) dm_n(\xi)\\
&=&\int_{\mathbb{R}^n} \hat{\phi}(\xi) \hat{\mu}(-\xi) dm_n(\xi)\\
&=&\int_{\mathbb{R}^n} \hat{\phi}(-\xi) \hat{\mu}(\xi) dm_n(\xi).
\end{eqnarray*}

\begin{theorem}
If $\mu$ and $\nu$ are finite Borel  measures on $\mathbb{R}^n$ and $\hat{\mu}=\hat{\nu}$, then $\mu=\nu$.
\end{theorem}
\begin{proof}
To prove that $\mu=\nu$ it suffices to prove that for any ball $B$ in $\mathbb{R}^n$ we have $\mu(B)=\nu(B)$. 
Let $\phi_n \in C_c^\infty(\mathbb{R}^n) \to \chi_B$ pointwise.  On the one hand, by the dominated convergence theorem,
$\inner{\phi_n}{\mu} \to \mu(B)$ and  $\inner{\phi_n}{\nu} \to \nu(B)$ as $n \to \infty$.
On the other hand, because $\hat{\mu}=\hat{\nu}$ we have
\[
\inner{\phi_n}{\mu}=\int_{\mathbb{R}^n} \hat{\phi}_n(-\xi) \hat{\mu}(\xi) dm_n(\xi)
=\int_{\mathbb{R}^n} \hat{\phi}_n(-\xi) \hat{\nu}(\xi) dm_n(\xi)
=\inner{\phi_n}{\nu}.
\]
Therefore $\mu(B)=\nu(B)$, and it follows that $\mu=\nu$.
\end{proof}


\section{Gaussian measures}
Let $\lambda_1,\ldots,\lambda_n>0$, and let $\Lambda:\mathbb{R}^n \to \mathbb{R}^n$ be the linear map
defined by $\Lambda e_i=\lambda_i e_i$. Define 
\[
d\mu(x) = \sqrt{\det \Lambda} \exp\left(-\frac{1}{2} x\cdot \Lambda x \right) dm_n(x),
\]
called a {\em Gaussian measure}. 


\begin{theorem}
\[
\hat{\mu}(\xi)=\exp\left(-\frac{1}{2} \xi \cdot \Lambda^{-1}\xi\right), \qquad \xi \in \mathbb{R}^n.
\]
\end{theorem}
\begin{proof}
We have
\begin{eqnarray*}
\hat{\mu}(\xi)&=&\int_{\mathbb{R}^n} e^{-i\xi \cdot x} \sqrt{\det \Lambda} \exp\left(-\frac{1}{2} x\cdot \Lambda x\right) dm_n(x)\\
&=&\int_{\mathbb{R}^d} e^{-i\xi_1 x_1-\cdots-i\xi_n x_n} \sqrt{\lambda_1 \cdots \lambda_n} \exp\left(-\frac{1}{2}\lambda_1 x_1^2 - \cdots - \frac{1}{2}\lambda_n
x_n^2 \right) dm_n(x)\\
&=&\prod_{j=1}^n I_j,
\end{eqnarray*}
where
\[
I_j=\int_\mathbb{R} e^{-i\xi_j x_j} \sqrt{\lambda_j} \exp\left(-\frac{1}{2}\lambda_j x_j^2\right) dm_1(x_j).
\]
Using
\[
-i\xi_j x_j-\frac{1}{2}\lambda_jx_j^2 = -\frac{\lambda_j}{2} \left( \left(x_j+\frac{i\xi_j}{\lambda_j}\right)^2 + \frac{\xi_j^2}{\lambda_j^2} \right)
=-\frac{\lambda_j}{2}  \left(x_j+\frac{i\xi_j}{\lambda_j}\right)^2 - \frac{\xi_j^2}{2\lambda_j},
\]
we get, doing contour integration,
\begin{eqnarray*}
I_j&=&\int_\mathbb{R} \sqrt{\lambda_j} \exp\left(-\frac{\lambda_j}{2} \left(x_j+\frac{i\xi_j}{\lambda_j}\right)^2 \right)
\exp \left( -\frac{\xi_j^2}{2\lambda_j} \right) dm_1(x_j)\\
&=&\int_\mathbb{R} \sqrt{\lambda_j} \exp\left(-\frac{\lambda_j x_j^2}{2} \right)\exp \left( -\frac{\xi_j^2}{2\lambda_j} \right) dm_1(x_j)\\
&=&\int_\mathbb{R} \sqrt{\lambda_j} \exp(-y_j^2) \exp \left( -\frac{\xi_j^2}{2\lambda_j} \right) \sqrt{\frac{2}{\lambda_j}} dm_1(y_j)\\
&=& \exp \left( -\frac{\xi_j^2}{2\lambda_j} \right) \int_\mathbb{R}  \sqrt{2} \exp(-y_j^2) dm_1(y_j)\\
&=& \exp \left( -\frac{\xi_j^2}{2\lambda_j} \right)  \int_{\mathbb{R}} \frac{1}{\sqrt{\pi}} \exp(-y_j^2) dy_j\\
&=& \exp \left( -\frac{\xi_j^2}{2\lambda_j} \right).
\end{eqnarray*}
Therefore, as $\Lambda^{-1}\xi =\sum_{j=1}^n \frac{\xi_j}{\lambda_j}e_j$ and $\xi \cdot \Lambda^{-1}\xi = \sum_{j=1}^n \frac{\xi_j^2}{\lambda_j}$,
\begin{eqnarray*}
\hat{\mu}(\xi)&=&\prod_{j=1}^n  \exp \left( -\frac{\xi_j^2}{2\lambda_j} \right)\\
&=&\exp\left(-\frac{1}{2} \sum_{j=1}^n \frac{\xi_j^2}{\lambda_j}\right)\\
&=&\exp\left(-\frac{1}{2} \xi \cdot \Lambda^{-1}\xi\right).
\end{eqnarray*}
\end{proof}

From the above theorem we get 
\[
\hat{\mu}(0)=1,
\]
and hence a Gaussian measure is a probability measure.

For $h \in \mathbb{R}^n$, define $T_h:\mathbb{R}^n \to \mathbb{R}^n$ by $T_h(x)=x-h$. 
If $E$ is a Borel subset of $\mathbb{R}^n$, because $\chi_{T_{-h}(E)}=\chi_E \circ T_h$,
\[
((T_h)_* \mu) (E)=\mu(T_h^{-1}(E))=\mu(T_{-h}(E))=\int_{\mathbb{R}^n} \chi_{T_{-h}(E)} d\mu=
\int_{\mathbb{R}^n} \chi_E \circ T_h d\mu.
\]
Then, because $T_h \circ T_{-h} = \id_{\mathbb{R}^n}$, 
\begin{eqnarray*}
\int_{\mathbb{R}^n} \chi_E \circ T_h d\mu&=&\int_{\mathbb{R}^n}\chi_E \circ T_h (x)  \sqrt{\det \Lambda} \exp\left(-\frac{1}{2} x\cdot \Lambda x \right) dm_n(x)\\
&=&\int_{\mathbb{R}^n}\chi_E(x)  \sqrt{\det \Lambda} \exp\left(-\frac{1}{2} (T_{-h}x)\cdot (\Lambda T_{-h}x) \right) d((T_{-h})_*m_n)(x)\\
&=&\int_{\mathbb{R}^n}\chi_E(x)  \sqrt{\det \Lambda} \exp\left(-\frac{1}{2} (T_{-h}x)\cdot (\Lambda T_{-h}x) \right) dm_n(x).
\end{eqnarray*}
As $\Lambda$ is self-adjoint $\Lambda x\cdot  h =  x \cdot \Lambda h$,
\begin{eqnarray*}
(T_{-h}x)\cdot (\Lambda T_{-h}x)&=&(x+h)\cdot (\Lambda(x+h))\\
&=&(x+h)\cdot (\Lambda x + \Lambda h)\\
&=&x\cdot \Lambda x + x\cdot \Lambda h + h \cdot \Lambda x + h \cdot \Lambda h\\
&=&x\cdot \Lambda x + 2x\cdot \Lambda h + h \cdot \Lambda h.
\end{eqnarray*}
Therefore,
\begin{eqnarray*}
((T_h)_* \mu) (E) &=& \int_{\mathbb{R}^n} \chi_E(x) \exp\left(-\frac{1}{2}\left(2x\cdot \Lambda h + h \cdot \Lambda h\right)\right) d\mu(x)\\
&=&\int_{\mathbb{R}^n} \chi_E(x) \exp\left(-x\cdot \Lambda h -\frac{1}{2}h \cdot \Lambda h\right) d\mu(x).
\end{eqnarray*}
This shows that the Radon-Nikodym derivative of $(T_h)_*\mu$ with respect to $\mu$ is 
\[
\frac{d(T_h)_*\mu}{d\mu}(x) = \exp\left(-x\cdot \Lambda h -\frac{1}{2}h \cdot \Lambda h\right).
\]


\section{Positive-definite functions}
We say that a function $\phi:\mathbb{R}^n \to \mathbb{C}$ is {\em positive-definite}
if $x_1,\ldots,x_r \in \mathbb{R}^n$ and $c_1,\ldots,c_r \in
\mathbb{C}$ imply
that
\[
\sum_{i,j=1}^r c_i \overline{c_j} \phi(x_i-x_j) \geq 0;
\]
in particular, the left-hand side is real.

Using $r=1$, $c_1=1$, we have for any $x_1 \in \mathbb{R}^n$ that $\phi(x_1-x_1) \geq 0$, i.e. $\phi(0) \geq 0$. For $x \in \mathbb{R}^n$,
using $r=2$, $x_1=x, x_2=0$ and choosing fitting  $c_1,c_2 \in \mathbb{C}$ gives
\[
\phi(-x)=\overline{\phi(x)},
\]
and using this with $c_2=1$ and for appropriate $c_1$ gives
\[
|\phi(x)| \leq \phi(0).
\]

For $f,g \in L^1(\mathbb{R}^n)$, the {\em convolution of $f$ and $g$} is the function $f*g:\mathbb{R}^n \to \mathbb{C}$ defined by
\[
(f*g)(x) = \int_{\mathbb{R}^n} f(y)g(x-y) dm_n(y), \qquad x \in \mathbb{R}^n,
\]
and $\norm{f*g}_{L^1} \leq \norm{f}_{L^1} \norm{g}_{L^1}$, a case of {\em Young's inequality}.
For $f:\mathbb{R}^n \to \mathbb{C}$, we denote by $\supp f$ the {\em essential support} of $f$; if $f$ is continuous, then $\supp f$ is the closure of the set
$\{x \in \mathbb{R}^n: f(x) \neq 0\}$.
A fact that we will use later is\footnote{Gerald B. Folland, {\em Real
Analysis: Modern Techniques and their Applications}, second ed., p.~240, Proposition 8.6.}
\[
\supp(f*g) \subseteq \overline{\supp f + \supp g}.
\]
We denote by $f^*$ the function defined by $f^*(x)=\overline{f(-x)}$.

$C_c(\mathbb{R}^n)$ is the set of all $f \in C(\mathbb{R}^n)$ for which $\supp f$ is a compact set.
The set $C_c(\mathbb{R}^n)$ is dense in the Banach space
$C_0(\mathbb{R}^n)$  and also in the Banach space $L^1(\mathbb{R}^n)$; $C_c(\mathbb{R}^n)$ is not a Banach space or even a Fr\'echet space,
and thus does not have a robust structure itself, but is used because it is easier to prove things for it which one then extends in some way to spaces
in which the set is dense.
The proof of the following theorem
follows Folland.\footnote{Gerald B. Folland, {\em A Course in Abstract Harmonic Analysis}, p.~85, Proposition 3.35.}

\begin{theorem}
If $\phi:\mathbb{R}^n \to \mathbb{C}$ is positive-definite and continuous and $f \in C_c(\mathbb{R}^n)$, then
\[
\int (f^* * f) \phi \geq 0.
\]
\end{theorem}
\begin{proof}
Write $K=\supp f$, and
define $F:\mathbb{R}^n \times \mathbb{R}^n \to \mathbb{C}$ by 
\[
F(x,y)=f(x)\overline{f(y)} \phi(x-y).
\]
$F$ is continuous, and $\supp F \subseteq K \times K$, hence $\supp F$ is compact. Thus
$F \in C_c(\mathbb{R}^n \times \mathbb{R}^n)$; in particular $F$ is uniformly continuous on
$K \times K$, and it follows that  for each $\epsilon>0$ there is some $\delta>0$ such that 
if $|x-a|<\delta$ and $|y-b|<\delta$ then 
$|F(x,y)-F(a,b)|<\epsilon$. 
The collection $\{B_\delta(x) :x \in K\}$ covers
$K$ and hence 
there are finitely many distinct $x_i \in K$
such that
the collection $\{B_\delta(x_i): i\}$ covers $K$.
Then 
$\{B_\delta(x_i) \times B_\delta(x_j):i,j\}$ covers $K \times K$. 
Let $E_i$ be pairwise disjoint, measurable, and satisfy $x_i \in E_i  \subseteq B_\delta(x_i)$.
The collection $\{E_i:i,\}$
covers $K$, so the collection  
$\{E_i \times E_j:i,j\}$
covers $K \times K$. 

Define
\[
R =\sum_{i,j} \int_{E_i \times E_j}  (F(x,y)-F(x_i,x_j)) dm_n(x)dm_n(y).
\]
$R$ satisfies
\begin{eqnarray*}
|R|&\leq&\sum_{i,j} \int_{E_i \times E_j} | F(x,y)- F(x_i,x_j)| dm_n(x)dm_n(y)\\
&\leq&\sum_{i,j} \int_{E_i \times E_j} \epsilon dm_n(x)dm_n(y)\\
&=&\epsilon \sum_{i,j}m_n(E_i)m_n(E_j)\\
&=&\epsilon m_n(K)^2.
\end{eqnarray*}
We obtain
\begin{eqnarray*}
\int_{K \times K} F(x,y) dm_n(x) dm_n(y)&=&\sum_{i,j} \int_{E_i \times E_j} F(x,y) dm_n(x)dm_n(y)\\
&=&\sum_{i,j} F(x_i,x_j) m_n(E_i)m_n(E_j) +R\\
&=&\sum_{i,j} f(x_i)\overline{f(x_j)}\phi(x_i-x_j)m_n(E_i)m_n(E_j)+R.
\end{eqnarray*}
Using $c_i=f(x_i)m_n(E_i)$, the fact that $\phi$ is positive-definite means that the sum is $\geq 0$. Therefore
\[
\int_{K \times K} F(x,y) dm_n(x) dm_n(y) \geq -|R| \geq -\epsilon m_n(K)^2.
\]
This is true for all $\epsilon>0$, hence 
\[
\int_{\mathbb{R}^n} \int_{\mathbb{R}^n} f(x)\overline{f(y)} \phi(x-y) dm_n(x)dm_n(y) = \int_{K \times K}
F(x,y) dm_n(x)dm_n(y) \geq 0.
\]
But
\begin{eqnarray*}
\int_{\mathbb{R}^n} (f^* * f)(x) \phi(x) dm_n(x)&=&\int_{\mathbb{R}^n} \left(\int_{\mathbb{R}^n} f^*(y)f(x-y) dm_n(y)\right) \phi(x) dm_n(x)\\
&=&\int_{\mathbb{R}^n} \int_{\mathbb{R}^n} \overline{f(-y)} f(x-y) \phi(x) dm_n(x) dm_n(y)\\
&=&\int_{\mathbb{R}^n} \int_{\mathbb{R}^n}  \overline{f(-y)} f(x) \phi(x+y) dm_n(x)dm_n(y)\\
&=&\int_{\mathbb{R}^n} \int_{\mathbb{R}^n}  \overline{f(y)} f(x) \phi(x-y) dm_n(x)dm_n(y).
\end{eqnarray*}
\end{proof}


\begin{corollary}
If $\phi:\mathbb{R}^n \to \mathbb{C}$ is positive-definite and continuous and $f \in L^1(\mathbb{R}^n)$, then
\[
\int (f^* * f) \phi \geq 0.
\]
\label{positivecor}
\end{corollary}
\begin{proof}
Let $f_n \in C_c(\mathbb{R}^n)$ converge to $f$ in $L^1(\mathbb{R}^n)$ as $n \to \infty$; that there is such a sequence is
given to us by the fact that $C_c(\mathbb{R}^n)$ is a dense subset of $L^1(\mathbb{R}^n)$. 
Using
\begin{eqnarray*}
f_n^* * f_n- f^* * f &=& f_n^* * f_n - f_n^** f + f_n^* *f - f^* * f\\
&=&f_n^**(f_n-f)+(f_n^*-f^*)*f\\
&=&f_n^**(f_n-f)+(f_n-f)^**f,
\end{eqnarray*}
and $\norm{g^*}_{L^1}=\norm{g}_{L^1}$,
we get
\begin{eqnarray*}
\norm{f_n^* * f_n - f^* * f}_{L^1}&\leq&\norm{f_n^**(f_n-f)}_{L^1}+\norm{(f_n-f)^**f}_{L^1}\\
&\leq&\norm{f_n^*}_{L^1} \norm{f_n-f}_{L^1} + \norm{(f_n-f)^*}_{L^1} \norm{f}_{L^1}\\
&=&\norm{f_n}_{L^1} \norm{f_n-f}_{L^1} + \norm{f_n-f}_{L^1} \norm{f}_{L^1},
\end{eqnarray*}
which converges to $0$ because $\norm{f_n-f}_{L^1} \to 0$. Therefore, because $\phi$ is bounded,
\[
\int_{\mathbb{R}^n} (f_n^* * f_n) \phi dm_n \to \int_{\mathbb{R}^n} (f^* * f)\phi dm_n.
\]
As $\int_{\mathbb{R}^n} (f_n^* * f_n)\phi dm_n \geq 0$ for each $n$, this implies that $\int_{\mathbb{R}^n} (f^* * f)\phi dm_n \geq 0$.
\end{proof}



It is straightforward to prove that the Fourier transform of a finite positive Borel measure is a positive-definite function; one ends up with the expression
\[
\int_{\mathbb{R}^n} \left| \sum_{j=1}^n c_j e^{i\xi_j\cdot x} \right|^2 d\mu(x),
\]
which is finite and nonnegative because $\mu$ is finite and positive respectively. We have established already that the
Fourier transform of a finite positive Borel measure $\mu$ on $\mathbb{R}^n$ is continuous and satisfies $\hat{\mu}(0)=1$. {\em Bochner's theorem}
is the statement that a function with these three properties is indeed the Fourier transform of a finite positive Borel measure.
Our proof of the following theorem follows Folland.\footnote{Gerald B. Folland, {\em A Course in Abstract Harmonic Analysis}, p.~95, Theorem 4.18.}

\begin{theorem}[Bochner]
If $\phi:\mathbb{R}^n \to \mathbb{C}$ is positive-definite, continuous, and satisfies $\phi(0)=1$, then there is some
Borel probability measure $\mu$ on $\mathbb{R}^n$ such that $\phi=\hat{\mu}$.
\end{theorem}
\begin{proof}
Let $\{\psi_U\}$ be an approximate identity. That is, for each  neighborhood $U$ of $0$, $\psi_U$ is a function such that
$\supp \psi_U$ is compact and contained in $U$,
$\psi \geq
0$,
$\psi_U(-x)=\psi_U(x)$, and  $\int_{\mathbb{R}^n} \psi_U dm_n =1$. For every $f \in L^1(\mathbb{R}^n)$, an approximate identity satisfies $\norm{f*\psi_U -f}_{L^1}\to 0$ as
$U \to \{0\}$.\footnote{Gerald B. Folland, {\em A Course in Abstract Harmonic Analysis}, p.~53, Proposition 2.42.}

We have $\psi_U^*=\psi_{-U}$, so
\[
\supp (\psi_U^* * \psi_U) \subseteq \overline{\supp \psi_{-U} + \supp \psi_U} = \supp \psi_{-U} + \supp \psi_U \subseteq -U +U,
\]
and as always, $\int_{\mathbb{R}^n} f*g dm_n
=\int_{\mathbb{R}^n} f dm_n \int_{\mathbb{R}^n} gdm_n$.
Therefore $\{\psi_U^* * \psi_U\}$ is an approximate identity:

For $f,g \in L^1(\mathbb{R}^n)$, define
\[
\inner{f}{g}_\phi = \int_{\mathbb{R}^n} (g^* * f) \phi dm_n.
\]
One checks that this is a positive Hermitian form; {\em positive} means that $\inner{f}{f}_\phi \geq 0$ for all $f \in L^1(\mathbb{R}^n)$, and this
is given to us by Corollary \ref{positivecor}.
Using the Cauchy-Schwarz inequality,\footnote{Jean Dieudonne, {\em Foundations of Modern Analysis}, 1969, p.~117, Theorem 6.2.1.}
\[
|\inner{f}{g}_\phi|^2 \leq \inner{f}{f}_\phi \inner{g}{g}_\phi.
\]

We have  laid out the tools that we will use. Let $f \in L^1(\mathbb{R}^n)$. $\psi_U * f  \to f$ in $L^1$ as $U \to \{0\}$, and as $\phi$ is bounded
this gives $\int_{\mathbb{R}^n} (\psi_U^* * f) \phi dm_n \to \int_{\mathbb{R}^n} f\phi dm_n$ as $U \to \{0\}$. 
Because $\{\psi_U^* * \psi_U\}$ is an approximate identity, $\int_{\mathbb{R}^n} (\psi_U^* * \psi_U) \phi dm_n \to \phi(0)$ as $U \to \{0\}$. That is,
we have $\inner{f}{\psi_U}_\phi \to \int_{\mathbb{R}^n} f\phi dm_n$ and $\inner{\psi_U}{\psi_U}_\phi \to \phi(0$ as $U \to \{0\}$, and as $\phi(0)=1$,
the above statemtn of the Cauchy-Schwarz inequality produces
\begin{equation}
\left| \int_{\mathbb{R}^n} f\phi dm_n \right|^2 \leq \int_{\mathbb{R}^n} (f^* * f)\phi dm_n.
\label{sweetestimate}
\end{equation}

With $h=f^* * f$, the inequality \eqref{sweetestimate} reads
\[
\left| \int_{\mathbb{R}^n} f\phi dm_n \right|^2 \leq \int_{\mathbb{R}^n} h\phi dm_n.
\]
Defining $h^{(1)}=h$, $h^{(2)}=h*h$, $h^{(3)}=h*h*h$, etc.,  applying \eqref{sweetestimate} to $h$ gives, because $h^*=h$,
\[
\left| \int_{\mathbb{R}^n} h \phi dm_n \right|^2 \leq \int_{\mathbb{R}^n} h^{(2)} \phi dm_n.
\]
Then applying \eqref{sweetestimate} to $h^{(2)}$, which satisfies $(h^{(2)})^*=h^{(2)}$,
\[
\left| \int_{\mathbb{R}^n} h^{(2)} \phi dm_n \right|^2 \leq \int_{\mathbb{R}^n} h^{(4)} \phi dm_n.
\]
Thus, for any $m \geq 0$ we have 
\begin{eqnarray*}
\left| \int_{\mathbb{R}^n} f \phi dm_n \right| &\leq& \left| \int_{\mathbb{R}^n} h^{\left(2^m\right)} \phi dm_n \right|^{2^{-(m+1)}}\\
&\leq&\norm{h^{\left(2^m\right)}}_{L^1}^{2^{-(m+1)}}\\
&=&\left(\norm{h^{\left(2^m\right)}}_{L^1}^{2^{-m}}\right)^{1/2},
\end{eqnarray*}
since $\norm{\phi}_\infty = \phi(0) =1$. 

With convolution as multiplication, $L^1(\mathbb{R}^n)$ is a commutative Banach algebra, and the Gelfand
transform is an algebra homomorphism $L^1(\mathbb{R}^n) \to C_0(\mathbb{R}^n)$ that satisfies\footnote{Gerald B. Folland,
{\em A Course in Abstract Harmonic Analysis}, p.~15, Theorem 1.30. Namely, this is the {\em Gelfand-Naimark theorem}.}
\[
\norm{\hat{g}}_\infty = \lim_{k \to \infty} \norm{g^{(k)}}_{L^1}^{1/k}, \qquad g \in L^1(\mathbb{R}^n);
\]
for $L^1(\mathbb{R}^n)$, the Gelfand transform is the Fourier transform. Write the Fourier transform as 
$\mathscr{F}:L^1(\mathbb{R}^n) \to C_0(\mathbb{R}^n)$. Stating that the Gelfand transform is a homomorphism
means that $\mathscr{F}(g_1*g_2)=\mathscr{F}(g_1) \mathscr{F}(g_2)$, because multiplication in the Banach algebra $C_0(\mathbb{R}^n)$
is pointwise multiplication.
Then, since a subsequence of a convergent sequence converges to the same limit,
\[
\lim_{m \to \infty} \left(\norm{h^{\left(2^m\right)}}_{L^1}^{2^{-m}}\right)^{1/2} = \left( \norm{\hat{h}}_\infty \right)^{1/2}.
\]
But
\[
\hat{h}=\mathscr{F}(f^* * f) = \mathscr{F}(f^*) \mathscr{F}(f) = \overline{\mathscr{F}(f)} \mathscr{F}(f)= \left| \mathscr{F}(f) \right|^2,
\]
so
\[
\left( \norm{\hat{h}}_\infty \right)^{1/2} = \left( \norm{|\hat{f}|^2}_\infty \right)^{1/2} = \norm{\hat{f}}_\infty.
\]
Putting things together, we have that for any $f \in L^1(\mathbb{R}^n)$,
\[
\left| \int_{\mathbb{R}^n} f \phi dm_n \right| \leq \norm{\hat{f}}_\infty.
\]
Therefore $\hat{f} \mapsto \int_{\mathbb{R}^n} f \phi dm_n$ is a bounded linear functional $\mathscr{F}(L^1(\mathbb{R}^n)) \to \mathbb{C}$, of norm
$\leq 1$. Using $\phi(0)=1$, one proves that this functional has norm $1$. (If we could apply this inequality to $\mathscr{F}(\delta)$ the two sides would be equal, thus
 to prove that the operator norm is 1, one applies the inequality to a sequence of functions that converge weakly to $\delta$.)
We take as known that $\mathscr{F}(L^1(\mathbb{R}^n))$ is dense in the Banach space $C_0(\mathbb{R}^n)$, so there is a 
bounded linear functional $\Phi:C_0(\mathbb{R}^n) \to \mathbb{C}$ whose restriction to
$\mathscr{F}(L^1(\mathbb{R}^n))$ is equal to $\hat{f} \mapsto \int_{\mathbb{R}^n} f \phi dm_n$, and  $\norm{\Phi}=1$.

Using the Riesz-Markov theorem,\footnote{Walter Rudin, {\em Real and Complex Analysis}, third ed., p.~130, Theorem 6.19.}
there is a regular complex Borel measure $\mu$ on $\mathbb{R}^n$ such that
\[
\Phi(g) = \int_{\mathbb{R}^n} g d\mu, \qquad g \in C_0(\mathbb{R}^n),
\]
and $\norm{\mu}=\norm{\Phi}$; $\norm{\mu}$ is the {\em total variation norm} of $\mu$, $\norm{\mu}=|\mu|(\mathbb{R}^n)$.
Then for $f \in L^1(\mathbb{R}^n)$ we have
\begin{eqnarray*}
\int_{\mathbb{R}^n} f\phi dm_n&=&\Phi(\hat{f})\\
&=&\int_{\mathbb{R}^n} \hat{f} d\mu \\
&=&\int_{\mathbb{R}^n}\left( \int_{\mathbb{R}^n} e^{-i\xi\cdot x} f(x) dm_n(x) \right) d\mu(\xi)\\
&=&\int_{\mathbb{R}^n} f(x) \left( \int_{\mathbb{R}^n} e^{-ix\cdot \xi} d\mu(\xi) \right) dm_n(x)\\
&=&\int_{\mathbb{R}^n} f(x) \hat{\mu}(x) dm_n(x).
\end{eqnarray*}
That this is true for all $f \in L^1(\mathbb{R}^n)$ implies that $\phi=\hat{\mu}$.
As $\mu(\mathbb{R}^n)=\hat{\mu}(0)=\phi(0)=1$ and $\norm{\mu}=\norm{\Phi}=1$ we have $\mu(\mathbb{R}^n) = \norm{\mu}$, and this implies that
$\mu$ is positive measure, hence, as $\mu(\mathbb{R}^n)=1$, a probability measure.

\end{proof}

\end{document}