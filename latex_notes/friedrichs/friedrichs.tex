\documentclass{article}
\usepackage{amsmath,amssymb,graphicx,subfig,mathrsfs,amsthm}
%\usepackage{tikz-cd}
\usepackage{hyperref}
\newcommand{\innerL}[2]{\langle #1, #2 \rangle_{L^2}}
\newcommand{\inner}[2]{\langle #1, #2 \rangle}
\newcommand{\HSnorm}[1]{\Vert #1 \Vert_{\ensuremath\mathrm{HS}}}
\newcommand{\HSinner}[2]{\left\langle #1, #2 \right\rangle_{\ensuremath\mathrm{HS}}}
\newcommand{\tr}{\textrm{tr}} 
\newcommand{\Span}{\textrm{span}} 
\newcommand{\id}{\textrm{id}} 
\newcommand{\Hom}{\textrm{Hom}}
\newcommand{\HS}{B_{\ensuremath\mathrm{HS}}} 
\newcommand{\norm}[1]{\Vert #1 \Vert}
\newtheorem{theorem}{Theorem}
\newtheorem{lemma}[theorem]{Lemma}
\newtheorem{proposition}[theorem]{Proposition}
\newtheorem{corollary}[theorem]{Corollary}
\newtheorem{definition}[theorem]{Definition}
\begin{document}
\title{Unbounded operators and the Friedrichs extension}
\author{Jordan Bell\\ \texttt{jordan.bell@gmail.com}\\Department of Mathematics, University of Toronto}
\date{\today}


\maketitle

\section{Introduction}
In this note, by $A \subset B$, I mean that $A$ is contained in $B$, and it may be that $A=B$; usually
I write this by $A \subseteq B$, but $A \subset B$  fits with the usual notation for saying that an operator
is an extension of another.

In this note, unless we say otherwise $H$ denotes a Hilbert space over $\mathbb{C}$, and
we do not presume $H$ to be separable.
We shall write the inner product $\inner{\cdot}{\cdot}$ on $H$ as conjugate linear in the second argument.

We say that {\em $T$ is an operator in $H$} if there is a linear subspace
$\mathscr{D}(T)$ of $H$ such that $T:\mathscr{D}(T) \to H$ is a linear map.
We call $\mathscr{D}(T)$ the {\em domain of $T$}
and  $\mathscr{R}(T)=T(\mathscr{D}(T))$ the {\em range of $T$}. We do not presume unless we say so that
$\mathscr{D}(T)$ is dense in $H$, and we say that $T$ is {\em densely defined} when this is so.


Define
\[
\mathscr{G}(T) = \{(x,Tx): x \in \mathscr{D}(T)\},
\]
called the {\em graph of $T$}.  We say that an operator $S$ is an {\em extension} of an operator $T$ if
$\mathscr{G}(T) \subset \mathscr{G}(S)$, and we write $T \subset S$.
The set of all extensions of an operator is a partially ordered set, so it makes sense to talk about a maximal
extension. In particular, if $\mathscr{D}(T) = H$ then $T$ is maximal.

$H \times H$ is a Hilbert space with the inner product
\[
\inner{(x,y)}{(v,w)} = \inner{x}{v}+\inner{y}{w}.
\]
We say that an operator $T$ is {\em closed} if $\mathscr{G}(T)$ is a closed subset of $H \times H$. Thus, to say
that $T$ is a closed operator means that if $x_n$ is a sequence in $\mathscr{D}(T)$ and $(x_n,Tx_n) \to (x,y) \in H \times H$,
then $(x,y) \in \mathscr{G}(T)$, i.e. $x \in \mathscr{D}(T)$ and $y=Tx$. It is apparent that if $T \in \mathscr{B}(H)$ then
$T$ is closed. On the other hand, if $T$ is closed and $\mathscr{D}(T)=H$, then the closed graph theorem tells
us that $T \in \mathscr{B}(H)$. 

\section{Adjoints}
Following Garrett,\footnote{Paul Garrett, {\em Unbounded operators, Friedrichs� extension theorem}, \url{http://www.math.umn.edu/~garrett/m/v/friedrichs.pdf}}
we say that an operator $T'$ is a {\em sub-adjoint} of an operator $T$ if
\[
\inner{Tv}{w} = \inner{v}{T'w}, \qquad v \in \mathscr{D}(T), w \in \mathscr{D}(T').
\]
Obviously, $T'=0$ with $\mathscr{D}(T')=\{0\}$ is a sub-adjoint of any operator. 
As well, if $T$ is densely defined, then for any linear subspace $V$ of $H$ there is at most one sub-adjoint of $T$ with domain $V$. 

We define $J:H \times H \to H \times H$ by $J(v,w)=(-w,v)$. $J$ is unitary. We follow Garrett's proof of the following theorem.\footnote{Paul Garrett, {\em Unbounded operators, Friedrichs� extension theorem}, \url{http://www.math.umn.edu/~garrett/m/v/friedrichs.pdf}}

\begin{theorem}
If $T$ is densely defined, then it has a a unique maximal sub-adjoint, denoted
$T^*$ and called the {\em adjoint} of $T$. $T^*$ is closed, with 
\[
\mathscr{G}(T^*) = J(\mathscr{G}(T))^\perp.
\]
\label{adjoint}
\end{theorem}
\begin{proof}
Write $X=J(\mathscr{G}(T))^\perp$. 
Suppose that $T'$ is a sub-adjoint of $T$. For $w \in \mathscr{D}(T')$ and $v \in \mathscr{D}(T)$,
\[
\inner{J(v,Tv)}{(w,T'w)} = \inner{(-Tv,v)}{(w,T'w)} = \inner{-Tv}{w}+\inner{v}{T'w} = 0,
\]
showing that $\mathscr{G}(T') \subset X$. 

For any $w \in H$, 
suppose that $(w,w_1),(w,w_2) \in X$. This means that for all $v \in \mathscr{D}(T)$,
$\inner{(w,w_1)}{(-Tv,v)}=0$ and $\inner{(w,w_2)}{(-Tv,v)}=0$, i.e.
$\inner{w}{-Tv}+\inner{w_1}{v}=0$ and $\inner{w}{-Tv}+\inner{w_2}{v}=0$, so
$\inner{w_1-w_2}{v}=0$.  Because $\mathscr{D}(T)$ is dense in $H$, this implies that
$w_1-w_2=0$ (lest a sequence of things that are each $0$ converge to something that is not $0$).
Therefore for any $w \in H$ there is at most one $w' \in H$ such that $(w,w') \in X$, and we define 
\[
W = \{w \in H: \textrm{there is some $w' \in H$ such that $(w,w') \in X$}\}.
\]
We define $T^*:W \to H$ by $Tw=w'$, so $T^*$ is an operator with $\mathscr{D}(T^*)=W$. 
It is apparent that $\mathscr{G}(T^*)=X$. 

For $v \in \mathscr{D}(T)$ and $w \in \mathscr{D}(T^*)$,  using $(w,T^*w) \in X$ and $(-Tv,v) \in J(\mathscr{G}(T))$ we get
\[
0=\inner{(-Tv,v)}{(w,T^*w)} = \inner{-Tv}{w} + \inner{v}{T^*w},
\]
showing that $T^*$ is a sub-adjoint of $T$. 

If $T'$ is a sub-adjoint of $T$, we have shown that $\mathscr{G}(T') \subset X$ and that $\mathscr{G}(T^*)=X$, giving
$T \subset T^*$. Hence $T^*$ is the unique maximal sub-adjoint of $T$. 

$\mathscr{G}(T^*)=X$ is an orthogonal complement hence closed in $H \times H$, meaning that $T^*$ is a closed operator,
completing the proof.
\end{proof}

Using the expression in the above theorem for the graph of the adjoint as an orthogonal complement, if
$T_1,T_2$ are densely defined and $T_1 \subset T_2$, then $T_2^* \subset T_1^*$.

\begin{definition}
Suppose that $T$ is an operator in $H$. We say that $T$ is {\em self-adjoint} if
$T$ is densely defined and  $T=T^*$, i.e. $\mathscr{G}(T) = \mathscr{G}(T^*)$.
\end{definition}


\begin{theorem}
If $T$ is a densely defined closed operator in $H$, then
\[
H \times H = J(\mathscr{G}(T)) \oplus \mathscr{G}(T^*)=\mathscr{G}(T) \oplus J(\mathscr{G}(T^*)).
\]
\label{orthogonal}
\end{theorem}
\begin{proof}
Because $T$ is densely defined we have $\mathscr{G}(T^*)=J(\mathscr{G}(T))^\perp$.
Then taking orthogonal complements, $\overline{J(\mathscr{G}(T))} = \mathscr{G}(T^*)^\perp$. But $T$ is closed and $J$ is
unitary, so $\overline{J(\mathscr{G}(T))}=J(\mathscr{G}(T))$, giving
$J(\mathscr{G}(T)) =  \mathscr{G}(T^*)^\perp$.
Moreover, $\mathscr{G}(T^*)$ is a closed linear subspace of $H \times H$, so
\[
H \times H =\mathscr{G}(T^*) \oplus \mathscr{G}(T^*)^\perp = \mathscr{G}(T^*) \oplus
 J(\mathscr{G}(T)).
\]
Because $J$ is unitary and $J^2=I$,
\[
H \times H = J(H \times H) = J(\mathscr{G}(T^*)) \oplus J^2(\mathscr{G}(T)) = J(\mathscr{G}(T^*)) \oplus \mathscr{G}(T).
\]
\end{proof}



We now use the above orthogonal direct sum to show that the adjoint of a densely
defined closed operator is itself densely defined; then since $T^*$ is densely defined it makes sense to talk about $T^{**}$, and
this is equal to $T$. The proof follows Rudin.\footnote{Walter
Rudin, {\em Functional Analysis}, second ed., p.~354, Theorem 13.12.}

\begin{theorem}
If $T$ is a densely defined  closed operator in $H$, then $\mathscr{D}(T^*)$ is dense in $H$ and $T^{**}=T$.
\end{theorem}
\begin{proof}
Suppose that $z \in \mathscr{D}(T^*)^\perp$. For all $y \in \mathscr{D}(T^*)$,
$\inner{z}{y}=0$, which can be written as
$\inner{(0,z)}{(-T^*y,y)}=0$, which means that $(0,z) \in (J(\mathscr{G}(T^*))^\perp$. But by Theorem \ref{orthogonal}, 
$(J(\mathscr{G}(T^*))^\perp=\mathscr{G}(T)$, so $(0,z) \in \mathscr{G}(T)$. That is, $T(0)=z$, hence $z=0$. Therefore
$\mathscr{D}(T^*)^\perp=\{0\}$, which implies that $\mathscr{D}(T^*)$ is dense in $H$. 

Because $T^*$ is densely defined, we can apply Theorem \ref{orthogonal} to get
\[
H \times H = J(\mathscr{G}(T^*)) \oplus \mathscr{G}(T^{**}).
\]
But we also have 
\[
H \times H = \mathscr{G}(T) \oplus J(\mathscr{G}(T^*)).
\]
Therefore $\mathscr{G}(T^{**})= \mathscr{G}(T)$.
\end{proof}



\section{Symmetric operators}
We say that an operator $T$ in $H$ is {\em symmetric} if  
\[
\inner{Tx}{y} = \inner{x}{Ty}, \qquad x,y \in \mathscr{D}(T).
\]

\begin{theorem}
Suppose that $T$ is a densely defined operator in $H$. Then $T$ is symmetric if and only if $T \subset T^*$. 
\label{symmetric}
\end{theorem}
\begin{proof}
Suppose that $T$ is symmetric. For $v,w \in \mathscr{D}(T)$, 
\[
\inner{(-Tv,v)}{(w,Tw)} = \inner{-Tv}{w}+\inner{v}{Tw} = 0,
\]
showing by Theorem \ref{adjoint} that $(w,Tw) \in \mathscr{G}(T^*)$. Therefore $\mathscr{G}(T) \subset \mathscr{G}(T^*)$, i.e.
$T \subset T^*$.

Suppose that $T \subset T^*$. For $x,y \in \mathscr{D}(T)$, the fact that $T \subset T^*$ gives $Tx=T^*x$, so 
\[
\inner{x}{Ty} = \inner{x}{T^*y} = \inner{Tx}{y},
\]
showing that $T$ is symmetric.
\end{proof}


\begin{lemma}
If $T$ is a symmetric operator in $H$ and $\lambda \in \mathbb{C}$ is an eigenvalue
of $T$, then $\lambda \in \mathbb{R}$. 
\end{lemma}
\begin{proof}
Let $v \in \mathscr{D}(T)$, $v \neq 0$ and $Tv=\lambda v$. $T$ being symmetric gives
$\inner{Tv}{v}=\inner{v}{Tv}$, hence $\inner{\lambda v}{v}=\inner{v}{\lambda v}$ and thus
$\lambda \norm{v} = \overline{\lambda} \norm{v}$, and $\norm{v} \neq 0$ so
$\lambda = \overline{\lambda}$, meaning $\lambda \in \mathbb{R}$. 
\end{proof}

\begin{definition}
An operator $T$ in $H$ is called {\em positive} if it is symmetric and if
\[
\inner{Tv}{v} \geq 0, \qquad v \in \mathscr{D}(T);
\]
we stipulate that $T$ is symmetric so that the left-hand side of the above inequality is real.
\end{definition}


\section{The Hellinger-Toeplitz theorem}
The Hellinger-Toeplitz theorem is the statement that if an operator in a Hilbert space is defined everywhere and is symmetric, then it is in fact bounded.
Our proofs follows Rudin.\footnote{Walter Rudin, {\em Functional Analysis}, second ed., p.~353, Theorem 13.11.}

\begin{theorem}[Hellinger-Toeplitz theorem]
If  $T$ is a symmetric operator in $H$ with $\mathscr{D}(T)=H$, then $T \in \mathscr{B}(H)$.
\end{theorem}
\begin{proof}
Because $\mathscr{D}(T)=H$, of course $T$ is densely defined, so because $T$ is symmetric, by Theorem \ref{symmetric} we have $T \subset T^*$;
it makes sense to talk about $T^*$ because $T$ is densely defined. $T \subset T^*$ and $\mathscr{D}(T)=H$ together imply $T=T^*$.
But from Theorem \ref{adjoint}, $\mathscr{G}(T^*)$ is closed, and hence $\mathscr{G}(T)$ is closed too.
Then, because $\mathscr{D}(T)=H$ and $\mathscr{G}(T)$ is closed, the closed graph theorem tells us that $T$ is continuous.
\end{proof}



\section{Friedrichs extension}
The proof of the following theorem expands on Garrett.\footnote{Paul Garrett, {\em Unbounded operators, Friedrichs� extension theorem}, \url{http://www.math.umn.edu/~garrett/m/v/friedrichs.pdf}}

\begin{theorem}[Friedrichs extension]
If $T$ is densely defined and positive, then there is an operator in $H$ that is self-adjoint and positive and whose restriction to $\mathscr{D}(T)$ is equal to $T$.
\end{theorem}
\begin{proof}
Define 
\[
\inner{v}{w}_1 = \inner{v}{w}+\inner{Tv}{w}, \qquad v,w \in \mathscr{D}(T).
\]
It is apparent that $\inner{\cdot}{\cdot}_1$ is a Hermitian form on the vector space $\mathscr{D}(T)$, conjugate linear in the second argument.
Moreover, for $v \in \mathscr{D}(T)$,
\[
\inner{v}{v}_1 = \inner{v}{v}+\inner{Tv}{v} \geq 0
\]
because $T$ is positive. Therefore $\inner{\cdot}{\cdot}_1$ is an inner product
on $\mathscr{D}(T)$.

Let $K$ be the completion of $\mathscr{D}(T)$ with respect
to the inner product $\inner{\cdot}{\cdot}_1$. That is, $K$ is a Hilbert space, and there is a one-to-one linear map
$k:\mathscr{D}(T) \to K$ such that $\inner{kv}{kw}_K = \inner{v}{w}_1$ for all $v,w \in \mathscr{D}(T)$, and
$k(\mathscr{D}(T))$ is dense in $K$. $k$ is an isometry, so it makes sense to define $j:k(\mathscr{D}(T)) \to H$ by
$j(k(x))=x$, and $j$ is itself an isometry. Because $k(\mathscr{D}(T))$ is dense in $K$ and $j$ is a bounded linear map,
there is a unique bounded linear map $\hat{j}:K \to H$ whose restriction to $k(\mathscr{D}(T))$ is equal to $j$, and 
$\norm{\hat{j}} = \norm{j} \leq 1$.
Suppose that $\hat{j}(\phi)=0$ for some $\phi \in K$. As $k(\mathscr{D}(T))$ is dense in $K$, there is a sequence $v_n \in \mathscr{D}(T)$
such that $\norm{kv_n - \phi}_K \to 0$, and as
\[
\norm{v_n} \leq \norm{v_n}_1 = \norm{j(kv_n)}_1 = \norm{\hat{j}(kv_n)}_1=\norm{\hat{j}(kv_n-\phi)}_1 \leq \norm{kv_n-\phi}_K,
\]
this means that $\norm{v_n} \to 0$. Then,
\begin{eqnarray*}
\norm{\phi}_K^2&=&\inner{\phi}{\phi}_K\\
&=&\lim_{n \to \infty} \inner{\phi}{kv_n}_K\\
&=&\lim_{m \to \infty} \lim_{n \to \infty} \inner{kv_m}{kv_n}_K\\
&=&\lim_{m \to \infty} \lim_{n \to \infty} \inner{v_m}{v_n}_1\\
&=&\lim_{m \to \infty} \lim_{n \to \infty} (\inner{v_m}{v_n}+\inner{Tv_m}{v_n})\\
&\leq&\limsup_{m \to \infty} \limsup_{n \to \infty} (\norm{v_m}\norm{v_n} + \norm{Tv_m} \norm{v_n})\\
&=&\limsup_{m \to \infty} 0\\
&=&0;
\end{eqnarray*}
this uses $\norm{v_n} \to 0$, and does not presume that $T$ is bounded. Hence $\norm{\phi}_K=0$, so
$\phi=0$. This shows 
\[
\textrm{$\hat{j}:K \to H$ is one-to-one.}
\]

For $h \in H$,  define $\lambda_h:K \to \mathbb{C}$ by
\[
\lambda_h \phi = \inner{\hat{j}\phi}{h}, \qquad \phi \in K,
\]
 which satisfies
\[
|\lambda_h \phi| \leq \norm{\hat{j}\phi} \norm{h} \leq \norm{\phi}_K \norm{h}, \qquad \phi \in K,
\] 
and hence $\norm{\lambda_h} \leq \norm{h}$, so by the Riesz representation theorem there is a unique $Ch \in K$ satisfying
$\norm{Ch}_1=\norm{\lambda_h} \leq \norm{h}$ and 
\[
\lambda_h \phi = \inner{\phi}{Ch}_K, \qquad \phi \in K.
\]
For $h_1,h_2 \in H$, $\alpha \in\mathbb{C}$, and $\phi \in K$,
\begin{eqnarray*}
\inner{\phi}{C(\alpha h_1+h_2)}_K&=&\lambda_{\alpha h_1+h_2} \phi\\
&=&\inner{\hat{j}\phi}{\alpha h_1+h_2}\\
&=&\overline{\alpha}\inner{\hat{j}\phi}{h_1} + \inner{\hat{j}\phi}{h_2}\\
&=&\overline{\alpha} \lambda_{h_1} \phi + \lambda_{h_2} \phi\\
&=&\overline{\alpha} \inner{\phi}{Ch_1}_K + \inner{\phi}{Ch_2}_K\\
&=&\inner{\phi}{\alpha Ch_1+Ch_2}_K.
\end{eqnarray*}
This being true for all $\phi \in K$ implies that $C(\alpha h_1+h_2) = \alpha Ch_1+Ch_2$.
Thus, $C:H \to K$ is linear, and $\norm{C} \leq 1$. Define $B:H \to H$ by
\[
B=\hat{j} \circ C,
\]
which satisfies $\norm{B} \leq \norm{\hat{j}} \norm{C} \leq 1$, so $B \in \mathscr{B}(H)$.

For $v,w \in H$,
\begin{eqnarray*}
\inner{Bv}{w}&=&\inner{\hat{j}(Cv)}{w}\\
&=&\lambda_w (Cv)\\
&=&\inner{Cv}{Cw}_K\\
&=&\overline{\inner{Cw}{Cv}_K}\\
&=&\overline{\lambda_v(Cw)}\\
&=&\overline{\inner{\hat{j}(Cw)}{v}}\\
&=&\overline{\inner{Bw}{v}}\\
&=&\inner{v}{Bw},
\end{eqnarray*}
so $B$ is self-adjoint.
Moreover, for $v \in H$,
\[
\inner{Bv}{v} = \inner{\hat{j}(Cv)}{v} = \lambda_v(Cv) = \inner{Cv}{Cv}_K \geq 0.
\]
Therefore, $B$ is a positive operator. 
As well, if $Bv=0$ then $(\hat{j} \circ C)(v)=0$, and as $\hat{j}$ is one-to-one this means that $Cv=0$, and hence 
for all $\phi \in K$,
\[
0=\inner{\phi}{Cv}_K =   \lambda_v \phi = \inner{\hat{j}\phi}{v}.
\]
But $\mathscr{D}(T) \subset \hat{j}(K)$, so the above holding for all $\phi \in K$
means in particular that $\inner{w}{v}=0$ for all $w \in \mathscr{D}(T)$. As $\mathscr{D}(T)$ is dense in $H$, this implies that
$v=0$. This shows that $B$ is one-to-one. Finally, suppose
that $\phi \in C(H)^\perp$, so for all $h \in H$,
\[
0=\inner{\phi}{Ch}_K = \lambda_h \phi =  \inner{\hat{j}\phi}{h}.
\]
Because this holds for all $h \in H$, we have $\hat{j}\phi=0$, and $\hat{j}$ is one-to-one so $\phi=0$. This shows
that $C(H)$ is dense in $K$. Then, as $\hat{j}:K \to \hat{j}(K)$ is a bijection, we have $B(H)$ is dense in $\hat{j}(K)$.
But $\mathscr{D}(T) \subset \hat{j}(K)$ and $\mathscr{D}(T)$ is dense in $H$, so $B(H)$ is dense in $H$.

We define $\mathscr{D}(A)=B(H)$, and define $A:\mathscr{D}(A) \to H$
by $Av=B^{-1}v$, which makes sense because $B$ is one-to-one.  We have just shown that $B(H)$ is
dense in $H$, so $A$ is a densely defined operator in $H$. As well, $A:\mathscr{D}(A) \to H$ is onto, because $B \in \mathscr{B}(H)$.
$A$ is symmetric: for $v,w \in \mathscr{D}(A)=B(H)$, there are $x,y \in H$ with $Bx=v$ and $By=w$, and using the fact that
$B$ is self-adjoint,
\[
\inner{Av}{w} = \inner{A(Bx)}{By} = \inner{x}{By} = \inner{Bx}{y} = \inner{v}{Aw}.
\]
Moreover, $A$ is positive: for $v \in \mathscr{D}(A)$ there is some $x \in H$ with $Bx=v$, and  the fact that $B$ is positive gives
\[
\inner{Av}{v} = \inner{A(Bx)}{Bx} = \inner{x}{Bx} \geq 0.
\]

In this paragraph we show that $A$ is self-adjoint; because $A$ is densely defined it indeed has an adjoint $A^*$. Define $U:H \times H \to H \times H$ by $U(v,w)=(w,v)$, which is unitary. It is
apparent that
\[
\mathscr{G}(A) = U(\mathscr{G}(B)).
\]
For any linear subspace $X$ of $H \times H$,
\[
(UX)^\perp = U(X^\perp).
\]
Then using $J \circ U = - U \circ J$ and Theorem \ref{adjoint} we obtain (since $-X=X$ when $X$ is a vector space)
\begin{eqnarray*}
\mathscr{G}(A^*) &=&J(\mathscr{G}(A))^\perp\\
&=&J(U(\mathscr{G}(B)))^\perp\\
&=&U(J(\mathscr{G}(B)))^\perp\\
&=&U(J(\mathscr{G}(B))^\perp)\\
&=&U(\mathscr{G}(B^*))\\
&=&U(\mathscr{G}(B))\\
&=&\mathscr{G}(A).
\end{eqnarray*}
Thus $A$ is self-adjoint.




Define $S:\mathscr{D}(T) \to H$ by $S=\id_H + T$. For $v,w \in \mathscr{D}(T)$,
\[
\inner{v}{Sw} = \inner{\hat{j}(kv)}{Sw} = \lambda_{Sw}(kv) = \inner{kv}{C(Sw)}_K = 
\]
and also
\[
\inner{v}{Sw} = \inner{v}{w}+\inner{v}{Tw}
= \inner{v}{w}+\inner{Tv}{w} = \inner{v}{w}_1 =
 \inner{kv}{kw}_K.
\]
Because $k(\mathscr{D}(T))$ is dense in $K$,  it follows that, for all $w \in \mathscr{D}(T)$,
$C(Sw)=kw$, or $\hat{j}(C(Sw))=\hat{j}(kw)$, i.e.,
$B(Sw)=w$. This shows that $w \in B(H) = \mathscr{D}(T)$, so
\[
\mathscr{D}(T) \subset \mathscr{D}(A),
\]
and we can apply
$A$ to $B(Sw)=w$ and get
$Sw=Aw$. Therefore,
\[
S \subset A.
\]


Define $\mathscr{D}(F)=\mathscr{D}(A)$ and $F=A-\id_H$. We verify that $F$ is self-adjoint:
\begin{eqnarray*}
&=&
\end{eqnarray*}
For $v \in \mathscr{D}(A)=B(H)$, there is some $x \in H$ with $Bx=v$, and
\[
\inner{v}{Av} = \inner{Bx}{B^{-1}Bx} = \inner{Bx}{x} = \inner{\hat{j}(Cx)}{x} = \lambda_x(Cx) =  \inner{Cx}{Cx}_K,
\]
but $\norm{v}=\norm{Bx}=\norm{\hat{j}(Cx)} \leq \norm{Cx}_K$. This shows that
\[
\inner{v}{Av-v} \geq 0, \qquad v \in \mathscr{D}(A),
\]
or
\[
\inner{v}{Fv} \geq 0.
\]
Therefore, $F$ is positive. For $v \in \mathscr{D}(T)$, which is contained in $\mathscr{D}(F)$, using the fact that $S \subset A$,
\[
Fv=Av-v=Sv-v=v+Tv-v=Tv.
\]
Therefore,
\[
T \subset F.
\]
We have established that $T$ is self-adjoint and positive, and thus $F$ is the operator we wish to obtain.
\end{proof}





\end{document}