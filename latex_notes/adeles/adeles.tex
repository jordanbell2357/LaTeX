\documentclass{article}
\usepackage{amsmath,amssymb,mathrsfs,amsthm}
%\usepackage{tikz-cd}
%\usepackage{hyperref}
\newcommand{\inner}[2]{\left\langle #1, #2 \right\rangle}
\newcommand{\tr}{\ensuremath\mathrm{tr}\,} 
\newcommand{\Span}{\ensuremath\mathrm{span}} 
\def\Re{\ensuremath{\mathrm{Re}}\,}
\def\Im{\ensuremath{\mathrm{Im}}\,}
\newcommand{\id}{\ensuremath\mathrm{id}} 
\newcommand{\var}{\ensuremath\mathrm{var}} 
\newcommand{\Lip}{\ensuremath\mathrm{Lip}} 
\newcommand{\GL}{\ensuremath\mathrm{GL}}
\newcommand{\diam}{\ensuremath\mathrm{diam}} 
\newcommand{\sgn}{\ensuremath\mathrm{sgn}\,} 
\newcommand{\lcm}{\ensuremath\mathrm{lcm}} 
\newcommand{\supp}{\ensuremath\mathrm{supp}\,}
\newcommand{\dom}{\ensuremath\mathrm{dom}\,}
\newcommand{\upto}{\nearrow}
\newcommand{\downto}{\searrow}
\newcommand{\norm}[1]{\left\Vert #1 \right\Vert}
\newtheorem{theorem}{Theorem}
\newtheorem{lemma}[theorem]{Lemma}
\newtheorem{proposition}[theorem]{Proposition}
\newtheorem{corollary}[theorem]{Corollary}
\theoremstyle{definition}
\newtheorem{definition}[theorem]{Definition}
\newtheorem{example}[theorem]{Example}
\begin{document}
\title{The adeles}
\author{Jordan Bell\\ \texttt{jordan.bell@gmail.com}\\Department of Mathematics, University of Toronto}
\date{April 11, 2016}

\maketitle

\section{Restricted products}
Let $I$ be a nonempty set and for $i \in I$ suppose that 
$X_i$ is a locally compact space and that $K_i$ is a compact open set in $K_i$. 
A subset $J$ of $I$ is said to be \textbf{almost all $I$} if $I \setminus J$ is finite.
Define the \textbf{restricted product}
\[
X = \widehat{\prod}_{i \in I}^{K_i} X_i = \left\{ x \in \prod_{i \in I} X_i: \textrm{$x_i \in K_i$ for almost
all $i \in I$}\right\}.
\]
The \textbf{restricted product topology} is the topology $\tau$ on $X$ generated by the collection $\mathscr{B}$
of sets of the form
$\prod_{i \in E} U_i \times \prod_{i \not \in E} K_i$ where $E \subset I$ is finite and for each $i \in E$,
$U_i$ is an open set in $X_i$.

\begin{lemma}
$\mathscr{B}$ is a base for $\tau$.
\end{lemma}
\begin{proof}
For $B_1=\prod_{i \in E_1} U_i \times \prod_{i \not \in E_1} K_i$
and  $B_2 =\prod_{i \in E_2} V_i \times \prod_{i \not \in E_2} K_i$ in $\mathscr{B}$, 
\begin{align*}
B_1 \cap B_2 &= \prod_{i \in E_1 \setminus E_2} (U_i \cap K_i)  \prod_{i \in E_2 \setminus E_1} 
(K_i \cap V_i)
\prod_{i \in E_1 \cap E_2} (U_i \cap V_i)
\prod_{i \not \in E_1 \cup E_2} K_i.
\end{align*}
Since $K_i$ is open, $U_i \cap K_i$ and $K_i \cap V_i$ are open, and because $E_1 \cup E_2$ is finite we
get that $B_1 \cap B_2$ belongs to $\mathscr{B}$.
\end{proof}

We prove that the restricted product is locally compact.\footnote{Anton Deitmar and Siegfried Echterhoff,
{\em Principles of Harmonic Analysis}, second ed., p.~258, Lemma 13.3.1.} This is  a motivation for using this object.

\begin{lemma}
$X$ is locally compact.
\label{LCH}
\end{lemma}
\begin{proof}
For $x \in X$, let $E \subset I$ be finite with $x \in K_i$ for $i \not \in E$. For $i \in E$, 
because $X_i$ is locally compact there is a compact neighborhood $N_i$ of $x_i$, with
$U_i$ open, $x \in U_i \subset N_i$.  
Then the product $\prod_{i \in E} N_i \times \prod_{i \not \in E} K_i$ is compact,
$\prod_{i \in E} U_i \times \prod_{i \not \in E} K_i$ is open, and
\[
x \in \prod_{i \in E} U_i \times \prod_{i \not \in E} K_i \subset \prod_{i \in E} N_i \times \prod_{i \not \in E} K_i,
\]
showing that $X$ is locally compact. 
\end{proof}


The following is part of the machinery of restricted products.\footnote{Anton Deitmar and Siegfried Echterhoff,
{\em Principles of Harmonic Analysis}, second ed., p.~258, Lemma 13.3.1.}

\begin{lemma}
For nonempty disjoint sets $A,B \subset I$ with $I = A \cup B$,
the topological spaces
\[
\widehat{\prod}_{i \in I}^{K_i} X_i
\]
and
\[
\left(\widehat{\prod}_{i \in A}^{K_i} X_i\right) \times \left(\widehat{\prod}_{i \in B}^{K_i} X_i\right).
\]
are homeomorphic.
\end{lemma}



\section{Adeles}
For a nonempty set of primes $S$, define
\[
\mathbb{A}_S = \widehat{\prod}_{p \in S}^{\mathbb{Z}_p} \mathbb{Q}_p,
\qquad \mathbb{A}^S = \widehat{\prod}_{p \not \in S}^{\mathbb{Z}_p} \mathbb{Q}_p.
\]
Because $\mathbb{Z}_p$ is a ring, $\mathbb{A}_S$ is a ring. 
We prove that $\mathbb{A}_S$ is a locally compact topological ring.\footnote{Anton Deitmar and Siegfried Echterhoff,
{\em Principles of Harmonic Analysis}, second ed., p.~258, Theorem 13.3.2.}

\begin{lemma}
For any nonempty set $S$ of primes, $\mathbb{A}_S$ is a locally compact topological ring.
\end{lemma}
\begin{proof}
For $a,b \in \mathbb{A}_S$, let 
$a+b \in U = \prod_{p \in E} U_p \times \prod_{p \in S \setminus E} \mathbb{Z}_p$, for $U_p$ open in $\mathbb{Q}_p$.
But $(x,y) \mapsto x+y$ is continuous $\mathbb{Q}_p \times \mathbb{Q}_p \to \mathbb{Q}$, so
each for $p \in E$ there is an open neighborhood $V_p$ of $a_p$ in $\mathbb{Q}_p$
 and an open neighborhood $W_p$ of $b_p$ in $\mathbb{Q}_p$ such that $x+y \in U_p$ for
 $(x,y) \in V_p \times W_p$. 
Let 
\[
V = \prod_{p \in E} V_p \times \prod_{p \in S \setminus E} \mathbb{Z}_p,
\qquad
W = \prod_{p \in E} W_p \times \prod_{p \in S \setminus E} \mathbb{Z}_p.
\]
which belong to $\mathscr{B}$. $(a,b) \in V \times W$, and if 
$(x,y) \in V \times W$ then $x+y \in U$, which shows that $(x,y) \mapsto x+y$ is continuous
$\mathbb{A}_S \times \mathbb{A}_S \to \mathbb{A}_S$.

Likewise, let $a \cdot b \in U = \prod_{p \in E} U_p \times \prod_{p \in S \setminus E} \mathbb{Z}_p$.
Because $(x,y) \mapsto x\cdot y$ is continuous $\mathbb{Q}_p \times \mathbb{Q}_p \to \mathbb{Q}_p$,
for each $p \in E$ there are open $a_p \in V_p \subset \mathbb{Q}_p$ and $b_p \in W_p \subset \mathbb{Q}_p$
such that $x\cdot y \in U_p$ for $(x,y) \in V_p \times W_p$. 

This shows that $\mathbb{A}_S$ is a topological ring. Finally, by Lemma \ref{LCH}, $\mathbb{A}_S$ 
is locally compact.
\end{proof}

Let $\mathbb{A}_{\textrm{fin}} = \widehat{\prod}_{p<\infty}^{\mathbb{Z}_p} \mathbb{Q}_p$, which is a locally
compact topological ring. Finally let 
\[
\mathbb{A} = \mathbb{R} \times \mathbb{A}_{\textrm{fin}}
=\mathbb{R} \times  \widehat{\prod}_{p<\infty}^{\mathbb{Z}_p} \mathbb{Q}_p,
\]
which is also a locally compact topological ring, whose elements are called \textbf{adeles}.






\section{Embedding the rationals in the adeles}
Write $N_p=\{0,\ldots,p-1\}$. $\mathbb{Q}_p \subset \prod_{\mathbb{Z}} N_p$. For 
$x \in \mathbb{Q}_p$,
\[
v_p(x) = \inf\{k \in \mathbb{Z}: x(k) \neq 0\}.
\]
$v_p(x) = \infty$ if and only if $x=0$.
\[
|x|_p = p^{-v_p(x)}.
\]
\[
\mathbb{Z}_p = \{x \in \mathbb{Q}_p: v_p(x) \geq 0\} = \{x \in \mathbb{Q}_p: |x|_p \leq 1\}.
\]
For $r \in \mathbb{Q}$ write $|r|_\infty = |r|$. 
It is straightforward that for $r \in \mathbb{Q}$, $r \neq 0$,
\[
|r|_\infty \cdot \prod_{p < \infty} |r|_p = \prod_{p \leq \infty} |r|_p = 1.
\]
Let $E_r=\{p<\infty: v_p(r)<0\}=\{p<\infty: |r|_p > 1\}$, which is finite.
Thus it makes sense to define
$\iota:\mathbb{Q} \to \mathbb{A}$ by 
$\iota(r)_p = r$ for $p \leq \infty$. It is immediate that $\iota$ is one-to-one.
Assign $\mathbb{Q}$ the discrete topology, and then $\iota$ is continuous. 
We shall prove that $\iota:\mathbb{Q} \to \iota(\mathbb{Q})$ is a homeomorphism.

\begin{theorem}[Chinese remainder theorem]
Let $S$ be a nonempty finite set of primes. For each $p \in S$ suppose $e_p$ is a positive integer and 
$c_p$ is an integer. Then there is a unique $x+\prod_{p \in S} p^{e_p} \mathbb{Z}$ such that
$x + p^{e_p} \mathbb{Z} = c_p + p^{e_p} \mathbb{Z}$ for all $p \in S$. 
\end{theorem}
\begin{proof}
Let $x,y \in \mathbb{Z}$ and 
suppose that $x+p^{e_p}\mathbb{Z}=c_p+p^{e_p}\mathbb{Z}$ and
$x+p^{e_p}\mathbb{Z}=c_p+p^{e_p}\mathbb{Z}$ for $p \in S$. 
This means that for each $p \in S$, $p^{e_p}$ divides $x-y$, and for
$p,q \in S$, $p \neq q$, $\gcd(p^{e_p},q^{e_q})=1$ so
$\prod_{p \in S} p^{e_p}$ divides $x-y$, meaning $x + \prod_{p \in S} p^{e_p} \mathbb{Z} = y + \prod_{p \in S} p^{e_p}$. 

Now let $N = \prod_{p \in S} p^{e_p}$ and for $p \in S$ let $N_p = p^{-e_p} N$.  
Then $\gcd(N_p,p^{e_p})=1$ so there is some $1 \leq u_p \leq p^{e_p}-1$ such that
\[
N_p u_p \equiv 1 \pmod{p^{e_p}}.
\]
Let $x = \sum_{p \in S} c_p N_p u_p \in \mathbb{Z}$. For $p,q \in S$, $q \neq p$,
$N_q \equiv 0 \pmod{p^{e_p}}$, so 
$x \equiv c_p \pmod{p^{e_p}}$. In other words, $x+ p^{e_p} \mathbb{Z} = c_p + p^{e_p}\mathbb{Z}$.
\end{proof}



\begin{theorem}[Weak approximation theorem]
Let $S$ be a nonempty finite set of primes and for $p \in S$ let $x_p \in \mathbb{Q}_p$. 
For $\epsilon>0$ there is some $r \in \mathbb{Q}$ such that
\[
|r-x_p|_p < \epsilon,\qquad p \in S.
\]
\label{weakapproximation}
\end{theorem}
\begin{proof}
Let $N=\prod_{p \in S} p$. For  each
$p \in S$ let $k_p>1$ such that $p^{-k_p} N<1$. 
Then define $y_p = p^{-k_p} N \in \mathbb{Q}_p$. 
$|y_p|_p = |p^{-k_p} p|=p^{k_p-1}$, so
\[
\left|\frac{y_p^n}{1+y_p^n}-1\right|_p =\frac{1}{|1+y_p^n|_p} 
\leq \frac{1}{|y_p|_p^n-1} = \frac{1}{p^{n(k_p-1)} -1} \to 0,
\]
and for $q \in S$ with $q \neq p$,
$|y_p|_q = |q|_q = q^{-1}$, so
\[
\left| \frac{y_p^n}{1+y_p^n} \right|_q =
\frac{q^{-n}}{|1+y_p^n|_q}
\leq \frac{q^{-n}}{1-q^{-n}} \to 0.
\]
For $p \in S$ take $r_p \in \mathbb{Q}$ with $|r_p-x_p|_p<\epsilon$.
For $n \geq 1$ define
\[
z_n = \sum_{p \in S} \frac{r_p y_p^n}{1+y_p^n} \in \mathbb{Q}.
\]
For $p \in S$,  $\sum_{q \in S} \frac{r_qy_q^n}{1+y_q^n} \to r_p$  in $\mathbb{Q}_p$.
Take $n_p$ with $|z_n-r_p|_p < \epsilon$, and for $n=\max\{n_p : p \in S\}$ and $r=z_n$, for any
$p \in S$ we have
\[
|r-x_p|_p = |r-r_p+r_p-x_p|_p \leq \max(|r-r_p|_p,|r_p-x_p|_p) < \epsilon.
\] 
\end{proof}




\begin{lemma}
Let $S$ be a nonempty finite set of primes and for each $p \in S$ suppose $x_p \in \mathbb{Z}_p$. 
For any $\epsilon>0$ there is some $x \in \mathbb{Z}$ such that
\[
|x-x_p|_p<\epsilon,\qquad p \in S.
\]
\end{lemma}
\begin{proof}
For each $p \in S$ let $y_p \in \mathbb{Z}_{\geq 0}$ with $|y_p-x_p|_p<\epsilon$. Take
$p^{-n_p}<\epsilon$ and let $n=\max\{n_p: p \in S\}$. 
By the Chinese remainder theorem, there is some $x \in \mathbb{Z}$ such that
$x + p^n \mathbb{Z} = y_p + p^n \mathbb{Z}$ for each $p \in  S$. 
Because $p^n$ divides $x-y_p$, $|x-y_p|_p \leq p^{-n}$. 
Then for any $p \in S$,
\[
|x-x_p|_p = |x-y_p + y_p-x_p|_p \leq \max(|x-y_p|_p,|y_p-x_p|_p)
<\epsilon.
\]
\end{proof}


In some arguments it is more convenient to work with the following \textbf{fundamental
domain} $D$ rather than $\mathbb{A}$.\footnote{Dorian Goldfeld and Joseph Hundley,
{\em Automorphic Representations and $L$-functions for the General Linear Group}, volume I, p.~10, Proposition 1.4.5.}

\begin{lemma}
Let
\[
D = [0,1) \times \prod_{p<\infty} \mathbb{Z}_p.
\]
The sets $\iota(r)+D$, $r \in \mathbb{Q}$, are pairwise disjoint and $\mathbb{A} = \bigcup_{r \in \mathbb{Q}} (\iota(r)+D)$.
\end{lemma}




\begin{theorem}
The subspace topology on $\iota(\mathbb{Q})$ inherited from $\mathbb{A}$ is discrete.
\end{theorem}












\end{document}