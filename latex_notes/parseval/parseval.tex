\documentclass{article}
\usepackage{amsmath,amssymb,graphicx,subfig,mathrsfs,amsthm}
%\usepackage{tikz-cd}
\newcommand{\inner}[2]{\left\langle #1, #2 \right\rangle}
\newcommand{\tr}{\textrm{tr}} 
\newcommand{\Span}{\textrm{span}} 
\newcommand{\SA}{B_{\textrm{sa}}(H)} 
\newcommand{\positive}{B_{\textrm{+}}(H)} 
\newcommand{\id}{\textrm{id}} 
\newcommand{\norm}[1]{\left\Vert #1 \right\Vert}
\newtheorem{theorem}{Theorem}
\newtheorem{lemma}[theorem]{Lemma}
\newtheorem{corollary}[theorem]{Corollary}
\begin{document}
\title{Abstract Fourier series and Parseval's identity}
\author{Jordan Bell\\ \texttt{jordan.bell@gmail.com}\\Department of Mathematics, University of Toronto}
\date{\today}
\maketitle

\section{Orthonormal basis}
Let $H$ be a separable complex Hilbert space.\footnote{One talk do everything we are doing and obtain the same results for nonseparable Hilbert spaces, but
one has to define what uncountable sums mean. This is done in John B. Conway, {\em A Course in Functional Analysis}, second ed., chapter I.}
 If $e_i \in H$, $i \geq 1$, and $\inner{e_i}{e_j}=\delta_{i,j}$, we say that the set $\{e_i\}$ is {\em orthonormal}. If
$\Span\{e_i: i \geq 1\}$ is a dense subspace of $H$, we say that $\{e_i: i \geq 1\}$ is an {\em orthonormal basis} for $H$. 
We can write this in another way. If $S_\alpha, \alpha \in I$ are subsets of $H$, let $\bigvee_{\alpha \in I} S_\alpha$ be the closure of the span of 
$\bigcup_{\alpha \in I} S_\alpha$. To say that $\{e_i\}$ is an orthonormal basis for $H$ is to say that $\{e_i\}$ is orthonormal and that
$H=\bigvee_{i \geq 1} \{e_i\}$.

\section{Abstract Fourier series}
If $a_k \in \mathbb{C}$ and the sequence
$\sum_{k=1}^n a_k e_k$ converges in $H$, we denote its limit by
\[
\sum_{k=1}^\infty a_k e_k.
\]
This is a {\em definition} of an infinite sum in $H$. Since $H$ is complete, one usually shows that a sequence converges by showing that the sequence is Cauchy, and hence
to show that $\sum_{k=1}^n a_k e_k$ converges it is equivalent to show that
\[
\sum_{k=m+1}^n a_k e_k \to 0
\]
as $m,n \to \infty$. And showing this is equivalent to showing that
\[
\inner{\sum_{k=m+1}^n a_k e_k}{\sum_{k=m+1}^n a_k e_k} \to 0
\] 
as $m, n \to \infty$. This is equivalent to
\[
\sum_{k=m+1}^n |a_k|^2 \to 0
\]
as $m,n \to \infty$, and this is equivalent to the series
\[
\sum_{k=1}^\infty |a_k|^2
\]
converging. Thus, the series $\sum_{k=1}^\infty a_k e_k$ converges if and only if the series $\sum_{k=1}^\infty |a_k|^2$ converges.\footnote{Furthermore, using the triangle
inequality rather than the orthonormality of the $e_k$, one can check that if
the series $\sum_{k=1}^\infty |a_k|$ converges then the series $\sum_{k=1}^\infty a_k e_k$ converges.}

Let $\{e_i:i \geq 1\}$ be an orthonormal basis for $H$; it is a fact that one exists. 
Let $v \in H$ and define
\[
s_n=\sum_{k=1}^n \inner{v}{e_k}e_k.
\]
If $1 \leq i \leq n$ then
\[
\inner{v-s_n}{e_i}=\inner{v}{e_i} - \sum_{k=1}^n \inner{v}{e_k}\inner{e_k}{e_i}
=\inner{v}{e_i}-\inner{v}{e_i}=0,
\]
hence
\[
\inner{v-s_n}{s_n}=0.
\]
It follows that
\begin{eqnarray*}
\sum_{k=1}^n |\inner{v}{e_k}|^2&=&\inner{s_n}{s_n}\\
&\leq&\inner{s_n}{s_n}+\inner{v-s_n}{v-s_n}\\
&=&\inner{v}{v},
\end{eqnarray*}
where we used $\inner{v-s_n}{s_n}=0$ in the third line.
Therefore the series $\sum_{k=1}^\infty |\inner{v}{e_k}|^2$ converges, and so the sequence $s_n$ converges to some
$v'=\sum_{k=1}^\infty \inner{v}{e_k}e_k \in H$. Since $s_n$ converges to $v'$, in particular it converges weakly to $v'$, i.e., for
any $w \in H$,
\[
\lim_{n \to \infty} \inner{s_n}{w}=\inner{v'}{w}.
\]
Therefore for any $j$,
\[
\inner{v-v'}{e_j}=\inner{v}{e_j}-\inner{v'}{e_j}=\inner{v}{v_j}-\lim_{n \to \infty} \inner{s_n}{e_j}
=\inner{v}{v_j}-\inner{v}{v_j}=0;
\]
this is because for $n \geq j$ we have $\inner{v-s_n}{e_j}=0$ and hence $\inner{v}{e_j}=\inner{s_n}{e_j}$.
As $\inner{v-v'}{e_j}=0$ for all $j$, it follows that $v-v'=0$, i.e. $v=v'$. Hence,
\[
v = \sum_{k=1}^\infty \inner{v}{e_k}e_k.
\]
We call this an {\em abstract Fourier series} for $v$.\footnote{If $H=L^2(\mathbb{T})$,
one checks that
$e_k=e^{ik}, k \in \mathbb{Z}$, is an orthonormal basis for $H$.
Then,
\[
\inner{f}{e_k}=\frac{1}{2\pi} \int_0^{2\pi} f(t) e^{-ik} dt
\]
and $f$ is the limit in $H$ of $\sum_{k=0}^n \inner{f}{e_k} e^{ik}$. Thus in $H$,
\[
f=\sum_{k \in \mathbb{Z}} \inner{f}{e_k} e^{ik}.
\]
}
It can be written as
\[
v= \sum_{k=1}^\infty (e_k \otimes e_k)v,
\]
and thus can be written without $v$ as
\[
\id_H = \sum_{k=1}^\infty e_k \otimes e_k;
\]
$e_k \otimes e_k \in B(H)$ is a projection with rank 1, and the above series conveges in the {\em strong operator topology} on $B(H)$.
Writing the identity map in this way is called a {\em resolution of the identity}.

\section{Parseval's identity}
On the one hand
\[
\lim_{n \to \infty} \norm{s_n}^2 = \sum_{k=1}^\infty |\inner{v}{e_k}|^2.
\]
On the other hand,
\[
\lim_{n \to \infty} \norm{s_n}^2 = \norm{v}^2.
\]
Hence
\[
\norm{v}^2 = \sum_{k=1}^\infty |\inner{v}{e_k}|^2,
\]
which is {\em Parseval's identity}. 
\end{document}