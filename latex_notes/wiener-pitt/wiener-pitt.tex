\documentclass{article}
\usepackage{amsmath,amssymb,mathrsfs,amsthm,leftidx}
%\usepackage{tikz-cd}
%\usepackage{hyperref}
\newcommand{\inner}[2]{\left\langle #1, #2 \right\rangle}
\newcommand{\tr}{\ensuremath\mathrm{tr}\,} 
\newcommand{\Span}{\ensuremath\mathrm{span}} 
\def\Re{\ensuremath{\mathrm{Re}}\,}
\def\Im{\ensuremath{\mathrm{Im}}\,}
\newcommand{\id}{\ensuremath\mathrm{id}} 
\newcommand{\ev}{\ensuremath\mathrm{ev}} 
\newcommand{\var}{\ensuremath\mathrm{var}} 
\newcommand{\Lip}{\ensuremath\mathrm{Lip}} 
\newcommand{\GL}{\ensuremath\mathrm{GL}} 
\newcommand{\diam}{\ensuremath\mathrm{diam}} 
\newcommand{\sgn}{\ensuremath\mathrm{sgn}\,} 
\newcommand{\lcm}{\ensuremath\mathrm{lcm}} 
\newcommand{\supp}{\ensuremath\mathrm{supp}\,}
\newcommand{\dom}{\ensuremath\mathrm{dom}\,}
\newcommand{\upto}{\nearrow}
\newcommand{\downto}{\searrow}
\newcommand{\norm}[1]{\left\Vert #1 \right\Vert}
\newcommand{\supnorm}[1]{\left\Vert #1 \right\Vert_{\mathrm{sup}}}
\newtheorem{theorem}{Theorem}
\newtheorem{lemma}[theorem]{Lemma}
\newtheorem{proposition}[theorem]{Proposition}
\newtheorem{corollary}[theorem]{Corollary}
\theoremstyle{definition}
\newtheorem{definition}[theorem]{Definition}
\newtheorem{example}[theorem]{Example}
\begin{document}
\title{The Wiener-Pitt tauberian theorem}
\author{Jordan Bell\\ \texttt{jordan.bell@gmail.com}\\Department of Mathematics, University of Toronto}
\date{\today}

\maketitle

\section{Introduction}
For $f \in L^1(\mathbb{R}^d)$, we write
\[
\hat{f}(\xi) = \int_{\mathbb{R}^d} f(x) e^{-2\pi i\xi\cdot x} dx, \qquad \xi \in \mathbb{R}^d.
\]
The Riemann-Lebesgue lemma tells us that $\hat{f} \in C_0(\mathbb{R}^d)$. 

For $f \in C^\infty(\mathbb{R}^d)$ and for multi-indices $\alpha,\beta$, write
\[
|f|_{\alpha,\beta} = \sup_{x \in \mathbb{R}^d} |x^\alpha (\partial^\beta f)(x)|.
\]
We say that $f$ is a \textbf{Schwartz function} if  for all multi-indices $\alpha$ and $\beta$ we have
$|f|_{\alpha,\beta}<\infty$. We denote by $\mathscr{S}$ the collection of Schwartz functions. It is a fact that 
$\mathscr{S}$
with this family of seminorms is a Fr\'echet space. 

Let $V_d = \frac{\pi^{d/2}}{\Gamma\left( \frac{d}{2}+1 \right)}$, the volume of the unit ball
in $\mathbb{R}^d$. 

\begin{lemma}
For $1 \leq p \leq \infty$, let
$m$ be the least integer $\geq \frac{d+1}{p}$.
There is some $C_d$ such that for any multi-index $\beta$,
\[
\norm{\partial^\beta f}_p \leq V_d^{1/p} |f|_{0,\beta} + C_d V_d^{1/p} \sum_{|\alpha|=m} |f|_{\alpha,\beta}, \qquad
f \in \mathscr{S}.
\]
\label{schwartzLp}
\end{lemma}
\begin{proof}
For $p=\infty$, the claim is true with $C_{d,\infty}=1$. 
For $1 \leq p < \infty$, let 
$g = \partial^\beta f$, which satisfies
\begin{align*}
\norm{g}_p&=\left( \int_{|x| \leq 1} |g(x)|^p dx + \int_{|x| \geq 1} |x|^{d+1} |g(x)|^p |x|^{-(d+1)} dx \right)^{1/p}\\
&\leq \left( \norm{g}_\infty^p V_d + \sup_{|x| \geq 1} \left( |x|^{d+1}|g(x)|^p \right) \int_{|x| \geq 1}
|x|^{-(d+1)} dx \right)^{1/p}\\
&=\left( \norm{g}_\infty^p V_d + \sup_{|x| \geq 1} \left( |x|^{d+1}|g(x)|^p \right) \int_1^\infty\left( \int_{S^{d-1}} |r\gamma|^{-(d+1)}
d\sigma(\gamma) \right) r^{d-1} dr \right)^{1/p}\\
&=\left( \norm{g}_\infty^p V_d + \sup_{|x| \geq 1} \left( |x|^{d+1}|g(x)|^p \right) \cdot V_d \int_1^\infty r^{-2} dr \right)^{1/p}\\
&=V_d^{1/p} \left( \norm{g}_\infty^p + \sup_{|x| \geq 1}  \left( |x|^{d+1}|g(x)|^p \right) \right)^{1/p}\\
&\leq V_d^{1/p} \norm{g}_\infty + V_d^{1/p}  \sup_{|x| \geq 1}  \left( |x|^{\frac{d+1}{p}}|g(x)| \right)\\
&\leq V_d^{1/p} \norm{g}_\infty + V_d^{1/p} \sup_{|x| \geq 1} \left( |x|^m |g(x)| \right).
\end{align*}
Using that the function $y \mapsto \sum_{|\alpha|=m} |y^\alpha|$ is continuous $S^{d-1} \to \mathbb{R}$,
there is some $C_d$ such that
\[
|x|^m \leq C_d \sum_{|\alpha|=m} |x^\alpha|, \qquad x \in \mathbb{R}^d.
\]
This gives us
\begin{align*}
\norm{g}_p& \leq V_d^{1/p} \norm{g}_\infty + V_d^{1/p} \sup_{|x| \geq 1} C_d  \sum_{|\alpha|=m} |x^\alpha| |g(x)|\\
&= V_d^{1/p} \norm{\partial^\beta f}_\infty  + C_d V_d^{1/p} \sum_{|\alpha|=m} \sup_{|x| \geq 1} |x^\alpha (\partial^\beta f)(x)|\\
&\leq V_d^{1/p} |f|_{0,\beta} + C_d V_d^{1/p} \sum_{|\alpha|=m} |f|_{\alpha,\beta}.
\end{align*}
\end{proof}

The dual space $\mathscr{S}'$ with the weak-*
topology is  a locally convex space, elements of which are called \textbf{tempered distributions}.
It is straightforward to check that if $u:\mathscr{S} \to \mathbb{C}$ is linear, then $u \in \mathscr{S}'$ if and only if
there is some  $C$ and some nonnegative integers $m,n$ such that
\[
|u(f)| \leq C \sum_{|\alpha| \leq m, |\beta| \leq n} |f|_{\alpha,\beta}, \qquad f \in \mathscr{S}.
\]
For $1 \leq p \leq \infty$ and $g \in L^p(\mathbb{R}^d)$,
define $u:\mathscr{S} \to \mathbb{C}$ by
\[
u(f) = \int_{\mathbb{R}^d} f(x) g(x) dx, \qquad f \in \mathscr{S}.
\]
For $\frac{1}{p}+\frac{1}{q}=1$, H\"older's inequality tells us
\[
|u(f)| \leq \norm{fg}_1 \leq \norm{g}_p \norm{f}_q.
\]
By Lemma \ref{schwartzLp}, with $m$ the least integer $\geq \frac{d+1}{q}$,
\[
\norm{f}_q \leq V_d^{1/q} |f|_{0,0} + C_d V_d^{1/q} \sum_{|\alpha|=m} |f_{\alpha,0}|.
\]
Therefore, 
\[
|u(f)| \leq C_{g,d,q} \sum_{|\alpha| \leq m, |\beta| \leq 0} |f|_{\alpha,\beta},
\]
showing that $u$ is continuous. 
We thus speak of elements of $L^p(\mathbb{R}^d)$ as tempered distributions, and
speak about the Fourier transform of an element of  $L^p(\mathbb{R}^d)$.

Let $u \in \mathscr{D}'$ be a distribution. For an open set $\omega$,
we say that \textbf{$u$ vanishes on $\omega$} if 
$u(\phi)=0$ for every $\phi \in \mathscr{D}(\omega)$. 
Let $\Gamma$ be the collection of open sets $\omega$ on which $u$ vanishes, and let
$\Omega = \bigcup_{\omega \in \Gamma} \omega$. 
$\Gamma$ is an open cover of $\Omega$, and
 thus there is a locally finite partition of unity $\psi_j$ subordinate to $\Gamma$.\footnote{Walter Rudin, {\em Functional
 Analysis}, second ed., p.~162, Theorem 6.20.}
For $\phi \in \mathscr{D}(\Omega)$, because 
$\supp \phi$ is compact,
there is some open set $W$, $\supp \phi \subset W \subset \Omega$, and some $m$ such that
\[
\psi_1(x)+\cdots+\psi_m(x)=1, \qquad x \in W.
\]
Then 
\[
u(\phi) = u(\phi(\psi_1+\cdots+\psi_m))
=u(\psi_1 \phi)+\cdots+u(\psi_m \phi).
\]
For each $j$, $1 \leq j \leq m$, there is some $\omega_j \in \Gamma$ such that $\supp \psi_j \subset \omega_j$,
which implies $\supp \psi_j \phi \subset \omega_j$, i.e.
$\psi_j \phi \in \mathscr{D}(\omega_j)$. But $\omega_j \in \Gamma$, so
$u(\psi_j \phi)=0$ and hence $u(\phi)=0$. This shows that $\Omega \in \Gamma$, namely, $\Omega$ is the largest
open set on which $u$ vanishes.
The \textbf{support of $u$} is
\[
\supp u = \mathbb{R}^d \setminus \Omega.
\]

For $u \in \mathscr{S}'$ we define $\hat{u}:\mathscr{S} \to \mathbb{C}$ by
\[
\hat{u}(\phi) = u(\hat{\phi}), \qquad \phi \in \mathscr{S}.
\]
It is a fact that $\hat{u} \in \mathscr{S}'$. 

For $f:\mathbb{R}^d \to \mathbb{C}$, write $\check{f}(x)=f(-x)$. For $\phi \in \mathscr{S}$,
\[
\mathscr{F}(\mathscr{F}(\phi))=\check{\phi}.
\]


\section{Tauberian theory}
\begin{lemma}
If $f \in L^1(\mathbb{R}^d)$, $\zeta \in \mathbb{R}^d$, and $\epsilon>0$, then there is some $h \in L^1(\mathbb{R}^d)$ with
$\norm{h}_1 < \epsilon$ and some $r>0$ such that 
\[
\hat{h}(\xi) = \hat{f}(\zeta)-\hat{f}(\xi), \qquad \xi \in B_r(\zeta).
\]
\label{nbdlemma}
\end{lemma}
\begin{proof}
It is a fact that there is a Schwartz function $g$ such that $\hat{g}(\xi)=1$ for $|\xi|<1$. 
For $\lambda>0$, let
\[
g_\lambda(x) = e^{2\pi i\zeta \cdot x} \lambda^{-d}g(\lambda^{-1}x), \qquad x \in \mathbb{R}^d,
\]
which satisfies, for $\xi \in \mathbb{R}^d$,
\begin{align*}
\widehat{g_\lambda}(\xi) &= \int_{\mathbb{R}^d} e^{-2\pi i\xi \cdot x} e^{2\pi i\zeta \cdot x} \lambda^{-d}g(\lambda^{-1}x) dx\\
&=\int_{\mathbb{R}^d} e^{-2\pi i \lambda \xi \cdot y} e^{2\pi i\lambda \zeta \cdot y}  g(y) dy\\
&=\hat{g}(\lambda \xi-\lambda \zeta).
\end{align*}
In particular, for $\xi \in V_\lambda = B_{\lambda^{-1}}(\zeta)$ we have
$\widehat{g_\lambda}(\xi) = 1$. 
We also define
\[
h_\lambda(x) = \hat{f}(\zeta) g_\lambda(x) - (f*g_\lambda)(x), \qquad
x \in \mathbb{R}^d,
\]
which satisfies, for $\xi \in \mathbb{R}^d$,
\[
\widehat{h_\lambda}(\xi)=\hat{f}(\zeta) \widehat{g_\lambda}(\xi)
-\widehat{f*g_\lambda}(\xi)\\
=\hat{f}(\zeta) \widehat{g_\lambda}(\xi) - \hat{f}(\xi) \widehat{g_\lambda}(\xi)
=\widehat{g_\lambda}(\xi)(\hat{f}(\zeta)-\hat{f}(\xi)).
\]
Hence, for $\xi \in V_\lambda$ we have $\widehat{h_\lambda}(\xi) = \hat{f}(\zeta)-\hat{f}(\xi)$.

For $x \in \mathbb{R}^d$,
\begin{align*}
h_\lambda(x)&=\int_{\mathbb{R}^d} f(y) e^{-2\pi i\zeta \cdot y} g_\lambda(x)
-\int_{\mathbb{R}^d} f(y) g_\lambda(x-y) dy\\
&=\int_{\mathbb{R}^d} f(y) \left( e^{-2\pi i\zeta \cdot y} g_\lambda(x)-g_\lambda(x-y) \right) dy,
\end{align*}
for which
\[
\begin{split}
&\left| e^{-2\pi i\zeta \cdot y} g_\lambda(x)-g_\lambda(x-y) \right|\\
=&\left| e^{-2\pi i\zeta \cdot y} e^{2\pi i\zeta \cdot x}
\lambda^{-d} g(\lambda^{-1} x) - e^{2\pi i\zeta\cdot (x-y)} \lambda^{-d} g(\lambda^{-1}(x-y)) \right|\\
=&\lambda^{-d} | g(\lambda^{-1}x)-g(\lambda^{-1}(x-y))|.
\end{split}
\]
Then
\begin{align*}
\norm{h_\lambda}_1&\leq \int_{\mathbb{R}^d} \left(   \int_{\mathbb{R}^d} |f(y)| \lambda^{-d} | g(\lambda^{-1}x)-g(\lambda^{-1}(x-y))|
dy \right) dx\\
&=\int_{\mathbb{R}^d} \left(  \int_{\mathbb{R}^d} |f(y)| |g(u)-g(\lambda^{-1}(\lambda u-y))| dy \right) du\\
&=\int_{\mathbb{R}^d} |f(y)| \left(  \int_{\mathbb{R}^d} |g(u)-g(u-\lambda^{-1} y))|  du \right) dy.
\end{align*}
For each $y \in \mathbb{R}^d$, 
\[
|f(y)| \left(  \int_{\mathbb{R}^d} |g(u)-g(u-\lambda^{-1} y))|  du \right) \leq 2 \norm{g}_1 |f(y),
\]
and hence by the dominated convergence theorem,
\[
\int_{\mathbb{R}^d} |f(y)| \left(  \int_{\mathbb{R}^d} |g(u)-g(u-\lambda^{-1} y))|  du \right) dy \to
0, \qquad \lambda \to \infty.
\]
Thus, there is some $\lambda_\epsilon$ such that $\norm{h_\lambda}_1 < \epsilon$ when
$\lambda \geq \lambda_\epsilon$.
For $h=h_{\lambda_\epsilon}$ and $r=\lambda_\epsilon^{-1}$, we have
$\hat{h}(\xi) = \hat{f}(\zeta)-\hat{f}(\xi)$ for $\xi \in V_{\lambda_\epsilon} = B_r(\zeta)$ and 
$\norm{h}_1<\epsilon$, proving the claim.
\end{proof}

We remind ourselves that for $\phi \in L^\infty(\mathbb{R}^d)$ and $f \in L^1(\mathbb{R}^d)$, the convolution
$f*\phi$ belongs to $C_u(\mathbb{R}^d)$, the collection of bounded uniformly continuous functions $\mathbb{R}^d \to \mathbb{C}$.
We also remind ourselves that any element of $L^\infty(\mathbb{R}^d)$  is a tempered distribution whose  Fourier
transform is a tempered distribution.\footnote{Walter Rudin, {\em Functional Analysis}, second ed.,
p.~228, Theorem 9.3.}


\begin{theorem}
If $\phi \in L^\infty(\mathbb{R}^d)$, $Y$ is a linear subspace of $L^1(\mathbb{R}^d)$, and 
\[
f*\phi = 0, \qquad f \in Y,
\]
then 
\[
Z(Y) =  \bigcap_{f \in Y} \{\xi \in \mathbb{R}^d: \hat{f}(\xi)=0\}
\]
contains $\supp \hat{\phi}$. 
\label{Ytheorem}
\end{theorem}
\begin{proof}
If $Y=\{0\}$, then  $Z(Y)=\mathbb{R}^d$, and the claim is true. If $Y$ has nonzero dimension, 
let $\zeta \in  \mathbb{R}^d \setminus Z(Y)$ and let  
$f \in Y$ such that $\hat{f}(\zeta)=1$; that there is such a function $f$ follows from $Y$ being a linear space.
Thus by Lemma \ref{nbdlemma} there is some $h \in L^1(\mathbb{R}^d)$ with
$\norm{h}_1 < 1$ and some $r>0$ such that 
\[
\hat{h}(\xi) =1-\hat{f}(\xi), \qquad \xi \in B_r(\zeta);
\]
because $Z(Y)$ is closed, we may take $r$ such that $B_r(\zeta) \subset \mathbb{R}^d \setminus Z(Y)$. 

Let $\rho \in \mathscr{D}(B_r(\zeta))$, and let $\psi \in \mathscr{S}$ with
$\hat{\psi}=\rho$. 
Define $g_0=\psi$ and $g_m = h * g_{m-1}$ for $m \geq 1$.
By Young's inequality
\[
\norm{g_m}_1 \leq  \norm{h}_1^m \norm{\psi}_1,
\]
and because $\norm{h}_1<1$, this means that the sequence $\sum_{m=0}^M g_m$ is Cauchy in $L^1(\mathbb{R}^d)$ so
converges to some $G$, for which, as $|\hat{h}| \leq \norm{h}_1 < 1$,
\[
\hat{G} = \sum_{m=0}^\infty \widehat{g_m} = \sum_{m=0}^\infty \hat{\psi}\cdot \hat{h}^m
=\hat{\psi} \cdot (1-\hat{h})^{-1}.
\]
For $\xi \in \supp \hat{\psi} \subset B_r(\zeta)$ we have
$\hat{h}(\xi)=1-\hat{f}(\xi)$ and so
\[
\hat{\psi}(\xi) = \hat{G}(\xi) (1-\hat{h}(\xi)) = \hat{G}(\xi)\hat{f}(\xi);
\]
on the other hand, for $\xi \not \in \supp \hat{\psi}$, $\hat{\psi}(\xi)=0=\hat{G}(\xi)\hat{f}(\xi)$, so
\[
\hat{\psi} = \hat{G} \cdot \hat{f},
\]
which implies that $\psi = G*f$. Then
\[
\psi*\phi = G*f*\phi = G*0 = 0,
\]
therefore
\[
\hat{\phi}(\rho) = \phi(\hat{\rho}) =\phi(\mathscr{F}^2(\psi))= \phi(\check{\psi}) = \int_{\mathbb{R}^d} \psi(-x) \phi(x) dx
=(\psi*\phi)(0)=0.
\]
This is true for all $\rho \in \mathscr{D}(B_r(\zeta))$, which 
means that $\hat{\phi}$ vanishes on $B_r(\zeta)$. This is true for any $\zeta \in \mathbb{R}^d \setminus Z(Y)$, 
so with $\Omega$ the union of those open sets on which $\hat{\phi}$ vanishes, 
$\mathbb{R}^d \setminus Z(Y) \subset \Omega$. Then
$Z(Y) \subset \mathbb{R}^d \setminus \Omega = \supp \hat{\phi}$. 
\end{proof}

If $X$ is a Banach space and $M$ is a linear subspace of $X$, we define the \textbf{annihilator of $M$}
as
\[
M^\perp = \{\gamma \in X^*: \textrm{if $x \in M$ then $\inner{x}{\gamma}=0$}\}.
\]
It is immediate that $M^\perp$ is a weak-* closed linear subspace of $X^*$. 
If $N$ is a linear subspace of $X^*$, we define the \textbf{annihilator of $N$} as
\[
^\perp N = \{x \in X: \textrm{if $\gamma \in N$ then $\inner{x}{\gamma}=0$}\}.
\]
It is immediate that $^\perp N$ is a norm closed linear subspace of the Banach space $X$. One proves using the Hahn-Banach theorem that
$^\perp(M^\perp)$ is the norm closure of $M$ in $X$.\footnote{Walter Rudin, {\em Functional Analysis}, second ed.,
p.~96, Theorem 4.7.}

We say that a subspace $Y$ of $L^1(\mathbb{R}^d)$ is \textbf{translation-invariant} if $f \in Y$ and $x \in \mathbb{R}^d$ imply
that $f_x \in Y$, where $f_x(y)=f(y-x)$. The following theorem gives conditions under which a closed translation-invariant
subspace of $L^1(\mathbb{R}^d)$ is equal to the entire space.\footnote{Walter Rudin, {\em Functional Analysis}, second ed.,
p.~228, Theorem 9.4.}

\begin{theorem}
If $Y$ is a closed translation-invariant subspace of $L^1(\mathbb{R}^d)$ and $Z(Y)=\emptyset$, then
$Y=L^1(\mathbb{R}^d)$. 
\label{translationinvariant}
\end{theorem}
\begin{proof}
Suppose that $\phi \in L^\infty(\mathbb{R}^d)$ and $\int f \check{\phi}=0$ for each $f \in Y$. 
Let $f \in Y$ and $x \in \mathbb{R}^d$.
As $Y$ is translation-invariant, $f_{-x} \in Y$ so $\int_{\mathbb{R}^d} f(y+x) \phi(-y) dy=0$, i.e.
$(f*\phi)(x)=0$. This is true for all $x \in \mathbb{R}^d$, which means that $f*\phi = 0$. 
Theorem \ref{Ytheorem} then tells us that $\supp \hat{\phi}$ is contained in
$Z(Y)$, namely, $\supp \hat{\phi}$ is empty, which  means that the tempered distribution $\hat{\phi}$ vanishes
on $\mathbb{R}^d$, i.e. $\supp \hat{\phi}$ is the zero element of the locally convex space $\mathscr{S}'$. 
As the Fourier transform $\mathscr{S}' \to \mathscr{S}'$ is linear and one-to-one, 
the tempered distribution $\phi$ is the zero element of $\mathscr{S}'$, which implies that $\phi \in
L^\infty(\mathbb{R}^d)$ is zero.
As Lebesgue measure on $\mathbb{R}^d$ is $\sigma$-finite, 
for $X$ the Banach space $L^1(\mathbb{R}^d)$ we have
$X^*=L^\infty(\mathbb{R}^d)$, with $\inner{f}{\gamma}=\int f \gamma$. Thus $Y^\perp$ is the zero subspace of $L^\infty(\mathbb{R}^d)$, hence
$^\perp (Y^\perp)=L^1(\mathbb{R}^d)$. This implies that $L^1(\mathbb{R}^d)$ is equal to the closure of $Y$ in $L^1(\mathbb{R}^d)$, and because
$Y$ is closed this means $Y=L^1(\mathbb{R}^d)$, completing the proof.
\end{proof}


\begin{theorem}
Suppose that $K \in L^1(\mathbb{R}^d)$ and that $Y$ is the smallest closed translation-invariant  subspace of $L^1(\mathbb{R}^d)$ that includes $K$. $Y=L^1(\mathbb{R}^d)$ if and only
if 
\[
\hat{K}(\xi) \neq 0, \qquad \xi \in \mathbb{R}^d.
\]
\label{intersection}
\end{theorem}
\begin{proof}
Suppose that $\hat{K}(\xi) \neq 0$ for all $\xi \in \mathbb{R}^d$. As $K \in Y$, this implies that $Z(Y) = \emptyset$. Thus by
 Theorem \ref{translationinvariant} we get  $Y=L^1(\mathbb{R}^d)$.
 
 Suppose that $Y=L^1(\mathbb{R}^d)$. Then $f(x)=e^{-\pi |x|^2}$ belongs to $Y$ and $\hat{f}(\xi)=e^{-\pi |\xi|^2}$, which has no zeros,
 hence $Z(Y)=\emptyset$. For $\xi \in \mathbb{R}^d$, define $\ev_\xi:C_0(\mathbb{R}^d) \to \mathbb{C}$ by $\ev_\xi(g)=g(\xi)$, which
 is a bounded linear operator. The Fourier transform $\mathscr{F}:L^1(\mathbb{R}^d) \to C_0(\mathbb{R}^d)$ is a bounded linear operator, hence
 for each $\xi \in \mathbb{R}^d$, $\ev_\xi \circ \mathscr{F}:L^1(\mathbb{R}^d) \to \mathbb{C}$ is a bounded linear operator.
Hence
\[
V_\xi = \{f \in L^1(\mathbb{R}^d): \hat{f}(\xi)=0\} = \ker (\ev_\xi \circ \mathscr{F})
\]
 is a closed subspace of $L^1(\mathbb{R}^d)$. 
 If $f \in V$ and $x \in \mathbb{R}^d$, then
 \[
 \widehat{f_x}(\xi)=\int_{\mathbb{R}^d} f(y-x) e^{-2\pi i\xi\cdot y} dy = 
 e^{-2\pi i\xi \cdot x} \hat{f}(\xi)=0,
 \]
 showing that $V_\xi$ is translation-invariant.  
 Therefore
 \[
 V = \bigcap_{\hat{K}(\xi)=0} V_\xi
 \]
 is a closed translation-invariant subspace of $L^1(\mathbb{R}^d)$, and because $Y$ is the smallest closed translation-invariant 
 subspace of $L^1(\mathbb{R}^d)$, $Y \subset V$. $Y \subset V$ implies $Z(V) \subset Z(Y)=\emptyset$,
 and applying Theorem \ref{translationinvariant} we get that $V=L^1(\mathbb{R}^d)$. But  there
 is no $\xi$ for which $V_\xi = L^1(\mathbb{R}^d)$, so  $V=L^1(\mathbb{R}^d)$ implies that
 $\{\xi \in \mathbb{R}^d: \hat{K}(\xi)=0\}=\emptyset$. 
 \end{proof}


\section{Slowly oscillating functions}
Let $B(\mathbb{R}^d)$ be the collection of bounded functions $\mathbb{R}^d \to \mathbb{C}$, which with the supremum norm 
$\norm{f}_u = \sup_{x \in \mathbb{R}^d} |f(x)|$ is a Banach algebra.

A function $\phi \in B(\mathbb{R}^d)$ is said to be \textbf{slowly oscillating} if for each $\epsilon>0$ there is some $A$ and some $\delta>0$ such that
if $|x|,|y|>A$ and $|x-y|<\delta$, then $|\phi(x)-\phi(y)|<\epsilon$. We now prove the \textbf{Wiener-Pitt tauberian theorem}; the statement 
supposing that a function is slowly oscillating is attributed to Pitt.\footnote{Walter Rudin, {\em Functional Analysis}, second ed.,
p.~229, Theorem 9.7; Walter Rudin, {\em Fourier Analysis on Groups}, p.~163, Theorem 7.2.7; Gerald B. Folland, {\em A Course in Abstract
Harmonic Analysis}, p.~116, Theorem 4.72; V. P. Havin and N. K. Nikolski,
{\em Commutative Harmonic Analysis II}, p.~134; Edwin Hewitt and Kenneth
A. Ross, {\em Abstract Harmonic Analysis II}, p.~511, Theorem 39.37.}

\begin{theorem}[Wiener-Pitt tauberian theorem]
If $\phi \in B(\mathbb{R}^d)$, $K \in L^1(\mathbb{R}^d)$, $\hat{K}(\xi) \neq 0$ for all $\xi \in \mathbb{R}^d$, and
\[
\lim_{|x| \to \infty} (K*\phi)(x)=a \hat{K}(0),
\]
then  for each $f \in L^1(\mathbb{R}^d)$,
\begin{equation}
\lim_{|x| \to \infty} (f*\phi)(x) = a\hat{f}(0).
\label{claim1}
\end{equation}
Furthermore, if such $\phi$ is slowly oscillating then
\begin{equation}
\lim_{|x| \to \infty} \phi(x) = a.
\label{claim2}
\end{equation}
\end{theorem}
\begin{proof}
Define $\psi(x)=\phi(x)-a$. Let $Y$ be the set of those $f \in L^1(\mathbb{R}^d)$ for which
\[
\lim_{|x| \to \infty} (f*\psi)(x)=0.
\]
It is immediate that $Y$ is a linear subspace of $L^1(\mathbb{R}^d)$. Suppose that $f_i \in Y$ tends to some $f \in L^1(\mathbb{R}^d)$. As $\psi \in B(\mathbb{R}^d)$,
$f*\psi$ and $f_i*\psi$ belong to $C_u(\mathbb{R}^d)$. 
Then
\[
\norm{f*\psi-f_i*\psi}_u = \norm{(f-f_i)*\psi}_u = \norm{\psi}_u \norm{f-f_i}_1.
\]
There is some $i_0$ such that $i \geq i_0$ implies $\norm{f-f_i}_1<\epsilon$, and because $f_{i_0} \in Y$ there is some $M$ such that
$|x| \geq M$ implies $|(f_{i_0}*\psi)(x)| < \epsilon$. Then for $|x| \geq M$,
\begin{align*}
|(f*\psi)(x)| & \leq |(f*\psi)(x)-(f_{i_0}*\psi)(x)|+|(f_{i_0}*\psi)(x)|\\
&\leq  \norm{\psi}_u \norm{f-f_i}_1 +|(f_{i_0}*\psi)(x)|\\
&<\epsilon \cdot (\norm{\psi}_u+1),
\end{align*}
showing that $f \in Y$, namely, that $Y$ is closed. 
Let $f \in Y$ and $x \in \mathbb{R}^d$. $f_x \in L^1(\mathbb{R}^d)$, and for $y \in \mathbb{R}^d$,
\[
((\tau_x f)*\psi)(y) = (f*\psi)(y-x),
\]
and as $|y| \to \infty$ we have $|y-x| \to \infty$ and thus $(f*\psi)(y-x) \to 0$, hence $\tau_x f \in Y$, i.e. $Y$ is translation-invariant.
Therefore $Y$ is a closed translation-invariant subspace of $L^1(\mathbb{R}^d)$. 
For $x \in \mathbb{R}^d$,
\[
(K*\psi)(x) = \int_{\mathbb{R}^d} K(y)(\phi(x-y)-a) dy = 
(K*\phi)(x)-a\hat{K}(0),
\]
and by hypothesis we get $(K*\psi)(x) \to 0$ as $|x| \to \infty$, i.e. $K \in Y$. 

Let $Y_0$ be the smallest closed translation-invariant subspace of $L^1(\mathbb{R}^d)$ that includes $K$.
On the one hand, because $Y$ is a closed translation-invariant subspace of $L^1(\mathbb{R}^d)$ and $K \in Y$ we have
$Y_0 \subset Y$. On the other hand, because $\hat{K}(\xi) \neq 0$ for all $\xi$ we have by 
Theorem \ref{intersection} 
that $Y_0 = L^1(\mathbb{R}^d)$. Therefore $Y=L^1(\mathbb{R}^d)$. This means that for each $f \in L^1(\mathbb{R}^d)$,
$(f*\psi)(x) \to 0$ as $|x| \to \infty$, i.e. $(f*\phi)(x) \to a\hat{f}(0)$ as $|x| \to \infty$, proving \eqref{claim1}

Assume further now that $\phi$ is slowly-oscillating and let $\epsilon>0$. There is some $A$ and some $\delta>0$ such that
if $|x|,|y|>A$ and $|x-y|<\delta$ then
\[
|\phi(x)-\phi(y)|<\epsilon.
\]
There is a test function $h$ such that $h \geq 0$, $h(x)=0$ for $|x| \geq \delta$, and $\hat{h}(0)=1$.  By \eqref{claim1}, 
\[
\lim_{|x| \to \infty} (h*\phi)(x) = a \hat{h}(0) = a.
\]
On the other hand, for $x \in \mathbb{R}^d$,
\begin{align*}
\phi(x)-(h*\phi)(x)&= \hat{h}(0) \phi(x) - (h*\phi)(x) \\
& = \int_{\mathbb{R}^d}  (h(y) \phi(x) -\phi(x-y) h(y)) dy\\
&=\int_{|y|<\delta} (\phi(x)-\phi(x-y)) h(y) dy,
\end{align*}
and so for $|x|>A+\delta$,
\[
|\phi(x)-(h*\phi)(x)| \leq \int_{|y| < \delta} \epsilon \cdot  |h(y)| dy= \epsilon \int_{\mathbb{R}^d} h(y) dy = \epsilon \hat{h}(0) = \epsilon.
\]
We have thus established that as $|x| \to \infty$, (i) $(h*\phi)(x) = a +o(1)$ and (ii) $\phi(x)=(h*\phi)(x)+o(1)$, which together yield
$\phi(x)=a+o(1)$, i.e. $\phi(x) \to a$ as $|x| \to \infty$, proving \eqref{claim2}.
\end{proof}


\section{Closed ideals in $L^1(\mathbb{R}^d)$}
$L^1(\mathbb{R}^d)$ is a Banach algebra using convolution as the product.\footnote{Eberhard Kaniuth, {\em A Course in Commutative Banach Algebras},
p.~25, Proposition 1.4.7.}

\begin{theorem}
Suppose that $I$ is a closed linear subspace of $L^1(\mathbb{R}^d)$. $I$ is translation-invariant if and only if $I$ is an ideal.
\end{theorem}
\begin{proof}
Assume that $I$ is translation-invariant and let $f \in I$ and $g \in L^1(\mathbb{R}^d)$.
For $\phi \in I^\perp \subset L^\infty(\mathbb{R}^d)$,
\begin{align*}
\inner{g*f}{\phi}&=\int_{\mathbb{R}^d} (g*f)(x) \phi(x) dx\\
&=\int_{\mathbb{R}^d} \phi(x) \left( \int_{\mathbb{R}^d} g(x-y) f(y) dy \right) dx\\
&=\int_{\mathbb{R}^d} g(z) \left( \int_{\mathbb{R}^d} \phi(x) f_z(x) dx \right) dz\\
&=\int_{\mathbb{R}^d} g(z) \inner{\phi}{f_z} dz\\
&=0,
\end{align*}
because $f_z \in I$ for each $z \in \mathbb{R}^d$. This shows that $f*g \in ^\perp(I^\perp)$.
But $^\perp(I^\perp)$ is the closure of $I$ in $L^1(\mathbb{R}^d)$,\footnote{Walter
Rudin, {\em Functional Analysis}, second ed., p.~96, Theorem 4.7.}  and $I$ is closed so
$f*g \in I$, showing that $I$ is an ideal.

Assume that $I$ is an ideal and let $f \in I$ and $x \in \mathbb{R}^d$. Let $V$ be a
closed ball centered at $0$, and let $\chi_A$ be the indicator function of a set $A$. We have
\begin{align*}
\norm{f_x - \frac{1}{\mu(V)}\chi_{x+V}*f}_1&= \int_{\mathbb{R}^d} \left| f_x(y) - \frac{1}{\mu(V)} (\chi_{x+V}*f)(y) \right| dy\\
&=\int_{\mathbb{R}^d} \left| \frac{1}{\mu(V)} \int_{V} f_x(y) dz - \frac{1}{\mu(V)} \int_{\mathbb{R}^d} \chi_{x+V}(z) f(y-z) dz \right| dy\\
&=\frac{1}{\mu(V)} \int_{\mathbb{R}^d} \left| \int_{V} f(y-x) dz - \int_{V} f(y-z-x) dz \right| dy\\
&=\frac{1}{\mu(V)} \int_{\mathbb{R}^d} \left| \int_{V} (f(y-x) - f(y-z-x)) dz \right| dy\\
&\leq \frac{1}{\mu(V)} \int_{V} \left( \int_{\mathbb{R}^d} \left| f(y-x) - f(y-z-x) \right| dy \right) dz\\
&= \frac{1}{\mu(V)} \int_{V} \norm{f_x-f_{z+x}}_1 dz\\
&=\frac{1}{\mu(V)} \int_{V} \norm{f-f_z}_1 dz\\
&\leq \sup_{z \in V} \norm{f-f_z}_1.
\end{align*}
Let $\epsilon>0$.
The map $z \mapsto f_z$ is continuous $\mathbb{R}^d \to L^1(\mathbb{R}^d)$, so there is some $\delta>0$ such that if
$|z|<\delta$ then $\norm{f_z-f_0}_1<\epsilon$, i.e. $\norm{f-f_z}_1<\epsilon$. Then let $V$ be the closed ball of radius $\delta$, with which
\begin{equation}
\norm{f_x - \frac{1}{\mu(V)}\chi_{x+V}*f}_1 \leq \sup_{z \in V} \norm{f-f_z}_1 \leq \epsilon.
\label{epsilonlimit}
\end{equation}
As $I$ is an ideal and  $\frac{1}{\mu(V)}\chi_{x+V} \in L^1(\mathbb{R}^d)$ we have $\frac{1}{\mu(V)}\chi_{x+V}*f \in L^1(\mathbb{R}^d)$, and
then \eqref{epsilonlimit} and the fact that $I$ is closed imply $f_x \in I$. Therefore $I$ is translation-invariant.
\end{proof}

\end{document}