\documentclass{article}
\usepackage{amsmath,amssymb,graphicx,subfig,mathrsfs,amsthm,siunitx}
%\usepackage{tikz-cd}
%\usepackage{hyperref}
\newcommand{\inner}[2]{\left\langle #1, #2 \right\rangle}
\newcommand{\tr}{\ensuremath\mathrm{tr}\,} 
\newcommand{\Span}{\ensuremath\mathrm{span}} 
\def\Re{\ensuremath{\mathrm{Re}}\,}
\def\Im{\ensuremath{\mathrm{Im}}\,}
\newcommand{\id}{\ensuremath\mathrm{id}} 
\newcommand{\rank}{\ensuremath\mathrm{rank\,}} 
\newcommand{\co}{\ensuremath\mathrm{co}\,} 
\newcommand{\cco}{\ensuremath\overline{\mathrm{co}}\,}
\newcommand{\supp}{\ensuremath\mathrm{supp}}
\newcommand{\epi}{\ensuremath\mathrm{epi}\,}
\newcommand{\Tr}{\ensuremath\mathrm{Tr}\,}
\newcommand{\lsc}{\ensuremath\mathrm{lsc}\,}
\newcommand{\ext}{\ensuremath\mathrm{ext}\,}
\newcommand{\cl}{\ensuremath\mathrm{cl}\,}
\newcommand{\dom}{\ensuremath\mathrm{dom}\,}
\newcommand{\LSC}{\ensuremath\mathrm{LSC}}
\renewcommand{\div}{\ensuremath\mathrm{div}\,}
\newcommand{\curl}{\ensuremath\mathrm{curl}\,}
\newcommand{\USC}{\ensuremath\mathrm{USC}}
\newcommand{\upto}{\nearrow}
\newcommand{\downto}{\searrow}
\newcommand{\norm}[1]{\left\Vert #1 \right\Vert}
\newtheorem{theorem}{Theorem}
\newtheorem{lemma}[theorem]{Lemma}
\newtheorem{proposition}[theorem]{Proposition}
\newtheorem{corollary}[theorem]{Corollary}
\theoremstyle{definition}
\newtheorem{definition}[theorem]{Definition}
\newtheorem{example}[theorem]{Example}
\begin{document}
\title{The Euler equations in fluid mechanics}
\author{Jordan Bell\\ \texttt{jordan.bell@gmail.com}\\Department of Mathematics, University of Toronto}
\date{\today}

\maketitle

\section{Continuity equation}
Let $\Omega$ be a domain in $\mathbb{R}^n$ and let $\rho \in C^\infty(\Omega \times \mathbb{R})$; perhaps later we will
care about functions  that are in larger spaces, and to justify making conclusions about those we will
have to check that what we have said here applies to them. 

Let $U$ be a Lipschitz domain in $\Omega$. Thinking of $\rho$ as a density, the amount of stuff in $U$ at time $t$
is 
\[
\int_U \rho(x,t) dx.
\]
Let $q \in C^\infty(\Omega \times \mathbb{R})$ and $F \in C^\infty(\Omega \times \mathbb{R}, \mathbb{R}^n)$. 
We think about  $q(x,t)$ as the rate at which new stuff appears at point $x$
at time $t$, and $F$ as the flux  of the stuff.
A change in the total amount of stuff in $U$ occurs from stuff appearing inside $U$ and from stuff going through the boundary
of $U$. We formalize this as the statement
\[
\frac{d}{dt}\int_U \rho(x,t)dx = \int_U q(x,t) dx - \int_{\partial U} F(s,t) \cdot N(s) ds,
\]
where $N(s)$ is the outward pointing unit normal to the surface $\partial U$ at the point $s \in \partial U$. 
Using the divergence theorem we get
\[
\int_{\partial U} F(s,t) \cdot N(s) ds = \int_U (\div F)(x,t) dx,
\]
and hence
\[
\int_U (\partial_t \rho)(x,t) = \frac{d}{dt} \int_U \rho(x,t)dx = \int_U (q(x,t) - (\div F)(x,t)) dx,
\]
or,
\[
\int_U (\partial_t \rho - q + \div F)(x,t) dx.
\]
Because this is true for any Lipschitz domain $U$ in $\Omega$, it follows that the integrand is $0$: for all $x \in \Omega$ and $t \in \mathbb{R}$, we have
\[
(\partial_t \rho - q + \div F)(x,t) = 0,
\]
i.e.
\[
\partial_t \rho + \div F = q.
\]
This is called a {\em continuity equation}.

If $\rho(x,t)$ denotes the density of stuff at the point $x$ at time $t$ and $u$ denotes the velocity of the stuff at the point $x$
and time $t$, then the flux $F$ (in other words, the  momentum), is $F=\rho u$. If there is no stuff spontaneously appearing, but rather stuff only moves around,
then $q=0$, and so
\begin{equation}
\partial_t \rho + \div(\rho u) = 0.
\label{continuityequation}
\end{equation}
One can describe the statement that stuff is not spontaneously appearing as {\em conservation of mass}, and hence \eqref{continuityequation}
 can be thought of as a consequence of conservation of mass.




\section{Momentum}
The integral $\int_U (\rho u)(x,t) dx$ is the total amount of momentum of the stuff at points in $U$ at time $t$.
We postulate that there is a function $p \in C^\infty(\Omega,\mathbb{R})$, which we call {\em pressure}, such that the 
 rate of change of the total amount of momentum over a set at time $t$ is equal to 
  the flow of momentum from outside to inside the set at time $t$ plus 
 the total amount of inward directed
 pressure over the boundary of the set at time $t$, which
 here means
 \[
 \frac{d}{dt} \int_U (\rho u)(x,t) dx =-\int_{\partial U} (\rho u)(s,t) u(s,t) \cdot N(s) ds  - \int_{\partial U} p(s,t) N(s) ds,
 \]
 where $N(s)$ is the outward pointing unit normal to the surface $\partial U$ at $s \in \partial U$.
Using the divergence theorem,
\[
 \frac{d}{dt} \int_U (\rho u)(x,t) dx=
 -\int_U \div (\rho u \otimes u)(x,t)
- \int_U (\nabla p)(x,t) dx.
\]
Combined with 
\[
\frac{d}{dt} \int_U (\rho u)(x,t) dx=\int_U \partial_t (\rho u)(x,t) dx,
\]
this gives
\[
\int_U (\partial_t(\rho u) + \div(\rho u \otimes u) + \nabla p)(x,t) dx.
\]
Because this is true for any Lipschitz domain $U$ in $\Omega$, we obtain
\[
(\partial_t(\rho u) + \div(\rho u \otimes u) + \nabla p)(x,t)=0
\] 
for all $x \in \Omega$ and $t \in \mathbb{R}$, or
\begin{equation}
\partial_t(\rho u) + \div(\rho u \otimes u) + \nabla p=0.
\label{momentum}
\end{equation}

To state that the stuff we are talking about is incompressible means that 
$\rho$ is constant. For the rest of this note, unless we state otherwise we take $\rho$ to be a nonzero constant, with which   equation \eqref{continuityequation} becomes
\begin{equation}
\div (u)=0,
\label{euler1}
\end{equation}
and \eqref{momentum} becomes
\begin{equation}
\partial_t u + \div(u \otimes u) + \frac{1}{\rho} \nabla p =0.
\label{euler2}
\end{equation}
The two equations \eqref{euler1} and \eqref{euler2} are called the {\em Euler equations for an incompressible fluid}. 




Taking the divergence of \eqref{euler2} yields
\[
\partial_t \div(u) + \div (\div (u \otimes u)) + \frac{1}{\rho}\div(\nabla p)=0.
\]
Using \eqref{euler1} and writing $\Delta p = \div(\nabla p)$,
\begin{equation}
\div(\div(u \otimes u)) + \frac{1}{\rho} \Delta p =0.
\label{divergencepressure}
\end{equation}

As
\[
\div(u \otimes u) = \partial_j(u_i u_j)e_i,
\]
we have, using \eqref{euler1},
\begin{eqnarray*}
\div(\div(u \otimes u))& =& \partial_i \partial_j(u_iu_j) \\
&=& \partial_i((\partial_ju_i)u_j + u_i \partial_j u_j)\\
&=&\partial_i((\partial_j u_i)u_j + u_i \div(u))\\
&=&\partial_i((\partial_j u_i)u_j)\\
&=&(\partial_i \partial_j u_i)u_j + (\partial_j u_i)\partial_i u_j\\
&=&(\partial_j (\div(u))) u_j + (\partial_j u_i)\partial_i u_j\\
&=&(\partial_j u_i)\partial_i u_j.
\end{eqnarray*}
Therefore, using this
with \eqref{divergencepressure} we get
\begin{equation}
-\Delta p = \rho (\partial_j u_i)\partial_i u_j.
\label{ellipticpressure}
\end{equation}
The use of this equation is to give us more information about the pressure $p$.

Furthermore,
with $u=u_i e_i$ and writing
\[
\nabla u = (\partial_j u) \otimes e_j = (\partial_j (u_i e_i)) \otimes e_j = \partial_j u_i e_i \otimes e_j,
\]
the contraction of the tensor $\nabla u$ with itself is
\begin{eqnarray*}
(\nabla u)(\nabla u) &=& (\partial_j u_i e_i \otimes e_j)(\partial_l u_k e_k \otimes e_l)\\
&=&(\partial_j u_i)(\partial_l u_k) (e_i \otimes e_j)(e_k \otimes e_l)\\
&=&(\partial_j u_i)(\partial_l u_k) \delta_{j,k} e_i \otimes e_l\\
&=&(\partial_j u_i)(\partial_l u_j) e_i \otimes e_l,
\end{eqnarray*}
for which
\[
\Tr((\nabla u)(\nabla u)) = (\partial_j u_i)(\partial_i u_j).
\]
With this, equation  \eqref{ellipticpressure} becomes
\[
-\Delta p = \rho \Tr((\nabla u)(\nabla u)).
\]


As
\[
\div(u \otimes u)=\partial_j(u_i u_j) e_i = (\partial_j u_i)u_je_i + u_i \partial_j u_j e_i = (\partial_j u_i)u_je_i + u_i \div(u) e_i,
\]
using \eqref{euler1} we have
\[
\div(u \otimes u) 
=(\partial_j u_i)u_je_i,
\]
and hence it follows from \eqref{euler1} that
\begin{equation}
\div(u \otimes u) = u \cdot \nabla u.
\label{materialform}
\end{equation}
This expression for $\div(u \otimes u)$ may be easier to work with than the original expression.

\section{Energy}
If $v=v_i e_i$ is a vector field,
we write
\[
\nabla v = (\partial_j v) \otimes e_j = (\partial_j v_i) e_i \otimes e_j.
\]
Then,
\[
v\cdot \nabla v = (v_k e_k) \cdot ( (\partial_j v_i) e_i \otimes e_j ) = v_j \partial_j v_i e_i = v_j\partial_j v.
\]


If $u$ (velocity of stuff) and $p$ (pressure of stuff) satisfy \eqref{euler1} and \eqref{euler2}, then
applying $u \cdot$ to both sides of \eqref{euler2} we get
\begin{equation}
u \cdot (\partial_t u ) + u \cdot  \div(u \otimes u ) + \frac{1}{\rho} u \cdot  \nabla p=0.
\label{udoteuler2}
\end{equation}
First,
\[
\partial_t (u \cdot u) = \partial_t (u_i u_i) = (\partial_t u_i) u_i + u_i (\partial_t u_i) = 2u_i (\partial_t u_i)= 
2 u \cdot (\partial_tu).
\]
Second,
\[
\div(u \otimes u) = \partial_j(u_iu_j)e_i
\]
so
\[
u \cdot \div(u \otimes u) = u_i \partial_j(u_i u_j) = u_i(\partial_j u_i)u_j + u_i u_i \partial_ju_j;
\]
but
\[
\div((u \cdot u) u) =  \div(u_i u_i u_j e_j) = \partial_j (u_i u_i u_j) = 2u_i(\partial_j u_i) u_j + u_i u_i \partial_j u_j,
\]
hence
\[
\div((u\cdot u)u) = 2 u\cdot \div(u \otimes u) - u\cdot u \div(u),
\]
and using \eqref{euler1} this is
\[
\div((u \cdot u)u) = 2 u\cdot \div(u \otimes u).
\]
Third,
\[
\div(pu) = (\nabla p) \cdot u + p \div (u),
\]
and using \eqref{euler1} this is
\[
\div(pu) = (\nabla p) \cdot u.
\]
Putting these three results  into \eqref{udoteuler2} gives
\[
\frac{1}{2} \partial_t(u\cdot u) + \frac{1}{2} \div((u\cdot u)u) + \frac{1}{\rho} \div(pu) =0,
\]
or
\begin{equation}
\partial_t \left( \frac{1}{2}\rho u\cdot u \right) + \div\left(\frac{1}{2}\rho (u\cdot u)u + pu \right)=0.
\label{energyequation}
\end{equation}

We define
\[
E=\frac{1}{2}\rho u\cdot u.
\]
If $\rho$ is thought of as mass density, with units of \si{kg/m^3}, and $u$ is thought of as the velocity of stuff, with units of \si{m/s},
then $E$ has units of 
$\si{kg.m^{-1}.s^{-2}}=\si{J/m^3}$. We choose to think of  $E$ defined this way as energy density; we say {\em choose} because although $E$ has the right units to be energy density,
any multiple would have the same units, and it is not apparent from what we have said so far why we care about $\frac{1}{2}\rho u\cdot u$ rather than some other multiple of 
$\rho u\cdot u$.
Writing equation \eqref{energyequation} using $E$ gives
\[
\partial_t E + \div(E u  + pu) = 0,
\]
which is thus a statement about the rate of change of  energy density.
We call $\frac{E+p}{\rho}$ the {\em total specific enthalpy} of the stuff. To say that a quantity  is specific means that it expresses some quantity per \si{kg}, and the dimensions of
enthalpy are \si{J}.



\section{Vorticity}
In this section, unless we say otherwise we take $n=3$.  For vector fields $v,w$,
\[
\nabla(v \cdot w) = v\cdot  \nabla w + w \cdot \nabla v + v \times \curl w + w \times \curl v.
\]
Using this identity $v=u$ and $w=u$ gives
\[
u \cdot \nabla u = u \times \curl u - \frac{1}{2} \nabla(u \cdot u),
\]
and therefore \eqref{materialform} can be written as
\begin{equation}
\div(u \otimes u) =\frac{1}{2}\nabla(u \cdot u) - u \times \curl u.
\label{curlequation}
\end{equation}

Taking the curl of \eqref{euler2} yields
\[
\partial_t \curl u + \curl \div(u \otimes u) =0;
\]
we used the fact that
the curl of the gradient of any scalar field is $0$ and so $\curl \nabla p=0$.
Using \eqref{curlequation}, this becomes
\[
\partial_t \curl(u) + \curl\left( \frac{1}{2} \nabla(u\cdot u) - u \times \curl u \right) = 0,
\]
and as the curl of the gradient of a scalar field is $0$, this is
\[
\partial_t \curl u = \curl(u \times \curl u).
\]
For vector fields $v,w$,
\[
\curl(v \times w) = v \div w - w \div v + (\nabla v)(w) - (\nabla w)(v),
\]
and with $v=u$ and $w=\curl u$ we obtain
\[
\partial_t \curl u = u \div \curl u - \curl(u) \div u + (\nabla u)(\curl u) - (\nabla \curl u)(u).
\]
Because the divergence of the curl of a vector field is $0$ and because $\div u=0$ by \eqref{euler1}, this becomes
\[
\partial_t \curl u  = (\nabla u)(\curl u) - (\nabla \curl u)(u).
\]
We call $\omega=\curl(u)$ the {\em vorticity} of the stuff, and with this notation the above equation can
be written as
\begin{equation}
\partial_t \omega  = (\nabla u)(\omega)-(\nabla \omega)(u).
\label{vorticity}
\end{equation}



\section{Material time derivative}
One often deals with expressions like $\partial_t \omega + u \cdot \nabla \omega$, and we 
write
\[
\frac{D}{Dt} = \partial_t + u\cdot \nabla,
\]
and call $\frac{D}{Dt}$ the {\em material time derivative}; it depends on the velocity $u$
of the stuff. With this notation, the equation \eqref{vorticity}
is
\[
\frac{D\omega}{Dt} = \omega \cdot \nabla u.
\]
Using \eqref{materialform} (which itself supposes \eqref{euler1}), we can write \eqref{euler2} using the material time derivative as
\[
\frac{Du}{Dt} + \nabla p=0.
\]



\section{Irrotational velocity fields}
In this section unless we say otherwise we take $n=3$ and we suppose that $\curl u =0$, which we describe as   $u$ being {\em irrotational}.
We suppose also in this section that $\Omega$ is simply connected,
which together with $\curl u=0$ implies that there is some 
 $\phi \in C^\infty(\Omega \times \mathbb{R})$ for which
\[
u(x,t) = (\nabla \phi)(x,t)
\]
for all $x \in \Omega$ and for all $t \in \mathbb{R}$; cf. the Helmholtz decomposition of a vector field in $\mathbb{R}^3$. We call $\phi$ a {\em potential function for $u$}.
Combining \eqref{euler2}, \eqref{materialform}, and $u=\nabla \phi$, we obtain
\begin{equation}
\partial_t \nabla \phi + (\nabla \phi)\cdot \nabla \nabla \phi + \frac{1}{\rho} \nabla p =0.
\label{conservativeeuler}
\end{equation}
We have
\begin{eqnarray*}
(\nabla \phi)\cdot \nabla \nabla \phi&=&(\partial_i \phi e_i) \cdot \nabla (\partial_k \phi e_k)\\
&=&(\partial_i \phi e_i)  \cdot (\partial_j \partial_k \phi e_k \otimes e_j)\\
&=&(\partial_i \phi)\partial_i \partial_k \phi e_k\\
&=&\frac{1}{2}\partial_k (( \partial_i \phi)(\partial_i \phi))e_k\\
&=&\frac{1}{2}\partial_k (\nabla \phi \cdot \nabla \phi) e_k\\
&=&\frac{1}{2}\nabla(\nabla \phi \cdot \nabla \phi),
\end{eqnarray*}
with which \eqref{conservativeeuler} becomes
\[
\partial_t \nabla \phi + \frac{1}{2}\nabla(\nabla \phi \cdot \nabla \phi) + \frac{1}{\rho}\nabla p=0,
\]
or
\[
\nabla\left( \partial_t \phi+ \frac{1}{2}\nabla\phi \cdot \nabla \phi+\frac{1}{\rho}p\right)=0.
\]
Then, defining $P$ to be
\[
P =  \partial_t \phi+ \frac{1}{2}\nabla\phi \cdot \nabla \phi+\frac{1}{\rho}p,
\]
we have that $P$ depends only on time. We call $P$ the {\em total pressure}, and the statement that the total pressure
depends only on time if the velocity $u$ is irrotational is called {\em Bernoulli's principle}. 

Furthermore, combining \eqref{euler1} with $u=\nabla \phi$ gives
\[
\Delta \phi = 0,
\]
i.e., for each $t \in \mathbb{R}$, $x \mapsto \phi(x,t)$ is a harmonic function on $\Omega$. 



\section{Euler equations in one dimension}
In this section we take $n=1$ and do not suppose that the pressure $\rho$ is constant. Since we do not take $\rho$ to be constant, we will use \eqref{continuityequation},
which tells us that
\[
\partial_t \rho + \div(\rho u) = 0,
\]
and \eqref{momentum}, which tells us that
\[
\partial_t(\rho u) + \div(\rho u \otimes u) + \nabla p=0.
\]
As $n=1$ here, we can write these two equations as
\begin{equation}
\partial_t \rho+ \partial_x(\rho u)=0
\label{1deuler1}
\end{equation}
and
\begin{equation}
\partial_t (\rho u)+\partial_x(\rho u^2) + \partial_x p =0.
\label{1deuler2}
\end{equation}
We suppose, giving no justification, that there are some constant $K$  and $\gamma$ for which $p=K\rho^\gamma$.
With this assumption, equation \eqref{1deuler2} becomes
\begin{equation}
\partial_t (\rho u)+\partial_x(\rho u^2) + \gamma  \frac{p}{\rho} \partial_x \rho = 0.
\label{pressuredensity}
\end{equation}

We write $\rho=\rho_0+\rho_1$ where $\rho_0$ is a constant, and
we also write $u=u_0+u_1$ where $u_0$ is a constant. With these definitions, the equation \eqref{1deuler1} becomes
\[
\partial_t (\rho_0+\rho_1) + \partial_x(\rho_0u_0+\rho_0u_1+\rho_1 u_0+\rho_1u_1)=0,
\]
i.e.
\[
\partial_t \rho_1 + \rho_0 \partial_x u_1 + u_0 \partial_x \rho_1 + \partial_x(\rho_1u_1)=0.
\]
Supposing that the last term is negligible, an approximation to the above equation is
\begin{equation}
\partial_t \rho_1 + \rho_0 \partial_x u_1 + u_0 \partial_x \rho_1=0.
\label{approx1}
\end{equation}
Furthermore, \eqref{pressuredensity} becomes
\[
\begin{split}
&\partial_t(\rho_0u_0+\rho_0u_1+\rho_1 u_0+\rho_1u_1) \\
&+ \partial_x(\rho_0u_0^2+2\rho_0u_0u_1+\rho_0u_1^2+\rho_1u_0^2+2\rho_1 u_0u_1 + \rho_0u_1^2)\\
&+\gamma \frac{p}{\rho_0+\rho_1} \partial_x(\rho_0+\rho_1)\\
=&0.
\end{split}
\]
Using that  $\rho_0$ and $u_0$ are constant and supposing that  $\partial_x(\rho_1 u_1)$ and $\partial_x(u_1^2)$ are negligible gives us the approximation
\[
\rho_0 \partial_t u_1 + u_0 \partial_t \rho_1 + 2\rho_0u_0 \partial_x u_1 + u_0^2 \partial_x \rho_1 + \gamma \frac{p}{\rho_0+\rho_1} \partial_x \rho_1.
\]
Expressing $\frac{1}{\rho_0+\rho_1}$ as a geometric series in powers of $\frac{\rho_1}{\rho_0}$ and supposing that
the sum of all the nonconstant terms is negligible, and approximating $p=K\rho^\gamma$ as $p_0=K\rho_0^\gamma$
gives us the approximation
\[
\rho_0 \partial_t u_1 + u_0 \partial_t \rho_1 + 2\rho_0u_0 \partial_x u_1 + u_0^2 \partial_x \rho_1 + \gamma \frac{p_0}{\rho_0} \partial_x \rho_1=0.
\]
Combining this equation with \eqref{approx1} multiplied by $u_0$ yields
\[
\rho_0\partial_t u_1 + \rho_0 u_0 \partial_x u_1 + \gamma \frac{p_0}{\rho_0}\partial_x \rho_1=0.
\]
We define $D_t = \partial_t + u_0 \partial_x$, with which we can write the above equation as
\begin{equation}
\rho_0 D_t u_1+\gamma \frac{p_0}{\rho_0} \partial_x \rho_1=0,
\label{wave1}
\end{equation}
and we can write \eqref{approx1} as
\begin{equation}
D_t \rho_1 + \rho_0 \partial_x u_1=0.
\label{wave2}
\end{equation}
Applying $D_t$ to \eqref{wave2}  gives
\[
D_t^2 \rho_1 + \rho_0 \partial_x D_t u_1 =0,
\]
and then using \eqref{wave1} this becomes
\[
D_t^2 \rho_1 + \partial_x \left( -\gamma \frac{p_0}{\rho_0} \partial_x \rho_1 \right) =0,
\]
or
\[
D_t^2 \rho_1 - \gamma \frac{p_0}{\rho_0} \partial_x^2 \rho_1 =0,
\]
which is a wave equation satisfied by $\rho_1$. 


\end{document}