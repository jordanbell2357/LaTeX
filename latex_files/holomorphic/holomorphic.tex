\documentclass{article}
\usepackage{amsmath,amssymb,graphicx,subfig,mathrsfs,amsthm}
\usepackage{hyperref}
\usepackage{tikz-cd}
\newcommand{\inner}[2]{\left\langle #1, #2 \right\rangle}
\newcommand{\tr}{\textrm{tr}} 
\newcommand{\Span}{\textrm{span}} 
\newcommand{\SA}{B_{\textrm{sa}}(H)} 
\newcommand{\positive}{B_{\textrm{+}}(H)} 
\newcommand{\id}{\textrm{id}} 
\newcommand{\norm}[1]{\left\Vert #1 \right\Vert}
\newtheorem{theorem}{Theorem}
\newtheorem{lemma}[theorem]{Lemma}
\newtheorem{corollary}[theorem]{Corollary}
\begin{document}
\title{The  Fr\'echet space of holomorphic functions on the unit disc}
\author{Jordan Bell\\ \texttt{jordan.bell@gmail.com}\\Department of Mathematics, University of Toronto}
\date{\today}
\maketitle

\section{Introduction}
The goal of this note is to develop all the machinery necessary to understand what it means to say that the set $H(D)$ of holomorphic
functions on the unit disc is a separable and reflexive Fr\'echet space that has the Heine-Borel property and is not normable.



\section{Topological vector spaces}
If $X$ is a topological space and $p \in X$, a {\em local basis at $p$} is a set $\mathscr{B}$ of open neighborhoods of $p$ such that if $U$ is an open neighborhood of 
$p$ then there is some $U_0 \in \mathscr{B}$ that is contained in $U$. 
We emphasize that to say that a topological vector space $(X,\tau)$ is normable is to say not just that there
is a norm on the vector space $X$, but moreover that the topology $\tau$ is induced by the norm.

A {\em topological vector space} over $\mathbb{C}$ is a vector space $X$  over $\mathbb{C}$ that is a  topological space such that singletons are closed sets and such that
vector addition $X \times X \to X$ and scalar multiplication $\mathbb{C} \times X \to X$ are continuous. It is not true that a topological space in which singletons are closed need be
Hausdorff, but one can prove that every topological vector space is a Hausdorff space.\footnote{Walter Rudin, {\em Functional Analysis}, second ed., p.~11, Theorem 1.12.}
For any $a \in X$, we check that the map $x \mapsto a+x$ is a homeomorphism. Therefore, a subset $U$ of $X$ is open if and only if 
$a+U$ is open for all $a \in X$. 
It follows that if $X$ is a vector space and $\mathscr{B}$ is a set of subsets of $X$ each of which contains $0$, then there is at most
one topology for $X$ such that $X$ is a topological vector space for which $\mathscr{B}$ is a local basis at $0$. In other words, the topology of a topological vector space
is determined by specifying a local basis at $0$.
A topological vector space $X$ is said to be {\em locally convex} if there is a local basis at $0$ whose elements are convex sets.

If $X$ is a vector space and $\mathscr{F}$ is a set of seminorms on $X$, 
we say that $\mathscr{F}$ is a {\em separating family} if $x \neq 0$ implies that
there is some $m \in \mathscr{F}$ with $m(x) \neq 0$. (Thus, if $m$ is a seminorm on $X$, the singleton $\{m\}$ 
is a separating family if and only if $m$ is a norm.)  The following theorem presents a local basis at $0$ for a topology and shows that there is a topology
for which the vector space is a locally convex space and for which this is a local basis at $0$.\footnote{Paul Garrett, {\em Seminorms and locally convex spaces}, \url{http://www.math.umn.edu/~garrett/m/fun/notes_2012-13/07b_seminorms.pdf}} We call  this topology  the {\em seminorm topology induced by $\mathscr{F}$}.


\begin{theorem}[Seminorm topology]
If $X$ is a vector space and  $\mathscr{F}$ is a separating family of seminorms on $X$, then there is 
a topology $\tau$ on $X$ such that $(X,\tau)$ is a locally convex space and  the collection $\mathscr{B}$ of finite intersections of sets of the form
\[
B_{m,\epsilon}=\{x\in X: m(x)<\epsilon\}, \qquad m\in \mathscr{F}, \epsilon>0
\]
is a local basis at $0$.
\label{seminormtopology}
\end{theorem}
\begin{proof}
We define $\tau$ to be those subsets $U$ of $X$ such that for all $x \in U$  there is some 
$N \in \mathscr{B}$ satisfying $x+N \subseteq U$. If $\mathscr{U}$ is a subset of $\tau$ and 
$x \in \bigcup_{U \in \mathscr{U}} U$, then 
there is some $U_0 \in \mathscr{U}$ with $x \in U_0$, and there is some $N_0 \in \mathscr{B}$ satisfying
$x+N_0 \subseteq U_0$. We have
\[
x+N_0 \subseteq U_0 \subseteq \bigcup_{U \in \mathscr{U}} U,
\]
which tells us that $\bigcup_{U \in \mathscr{U}}  U \in \tau$. If $U_1,\ldots,U_n \in \tau$ and
$x \in \bigcap_{k=1}^n U_k$, then there are $N_1,\ldots,N_n \in \mathscr{B}$ satisfying
$x + N_k \in U_k$ for $1 \leq k \leq n$. But the intersection of finitely many elements of $\mathscr{B}$ is itself an element
of $\mathscr{B}$, so $N=\bigcap_{k=1}^n N_k \in \mathscr{B}$, and 
\[
x+N \subseteq \bigcap_{k=1}^n U_k,
\]
showing that $\bigcap_{k=1}^n U_k \in \tau$. Therefore, $\tau$ is a topology.

Suppose that $x \in X$. For $y \neq x$, let $m_y \in \mathscr{F}$ with $\epsilon_y=m_y(x-y) \neq 0$; there is such a seminorm because $\mathscr{F}$ is
a separating family.
Then
$U_y=y+B_{m_y,\epsilon_y}$ is an open set that contains $y$ and does not contain $x$. Therefore $X \setminus U_y$ is a closed set that contains $x$ and does not contain $y$, 
and 
\[
\bigcap_{y \neq x} X \setminus U_y = \{x\}
\]
is a closed set, showing that singletons are closed.


Let $x,y \in X$ and $N \in \mathscr{B}$. There are  $m_k \in \mathscr{F}$ and $\epsilon_k>0$, $1 \leq k \leq n$, such that
$N = \bigcap_{k=1}^n B_{m_k,\epsilon_k}$. Let $U = \bigcap_{k=1}^n B_{m_k,\epsilon_k/2}$. If $v \in (x+U) + (y+U)$ and
$1 \leq k \leq n$, then there are $x_k \in B_{m_k,\epsilon_k/2}$ and $y_k \in B_{m_k,\epsilon_k/2}$ such that
$v = x+x_k + y + y_k$, and
\[
m_k(v-(x+y)) = m_k(x_k+y_k) \leq m_k(x_k)+ m_k(y_k) < \frac{\epsilon_k}{2}+\frac{\epsilon_k}{2} = \epsilon_k,
\]
so $v \in x+y+B_{m_k,\epsilon_k}$. This is true for each $k$, $1 \leq k \leq n$, so $v \in x+y+N$. Hence
\[
(x+U)+(y+U) \subseteq x+y+N,
\]
showing that vector addition is continuous at $(x,y) \in X \times X$: for every basic open neighborhood $x+y+N$ of the image $x+y$, there
is an open neighborhood $(x+U) \times (y+U)$ of $(x,y)$ whose image under vector addition is contained in $x+y+N$. 

Let $\alpha \in \mathbb{C}$, $x \in X$, and $N \in \mathscr{B}$, say $N = \bigcap_{k=1}^n B_{m_k,\epsilon_k}$. Let
$\epsilon=\min \{\epsilon_k: 1 \leq k \leq n\}$, 
let $\delta>0$ be small enough so that $\delta(\delta+|\alpha|+m_k(x))<\epsilon$ for each $1 \leq k \leq n$,
let $\Delta=\{\beta \in \mathbb{C}: |\beta-\alpha|<\delta\}$, and
let $U=\bigcap_{k=1}^n B_{m_k,\delta}$.
If $(\beta,v) \in \Delta \times (x+U)$ and $1 \leq k \leq n$, then
\begin{align*}
m_k(\beta v - \alpha x)&=m_k(\beta v - \beta x + \beta x - \alpha x)\\
&\leq m_k(\beta (v-x)) +m_k((\beta-\alpha) x)\\
&=|\beta| m_k(v-x) + |\beta-\alpha| m_k(x)\\
&< (\delta+|\alpha|) \delta+\delta m_k(x)\\
&=\delta(\delta+|\alpha|+m_k(x))\\
&<\epsilon\\
&\leq \epsilon_k,
\end{align*}
showing that $\beta v \in \alpha x + B_{m_k,\epsilon_k}$. This is true for each $k$, so $\beta v \in N$, which shows
that scalar multiplication is continuous at $(\alpha,x)$: for every basic open neighborhood $\alpha x +N$ of the image $\alpha x$,
there is an open neighborhood $\Delta \times (x+U)$ of $(\alpha,x)$ whose image under scalar multiplication
is contained in $\alpha x+N$.


We have shown that $X$ with the topology $\tau$ is a topological vector space. To show that $X$ is a locally convex space it
suffices to prove that each element of the local basis $\mathscr{B}$ is convex. An intersection of convex sets is a convex set,
so to prove that each element of $\mathscr{B}$ is convex it suffices to prove that each $B_{m,\epsilon}$ is convex, 
$m \in \mathscr{F}$ and $\epsilon>0$. If $0 \leq t \leq 1$ and $x,y \in B_{m,\epsilon}$, then
\[
m(tx+(1-t)y) \leq m(tx)+m((1-t)y) = tm(x)+(1-t)m(y)<t\epsilon+(1-t)\epsilon=\epsilon,
\]
showing that $tx+(1-t)y \in B_{m,\epsilon}$ and thus that $B_{m,\epsilon}$ is a convex set. Therefore, $(X,\tau)$ is a locally
convex space. 
\end{proof}


In the other direction, we will now explain how the topology of a locally convex space is induced by a separating family of seminorms.
We say that a subset $S$ of a vector space $X$ is {\em absorbing} if $x \in X$ implies that there is some $t > 0$ such that $x \in tS$. 
The {\em Minkowski functional} $\mu_S:X \to [0,\infty)$ of an absorbing set $S$ is defined by
\[
\mu_S(x)=\inf\{t  \geq 0: x \in tS\}, \qquad x \in X.
\]
If $U$ is an open set containing $0$ and $x \in X$, then $0\cdot x=0 \in U$, and because scalar multiplication is continuous there is some $t > 0$ such that $tx \in U$. Thus an open set
containing $0$ is absorbing.
We say that  a subset $S$ of a vector space $X$ is {\em balanced} if $|\alpha| \leq 1$ implies that $\alpha S \subseteq S$.
One proves that in a topological vector space, every convex open neighborhood of $0$ contains a balanced convex open neighhborhood of $0$.\footnote{Walter Rudin,
{\em Functional Analysis}, second ed., p.~12, Theorem 1.14.} It follows that a locally convex space has a local basis at $0$ whose elements are balanced convex open sets. 
The following lemma shows that the Minkowski functional of each member of this local basis is a seminorm.

\begin{lemma}
If $X$ is a topological vector space and 
 $U$ is a balanced convex open neighborhood of  $0$, then the Minkowski functional of $U$ is a seminorm on $X$.
\end{lemma}
\begin{proof}
Let $\alpha \in \mathbb{C}$ and $x \in X$. If $\alpha = 0$, then
\[
\mu_U(\alpha x)= \mu_U(0)=0  = |\alpha| \mu_U(x).
\]
Otherwise, write $\alpha = ru$ with $r>0$ and $|u|=1$. 
Because $U$ is balanced and $|u^{-1}|=1$, we have
\begin{align*}
\mu_U(\alpha x)& = \inf\{t \geq 0: \alpha x \in tU\}\\
&=\inf\{t \geq 0: rux \in tU\}\\
&=\inf\{t \geq 0: x \in r^{-1}t u^{-1} U\}\\
&=\inf\{t \geq 0: x \in r^{-1}t U\}\\
&=\inf\{rs \geq 0: x \in sU\}\\
&=r \inf\{s \geq 0: x \in sU\}\\
&=r\mu_U(x).
\end{align*}
Therefore, if $\alpha \in \mathbb{C}$ and $x \in X$, then $\mu_U(\alpha x)=|\alpha| \mu_U(x)$.

Let $x,y \in X$. $U$ is absorbing, so let $s=\mu_U(x)$ and $t = \mu_U(y)$. If $\epsilon>0$ then
$x \in (s+\epsilon)U$ and $y \in (t+\epsilon)U$. 
We have
\[
x+y \in (s+\epsilon)U + (t+\epsilon)U = \{(s+\epsilon)u + (t+\epsilon)v: u,v \in U\},
\]
and for $u,v \in U$, because $U$ is convex we have
\[
s'u+t'v = (s'+t') \left( \frac{s'}{s'+t'}u + \frac{t'}{s'+t'}v \right)  \in (s'+t')U,
\]
where $s'=s+\epsilon$ and $t' = t+\epsilon$,
so
\[
x+y \in (s+t+2\epsilon)U.
\]
This is true for every $\epsilon>0$, which means that $\mu_U(x+y) \leq s+t$. Therefore
\[
\mu_U(x+y) \leq s+t = \mu_U(x)+\mu_U(y),
\]
showing that $\mu_U$ satisfies the triangle inequality and hence that $\mu_U$ is a seminorm on $X$.
\end{proof}


We  proved above that the Minkowski functional of a balanced convex open neighborhood of $0$ is a seminorm. 
The following lemma shows  that the collection of Minkowski functionals corresponding to a balanced convex local basis at $0$ are a separating family.\footnote{Walter Rudin, {\em Functional Analysis}, second ed., p.~27, Theorem 1.36.}

\begin{lemma}
If $X$ is a topological vector space and $U$ is a balanced convex open neighborhood of $0$, then 
\[
U = \{x \in X: \mu_U(x)<1\}.
\]
If  $\mathscr{B}$ is a local basis at $0$ whose elements are balanced and convex, then
\[
\{\mu_U: U \in \mathscr{B}\}
\]
 is a separating family of seminorms on $X$.
 \label{minkowski}
\end{lemma}
\begin{proof}
Let $U \in \mathscr{B}$.  If $x \in U$, then
because $1 \cdot x \in U$ and scalar multiplication
is continuous, there is some $\delta>0$ and some open neighborhood $N$ of $x$ such that the image of $[1-\delta,1+\delta] \times N$ under scalar multiplication
is contained in $U$. In particular, if $(1+\delta)x \in U$ and so $x \in \frac{1}{1+\delta}U$. 
Thus we have
\[
\mu_U(x)=\inf \{t \geq 0: x \in tU\} \leq \frac{1}{1+\delta}<1.
\]
Therefore, if $x \in U$ then $\mu_U(x)<1$.  On the other hand,
if $x \in X$ and $\mu_U(x)<1$, then there is some $t<1$ such that $x \in tU$. As $U$ is balanced, we have $x \in U$. Therefore,
if $\mu_U(x)<1$ then $x \in U$. This establishes that if $U \in \mathscr{B}$ then
\[
U = \{x \in X: \mu_U(x)<1\}.
\]

If $x \neq 0$, then because singletons are closed, the set $X \setminus \{x\}$ is  open and contains $0$, and thus
there is some $U \in \mathscr{B}$ with $U \subseteq X \setminus \{x\}$.  Hence $x \not \in U$, which implies by the first claim that
 $\mu_U(x) \geq 1$. In particular, $\mu_U(x) \neq 0$, proving the second claim.
\end{proof}

If $X$ is a locally convex space then there is a local basis at $0$, call it $\mathscr{B}$, whose elements are balanced and convex, and we have established that $\mathscr{F}=\{\mu_U:U \in \mathscr{B}\}$ is a separating family of seminorms on $X$. Therefore by Theorem \ref{seminormtopology}, $X$ with the seminorm topology induced by $\mathscr{F}$ 
is a locally convex space. The following theorem states that the seminorm topology is equal to the original topology of the space.\footnote{Paul Garrett, {\em Seminorms and locally convex spaces}, \url{http://www.math.umn.edu/~garrett/m/fun/notes_2012-13/07b_seminorms.pdf}}

\begin{theorem}
If $(X,\tau)$ is a locally convex space, then there is a separating family of seminorms on $X$ such that $\tau$ is equal to the seminorm topology.
\end{theorem}
\begin{proof}
Let $\mathscr{B}$ be a local basis at $0$ whose elements are balanced and convex and let $\mathscr{F} = \{\mu_U: U \in \mathscr{B}\}$. If $U \in \mathscr{B}$, then 
$U=\{x \in X: \mu_U(x)<1\}$, which is an open neighborhood of $0$ in the seminorm topology induced by $\mathscr{F}$, and this implies
that the seminorm topology is at least as fine as $\tau$. 

If $U \in \mathscr{B}$ and $\epsilon>0$, then
\[
\{x \in X:\mu_U(x)<\epsilon\} = \left\{x \in X: \mu_U\left(\frac{x}{\epsilon}\right)<1\right\}=\{\epsilon x \in X: \mu_U(x)<1\}=\epsilon U.
\]
$\epsilon U \in \tau$ and $0 \in \epsilon U$, and it follows that $\tau$ is at least as fine as the seminorm topology. Therefore $\tau$ is equal to the seminorm topology
induced by $\mathscr{F}$. 
\end{proof}

We have shown that if $X$ is a vector space and $\mathscr{F}$ is a separating family of seminorms on $X$, then $X$ with the seminorm topology induced by $\mathscr{F}$
 is a locally convex space. Furthermore, we have shown that if $X$ is a locally convex space then there is a separating family $\mathscr{F}$ of seminorms on $X$ such that
the topology of $X$ is equal to the seminorm topology induced by $\mathscr{F}$. In other words, the topology of any locally convex space is the seminorm topology induced by some
separating family of seminorms on the space.


A subset $E$ of a topological vector space $X$ is said to be {\em bounded} if for every open neighborhood $N$ of $0$ there is some $s>0$
such that $t>s$ implies that $E \subseteq tN$. 

\begin{lemma}
If $X$ is a locally convex space with the seminorm topology induced by a separating family $\mathscr{F}$ of seminorms on $X$, then a 
subset $E$ of $X$ is bounded if and only if each $m \in \mathscr{F}$ is a bounded function on $E$.
\label{boundedset}
\end{lemma}
\begin{proof}
Suppose that $E$ is bounded and  $m \in \mathscr{F}$. The set $U=\{x \in X: m(x)<1\}$ is an open neighborhood of $0$, so there is some $t>0$ such that 
$E \subseteq tU$. Hence if $x \in E$ then $m(x)<t$, so $m$ is a bounded function on $E$.

Suppose that for each $m \in \mathscr{F}$ there is some $M_m$ such that $x \in E$ implies that $m(x) \leq M_m$.
If $U$ is an open neighborhood of $0$, then there are $m_1,\ldots,m_n \in \mathscr{F}$ and $\epsilon_1,\ldots,\epsilon_n>0$
such that 
\[
\bigcap_{k=1}^n \{x \in X: m_k(x)<\epsilon_k\} \subseteq U.
\]
Let $M=\max \left\{ \frac{M_{m_k}}{\epsilon_k}: 1 \leq k \leq n\right\}$. For $t>M$,
\[
\bigcap_{k=1}^n \{tx \in X: m_k(x) < \epsilon_k\} \subseteq tU,
\]
i.e.,
\[
\bigcap_{k=1}^n \{x \in X: m_k(x) < \epsilon_k t\} \subseteq tU.
\]
But if $x \in E$ and $1 \leq k \leq n$ then
\[
m_k(x) \leq M_{m_k} \leq \epsilon_k M < \epsilon_k t,
\]
hence $x$ is in the above intersection and thus is in $tU$. Therefore $E \subseteq tU$, showing that $E$ is bounded.
\end{proof}


We now prove that if the topology of a locally convex space is induced by a countable separating family of seminorms then the topology is metrizable.

\begin{theorem}
If $(X,\tau)$ is a locally convex space with the seminorm topology induced by a countable separating family of seminorms $\{m_n:n \in \mathbb{N}\}$
and $c_n$ is a summable nonincreasing sequence of positive numbers,
then 
\[
d(x,y) =\sum_{n=1}^\infty c_n \frac{m_n(x-y)}{1+m_n(x-y)}, \qquad x,y \in X,
\]
is a translation invariant metric on $X$, $\tau$ is equal to the metric topology for $d$, and with this metric the open balls centered at $0$  are balanced.
\label{metrizable}
\end{theorem}
\begin{proof}
For any $x,y \in X$ we have
\[
d(x,y) < \sum_{n=1}^\infty c_n < \infty,
\]
because the sequence $c_n$ is summable. It is apparent that $d(x,y)=d(y,x)$. 

If $m$ is any seminorm on $X$, then
\[
\frac{m(x)+m(y)}{1+m(x)+m(y)}-\frac{m(x+y)}{1+m(x+y)}=\frac{m(x)+m(y)-m(x+y)}{(1+m(x)+m(y))(1+m(x+y))} \geq 0,
\]
so
\[
\frac{m(x+y)}{1+m(x+y)} \leq \frac{m(x)+m(y)}{1+m(x)+m(y)}.
\]
Also, it is straightforward to check that the function $f:[0,\infty) \to [0,\infty)$ defined by $f(a)=\frac{a}{1+a}$ satisfies $f(a+b) \leq f(a)+f(b)$. 
Define $d_0(x)=d(x,0)$. If $x,y \in X$, then
\begin{align*}
d_0(x+y)&=\sum_{n=1}^\infty c_n \frac{m_n(x+y)}{1+m_n(x+y)}\\
&\leq \sum_{n=1}^\infty c_n \frac{m_n(x)+m_n(y)}{1+m_n(x)+m_n(y)}\\
&\leq \sum_{n=1}^\infty c_n \frac{m_n(x)}{1+m_n(x)} + c_n \frac{m_n(y)}{1+m_n(y)}\\
&=d_0(x)+d_0(y).
\end{align*}
Hence, for $x,y \in X$,
\[
d(x,z) = d_0(x-y+y-z) \leq d_0(x-y)+d_0(y-z) = d(x,y)+d(y,z),
\]
showing that $d$ satisfies the triangle inequality.

If $d(x,y)=0$, then 
\[
\sum_{n=1}^\infty c_n \frac{m_n(x-y)}{1+m_n(x-y)}=0.
\]
As each term is nonnegative, each term must be equal to $0$. As each $c_n$ is positive, this implies that each $m_n(x-y)$ is equal to $0$.
But $\{m_n: n \in \mathbb{N}\}$ is a separating family so if $x-y \neq 0$ then there is some $m_n$ with $m_n(x-y) \neq 0$, and this shows that
$x-y=0$, i.e. $x=y$. Therefore $d$ is a metric on $X$.

If $x_0 \in X$, then $d(x+x_0,y+x_0)=d(x,y)$: the metric $d$ is translation invariant.

If $|\alpha| \leq 1$ and $x \in X$, then
\begin{align*}
d_0(\alpha x)&=\sum_{n=1}^\infty c_n \frac{m_n(\alpha x)}{1+m_n(\alpha x)}\\
&=\sum_{n=1}^\infty c_n \frac{|\alpha| m_n(x)}{1+|\alpha| m_n(x)}\\
&=\sum_{n=1}^\infty c_n \frac{m_n(x)}{\frac{1}{|\alpha|} + m_n(x)}\\
&\leq \sum_{n=1}^\infty c_n \frac{m_n(x)}{1+m_n(x)}\\
&=d_0(x).
\end{align*}
Thus, if $d(x,0)<\epsilon$ and $|\alpha| \leq 1$ then $d(\alpha x,0)<\epsilon$, so the open ball
\[
\{x \in X: d(x,0)<\epsilon\}
\]
 is balanced.
 
 $(X,\tau)$ has a local basis at $0$ whose elements are finite intersections of sets of the form $\{x \in X: m_n(x)<\epsilon\}$. 
 Suppose that $\epsilon>0$,
  let $N$ be large enough so that $\sum_{n=N+1}^\infty c_n<\frac{\epsilon}{2}$, and let
  $M$ be large enough so that $\frac{1}{M} \sum_{n=1}^N c_n<\frac{1}{2}$. 
 If $x \in \bigcap_{n=1}^N \{y \in X: m_n(y)<\frac{\epsilon}{M}\}$, then 
 \begin{align*}
 d(x,0) &= \sum_{n=1}^N c_n \frac{m_n(x)}{1+m_n(x)} + \sum_{n=N+1}^\infty c_n \frac{m_n(x)}{1+m_n(x)}\\
 &< \sum_{n=1}^N c_n m_n(x) + \sum_{n=N+1}^\infty c_n\\
 &< \sum_{n=1}^N c_n \frac{\epsilon}{M} + \frac{\epsilon}{2}\\
 &<\frac{\epsilon}{2}+\frac{\epsilon}{2}\\
 &=\epsilon.
 \end{align*}
 This shows that 
 \[
 \bigcap_{n=1}^N \left\{x \in X: m_n(x)<\frac{\epsilon}{M}\right\} \subseteq \{x \in X: d(x,0)<\epsilon\},
 \]
 and this entails that $\tau$ is at least as fine as the metric topology induced by $d$.
 
Suppose that $0<\epsilon<\frac{1}{2}$ and  $N \in \mathbb{N}$. If $d(x,0) < c_N \epsilon$, then  of course
for each $n$  we have
\[
c_n \frac{m_n(x)}{1+m_n(x)} <c_N \epsilon,
\]
and hence if $1 \leq n \leq N$ then
\[
\frac{m_n(x)}{1+m_n(x)} < \frac{c_N}{c_n} \epsilon \leq \epsilon,
\]
and hence if $1 \leq n \leq N$ then
\[
m_n(x) < \frac{\epsilon}{1-\epsilon} < 2\epsilon.
\]
Therefore,
\[
\{x \in X: d(x,0)< c_N \epsilon\} \subseteq \bigcap_{n=1}^N \{x \in X: m_n(x)<2\epsilon\}.
\]
It follows from this that the metric topology induced by $d$ is at least as fine as $\tau$.
\end{proof}

If a locally convex space is metrizable with a complete metric, then it is called a {\em Fr\'echet space}.

We now prove conditions under which a topological vector space is
normable. 

\begin{theorem}
A topological vector space $(X,\tau)$ is normable if and only if there is a convex bounded open neighborhood of the origin.
\label{normable}
\end{theorem}
\begin{proof}
Suppose that $V$ is a convex bounded open neighborhood of $0$. $V$ contains a 
balanced convex open neighborhood $U$ of $0$,\footnote{Walter Rudin,
{\em Functional Analysis}, second ed., p.~12, Theorem 1.14.} and because $V$ is bounded so is $U$. We define
$\norm{x}=\mu_U(x)$, where $\mu_U$ is the Minkowski functional of $U$.  If $x \neq 0$,
then because $N=X \setminus \{x\}$ is an open neighborhood of $0$ and $U$ is bounded, there is some $t>0$ such that $U \subseteq tN$. Hence
$x \not \in \frac{1}{t}U$, i.e., $tx \not \in U$. As $U$ is balanced, 
by Lemma \ref{minkowski} we get $\mu_U(tx) \geq 1$. $\mu_U$ is a seminorm, so $\mu_U(x) \geq \frac{1}{t}>0$, showing that if $x \neq 0$ then $\mu_U(x) >0$, and hence
that $\norm{\cdot}$ is a norm on $X$. 
Also, we check that
\[
\{x \in X: \norm{x} < r\} = rU.
\]
Because $U$ is bounded, for any open neighborhood $N$ of $0$ there is some $t>0$ such that $U \subseteq tN$, hence
\[
\left\{x \in X: \norm{x} < \frac{1}{t} \right\} \subseteq N.
\]
This implies that the norm topology for $\norm{\cdot}$ is at least as fine as $\tau$.
And $\{x \in X: \norm{x} < r\}=rU$ is an open set because scalar multiplication is continuous, so $\tau$ is at least as fine as the norm topology for $\norm{\cdot}$. Therefore
that $(X,\tau)$ is normable with the norm $\norm{\cdot}$.

In the other direction, 
if $\tau$ is the norm topology for some norm $\norm{\cdot}$ on $X$, then 
\[
U=\{x \in X: \norm{x} < 1\}
\]
is indeed a convex open neighborhood of the origin. Suppose that $N$ is an open neighborhood of $0$. There is some
$r>0$ such that
\[
\{x \in X: \norm{x}<r\} \subseteq N,
\] 
and thus such that $U \subseteq \frac{1}{r}N$, and hence $U$ is bounded, showing that there exists a convex bounded open neighborhood of the origin.
\end{proof}


A topological vector space is called {\em locally bounded} if there is a bounded open neighborhood of the origin. A topological vector
space is said to have the {\em Heine-Borel property} if every closed and bounded subset of it is compact. 

\begin{theorem}
If $X$ is a topological vector space that is locally bounded and has the Heine-Borel property, then $X$ has finite dimension.
\label{finitedim}
\end{theorem}
\begin{proof}
Let $V$ be a bounded neighborhood of $0$. It is a fact that the closure of a bounded set is itself bounded,\footnote{Walter Rudin,
{\em Functional Analysis}, second ed., p.~11, Theorem 1.13(f).} and therefore $\overline{V}$ is compact. 
For any $x \in X$, the set $x+\overline{V}$ is a compact neighborhood of $x$, hence $X$ is locally compact. 
But a locally compact topological vector space is finite dimensional,\footnote{Walter Rudin, {\em Functional Analysis}, second ed.,
p.~17, Theorem 1.22.} so $X$ is finite dimensional.
\end{proof}




\section{Continuous functions on the unit disc}
Let $D=\{z \in \mathbb{C}:|z|<1\}$, the open unit disc.
Let $C(D)$ be the set of continuous functions $D \to \mathbb{C}$.  $C(D)$ is a complex vector space.
If $K$ is a compact subset of $D$, define
\[
\nu_K(f)=\sup \{ | f(z)|: z \in K\}, \qquad f \in C(D).
\]
It is straightforward to check that $\nu_K$ is a seminorm on $C(D)$.
If $f \in C(D)$ is nonzero then there is some $z \in D$ with $f(z) \neq 0$, and hence
$\nu_{\{z\}}(f) =|f(z)| >0$, so the set of all $\nu_K$ is a separating family of seminorms on $C(D)$. Thus, $C(D)$ with the seminorm topology induced by
the set of all $\nu_K$ is a locally convex space.



Define $K_n=\{z \in \mathbb{C}: |z| \leq 1-\frac{1}{n}\}$, $n \geq 1$. If $K$ is a compact subset of $D$, then there is some $n$ with $K \subseteq K_n$,
so $\nu_K(f) \leq \nu_{K_n}(f)$, and hence
\[
\{f \in C(D): \nu_{K_n}(f)<\epsilon\} \subseteq \{f \in C(D): \nu_K(f) < \epsilon\}. 
\]
It follows that the seminorm topology induced by $\{\nu_{K_n}:n \in \mathbb{N}\}$ is at least as fine as the seminorm topology
induced by $\{\nu_K: \textrm{$K$ is compact}\}$, thus the topologies are equal. 
Because the topology of $C(D)$ is induced by the countable family $\{\nu_{K_n}: n \in \mathbb{N} \}$,  by Theorem \ref{metrizable} it is metrizable:  for any summable nonincreasing sequence of positive real numbers $c_n$, the topology is induced by the metric
\begin{equation}
d(f,g) = \sum_{n=1}^\infty c_n \frac{\nu_{K_n}(f-g)}{1+\nu_{K_n}(f-g)}, \qquad f,g \in C(D).
\label{Cmetric}
\end{equation}
Suppose that $f_i \in C(D)$ is a Cauchy sequence. For $n \in \mathbb{N}$,  the fact that $f_i$ is a Cauchy sequence in $C(D)$ implies that
$\nu_{K_n}(f_i-f_j) \to 0$ as $i,j \to \infty$. $C(K_n)$ is a Banach space with the norm $\nu_{K_n}$, and hence there is some $f_{K_n} \in C(K_n)$ satisfying
$\nu_{K_n}(f_i-f_{K_n}) \to 0$ as $i \to \infty$. We define $f:D \to \mathbb{C}$ to be $f_{K_n}(z)$, for $z \in K_n$; this makes sense because the restriction of
$f_{K_n}$ to $K_m$ is $f_{K_m}$ if $n \geq m$. 
$f$ is continuous at each point in $D$ because for each point in $D$ there is some $K_n$ containing an open neighborhood of the point, and $f_{K_n}$ is continuous.
Hence $f \in C(D)$. Therefore $C(D)$ with the metric \eqref{Cmetric} is a complete metric space, which means that it is a Fr\'echet space.

\begin{theorem}
The topology of $C(D)$  is not induced by a  norm.
\end{theorem}
\begin{proof}
Because the topology of $C(D)$ is the seminorm topology induced by the separating family of seminorms $\{\nu_{K_n}: n \in \mathbb{N}\}$,
by Lemma \ref{boundedset} a subset $E$ of $C(D)$ is bounded if and only if each $\nu_{K_n}$ is a bounded function
on $E$, i.e., for each $n \in \mathbb{N}$ there is some $M_n$ such that $f \in E$ implies $\nu_{K_n}(f) \leq M_n$.

Suppose by contradiction that there is a  bounded convex open neighborhood $V$ of the origin. Because $\nu_{K_n}(f) \leq \nu_{K_{n+1}}(f)$ for any $f \in C(D)$, there is
some $N \in \mathbb{N}$ and some $\epsilon>0$ such that
\[
U=\{f \in C(D): \nu_{K_N}(f)<\epsilon\} \subseteq V.
\]
$V$ being bounded implies that $U$ is bounded. Let 
\[
\Delta_1 = \left\{z \in \mathbb{C}: |z|<1-\frac{1}{N}+\frac{1}{N(N+1)} \right\}, \quad \Delta_2 = \left\{z \in \mathbb{C}: 1-\frac{1}{N} < |z| < 1 \right\},
\]
and let $\phi_1,\phi_2$ be a partition of unity subordinate to this open cover of $D$. For any constant $M>0$, the restriction of $M\phi_2$ to $K_N$ is $0$ and
hence  belongs to $U$. But $\nu_{K_{N+1}}(M\phi_2)=M$, so $\nu_{K_{N+1}}$ is not a bounded function on $U$, contradicting that $U$ is bounded. Therefore, there is no bounded convex
open neighborhood of $0$. 
By Theorem \ref{normable},  this tells us that $C(D)$ is not normable.
\end{proof}



For each $n$, the set $C(K_n)$ is a Banach space with norm $\nu_{K_n}$. 
If $n \geq m$ and $f \in C(K_n)$, let $r_{n,m}(f)$ be the restriction of $f$ 
to $K_m$. For $n \geq m$, the function $r_{n,m}$ is a continuous linear map $C(K_n) \to C(K_m)$, and  if $n \geq m \geq l$ then $r_{n,l} = r_{m,l} \circ r_{n,m}$. Thus the Banach spaces $C(K_n)$ and the maps
$r_{n,m}$ are a {\em projective system} in the category of locally convex spaces, and it is a fact that any projective system in this category has a projective limit  that is unique up to unique isomorphism. 


\begin{theorem}
$C(D)=\varprojlim C(K_n)$.
\end{theorem}
\begin{proof}
Define $r_n:C(D) \to C(K_n)$ by taking $r_n(f)$ to be the restriction of $f$ to $K_n$. Each $r_n$ is continuous and linear. Certainly, if $n \geq m$ then $r_m = r_{n,m} \circ r_n$. 
Suppose the $Y$ is a locally convex space, that $\phi_n:Y \to C(K_n)$ are continuous linear maps, and that if $n \geq m$ then
\begin{equation}
\phi_m = r_{n,m} \circ \phi_n.
\label{Yspace}
\end{equation}
If $z \in K_m$ and $n \geq m$, then by \eqref{Yspace} we have $\phi_n(y)(z)=\phi_m(y)(z)$.
For $z \in D$, eventually $z \in K_n$, and 
define $\phi(y)(z)$ to be $\phi_n(y)(z)$ for any $n$ such that $z \in K_n$. For each $z \in D$ there is some $n$ such that $z$ is in the interior of $K_n$,
and the restriction of $\phi(y)$ to $K_n$ is equal to $\phi_n(y)$, hence $\phi(y)$ is continuous at $z$. Therefore $\phi(y)
\in C(D)$, so $\phi:Y \to C(D)$. 

Suppose that $y_1,y_2 \in Y$ and $\alpha \in \mathbb{C}$. If $z \in D$, 
then there is some $n$ with $z \in K_n$, and because $\phi_n$ is linear,
\[
\phi(\alpha y_1+y_2)(z) = \phi_n(\alpha y_1+y_2)(z)=\alpha \phi_n(y_1)(z)+\phi_n(y_2)(z) = \alpha \phi(y_1)(z)+\phi(y_2)(z).
\]
Therefore $\phi$ is linear. 

Suppose that $y_\alpha \in Y$ is a net with limit $y \in Y$. For $\phi(y_\alpha)$ to converge to $\phi(y)$ means that
for each $n \in \mathbb{N}$ we have $\nu_{K_n}(\phi(y_\alpha)-\phi(y)) \to 0$.  But
\[
\nu_{K_n}(\phi(y_\alpha)-\phi(y)) = \nu_{K_n}(\phi_n(y_\alpha)-\phi_n(y)),
\]
and $\phi_n(y_\alpha) \to \phi_n(y)$ because $\phi_n$ is continuous. Therefore, for each $n \in \mathbb{N}$ we have
$\nu_{K_n}(\phi(y_\alpha)-\phi(y)) \to 0$, so $\phi$ is continuous.
\end{proof}


We proved in the above theorem that the Fr\'echet space $C(D)$ is the projective limit of the Banach spaces $C(K_n)$. It is a fact that 
the projective limit of any projective system of Banach spaces is a Fr\'echet space.\footnote{J. L .Taylor, {\em Notes on locally convex topological vector spaces},
\url{http://www.math.utah.edu/~taylor/LCS.pdf}, p.~8, Proposition 2.6, and
cf. Paul Garret,
{\em Functions on circles: Fourier series, I}, \url{http://www.math.umn.edu/~garrett/m/fun/notes_2012-13/04_blevi_sobolev.pdf}, p.~37, \S 13.}
 

A topological space is said to be {\em separable} if it has a countable subset that is dense.

\begin{theorem}
$C(D)$ is separable.
\label{separable}
\end{theorem}
\begin{proof}
One proves using the Stone-Weierstrass theorem that the Banach
space $C(K_n)$ is separable.
The product of at most continuum many separable Hausdorff spaces each with at least two points is itself  separable
 with the product topology.\footnote{Stephen Willard, {\em General Topology}, p.~109, Theorem 16.4.} Therefore, $\prod_{n=1}^\infty C(K_n)$ is separable. 
Because each $C(K_n)$ is a metric space, this countable product $\prod_{n=1}^\infty C(K_n)$ is metrizable, and any subset of a separable metric space is itself separable with the subspace
topology.
The projective limit of a projective system of topological vector spaces is a closed subspace
of the product of the spaces; thus, using merely that the projective limit is a subset of the product $\prod_{n=1}^\infty C(K_n)$ and has the subspace topology inherited from the direct
product, 
we get that $C(D)$ is separable.
\end{proof}





\section{Holomorphic functions on the unit disc}
Let $H(D)$ be the set of holomorphic functions $D \to \mathbb{C}$. $H(D)$ is a linear subspace of $C(D)$. 
Let $H(D)$ have the subspace topology inherited from $C(D)$. One proves that this topology is equal to the seminorm topology induced 
by $\{\nu_{K_n}: n \in \mathbb{N}\}$.  Any subset of a separable metric space with the subspace topology is separable. By Theorem \ref{separable} the Fr\'echet space
$C(D)$ is separable, and thus $H(D)$ is separable too.

We now prove that $H(D)$ is a closed subspace of $C(D)$.\footnote{Paul Garrett, {\em Holomorphic vector-valued functions}, \url{http://www.math.umn.edu/~garrett/m/fun/notes_2012-13/08b_vv_holo.pdf}} A closed linear subspace of a Fr\'echet space is itself a Fr\'echet space, hence this theorem shows that $H(D)$ is a Fr\'echet space.

\begin{theorem}
$H(D)$ is a closed subset of $C(D)$.
\end{theorem}
\begin{proof}
Suppose that $f_j \in H(D)$ is a net and that $f_j \to f \in C(D)$. We shall show that $f \in H(D)$. (In fact it suffices to prove this for a sequence of elements in $H(D)$ because
we have shown that $C(D)$ is metrizable, but that will not simplify this argument.) To show this we have to prove that if $z \in D$ then $\frac{f(z+h)-f(z)}{h}$ has a limit as $h \to 0$,
$h \in \mathbb{C}$. Let $\gamma$ be a counterclockwise oriented circle contained in $D$ with center $z$, say of radius $r=\frac{1-|z|}{2}>0$. For each $j$ the function $f_j$ is holomorphic on $D$,  and so
Cauchy's integral formula gives
\[
f_j(w)=\frac{1}{2\pi i} \int_\gamma \frac{f_j(\zeta)}{\zeta-w} d\zeta, \qquad w \in B_r(z).
\]
Therefore
\begin{eqnarray*}
f(w)-\frac{1}{2\pi i} \int_\gamma \frac{f(\zeta)}{\zeta-w} d\zeta&=&f(w)-f_j(w)+\frac{1}{2\pi i} \int_\gamma \frac{f_j(\zeta)}{\zeta-w} d\zeta-\frac{1}{2\pi i} \int_\gamma \frac{f(\zeta)}{\zeta-w} d\zeta\\
&=&f(w)-f_j(w)+\frac{1}{2\pi i} \int_\gamma \frac{f_j(w)-f(w)}{\zeta-w} d\zeta.
\end{eqnarray*}
As $\gamma$ is a compact subset of $D$ this gives us
\[
\left| f(w)-\frac{1}{2\pi i} \int_\gamma \frac{f(\zeta)}{\zeta-w} d\zeta \right| \leq |f(w)-f_j(w)|+ \frac{1}{2\pi}\cdot 2\pi r \cdot \frac{\nu_\gamma(f_j-f)}{r-|w-z|}.
\]
The right-hand side tends to 0, while the left-hand side does not depend on $j$. Hence 
\begin{equation}
f(w)=\frac{1}{2\pi i} \int_\gamma \frac{f(\zeta)}{\zeta-w} d\zeta, \qquad w \in B_r(z).
\label{cauchy}
\end{equation}
Applying \eqref{cauchy}, we have for $0 \leq |h|<r$,
\[
f(z+h)=\frac{1}{2\pi i} \int_\gamma \frac{f(\zeta)}{\zeta-(z+h)} d\zeta,
\]
hence
\begin{eqnarray*}
f(z+h)-f(z)&=&\frac{1}{2\pi i} \int_\gamma \frac{f(\zeta)}{\zeta-(z+h)} d\zeta- \frac{1}{2\pi i} \int_\gamma \frac{f(\zeta)}{\zeta-z} d\zeta\\
&=&\frac{1}{2\pi i} \int_\gamma f(\zeta)\left( \frac{1}{\zeta-(z+h)} - \frac{1}{\zeta-z} \right) d\zeta\\
&=&\frac{1}{2\pi i}\int_\gamma f(\zeta) \cdot \frac{h}{(\zeta-(z+h))(\zeta-z)} d\zeta,
\end{eqnarray*}
thus
\[
\frac{f(z+h)-f(z)}{h} = \frac{1}{2\pi i} \int_\gamma \frac{f(\zeta)}{(\zeta-(z+h))(\zeta-z)} d\zeta.
\]
For $\zeta \in \gamma$ we have $\left| \frac{f(\zeta)}{(\zeta-(z+h))(\zeta-z)} \right| \leq \frac{\nu_\gamma(f)}{(r-|h|)r}$, and so by the dominated convergence theorem we get
\[
\lim_{h \to 0} \frac{f(z+h)-f(z)}{h} = \frac{1}{2\pi i} \int_\gamma \frac{f(\zeta)}{(\zeta-z)^2} d\zeta.
\]
Thus, for every $z \in D$, the function $f$ is complex differentiable at $z$. Hence $f \in H(D)$, and therefore $H(D)$ is  a closed subset of $C(D)$. 
\end{proof}


We remind ourselves that a topological vector space is said to have the {\em Heine-Borel property} if every closed and bounded subset of it is compact. 
 Lemma \ref{boundedset} tells us that a subset $E$ of $H(D)$ is bounded if and only if each seminorm $\nu_{K_n}$ is a bounded function on $E$. 
 The following theorem states that $H(D)$ has the Heine-Borel property.\footnote{Henri Cartan, {\em Elementary Theory of Analytic Functions of One or Several Complex Variables},
pp.~162--167, chapter V, \S 4.} An equivalent statement is called {\em Montel's theorem}.

\begin{theorem}[Heine-Borel property]
The Fr\'echet space $H(D)$ has the Heine-Borel property.
\end{theorem}

That $H(D)$ has the Heine-Borel property is a useful tool, and lets us prove that the topology of $H(D)$ is not induced by a norm.

\begin{theorem}
$H(D)$ is not normable.
\end{theorem}
\begin{proof}
If $H(D)$ were normable then by Theorem \ref{normable} there would be a convex bounded open neighborhood of the origin. This would
imply that $H(D)$ is locally bounded (has a bounded open neighborhood of the origin). But $H(D)$ has the Heine-Borel property, and a topological
vector space that is locally bounded and has the Heine-Borel property is finite dimensional by Theorem \ref{finitedim}. It is straightforward to
check that $H(D)$ is not finite dimensional, and hence $H(D)$ is not normable.
\end{proof}




For $f \in H(D)$, let $(d f)(z)=\lim_{h \to 0} \frac{f(z+h)-f(z)}{h}$.
First, if $f \in H(D)$ then one proves that $df \in H(D)$.  Then,
the following theorem states that $d: H(D) \to H(D)$ is a morphism in the category of locally convex spaces.\footnote{Henri
Cartan,  {\em Elementary Theory of Analytic Functions of One or Several Complex Variables}, p.~143, chapter V, \S 1.}


\begin{theorem}
Differentiation $H(D) \to H(D)$ is a continuous linear map.
\end{theorem}

If $K$ is a compact subset of $D$ and $f \in H(D)$, let $r_K(f)$ be the restriction of $f$ to $K$, and let $\overline{H}(K)$ be the closure
in  $C(K)$ of the set $\{r_K(f):f \in H(D)\}$. Each element of $\overline{H}(K)$ is holomorphic on the interior of $K$.
$C(K)$ is a Banach space with the norm  $\nu_K$, and hence $\overline{H}(K)$ is a Banach space with the same norm, because it is indeed a linear subspace.
If $n \geq m$ and $f \in \overline{H}(K_n)$, let $r_{n,m}(f)=r_{K_m}(f) \in \overline{H}(K_m)$.
The $r_{n,m}$ are continuous and linear, and if $n \geq m \geq l$ then $r_{n,l} = r_{m,l} \circ r_{n,m}$. Thus the Banach spaces $\overline{H}(K_n)$ and the continuous linear
maps
$r_{n,m}$ are a projective system in the category of locally convex spaces, and this projective system has a projective limit $\varprojlim \overline{H}(K_n)$.
The following theorem states that this projective limit is equal to the Fr\'echet space $H(D)$.\footnote{J. L .Taylor, {\em Notes on locally convex topological vector spaces},
\url{http://www.math.utah.edu/~taylor/LCS.pdf}, p.~8} 


\begin{theorem}
$H(D)=\varprojlim \overline{H}(K_n)$.
\end{theorem}


\section{Dual spaces}
The {\em dual  of a topological vector space $X$} is the set $X^*$ of continuous linear maps $X \to \mathbb{C}$. 
If $E$ is a bounded subset of $X$ and $\lambda \in X^*$, then $\lambda(E)$ is a bounded subset of $\mathbb{C}$ (the image of a bounded set under a continuous linear map is a bounded set).
Hence
\[
p_E(\lambda) = \sup \{ |\lambda x| : x \in E\} < \infty.
\]
The function $p_E$ is a seminorm on $X^*$, and if $\lambda \neq 0$ then there is some $x \in X$ with $\lambda x \neq 0$,
 hence $p_{\{x\}}(\lambda) >0$. 
The {\em strong dual  topology on $X^*$} is the seminorm topology induced by the separating family
\[
\{p_E: \textrm{$E$ is a bounded subset of $X$}\}.
\]
(To add to our vocabulary: the set of all bounded subsets of a topological vector space is called the {\em bornology of the space}. Similar
to how one can define a topology as a collection of sets satisfying certain properties, one can also define a bornology on a set
 without first having the structure of a topological vector space.)
We denote by $X_\beta^*$ the dual space $X^*$ with the strong dual topology. $X_\beta^*$ is a locally convex space.
If $X$ is a normed space, one can prove\footnote{K. Yosida,
{\em Functional Analysis}, sixth ed., p.~111,  Theorem 1.} that $X_\beta^*$ is normable with the operator norm
\[
\norm{\lambda} = \sup \{ |\lambda x| : \norm{x} \leq 1\}.
\]
We say that a topological vector space $X$ is {\em reflexive} if $(X_\beta^*)_\beta^*=X$; since the strong dual
of a topological vector space is locally convex, for a topological vector space to be reflexive it is necessary
that it be locally convex.

Let $X$ be a locally convex space. The Hahn-Banach separation theorem\footnote{Walter Rudin, {\em Functional Analysis}, second ed., p.~59, Theorem 3.4.}
yields that $X^*$ separates $X$: if $x \neq 0$ then there is some $\lambda \in X^*$ with $\lambda x \neq 0$. If $\lambda \in X^*$, then $|\lambda|$ is a seminorm
on $X$ and $\{|\lambda|: \lambda \in X^*\}$ is therefore a separating family of seminorms on $X$. We call the seminorm topology induced by this separating family
the {\em weak topology on $X$}, and $X$ with the weak topology is a locally convex space.
The original topology on $X$ is at least as fine as the weak topology on $X$: any set that is open using
the weak topology is open using the original topology.

 
 The following lemma shows that a Fr\'echet space with the Heine-Borel property is reflexive, and therefore that $H(D)$ is reflexive.
 
 \begin{lemma}
 If a Fr\'echet space has the Heine-Borel property, then it is reflexive.
 \end{lemma}
 \begin{proof}
A subset of a locally convex space is called a {\em barrel} if it is closed, convex, balanced, and absorbing.
A locally convex space is said to be {\em barreled} if each barrel is a neighborhood of $0$. It is a fact that every Fr\'echet space is barreled.\footnote{K. Yosida,
{\em Functional Analysis}, sixth ed., p.~138, Corollary 1.}
A locally convex space is reflexive if and only if it is barreled and 
if every set that is closed, convex, balanced, and bounded is weakly compact.\footnote{K. Yosida,
{\em Functional Analysis}, sixth ed., p.~140, Theorem 2.} Therefore, for a Fr\'echet space with the Heine-Borel property
 to be reflexive it is necessary and sufficient that every set that is compact, convex, and balanced be weakly compact.
But if a subset of a locally convex space is compact then it is weakly compact, because the original topology is at least as fine as the weak topology and hence
any cover of a set by elements of the weak topology is also a cover of the set by elements of the original topology. Therefore, any Fr\'echet space with the Heine-Borel property
is reflexive.
\end{proof}


Morphisms in the category of locally convex spaces are continuous linear maps. If $X$ and $Y$ are locally convex spaces and 
$\phi:X \to Y$ is a morphism, the {\em dual of $\phi$} is the morphism
\[
\phi^*:Y_\beta^* \to X_\beta^*
\]
defined by
\[
\phi^*(\lambda)=\lambda \circ \phi, \qquad \lambda \in Y_\beta^*.
\]
One verifies that $\phi^*$ is in fact a morphism.
If the spaces $X_j$ and the morphisms $\phi_{i,j}:X_i \to X_j$, $i \geq j$, are a projective system
in the category of locally convex spaces, then the dual spaces $(X_j)_\beta^*$ and the morphisms $\phi_{i,j}^*:(X_j)_\beta^* 
\to (X_i)^*_\beta$, $i \geq j$, are a {\em direct system} in this category.
It is a fact that the dual of a projective limit of Banach spaces is isomorphic  to the direct limit
of the duals of the Banach spaces.\footnote{Paul Garrett, {\em Functions on circles: Fourier series, I},  \url{http://www.math.umn.edu/~garrett/m/fun/notes_2012-13/04_blevi_sobolev.pdf}, p.~15, Theorem 5.1.1.}
Thus, as $H(D)$ is the projective limit of the Banach spaces $\overline{H}(K_n)$, its dual space $H^*(D)=(H(D))_\beta^*$ is isomorphic to the direct limit of the duals 
of these Banach spaces:
\[
H^*(D)=\varinjlim (\overline{H}(K_n))_\beta^*.
\]

Cooper\footnote{J. B. Cooper, {\em Functional analysis-- spaces of holomorphic functions and their duality}, \url{http://www.dynamics-approx.jku.at/lena/Cooper/holloc.pdf}, p.~11, \S 5.}
shows that $H^*(D)$ is isomorphic to the space of germs of functions on the complement of $D$ in the extended complex plane that vanish at infinity. 
Let $\mathfrak{A}$ be those sequences $a \in \mathbb{C}^\mathbb{N}$ satisfying
\[
\limsup |a_n|^{1/n} \leq 1.
\]
By Hadamard's formula for the radius of convergence of a power series, these are precisely the sequences of coefficients of power series with
radius of convergence $\geq 1$, and $\mathfrak{A}$ is a complex vector space. The map
\[
a \mapsto \sum_{n=0}^\infty a_n z^n
\]
is linear and has the linear inverse
\[
f \mapsto \left( \frac{f^{(n)}(0)}{n!} \right),
\]
so $H(D)$ and $\mathfrak{A}$ are linearly isomorphic. 
For $0<r<1$, define 
\[
q_r(a) = \max \{|a_n| r^n : n \in \mathbb{N}\}.
\]
Each $q_r$ is a norm, yet we do not give $\mathfrak{A}$ the norm topology. Rather, we give $\mathfrak{A}$ the seminorm topology induced by 
the family $\{ q_r : 0<r<1 \}$, and with this topology $\mathfrak{A}$ is a  locally convex space. 
One proves that the above two linear maps are continuous, and hence that $H(D)$ is isomorphic as a locally
convex space to $\mathfrak{A}$. Then, one proves that the dual space of $\mathfrak{A}$ are those
sequences $b \in \mathbb{C}^\mathbb{N}$ such that
\[
\limsup |b_n|^{1/n}<1,
\]
and $b$ corresponds to
\[
\sum_{n=0}^\infty b_n \left( \frac{1}{z} \right)^{n+1}.
\]

\end{document}