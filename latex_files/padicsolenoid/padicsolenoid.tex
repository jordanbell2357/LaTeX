\documentclass{article}
\usepackage{amsmath,amssymb,mathrsfs,amsthm}
\usepackage{tikz-cd}
\usepackage{hyperref}
\newcommand{\inner}[2]{\left\langle #1, #2 \right\rangle}
\newcommand{\tr}{\ensuremath\mathrm{tr}\,} 
\newcommand{\Span}{\ensuremath\mathrm{span}} 
\def\Re{\ensuremath{\mathrm{Re}}\,}
\def\Im{\ensuremath{\mathrm{Im}}\,}
\newcommand{\id}{\ensuremath\mathrm{id}} 
\newcommand{\diam}{\ensuremath\mathrm{diam}} 
\newcommand{\lcm}{\ensuremath\mathrm{lcm}} 
\newcommand{\supp}{\ensuremath\mathrm{supp}\,}
\newcommand{\dom}{\ensuremath\mathrm{dom}\,}
\newcommand{\upto}{\nearrow}
\newcommand{\downto}{\searrow}
\newcommand{\norm}[1]{\left\Vert #1 \right\Vert}
\newtheorem{theorem}{Theorem}
\newtheorem{lemma}[theorem]{Lemma}
\newtheorem{proposition}[theorem]{Proposition}
\newtheorem{corollary}[theorem]{Corollary}
\theoremstyle{definition}
\newtheorem{definition}[theorem]{Definition}
\newtheorem{example}[theorem]{Example}
\begin{document}
\title{The $p$-adic solenoid}
\author{Jordan Bell\\ \texttt{jordan.bell@gmail.com}\\Department of Mathematics, University of Toronto}
\date{\today}

\maketitle
\section{Definition}
We shall be speaking about locally compact abelian groups, and unless we say otherwise,
by \textbf{morphism} we mean a continuous group homomorphism. 

For $p$ prime and $n \in \mathbb{Z}_{\geq 0}$, $p^n\mathbb{Z}$ is a closed subgroup of the locally compact abelian group
$\mathbb{R}$, and the quotient $\mathbb{R}/p^n \mathbb{Z}$ is a compact abelian group.
For $n \geq m$, let $\phi_{n,m}:\mathbb{R}/p^n \mathbb{Z} \to \mathbb{R}/p^m \mathbb{Z}$
be the projection map, which is a morphism. The compact abelian groups $\mathbb{R}/p^n \mathbb{Z}$ and the morphisms $\phi_{n,m}$
are an inverse system, and the inverse limit is a compact abelian group denoted $\mathbb{T}_p$, called the \textbf{$p$-adic solenoid},
with morphisms $\phi_n:\mathbb{T}_p \to \mathbb{R}/p^n \mathbb{Z}$. Because the maps $\phi_{n,m}:\mathbb{R}/p^n \mathbb{Z} \to \mathbb{R}/p^m \mathbb{Z}$
are surjective, the maps $\phi_n:\mathbb{T}_p \to \mathbb{R}/p^n\mathbb{Z}$ are surjective.\footnote{Alain M. Robert, {\em A Course in $p$-adic Analysis},
Chapter 1, \S 4, p.~29.} 

Let $\pi_n:\mathbb{R} \to \mathbb{R}/p^n\mathbb{Z}$ be the projection map, which is a  morphism. The projection maps $\pi_n$ are compatible with the inverse
system $\phi_{n,m}$, so there is a unique morphism $\pi:\mathbb{R} \to \mathbb{T}_p$ such that $\phi_n \circ \pi = \pi_n$ for all $n \in \mathbb{Z}_{\geq 0}$. 
If $x,y \in \mathbb{R}$ are distinct, then for sufficiently large $n$ we have $\pi_n(x) \neq \pi_n(y)$. If $\pi(x) = \pi(y)$ then
$\pi_n(x)=\phi_n(\pi(x))=\phi_n(\pi(y)) = \pi_n(y)$, a contradiction. Therefore $\pi:\mathbb{R} \to \mathbb{T}_p$ is injective. 
Furthermore,  the maps $\pi_n:\mathbb{R} \to \mathbb{R}/p^n \mathbb{Z}$ being surjective implies that the image $\pi(\mathbb{R})$ is dense in
$\mathbb{T}_p$.\footnote{Luis Ribes and Pavel Zalesskii, {\em Profinite Groups}, p.~7, Lemma 1.1.7.}

\section{Pontryagin dual}
If $G$ is a locally compact abelian group, we denote
by $G^*$ the collection of morphisms $G \to S^1$. We assign $G^*$ the coarsest topology such that for all
$g \in G$, the map $\gamma \mapsto \gamma(x)$ is continuous $G^* \to S^1$, and with this topology, $G^*$ is a locally compact
abelian group, called the \textbf{Pontryagin dual of $G$}.

If $\phi:G \to H$ is a morpism of locally compact abelian groups, then $\phi^*:H^* \to G^*$ defined by
\[
\phi^*(\theta)(g) = \theta(\phi(g)), \qquad \theta \in H^*, \quad g\in G,
\]
is a morphism. Say $\phi$ is surjective, and  $\phi^*(\theta_1)=\phi^*(\theta_2)$ but that $\theta_1 \neq \theta_2$. Then there
is some $h \in H$ such that $\theta_1(h) \neq \theta_2(h)$. Since $\phi:G \to H$ is surjective, there is some $g \in G$ such that
$\phi(g)=h$. But then
\[
\theta_1(h) = \theta_2(\phi(g))=\phi^*(\theta_1)(g) = \phi^*(\theta_2)(g) = \theta_2(\phi(g)) = \theta_2(h),
\]
contradicting $\theta_1(h) \neq \theta_2(h)$. Therefore, if $\phi:G \to H$ is surjective then $\phi^*:H^* \to G^*$ is injective.

Let
\[
\frac{1}{p^n} \mathbb{Z} = \left\{ \frac{j}{p^n}: j \in \mathbb{Z} \right\} \subset \mathbb{Q},
\]
which with the discrete topology is a discrete abelian group. 

\begin{theorem}
For prime $p$ and $n \in \mathbb{Z}_{\geq 0}$, the map
$\Phi_n:\frac{1}{p^n} \mathbb{Z} \to  (\mathbb{R}/p^n \mathbb{Z})^*$ defined by
\[
\Phi_n(a)(x+p^n \mathbb{Z}) = e^{2\pi i ax}, \qquad a \in \frac{1}{p^n}\mathbb{Z},\quad
x + p^n \mathbb{Z} \in \mathbb{R}/p^n \mathbb{Z},
\]
is an isomorphism of topological groups.
\end{theorem}
\begin{proof}
Write $a=\frac{j}{p^k}$, $j \in \mathbb{Z}$. 
If $x+p^n \mathbb{Z} = y+p^n \mathbb{Z}$, then $x-y \in  p^n \mathbb{Z}$, so $x-y=p^nk$ for some $k \in \mathbb{Z}$. Then
\[
\Phi_n(a)(x+p^n\mathbb{Z}) = e^{2\pi iax} = e^{2\pi i \frac{j}{p^n}(p^nk+y)} = e^{2\pi ik + 2\pi i \frac{j}{p^n}y}
=e^{2\pi i ay} = \Phi_n(a)(y+p^n\mathbb{Z}),
\]
showing that $\Phi_n$ is well-defined. Furthermore, one checks that indeed $\Phi_n(a) \in (\mathbb{R}/p^n \mathbb{Z})^*$ for each
$a \in \frac{1}{p^n} \mathbb{Z}$. 

It is apparent that $\Phi_n(a+b)=\Phi_n(a) \cdot \Phi_n(b)$. $\Phi$ is continuous because $\frac{1}{p^n} \mathbb{Z}$ is discrete. If 
$\Phi_n(a)=\Phi_n(b)$, this means that for all $x + p^n \mathbb{Z} \in \mathbb{R}/p^n \mathbb{Z}$, $e^{2\pi i ax}=e^{2\pi ibx}$,
equivalently, that $(a-b)x \in \mathbb{Z}$ for all $x \in \mathbb{R}$, whence $a=b$. Thus $\Phi_n$ is injective. 

Let $\gamma \in (\mathbb{R}/p^n \mathbb{Z})^*$. Define $\Gamma:\mathbb{R} \to S^1$ by $\Gamma = \gamma \circ \pi_n$,
so that $\Gamma \in \mathbb{R}^*$. We take as given that  because $\Gamma \in \mathbb{R}^*$, there is some
 $y \in \mathbb{R}$ such that 
$\Gamma(x)=e^{2\pi iyx}$ for all $x \in \mathbb{R}$. In particular, for $x=p^n$, on the one hand
\[
\Gamma(p^n)=\gamma(\pi_n(p^n)) = \gamma(0+p^n \mathbb{Z}) = 1,
\]
and on the other hand
\[
\Gamma(p^n) = e^{2\pi iy p^n},
\]
so $y p^n \in \mathbb{Z}$, i.e. $y \in \frac{1}{p^n} \mathbb{Z}$, and it follows that $\gamma = \Phi_n(y)$. Therefore $\Phi_n$ is surjective.


The \textbf{open mapping theorem for topological groups} states that if $G,H$ are locally compact groups, $f:G \to H$
is a surjective morphism, and $G$ is $\sigma$-compact, then $f$ is open. $\mathbb{Z}$ is discrete and countable, hence is $\sigma$-compact, so
$\Phi_n$ is open. Therefore $\Phi_n$ is an isomorphism of topological groups.
\end{proof}

Because the morphisms $\phi_{n,m}:\mathbb{R}/p^n \mathbb{Z} \to \mathbb{R}/p^m \mathbb{Z}$ are surjective,
the morphisms $\phi_{n,m}^*: (\mathbb{R}/p^m \mathbb{Z})^* \to  (\mathbb{R}/p^n \mathbb{Z})^*$ are injective. 
For $m \leq n$,
define $\iota_{m,n}: \frac{1}{p^m} \mathbb{Z} \to \frac{1}{p^n}\mathbb{Z}$ by
$\iota\left(\frac{j}{p^m}\right)=\frac{j}{p^m}=\frac{p^{n-m}j}{p^n} \in \frac{1}{p^n} \mathbb{Z}$; this is an injective morphism.
One checks that the following diagram commutes.
 
 \begin{center}
\begin{tikzcd}
 (\mathbb{R}/p^m \mathbb{Z})^* \arrow{r}{\phi_{n,m}^*} 
&(\mathbb{R}/p^n\mathbb{Z})^*\\
\frac{1}{p^m}\mathbb{Z} \arrow{r}{\iota_{m,n}} \arrow{u}{\Phi_m}&\frac{1}{p^n}\mathbb{Z} \arrow{u}{\Phi_n}
\end{tikzcd}
\end{center}
The discrete groups $\frac{1}{p^m}\mathbb{Z}$ and the morphisms $\iota_{m,n}$ are a direct system. The \textbf{localization of $\mathbb{Z}$
away from $p$} is the abelian group
\[
\mathbb{Z}[1/p] = \left\{ \frac{j}{p^m}: j \in \mathbb{Z}, m \in \mathbb{Z}_{\geq 0}\right\} \subset \mathbb{Q}.
\]
We assign $\mathbb{Z}[1/p]$ the discrete topology. One proves that
$\mathbb{Z}[1/p]$ with the maps $\iota_m:\frac{1}{p^m}\mathbb{Z} \to \mathbb{Z}[1/p]$ defined by
\[
\iota_m\left(\frac{j}{p^m}\right) = \frac{j}{p^m}
\]
is the direct limit of this direct system.\footnote{A direct limit of discrete abelian groups
is the direct limit of abelian groups. On direct limits of abelian groups, cf. Luis Ribes and Pavel Zalesskii, {\em Profinite Groups}, p.~15, Proposition 1.2.1.}
The direct system $\iota_{m,n}:\frac{1}{p^m} \mathbb{Z} \to \frac{1}{p^n} \mathbb{Z}$
is dual to the inverse system $\phi_{n,m}:\mathbb{R}/p^n\mathbb{Z} \to \mathbb{R}/p^m \mathbb{Z}$. It follows that
the Pontryagin dual of the limit of either system  is isomorphic as a topological group to the limit of the other system. That is,
\[
\mathbb{T}_p^* \cong \mathbb{Z}[1/p], \qquad (\mathbb{Z}[1/p])^* \cong \mathbb{T}_p,
\]
as topological groups.


\section{$p$-adic integers}
For $n \geq m$, let $\psi_{n,m}:\mathbb{Z}/p^n \mathbb{Z} \to \mathbb{Z}/p^m \mathbb{Z}$ be the projection map. With the discrete topology, 
$\mathbb{Z}/p^n\mathbb{Z}$ is a compact abelian group, as it is finite. Then $\psi_{n,m}:\mathbb{Z}/p^n \mathbb{Z} \to \mathbb{Z}/p^m\mathbb{Z}$ is an inverse system,
and its inverse limit is a compact abelian group denoted $\mathbb{Z}_p$, called the \textbf{$p$-adic integers}, with morphisms $\psi_n:\mathbb{Z}_p \to \mathbb{Z}/p^n\mathbb{Z}$. 
Because the morphisms $\psi_{n,m}$ are surjective, the morphisms $\psi_n$ are surjective.

Let $\lambda_n:\mathbb{Z}/p^n \mathbb{Z} \to \mathbb{R}/p^n\mathbb{Z}$ be the inclusion map. Then the morphisms $\Lambda_n = \lambda_n \circ \psi_n:\mathbb{Z}_p \to 
\mathbb{R}/p^n \mathbb{Z}$ are compatible with the inverse system $\phi_{n,m}:\mathbb{R}/p^n \mathbb{Z} \to \mathbb{R}/p^m \mathbb{Z}$, so there is a unique morphism
$\Lambda:\mathbb{Z}_p \to \mathbb{T}_p$ such that $\phi_n \circ \Lambda = \Lambda_n$ for all $n \in \mathbb{Z}_{\geq 0}$. 
Suppose that $x,y \in \mathbb{Z}_p$ are distinct and that $\Lambda(x) = \Lambda(y)$. It is a fact that there is some $n$ such that $\psi_n(x) \neq \psi_n(y)$.  Because $\lambda_n$ is injective,
this implies that $\Lambda_n(x) \neq \Lambda_n(y)$, and this contradicts that $\Lambda(x) = \Lambda(y)$. Therefore $\Lambda:\mathbb{Z}_p
\to \mathbb{T}_p$ is injective.


It can be proved that $\ker \phi_0 = \Lambda(\mathbb{Z}_p)$, which implies that
\[
0 \to \mathbb{Z}_p \to \mathbb{T}_p \to \mathbb{R}/\mathbb{Z} \to 0
\]
is a short exact sequence of topological groups.\footnote{Alain M. Robert, {\em A Course in $p$-adic Analysis},
Chapter 1, Appendix, p.~55.} 

It can be proved that for each $m \in \mathbb{Z}_{>0}$ such that $\gcd(m,p)=1$, the $p$-adic solenoid $\mathbb{T}_p$ has a unique cyclic subgroup of order $m$,
and on the other hand that there is no element in $\mathbb{T}_p$ whose order is a power of $p$, namely, $\mathbb{T}_p$ has no $p$-torsion.\footnote{Alain M. Robert, {\em A Course in $p$-adic Analysis},
Chapter 1, Appendix, pp.~55--56.} 



\section{Further reading}
Garrett has written several notes on the $p$-adic solenoid.\footnote{Paul Garrett, {\em Solenoids}, \url{http://www.math.umn.edu/~garrett/m/mfms/notes/02_solenoids.pdf};
Paul Garrett, {\em Bigger diagrams for solenoids, more automorphisms, colimits}, \url{http://www.math.umn.edu/~garrett/m/mfms/notes/03_more_autos.pdf};
Paul Garrett, {\em The ur-solenoid and the adeles}, \url{http://www.math.umn.edu/~garrett/m/mfms/notes/04_ur_solenoid.pdf}}
The $p$-adic solenoid occurs in several places in the books of Hofmann and Morris.\footnote{Karl H. Hofmann and Sidney A. Morris, {\em The Structure of Compact
Groups}, 2nd revised and augmented edition; Karl H. Hofmann and Sidney A. Morris, {\em The Lie Theory of Connected
Pro-Lie Groups}; see also H. Salzmann, T. Grundh\"ofer, H. H\"ahl, and R. L\"owen, {\em The Classical Fields: Structural Features of the Real and Rational Numbers},
p.~99.}
For properties of the $p$-adic solenoid involving homological algebra, see the below references.\footnote{For $\mathrm{Ext}(\mathbb{Z},\mathbb{T}_p)$ see Jean Dieudonn\'e,
{\em A History of Algebraic and Differential Topology, 1900 -- 1960}, p.~94; see also
J. M. Cordier and T. Porter, {\em Shape Theory: Categorical Methods of Approximation}, p.~83; and \url{http://mathoverflow.net/questions/4478/torsion-in-homology-or-fundamental-group-of-subsets-of-euclidean-3-space}}

\end{document}