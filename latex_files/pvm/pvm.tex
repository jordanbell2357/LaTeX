\documentclass{article}
\usepackage{amsmath,amssymb,graphicx,subfig,mathrsfs,amsthm}
%\usepackage{tikz-cd}
\newcommand{\inner}[2]{\left\langle #1, #2 \right\rangle}
\newcommand{\tr}{\textrm{tr}} 
\newcommand{\im}{\textrm{im\,}} 
\newcommand{\Span}{\textrm{span}} 
\newcommand{\SA}{B_{\textrm{sa}}(H)} 
\newcommand{\positive}{B_{\textrm{+}}(H)} 
\newcommand{\id}{\textrm{id}} 
\newcommand{\norm}[1]{\left\Vert #1 \right\Vert}
\newtheorem{theorem}{Theorem}
\newtheorem{lemma}[theorem]{Lemma}
\newtheorem{corollary}[theorem]{Corollary}
\begin{document}
\title{Projection-valued measures and spectral integrals}
\author{Jordan Bell\\ \texttt{jordan.bell@gmail.com}\\Department of Mathematics, University of Toronto}
\date{\today}
\maketitle
\begin{abstract}
The purpose of these notes is to precisely define all the objects one needs to talk about projection-valued measures and the spectral theorem. 
\end{abstract}



\section{Orthogonal complements}
Let $H$ be a separable complex Hilbert space. 

If $W$ is a subset of $H$, the {\em span} of $W$ is the set of all finite linear combinations of elements of $W$. 
If $W_\alpha$, $\alpha \in I$, are subsets of $H$, let 
\[
\bigvee_{\alpha \in I} W_\alpha
\]
denote the closure of the span of $\bigcup_{\alpha \in I} W_\alpha$. It is straightforward to check that this is the intersection of all closed subspaces of $H$ that contain each of the $W_\alpha$.\footnote{Let $\mathfrak{L}$ be the set of all closed subspaces of $H$. Set inclusion $\subseteq$ is a partial order on $\mathfrak{L}$. $\mathfrak{L}$ is 
a {\em lattice}: if $M,N \in \mathfrak{L}$, then $M \cap N$ is the greatest element in $\mathfrak{L}$ that is contained both in $M$ and in $N$, and
$M \vee N$ is the least element in $\mathfrak{L}$ that contains each of $M$ and $N$. $\mathfrak{L}$ is a {\em complete} lattice: if $M_\alpha \in \mathfrak{L}$,
then $\bigcup_\alpha M_\alpha$ is the greatest element in $\mathfrak{L}$ that is contained in each $M_\alpha$, and 
$\bigvee_\alpha M_\alpha$ is the least element in $\mathfrak{L}$ that contains each $M_\alpha$. (For comparison, the set of finite dimensional subspaces of an infinite
dimensional Hilbert space is a lattice but is not a complete lattice.)}

If $V$ is a subspace of $H$, define
\[
V^\perp=\{f \in H: \textrm{$\inner{f}{g}=0$ for all $g \in V$}\},
\]
called the {\em orthogonal complement} of $V$.
If $V$ is a subspace of $H$ then $V^\perp$ is a closed subspace of $H$.



Let $V$ be a closed subspace of $H$ and let $f \in H$. It is a fact that there is a unique $g_0 \in V$ such that 
\[
\norm{f-g_0}=\inf_{g \in V} \norm{f-g},
\]
and that $f - g_0 \in V^\perp$. 
Then, $f=g_0+(f-g_0) \in V+V^\perp$. Therefore, if
$V$ is a closed subspace of $H$ then
\begin{equation}
\label{Hdecomp}
H=V + V^\perp.
\end{equation}

If $V$ and $W$ are subspaces of $H$, we write $V \perp W$ if $\inner{v}{w}=0$ for all $v \in V, w \in W$. If $V,W$ are closed subspaces of $H$ and
$V \perp W$, we
can show that $V+W=V \vee W$, and thus $V+W$ is in particular a closed subspace of $H$.\footnote{If $z_n \in V+W$ and $z_n \to z \in H$, write $z_n=v_n+w_n$,
$v_n \in V$ and $w_n \in W$. Using $V \perp W$ we get $\norm{z_n-z_m}^2=\norm{v_n-v_m}^2+\norm{w_n-w_m}^2$, which we use to
prove the claim. On the other hand, there are examples of closed subspaces $V,W$ of a Hilbert space that do not satisfy $V \perp W$ for
which $V+W$ is not itself closed.}   By induction, we obtain $V_1 + \cdots+V_n = \bigvee_{k=1}^n V_k$. 
If $V_k$, $k \geq 1$, are mutually orthogonal closed subspaces of $H$, we define
\[
\bigoplus_{k=1}^\infty V_k= \bigvee_{k=1}^\infty V_k,
\]
called the  {\em orthogonal direct sum} of the subspaces $V_k$. The orthogonal direct sum $\bigoplus_{k=1}^\infty V_k$ is equal
to the closure of the set of all  finite sums of the form $\sum  v_k$, where $v_k \in V_k$.\footnote{See
Paul
Halmos, {\em Introduction to Hilbert Space and the Theory of Spectral Multiplicity}, p.~26, \S 13, Theorem 2.}

If $V$ is a closed subspace of $H$, then \eqref{Hdecomp} is
\[
H=V \oplus V^\perp.
\]

The subobjects of a Hilbert space that we care about are mostly closed subspaces, because they are themselves Hilbert spaces (a closed subset
of a complete metric space is itself a complete metric space). Generally, when talking about one type of object, we care mostly about those
subsets of it that are the same type of object, which is why, for example, that we are so glad to know when closed subspaces are mutually orthogonal, because
then  their sum is itself a closed subspace.\footnote{Paul
Halmos in his {\em Introduction to Hilbert Space and the Theory of Spectral Multiplicity} uses the term linear manifold to
refer to what I call a  subspace and subspace to refer to what I call a closed subspace. I think even better terms
would be {\em linear subspace} and {\em Hilbert subspace}, respectively; this would emphasize the category of objects with  which one is working.}

\section{Projections}
Let $V$ be a closed subspace of $H$.
The {\em  projection} onto $V$ is the map $P_V:H \to H$ defined in the following way: if $f \in H=V \oplus V^\perp$, let $f=g+h, g \in V, h \in V^\perp$, and set $P_V(f)=g$. The image of $P_V$ is $V$:
if $f \in \im P_V$ then there is some $f_0 \in H$ such that $P_V f_0 =f$ and $f_0=g+h$, $g \in V, h \in V^\perp$, so by
the definition of $P_V$ we get $P_V f_0=g$, giving $f=g \in V$; on the other hand, if $f \in V$ then $P_V f = f$, so $f \in \im P$.

The image of a projection is closed.\footnote{We call $P:H \to H$ a projection if there is some closed subspace $V$ of $H$ such that $P$ is the projection onto $V$, and this $V$ will be the image of $P$. We often talk about a projection in $H$ without specifying what it is a projection onto.}
Hence if $P$ is a projection, we have
\[
H=\im P_V \oplus (\im P_V)^\perp = \im P_V \oplus \ker P_V.
\]


If $P_V$ is a projection onto a closed subspace $V$ of $H$,
we check that
$P_V \in B(H)$. If $g \in V, h \in V^\perp$, then, because $\inner{g}{h}=0$,
\[
\norm{P_V(g+h)}^2=\norm{g}^2 \leq \norm{g}^2+\norm{h}^2 =\inner{g}{g}+\inner{h}{h}=\inner{g+h}{g+h}=\norm{g+h}^2,
\] 
so $\norm{P_V} \leq 1$. If $V \neq \{0\}$, let $g \in V$, $g \neq 0$. Then
\[
\norm{\frac{g}{\norm{g}}}=1, \qquad \norm{P_V \left( \frac{g}{\norm{g}} \right)}=\norm{\frac{g}{\norm{g}}}=1.
\]
Hence if $P_V$ is a projection and $P_V \neq 0$, then $\norm{P_V}=1$. Of course, if $P_V=0$ then $\norm{P_V}=0$. 
Using $H=V \oplus V^\perp$, we check that
if  $f_1,f_2 \in H$ then
\[
\inner{P_V f_1}{f_2}=\inner{f_1}{P_V f_2},
\]
hence  $P_V$ is self-adjoint.


We call $P \in B(H)$  {\em idempotent} if $P^2=P$.\footnote{Often the term projection is used to refer to a thing we would call an idempotent, and the term orthogonal projection is used to refer to a thing   we would call a projection.} The following conditions are each equivalent to a nonzero idempotent $P$ being a  projection:
\begin{itemize}
\item $P$ is positive
\item $P$ is self-adjoint
\item $P$ is normal
\item $\ker P=(\im P)^\perp$
\item $\norm{P}=1$
\end{itemize}





\section{The lattice of projections}
Two important projections: $\id_H$ is the projection onto $H$, and $0$ is the projection onto $\{0\}$. 


If $T \in B(H)$ is self-adjoint, we say that $T$ is  {\em positive}  if, for all $v \in H$,
\[
\inner{Tv}{v} \geq 0.
\]
If $S,T \in B(H)$ are self-adjoint, we say that $S \geq T$ if, for all $v \in H$,
\[
\inner{Sv}{v} \geq \inner{Tv}{v}.
\]
This is equivalent to $S-T$ being a positive operator. Thus, $T \in B(H)$ is a positive operator if and only if $T \geq 0$. 
One checks that $\leq$ is a partial order on the set of bounded self-adjoint operators on $H$. 


Let $P \in B(H)$ be a projection. As $H = \im P \oplus (\im P)^\perp=\im P \oplus \ker P$, if $v \in H$ then $v=v_1+v_2$, $v_1 \in \im P, v_2 \in (\im P)^\perp$,
\[
\inner{Pv}{v}=\inner{v_1}{v_1+v_2}=\inner{v_1}{v_1}+\inner{v_1}{v_2}=\inner{v_1}{v_1} \geq 0.
\]
Hence $P \geq 0$.

Let $P,Q \in B(H)$ be projections. It is a fact that the following statements are equivalent:
\begin{itemize}
\item $P \leq Q$
\item $\im P \subseteq \im Q$
\item $QP=P$
\item $PQ=P$
\item $\norm{Pv} \leq \norm{Qv}$ for all $v \in H$
\end{itemize}

The set of projections is a complete lattice.\footnote{If $P,Q \in B(H)$ are projections, then the infimum of $P$ and $Q$ is the projection onto
$\im P \cap \im Q$, and the supremum of $P$ and $Q$ is the projection onto $\im P \vee \im Q$.
See Problem 96 of  Paul Halmos's {\em Hilbert Space Problem Book}.}



\section{Sums of projections}
\label{sums}
If $P,Q \in B(H)$ are projections, we write $P \perp Q$ if $\im P \perp \im Q$, which is equivalent to $PQ=0$ and $QP=0$; neither one of those by itself is sufficient. It is a
fact\footnote{See Steven Roman,
{\em Advanced Linear Algebra}, third ed., p.~78} that $P+Q$ is a projection if and only if
$P \perp Q$, in which case $P+Q$ is a projection onto $\im P \oplus \im Q$. The product $PQ$ is a projection if and only if
$PQ=QP$, in which case it is a projection onto $\im P \cap \im Q$. 


Paul Halmos shows the following in Question 94 of his {\em Hilbert Space Problem Book}: If $T_n \in B(H)$ are self-adjoint such that
$T_n \leq T_{n+1}$ for all $n$ and if there is some self-adjoint
$T' \in B(H)$ such that
$T_n \leq T'$ for all $n$, then 
there is some self-adjoint $T \in B(H)$ such that $T_n \to T$ in the {\em strong operator topology}.\footnote{Recall that if $T_n \in B(H)$ and $T \in B(H)$, we say that $T_n \to T$ in the strong
operator topology if for all $v \in H$ we have $T_n v \to Tv$. (If $T_n \to T$ in $B(H)$ then $T_n \to T$ in the strong operator topology.)} This is analogous to how a nondecreasing
sequence of real numbers that has an upper bound converges to a real number.

Let $M_n$, $n \geq 1$, be closed subspaces of $H$ such that $M_n \subseteq M_{n+1}$ and let $P_n$ be the projection onto $M_n$. As $M_n \subseteq M_{n+1}$ we have $P_n \leq P_{n+1}$ for all $n$. Also,
$P_n \leq \id_H$ for all $n$, as $\id_H$ is the projection onto $H$ and $M_n \subseteq H$. Hence by the result from Halmos stated above, there is some self-adjoint $P \in B(H)$ such that
$P_n \to P$ in the strong operator topology. Let
\[
M=\bigvee_{n=1}^\infty M_n.
\]
I claim that $P$ is the projection onto $M$.\footnote{cf. Paul Halmos, {\em Introduction to Hilbert Space and the Theory of Spectral Multiplicity},
p.~46, \S 28, Theorem 1.}

If $P_n \in B(H)$, $n \geq 1$, are projections and $P_i \perp P_j$ for $i \neq j$, 
let 
\[
M_n = \bigoplus_{k=1}^n \im P_k.
\]
Define $T_n v=\sum_{k=1}^n P_k v$. As $M_n$ is a closed subspace of $H$, we have
\[
H=M_n \oplus M_n^\perp=\im P_1 \oplus \cdots P_n \oplus M_n^\perp.
\]
If $v =v_1+\cdots+v_n+v'$, $v_1 \in \im P_1,\ldots, v_n \in \im P_n, v' \in M_n^\perp$, then
\[
T_n v=\sum_{k=1}^n  \left(P_k (v_1+\cdots+v_n+v')\right)=\sum_{k=1}^n v_k.
\]
Thus $T_n$ is the projection onto $M_n$. Therefore, there is some self-adjoint $P \in B(H)$ such that
$T_n \to P$ in the strong operator topology, and with
\[
M=\bigvee_{n=1}^\infty \bigvee_{k=1}^n \im P_k = \bigvee_{n=1}^\infty \im P_k = \bigoplus_{n=1}^\infty \im P_n,
\]
$P$ is the projection onto $M$.
 $\sum_{k=1}^n P_k \to P$ in the strong operator topology, and we denote $P$
by $\sum_{k=1}^\infty P_k$.


\section{Definition  of projection-valued measures}
Let $\mathscr{P}(H)$ be the set of projections in $B(H)$. Let $\mathscr{B}(\mathbb{C})$ be the Borel $\sigma$-algebra of $\mathbb{C}$. A {\em projection-valued measure} on
$\mathbb{C}$ is a map $E:\mathscr{B}(\mathbb{C}) \to \mathscr{P}(H)$ 
such that
\begin{itemize}
\item $E(\emptyset)=0$ and $E(\mathbb{C})=\id_H$
\item If $B_n \in \mathscr{B}(\mathbb{C})$, $n \geq 1$, are pairwise disjoint, then
\[
E\Big(\bigcup_{n \geq 1} B_n\Big)=\sum_{n \geq 1} E(B_n),
\]
where $\sum_{n \geq 1} E(B_n)$ is the limit of $\sum_{1 \leq n \leq N} E(B_n)$ in the strong operator topology.
\end{itemize}

\section{Finite additivity}
\label{properties}
If $E:\mathscr{B}(\mathbb{C}) \to \mathscr{P}(H)$ is any function that satisfies
$E(B_1 \cup B_2)=E(B_1)+E(B_2)$ for disjoint $B_1, B_2 \in \mathscr{B}(\mathbb{C})$, then it satisfies the following
four properties. In particular, if $E$ is a projection-valued measure it satisfies them.

\begin{enumerate}
\item $E(\emptyset)=E(\emptyset \cup \emptyset)=E(\emptyset)+E(\emptyset)$, so $E(\emptyset)=0$.

\item  If $B_1,B_2 \in \mathscr{B}(\mathbb{C})$ and $B_1 \subseteq B_2$, then
\[
E(B_2)=E(B_2 \setminus B_1 \cup B_1)=E(B_2 \setminus B_1)+E(B_1) \geq E(B_1),
\]
since $E(B_2 \setminus B_1)$ is a projection and hence is a positive operator. Therefore, if $B_1 \subseteq B_2$ then
$E(B_1) \leq E(B_2)$. 

\item Let $B_1, B_2 \in \mathscr{B}(\mathbb{C})$. We have
\[
E(B_1)=E(B_1 \cap B_2)+E(B_1 \setminus B_2),
\]
and
\[
E(B_2)=E(B_1 \cap B_2)+E(B_2 \setminus B_1),
\]
and
\[
E(B_1 \cup B_2)=E(B_1 \cap B_2) + E(B_1 \setminus B_2) + E(B_2 \setminus B_1), 
\]
and combining these gives, for any $B_1, B_2 \in \mathscr{B}(\mathbb{C})$,
\begin{eqnarray*}
E(B_1)+E(B_2)&=&E(B_1 \cap B_2)+E(B_1 \cap B_2)+E(B_1 \setminus B_2)+E(B_2 \setminus B_1)\\
&=&E(B_1 \cap B_2)+E(B_1 \cup B_2).
\end{eqnarray*}

\item Let $B_1,B_2 \in \mathscr{B}(\mathbb{C})$. Multiplying both sides of the above equation on the left by $E(B_2)$ gives
\[
E(B_1)E(B_2)+E(B_2)E(B_2)=E(B_1 \cap B_2)E(B_2)+E(B_1 \cup B_2) E(B_2).
\]
As $E(B_1 \cup B_2)$ and $E(B_2)$ are projections and $E(B_1 \cup B_2) \geq E(B_2)$, we have
\[
E(B_1 \cup B_2) E(B_2)=E(B_2),
\]
which with $E(B_2)E(B_2)=E(B_2)$ gives
\[
E(B_1)E(B_2)+E(B_2)=E(B_1 \cap B_2)+E(B_2).
\]
Hence, for any $B_1, B_2 \in \mathscr{B}(\mathbb{C})$,
\[
E(B_1)E(B_2)=E(B_1 \cap B_2).
\]
\end{enumerate}


\section{Complex measures}
Suppose that $E:\mathscr{B}(\mathbb{C}) \to \mathscr{P}(H)$ is a function such that
$E(\mathbb{C})=\id_H$, and that for all $v, w\in H$, the function
$E_{v,w}:\mathscr{B}(\mathbb{C}) \to \mathbb{C}$ defined by
\[
E_{v,w}(B)=\inner{E(B)v}{w}
\]
is a complex measure. I will show  that $E$ is a projection-valued measure. Let $B_n \in \mathscr{B}(\mathbb{C})$, $n \geq 1$, be pairwise disjoint.\footnote{cf. 
Paul Halmos, {\em Introduction to Hilbert Space and the Theory of Spectral Multiplicity}, p.~59, \S 36, Theorem 3.}
If $B_1,B_2 \in \mathscr{B}(\mathbb{C})$ are disjoint, then for all $v,w \in H$,
\begin{eqnarray*}
\inner{E(B_1 \cup B_2)v}{w}&=&E_{v,w}(B_1 \cup B_2)\\
&=&E_{v,w}(B_1)+E_{v,w}(B_2)\\
&=&\inner{E(B_1)v}{w}+\inner{E(B_2)v}{w}\\
&=&\inner{(E(B_1)+E(B_2))v}{w},
\end{eqnarray*}
and since this holds for all $v, w \in H$, we obtain $E(B_1 \cup B_2)=E(B_1)+E(B_2)$. Therefore, from \S \ref{properties},
if $B_1,B_2 \in \mathscr{B}(\mathbb{C})$ are disjoint then
\[
E(B_1)E(B_2)=E(B_1 \cap B_2)=E(\emptyset)=0.
\]

If $v \in H$ and  $v_n=E(B_n)v$, then, for $m \neq n$,
\[
\inner{v_n}{v_m}=\inner{E(B_n)v}{E(B_m)v}=\inner{E(B_m)E(B_n)v}{v}=\inner{0v}{0}=0.
\]
For $a_n \in \mathbb{C}$, for the sequence $\sum_{n=1}^N a_n v_n$ to converge
in $H$ it is equivalent to $\sum_{n=1}^\infty \norm{a_n v_n}^2 < \infty$;\footnote{e.g. Walter Rudin's
{\em Functional Analysis}, p.~295, Theorem 12.6: if $x_n \in H$ are pairwise orthogonal, not necessarily of unit norm,
then for $\sum_{n=1}^\infty x_n$ to converge is equivalent to $\sum_{n=1}^\infty \norm{x_n}^2<\infty$.} but
$\norm{v_n} \leq \norm{v}$, so it suffices to show that $\sum_{n=1}^\infty |a_n|^2<\infty$.
Using that if $T$ is a projection then $T$ is self-adjoint and $T^2=T$, and that $E_{v,v}$ is a complex measure,
\begin{eqnarray*}
\sum_{n=1}^\infty \norm{E(B_n)v}^2&=&\sum_{n=1}^N \inner{E(B_n)v}{E(B_n)v}\\
&=&\sum_{n=1}^\infty \inner{E(B_n)E(B_n)v}{v}\\
&=&\sum_{n=1}^\infty \inner{E(B_n)v}{v}\\
&=&\sum_{n=1}^\infty E_{v,v}(B_n)\\
&=&E_{v,v}\left(\bigcup_{n=1}^\infty B_n \right)\\
&=&\inner{E\left(\bigcup_{n=1}^\infty B_n\right)v}{v}\\
&=&\norm{E\left(\bigcup_{n=1}^\infty B_n\right)v}^2\\
&\leq&\norm{v}.
\end{eqnarray*}
Therefore,  the sequence $\sum_{n=1}^N E(B_n)v$ converges in $H$; namely, $\sum_{n=1}^N E(B_n)$ converges
in the strong operator topology. Let $P$ be its limit. 
By \S \ref{sums}, $P$ is the projection onto $\bigoplus_{n=1}^\infty \im E(B_n)$.


For $v, w \in H$,
\begin{eqnarray*}
\inner{E\left(\bigcup_{n=1}^\infty B_n \right)v}{w}&=&E_{v,w}\left(\bigcup_{n=1}^\infty B_n \right)\\
&=&\sum_{n=1}^\infty E_{v,w}(B_n)\\
&=&\sum_{n=1}^\infty \inner{E(B_n)v}{w}\\
&=&\inner{Pv}{w},
\end{eqnarray*}
and since this is true for all $v,w \in H$, we obtain
\[
E\left(\bigcup_{n=1}^\infty B_n \right) = P,
\]
where $P$ is the limit of $\sum_{n=1}^N E(B_n)$ in the strong operator topology. This completes the proof that $E$ is a 
projection-valued measure.

On the other hand, if $E:\mathscr{B}(\mathbb{C}) \to \mathscr{P}(H)$ is a projection-valued measure, we can show that
for each $v,w \in H$ the function $E_{v,w}:\mathscr{B}(\mathbb{C}) \to \mathbb{C}$ defined by
$E_{v,w}(B)=\inner{E(B)v}{w}$ is a complex measure.







\section{Spectral integrals}
Let $\mathfrak{B}(\mathbb{C})$ be the set of bounded measurable functions $\mathbb{C} \to \mathbb{C}$.\footnote{This is not
$L^\infty(\mathbb{C})$, the set of equivalence classes of essentially bounded measurable functions, where two functions
are equivalent if they are equal almost everywhere. Moreover, it is not even the set $\mathscr{L}^\infty(\mathbb{C})$ of
essentially bounded measurable functions.

When does one speak about $\mathscr{L}^\infty(\mathbb{C})$? $\mathscr{L}^\infty(\mathbb{C})$ is a vector space; let $\mathscr{N}(\mathbb{C})$
be the set of those functions that are equal to $0$ almost everywhere in $\mathbb{C}$; $\mathscr{N}(\mathbb{C})$ is a vector space; then $L^\infty(\mathbb{C})$ is the vector space quotient
$\mathscr{L}^\infty(\mathbb{C})/\mathscr{N}(\mathbb{C})$.} It is a complex vector space, and we define the norm
$\norm{f}=\sup_{z \in \mathbb{C}} |f(z)|$; one checks that $\mathfrak{B}(\mathbb{C})$ is a Banach space.
Paul Halmos, {\em Introduction to Hilbert Space and the Theory of Spectral Multiplicity},
p.~60, \S 37, proves\footnote{The operator $A$ is the operator obtained from the following statement, which
  itself follows
from the Riesz representation theorem.
If $\phi:H \times H \to \mathbb{C}$ is sesquilinear (we take sesquilinear to mean linear in the first entry and conjugate
linear in the second entry) and 
\[
M=\sup_{\norm{v}=\norm{w}=1} |\phi(v,w)|<\infty,
\]
then there exists a unique $A \in B(H)$ such that
\[
\phi(v,w)=\inner{Av}{w}, \qquad v, w \in H,
\]
and $\norm{A}=M$.

One also has to prove that for each $f \in \mathfrak{B}$, $\phi(v,w)=E_{v,w}(f)$ is sesquilinear.} that if $E:\mathscr{B}(\mathbb{C}) \to \mathscr{P}(H)$ is a projection-valued measure
and $f \in \mathfrak{B}$, then there is a unique $A \in B(H)$ such that, for all $v, w \in H$,
\[
\inner{Av}{w}=E_{v,w}(f)=\int f dE_{v,w}=\int_\mathbb{C} f(\lambda) dE_{v,w}(\lambda),
\]
and $\norm{A} \leq 2\norm{f}$. 
We write
\[
A=E(f)=\int f dE=\int_\mathbb{C} f(\lambda) dE(\lambda).
\]

We can  check that $\mathfrak{B}(\mathbb{C})$ is a $C^*$-algebra, with $f^*$ defined by $f^*(z)= \overline{f(z)}$. It is a fact that $B(H)$ is a $C^*$-algebra. If $\alpha \in \mathbb{C}$ and $f,g \in 
\mathfrak{B}(\mathbb{C})$,  then\footnote{See Paul Halmos, {\em Introduction to Hilbert Space and the Theory of Spectral Multiplicity},
p.~60, \S 37, Theorem 2.}
\begin{equation}
E(\alpha f)=\alpha E(f), \qquad E(f+g)=E(f)+E(g); \qquad E(f^*)=(E(f))^*.
\label{Efunctions}
\end{equation}
The first two of these together with $\norm{E(f)} \leq 2\norm{f}$ show that $E:\mathfrak{B}(\mathbb{C}) \to B(H)$ is a bounded linear map.
If $f, g \in \mathfrak{B}(\mathbb{C})$, then\footnote{From Paul Halmos, {\em Introduction to Hilbert Space and the Theory of Spectral Multiplicity},
p.~61, \S 37, Theorem 3. To understand the proof by Halmos (which is symbolically convincing because of our familiarity with the permissible  moves
one can make when
integrating functions using complex measures), keep in mind that if $B_1,B_2 \in \mathscr{B}(\mathbb{C})$ then
\[
E_{E(B_1)v,w}(B_2)=\inner{E(B_2)E(B_1)v}{w}=\inner{E(B_1 \cap B_2)v}{w}=E_{v,w}(B_1 \cap B_2).
\]
}
\[
E(f)E(g)=E(fg).
\]
This and the third statement in \eqref{Efunctions}
show that $E:\mathfrak{B}(\mathbb{C}) \to B(H)$ is a homomorphism of $C^*$-algebras:

From the fact that $E:\mathfrak{B}(\mathbb{C}) \to B(H)$ is a homomorphism of $C^*$-algebras, it follows in particular that 
$E(f)$ is a normal operator for each $f \in \mathfrak{B}(\mathbb{C})$. 

\section{The spectrum of a projection-valued measure}
If $E:\mathscr{B}(\mathbb{C}) \to \mathscr{P}(H)$ is a projection-valued measure,
let $U_\alpha, \alpha \in I$, be those open sets $U_\alpha \subseteq \mathbb{C}$ such that $E(U_\alpha)=0$. The
 {\em spectrum}
of $E$ is
\[
\sigma(E) = \mathbb{C} \setminus \bigcup_{\alpha \in I} U_\alpha;
\]
this may also be called the support of $E$, and is analogous to the support of a nonnegative measure. 
Since $\sigma(E)$ is the complement of a union of open sets, it is closed. 

For each $n \geq 1$, let $D_n=\{|z| \leq n\}$. $D_n$ is compact, so there are $\alpha_{n,1},\ldots,\alpha_{n,N}$ such that 
\[
D_n \subseteq \bigcup_{k=1}^N U_{\alpha_{n,k}}.
\]
But, using \S \ref{properties} and the fact that $E(U_{\alpha_{n,k}})=0$ for $1 \leq k \leq N$,
\begin{eqnarray*}
E\left( \bigcup_{k=1}^N U_{\alpha_{n,k}} \right) &=& E(U_{\alpha_{n,1}})+E\left(\bigcup_{k=2}^N U_{\alpha_{n,k}}\right)
-E\left(U_{\alpha_{n,1}} \cap \bigcup_{k=2}^N U_{\alpha_{n,k}} \right)\\
&=&E\left(\bigcup_{k=2}^N U_{\alpha_{n,k}}\right)-E(U_{\alpha_{n,1}} ) E\left(\bigcup_{k=2}^N U_{\alpha_{n,k}} \right)\\
&=&E\left(\bigcup_{k=2}^N U_{\alpha_{n,k}}\right)\\
&=&\cdots\\
&=&0,
\end{eqnarray*}
therefore $E(D_n)=0$ (it will be $\leq 0$, and as a projection is a positive operator it must equal $0$). Let $B_n=D_{n+1} \setminus D_n$;
as $B_n$ is a subset of $D_{n+1}$ we get $E(B_n) = 0$. We have $\bigcup_{n=1}^\infty B_n = \mathbb{C}$, and as the $B_n$ are pairwise
disjoint we get
\[
E(\mathbb{C})=E\left(\bigcup_{n=1}^\infty B_n\right)=\sum_{n=1}^\infty E(B_n)=0,
\]
contradicting that $E(\mathbb{C})=\id_H$. Therefore, if $E$ is a projection-valued measure then its spectrum is not empty.

Here are some facts about projection-valued measures.
$E(\mathbb{C} \setminus \sigma(E))=0$.\footnote{Paul Halmos,
{\em Introduction to Hilbert Space and the Theory of Spectral Multiplicity},
p.~62, \S 38, Theorem 1. His statement is about regular measures, but  a projection-valued measure on the Borel
$\sigma$-algebra of $\mathbb{C}$ is regular, as he shows on the page after that.} 

Let $\mathfrak{B}_E(\mathbb{C})$ be the set of
bounded measurable functions $\sigma(E) \to \mathbb{C}$, and let $\norm{f}_E = \sup_{z \in \sigma(E)} |f(z)|$. 
If $E$ is a projection-valued measure with compact spectrum and 
$f:\mathbb{C} \to \mathbb{C}$ is continuous, then\footnote{Paul Halmos,
{\em Introduction to Hilbert Space and the Theory of Spectral Multiplicity}, p.~62, \S 38, Theorem 2.}
\[
\norm{E(\chi_{\sigma(E)} f)}=\norm{f}_E.
\]
(We  often talk about projection-valued measures whose spectrum is compact;
since their spectrum is closed, to demand that the spectrum of a projection-valued measure is compact is to demand that it is bounded.)

If $E$ is a projection-valued measure with compact spectrum and if $A=E(\chi_{\sigma(E)} \lambda)$, then\footnote{Paul Halmos,
{\em Introduction to Hilbert Space and the Theory of Spectral Multiplicity}, p.~64, \S 39, Theorem 2. The proof uses the fact that $T \in B(H)$ is invertible if and only if
there is some $\alpha>0$ such that $\norm{Tv} \geq \alpha \norm{v}$ for all $v \in H$.}
\[
\sigma(A)=\sigma(E).
\]

\section{Statement of the spectral theorem}
The spectral theorem, proved by in Paul Halmos, {\em Introduction to Hilbert Space and the Theory of Spectral Multiplicity}, p.~69, \S 43, Theorem 1,
states the following:

If $A \in B(H)$ is self-adjoint, then there exists a unique projection-valued measure $E:\mathscr{B}(\mathbb{C}) \to \mathscr{P}(H)$, with $\sigma(E)$ compact and $\sigma(E) \subset \mathbb{R}$, such that
\[
A=E(\chi_{\sigma(E)} \lambda)=\int_{\sigma(E)} \lambda dE(\lambda).
\]
Since $\sigma(E) \subset \mathbb{R}$, we can write $E:\mathscr{B}(\mathbb{R}) \to \mathscr{P}(H)$. 

\end{document}
