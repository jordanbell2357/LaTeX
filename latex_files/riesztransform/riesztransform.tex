\documentclass{article}
\usepackage{amsmath,amssymb,mathrsfs,amsthm,mathabx}
%\usepackage{tikz-cd}
\usepackage{hyperref}
\newcommand{\inner}[2]{\left\langle #1, #2 \right\rangle}
\newcommand{\tr}{\ensuremath\mathrm{tr}\,} 
\newcommand{\Span}{\ensuremath\mathrm{span}} 
\def\Re{\ensuremath{\mathrm{Re}}\,}
\def\Im{\ensuremath{\mathrm{Im}}\,}
\newcommand{\id}{\ensuremath\mathrm{id}} 
\newcommand{\var}{\ensuremath\mathrm{var}} 
\newcommand{\arcsinh}{\ensuremath\mathrm{arcsinh}} 
\newcommand{\Lip}{\ensuremath\mathrm{Lip}} 
\newcommand{\GL}{\ensuremath\mathrm{GL}} 
\newcommand{\diam}{\ensuremath\mathrm{diam}} 
\newcommand{\sgn}{\ensuremath\mathrm{sgn}\,} 
\newcommand{\lcm}{\ensuremath\mathrm{lcm}} 
\newcommand{\supp}{\ensuremath\mathrm{supp}\,}
\newcommand{\dom}{\ensuremath\mathrm{dom}\,}
\newcommand{\upto}{\nearrow}
\newcommand{\downto}{\searrow}
\newcommand{\norm}[1]{\left\Vert #1 \right\Vert}
\newcommand{\HS}[1]{\left\Vert #1 \right\Vert_{\mathrm{HS}}}
\newtheorem{theorem}{Theorem}
\newtheorem{lemma}[theorem]{Lemma}
\newtheorem{proposition}[theorem]{Proposition}
\newtheorem{corollary}[theorem]{Corollary}
\theoremstyle{definition}
\newtheorem{definition}[theorem]{Definition}
\newtheorem{example}[theorem]{Example}
\begin{document}
\title{Singular integral operators and the Riesz transform}
\author{Jordan Bell\\ \texttt{jordan.bell@gmail.com}\\Department of Mathematics, University of Toronto}
\date{\today}

\maketitle

\section{Calder\'on-Zygmund kernels}
Let $\omega_{n-1}$ be the measure of $S^{n-1}$. It is 
\[
\omega_{n-1} = \frac{2\pi^{n/2}}{\Gamma(n/2)}.
\]
Let $v_n$ be the measure
of the unit ball in $\mathbb{R}^n$. It is
\[
v_n = \frac{\omega_{n-1}}{n} = \frac{2\pi^{n/2}}{n\Gamma(n/2)}.
\]

For $k,N \geq 0$ and $\phi \in C^\infty(\mathbb{R}^n)$ let
\[
p_{k,N}(\phi) = \max_{|\alpha| \leq k} \sup_{x \in \mathbb{R}^n} 
(1+|x|)^{N} |(\partial^\alpha \phi)(x)|.
\]
 

A Borel measurable function $K:\mathbb{R}^n \setminus \{0\} \to \mathbb{C}$ is called a 
\textbf{Calder\'on-Zygmund kernel} if there is some $B$ such that
\begin{enumerate}
\item $|K(x)| \leq B|x|^{-n}$, $x \neq 0$ \label{cond1}
\item $\int_{|x| \geq 2|y|} |K(x-y)-K(x)| dx \leq B$, $y \neq 0$ \label{cond2}
\item $\int_{R_1<|x|<R_2} K(x) dx = 0$, $0<R_1<R_2<\infty$. \label{cond3}
\end{enumerate}

The following lemma gives a tractable condition under which 
Condition \ref{cond2} is satisfied.\footnote{Camil Muscalu and Wilhelm Schlag, {\em Classical
and Multilinear Harmonic Analysis}, volume I, p.~167, Lemma 7.2.}

\begin{lemma}
If $|(\nabla K)(x)| \leq C |x|^{-n-1}$ for all $x \neq 0$ then for $y \neq 0$,
\[
\int_{|x| \geq 2|y|} |K(x-y)-K(x)| dx \leq v_n 2^n C.
\]
\end{lemma}
\begin{proof}
For $|x|>2|y|>0$, if $0 \leq t \leq 1$ then
\[
|x-ty| \geq |x| - t|y|  \geq |x| - |y| > |x|-\frac{|x|}{2} = \frac{|x|}{2}.
\]
Write $f(t) = K(x-ty)$, for which $f'(t) = -(\nabla K)(x-ty) \cdot y$. By the fundamental theorem of calculus,
\[
K(x-y)-K(x) = f(1)-f(0) = \int_0^1 f'(t) dt = -\int_0^1 (\nabla K)(x-ty) \cdot y dt,
\]
thus
\[
|K(x-y)-K(x)| \leq \int_0^1 |(\nabla K)(x-ty)| |y| dt
\leq C |y| \int_0^1 |x-ty|^{-n-1} dt.
\]
Then using $|x-ty| > \frac{|x|}{2}$,
\[
|K(x-y)-K(x)| \leq C|y| \left( \frac{|x|}{2} \right)^{-n-1} 
=2^{n+1} C |y| |x|^{-n-1}.
\]
For $|y|>0$, using spherical coordinates,\footnote{See \url{http://individual.utoronto.ca/jordanbell/notes/sphericalmeasure.pdf}}
\begin{align*}
\int_{|x| \geq 2|y|} |K(x-y)-K(x)| dx&\leq \int_{|x| \geq 2|y|} 2^{n+1} C |y| |x|^{-n-1} dx\\
&=2^{n+1} C |y| \int_{2|y|}^\infty \left( \int_{S^{n-1}} |r\gamma|^{-n-1}  d\sigma(\gamma) \right) r^{n-1} dr\\
&=v_n 2^{n+1} C|y| \int_{2|y|}^\infty r^{-2} dr\\
&=v_n 2^{n+1} C |y| \cdot \frac{1}{2|y|}\\
&=v_n 2^n C.
\end{align*}
\end{proof}


For a Calder\'on-Zygmund kernel $K$, for $f \in \mathscr{S}(\mathbb{R}^n)$,  for $x \in \mathbb{R}^n$, and for $\epsilon>0$,
using Condition \ref{cond3} with $R_1=\epsilon$ and $R_2=1$,\footnote{\url{https://math.aalto.fi/~parissi1/notes/harmonic.pdf},
p.~115, Lemma 6.15.}
\[
\begin{split}
&\int_{|x-y| \geq \epsilon} K(x-y) f(y) dy\\
=&\int_{\epsilon \leq |x-y| \leq 1} K(x-y) (f(y)-f(x)) dy + \int_{|x-y| \geq 1} K(x-y) f(y) dy.
\end{split}
\]
By Condition \ref{cond1} there is some $B$ such that 
$|K(x)| \leq B|x|^{-n}$, and combining this with $|f(y)-f(x)| \leq \norm{\nabla f}_{\infty} |y-x|$,
\[
|K(x-y)(f(y)-f(x))| \leq B\norm{\nabla f}_\infty  |y-x|^{-n+1},
\]
which is integrable on $\{|x-y| \leq 1\}$. Then by the dominated convergence theorem,
\[
\lim_{\epsilon \to 0} \int_{\epsilon \leq |x-y| \leq 1} K(x-y) (f(y)-f(x)) dy 
=\int_{|x-y| \leq 1} K(x-y) (f(y)-f(x)) dy.
\]

\begin{lemma}
For a Calder\'on-Zygmund kernel $K$, for $f \in \mathscr{S}(\mathbb{R}^n)$,  and for $x \in \mathbb{R}^n$, 
the limit
\[
\lim_{\epsilon \to 0} \int_{|x-y| \geq \epsilon} K(x-y) f(y) dy
\]
exists.
\label{CZK}
\end{lemma}



\section{Singular integral operators}
For a Calder\'on-Zygmund kernel $K$ on $\mathbb{R}^n$, for $f \in \mathscr{S}(\mathbb{R}^n)$, and for $x \in \mathbb{R}^n$, let
\[
(Tf)(x) = \lim_{\epsilon \to 0} \int_{|x-y| \geq \epsilon} K(x-y) f(y) dy.
\]
We call $T$ a \textbf{singular integral operator}. By Lemma \ref{CZK} this makes sense.

We prove that singular integral operators are $L^2 \to L^2$ bounded.\footnote{Camil Muscalu and Wilhelm Schlag, {\em Classical
and Multilinear Harmonic Analysis}, volume I, p.~168, Proposition 7.3;
Elias M. Stein, {\em Singular Integrals and Differentiability Properties of Functions},
p.~35, \S 3.2, Theorem 2; \url{http://math.uchicago.edu/~may/REU2013/REUPapers/Talbut.pdf}}

\begin{theorem}
There is some $C_n$ such that for any Calder\'on-Zygmund kernel $K$
and any $f \in \mathscr{S}(\mathbb{R}^n)$,
\[
\norm{Tf}_2 \leq C_n B\norm{f}_2.
\]
\end{theorem}
\begin{proof}
For $0<r<s<\infty$  and
for $\xi \in \mathbb{R}^n$ define
\begin{align*}
m_{r,s}(\xi) &= \int_{\mathbb{R}^n} e^{-2\pi ix\cdot \xi}  1_{r<|x|<s}(x)  K(x) dx.
\end{align*}
Take $r<|\xi|^{-1}<s$, for which
\[
m_{r,s}(\xi) = \int_{r<|x|<|\xi|^{-1}} e^{-2\pi ix\cdot \xi} K(x) dx 
+ \int_{|\xi|^{-1}<|x|<s} e^{-2\pi ix\cdot \xi} K(x) dx.
\]
For the first integral, using Condition \ref{cond3} with $R_1=r$ and $R_2=|\xi|^{-1}$ and then using Condition \ref{cond1},
\begin{align*}
\left|\int_{r<|x|<|\xi|^{-1}} e^{-2\pi ix\cdot \xi} K(x) dx\right| &
= \left|\int_{r<|x|<|\xi|^{-1}} (e^{-2\pi ix\cdot \xi}-1) K(x) dx\right|\\
&\leq \int_{|x|<|\xi|^{-1}} |e^{-2\pi ix\cdot \xi}-1| |K(x)| dx\\
&\leq  \int_{|x|<|\xi|^{-1}} 2\pi |x| |\xi| |K(x)| dx\\
&\leq 2\pi |\xi| \int_{|x|<|\xi|^{-1}} B |x|^{-n+1} dx\\
&=2\pi |\xi| \cdot v_n |\xi|^{-1}.
\end{align*}
For the second integral, let $z=\frac{\xi}{2|\xi|^2}$, and 
\begin{align*}
\int_{|\xi|^{-1} < |x|<s} e^{-2\pi ix\cdot \xi} K(x) dx&= -\int_{|\xi|^{-1}<|x|<s}e^{-2\pi i(x+z) \cdot \xi} K(x) dx\\
&=-\int_{|\xi|^{-1}<|x-z|<s} e^{-2\pi ix\cdot \xi} K(x-z) dx.
\end{align*}
Let
\begin{align*}
R &= \int_{|\xi|^{-1}<|x|<s} e^{-2\pi ix\cdot \xi}K(x-z) dx - \int_{|\xi|^{-1}<|x-z|<s} e^{-2\pi ix\cdot \xi} K(x-z) dx\\
&=-\int_{|\xi|^{-1}<|x+z|<s} e^{-2\pi ix\cdot \xi} K(x) dx + \int_{|\xi|^{-1}<|x|<s} e^{-2\pi ix\cdot \xi} K(x) dx,
\end{align*}
with which
\[
\int_{|\xi|^{-1} < |x|<s} e^{-2\pi ix\cdot \xi} K(x) dx = \frac{1}{2} \int_{|\xi|^{-1}<|x|<s} e^{-2\pi ix\cdot \xi} (K(x)-K(x-z)) dx
+\frac{R}{2}.
\]
On the one hand, applying Condition \ref{cond2}, as $|z| = \frac{1}{2|\xi|}$,
\begin{align*}
\left| \int_{|\xi|^{-1}<|x|<s} e^{-2\pi ix\cdot \xi} (K(x)-K(x-z)) dx \right|&\leq \int_{|x|>|\xi|^{-1}} |K(x)-K(x-z)| dx\\
&=\int_{|x|>2|z|} |K(x)-K(x-z)| dx\\
&\leq B.
\end{align*}
On the other hand, let
\[
D =D_1 \triangle D_2=  \{x:|\xi|^{-1}<|x+z|<s\} \triangle \{x:|\xi|^{-1}<|x|<s\}.
\]
For $x \in D_1$ we have
\[
|x| \geq |x+z| - |z| > \frac{1}{|\xi|} - \frac{1}{2|\xi|}= \frac{1}{2|\xi|},
\]
and for $x \in D_2$ we have
$|x| > \frac{1}{|\xi|}$, so for $x \in D$,
\[
|x| > \frac{1}{2|\xi|}.
\]
Applying Condition \ref{cond1},
\[
|K(x)| \leq B|x|^{-n} < 2^nB|\xi|^n. 
\]
Furthermore, for $x \in D_1 \setminus D_2$ we have $|x| \leq |\xi|^{-1}$, and for $x \in D_2 \setminus D_1$ we have
\[
|x| \leq |x+z|+|z| = |x+z| + \frac{1}{2|\xi|} \leq \frac{1}{|\xi|}+\frac{1}{2|\xi|}
=\frac{3}{2|\xi|}.
\]
Hence
\[
D \subset \left\{x: \frac{1}{2|\xi|} < |x| \leq \frac{3}{2|\xi|}\right\},
\]
so
\[
\lambda(D) \leq \left(\frac{3}{2|\xi|}\right)^n v_n =  \left(\frac{3}{2}\right)^n |\xi|^{-n} v_n.
\]
Therefore
\[
|R| \leq 2^nB|\xi|^n \cdot  \left(\frac{3}{2}\right)^n |\xi|^{-n} v_n
=3^n v_n B
\]
and then
\[
\left| \int_{|\xi|^{-1} < |x|<s} e^{-2\pi ix\cdot \xi} K(x) dx  \right| \leq \frac{1}{2}B + \frac{1}{2} \cdot 3^n v_n B,
\]
and finally\footnote{The way I organize the argument, I want to use $\norm{m_{r,s}}_\infty \leq C_nB$, while
we have only obtained this bound for $r<|\xi|^{-1}<s$. To make the argument correct I may need to 
do things in a different order, e.g. apply Fatou's lemma and then use an inequality instead of using an inequality
and then apply Fatou's lemma.}
\[
|m_{r,s}(\xi)| \leq 2\pi  v_n+\frac{1}{2}B + \frac{1}{2} \cdot 3^n v_n B=C_nB.
\]

Define
\[
(T_{r,s}f)(x) = \int_{\mathbb{R}^n} 1_{r<|y|<s}(y)  K(y) f(x-y) dy,\qquad x \in \mathbb{R}^n.
\]
Then
\begin{align*}
\widehat{T_{r,s}f}(\xi)&=\int_{\mathbb{R}^n} e^{-2\pi ix\cdot \xi} \left( \int_{\mathbb{R}^n} 1_{r<|y|<s}(y)  K(y) f(x-y) dy\right) dx\\
&=\int_{\mathbb{R}^n} 1_{r<|y|<s}(y)  K(y) \left( \int_{\mathbb{R}^n} e^{-2\pi ix\cdot \xi} f(x-y) dx \right) dy\\
&=\int_{\mathbb{R}^n} 1_{r<|y|<s}(y)  K(y) e^{-2\pi iy\cdot \xi} \widehat{f}(\xi) dy\\
&=m_{r,s}(\xi) \widehat{f}(\xi),
\end{align*}
and so
\[
\norm{\widehat{T_{r,s}f}}_2^2=\int_{\mathbb{R}^n} |\widehat{T_{r,s}f}(\xi)|^2 d\xi
=\int_{\mathbb{R}^n} |m_{r,s}(\xi) \widehat{f}(\xi)|^2 d\xi
\leq \norm{m_{r,s}}_\infty^2 \norm{\widehat{f}}_2^2,
\]
by Plancherel's theorem and the inequality we got for $|m_{r,s}(\xi)|$,
\[
\norm{T_{r,s}f}_2^2 \leq \norm{m_{r,s}}_\infty^2 \norm{f}_2^2 \leq (C_nB)^2 \norm{f}_2^2.
\]
For each $x \in \mathbb{R}^n$, 
$(T_{r,s}f)(x) \to (Tf)(x)$ as $r \to 0$ and $s \to \infty$, and thus using Fatou's lemma,
\[
\int_{\mathbb{R}^n} |(Tf)(x)|^2 dx \leq
\liminf_{r \to 0, s \to \infty} \int_{\mathbb{R}^n} |(T_{r,s}f)(x)|^2 dx
=(C_nB)^2 \norm{f}_2^2.
\]
That is,
\[
\norm{Tf}_2 \leq C_nB \norm{f}_2.
\]
\end{proof}





\section{The Riesz transform}
Let
\[
c_n = \frac{1}{\pi v_{n-1}} = \frac{\Gamma\left(\frac{n+1}{2}\right)}{\pi^{(n+1)/2}}.
\]
For $1 \leq j \leq n$, let
\[
K_j(x) = c_n \frac{x_j}{|x|^{n+1}}.
\]
This is a Calder\'on-Zygmund kernel. 
For $\phi \in \mathscr{S}(\mathbb{R}^n)$ define
\[
(R_j \phi)(x) = \lim_{\epsilon \to 0} \int_{|y-x| \geq \epsilon} K_j(x-y) \phi(y) dy
=\lim_{\epsilon \to 0} \int_{|y| \geq \epsilon} K_j(y) \phi(x-y) dy.
\]
We call each $R_j$, $1 \leq j \leq n$, a \textbf{Riesz transform}. 





For $1 \leq j \leq n$ define $W_j:\mathscr{S}(\mathbb{R}^n) \to \mathbb{C}$ by
\begin{equation}
\inner{W_j}{\phi} = \frac{\Gamma\left(\frac{n+1}{2}\right)}{\pi^{\frac{n+1}{2}}} \lim_{\epsilon \to 0} \int_{|y| \geq \epsilon} K_j(y) \phi(y) dy.
\label{Wj}
\end{equation}
For $\epsilon>0$, 
\begin{align*}
\left| \int_{\epsilon \leq |y| \leq 1} K_j(y) \phi(y) dy \right|&=\left| \int_{\epsilon \leq |y| \leq 1} K_j(y)(\phi(y)-\phi(0)) dy \right|\\
&\leq \int_{\epsilon \leq |y| \leq 1} c_n |y|^{-n} \cdot \norm{\nabla \phi}_\infty |y| dy\\
&= c_n \norm{\nabla \phi}_\infty \omega_{n-1} \int_\epsilon^1 r^{-n+1} \cdot r^{n-1} dr\\
&=c_n \norm{\nabla \phi}_\infty \omega_{n-1} (1-\epsilon).
\end{align*}
For $|y| \geq 1$,
\begin{align*}
\int_{|y| \geq 1} |K_j(y) \phi(y)| dy &\leq c_n \int_{|y| \geq 1} |y|^{-n} (1+|y|^2)^{-1/2} p_{0,1}(\phi) dy\\
&=c_n \omega_n \int_1^\infty r^{-n} (1+r)^{-1} \cdot r^{n-1} dr\\
&=c_n \omega_n \log 2.
\end{align*}
It then follows from the dominated convergence theorem that the limit
\[
\lim_{\epsilon \to 0} \int_{|y| \geq \epsilon} K_j(y) \phi(y) dy
\]
exists, which shows that the definition \eqref{Wj} makes sense.
It is apparent that $W_j$ is linear.
Then prove that if $\phi_k \to \phi$ in $\mathscr{S}(\mathbb{R}^n)$ then
$\inner{W_j}{\phi_k} \to \inner{W_j}{\phi}$. This being true means that $W_j \in \mathscr{S}'(\mathbb{R}^n)$, namely
that each $W_j$ is a tempered distribution. 

For a function $f:\mathbb{R}^n \to \mathbb{C}$, write
\[
\widetilde{f}(x) = f(-x), \qquad (\tau_yf)(x)=f(x-y).
\]
For $u \in \mathscr{S}'(\mathbb{R}^n)$ and $h \in \mathscr{S}(\mathbb{R}^n)$, define
\[
\inner{h*u}{\phi} = \inner{u}{\widetilde{h}*\phi},\qquad \phi \in \mathscr{S}(\mathbb{R}^n).
\]
It is a fact that $h*u \in \mathscr{S}'(\mathbb{R}^n)$, and this tempered distribution is induced by the $C^\infty$ function
$x \mapsto \inner{u}{\tau_x \widetilde{h}}$.\footnote{Loukas Grafakos, {\em Classical Fourier Analysis}, second ed.,
p.~116, Theorem 2.3.20.}
The Fourier transform of a tempered distribution $u$ is defined by
\[
\inner{\widehat{u}}{\phi} = \inner{u}{\widehat{\phi}},\qquad \phi \in \mathscr{S}(\mathbb{R}^n),
\]
where
\[
\widehat{\phi}(\xi) = \int_{\mathbb{R}^n} e^{-2\pi ix\cdot \xi} \phi(x) dx,\qquad \xi \in \mathbb{R}^n.
\]
It is a fact that $\widehat{u}$ is itself a tempered distribution.
Finally, for a tempered distribution $u$ and a Schwartz function $h$, we define
\[
\inner{hu}{\phi} = \inner{u}{h\phi},\qquad \phi \in \mathscr{S}(\mathbb{R}^n).
\]
It is a fact that $hu$ is itself a tempered distribution.
It is proved that\footnote{Loukas Grafakos, {\em Classical Fourier Analysis}, second ed.,
p.~120, Proposition 2.3.22.}
\[
\widehat{\phi*u} = \widehat{\phi} \widehat{u}.
\]
The left-hand side is the Fourier transform of the tempered distribution $\phi*u$, and the right-hand side is the product
of the Schwartz function $\widehat{\phi}$ and the tempered distribution $\widehat{u}$. 

\begin{lemma}
For $1 \leq j \leq n$, for $\phi \in \mathscr{S}(\mathbb{R}^n)$, and for $x \in \mathbb{R}^n$,
\[
(R_j \phi)(x) = (\phi*W_j)(x).
\]
\end{lemma}

We will use the following identity for integrals over $S^{n-1}$.\footnote{Loukas Grafakos, {\em Classical Fourier Analysis}, second ed.,
p.~261, Lemma 4.1.15.}

\begin{lemma}
For $\xi \neq 0$ and for $1 \leq j \leq n$,
\[
\int_{S^{n-1}} \sgn(\xi \cdot \theta) \theta_j d\sigma(\theta)    =\frac{2\omega_{n-2}}{n-1}
\frac{\xi_j}{|\xi|}.
\]
\label{spherical}
\end{lemma}
\begin{proof}
It is a fact that
\begin{equation}
\int_{S^{n-1}} \sgn(\theta_k)\theta_j d\sigma(\theta)=
\begin{cases}
0&k \neq j\\
\int_{S^{n-1}} |\theta_j| d\sigma(\theta)&k = j.
\end{cases}
\label{cancellation}
\end{equation}

It suffices to prove the claim when $\xi \in S^{n-1}$. For $1 \leq j \leq n$ there is 
$A_j=(a_{i,k})_{i,k} \in SO_n(\mathbb{R})$ such that\footnote{\url{http://www.math.umn.edu/~garrett/m/mfms/notes/08_homogeneous.pdf}}
\[
A_j e_j = \xi,
\]
for which $a_{i,j}=\xi_i$.
Using that $A_j^T=A_j^{-1}$ and that 
$\sigma$ is invariant under $O(n)$
 we calculate
\begin{align*}
\int_{S^{n-1}} \sgn(\xi\cdot \theta) \theta_j d\sigma(\theta)&=
\int_{S^{n-1}} \sgn(A_j e_j \cdot \theta) (AA^{-1} \theta)_j d\sigma(\theta)\\
&=\int_{S^{n-1}} \sgn(e_j \cdot A_j^{-1} \theta) (AA^{-1} \theta)_j d\sigma(\theta)\\
&=\int_{S^{n-1}} \sgn(e_j \cdot \theta) (A\theta)_j d({A_j^{-1}}_*\sigma)(\theta)\\
&=\int_{S^{n-1}} \sgn(e_j \cdot \theta) (A\theta)_j d\sigma(\theta)\\
&=\int_{S^{n-1}} \sgn(\theta_j)  \sum_{k=1}^n a_{j,k} \theta_k d\theta.
\end{align*}
Applying Lemma \ref{cancellation} and $a_{j,j}=\xi_j$, this becomes
\[
\int_{S^{n-1}} \sgn(\xi\cdot \theta) \theta_j d\sigma(\theta)=
\int_{S^{n-1}} \xi_j |\theta_j| d\sigma(\theta)
=\frac{\xi_j}{|\xi|} \int_{S^{n-1}} |\theta_j| d\sigma(\theta).
\]
Hence for each $1 \leq j \leq n$,
\[
\int_{S^{n-1}} \sgn(\xi\cdot \theta) \theta_j d\sigma(\theta)=
\frac{\xi_j}{|\xi|} \int_{S^{n-1}} |\theta_1| d\sigma(\theta).
\]
It is a fact that\footnote{Loukas Grafakos, {\em Classical Fourier Analysis}, second ed.,
p.~441, Appendix D.2.}
\[
\int_{RS^{n-1}} f(\theta) d\sigma(\theta) = 
\int_{-R}^R \int_{\sqrt{R^2-s^2} S^{n-2}} f(s,\phi) d\phi \frac{Rds}{\sqrt{R^2-s^2}}.
\]
Using this with $f(\theta)=f(\theta_1,\ldots,\theta_n)=|\theta_1|$ 
and using that the measure of $RS^{n-2}$ is
$R^{n-2} \omega_{n-1}$,
we calculate
\begin{align*}
\int_{S^{n-1}} |\theta_1| d\sigma(\theta)&=\int_{-1}^1 \int_{\sqrt{1-s^2}S^{n-2}} |s| d\phi \frac{ds}{\sqrt{1-s^2}}\\
&=\int_{-1}^1 (1-s^2)^{\frac{n-2}{2}-\frac{1}{2}} \omega_{n-2} |s| d\phi\\
&=2\omega_{n-2} \int_{0}^1 (1-s^2)^{\frac{n-3}{2}}s ds\\
&=\omega_{n-2} \int_0^1 u^{\frac{n-3}{2}} du\\
&=\frac{2\omega_{n-2}}{n-1}.
\end{align*}
\end{proof}


We now calculate the Fourier transform
of the $W_j$.
We show that the Fourier transform of the tempered distribution $W_j$
is  induced by the function $\xi \mapsto -i \frac{\xi_j}{|\xi|}$.\footnote{Loukas Grafakos, {\em Classical Fourier Analysis}, second ed.,
p.~260, Proposition 4.1.14.}

\begin{theorem}
For $1 \leq j \leq n$ and for $\phi \in \mathscr{S}(\mathbb{R}^n)$,
\[
\inner{\widehat{W}_j}{\phi} = \int_{\mathbb{R}^n} -i \phi(x) \frac{x_j}{|x|} dx.
\]
\label{Wjtheorem}
\end{theorem}
\begin{proof}
We calculate
\begin{align*}
\inner{W_j}{\widehat{\phi}}&= \frac{\Gamma\left(\frac{n+1}{2}\right)}{\pi^{\frac{n+1}{2}}} \lim_{\epsilon \to 0}
\int_{|\xi| \geq \epsilon}K_j(\xi) \widehat{\phi}(\xi)  d\xi\\
&= \frac{\Gamma\left(\frac{n+1}{2}\right)}{\pi^{\frac{n+1}{2}}} \lim_{\epsilon \to 0}
\int_{\epsilon \leq |\xi| \leq 1/\epsilon}K_j(\xi) \widehat{\phi}(\xi)  d\xi\\
&=\frac{\Gamma\left(\frac{n+1}{2}\right)}{\pi^{\frac{n+1}{2}}} \lim_{\epsilon \to 0}
\int_{\epsilon \leq |\xi| \leq 1/\epsilon} \left(\int_{\mathbb{R}^n} e^{-2\pi ix\cdot \xi} \phi(x) dx\right)
\frac{\xi_j}{|\xi|^{n+1}} d\xi\\
&=\frac{\Gamma\left(\frac{n+1}{2}\right)}{\pi^{\frac{n+1}{2}}} \lim_{\epsilon \to 0} \int_{\mathbb{R}^n}
\phi(x) \left( \int_{\epsilon \leq |\xi| \leq 1/\epsilon} e^{-2\pi ix\cdot \xi} \frac{\xi_j}{|\xi|^{n+1}} d\xi \right) dx.
\end{align*}
For the inside integral, because $\theta \mapsto \cos(-2\pi rx_j \theta_j) \theta_j$  is an odd function,
\begin{align*}
 \int_{\epsilon \leq |\xi| \leq 1/\epsilon} e^{-2\pi ix\cdot \xi} \frac{\xi_j}{|\xi|^{n+1}} d\xi&=
 \int_{\epsilon \leq r \leq 1/\epsilon} \left( \int_{S^{n-1}} e^{-2\pi ix\cdot (r\theta)} 
\frac{r\theta_j}{r^{n+1}} d\sigma(\theta) \right) r^{n-1} dr \\
&=
 \int_{\epsilon \leq r \leq 1/\epsilon} \left( \int_{S^{n-1}} e^{-2\pi ir x\cdot \theta} 
\theta_j d\sigma(\theta) \right) r^{-1} dr \\
&= \int_{\epsilon \leq r \leq 1/\epsilon} \left( \int_{S^{n-1}} i\sin(-2\pi rx\cdot \theta) \theta_j d\sigma(\theta) \right)
r^{-1} dr\\
&= -i \int_{\epsilon \leq r \leq 1/\epsilon}\left( \int_{S^{n-1}} \sin(2\pi rx\cdot\theta) \theta_j d\sigma(\theta) \right)
r^{-1} dr\\
&=-i \int_{S^{n-1}} \left( \int_{\epsilon \leq r \leq 1/\epsilon} \sin(2\pi rx\cdot \theta) r^{-1} dr \right) \theta_j
d\sigma(\theta). 
\end{align*}
Call the whole last expression $f_\epsilon(x)$.
It is a fact that
for $0<a<b<\infty$,
\[
\left| \int_a^b \frac{\sin t}{t} dt \right| \leq 4,
\]
thus for $x \neq 0$,
\[
|f_\epsilon(x)| \leq 4\omega_{n-1}.
\]
As\footnote{Loukas Grafakos, {\em Classical Fourier Analysis}, second ed.,
p.~263, Exercise 4.1.1.}
\[
\lim_{\epsilon \to 0}f_\epsilon(x)=- i\int_{S^{n-1}} \sgn(x\cdot \theta) \frac{\pi}{2} \theta_j d\sigma(\theta),
\]
applying the dominated convergence theorem yields
\[
\begin{split}
& \lim_{\epsilon \to 0} \int_{\mathbb{R}^n}
\phi(x) \left( \int_{\epsilon \leq |\xi| \leq 1/\epsilon} e^{-2\pi ix\cdot \xi} \frac{\xi_j}{|\xi|^{n+1}} d\xi \right) dx\\
=&\int_{\mathbb{R}^n} \phi(x) \left(- i\int_{S^{n-1}} \sgn(x\cdot \theta) \frac{\pi}{2} \theta_j d\sigma(\theta)\right) dx\\
=&-i\frac{\pi}{2} \int_{\mathbb{R}^n}\phi(x) \left( \int_{S^{n-1}} \sgn(x\cdot \theta) \theta_j d\sigma(\theta) \right) dx.
\end{split}
\]
Then using Lemma \ref{spherical} and putting the above together we get
\begin{align*}
\inner{W_j}{\widehat{\phi}}&=\frac{\Gamma\left(\frac{n+1}{2}\right)}{\pi^{\frac{n+1}{2}}}
\cdot -i\frac{\pi}{2} \int_{\mathbb{R}^n} \phi(x) \left( \int_{S^{n-1}} \sgn(x\cdot \theta) \theta_j d\sigma(\theta) \right) dx\\
&=-i\frac{\pi}{2} \frac{\Gamma\left(\frac{n+1}{2}\right)}{\pi^{\frac{n+1}{2}}} 
\int_{\mathbb{R}^n}\phi(x) \frac{2\omega_{n-2}}{n-1} \frac{x_j}{|x|} dx.
\end{align*}
We work out that
\[
\frac{\pi}{2} \frac{\Gamma\left(\frac{n+1}{2}\right)}{\pi^{\frac{n+1}{2}}} \cdot \frac{2\omega_{n-2}}{n-1} = 1,
\]
and therefore
\[
\inner{W_j}{\widehat{\phi}} = -i \int_{\mathbb{R}^n} \phi(x) \frac{x_j}{|x|} dx,
\]
completing the proof.
\end{proof}

Because $R_j h = h*W_j$, 
\[
\inner{\widehat{R_j h}}{\phi}=
\inner{\widehat{h} \widehat{W}_j}{\phi}
=\inner{\widehat{W}_j}{\widehat{h} \phi}
=\int_{\mathbb{R}^n} -i \widehat{h}(\xi) \phi(\xi) \frac{\xi_j}{|\xi|} d\xi.
\]

\begin{theorem}
For $1 \leq j \leq n$ and for $h \in \mathscr{S}(\mathbb{R}^n)$,
\[
\widehat{R_j h}(\xi) = -i\frac{\xi_j}{|\xi|} \widehat{h}(\xi),\qquad \xi \in \mathbb{R}^n.
\]
\label{rieszfourier}
\end{theorem}

In other words, the \textbf{multiplier} of the Riesz transform $R_j$ is $m_j(\xi) =  -i\frac{\xi_j}{|\xi|}$. 


\section{Properties of the Riesz transform}
\begin{theorem}
\[
-I = \sum_{j=1}^n R_j^2,
\]
where $I(h)  = h$ for $h \in \mathscr{S}(\mathbb{R}^n)$.
\end{theorem}
\begin{proof}
For $h \in \mathscr{S}(\mathbb{R}^n)$,
\begin{align*}
\widehat{R_j^2 h}(\xi) = -i\frac{\xi_j}{|\xi|} \widehat{R_j h}(\xi)
=-i\frac{\xi_j}{|\xi|} \cdot -i\frac{\xi_j}{|\xi|} \widehat{h}(\xi)
=-\frac{\xi_j^2}{|\xi|^2} \widehat{h}(\xi),
\end{align*}
hence
\[
\sum_{j=1}^n \widehat{R_j^2 h}  = -\widehat{h}.
\]
Taking the inverse Fourier transform,
\[
\sum_{j=1}^n R_j^2 h  = -h,
\]
i.e.
\[
\sum_{j=1}^n R_j^2 = -I.
\]
\end{proof}


For a tempered distribution $u$, for $1 \leq j \leq n$, we define
\[
\inner{\partial_j u}{\phi} = (-1) \inner{u}{\partial_j \phi},\qquad \phi \in \mathscr{S}(\mathbb{R}^n).
\]
It is a fact that $\partial_j u$ is itself a tempered distribution. 
One proves that
\[
\widehat{\partial_j u} = (2\pi i\xi_j) \widehat{u}.
\]
Each side of the above equation is a tempered distribution. Then
\[
\widehat{\Delta u} = \sum_{j=1}^n \widehat{\partial_j^2 u}
=\sum_{j=1}^n (2\pi i\xi_j)^2 \widehat{u}
=-4\pi^2  \sum_{j=1}^n \xi_j^2 \widehat{u}
=-4\pi^2 |\xi|^2 \widehat{u}.
\]

Suppose that $f$ is a Schwartz function and that $u$ is a tempered distribution satisfying
\[
\Delta u = f,
\]
called \textbf{Poisson's equation}.
Then
\[
-4\pi^2 |\xi|^2 \widehat{u} = \widehat{f}.
\]
For $1 \leq j,k \leq n$,
\begin{align*}
\partial_j \partial_k u&=\mathscr{F}^{-1}(\mathscr{F}(\partial_j\partial_k u))\\
&=\mathscr{F}^{-1}((2\pi i\xi_j)(2\pi i\xi_k) \widehat{u})\\
&=\mathscr{F}^{-1}\left( -4\pi^2 \xi_j \xi_k \cdot \frac{\widehat{f}}{-4\pi^2 |\xi|^2}\right)\\
&=\mathscr{F}^{-1}\left(\frac{\xi_j \xi_k}{|\xi|^2} \widehat{f}\right).
\end{align*}
Using Theorem \ref{rieszfourier},
\begin{align*}
R_j R_k f &= \mathscr{F}^{-1} \mathscr{F}(R_j R_k f)\\
&=\mathscr{F}^{-1}\left( -i\frac{\xi_j}{|\xi|} \widehat{R_k f}\right)\\
&=\mathscr{F}^{-1}\left( -i\frac{\xi_j}{|\xi|} \cdot -i\frac{\xi_k}{|\xi|} \widehat{f}\right)\\
&=\mathscr{F}^{-1}\left(-\frac{\xi_j \xi_k}{|\xi|^2} \widehat{f}\right).
\end{align*}
Therefore
\[
\partial_j \partial_k u = -R_j R_k f.
\]

\begin{theorem}
If $f$ is a Schwartz function and $u$ is a tempered distribution satisfying
\[
\Delta u =f,
\]
then for $1 \leq j,k \leq n$,
\[
\partial_j \partial_k u = -R_j R_k f.
\]
\end{theorem}





\end{document}