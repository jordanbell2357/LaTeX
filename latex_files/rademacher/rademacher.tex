\documentclass{article}
\usepackage{amsmath,amssymb,graphicx,subfig,mathrsfs,amsthm,siunitx}
%\usepackage{tikz-cd}
%\usepackage{hyperref}
\newcommand{\inner}[2]{\left\langle #1, #2 \right\rangle}
\newcommand{\tr}{\ensuremath\mathrm{tr}\,} 
\newcommand{\Span}{\ensuremath\mathrm{span}} 
\def\Re{\ensuremath{\mathrm{Re}}\,}
\def\Im{\ensuremath{\mathrm{Im}}\,}
\newcommand{\id}{\ensuremath\mathrm{id}} 
\newcommand{\rank}{\ensuremath\mathrm{rank\,}} 
\newcommand{\diam}{\ensuremath\mathrm{diam}} 
\newcommand{\osc}{\ensuremath\mathrm{osc}} 
\newcommand{\co}{\ensuremath\mathrm{co}\,} 
\newcommand{\cco}{\ensuremath\overline{\mathrm{co}}\,}
\newcommand{\supp}{\ensuremath\mathrm{supp}\,}
\newcommand{\ext}{\ensuremath\mathrm{ext}\,}
\newcommand{\ba}{\ensuremath\mathrm{ba}\,}
\newcommand{\cl}{\ensuremath\mathrm{cl}\,}
\newcommand{\dom}{\ensuremath\mathrm{dom}\,}
\newcommand{\Cyl}{\ensuremath\mathrm{Cyl}\,}
\newcommand{\extreals}{\overline{\mathbb{R}}}
\newcommand{\upto}{\nearrow}
\newcommand{\downto}{\searrow}
\newcommand{\norm}[1]{\left\Vert #1 \right\Vert}
\newtheorem{theorem}{Theorem}
\newtheorem{lemma}[theorem]{Lemma}
\newtheorem{proposition}[theorem]{Proposition}
\newtheorem{corollary}[theorem]{Corollary}
\theoremstyle{definition}
\newtheorem{definition}[theorem]{Definition}
\newtheorem{example}[theorem]{Example}
\begin{document}
\title{Rademacher functions}
\author{Jordan Bell\\ \texttt{jordan.bell@gmail.com}\\Department of Mathematics, University of Toronto}
\date{\today}

\maketitle


\section{Binary expansions}
Define $S:\{0,1\}^\mathbb{N} \to [0,1]$ by
\[
S(\sigma) = \sum_{k=1}^\infty \frac{\sigma_k}{2^k}, \qquad \sigma \in \{0,1\}^\mathbb{N}.
\]
For example, for $\sigma_1=0$ and $\sigma_2=1,\sigma_3=1,\ldots$, 
\[
S(\sigma) = \frac{0}{2}+\frac{1}{4}+\frac{1}{8}+\cdots = \frac{1}{2};
\]
  for $\sigma_1=1$ and $\sigma_2=0$, $\sigma_3=0, \ldots$,
\[
S(\sigma) = \frac{1}{2}+\frac{0}{4}+\frac{0}{8} + \cdots = \frac{1}{2}.
\]
Let $\sigma \in \{0,1\}^\mathbb{N}$. If there is some $n  \in \mathbb{N}$ such that $\sigma_n=0$ and $\sigma_k=1$ for all $k \geq n+1$, then
defining
\[
\tau_k = \begin{cases}
\sigma_k&k \leq n-1\\
1&k=n\\
0&k \geq n,
\end{cases}
\]
we have
\[
S(\sigma) = \sum_{k=1}^{n-1} \frac{\sigma_k}{2^k} + \sum_{k=n+1}^\infty \frac{1}{2^k} = 
 \sum_{k=1}^{n-1} \frac{\sigma_k}{2^k} + \frac{1}{2^n}
 =S(\tau).
\]
One proves that if either (i) there is some $n \in \mathbb{N}$ such that
$\sigma_n=0$ and $\sigma_k=1$ for all $k \geq n+1$ or (ii) there is some $n \in \mathbb{N}$
such that $\sigma_n=1$ and $\sigma_k=0$ for all $k \geq n+1$, then $S^{-1}(S(\sigma))$ contains exactly two elements, and that
otherwise $S^{-1}(S(\sigma))$ contains exactly one element. 

In words, except for the sequence whose terms are only 0 or the sequence whose terms are only 1, $S^{-1}(S(\sigma))$ contains
exactly two elements when $\sigma$ is eventually $0$ or eventually $1$, and $S^{-1}(S(\sigma))$ contains exactly
one element otherwise.

We define $\epsilon:[0,1] \to \{0,1\}^\mathbb{N}$ by taking $\epsilon(t)$
to be  the unique element of $S^{-1}(t)$ if $S^{-1}(t)$ contains exactly one element, and to be the element of $S^{-1}(t)$
that is eventually $0$ if $S^{-1}(t)$ contains exactly two elements.
For $k \in \mathbb{N}$ we define $\epsilon_k:[0,1] \to \{0,1\}$ by
\[
\epsilon_k(t) = \epsilon(t)_k, \qquad t \in [0,1].
\]
Then, for all $t \in [0,1]$,
\begin{equation}
t =S(\epsilon(t))= \sum_{k=1}^\infty \frac{\epsilon_k(t)}{2^k},
\label{binexpansion}
\end{equation}
which we call \textbf{the binary expansion of $t$}.


\section{Rademacher functions}
For $k \in \mathbb{N}$, the \textbf{$k$th Rademacher function} $r_k:[0,1] \to \{-1,+1\}$ is defined by
\[
r_k(t) = 1-2\epsilon_k(t), \qquad t \in [0,1].
\]
We can rewrite the binary expansion of $t \in [0,1]$ in \eqref{binexpansion} as
\begin{equation}
\sum_{k=1}^\infty \frac{r_k(t)}{2^k} = \sum_{k=1}^\infty \left(\frac{1}{2^k} - 2\cdot \frac{\epsilon_k(t)}{2^k} \right)
=1-2\sum_{k=1}^\infty \frac{\epsilon_k(t)}{2^k}=1-2t.
\label{binary}
\end{equation}


Define $r:\mathbb{R} \to \{-1,+1\}$ by 
\[
r(x) = (-1)^{[x]},
\]
where $[x]$ denotes the greatest integer $\leq x$. Thus, for $0 \leq x <1$ we have $r(x)=1$, for $1 \leq x < 2$ we have
$r(x)=-1$, and $r$ has period $2$. 

\begin{lemma}
For any $n \in \mathbb{N}$,
\[
r_n(t) =(-1)^{[2^nt]}= r(2^n t), \qquad t \in [0,1]
\]
\label{rformula}
\end{lemma}



In the following theorem we use the Rademacher functions to prove an identity for trigonometric functions.\footnote{Mark Kac, {\em Statistical Independence in Probability, Analysis and Number Theory}, p.~4, \S 3.}

\begin{theorem} For any nonzero real $x$,
\[
 \prod_{k=1}^\infty \cos \frac{x}{2^k} = \frac{\sin x}{x}.
\]
\end{theorem}
\begin{proof}
Let $n \in \mathbb{N}$ and let $c_1,\ldots,c_n \in \mathbb{R}$. 
The function
\[
\sum_{k=1}^n c_k r_k
\]
is constant on each of the intervals
 \begin{equation}
 \left[\frac{s}{2^n},\frac{s+1}{2^n} \right), \qquad 0 \leq s \leq 2^n-1.
 \label{sintervals}
 \end{equation}
 There is a bijection between  $\Delta_n=\{-1,+1\}^n$
 and 
the collection of intervals \eqref{sintervals}. Without explicitly describing this bijection,
we have
\begin{eqnarray*}
\int_0^1 \exp\left(i\sum_{k=1}^n c_k r_k(t) \right) dt &=& 
\sum_{s=0}^{2^n-1} \int_{s\cdot 2^{-n}}^{(s+1)\cdot 2^{-n}} 
\exp\left(i\sum_{k=1}^n c_k r_k(t) \right) dt\\
&=&\sum_{\delta \in \Delta_n} \frac{1}{2^n}  \exp\left(i\sum_{k=1}^n \delta_k c_k \right)\\
&=&\sum_{\delta \in \Delta_n} \prod_{k=1}^n \frac{e^{i\delta_kc_k}}{2}\\
&=&\prod_{k=1}^n \frac{e^{ic_k}+e^{-ic_k}}{2},
\end{eqnarray*}
giving
\begin{equation}
\int_0^1 \exp\left(i\sum_{k=1}^n c_k r_k(t) \right) dt =\prod_{k=1}^n \cos c_k.
\label{cosformula}
\end{equation}

We have
\begin{equation}
\int_0^1 e^{ix(1-2t)} dt  = e^{ix} \frac{e^{-2ixt}}{-2ix} \bigg|_0^1
 = e^{ix} \left(\frac{e^{-2ix}}{-2ix} + \frac{1}{2ix} \right)
 =\frac{\sin x}{x}.
 \label{sinformula}
\end{equation}
Using \eqref{binary} we check that
the sequence of functions $\sum_{k=1}^n \frac{r_k(t)}{2^k}$ converges uniformly on $[0,1]$
to $1-2t$, and hence using \eqref{sinformula} we get
\[
\int_0^1 \exp\left(ix \sum_{k=1}^n  \frac{r_k(t)}{2^k} \right) dt \to \int_0^1 e^{ix(1-2t}) dt
=\frac{\sin x}{x}
\]
as $n \to \infty$. Combining this with \eqref{cosformula}, which we apply with $c_k=\frac{x}{2^k}$,  we get
\[
 \prod_{k=1}^n \cos \frac{x}{2^k} \to \frac{\sin x}{x}
\]
as $n \to \infty$, proving the claim.
\end{proof}

We now give an explicit formula for the measure of those $t$ for which exactly $l$ of $r_1(t),\ldots,r_n(t)$ are equal to $1$.\footnote{Mark Kac, {\em Statistical Independence in Probability, Analysis and Number Theory}, pp.~8--9.}
We denote by $\mu$ Lebesgue measure on $\mathbb{R}$. We can interpret the following formula as stating the probability that out of $n$
tosses of a coin, exactly $l$ of the outcomes are heads. 

\begin{theorem}
For $n \in \mathbb{N}$ and $0 \leq l \leq n$,
\[
\mu\{t \in [0,1]: r_1(t)+\cdots+r_n(t)=2l-n\} = \frac{1}{2^n} \binom{n}{l}.
\]
\label{numberheads}
\end{theorem}
\begin{proof}
Define $\phi:[0,1] \to \mathbb{R}$ by
\[
\phi(t) = \frac{1}{2\pi} \int_0^{2\pi} e^{ix\left(-(2l-n)+\sum_{k=1}^n r_k(t)\right)} dx
\]
But for $m \in \mathbb{Z}$,
\begin{equation}
\frac{1}{2\pi} \int_0^{2\pi} e^{imx} dx =\delta_{m,0}= \begin{cases}
1&m=0\\
0&m \neq0,
\end{cases}
\label{orthogonality}
\end{equation}
hence
\[
\phi(t)=
\begin{cases}
1&\sum_{k=1}^n r_k(t)=2l-n\\
0&\sum_{k=1}^n r_k(t) \neq 2l-n.
\end{cases}
\]
Therefore
\begin{eqnarray*}
\mu \bigg\{ t \in [0,1]: \sum_{k=1}^n r_k(t) = 2l-n\bigg\} & = & \int_0^1 \phi(t) dt\\
&=&\int_0^1  \frac{1}{2\pi} \int_0^{2\pi} e^{ix\left(-(2l-n)+\sum_{k=1}^n r_k(t)\right)} dx dt\\
&=&\frac{1}{2\pi}  \int_0^{2\pi} e^{-ix(2l-n)} \int_0^1  e^{ix\sum_{k=1}^n r_k(t)} dt dx\\
&=&\frac{1}{2\pi}  \int_0^{2\pi} e^{-ix(2l-n)} \cos^n x dx;
\end{eqnarray*}
the last equality uses \eqref{cosformula} with $c_1=x,\ldots,c_n=x$. 
Furthermore, writing
\[
\cos^n x = 2^{-n} (e^{ix}+e^{-ix}) = 2^{-n} \sum_{k=0}^n \binom{n}{k} e^{ix(2k-n)},
\]
we calculate using \eqref{orthogonality} that
\begin{eqnarray*}
\frac{1}{2\pi} \int_0^{2\pi} e^{-ix(2l-n)} \cos^n x dx&=&2^{-n} \sum_{k=0}^n \binom{n}{k}  \frac{1}{2\pi}\int_0^1 e^{-ix(2l-n)} e^{ix(2k-n)} dx\\
 &=&2^{-n}  \sum_{k=0}^n \binom{n}{k} \frac{1}{2\pi} \int_0^1 e^{ix(2k-2l)} dx\\
 &=&2^{-n}  \sum_{k=0}^n \binom{n}{k} \delta_{2k-2l,0}\\
 &=&2^{-n}  \sum_{k=0}^n \binom{n}{k} \delta_{k,l}\\
 &=&2^{-n} \binom{n}{l},
\end{eqnarray*}
proving the claim.
\end{proof}



We now prove that the expected value of a product of distinct Rademacher functions is equal to the product of their
expected values.\footnote{Masayoshi Hata, {\em Problems and Solutions in Real
Analysis}, p.~185, Solution 13.2.}

\begin{theorem}
If $k_1,\ldots,k_n$ are positive integers and $k_1<\cdots<k_n$, then
\[
\int_0^1 r_{k_1}(t) \cdots r_{k_n}(t) dt = 0.
\]
\label{independent}
\end{theorem}
\begin{proof}
Write $J=\int_0^1 r_{k_1}(t) \cdots r_{k_n}(t) dt$ and define
\[
\phi(x) = \prod_{s=2}^n r(2^{k_s-k_1}x), \qquad x \in \mathbb{R},
\]
which satisfies
\[
\phi(x+1)=\prod_{s=2}^n r(2^{k_s-k_1}x+2^{k_s-k_1})
=\prod_{s=2}^n r(2^{k_s-k_1}x)=\phi(x).
\]
Hence, as $\phi$ has period $1$ and $r$ has period $2$,
\begin{eqnarray*}
J &=& \int_0^1 r_{k_1}(t) \phi(2^{k_1}t) dt\\
& =& \int_0^1 r(2^{k_1}t) \phi(2^{k_1}t) dt \\
&=& \frac{1}{2^{k_1}} \int_0^{2^{k_1}} r(x) \phi(x) dx\\
&=&\frac{1}{2^{k_1}} \sum_{j=0}^{2^{k_1-1}-1} \int_{2j}^{2j+2} r(x) \phi(x) dx\\
&=&\frac{1}{2^{k_1}} \sum_{j=0}^{2^{k_1-1}-1} \int_0^2 r(x) \phi(x) dx\\
&=&\frac{1}{2} \int_0^2 r(x) \phi(x) dx.
\end{eqnarray*}
But, as $\phi$ has period $1$,
\[
 \int_0^2 r(x) \phi(x) dx = \int_0^1 \phi(x) dx - \int_1^2 \phi(x) dx =  \int_0^1 \phi(x) dx - \int_0^1 \phi(x) dx =0,
\]
hence  
$J=0$,
proving the claim.
\end{proof}

For each $n \in \mathbb{N}$, if $f$ is a function defined on the integers we define
\[
I_n(f) = \int_0^1 f\left( \sum_{k=1}^n r_k(t) \right) dt.
\]

\begin{lemma}
For any $n \in \mathbb{N}$,
\[
I_n(x^2)=n, \qquad I_n(x^4)=3n^2-2n.
\]
\end{lemma}
\begin{proof}
Using Theorem \ref{independent} we get
\begin{eqnarray*}
I_n(x^2)&=&\int_0^1 \left(\sum_{k=1}^n r_k(t) \right)^2 dt\\
&=&\int_0^1 \sum_{k=1}^n r_k(t)^2 + \sum_{j \neq k} r_j(t) r_k(t) dt\\
&=&\int_0^1 \sum_{k=1}^n r_k(t)^2 dt\\
&=&n.
\end{eqnarray*}

Using Theorem \ref{independent} we get, since $r_j(t)^4=r_j(t)^2=1$ and $r_j(t)^3=r_j(t)$,
\begin{eqnarray*}
I_n(x^4)&=&\int_0^1 \left(\sum_{k=1}^n r_k(t) \right)^4 dt\\
&=&\int_0^1 \sum_{k=1}^n r_k(t)^4 +\binom{4}{3} \sum_{j=1}^n \sum_{k \neq j} r_j(t)^3 r_k(t)+
\binom{4}{2} \sum_{j=1}^n \sum_{k \neq j} r_j(t)^2 r_k(t)^2\\
&&+\binom{4}{2}\sum_{j=1}^n \sum_{\textrm{$j,k,l$ all distinct}} r_j(t)^2 r_k(t) r_l(t)\\
&&+\sum_{\textrm{$j,k,l,m$ all distinct}} r_j(t) r_k(t) r_l(t) r_m(t) dt\\
&=& n+\binom{4}{2}n(n-1).
\end{eqnarray*}
\end{proof}


Our proof of the next identity follows Hata.\footnote{Masayoshi Hata, {\em Problems and Solutions in Real Analysis}, p.~188, Solution 13.6.}

\begin{lemma}
For any $n \in \mathbb{N}$,
\[
I_n(|x|) = \frac{2}{\pi} \int_0^\infty \frac{1-\cos^n x}{x^2} dx.
\]
\label{absoluteformula}
\end{lemma}
\begin{proof}
For $n \in \mathbb{N}$ and $c_1,\ldots,c_n \in \mathbb{R}$,
\[
\int_0^1 \exp\left(i\sum_{k=1}^n c_k r_k(t) \right) dt = 
\int_0^1 \cos \left(\sum_{k=1}^n c_k r_k(t) \right) dt
+i\int_0^1 \sin\left(\sum_{k=1}^n c_k r_k(t) \right) dt,
\]
and since \eqref{cosformula} tells us that the left-hand side of the above is real, it follows that we can
write \eqref{cosformula} as
\begin{equation}
\int_0^1 \cos \left(\sum_{k=1}^n c_k r_k(t) \right) dt =\prod_{k=1}^n \cos c_k.
\label{cosproduct}
\end{equation}

Suppose that $\xi$ is a positive real number. Using $t=x\xi$ and doing integration by parts,
\begin{eqnarray*}
\int_0^\infty \frac{1-\cos x\xi}{x^2} dx&=&\xi\int_0^\infty \frac{1-\cos t}{t^2} dt\\
&=&\xi\frac{1-\cos t}{-t} \bigg|_0^\infty  
+\xi\int_0^\infty \frac{\sin t}{t} dt \\
&=&\xi\int_0^\infty \frac{\sin t}{t} dt\\
&=&\xi \frac{\pi}{2}.
\end{eqnarray*}
It is thus apparent that for any real $\xi$,
\[
\int_0^\infty \frac{1-\cos x\xi}{x^2} dx = |\xi| \frac{\pi}{2}.
\]
For any $n \in \mathbb{N}$, applying the above with $\xi=\sum_{k=1}^n r_k(t)$ we get
\begin{eqnarray*}
I_n(|x|)&=&\frac{2}{\pi} \int_0^1 \left| \sum_{k=1}^n r_k(t) \right|\frac{\pi}{2} dt\\
&=&\frac{2}{\pi} \int_0^1 \int_0^\infty \frac{1-\cos\left(x \sum_{k=1}^n r_k(t) \right)}{x^2} dx dt\\
&=&\frac{2}{\pi} \int_0^\infty \frac{1}{x^2} \int_0^1 1-\cos\left(x \sum_{k=1}^n r_k(t) \right) dt dx\\
&=&\frac{2}{\pi} \int_0^\infty \frac{1}{x^2} \left( 1- I_n(\cos x\cdot) \right) dx.
\end{eqnarray*}
Applying \eqref{cosproduct} with $c_k=x$ for each $k$, this is equal to
\[
\frac{2}{\pi}\int_0^\infty \frac{1}{x^2}\left(1-\prod_{k=1}^n \cos x\right) dx
=\frac{2}{\pi} \int_0^\infty \frac{1-\cos^n x}{x^2} dx,
\]
completing the proof.
\end{proof}

We use the above formula for $I_n(|x|)$  to obtain an asymptotic formula for $I_n(|x|)$.\footnote{Mark Kac, {\em Statistical Independence in Probability, Analysis and Number Theory}, p.~12, Masayoshi Hata, {\em Problems and Solutions in Real Analysis}, p.~188, Solution 13.6.}

\begin{theorem}
\[
I_n(|x|) \sim \sqrt{\frac{2}{\pi}} \sqrt{n}.
\]
\end{theorem}
\begin{proof}
By Lemma \ref{absoluteformula},
\[
I_n(|x|) = \frac{2}{\pi} \int_0^\infty \frac{1-\cos^n x}{x^2} dx.
\]
For $0 \leq \epsilon <1$, define $\phi_\epsilon:\left[0,\frac{\pi}{2} \right) \to \mathbb{R}$ by
\[
\phi_\epsilon(x) = \frac{x^2}{2(1-\epsilon)}+\log \cos x.
\]
We also define
\[
\alpha_\epsilon = \arccos \sqrt{1-\epsilon}, \qquad \beta_\epsilon = \int_{\alpha_\epsilon}^\infty \frac{1-\cos^n x}{x^2} dx,
\]
and for $\sigma>0$,
\[
K_{\epsilon,\sigma} = \int_0^{\alpha_\epsilon} \frac{1-\exp\left(-\frac{nx^2}{\sigma} \right)}{x^2} dx.
\]

Let $0<\epsilon<1$. Until the end of the proof, at which point we take $\epsilon \to 0$, we shall keep $\epsilon$ fixed.
For $0<x<\alpha_\epsilon$ we have, using $\arccos\sqrt{1-\epsilon} \leq \sqrt{\epsilon}$,
\[
\phi_0(x) = \frac{x^2}{2}+\log \cos x < \frac{\epsilon}{2}+ \log \sqrt{1-\epsilon}
= \frac{\epsilon}{2}+ \frac{1}{2} \log(1-\epsilon)<0,
\]
hence 
\[
\cos x < \exp\left(-\frac{x^2}{2}\right).
\]
On the other hand, 
\[
\phi_\epsilon'(x) = \frac{x}{1-\epsilon}-\tan x, \qquad \phi_\epsilon''(x) = \frac{1}{1-\epsilon}-\sec^2 x,
\]
so $\phi_\epsilon(0)=\phi_\epsilon'(0)=0$ and $\phi_\epsilon''(t)> 0$ for all $0 \leq t<\alpha_\epsilon$,
giving
\[
\phi_\epsilon(x)>0,
\]
and hence
\[
\exp\left(-\frac{x^2}{2(1-\epsilon)}\right)<\cos x.
\]
Collecting what we have established so far, for $0<x<\alpha_\epsilon$ we have
\[
\exp\left(-\frac{x^2}{2(1-\epsilon)}\right)<\cos x < \exp\left(-\frac{x^2}{2}\right).
\]
This shows that
\[
K_{\epsilon,2(1-\epsilon)} = \int_0^{\alpha_\epsilon}  \frac{1-\exp\left(-\frac{nx^2}{2(1-\epsilon)} \right)}{x^2} dx
\geq \int_0^{\alpha_\epsilon} \frac{1-\cos^n x}{x^2} dx,
\]
and therefore
\[
K_{\epsilon,2(1-\epsilon)}  + \beta_\epsilon \geq \frac{\pi}{2} I_n(|x|).
\]
On the other hand,
\[
K_{\epsilon,2} = \int_0^{\alpha_\epsilon} \frac{1-\exp\left(-\frac{nx^2}{2} \right)}{x^2} dx
\leq \int_0^{\alpha_\epsilon} \frac{1-\cos^n x}{x^2} dx,
\]
so
\[
K_{\epsilon,2}  + \beta_\epsilon \leq \frac{\pi}{2} I_n(|x|).
\]
Now summarizing what we have obtained, we have
\begin{equation}
K_{\epsilon,2}  + \beta_\epsilon \leq 
\frac{\pi}{2} I_n(|x|) \leq K_{\epsilon,2(1-\epsilon)}  + \beta_\epsilon.
\label{Kinequality}
\end{equation}

For $\sigma>0$, doing the change of variable $t=\sqrt{\frac{n}{\sigma}} x$,
\[
K_{\epsilon,\sigma} = \int_0^{\alpha_\epsilon} \frac{1-\exp\left(-\frac{nx^2}{\sigma} \right)}{x^2} dx
=\sqrt{\frac{n}{\sigma}} \int_0^{\sqrt{\frac{n}{\sigma}} \alpha_\epsilon} \frac{1-e^{-t^2}}{t^2} dt.
\]
As $n \to \infty$,
the right-hand side of this is asymptotic to
\[
\sqrt{\frac{n}{\sigma}} \int_0^{\infty} \frac{1-e^{-t^2}}{t^2} dt = \sqrt{\frac{n}{\sigma}} \sqrt{\pi}.
\]
Dividing  \eqref{Kinequality} by $\sqrt{n}$ and taking the limsup then gives 
\[
\limsup_{n \to \infty} \frac{\pi}{2} \frac{I_n(|x|)}{\sqrt{n}} \leq \sqrt{\frac{\pi}{2(1-\epsilon)}},
\]
or
\[
\limsup_{n \to \infty} \frac{I_n(|x|)}{\sqrt{n}} \leq \sqrt{\frac{2}{\pi(1-\epsilon)}};
\]
indeed $\beta_\epsilon$ depends on $n$, but $\beta_\epsilon < \frac{2}{\alpha_\epsilon}$, which does not depend on $n$.
Taking $\epsilon \to 0$ yields
\[
\limsup_{n \to \infty} \frac{I_n(|x|)}{\sqrt{n}} \leq \sqrt{\frac{2}{\pi}}.
\]
On the other hand, taking the liminf of \eqref{Kinequality} divided by $\sqrt{n}$ gives
\[
\liminf_{n \to \infty} \frac{\pi}{2} \frac{I_n(|x|)}{\sqrt{n}} \geq \sqrt{\frac{\pi}{2}},
\]
or
\[
\liminf_{n \to \infty}\frac{I_n(|x|)}{\sqrt{n}} \geq \sqrt{\frac{2}{\pi}}.
\]
Combining the limsup and the liminf inequalities proves the claim.
\end{proof}


\begin{lemma}
For any $\xi \in \mathbb{R}$  and $n \in \mathbb{N}$,
\[
I_n(e^{\xi|x|}) <I_n(e^{\xi x})+I_n(e^{-\xi x})= 2 (\cosh \xi)^n.
\]
\label{coshestimate}
\end{lemma}




We will use the following theorem to establish an estimate similar to but weaker than the \textbf{law of the iterated logarithm}.\footnote{Masayoshi Hata, {\em Problems and Solutions in Real Analysis}, p.~189, Solution 13.7.}


\begin{theorem}
For any $\epsilon>0$, for almost all $t \in [0,1]$,
\[
\sum_{n=1}^\infty \frac{1}{n^{2+\epsilon}} \exp\left(\sqrt{\frac{2\log n}{n}} \left| \sum_{k=1}^n r_k(t) \right| \right) dt<\infty.
\]
\label{seriesaa}
\end{theorem}
\begin{proof}
Define $f_n:[0,1] \to (0,\infty)$ by
\[
f_n(t) = \frac{1}{n^{2+\epsilon}} \exp\left( \sqrt{\frac{2\log n}{n}} \left| \sum_{k=1}^n r_k(t) \right| \right).
\]
Applying Lemma \ref{coshestimate} with $\xi =  \sqrt{\frac{2\log n}{n}}$,
\[
\int_0^1 f_n(t) dt \leq \frac{1}{n^{2+\epsilon}} \cdot 2 \cdot \left( \cosh  \sqrt{\frac{2\log n}{n}} \right)^n.
\]
It is not obvious, but we take as given the asymptotic expansion
\[
 \left( \cosh  \sqrt{\frac{2\log n}{n}} \right)^n = n-\frac{1}{3} (\log n)^2 +
 \frac{\frac{8}{45} (\log n)^3+\frac{1}{18} (\log n)^4}{n} + O(n^{-3/2}),
\]
and using this,
\[
\frac{1}{n^{2+\epsilon}} \cdot 2 \cdot \left( \cosh  \sqrt{\frac{2\log n}{n}} \right)^n = 
\frac{2}{n^{1+\epsilon}}+O\left( \frac{(\log n)^2}{n^{2+\epsilon}}\right)
=\frac{2}{n^{1+\epsilon}}+O(n^{-2}).
\]
Thus 
\[
\sum_{n=1}^\infty \int_0^1 f_n(t) dt  = \sum_{n=1}^\infty\left( \frac{2}{n^{1+\epsilon}}+O(n^{-2}) \right)
<\infty.
\]
Because each $f_n$ is nonnegative, using this with the \textbf{monotone convergence theorem} gives the claim.
\end{proof}


\begin{theorem}
For almost all $t \in [0,1]$,
\[
\limsup_{n \to \infty} \frac{\left| \sum_{k=1}^n r_k(t) \right|}{\sqrt{n \log n}} \leq \sqrt{2}.
\]
\end{theorem}
\begin{proof}
Let $\epsilon>0$.
By Theorem \ref{seriesaa}, for almost all $t \in [0,1]$ there is some $n_t$ such that $n \geq n_t$ implies that $f_n(t)<1$, where we are talking about the functions
$f_n$ defined in the proof of that theorem; certainly the terms of a convergent series are eventually less than $1$. 
That is, for almost all $t \in [0,1]$ there is some $n_t$ such that $n \geq n_t$ implies that (taking logarithms),
\[
(-2-\epsilon) \log n +  \sqrt{\frac{2\log n}{n}} \left| \sum_{k=1}^n r_k(t) \right|<0,
\]
and rearranging,
\[
\frac{\left| \sum_{k=1}^n r_k(t) \right|}{\sqrt{n \log n}} < \sqrt{2}+\frac{\epsilon}{\sqrt{2}}=\sqrt{2}+\epsilon'.
\]

For each $s \in \mathbb{N}$, let $E_s$ be those $t \in [0,1]$ such that
\[
\limsup_{n \to \infty} \frac{\left| \sum_{k=1}^n r_k(t) \right|}{\sqrt{n \log n}} > \sqrt{2}+\frac{1}{s}.
\]
For each $s$, taking $0<\epsilon'<\frac{1}{s}$ we get that almost all $t \in [0,1]$ do not belong to $E_s$. That is, for each
$s$, the set $E_s$ has measure $0$. Therefore
\[
E = \bigcup_{s=1}^\infty E_s
\]
has measure $0$. 
That is, for almost all $t \in [0,1]$, for all  $s \in \mathbb{N}$ we have
$t \not \in E_s$, i.e.
\[
\limsup_{n \to \infty} \frac{\left| \sum_{k=1}^n r_k(t) \right|}{\sqrt{n \log n}} \leq \sqrt{2}+\frac{1}{s},
\]
and this holding for all $s \in \mathbb{N}$ yields
\[
\limsup_{n \to \infty} \frac{\left| \sum_{k=1}^n r_k(t) \right|}{\sqrt{n \log n}} \leq \sqrt{2},
\]
completing the proof.
\end{proof}


\section{Hypercubes}
Let $m_n$ be Lebesgue measure on $\mathbb{R}^n$, and let $Q_n=[0,1]^n$.\footnote{Masayoshi Hata, {\em Problems and Solutions in Real
Analysis}, p.~161, Solution 11.1.}

\begin{theorem}
If $f \in C([0,1])$, then
\[
\lim_{n \to \infty} \int_{Q_n} f\left(\frac{x_1+\cdots+x_n}{n}\right) dm_n(x) = f\left(\frac{1}{2}\right).
\]
\end{theorem}
\begin{proof}
Define $X_n:Q_n \to \mathbb{R}$ by 
\[
X_n = \frac{x_1+\cdots+x_n}{n}, \qquad x \in Q_n.
\]
We have
\[
\int_{Q_n} X_n dm_n(x)=\frac{1}{n} \sum_{k=1}^n \int_0^1 x_k \cdot 1 dx_k
=\frac{1}{n} \sum_{k=1}^n \frac{1}{2}
=\frac{1}{2},
\]
and we define
\begin{eqnarray*}
V_n&=&\int_{Q_n} \left(X_n-\frac{1}{2} \right)^2 dm_n(x)\\
&=&\int_{Q_n} \sum_{k=1}^n \frac{x_k^2}{n^2} + \sum_{j \neq k} \frac{x_j x_k}{n^2} - X_n + \frac{1}{4} dm_n(x)\\
&=&\frac{1}{n^2} \sum_{k=1}^n \int_0^1 x_k^2 dx_k +  \frac{1}{n^2} \sum_{j=1}^n \sum_{k \neq j} \int_0^1 x_j dx_j  \int_0^1 x_k dx_k
-\frac{1}{2}+\frac{1}{4}\\
&=&\frac{1}{n^2} \sum_{k=1}^n \frac{1}{3} + \frac{1}{n^2} \sum_{j=1}^n \sum_{k \neq j} \frac{1}{4} -\frac{1}{4}\\
&=&\frac{1}{3n}+\frac{n-1}{4n} - \frac{1}{4}\\
&=&\frac{n^{-1}}{12}.
\end{eqnarray*}

Suppose that $c_n$ is a sequence of positive real numbers tending to $0$, and define $J_n=J_n(c)$ to be those $x \in Q_n$ such that
\[
\left|X_n(x)-\frac{1}{2} \right| \geq c_n.
\]
Then
\begin{eqnarray*}
V_n &=& \int_{Q_n} \left(X_n-\frac{1}{2} \right)^2 dm_n(x)\\
&\geq& \int_{J_n} \left(X_n-\frac{1}{2} \right)^2 dm_n(x)\\
&\geq& \int_{J_n} c_n^2 dm_n(x)\\
&=&c_n^2 m_n(J_n),
\end{eqnarray*}
so
\[
m_n(J_n) \leq \frac{V_n}{c_n^2} = \frac{n^{-1}}{12c_n^2}. 
\]
Take $c_n=n^{-1/3}$, giving
\[
m_n(J_n) \leq \frac{n^{-1/3}}{12}.
\]
Let $\epsilon>0$.
Because $f$ is continuous, there is some $\delta>0$ such that $|t-\frac{1}{2}|<\delta$ implies that
$|f(t)-f(\frac{1}{2})|<\epsilon$; furthermore, we take $\delta$ such that
\[
\frac{\norm{f}_\infty \delta}{6}<\epsilon.
\]
Set $N>\delta^{-3}$. For $n \geq N$ and $x \in Q_n \setminus J_n$,
\[
\left|X_n(x)-\frac{1}{2}\right| < c_n=n^{-1/3} \leq N^{-1/3} < \delta,
\]
and so
\[
\left|(f(X_n(x))-f\left(\frac{1}{2}\right)\right|<\epsilon.
\]
This gives us
\begin{eqnarray*}
\left| \int_{Q_n} f(X_n(x))  dm_n(x)- f\left(\frac{1}{2}\right) \right|&=&
\left|   \int_{Q_n} f(X_n(x)) - f\left(\frac{1}{2}\right) dm_n(x) \right|\\
&\leq& \int_{J_n} \left|f(X_n(x)) - f\left(\frac{1}{2}\right)\right| dm_n(x)\\
&&+\int_{Q_n \setminus J_n}  \left|f(X_n(x)) - f\left(\frac{1}{2}\right)\right| dm_n(x)\\
&\leq& \int_{J_n}2\norm{f}_\infty dm_n(x)
+\int_{Q_n \setminus J_n} \epsilon dm_n(x)\\
&\leq&2\norm{f}_\infty m_n(J_n)+\epsilon\\
&\leq&2\norm{f}_\infty  \frac{n^{-1/3}}{12} + \epsilon\\
&<&\frac{\norm{f}_\infty \delta}{6}+\epsilon\\
&<&2\epsilon,
\end{eqnarray*}
which proves the claim.
\end{proof}




\end{document}