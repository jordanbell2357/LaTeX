\documentclass{article}
\usepackage{amsmath,amssymb,graphicx,subfig,mathrsfs,amsthm}
\newcommand{\norm}[1]{\left\Vert #1 \right\Vert}
\newtheorem{theorem}{Theorem}
\newtheorem{lemma}[theorem]{Lemma}
\newtheorem{corollary}[theorem]{Corollary}
\begin{document}
\title{Liouville's theorem and Gibbs measures}
\author{Jordan Bell\\ \texttt{jordan.bell@gmail.com}\\Department of Mathematics, University of Toronto}
\date{\today}

\maketitle


Let $M$ be a symplectic manifold with symplectic form $\omega$. Define $\omega^\sharp:TM \to T^*M$
by
\[
\omega^\sharp(X)Y=\omega(X,Y), \qquad Y \in C^\infty(M,TM),
\]
in other words,
\[
(\omega^\sharp(X))_x v=\omega_x(X_x,v),\qquad x\in M, v \in T_x M.
\]
$\omega^\sharp:TM \to T^*M$ is a vector bundle isomorphism.

Let $H \in C^\infty(M,\mathbb{R})$. We define
\[
X_H=(\omega^\sharp)^{-1}(dH),
\]
i.e.,
\[
X_H(x)=(\omega^\sharp)^{-1}(dH(x)), \qquad x \in M.
\]
Thus, $X_H$ is the unique element of $C^\infty(M,TM)$ such that
\[
\omega(X_H,Y)=dH(Y), \qquad Y \in C^\infty(M,TM).
\]
We call $X_H \in C^\infty (M,TM)$ the Hamiltonian vector field of $H$, or the symplectic gradient $\nabla_\omega H$ of $H$.

If $\omega=\sum_{i=1}^n dq^i \wedge dp_i$, define $X \in C^\infty(M,TM)$ by
\[
X=\sum_{i=1}^n \frac{\partial H}{dp_i} \frac{\partial}{\partial q^i } -\frac{\partial H}{dq^i} \frac{\partial}{\partial p_i}.
\]
We have
\[
i_X dq^i=\frac{\partial H}{\partial dp_i}
\]
and
\[
i_X dp_i=-\frac{\partial H}{dq^i},
\]
hence, as $i_X(\alpha \wedge \beta)=(i_X \alpha) \wedge \beta + (-1)^k \alpha \wedge (i_X \beta)$ for $\alpha \in \Omega^k$ \cite[p.~115, Theorem 2.4.13]{foundations},
\begin{eqnarray*}
i_X \omega&=&\sum_{i=1}^n i_X ( dq^i \wedge dp_i)\\
&=&\sum_{i=1}^n (i_X dq^i) \wedge dp_i - dq^i \wedge (i_X dp_i)\\
&=&\sum_{i=1}^n \frac{\partial H}{\partial p_i} dp_i +\frac{\partial H}{\partial q^i} dq^i\\
&=&dH.
\end{eqnarray*}
Hence $X=X_H$.

For a vector field $X$, the Lie derivative $L_X \omega$ of $\omega$ is defined by, 
\[
L_X \omega= (F_t^*)^{-1} \frac{d}{dt} F_t^* \omega,
\]
which one checks is independent of $t$, where $F_t^* \omega$ is the pull-back of $\omega$ by $F_t$.

Let $F_t$ be the flow of $X_H$, for $t \in I$ where $I$ is some open interval with $0 \in I$. For $t \in I$, we have by \cite[p.~115, Theorem 2.3.13]{foundations},
\begin{eqnarray*}
\frac{d}{dt} \left( F_t^* \omega \right)&=&F_t^*\left( L_{X_H} \omega\right)\\
&=&F_t^* \left( i_X d\omega + d(i_{X_H} \omega) \right)\\
&=&F_t^*(i_X 0 + ddH)\\
&=&F_t^*(0+0)\\
&=&0.
\end{eqnarray*}
Thus, for $t \in I$ we have $F_t^* \omega=F_0^* \omega=\omega$. So for each $t \in I$,  the map $F_t:M \to M$ is a symplectomorphism.

Let
\[
\mu=\frac{\omega^n}{n!}=\frac{1}{n!} \underbrace{\omega \wedge \cdots \wedge \omega}_{n}.
\]
$\mu$ is equal to the degree $2n$ term in
\[
\exp(\omega).
\]

We have, as $F_t^*$ is a homomorphism of differential algebras \cite[p.~113, Theorem 2.4.9]{foundations} and as $F_t$ is a symplectomorphism,
\begin{eqnarray*}
F_t^* \mu&=&\frac{1}{n!} (F_t^* \omega) \wedge \cdots (F_t^* \omega)\\
&=&\frac{1}{n!} \omega \wedge \cdots \wedge \omega\\
&=&\mu.
\end{eqnarray*}


If $\omega=\sum_{i=1}^n dq^i \wedge dp_i$ then
\[
\omega^n=(-1)^{\frac{n(n-1)}{2}}n! dq^1 \wedge \cdots dq^n \wedge dp_1 \wedge \cdots \wedge dp_n;
\]
the sign comes up getting all the $q^i$'s together; since we have to reorder both the $q^i$'s and the $p_i$'s the signs we get from doing those cancel.


If $f \in C^\infty(\mathbb{R},\mathbb{R})$, then, as $H \circ F_t = H$ for all $t \in I$,
\begin{eqnarray*}
F_t^* ((f\circ H) \mu)&=&(f\circ H \circ F_t) F_t^* \mu\\
&=&(f \circ H) \mu.
\end{eqnarray*}
Let $\mu_\beta=e^{-\beta H}\mu$. We call $\mu_\beta \in \Omega^{2n}(M)$ a Gibbs measure on $M$.

One can motivate the choice of $e^{-\beta H}$ as a function  by which to multiply $\mu$ (rather than any other function invariant under the Hamiltonian flow $F_t$) through 
equivariant cohomology. See \cite[pp.~197--198]{MR1853077}. Let $z$ be a formal variable. An equivariant differential form (for the Hamiltonian flow of $H$) is a finite sum
$\alpha=\sum_n \alpha_n z^n$, where $\alpha_n$ is a differential form on $M$ such that $L_{X_H} \alpha_n=0$. We define the equivariant differential $D$ (for the Hamiltonian flow of $H$)
by
\[
D \alpha = d\alpha-zi_{X_H} \alpha=\sum_n d(\alpha_n)z^n-z\sum_n i_X (\alpha_n) z^n.
\]
But
\begin{eqnarray*}
D^2 \alpha&=&d^2 \alpha-z di_{X_H}\alpha-zi_{X_H} d\alpha   + z^2 i_{X_H} i_{X_H} \alpha\\
&=&-z\sum_n \left(d(i_{X_H} \alpha_n) +i_{X_H} (d\alpha_n) \right)z^n+z^2 \sum_n i_{X_H} i_{X_H} \alpha_n z^n\\
&=&-z\sum_n L_{X_H} \alpha_n z^n+0\\
&=&0.
\end{eqnarray*}
Thus $D^2=0$. If $L_{X_H} \alpha=0$, then $L_{X_H} (D\alpha)=0$, while the differential of a regular differential form that is invariant
under a Hamiltonian flow is not necessarily itself invariant under the Hamiltonian flow. 

$D\omega=d\omega-zi_{X_H} \omega=-(dH) z$. As $i_{X_H} f=0$ for a function $f$, we have
$D(\omega+zH)=0$; thus while $\omega$ is closed under the usual differential $d$,
$\omega+zH$ is closed under the equivariant differential $D$. The degree $2n$ term of $\exp(\omega+zH)$ is
\[
e^{zH} \frac{\omega^n}{n!}=e^{zH} \mu.
\]
Taking $z=-\beta$ gives us the Gibbs measure $\mu_\beta$.

\bibliographystyle{amsplain}
\bibliography{equivariant}

\end{document}