\documentclass{article}
\usepackage{amsmath,amssymb,graphicx,subfig,mathrsfs,amsthm}
%\usepackage{tikz-cd}
\usepackage{hyperref}
\newcommand{\innerL}[2]{\langle #1, #2 \rangle_{L^2}}
\newcommand{\inner}[2]{\langle #1, #2 \rangle}
\def\Re{\ensuremath{\mathrm{Re}}\,}
\def\Im{\ensuremath{\mathrm{Im}}\,}
\newcommand{\HSnorm}[1]{\Vert #1 \Vert_{\ensuremath\mathrm{HS}}}
\newcommand{\HSinner}[2]{\left\langle #1, #2 \right\rangle_{\ensuremath\mathrm{HS}}}
\newcommand{\tr}{\textrm{tr}} 
\newcommand{\supp}{\mathrm{supp}\,} 
\newcommand{\Span}{\textrm{span}} 
\newcommand{\id}{\textrm{id}} 
\newcommand{\Hom}{\textrm{Hom}}
\newcommand{\HS}{B_{\ensuremath\mathrm{HS}}} 
\newcommand{\norm}[1]{\Vert #1 \Vert}
\renewcommand{\div}{\mathrm{div}}
\newtheorem{theorem}{Theorem}
\newtheorem{lemma}[theorem]{Lemma}
\newtheorem{proposition}[theorem]{Proposition}
\newtheorem{corollary}[theorem]{Corollary}
\newtheorem{definition}[theorem]{Definition}
\begin{document}
\title{The infinite-dimensional torus}
\author{Jordan Bell\\ \texttt{jordan.bell@gmail.com}\\Department of Mathematics, University of Toronto}
\date{\today}
\maketitle

\section{Locally compact abelian groups}
Let $\mathbb{N}$ denote the positive integers.

If $G_i$, $i \in I$, are compact abelian groups, we define their \textbf{direct product} to be the cartesian product
\[
\prod_{i \in I} G_i
\]
with the coarsest topology such that the projection maps $\pi_i:\prod_{j \in I} G_j \to G_i$ are continuous (namely the product topology), with which 
the direct product is a compact abelian group. We write
\[
G^\omega = \prod_{\mathbb{N}} G.
\]
We shall be interested especially in the compact abelian group $\mathbb{T}=S^1$, and we call $\mathbb{T}^\omega$ the
\textbf{infinite-dimensional torus}.


If $\Gamma_i$, $i \in I$, are discrete abelian groups, their 
 \textbf{direct sum}, denoted by
 \[
\bigoplus_{i \in I} \Gamma_i,
\]
 consists of those elements $x$ of the cartesian product $\prod_{i \in I} \Gamma_i$  such that 
 the set $\{i \in I: \pi_i(x) \neq 0\}$ is finite. Let $p_i:\bigoplus_{j \in I} \Gamma_j \to \Gamma_i$ be the restriction of $\pi_i$ to $\bigoplus_{j \in I} \Gamma_j$.
   We give the direct sum the finest topology such that the inclusion maps
$q_i: \Gamma_i \to \bigoplus_{j \in I} \Gamma_j$, defined by
\[
(p_j \circ q_i)(x)=\begin{cases}
x&j=i\\
0&j \neq i
\end{cases}, \qquad x\in \Gamma_i,
\]
 are continuous. With this topology, the direct sum is a discrete abelian group. We write 
 \[
 \Gamma^\infty = \bigoplus_{\mathbb{N}} \Gamma.
 \]
 We shall be interested especially in the discrete abelian group $\mathbb{Z}$, and in the infinite direct sum $\mathbb{Z}^\infty$.
 (I don't know how significant an object it is, but I mention that the abelian group $\prod_{\mathbb{N}} \mathbb{Z}$ is
 called the Baer�Specker group.)
 
When speaking about $0$ or $1$ in a locally compact abelian group, it is unambiguous that this symbol denotes the identity element of the group,
because  there is only one distinguished element in a locally compact abelian group.
Often we denote the identity element of a compact abelian group by $1$
and the identity element of a discrete abelian group by $0$.

If $G_1,\ldots,G_n$ are locally compact abelian groups, it is straightforward to check that the cartesian product
\[
\prod_{k=1}^n G_k
\]
with the product topology is a locally compact abelian group. We call this both the direct product and the direct sum
and write
\[
G_1 \oplus \cdots \oplus G_n = \bigoplus_{k=1}^n G_k = \prod_{k=1}^n G_k = G_1 \times \cdots \times G_n.
\]


\section{Dual groups}
If $G$ is a locally compact abelian group, denote by $\widehat{G}$ its \textbf{dual group}, that is, the set of continuous group homomorphisms $G \to S^1$.
For $g \in G$ and $\phi \in \widehat{G}$we write
\[
\inner{x}{\phi} = \phi(x).
\]
$\widehat{G}$ has the initial topology induced by $\{\phi \mapsto \inner{x}{\phi}: x \in G\}$, with which it is a locally compact abelian group. If $G$ is compact then
$\widehat{G}$ is discrete, and if $G$ is discrete then $\widehat{G}$ is compact.


\begin{theorem}
Suppose that $G_1,\ldots,G_n$ are locally compact abelian groups. Then the dual
group of  $G_1 \oplus \cdots \oplus G_n$ is
isomorphic as a topological group to  $\widehat{G}_1 \oplus \cdots \oplus \widehat{G}_n$.
\end{theorem}

We prove in the following theorem that for discrete abelian groups, the dual group of a direct sum is the direct product of the dual groups.\footnote{Karl H. Hofmann and Sidney A. Morris, {\em The Structure of Compact Groups}, second ed., p.~12, Proposition 1.17. Cf. Walter Rudin, {\em Fourier Analysis on Groups}, p.~37, \S 2.2.3.}
In particular, this shows that the dual group of $\mathbb{Z}^\infty$ is $\mathbb{T}^\omega$. Then by the \textbf{Pontryagin duality theorem}\footnote{Walter
Rudin, {\em Fourier Analysis on Groups}, p.~28, Theorem 1.7.2.}
we get that the dual group of $\mathbb{T}^\omega$ is $\mathbb{Z}^\infty$.

\begin{theorem}
Suppose that $\Gamma_i$, $i \in I$, are discrete abelian groups
and let
\[
 \Gamma = \bigoplus_{i \in I} \Gamma_i, \qquad G= \prod_{i \in I} \widehat{\Gamma}_i.
\]
Then $\Phi:G \to \widehat{\Gamma}$, defined by
\[
(\Phi g)(\gamma) = \prod_{i \in I} \inner{p_i(\gamma)}{\pi_i(g)}, \qquad g \in G, \gamma \in \Gamma,
\]
is an isomorphism of topological groups. Here, $\pi_i:G \to \widehat{\Gamma}_i$   and $p_i:\Gamma  \to \Gamma_i$  
are the projection maps.
\end{theorem}
\begin{proof}
The definition of $(\Phi g)(\gamma)$ makes sense because $\{i \in I: p_i(\gamma) \neq 0\}$ is finite and hence
$\{i \in I:  \inner{p_i(\gamma)}{\pi_i(g)} \neq 1\}$ is finite. For $g,h \in G$ and $\gamma \in \Gamma$,
\begin{eqnarray*}
(\Phi (gh))(\gamma)& =& \prod_{i \in I}\inner{p_i(\gamma)}{\pi_i(gh)}\\
& =& \prod_{i \in I}\inner{p_i(\gamma)}{\pi_i(g)}\inner{p_i(\gamma)}{\pi_i(h)}\\
& =& (\Phi g)(\gamma) (\Phi h)(\gamma)\\
& =& ((\Phi g)(\Phi h))(\gamma),
\end{eqnarray*}
showing that $\Phi(gh)=\Phi(g)\Phi(h)$ and 
hence that $\Phi$ is a homomorphism. Suppose that $g \in \ker \Phi$. 
For each $i \in I$ and each $\gamma \in \Gamma_i$,
\[
((\Phi g) \circ q_i )(\gamma) = (\Phi g)(q_i(\gamma))=1,
\]
where $q_i:\Gamma_i \to \\Gamma$ is the inclusion map.
 This is true for all $\gamma \in \Gamma_i$, so $(\Phi g) \circ q_i$ is the identity element of $\widehat{\Gamma}_i$. And this is true for all
$i \in I$, so $\Phi g$ is the identity element of $G$. Therefore $\Phi$ is one-to-one. Suppose that
$\alpha \in \widehat{\Gamma}$.  Define $g \in G$ as follows: for each $i \in I$, take $\pi_i(g) = \alpha \circ q_i \in \widehat{\Gamma}_i$. Then $g$
satisfies  
$\Phi g = \alpha$, hence $\Phi$ is onto and is therefore a group isomorphism.

A continuous bijection from a compact topological space to a Hausdorff space is a homeomorphism, so to prove
that $\Phi$ is a homeomorphism it suffices to prove that $\Phi$ is continuous. $\widehat{\Gamma}$ has the initial topology induced by
$\{\alpha \mapsto  \inner{\gamma}{\alpha}: \gamma \in \Gamma\}$, which are  maps $\widehat{\Gamma} \to S^1$, so by the \textbf{universal property} of the initial topology, to prove that
$\Phi$ is continuous it suffices to prove  that for each $\gamma \in \Gamma$,
\[
g \mapsto \inner{\gamma}{\Phi g}
\]
 is continuous
$G \to S^1$. For $\gamma \in \Gamma$, let $J_\gamma=\{i \in I: p_i(\gamma) \neq 0\}$, which is a finite set. 
For each $i \in J_\gamma$,  it is straightforward to check that the map $g \mapsto \inner{p_i( \gamma)}{\pi_i(g)}$  is continuous $G \to S^1$.
Hence the map
\[
g \mapsto (\Phi g)(\gamma) = \prod_{i \in J_\gamma} \inner{p_i(\gamma)}{\pi_i(g)} 
\]
is continuous $G \to S^1$, being a product of finitely many continuous functions $G \to S^1$, and this completes the proof.
\end{proof}

Let $G$ be a locally compact abelian group.
If $\Gamma_0$ is a finite subset of $\widehat{G}$ and $a_\gamma \in \mathbb{C}$ for each $\gamma \in \Gamma_0$,
we call the function $G \to \mathbb{C}$ defined by
\[
x \mapsto \sum_{\gamma \in \Gamma_0} a_\gamma \inner{x}{\gamma}
\]
a \textbf{trigonometric polynomial} on $G$.
Suppose that $G$ is a compact abelian group.  Its dual group $\widehat{G}$ separates points in $G$; this is not immediate and is proved using the inversion
theorem for the Fourier transform.\footnote{Walter Rudin, {\em Fourier Analysis on Groups}, p.~24, \S 1.5.2.}
The set of trigonometric polynomials on $G$ is a self-adjoint algebra that contains the constant functions,
so the Stone-Weierstrass theorem then tells us that it is dense in the Banach algebra $C(G)$. 
Because $\mathbb{C}$ is separable, it follows that if $\widehat{G}$ is countable then $C(G)$ is separable. 
In particular, any closed subgroup $G$ of $\mathbb{T}^\omega$ is a compact abelian group whose dual group one checks to be countable, so
$C(G)$ is separable.

A compact Hausdorff space $X$ is metrizable if and only if the Banach algebra $C(X)$ is separable.\footnote{Charalambos D. Aliprantis and Kim C. Border,
{\em Infinite Dimensional Analysis: A Hitchhiker's Guide}, third ed., p.~353, Theorem 9.14.} 
We established in the previous paragraph that if $G$ is a compact abelian group with countable dual group then the trigonometric polynomials are dense in the Banach
algebra $C(G)$. Therefore, every compact abelian group with countable dual group is metrizable. In particular, $\mathbb{T}^\omega$ and all its closed subgroups
are metrizable. In fact, it is proved in Rudin that for a compact abelian group, (i) being metrizable, (ii) having a countable dual group, and (iii) being isomorphic as a topological group
to a closed subgroup of $\mathbb{T}^\omega$ are equivalent.\footnote{Walter Rudin, {\em Fourier Analysis on Groups},
p.~38, \S 2.2.6.}



\section{$\mathbb{T}^\omega$ and $\mathbb{Z}^\infty$}
Let $\pi_n:\mathbb{T}^\omega \to S^1$ and
$p_n:\mathbb{Z}^\infty \to \mathbb{Z}$ be the projection maps and let $q_n:\mathbb{Z} \to \mathbb{Z}^\infty$ be the inclusion map.

For $x \in \mathbb{T}^\omega$
and $\gamma \in \mathbb{Z}^\infty$,
\[
\inner{x}{\gamma} = \prod_{n \in \mathbb{N}} \inner{\pi_n(x)}{p_n(\gamma)} = \prod_{n \in \mathbb{N}} \pi_n(x)^{p_n(\gamma)},
\]
where for each $n$, $\pi_n(x) \in S^1$ and $p_n(\gamma) \in \mathbb{Z}$. 




Let $m$ be the Haar measure on $\mathbb{T}^\omega$ such that $m(\mathbb{T}^\omega)=1$. 
Because the dual group of $\mathbb{T}^\omega$ is $\mathbb{Z}^\infty$, for any $f \in L^1(m)$ the Fourier transform of $f$ is
the function $\hat{f} \in C_0(\mathbb{Z}^\infty)$ defined by
\[
\hat{f}(\gamma) = \int_{\mathbb{T}^\omega} f(x) \inner{-x}{\gamma} dm(x)
=\int_{\mathbb{T}^\omega} f(x) \prod_{n \in \mathbb{N}} \pi_n(\gamma)^{-p_n(x)} dm(x),\qquad \gamma \in \mathbb{Z}^\infty.
\]






\section{Kronecker sets}
Suppose that $G$ is a locally compact abelian group and that $E$ is a subset of $G$, which we give the subspace
topology.
$E$ is called a \textbf{Kronecker set} if
for every continuous $f:E \to S^1$ and every $\epsilon>0$, there is some $\gamma \in \widehat{G}$ such that
\[
\sup_{x \in E} |f(x)-\inner{x}{\gamma}|<\epsilon.
\] 

We first prove the following lemma from Rudin.\footnote{Walter Rudin, {\em Fourier Analysis on Groups}, p.~104, Lemma 5.2.8.}

\begin{lemma}
If $0<\alpha<\beta<1$, then
the set of polynomials with integer coefficients and $0$ constant term is dense in the real Banach algebra $C([\alpha,\beta])$ of continuous functions $[\alpha,\beta] \to 
\mathbb{R}$.
\label{integerpoly}
\end{lemma}
\begin{proof}
Let $R$ be the closure in $C([\alpha,\beta])$ of the set of polynomials with integer coefficients and $0$ constant term. Because $x \in R$, $R$ separates points
in $[\alpha,\beta]$ and  for every $a \in [\alpha,\beta]$ there is some $f \in R$ such that $f(a) \neq 0$. It is straightforward
to check that $R$ is closed under addition and multiplication. If we show that $\mathbb{R} \subset R$, it will follow that $R$ is an algebra over $\mathbb{R}$,
and then by the Stone-Weierstrass theorem we will get that $R$ is dense in $C([\alpha,\beta])$, and hence equal to $C([\alpha,\beta])$ as $R$ is closed.

Let $c \in \mathbb{R}$, let $p$ be prime, and define
\[
S_p(x) = \frac{1-x^p-(1-x)^p}{p}, \qquad x \in [\alpha,\beta].
\]
Using that $p$ is prime, by the binomial theorem it follows that $S_p$ is a polynomial with integer coefficients and $0$ constant term.
Partitioning $\mathbb{R}$ into intervals of length $p$, $c$ lies in one of these intervals and hence there is some integer $q_p$ such that
$\left|c-\frac{q_p}{p}\right|<\frac{1}{p}$. 
For $x \in [\alpha,\beta]$,
\begin{eqnarray*}
|q_p S_p(x)-c| &\leq& \left|c-\frac{q_p}{p}\right| +   \frac{|q_p|}{p} (\beta^p+(1-\alpha)^p)\\
& <& \frac{1}{p}+
\left(|c|+\frac{1}{p} \right)(\beta^p+(1-\alpha)^p).
\end{eqnarray*}
Hence $\norm{q_pS_p - c}_\infty \to 0$ as $p \to \infty$. $q_p$ is an integer so for each $p$, $q_pS_p$ is a polynomial with integer coefficients and $0$ constant term, so this shows
that $c \in R$, completing the proof.
\end{proof}

An \textbf{arc} in a topological space is a homeomorphic image of a  compact subset of $\mathbb{R}$ of nonzero length. The following theorem shows that
there is an arc in $\mathbb{T}^\omega$ that is a Kronecker set.\footnote{Walter Rudin, {\em Fourier Analysis on Groups}, p.~103, Theorem 5.2.7.}

\begin{theorem}
$\mathbb{T}^\omega$ contains an arc that is a Kronecker set.
\end{theorem}
\begin{proof}
Let $0<\alpha<\beta<1$, define $x:[\alpha,\beta] \to \mathbb{T}^\omega$ by
\[
(\pi_n \circ x)(t) =\exp\left(2\pi i t^n \right), \qquad t \in [\alpha,\beta], \quad n \in \mathbb{N},
\]
and let $L$ be the image of $[\alpha,\beta]$ under $x$. Assign $L$ the subspace topology inherited from $\mathbb{T}^\omega$,
and suppose that $f:L \to S^1$ is continuous.
One proves that there is a continuous function $h:[\alpha,\beta] \to \mathbb{R}$  that satisfies
\[
(f \circ x)(t) = \exp(2\pi i h(t)), \qquad \alpha \leq t \leq \beta.
\]
Let $\epsilon>0$, and by Lemma \ref{integerpoly}, let $S_m(x)=\sum_{j=1}^m a_j x^j$ be a polynomial with integer coefficients such that $\norm{S_m-h}_\infty < \epsilon$.
Define $\gamma \in \mathbb{Z}^\infty$ by $p_j(\gamma) = a_j$ for $1 \leq j \leq m$ and $p_j(\gamma)=0$ otherwise. For $t \in [\alpha,\beta]$,
\begin{eqnarray*}
|f(x(t))-\inner{x(t)}{\gamma}| &=& \left| \exp(2\pi i h(t)) - \prod_{n \in \mathbb{N}} \inner{\pi_n(x(t))}{p_n(\gamma)} \right|\\
&=& \left| \exp(2\pi i h(t)) - \prod_{n=1}^m  \inner{\pi_n(x(t))}{a_n}  \right|\\
&=& \left| \exp(2\pi i h(t)) - \prod_{n=1}^m \exp(2\pi ia_n t^n) \right|\\
&=&\left| \exp(2\pi i h(t)) - \exp\left( \sum_{n=1}^m 2\pi ia_n t^n \right) \right| \\
&\leq&\left|2\pi h(t) -  \sum_{n=1}^m  2\pi a_n t^n \right|\\
&=&2\pi |h(t)-S_m(t)|\\
&<&2\pi \epsilon,
\end{eqnarray*}
using the fact that $|\exp(i A)-\exp(i B)| \leq |A-B|$ for $A,B \in \mathbb{R}$. Hence, for every $\epsilon>0$ there is some $\gamma \in \mathbb{Z}^\infty$ such
that
\[
\sup_{y \in L} |f(y)-\inner{y}{\gamma}|<\epsilon,
\]
showing that $L$ is a Kronecker set.
\end{proof}




\section{Subgroups}
Suppose that $G$ is a locally compact abelian group. For each $x \in G$, let $t_x:G \to G$ be defined by $t_x(y)=x+y$, which is a homeomorphism,
and let $\sigma:G \to G$ be defined by $\sigma(x)=-x$, which is also a homeomorphism.
If $A$ is an open set in $G$ and $B$ is a subset of $G$, 
then
\[
A+B = \bigcup_{x \in B} t_x(A),
\]
which is open because $t_x(A)$ is open for each $x \in B$. 
Furthermore, if $A$ and $B$ are both compact sets in $G$ then $A \times B$ is compact in $G \times G$ and $A+B$ is the image of $A \times B$
under the continuous map $(x,y) \mapsto x+y$ hence is compact. 

By a \textbf{neighborhood} of a point $x$ in a topological space we mean a set such that $x$ lies in the interior of the set, in other words, a set that contains an open neighborhood of the point.
The collection of all neighborhoods of a point $x$ is a filter, and a \textbf{neighborhood base at $x$} is a filter base for the neighborhood filter of $x$.
In a locally compact Hausdorff space, every point $x$ has a neighborhood base  consisting of 
compact neighborhoods of $x$. 

Let $A:G \times A \to G$ be $A(x,y)=x+y$, which is continuous. 
If $W$ is a neighborhood of $0$ in $G$, then $A^{-1}(W)$ is a neighborhood of $(0,0)$ in $G \times G$. A base for the product
topology on $G \times G$ consists of sets of the form $U_1 \times U_2$ where $U_1,U_2$ are open sets in $G$, so there are open sets $U_1,U_2$ in $G$ such that
$(0,0) \in U_1 \times U_2 \subset A^{-1}(W)$. Each of $U_1$ and $U_2$ are then open neighborhoods of $0$ in $G$, so $V=U_1 \cap U_2$ is also an open neighborhood
of $0$ in $G$, and then $V \times V$ is open in $G \times G$ and
\[
(0,0) \in V \times V \subset U_1 \times U_2 \subset A^{-1}(W).
\]
Hence $A(0,0) \subset A(V \times V) \subset W$, i.e. $0 \in V+V \subset W$, and $V+V$ is open because $V$ is open. Therefore, for every neighborhood $W$ of $0$ in a locally
compact abelian group, there is some $V$ that is an open neigborhood of $0$ and that satisfies $V+V \subset W$. 

Suppose that $G$ is a locally compact abelian group.
A subset $E$ of $G$ is called \textbf{symmetric} if $E=-E$. 
If $N$ is a compact neighborhood of $0$ then $N$ contains an open neighborhood $U$ of $0$.
The set $U \cap \sigma(U)$ is an open neighborhood of
$0$ and the set $N \cap \sigma(N)$ is compact (an intersection of compact sets in a Hausdorff space is compact) and contains
$U \cap \sigma(U)$, hence $N \cap \sigma(N)$ is a compact symmetric neighborhood of $0$ that is contained in $N$. It follows that in a locally compact abelian group, there is a neighborhood base at $0$  consisting of compact symmetric neighborhoods of $0$.

Suppose that $G$ is an abelian group and that $H$ is a  subgroup of $G$. We define the \textbf{quotient group}
 $G/H$ be the collection of cosets of $H$,
which is an abelian group where we define
\[
(x+H)+(y+H) = (x+y) + H, \qquad x,y \in G.
\]
Let $\pi:G \to G/H$ be the projection map, which is a homomorphism with $\ker \pi = H$.

We are now equipped to define quotient groups in the category of locally compact abelian groups. 
Suppose that $G$ is a locally compact abelian group and that $H$ is a closed subgroup of $G$. 
We assign $G /H$ the final topology induced by the
projection map $\pi$ (namely, the quotient topology).  For  $x+ H \in G/H$, there is a compact neighborhood $N$ of $x$ in $G$; that is,
there is a compact set $N$ and an open set $U$ such that $x \in U \subset N$. Because $\pi$ is continuous, $\pi(N)$ is compact,
and because $\pi$ is open, $\pi(U)$ is open, so $\pi(N)$ is a compact neighborhood of $x+H$ in $G/H$. Therefore $G/H$ is locally compact.
It remains to prove that $G/H$ is Hausdorff and that addition and negation are continuous to prove that $G/H$ is  a locally compact
abelian group. Suppose that $x+H,y+H$ are distinct elements of $G/H$, i.e. $x-y \not \in H$. The set $y+H = t_y(H)$ is closed because $H$ is closed,
and $x \not \in y+H$ so $G \setminus  t_y(H)$ is an open neighborhood of $x$, and hence $W=t_{-x}(G \setminus t_y(H))$ is an open neighborhood of $0$ such that
$x+W$ is disjoint from $y+H$. Because $W$ is an open neighborhood of $0$ there is an open  neighborhood $V$ of $0$ such that $V+V \subset W$. Furthermore,
there is a compact symmetric neighborhood of $0$, $N$, contained in $V$. 
If $(x+H +N) \cap (y+H+N) \neq \emptyset$ then there are $h_1,h_2 \in H$ and $n_1,n_2 \in N$ such that
$x+h_1+n_1 = y+h_2+n_2$, and then $x+(n_1-n_2) = y + (h_2-h_1)$. But $-n_2 \in N$ because $N$ is symmetric and so
$n_1-n_2 \in N + N \subset V+V \subset W$, so $x+(n_1-n_2) \in x+W$,  and $h_2-h_1 \in H$, so $y+(h_2-h_1) \in y+H$, contradicting that $x+W$ and
$y+H$ are disjoint. Therefore $x+H+N$ and $y+H+N$ are disjoint, and their images under $\pi$ are then disjoint neighborhoods of $x+H$ and $y+H$ in $G/H$, showing that
$G/H$ is Hausdorff. It is straightforward to prove that addition and negation are continuous in $G/H$, and therefore $G/H$
is a locally compact abelian group. 

If $H$ is a closed subgroup of a locally compact abelian group $G$, the \textbf{annihilator of $H$}, denoted $\Lambda_H$, is the set
of all $\gamma \in \widehat{G}$ such that
\[
\inner{x}{\gamma}=1, \qquad x \in H.
\]
 For each $x \in H$, the map $\gamma \mapsto \inner{x}{\gamma}$ is continuous
$\widehat{G} \to S^1$ so the inverse image of $\{1\}$ under this map is closed. $\Lambda_H$ is the intersection of all these inverse images hence is closed, and is a closed
subgroup because it is apparent that $\Lambda_H$ is a subgroup of $\widehat{G}$. 
It can be proved that $\Lambda_H$ is the dual of the quotient group $G/H$ and that the quotient group $\widehat{G}/\Lambda_H$ is the dual of
 $H$.\footnote{Walter Rudin, {\em Fourier Analysis on Groups}, p.~35, Theorem 2.1.2.}

The following lemma shows that we can extend continuous characters on a closed subgroup to the entire group.\footnote{Walter
Rudin, {\em Fourier Analysis on Groups}, p.~36, Theorem 2.1.4.}

\begin{lemma}
Suppose that $H$ is a closed subgroup of a locally compact abelian group $G$. If $\phi \in \widehat{H}$, then there is some $\gamma \in \widehat{G}$ whose restriction to
$H$ is equal to $\phi$.
\end{lemma}
\begin{proof}
$\phi \in \widehat{H} = \widehat{G} / \Lambda_H$, so there is some $\gamma \in \widehat{G}$ such that
for all $x \in H$, $\gamma(x)=\phi(x)$.
\end{proof}

Suppose that $G$ is a locally compact abelian group. It can be proved that if $E$ is a compact open set in $G$ and $0 \in
E$, then $E$ contains a compact open subgroup of $G$.\footnote{Walter Rudin, {\em Fourier Analysis on Groups}, p.~41,
Lemma 2.4.3.}

We are now equipped to prove the following theorem.\footnote{Walter Rudin, {\em Fourier Analysis on Groups}, p.~47, Theorem 2.5.6.}

\begin{theorem}
Suppose that $G$ is a compact group. $G$ is connected if and only if $\gamma \in \widehat{G}$ having finite order implies that $\gamma=0$.
\end{theorem}
\begin{proof}
Assume that $G$ is not connected. Then there is a clopen subset $A$ that is neither $G$ nor $\emptyset$. Because $G$ is compact,
both $A$ and $G \setminus A$ are compact and open, and one of them, call it $E$, contains $0$. 
Because $E$ is a compact open set containing $0$, $E$ contains a compact open subgroup $H$ of $G$, and $H \neq G$ because $E \neq G$.
Because $H$ is open, the singleton $\{0+H\}$ in the quotient group $G/H$ is an open set, and therefore $G/H$ is discrete. But
$G$ is compact and $G/H$ is the image of $G$ under the projection map, so $G/H$ is compact. Hence $G/H$ is finite. 
The dual of $G/H$ is $\Lambda_H$, which is a   subgroup of $\widehat{G}$. Because $G/H$
contains more than one element (as $H \neq G$), $\Lambda_H$ contains some $\gamma \neq 0$, and $\gamma$ has finite order because it is contained in the finite
subgroup $\Lambda_H$.

Assume that $\gamma \in \widehat{G}$ has order finite order and that $\gamma \neq 0$. Every element of $\gamma(G)$ has finite order and $\gamma(G) \neq \{1\}$,
so $\gamma(G)$ is not connected. But if $G$ were connected then $\gamma(G)$, a continuous image of $G$, would be connected, hence $G$ is not connected.
\end{proof}



\begin{lemma}
Suppose that $G$ is a locally compact abelian group. If $A$ is an open subgroup of $G$, then $A$ is closed.
\end{lemma}
\begin{proof}
$A$ is a subgroup of $G$, which gives us 
\[
A = G \setminus \bigcup_{x \in G \setminus A} (x+A).
\]
Because each set $x+A$ is open, this shows that $A$ is closed.
\end{proof}


\section{Measures}
Suppose that $\mathscr{M}$ is a $\sigma$-algebra on a set $X$.
If $\mu$ is a complex measure on $\mathscr{M}$ we denote by $|\mu|$ its \textbf{total variation}, which is a finite positive
measure on $\mathscr{M}$.\footnote{Walter Rudin, {\em Real and Complex Analysis}, third ed., p.~117, Theorem 6.2 and p.~118, Theorem 6.4.} The \textbf{total variation norm} of $\mu$ is $\norm{\mu}=|\mu|(X)$. 

Suppose that $X$ is a Hausdorff space with Borel $\sigma$-algebra $\mathscr{B}_X$ and that $\mu$ is a complex Borel measure on $X$. 
We say that $\mu$ is
\textbf{outer regular} if for each $E \in \mathscr{B}_X$,
\[
|\mu|(E) = \inf\{|\mu|(V): \textrm{$E \subset V$ and $V$ is open}\}
\]
\textbf{inner regular} if for each $E \in \mathscr{B}_X$,
\[
|\mu|(E) = \sup\{|\mu|(F): \textrm{$F \subset E$ and $F$ is closed}\},
\]
and \textbf{tight} if  for each $E \in \mathscr{B}_X$,
\[
|\mu|(E) = \sup\{|\mu|(K): \textrm{$K \subset E$ and $K$ is compact}\}.
\]
(Because we demand that $X$ be Hausdorff, a compact set is closed and hence belongs to the Borel
$\sigma$-algebra of $X$; compact sets need not belong to the Borel $\sigma$-algebra of a topological space that is not Hausdorff.)
We remark that the words ``inner regular'' often means what we call tight. We say that $\mu$ is \textbf{regular} if it is both outer
regular and tight, and we also remark that calling a measure regular often means being outer regular
and what we call inner regular. What we call a regular complex Borel measure means precisely what Rudin means by these
words in  {\em Fourier Analysis on Groups}, and using Rudin's notation we define
\[
M(X) = \{\mu: \textrm{$\mu$ is a regular complex Borel measure on $X$}\}.
\]

 It is a fact that a complex Borel measure on a metrizable space
is outer regular and inner regular,\footnote{Charalambos D. Aliprantis and Kim C. Border,
{\em Infinite Dimensional Analysis: A Hitchhiker's Guide}, third ed., p.~436, Theorem 12.5.} and that a complex Borel measure on a Polish
space is regular.\footnote{Charalambos D. Aliprantis and Kim C. Border,
{\em Infinite Dimensional Analysis: A Hitchhiker's Guide}, third ed., p.~438, Theorem 12.7.} 

Suppose that $X$ and $Y$ are locally compact Hausdorff spaces and that $\mu \in M(X)$ and $\lambda \in M(Y)$. It is a fact that there
is a unique element of $M(X \times Y)$, denoted $\mu \times \lambda$, such that for any $A \in \mathscr{B}_X$ and $B \in \mathscr{B}_Y$,
\[
(\mu \times \lambda)(A \times B) = \mu(A) \lambda(B).
\]
We call $\mu \times \lambda$ the \textbf{product measure} of $\mu$ and $\lambda$. 

Suppose that $G$ is a locally compact abelian group with addition $A:G \times G \to G$. For $\mu,\lambda \in M(G)$,
we define the \textbf{convolution} of $\mu$ and $\lambda$ to be the pushforward of the product $\mu \times \lambda$ by
$A$,
\[
\mu * \lambda = A_*(\mu \times \lambda),
\]
and it can be proved that $\mu * \lambda \in M(G)$, that convolution is commutative and associative, and that
$\norm{\mu * \lambda} \leq \norm{\mu} \norm{\lambda}$.\footnote{Walter Rudin, {\em Fourier Analysis on Groups}, p.~13,
Theorem 1.3.2; Karl Stromberg, {\em A note on the convolution of regular measures}, Math. Scand. \textbf{7} (1959), 347--352.} Then, with convolution as multiplication and using the total variation norm,
$M(G)$ is a unital commutative Banach algebra, with unity $\delta_0$.

For $\mu \in M(G)$, the \textbf{Fourier transform of $\mu$}
is the function $\hat{\mu}:\widehat{G} \to \mathbb{C}$ defined by
\[
\hat{\mu}(\gamma) = \int_G \inner{-x}{\gamma} d\mu(x), \qquad \gamma \in \widehat{G}.
\]
One proves that $\hat{\mu}$ is bounded and uniformly continuous, and we define 
\[
B(\widehat{G}) = \{\hat{\mu}: \mu \in M(G)\}.
\]


\section{Idempotent measures}
If $G$ is a locally compact abelian group and $\mu \in M(G)$, we say that $\mu$ is \textbf{idempotent} if $\mu * \mu=\mu$,
and we denote the set of idempotent elements of $M(G)$ by $J(G)$. 
Because the Fourier transform of a convolution is the product of the Fourier transforms, 
for $\mu \in M(G)$ we have $\mu*\mu=\mu$ if and only if $\hat{\mu}^2 = \hat{\mu}$. But $\hat{\mu}^2 = \hat{\mu}$ is equivalent to $\hat{\mu}$
having range contained in $\{0,1\}$, so for $\mu \in M(G)$, we have that
$\mu \in J(G)$ if and only if $\hat{\mu}$ is the characteristic function of some subset of $\widehat{G}$. 
For $\mu \in J(G)$, we write
\[
S(\mu) = \{\gamma \in \widehat{G}: \hat{\mu}(\gamma)=1\}.
\]

Suppose that $\Lambda$ is an open subgroup of $\widehat{G}$. Then $\Lambda$ is closed,  and
the fact that $\Lambda$ is open implies that the singleton containing the identity in $\widehat{G}/\Lambda$ is open and hence that
$\widehat{G}/\Lambda$ is a discrete abelian group. 
Denoting the annihilator of $\Lambda$ by $H$, which is a closed subgroup of $G$, 
the quotient group $\widehat{G}/\Lambda$ is the dual group of $H$ and hence $H$ is compact. 
Let $m_H$ be the Haar measure on $H$ such that $m_H(H)=1$. Taking $m_H(E) = m_H(E \cap H)$, $m_H \in M(G)$.
 If $\gamma \in \Lambda$ then
\[
\hat{m}_H(\gamma) = \int_G \inner{-x}{\gamma} dm_H(x) = \int_H \inner{-x}{\gamma} dm_H(x) = \int_H dm_H(x) = m_H(H)=1.
\]
If $\gamma \in \widehat{G} \setminus \Lambda$ then there is some $x_0 \in H$ such that $\inner{x_0}{\gamma} \neq 1$,
and then
\[
\int_H \inner{-x}{\gamma} dm_H(x) = \inner{x_0}{\gamma} \int_H \inner{-x_0-x}{\gamma} dm_H(x) = \inner{x_0}{\gamma} \int_H \inner{-x}{\gamma} dm_H(x),
\]
showing that $\hat{m}_H(\gamma) = \inner{x_0}{\gamma} \hat{m}_H(\gamma)$, and because $\inner{x_0}{\gamma} \neq 1$ this implies that $\hat{m}_H(\gamma)=0$.
Therefore, $\Lambda = S(m_H)$. 

If $E=\gamma_0+\Lambda$, then with 
\[
d\mu(x)=\inner{x}{\gamma_0}dm_(H)
\]
we have $\mu \in J(G)$ and $E=S(\mu)$. 


\section{Sidon sets}
Let $G$ be a compact abelian group and let $E \subset \widehat{G}$. A function $f \in L^1(G)$ is called an \textbf{$E$-function} if
$\gamma \in \widehat{G} \setminus E$ implies that $\hat{f}(\gamma)=0$. An \textbf{$E$-polynomial} is a trigonometric polynomial
$f$ on $G$ that is an $E$-function.

We call a subset $E$ of $\widehat{G}$ a \textbf{Sidon set} if there is some $B_E \geq 0$ such that
for every $E$-polynomial $f$ on $G$,
\[
\sum_{\gamma \in E} |\hat{f}(\gamma)| \leq B_E \norm{f}_\infty. 
\]

We shall use the following lemma later.\footnote{Walter
Rudin, {\em Fourier Analysis on Groups}, p.~121, Theorem 5.7.3.}

\begin{lemma}
Suppose that $\Gamma$ is a discrete abelian group that is the dual group of a compact abelian group $G$. If $E \subset \Gamma$ is a Sidon set with constant
$B_E$, then  every bounded $E$-function $f$ on $G$ satisfies
\[
\sum_{\gamma \in E} |\hat{f}(\gamma)| \leq B_E \norm{f}_\infty.
\]
\label{sidon}
\end{lemma}


\section{Dirichlet series}
Define $\sigma:\mathbb{Z}^\infty \to \mathbb{Z}$ by $\sigma(\gamma) = \sum_{n \in \mathbb{N}} p_n(\gamma)$, i.e. the sum of the entries of $\gamma$, which makes
sense because any element of $\mathbb{Z}^\infty$ has only finitely many nonzero entries. 

Let $Y$ be those $\gamma\in \mathbb{Z}^\infty$ such that $p_n(\gamma) \geq 0$ for all $n \in \mathbb{N}$,
and let $E = Y \cap \sigma^{-1}(1)$. 
 In other words, the elements of $E$ are those $\gamma \in \mathbb{Z}^\infty$ one coordinate of which is $1$ and all other coordinates of which are $0$.
The proof of the following theorem is from Rudin.\footnote{Walter Rudin, {\em Fourier Analysis on Groups}, p.~224, Theorem 8.7.9.}

\begin{theorem}
If $f \in L^\infty(\mathbb{T}^\omega)$ and $\hat{f}(\gamma)=0$ for all $\gamma \in X \setminus Y$, then 
\[
\sum_{\gamma \in E} |\hat{f}(\gamma)| \leq \norm{f}_\infty.
\]
\label{infinitySidon}
\end{theorem}
\begin{proof}
$\sigma:\mathbb{Z}^\infty \to \mathbb{Z}$ is a continuous group homomorphism, and $\ker \sigma$ is an open subgroup of
$\mathbb{Z}^\infty$, because $\mathbb{Z}^\infty$ is discrete. Because $\sigma^{-1}(1)$ is a coset of this open subgroup,
there is some $\mu \in J(\mathbb{T}^\omega)$ such that $\hat{\mu}$ is the characteristic function of $\sigma^{-1}(1)$, and this $\mu$ satisfies
$\norm{\mu}=1$. Define $g:\mathbb{T}^\omega \to \mathbb{C}$ by 
\[
g(x)=(f*\mu)(x) = \int_{\mathbb{T}^\omega} f(x-y) d\mu(y), \qquad x \in \mathbb{T}^\omega,
\]
whose Fourier transform is $\hat{g}(\gamma)=\hat{f}(\gamma) \hat{\mu}(\gamma)$. 
If $\gamma \not \in E$ then $\gamma \not \in Y$ or $\gamma \not \in \sigma^{-1}(1)$. In the first case
$\hat{f}(\gamma)=0$ and in the second case $\hat{\mu}(\gamma) =0$, and hence $\gamma \not \in E$ implies that
$\hat{g}(\gamma)=0$, namely, $g$ is an $E$-function. Also, it is apparent from the definition of $g$ that $\norm{g}_\infty 
\leq \norm{f}_\infty$. 

Suppose that $P$ is an $E$-polynomial. Hence there is a finite subset $E_0$ of $E$ such that
$\gamma \not \in E_0$ implies that $\hat{P}(\gamma)=0$, and thus there are $c_\gamma \in \mathbb{C}$, $\gamma \in E_0$, such that 
\[
P(x) = \sum_{\gamma \in E_0} c_\gamma \inner{x}{\gamma}
=\sum_{\gamma \in E_0} c_\gamma \prod_{n \in \mathbb{N}} \inner{\pi_n(x)}{p_n(\gamma)}, \qquad
x \in \mathbb{T}^\omega.
\]
$E_0 \subset E$, so any element of $E_0$ has one entry $1$, say $p_{n_\gamma}(\gamma)=1$, and all other entries $0$, so
\[
P(x) = \sum_{\gamma \in E_0} c_\gamma \pi_{n_\gamma}(x).
\]
Define $x \in \mathbb{T}^\omega$ by taking $c_\gamma \cdot  \pi_{n_\gamma}(x)=|c_\gamma|$ for each $\gamma \in E_0$, and all
other entries of $x$ to be $1 \in S^1$; this makes sense because if $\gamma_1,\gamma_2 \in E_0$ and $n_{\gamma_1}= n_{\gamma_2}$ then
$\gamma_1=\gamma_2$. For this $x$, $P(x)=\sum_{\gamma \in E_0} |c_\gamma|$. But  it is apparent that
$\norm{P}_\infty \leq \sum_{\gamma \in E_0} |c_\gamma|$, so
\[
\norm{P}_\infty = \sum_{\gamma \in E_0} |c_\gamma|.
\]
This shows that $E$ is a Sidon set with $B_E=1$. Therefore by Lemma \ref{sidon}, because $g$ is a bounded $E$-function on $\mathbb{T}^\omega$ we
get
$\sum_{\gamma \in E} |\hat{g}(\gamma)| \leq \norm{g}_\infty$.
But $\hat{\mu}$ is the characteristic function of $\sigma^{-1}(1)$ and $E=Y \cap \sigma^{-1}(1)$, so
\[
\sum_{\gamma \in E} \hat{f}(\gamma) 
= \sum_{\gamma \in E} \hat{f}(\gamma) \hat{\mu}(\gamma)
= \sum_{\gamma \in E} \hat{g}(\gamma) \leq \norm{g}_\infty \leq \norm{f}_\infty,
\]
proving the claim.
\end{proof}


Following Rudin, we use the above theorem to prove a theorem about Dirichlet series
due to Bohr.\footnote{Walter Rudin, {\em Fourier Analysis on Groups}, pp.~224--225. See also 
Maxime Bailleul and Pascal Lef\`evre,
{\em Some Banach spaces of Dirichlet series},
\url{arxiv.org/abs/1311.3845}} 

\begin{theorem}[Bohr]
If
\[
\phi(s) = \sum_{k=1}^\infty \frac{c_k}{k^s}
\]
and $|\phi(s)| \leq 1$ for all $s$ such that $\Re s>0$, then
\[
\sum_p |c_p| \leq 1.
\]
\end{theorem}
\begin{proof}
For $k \in \mathbb{N}$, let $\gamma(k) \in Y$ such that $k=\prod_{n=1}^\infty p_n^{h_n (\gamma(k))}$, where $p_n$ are the primes and
where $h_n:\mathbb{Z}^\infty \to \mathbb{Z}$ are the projection maps; so far we have denoted these projection maps by $p_n$, rather than using $h_n$,
but the symbol $p_n$ has such a strong association with the primes that we change notation here. The map $k \mapsto \gamma(k)$ is a bijection
$\mathbb{N} \to Y$, and we write $c_\gamma=c_k$. We shall use the fact that the image of the primes under this bijection is $E$.

Let $s$ be a complex number in the half-plane of  convergence of $\phi$ and write $z_n(s)=p_n^{-s}=\exp(-s \log p_n)$. Then,
\begin{eqnarray*}
\phi(s) &=&\sum_{k =1}^\infty c_k k^{-s} \\
&=&\sum_{\gamma \in Y} c_\gamma \left(\prod_{n=1}^\infty p_n^{h_n (\gamma)} \right)^{-s}\\
&=&\sum_{\gamma \in Y} c_\gamma \prod_{n=1}^\infty p_n^{-s h_n(\gamma)}\\
&=&\sum_{\gamma \in Y} c_\gamma \prod_{n=1}^\infty z_n(s)^{h_n(\gamma)}
\end{eqnarray*}
Defining
 $T:\mathbb{R} \to \mathbb{T}^\omega$ by
\[
(\pi_n \circ T)(\sigma)=\exp(-i \sigma \log p_n), \qquad n \in \mathbb{N}, \sigma \in \mathbb{R}, 
\]
we have, as $z_n(i\sigma)=\exp(-i\sigma \log p_n)$,
\[
\phi(i\sigma) 
=\sum_{\gamma \in Y} c_\gamma \prod_{n=1}^\infty \inner{\pi_n(T(\sigma))}{h_n(\gamma)}
=\sum_{\gamma \in Y} c_\gamma \inner{T(\sigma)}{\gamma}.
\]
One checks that the function $f:\mathbb{T}^\omega \to \mathbb{C}$ defined by $f(x) = \sum_{\gamma \in Y} c_\gamma \inner{x}{\gamma}$
satisfies the conditions of
 Theorem \ref{infinitySidon}, and thus gets
 \[
\sum_p |c_p| = \sum_{\gamma \in E} |c_\gamma| =  \sum_{\gamma \in E} |\hat{f}(\gamma)| \leq \norm{f}_\infty 
 \]
 I do not see why $\norm{f}_\infty \leq 1$. However, granted this, the claim follows.
\end{proof}








\section{Descriptive set theory}
If $(X,d)$ is a compact metric space, 
$C(X,X)$ is a Polish space with the \textbf{uniform metric} $(f,g) \mapsto \sup_{x \in X} d(f(x),g(x))$. 
We denote by $H(X)$ the group of homeomorphisms of $X$, which one proves is a $G_\delta$ set in $C(X,X)$.  
Because $H(X)$ is a $G_\delta$ set in a Polish space, it is a Polish space with the subspace topology. 
A homeomorphism $h$ of $X$ is said to be \textbf{minimal} if there is no proper closed subset of $X$ that is invariant under $h$, and is called
\textbf{distal} if $x \neq y$ implies that there is some $\epsilon>0$ such that for all $n \in \mathbb{N}$, $d(h^n(x),h^n(y))>\epsilon$. 
It has been proved (Beleznay-Foreman) that the collection of minimal distal homeomorphisms of $\mathbb{T}^\omega$ is a Borel $\mathbf{\Sigma}_1^1$-complete
set in $H(\mathbb{T}^\omega)$.\footnote{Alexander S. Kechris, {\em Classical Descriptive Set Theory}, p~262, Theorem 33.22.}



\section{Further reading}
Albeverio, Sergio, Daletskii, Alexei and Kondratiev, Yuri, {\em Stochastic analysis on (infinite-dimensional) product manifolds},
Stochastic dynamics (Bremen, 1997), 339--369, Springer, New York, 1999. 

Aoki, Nobuo and Totoki, Haruo, {\em Ergodic automorphisms of $T^\infty$ are Bernoulli automorphisms},
Publ. RIMS, Kyoto Univ. \textbf{10} (1975), 535--544.

Balasubramanian, R., Calado, B. and Queff\'elec, H.,
{\em The Bohr inequality for ordinary Dirichlet series},
Studia Math. \textbf{175} (2006), no. 3, 285--304. 

Bendikov, A. and Saloff-Coste, L., {\em On the sample paths of diagonal Brownian motions on the infinite dimensional torus},
Ann. Inst. H. Poincar\'e Probab. Statist. \textbf{40} (2004), no. 2, 227--254. 

Bendikov, A., {\em L\'evy measures for symmetric stable semigroups on torus $\mathbb{T}^\infty$}. Functional analysis, 173--181, Narosa, New Delhi, 1998. 

Bendikov, A. and Saloff-Coste, L., {\em Spaces of smooth functions and distributions on infinite-dimensional compact groups}, 
J. Funct. Anal. \textbf{218} (2005), no. 1, 168--218. 

Bendikov, A. D., {\em Symmetric stable semigroups on the infinite-dimensional torus},
Exposition. Math. \textbf{13} (1995), no. 1, 39--79. 

Berg, Christian, {\em On Brownian and Poissonian convolution semigroups on the infinite dimensional torus},
Invent. Math. \textbf{38} (1976/77), no. 3, 227--235.

Berg, Christian, {\em Potential theory on the infinite dimensional torus}, Invent. Math. \textbf{32} (1976), no. 1, 49--100. 

Biroli, Marco and Maheux, Patrick, {\em Logarithmic Sobolev inequalities and Nash-type inequalities for sub-markovian symmetric semigroups},
\url{https://hal.inria.fr/hal-00465177v1/document}

Fournier, John, {\em Extensions of a Fourier multiplier theorem of Paley}, Pacific J. Math. \textbf{30} (1969), 415--431. 

Haezendonck, J., 
{\em Les groupes commutatifs de Lebesgue-Rohlin},
Acad. Roy. Belg. Bull. Cl. Sci. (5) \textbf{58} (1972), 344--353. 

Hastings, H. M., {\em On expansive homeomorphisms of the in�nite torus}, The structure of
attractors in dynamical systems (Proc. Conf., North Dakota State Univ.)

Hedenmalm, H{\aa}kan and Saksman, Eero, {\em Carleson's convergence theorem for Dirichlet series},
Pacific Journal of Mathematics \textbf{208} (2003), no.~1, 85--109.

Helson, Henry, {\em Hankel forms and sums of random variables},  Studia Math. \textbf{176} (2006), no. 1, 85--92. 

Holley, R. and Stroock, D., {\em Diffusions on an infinite-dimensional torus}, J. Funct. Anal. \textbf{42} (1981), no. 1, 29--63. 

Jessen, B., {\em The theory of integration in a space of an infinite number of dimensions}, Acta Math. \textbf{63} (1934), no. 1,
249--323.

Kholshchevnikova, N. N., {\em Uniqueness for trigonometric series with respect to an increasing number of variables},
Proc. Steklov Inst. Math. 2005, Function Theory, suppl. 2, S160--S166. 

Lind, D. A., {\em Ergodic automorphisms of the infinite torus are Bernoulli},
Israel J. Math. \textbf{17} (1974), 162--168. 

Olsen, Jan-Fredrik, {\em Local properties of Hilbert spaces of Dirichlet series},
J. Funct. Anal. \textbf{261} (2011), no. 9, 2669--2696. 

Saksman, Eero and Seip, Kristian, {\em Integral means and boundary limits of Dirichlet series},
Bull. Lond. Math. Soc. \textbf{41} (2009), no. 3, 411--422. 

Saloff-Coste, Laurent, {\em Probability on groups: random walks and invariant diffusions}, 
Notices Amer. Math. Soc. \textbf{48} (2001), no. 9, 968--977. 

Skorikov, A. V., {\em Bessel potentials in spaces with a mixed norm on the group $\mathbb{T}^\infty$},
Moscow Univ. Math. Bull. \textbf{48} (1993), no. 6, 1--4 

Stackelberg, Olaf P., {\em Metric theorems related to the Kronecker-Weyl theorem mod $m$}, 
Monatsh. Math. \textbf{82} (1976), no. 1, 57--69. 

Sugita, Hiroshi and Takanobu, Satoshi,
{\em A limit theorem for Weyl transformation in infinite-dimensional torus and central limit theorem for correlated multiple Wiener integrals},
J. Math. Sci. Univ. Tokyo \textbf{7} (2000), no. 1, 99--146. 

Tanaka, Jun-ichi, {\em Dirichlet series induced by the Riemann zeta-function}, Studia Math. \textbf{187} (2008), no. 2, 157--184. 

Taylor, Thomas J. S., {\em Applications of harmonic analysis on the infinite-dimensional torus to the theory of the Feynman integral},
Functional integration with emphasis on the Feynman integral (Sherbrooke, PQ, 1986),
Rend. Circ. Mat. Palermo (2) Suppl. No. 17 (1987), 349--362 (1988). 

\end{document}