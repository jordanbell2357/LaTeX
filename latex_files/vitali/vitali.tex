\documentclass{article}
\usepackage{amsmath,amssymb,mathrsfs,amsthm}
%\usepackage{tikz-cd}
%\usepackage{hyperref}
\newcommand{\inner}[2]{\left\langle #1, #2 \right\rangle}
\newcommand{\tr}{\ensuremath\mathrm{tr}\,} 
\newcommand{\Span}{\ensuremath\mathrm{span}} 
\def\Re{\ensuremath{\mathrm{Re}}\,}
\def\Im{\ensuremath{\mathrm{Im}}\,}
\newcommand{\id}{\ensuremath\mathrm{id}} 
\newcommand{\var}{\ensuremath\mathrm{var}} 
\newcommand{\Lip}{\ensuremath\mathrm{Lip}} 
\newcommand{\GL}{\ensuremath\mathrm{GL}} 
\newcommand{\diam}{\ensuremath\mathrm{diam}} 
\newcommand{\sgn}{\ensuremath\mathrm{sgn}\,} 
\newcommand{\lcm}{\ensuremath\mathrm{lcm}} 
\newcommand{\supp}{\ensuremath\mathrm{supp}\,}
\newcommand{\dom}{\ensuremath\mathrm{dom}\,}
\newcommand{\upto}{\nearrow}
\newcommand{\downto}{\searrow}
\newcommand{\norm}[1]{\left\Vert #1 \right\Vert}
\newtheorem{theorem}{Theorem}
\newtheorem{lemma}[theorem]{Lemma}
\newtheorem{proposition}[theorem]{Proposition}
\newtheorem{corollary}[theorem]{Corollary}
\theoremstyle{definition}
\newtheorem{definition}[theorem]{Definition}
\newtheorem{example}[theorem]{Example}
\begin{document}
\title{Vitali coverings on the real line}
\author{Jordan Bell\\ \texttt{jordan.bell@gmail.com}\\Department of Mathematics, University of Toronto}
\date{\today}

\maketitle

For $x \in \mathbb{R}$ and $r>0$ write
\[
B(x,r) = \{y \in \mathbb{R}: |y-x|<r\}.
\]

Let
$\lambda$ be Lebesgue measure on the Borel $\sigma$-algebra of $\mathbb{R}$ and
let $\lambda^*$ be Lebesgue outer measure on $\mathbb{R}$. 

A \textbf{Vitali covering} of a set $E \subset \mathbb{R}$ is a collection $\mathcal{V}$ 
of  closed intervals such that for $\epsilon>0$ and for $x \in E$ there is some $I \in \mathcal{V}$ 
with $x \in I$ and
$0<\lambda(I)<\epsilon$.

The following is the \textbf{Vitali covering theorem}.\footnote{Klaus Bichteler, {\em Integration -- A Functional Approach},
p.~161, Lemma 10.5;
John J. Benedetto and Wojciech Czaja, {\em Integration and Modern Analysis}, p.~179, Theorem 4.3.1;
Russell A. Gordon, {\em The Integrals of Lebesgue, Denjoy, Perron, and Henstock}, p.~52, Lemma 4.6.}



\begin{theorem}[Vitali covering theorem]
Let $U$ be an open set in $\mathbb{R}$ with $\lambda(U)<\infty$, let $E \subset U$, and let $\mathcal{V}$ be a Vitali covering of $E$
each interval of which is contained in $U$. Then for any $\epsilon>0$, there are disjoint $I_1,\ldots,I_n \in \mathcal{V}$ such that
\[
\lambda^*\left(E \setminus \bigcup_{j=1}^n I_j \right)<\epsilon.
\]
\end{theorem}
\begin{proof}
Suppose that $I_1,\ldots,I_n \in \mathcal{V}$ are pairwise disjoint. If $E \subset \bigcup_{j=1}^n I_j$ then $I_1,\ldots,I_n$ satisfy the claim, and 
otherwise,
let 
\[
U_n = U \setminus \bigcup_{j=1}^n I_j,
\]
and there exists some $x \in E \cap U_n$. As $x \in U_n$ and $U_n$ is open, there is some $\eta>0$ such that $B(x,\eta) \subset U_n$ and then
as $\mathcal{V}$ is a Vitali covering of $E$ there is some $I \in \mathcal{V}$ with $x \in I \subset B(x,\eta) \subset U_n$. 
Thus $\delta_n>0$ for
\[
\delta_n = \sup\left\{ \lambda(I): I \in \mathcal{V}, I \subset U_n\right\},
\]
and there is some $I_{n+1} \in \mathcal{V}$ with $I_{n+1} \subset U_n$ and  $\lambda(I_{n+1}) > \frac{\delta_n}{2}$.

For $j \geq 1$ write  $I_j=[x_j-r_j,x_j+r_j]$ and let $J_j = [x_j-5r_j,x_j+5r_j]$, namely
$J_j$ is concentric with $I_j$ and $\lambda(J_j) = 5 \lambda(I_j)$. 
Then, as the intervals $I_1,I_2,\ldots$ are pairwise disjoint
Borel sets each contained in $U$,
\[
 \sum_{j=1}^\infty \lambda(J_j)
= 5 \sum_{j=1}^\infty \lambda(I_j)
= 5 \lambda\left( \bigcup_{j=1}^\infty I_j\right)
\leq 5 \lambda(U)
<\infty
\]
and it follows from $\sum_{j=1}^\infty \lambda(J_j)<\infty$  that $\sum_{j=M}^\infty \lambda(J_j) \to 0$ as $M \to \infty$,
which with
\[
\lambda\left(\bigcup_{j=M}^\infty J_j \right) \leq \sum_{j=M}^\infty \lambda(J_j)
\]
yields $\lambda\left( \bigcup_{j=M}^\infty J_j\right) \to 0$ as $M \to \infty$. 

Let $M \geq 1$. 
If $x \in E \setminus \bigcup_{j=1}^\infty I_j$ then $x \in E \setminus \bigcup_{j=1}^M I_j$ and so
$x \in U_M$, and as $U_M$ is open there is some $\eta>0$ with $B(x,\eta) \subset U_M$. But
$x \in E$ and $\mathcal{V}$ is a Vitali covering of $E$, so there is some $I \in \mathcal{V}$ with
$x \in I$ and $I \subset B(x,\eta) \subset U_M$. Now,
$\lambda(I_{j+1}) > \frac{\delta_j}{2}$ and $\sum_{j=1}^\infty \lambda(I_j)<\infty$ together imply 
 $\delta_n \to 0$ as $n \to \infty$, so there is some $n$ for which $\delta_n<\lambda(I)$. By the definition of
$\delta_n$ as a supremum, this means that $I \not \subset U_n$ and so it makes sense to define $N$ to be a minimal
positive integer such that $I \not \subset U_N$. $M<N$: if $M \geq N$ then
$I \subset U_M \subset U_N$, contradicting $I \not \subset U_N$. (We shall merely use that $M \leq N$.)
The fact that $I \not \subset U_N$ and $I \subset U_{N-1}$ means that $I \cap I_N \neq \emptyset$ and also,
by the definition of $\delta_{N-1}$, $\lambda(I) \leq \delta_{N-1} < 2 \lambda(I_N)$. 
Write $I=[y-r,y+r]$. $I \cap I_N \neq \emptyset$ tells us $y-r \leq x_N+r_N$ and $y+r \geq x_N-r_N$, and 
$\lambda(I) < 2\lambda(I_N)$ tells us
$2r < 4r_N$, hence
\[
y+r \leq x_N+r_N+2r \leq x_N + 5r_N,\qquad y-r \geq x_N-r_N-2r \geq x_N-5r_N,
\]
showing that
\[
x \in I=[y-r,y+r] \subset J_N \subset \bigcup_{j=M}^\infty J_j.
\] 
This is true for each $x \in E \setminus \bigcup_{j=1}^\infty I_j$, which means that
\[
E \setminus \bigcup_{j=1}^\infty I_j \subset \bigcup_{j=M}^\infty J_j.
\]
Because $\lambda( \bigcup_{j=M}^\infty J_j) \to 0$ as $M \to \infty$, this yields 
\[
\lambda^*\left(E \setminus \bigcup_{j=1}^\infty I_j\right) = 0.
\]
But $E \setminus \bigcup_{j=1}^n I_j$ is an increasing sequence of sets tending to
$E \setminus \bigcup_{j=1}^\infty I_j$, therefore
\[
\lambda^*\left(E \setminus \bigcup_{j=1}^n I_j\right) \to \lambda^*\left(E \setminus \bigcup_{j=1}^\infty I_j\right) = 0,
\qquad  n \to \infty,
\]
so there is some $n$ such that $\lambda^*\left(E \setminus \bigcup_{j=1}^n I_j\right) < \epsilon$ and then
$I_1,\ldots,I_n$ satisfy the claim.
\end{proof}

\end{document}