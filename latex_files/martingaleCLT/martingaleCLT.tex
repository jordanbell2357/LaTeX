\documentclass{article}
\usepackage{amsmath,amssymb,mathrsfs,amsthm}
%\usepackage{tikz-cd}
\usepackage{hyperref}
\newcommand{\inner}[2]{\left\langle #1, #2 \right\rangle}
\newcommand{\tr}{\ensuremath\mathrm{tr}\,} 
\newcommand{\Span}{\ensuremath\mathrm{span}} 
\def\Re{\ensuremath{\mathrm{Re}}\,}
\def\Im{\ensuremath{\mathrm{Im}}\,}
\newcommand{\id}{\ensuremath\mathrm{id}} 
\newcommand{\var}{\ensuremath\mathrm{var}} 
\newcommand{\Lip}{\ensuremath\mathrm{Lip}} 
\newcommand{\Sh}{\ensuremath\mathrm{Sh}} 
\newcommand{\GL}{\ensuremath\mathrm{GL}} 
\newcommand{\diam}{\ensuremath\mathrm{diam}} 
\newcommand{\sgn}{\ensuremath\mathrm{sgn}} 
\newcommand{\lcm}{\ensuremath\mathrm{lcm}} 
\newcommand{\supp}{\ensuremath\mathrm{supp}\,}
\newcommand{\dom}{\ensuremath\mathrm{dom}\,}
\newcommand{\upto}{\nearrow}
\newcommand{\downto}{\searrow}
\newcommand{\norm}[1]{\left\Vert #1 \right\Vert}
\newtheorem{theorem}{Theorem}
\newtheorem{lemma}[theorem]{Lemma}
\newtheorem{proposition}[theorem]{Proposition}
\newtheorem{corollary}[theorem]{Corollary}
\theoremstyle{definition}
\newtheorem{definition}[theorem]{Definition}
\newtheorem{example}[theorem]{Example}
\begin{document}
\title{Martingales, L\'evy's continuity theorem, and the martingale central limit theorem}
\author{Jordan Bell\\ \texttt{jordan.bell@gmail.com}\\Department of Mathematics, University of Toronto}
\date{\today}

\maketitle


\section{Introduction}
In this note, any statement we make about filtrations and martingales is about filtrations and martingales indexed by the positive integers,
rather than the nonnegative real numbers.

We take
\[
\inf \emptyset = \infty,
\]
and for $m>n$, we take
\[
\sum_{k=m}^n = 0.
\]
(Defined rightly, these are not merely convenient  ad hoc definitions.)


\section{Conditional expectation}
Let $(\Omega,\mathscr{A},P)$ be a probability space and let
$\mathscr{B}$ be a sub-$\sigma$-algebra of $\mathscr{A}$. 
For each $f \in L^1(\Omega,\mathscr{A},P)$, there is some
$g:\Omega \to \mathbb{R}$ such that
(i) $g$ is $\mathscr{B}$-measurable and (ii) for each $B \in \mathscr{B}$,
$\int_B g dP = \int_B f dP$, and if $h:\Omega \to \mathbb{R}$ satisfies (i) and (ii) then
$h(\omega)=g(\omega)$ for almost all $\omega \in \Omega$.\footnote{Manfred Einsiedler and Thomas Ward,
{\em Ergodic Theory: with a view towards Number Theory}, p.~121, Theorem 5.1.}
We denote some $g:\Omega \to \mathbb{R}$ satisfying (i) and (ii) by $E(f|\mathscr{B})$, called
the \textbf{conditional expectation of $f$ with respect to $\mathscr{B}$}.
In other words,
$E(f|\mathscr{B})$ is the unique element of $L^1(\Omega,\mathscr{B},P)$
such that for each $B \in \mathscr{B}$,
\[
\int_B E(f|\mathscr{B}) dP = \int_B f dP.
\] 


The map $f \mapsto E(f|\mathscr{B})$ satisfies the following:
\begin{enumerate}
\item $f \mapsto E(f|\mathscr{B})$ is positive linear operator $L^1(\Omega,\mathscr{A},P)
\to L^1(\Omega,\mathscr{B},P)$ with norm $1$.
\item If $f \in L^1(\Omega,\mathscr{A},P)$ and $g \in L^\infty(\Omega,\mathscr{B},P)$, then for almost all
$\omega \in \Omega$,
\[
E(gf|\mathscr{B})(\omega) = g(\omega) E(f|\mathscr{B})(\omega).
\]
\item If $\mathscr{C}$ is a sub-$\sigma$-algebra of $\mathscr{B}$, then
for almost all $\omega \in \Omega$,
\[
E(E(f|\mathscr{B})|\mathscr{C})(\omega) = E(f|\mathscr{C})(\omega).
\]
\item If $f \in L^1(\Omega,\mathscr{B},P)$ then for almost all $\omega \in \Omega$,
\[
E(f|\mathscr{B})(\omega) = f(\omega).
\]
\item If $f \in L^1(\Omega,\mathscr{A},P)$, then
for almost all $\omega \in \Omega$,
\[
|E(f|\mathscr{B})(\omega)| \leq E(|f| |\mathscr{B})(\omega).
\]
\item If $f \in L^1(\Omega,\mathscr{A},P)$ is independent of $\mathscr{B}$, then for almost all $\omega \in \Omega$,
\[
E(f|\mathscr{B})(\omega) = E(f).
\]
\end{enumerate}




\section{Filtrations}
A \textbf{filtration} of a $\sigma$-algebra $\mathscr{A}$ is a sequence
$\mathscr{F}_n$, $n \geq 1$, of sub-$\sigma$-algebras of $\mathscr{A}$ such that
$\mathscr{F}_m \subset \mathscr{F}_n$ if $m \leq n$. 
We set $\mathscr{F}_0=\{\emptyset,\Omega\}$. 

A sequence of random variables $\xi_n:(\Omega,\mathscr{A},P) \to \mathbb{R}$ is said to be \textbf{adapted to the filtration $\mathscr{F}_n$}
if for each $n$, $\xi_n$ is $\mathscr{F}_n$-measurable.

Let $\xi_n:(\Omega,\mathscr{A},P) \to \mathbb{R}$, $n \geq 1$, be a sequence of random variables. The
\textbf{natural filtration of $\mathscr{A}$ corresponding to $\xi_n$} is
\[
\mathscr{F}_n =\sigma(\xi_1,\ldots,\xi_n).
\]
It is apparent that $\mathscr{F}_n$ is a filtration and that the sequence $\xi_n$ is adapted to $\mathscr{F}_n$. 


\section{Martingales}
Let $\mathscr{F}_n$ be a filtration of a $\sigma$-algebra $\mathscr{A}$ and let
$\xi_n:(\Omega,\mathscr{A},P) \to \mathbb{R}$ be a sequence of random variables.
We say that \textbf{$\xi_n$ is a martingale with respect to $\mathscr{F}_n$} if
(i) the sequence $\xi_n$ is adapted to the filtration $\mathscr{F}_n$, 
(ii) for each $n$, $\xi_n \in L^1(P)$, and (iii) for each $n$, for almost all $\omega \in \Omega$,
\[
E(\xi_{n+1}|\mathscr{F}_n)(\omega) = \xi_n(\omega).
\]
In particular, 
\[
E(\xi_1)=E(\xi_2)=\cdots,
\]
i.e.
\[
E(\xi_m) = E(\xi_n), \qquad m \leq n.
\]

We say that \textbf{$\xi_n$ is a submartingale with respect to $\mathscr{F}_n$} if (i) and (ii) above are true, and if
for each $n$, for almost all $\omega \in \Omega$,
\[
E(\xi_{n+1}|\mathscr{F}_n)(\omega) \geq \xi_n(\omega).
\]
In particular,
\[
E(\xi_1) \leq E(\xi_2) \leq \cdots,
\]
i.e.
\[
E(\xi_m) \leq E(\xi_n), \qquad m \leq n.
\]

We say that 
\textbf{$\xi_n$ is a supermartingale with respect to $\mathscr{F}_n$} if (i) and (ii) above are true, and if
for each $n$, for almost all $\omega \in \Omega$,
\[
\xi_n(\omega) \geq E(\xi_{n+1}|\mathscr{F}_n)(\omega).
\]
In particular,
\[
E(\xi_1) \geq E(\xi_2) \geq \cdots,
\]
i.e.
\[
E(\xi_m) \geq E(\xi_n), \qquad m \leq n.
\]

If we speak about a martingale without specifying a filtration, we mean a martingale with respect to the natural filtration
corresponding to the sequence of random variables.


\section{Stopping times}
Let $\mathscr{F}_n$ be a filtration of a $\sigma$-algebra $\mathscr{A}$. A \textbf{stopping time with respect to $\mathscr{F}_n$} is a 
function $\tau:\Omega \to \{1,2,\ldots\} \cup \{\infty\}$ such that for each $n \geq 1$,
\[
\{\omega \in \Omega: \tau(\omega) = n\} \in \mathscr{F}_n.
\]

It is straightforward to check that a function $\tau:\Omega \to \{1,2,\ldots\} \cup \{\infty\}$ is a stopping time with respect to $\mathscr{F}_n$ if and only if
for each $n \geq 1$,
\[
\{\omega \in \Omega: \tau(\omega) \leq n\} \in \mathscr{F}_n.
\]

The following lemma shows that the time of first entry into a Borel subset of $\mathbb{R}$ of a sequence of random variables adapted to a filtration
is a stopping time.\footnote{Zdzis\l{}aw Brze\'zniak and Tomasz Zastawniak, {\em Basic Stochastic Processes}, p.~55, Exercise 3.9.}

\begin{lemma}
Let $\xi_n$ be a sequence of random variables adapted to a filtration $\mathscr{F}_n$ and let $B \in \mathscr{B}_{\mathbb{R}}$. Then
\[
\tau(\omega) = \inf\{n \geq 1: \xi_n(\omega) \in B\}
\]
is a stopping time with respect to $\mathscr{F}_n$. 
\label{hitting}
\end{lemma}
\begin{proof}
Let $n \geq 1$. Then 
\begin{align*}
\{\omega \in \Omega: \tau(\omega) =n\} &= \left( \bigcap_{k=1}^{n-1} \{\omega \in \Omega: \xi_k(\omega) \not \in B\}\right)
\cap \{\omega \in \Omega: \xi_n(\omega) \in B\}\\
&=\left(\bigcap_{k=1}^{n-1} A_k^c \right) \cap A_n,
\end{align*}
where
\[
A_k = \{\omega \in \Omega: \xi_k(\omega) \in B\}.
\]
Because the sequence $\xi_k$ is adapated to the filtration $\mathscr{F}_k$,
$A_k^c \in \mathscr{F}_k$ and $A_n \in \mathscr{F}_n$, and because
$\mathscr{F}_k$ is a filtration, 
the right-hand side of the above belongs to $\mathscr{F}_n$.
\end{proof}


If $\xi_n$ is a sequence of random variables adapted to a filtration $\mathscr{F}_n$ and a stopping time $\tau$ with respect to $\mathscr{F}_n$,
for $n \geq 1$ we define $\xi_{\tau \wedge n}:\Omega \to \mathbb{R}$ by
\[
\xi_{\tau \wedge n}(\omega) = \xi_{\tau(\omega) \wedge n}(\omega), \qquad \omega \in \Omega.
\]
$\xi_{\tau \wedge n}$ is called \textbf{the sequence $\xi_n$ stopped at $\tau$}.\footnote{Zdzis\l{}aw Brze\'zniak and Tomasz Zastawniak, {\em Basic Stochastic Processes},
p.~55, Exercise 3.10.}

\begin{lemma}
$\xi_{\tau \wedge n}:(\Omega,\mathscr{A},P) \to \mathbb{R}$ is a sequence of random variables adapted to the filtration
$\mathscr{F}_n$.
\end{lemma}
\begin{proof}
Let $n \geq 1$ and let $B \in \mathscr{B}_{\mathbb{R}}$. 
Because
\[
\{\omega: \xi_{\tau \wedge n}(\omega) \in B, \tau(\omega)>n\} = \{\omega: \xi_n(\omega) \in B, \tau(\omega)>n\}
\]
and
 for any $k$,
\[
\{\omega: \xi_{\tau \wedge n}(\omega) \in B, \tau(\omega)=k\} = \{\omega: \xi_k \in B, \tau(\omega) = k\},
\]
we get
\[
\{\omega: \xi_{\tau \wedge n}(\omega) \in B\} = \{\omega: \xi_n(\omega) \in B, \tau(\omega)>n\} \cup \bigcup_{k=1}^n \{\omega: \xi_k(\omega) \in B, \tau(\omega) = k\}.
\]
But 
\[
\{\xi_n \in B, \tau >n\} = \{\xi_n \in B\} \cap \{\tau > n\}
\in \mathscr{F}_n
\]
and
\[
\{\xi_k \in B, \tau=k\} = \{\xi_k \in B\} \cap \{\tau=k\} \in \mathscr{F}_k,
\]
and therefore
\[
\{\xi_{\tau \wedge n} \in B\} \in \mathscr{F}_n.
\]
In particular, $\{\xi_{\tau \wedge n} \in B\} \in \mathscr{A}$, namely, $\xi_{\tau \wedge n}$ is a random variable, and the above
shows that this sequence is adapted to the filtration $\mathscr{F}_n$.
\end{proof}

We now prove that a stopped martingale is itself a martingale with respect to the same filtration.\footnote{Zdzis\l{}aw Brze\'zniak and Tomasz Zastawniak, {\em Basic Stochastic Processes},
p.~56, Proposition 3.2.}

\begin{theorem}
Let $\mathscr{F}_n$ be a filtration of a $\sigma$-algebra $\mathscr{A}$ and let $\tau$ be a stopping time with respect to $\mathscr{F}_n$. 
\begin{enumerate}
\item If $\xi_n$ is a submartingale with respect to $\mathscr{F}_n$ then so is $\xi_{\tau \wedge n}$.
\item If $\xi_n$ is a supermartingale with respect to $\mathscr{F}_n$ then so is $\xi_{\tau \wedge n}$.
\item If $\xi_n$ is a martingale with respect to $\mathscr{F}_n$ then so is $\xi_{\tau \wedge n}$.
\end{enumerate}
\label{stopped}
\end{theorem}
\begin{proof}
For $n \geq 1$, define
\[
\alpha_n(\omega) = \begin{cases}
1&\tau(\omega) \geq n\\
0&\tau(\omega) < n;
\end{cases}
\]
we remark that
$\tau(\omega) \geq n$ if and only if $\tau(\omega)>n-1$ and
$\tau(\omega)<n$ if and only if $\tau(\omega) \leq n-1$.
For $B \in \mathscr{B}_{\mathbb{R}}$,
(i) if $0,1 \not \in B$ then
\[
\{\omega \in \Omega: \alpha_n(\omega) \in B\} = \emptyset \in \mathscr{F}_{n-1},
\]
(ii) if $0,1 \in B$ then
\[
\{\omega \in \Omega: \alpha_n(\omega) \in B\} = \Omega \in \mathscr{F}_{n-1},
\]
(iii) if $0 \in B$ and $1 \not \in B$ then
\[
\{\omega \in \Omega: \alpha_n(\omega) \in B\}  = \{\omega \in \Omega: \alpha_n(\omega)=0\}
=\{\omega \in \Omega: \tau(\omega)\leq n-1\} \in \mathscr{F}_{n-1},
\]
and (iv) if $1 \in B$ and $0 \not \in B$ then
\[
\{\omega \in \Omega: \alpha_n(\omega) \in B\}  = \{\omega \in \Omega: \alpha_n(\omega)=1\}
=\{\omega \in \Omega: \tau(\omega) >  n-1\} \in \mathscr{F}_{n-1},
\]
Therefore $\{ \alpha_n \in B\} \in \mathscr{F}_{n-1}$. 

Set $\xi_0=0$, and we check that
\[
\xi_{\tau \wedge n} = \sum_{k=1}^n \alpha_k (\xi_k - \xi_{k-1}).
\]
It is apparent from this expression that if $\xi_n$ is adapted to $\mathscr{F}_n$ then $\xi_{\tau \wedge n}$ is adapted to $\mathscr{F}_n$, and that
if each $\xi_n$ belongs to $L^1(P)$ then each $\xi_{\tau \wedge n}$ belongs to $L^1(P)$. 
As each of $\alpha_1,\ldots,\alpha_{n+1}$ is $\mathscr{F}_n$-measurable and is bounded,
\begin{equation}
E(\xi_{\tau \wedge (n+1)}|\mathscr{F}_{n}) = \sum_{k=1}^{n+1} E(\alpha_k(\xi_k-\xi_{k-1})|\mathscr{F}_n)
=\sum_{k=1}^{n+1} \alpha_k E(\xi_k-\xi_{k-1}|\mathscr{F}_n).
\label{alphaeq}
\end{equation}
Suppose that $\xi_n$ is a submartingale. By \eqref{alphaeq},
\begin{align*}
E(\xi_{\tau \wedge (n+1)}|\mathscr{F}_{n})  &= \sum_{k=1}^{n} \alpha_k (\xi_k-\xi_{k-1}) + \alpha_{n+1}E(\xi_{n+1}|\mathscr{F}_n)
-\alpha_{n+1} \xi_n\\
&\geq \xi_{\tau \wedge n} + \alpha_{n+1} \xi_n -\alpha_{n+1} \xi_n\\
&=\xi_{\tau \wedge n},
\end{align*}
which shows that $\xi_{\tau \wedge n}$ is a submartingale; the statement that
$E(\xi_{\tau \wedge (n+1)}|\mathscr{F}_{n})  \geq \xi_{\tau \wedge n}$ means that
$E(\xi_{\tau \wedge (n+1)}|\mathscr{F}_n)(\omega) \geq \xi_{\tau \wedge n}(\omega)$ for almost all
$\omega \in \Omega$.
\end{proof}


We now prove the \textbf{optional stopping theorem}.\footnote{Zdzis\l{}aw Brze\'zniak and Tomasz Zastawniak, {\em Basic Stochastic Processes},
p.~58, Theorem 3.1.}

\begin{theorem}[Optional stopping theorem]
Let $\mathscr{F}_n$ be a filtration of a $\sigma$-algebra $\mathscr{A}$, let $\xi_n$ be a martingale with respect to $\mathscr{F}_n$, and let
$\tau$ be a stopping time with respect to $\mathscr{F}_n$. Suppose that:
\begin{enumerate}
\item For almost all $\omega \in \Omega$, $\tau(\omega)<\infty$.
\item $\xi_\tau \in L^1(\Omega,\mathscr{A},P)$.
\item $E(\xi_n 1_{\{\tau>n\}}) \to 0$ as $n \to \infty$.
\end{enumerate}
Then
\[
E(\xi_\tau)=E(\xi_1).
\]
\end{theorem}
\begin{proof}
For each $n$, $\Omega = \{\tau \leq n\} \cup \{\tau>n\}$, and therefore
\[
\xi_\tau = \xi_{\tau \wedge n} + \xi_\tau 1_{\{\tau>n\}} -  \xi_n 1_{\{\tau>n\}}
=\xi_{\tau \wedge n}  + \sum_{k=n+1}^\infty \xi_k 1_{\{\tau=k\}} - \xi_n 1_{\{\tau>n\}}.
\]
Theorem \ref{stopped} tells us that $\xi_{\tau \wedge n}$ is a martingale with respect to to $\mathscr{F}_n$, and hence
\[
E(\xi_{\tau \wedge n}) = E(\xi_{\tau \wedge 1}) = E(\xi_1),
\]
so
\begin{equation}
E(\xi_\tau) = E(\xi_1) + \sum_{k=n+1}^\infty E(\xi_k 1_{\{\tau=k\}}) -  E(\xi_n 1_{\{\tau>n\}}).
\label{xitau}
\end{equation}
But as $\xi_\tau \in L^1(P)$,
\[
\int_{\Omega} (\xi_\tau)(\omega) dP(\omega) = \sum_{k=1}^\infty 
\int_{\{\tau=k\}} \xi_k(\omega) dP(\omega)
=\sum_{k=1}^\infty E(\xi_k 1_{\{\tau=k\}}),
\]
and the fact that this series converges means that $\sum_{k=n+1}^\infty E(\xi_k 1_{\{\tau=k\}}) \to 0$. With the hypothesis
$E(\xi_n 1_{\{\tau>n\}}) \to 0$, as $n \to \infty$ we have
\[
E(\xi_1) + \sum_{k=n+1}^\infty E(\xi_k 1_{\{\tau=k\}}) -  E(\xi_n 1_{\{\tau>n\}}) \to E(\xi_1).
\]
But \eqref{xitau} is true for each $n$, so we get $E(\xi_\tau)=E(\xi_1)$, proving the claim.
\end{proof}

Suppose that $\eta_n$ is a sequence of independent random variables each with the Rademacher distribution:
\[
P(\eta_n=1)=\frac{1}{2},\qquad P(\eta_n=-1)=\frac{1}{2}.
\]
Let $\xi_n=\sum_{k=1}^n \eta_k$ and let $\mathscr{F}_n=\sigma(\eta_1,\ldots,\eta_n)$.
Because
\[
\xi_{n+1}^2 = (\xi_n+\eta_{n+1})^2 = \eta_{n+1}^2 + 2 \eta_{n+1} \xi_n + \xi_n^2,
\]
we have, as $\xi_n$ is $\mathscr{F}_n$-measurable and belongs to $L^\infty(P)$ and as $\eta_{n+1}$ is independent of the $\sigma$-algebra
$\mathscr{F}_n$,
\begin{align*}
E(\xi_{n+1}^2-(n+1) | \mathscr{F}_n) &= E( \eta_{n+1}^2 + 2 \eta_{n+1} \xi_n + \xi_n^2 - (n+1)| \mathscr{F}_n)\\
&=E(\eta_{n+1}^2)+2\xi_n E(\eta_{n+1})+\xi_n^2 - (n+1)\\
&=1+0+\xi_n^2-(n+1)\\
&=\xi_n^2-n.
\end{align*}
Therefore, $\xi_n^2-n$ is a martingale with respect to $\mathscr{F}_n$. 

Let $K$ be a positive integer and let
\[
\tau = \inf\{n \geq 1: |\xi_n|=K\}.
\]
Namely, $\tau$ is the time of first entry in the Borel subset $\{-K,K\}$ of $\mathbb{R}$, hence by Lemma \ref{hitting} is a stopping time
with respect to the filtration $\mathscr{F}_n$. With some work,\footnote{Zdzis\l{}aw Brze\'zniak and Tomasz Zastawniak, {\em Basic Stochastic Processes},
p.~59, Example 3.7.}
one shows that
(i) $P(\tau>2Kn) \to 0$ as $n \to \infty$,
(ii) $E(|\xi_\tau^2-\tau|)<\infty$,
and (iii) $E((\xi_n^2-n)1_{\{\tau>n\}}) \to 0$ as $n \to \infty$. 
Then we can apply the optional stopping theorem to the martingale $\xi_n^2-n$: we get that
\[
E(\xi_\tau^2-\tau) = E(\xi_1^2-1) = E(\xi_1^2) - 1 = E(\eta_1^2) - 1 = 0.
\]
Hence
\[
E(\tau) = E(\xi_\tau^2).
\]
But $|\xi_\tau|=K$, so $\xi_\tau^2=K^2$, hence
\[
E(\tau) = E(K^2) = K^2.
\]



\section{Maximal inequalities}
We now prove \textbf{Doob's maximal inequality}.\footnote{Zdzis\l{}aw Brze\'zniak and Tomasz Zastawniak, {\em Basic Stochastic Processes},
p.~68, Proposition 4.1.}

\begin{theorem}[Doob's maximal inequality]
Suppose that $\mathscr{F}_n$ is a filtration of a $\sigma$-algebra $\mathscr{A}$, that
$\xi_n$ is a submartingale with respect to $\mathscr{F}_n$, and that
for each $n$, $\xi_n \geq 0$. Then for each $n \geq 1$ and $\lambda>0$,
\[
\lambda P\left(\max_{1 \leq k \leq n} \xi_k \geq \lambda \right) \leq 
E\left(\xi_n 1_{\{\max_{1 \leq k \leq n} \xi_k \geq \lambda\}}\right).
\]
\end{theorem}
\begin{proof}
Define $\zeta_n(\omega) = \max_{1 \leq k \leq n} \xi_k(\omega)$, which is $\mathscr{F}_n$-measurable, and define
$\tau:\Omega \to \{1,\ldots,n\}$ by
\[
\tau(\omega) = \min\{1 \leq k \leq n: \xi_k(\omega) \geq \lambda\}
\]
if there is some $1 \leq k \leq n$ for which $\xi_k(\omega) \geq \lambda$, and $\tau(\omega)=n$ otherwise.
For $1 \leq k \leq n$,
\[
\{\tau = k\} = \left( \bigcap_{j=1}^{k-1} \{\xi_k<\lambda\}\right) \cap \{\xi_k \geq \lambda\} \in \mathscr{F}_k,
\]
and for $k>n$, 
\[
\{\tau = k\} = \emptyset \in \mathscr{F}_k,
\]
showing that $\tau$ is a stopping time with respect to the filtration $\mathscr{F}_k$. 

For $k \geq 1$,
\begin{align*}
\xi_{k+1}-\xi_{\tau \wedge (k+1)}&=\sum_{j=1}^k  1_{\{\tau=j\}} (\xi_{k+1}-\xi_{\tau \wedge (k+1)})
=\sum_{j=1}^k 1_{\{\tau=j\}} (\xi_{k+1}-\xi_j),
\end{align*}
hence, because $\tau$ is a stopping time with respect to the filtration $\mathscr{F}_k$ and because $\xi_k$ is a submartingale
with respect to this filtration,
\begin{align*}
E(\xi_{k+1}-\xi_{\tau \wedge (k+1)}|\mathscr{F}_k)&=\sum_{j=1}^k 1_{\{\tau=j\}} E((\xi_{k+1}-\xi_j)  | \mathscr{F}_k)\\
&=\sum_{j=1}^k 1_{\{\tau=j\}} (E(\xi_{k+1}|\mathscr{F}_k) - \xi_j)\\
&\geq \sum_{j=1}^k 1_{\{\tau=j\}} (\xi_k -\xi_j)\\
&=\sum_{j=1}^{k-1} 1_{\{\tau=j\}} (\xi_k-\xi_j)\\
&=\xi_k - \xi_{\tau \wedge k},
\end{align*}
from which we have that the sequence $\xi_k - \xi_{\tau \wedge k}$ is a submartingale with respect to the filtration
$\mathscr{F}_k$. Therefore
\[
E(\xi_k - \xi_{\tau \wedge k}) \geq E(\xi_1 - \xi_{\tau \wedge 1}) = E(\xi_1)-E( \xi_{\tau \wedge 1})=
E(\xi_1)-E(\xi_1)=0,
\]
and so
$E(\xi_{\tau \wedge k}) \leq E(\xi_k)$.
Because $\tau \wedge n = \tau$, this yields
\[
E(\xi_\tau) \leq E(\xi_n).
\]

We have
\[
E(\xi_\tau) = E(\xi_\tau 1_{\{\zeta_n \geq \lambda\}}) + E(\xi_\tau 1_{\{\zeta_n < \lambda\}}).
\]
If $\omega \in \{\zeta_n \geq \lambda\}$ then $(\xi_\tau)(\omega) \geq \lambda$, and if
$\omega \in  \{\zeta_n < \lambda\}$ then $\tau(\omega)=n$ and so $(\xi_\tau)(\omega)=\xi_n(\omega)$. Therefore
\[
E(\xi_\tau) \geq E(\lambda \cdot 1_{\{\zeta_n \geq \lambda\}}) + E(\xi_n 1_{\{\zeta_n < \lambda\}})
=\lambda P(\zeta_n \geq \lambda)+E(\xi_n 1_{\{\zeta_n < \lambda\}}).
\]
Therefore
\[
\lambda P(\zeta_n \geq \lambda)+E(\xi_n 1_{\{\zeta_n < \lambda\}}) \leq E(\xi_n).
\]
But $\xi_n = \xi_n 1_{\zeta_n < \lambda} + \xi_n 1_{\zeta_n \geq \lambda}$, hence
\[
\lambda P(\zeta_n \geq \lambda) \leq E(\xi_n 1_{\{\zeta_n \geq \lambda\}}),
\]
which proves the claim.
\end{proof}

The following is \textbf{Doob's  $L^2$ maximal  inequality}, which we prove using Doob's maximal inequality.\footnote{Zdzis\l{}aw Brze\'zniak and Tomasz Zastawniak, {\em Basic Stochastic Processes},
p.~68, Theorem 4.1.}

\begin{theorem}[Doob's $L^2$ maximal  inequality]
Suppose that $\mathscr{F}_n$ is a filtration of a $\sigma$-algebra $\mathscr{A}$ and that $\xi_n$ is a submartingale with
respect to $\mathscr{F}_n$ such that for each $n \geq 1$, $\xi_n \geq 0$ and $\xi_n \in L^2(P)$.
Then for each $n \geq 1$,
\[
E\left( \left| \max_{1 \leq k \leq n} \xi_k \right|^2 \right) \leq 4 E(\xi_n^2).
\]
\end{theorem}
\begin{proof}
Define $\zeta_n(\omega) = \max_{1 \leq k \leq n} \xi_k(\omega)$. 
It is a fact that if $\eta \in L^2(P)$ and $\eta \geq 0$ then
\[
E(\eta^2) = 2\int_0^\infty t P(\eta \geq  t) dt.
\]
Using this, Doob's maximal inequality, Fubini's theorem, and the Cauchy-Schwarz inequality, 
\begin{align*}
E(\zeta_n^2)&=2\int_0^\infty t P(\zeta_n > t) dt\\
&\leq 2\int_0^\infty E(\xi_n 1_{\{\zeta_n \geq t\}} dt\\
&=2 \int_0^\infty \left( \int_{\{\zeta_n \geq t\}} \xi_n(\omega) dP(\omega) \right) dt\\
&=2 \int_\Omega \left(\int_0^{\zeta_n(\omega)} dt \right) \xi_n(\omega) dP(\omega)\\
&=2\int_\Omega \zeta_n(\omega) \xi_n(\omega) dP(\omega)\\
&\leq 2 (E(\zeta_n^2))^{1/2} (E(\xi_n^2))^{1/2}.
\end{align*}
If $E(\zeta_n^2)=0$ the claim is immediate. Otherwise, we divide this inequality by $(E(\zeta_n^2))^{1/2}$ and obtain
\[
(E(\zeta_n^2))^{1/2} \leq 2 (E(\xi_n^2))^{1/2},
\]
and so
\[
E(\zeta_n^2) \leq 4 E(\xi_n^2),
\]
proving the claim.
\end{proof}



\section{Upcrossings}
Suppose that $\xi_n$ is a sequence of random variables that is adapted to a filtration $\mathscr{F}_n$ and let
$a<b$ be real numbers. Define
\[
\tau_0=0,
\]
and by induction for $m \geq 1$,
\[
\sigma_m(\omega) = \inf\{k \geq \tau_{m-1}(\omega): \xi_k(\omega) \leq a\}
\]
and
\[
\tau_m(\omega)=\inf\{k \geq \sigma_m(\omega): \xi_k(\omega) \geq b\},
\]
where $\inf \emptyset = \infty$. For each $m$, $\tau_m$ and $\sigma_m$ are each 
stopping times with respect to the filtration $\mathscr{F}_k$. 
For $n \geq 0$ we define
\[
U_n[a,b](\omega) = \sup\{m \geq 0: \tau_m(\omega) \leq n\}.
\]

For $x \in \mathbb{R}$, we write
\[
x^- = \max\{0,-x\} = -\min\{0,x\}.
\]
namely, the \textbf{negative part of $x$}.

We now prove the \textbf{upcrossings inequality}.

\begin{theorem}[Upcrossings inequality]
If $\xi_n$, $n \geq 1$, is a supermartingale with respect to a filtration $\mathscr{F}_n$ and $a<b$, then for each $n \geq 1$,
\[
(b-a) E(U_n[a,b]) \leq E((\xi_n-a)^-).
\]
\end{theorem}
\begin{proof}
For $n \geq 1$ and $\omega \in \Omega$, and writing $N=U_n[a,b](\omega)$, for which $N \leq n$, we have
\[
\begin{split}
&\sum_{m=1}^n (\xi_{\tau_m \wedge n}(\omega) - \xi_{\sigma_m \wedge n}(\omega))\\
=&\sum_{m=1}^N (\xi_{\tau_m \wedge n}(\omega) - \xi_{\sigma_m \wedge n}(\omega))
+\xi_{\tau_{N+1} \wedge n}(\omega)-\xi_{\sigma_{N+1} \wedge n}(\omega)\\
&
+\sum_{m=N+2}^n (\xi_{\tau_m \wedge n}(\omega) - \xi_{\sigma_m \wedge n}(\omega))\\
=&\sum_{m=1}^N (\xi_{\tau_m}(\omega)-\xi_{\sigma_m}(\omega))
+\xi_n(\omega)-\xi_{\sigma_{N+1} \wedge n}(\omega)
+\sum_{m=N+1}^n (\xi_n(\omega)-\xi_n(\omega))\\
=&\sum_{m=1}^N (\xi_{\tau_m}(\omega)-\xi_{\sigma_m}(\omega))
+1_{\{\sigma_{N+1} \leq n\}}(\omega) (\xi_n(\omega)-\xi_{\sigma_{N+1}}(\omega))\\
\geq&\sum_{m=1}^N (b-a)+1_{\{\sigma_{N+1} \leq n\}}(\omega) (\xi_n(\omega)-\xi_{\sigma_{N+1}}(\omega)).
\end{split}
\]
Because $\xi_{\sigma_{N+1}}(\omega) \leq a$, we have
\[
(b-a)N \leq 1_{\{\sigma_{N+1} \leq n\}}(\omega) (a-\xi_n(\omega)) + 
\sum_{m=1}^n (\xi_{\tau_m \wedge n}(\omega) - \xi_{\sigma_m \wedge n}(\omega)).
\]
One proves that\footnote{I am not this ``one''. I have not sorted out why this inequality is true. In every proof of the upcrossings inequality
I have seen there are pictures and things like this are asserted to be obvious. I am not satisfied with that reasoning; one should
not have to interpret an inequality visually to prove it.}
\[
1_{\{\sigma_{N+1} \leq n\}}(\omega) (a-\xi_n(\omega))  \leq \min\{0,a-\xi_n(\omega)\} = (\xi_n(\omega)-a)^-.
\]
Thus
\[
(b-a)E(U_n[a,b]) \leq E((\xi_n-a)^-)+\sum_{m=1}^n E(\xi_{\tau_m \wedge n} - \xi_{\sigma_m \wedge n}).
\]
Using that $\xi_n$ is a supermartingale, for each $1 \leq m \leq n$,
\begin{align*}
E(\xi_{\tau_m \wedge n} - \xi_{\sigma_m \wedge n})&\leq 0.
\end{align*}
Therefore
\[
(b-a)E(U_n[a,b]) \leq E((\xi_n-a)^-).
\]
\end{proof}


\section{Doob's martingale convergence theorem}
We now use the uprossings inequality to prove \textbf{Doob's martingale convergence theorem}.\footnote{Zdzis\l{}aw Brze\'zniak and Tomasz Zastawniak, {\em Basic Stochastic Processes},
p.~71, Theorem 4.2.}

\begin{theorem}[Doob's martingale convergence theorem]
Suppose that $\xi_n$, $n \geq 1$, is a supermartingale with respect to a filtration $\mathscr{F}_n$ and that
\[
M=\sup_n E(|\xi_n|)<\infty.
\]
Then there is some $\xi \in L^1(\Omega,\mathscr{A},P)$ such that for almost all $\omega \in \Omega$,
\[
\lim_{n \to \infty} \xi_n(\omega) = \xi(\omega)
\]
and with $E(|\xi|) \leq M$. 
\end{theorem}
\begin{proof}
For any $a<b$ and $n \geq 1$, the upcrossings inequality tells us that
\[
E(U_n[a,b]) \leq \frac{E(\xi_n-a)^-)}{b-a} \leq 
\frac{E(|\xi_n-a|)}{b-a} \leq
\frac{E(|\xi_n|+|a|)}{b-a} \leq \frac{M+|a|}{b-a}.
\] 
For each $\omega \in \Omega$, the sequence $U_n[a,b](\omega) \in [0,\infty)$ is nondecreasing, so by the monotone
convergence theorem,
\[
E\left(\lim_{n \to \infty} U_n[a,b] \right) = \lim_{n \to \infty} E(U_n[a,b]) \leq \frac{M+|a|}{b-a}.
\]
This implies that
\[
P\left(\omega \in \Omega: \lim_{n \to \infty} U_n[a,b](\omega) <\infty\right)=1.
\]
Let 
\[
A= \bigcap_{a,b \in \mathbb{Q}, a<b} \left\{\omega \in \Omega:
\lim_{n \to \infty} U_n[a,b](\omega)<\infty\right\}.
\]
This is an intersection of countably many sets each with measure $1$, so $P(A)=1$.

Let
\[
B = \{\omega \in \Omega: \liminf_n \xi_n(\omega) < \limsup_n \xi_n(\omega)\}.
\]
If $\omega \in B$, then there are $a,b \in \mathbb{Q}$, $a<b$, such that
\[
\liminf_n \xi_n(\omega)<a<b<\limsup_n \xi_n(\omega).
\]
It follows from this $\lim_{n \to \infty} U_n[a,b](\omega)=\infty$. Thus $\omega \not \in A$, so
$B \cap A = \emptyset$, and because $P(A)=1$ we get $P(B)=0$. 

We define $\xi:\Omega \to \mathbb{R}$ by
\[
\xi(\omega)=\begin{cases}
\lim_{n \to \infty} \xi_n(\omega)&\omega \not \in B\\
0&\omega \in B,
\end{cases}
\]
which is Borel measurable. Furthermore, since $|\xi|=\liminf_n |\xi_n|$ almost everywhere,
by Fatou's lemma we obtain
\begin{align*}
E(|\xi|)&=E(\liminf_n |\xi_n|)\\
&\leq \liminf_n E(|\xi_n|)\\
&\leq \sup_n E(|\xi_n|)\\
&=M.
\end{align*}
\end{proof}



\section{Uniform integrability}
Let $\xi:(\Omega,\mathscr{A},P) \to \mathbb{R}$ be a random variable.
It is a fact\footnote{Zdzis\l{}aw Brze\'zniak and Tomasz Zastawniak, {\em Basic Stochastic Processes},
p.~73, Exercise 4.3.} that $\xi \in L^1$ if and only if for each $\epsilon>0$ there is some
$M$ such that
\[
\int_{\{|\xi|>M\}} |\xi| dP < \epsilon.
\]
(One's instinct might be to try to use the Cauchy-Schwarz inequality to prove this. This doesn't work.)
Thus, if $\xi_n$ is a sequence in $L^1(\Omega,\mathscr{A},P)$ then  for each $\epsilon>0$ there
are $M_n$ such that, for each $n$,
\[
\int_{\{|\xi_n|>M_n\}} |\xi_n| dP < \epsilon.
\]

A sequence of random variables $\xi_n$ is said to be \textbf{uniformly integrable}
if for each $\epsilon>0$ there is some $M$ such that, for each $n$,
\[
\int_{\{|\xi_n|>M\}} |\xi_n| dP < \epsilon.
\]
If a sequence $\xi_n$ is uniformly integrable, then there is some $M$ such that for each $n$,
\[
\int_{\{|\xi_n|>M\}} |\xi_n| dP < 1,
\]
and so
\[
E(|\xi_n|) = \int_{\{|\xi_n| \leq M\}} |\xi_n| dP + \int_{\{|\xi_n|>M\}} |\xi_n| dP
< \int_{\{|\xi_n| \leq M\}} M dP + 1
\leq M+1.
\]

The following lemma states that the conditional expectations of an integrable random variable
with respect to a filtration is a uniformly integrable martingale with respect to that filtration.\footnote{Zdzis\l{}aw Brze\'zniak and Tomasz Zastawniak, {\em Basic Stochastic Processes},
p.~75, Exercise 4.5.}

\begin{lemma}
Suppose that $\xi \in L^1(\Omega,\mathscr{A},P)$ and that $\mathscr{F}_n$ is a filtration of $\mathscr{A}$.
Then $E(\xi|\mathscr{F}_n)$ is a martingale with respect to $\mathscr{F}_n$ and is uniformly integrable.
\label{integrable}
\end{lemma}

We now prove that a uniformly integrable supermartingale converges in $L^1$.\footnote{Zdzis\l{}aw Brze\'zniak and Tomasz Zastawniak, {\em Basic Stochastic Processes},
p.~76, Theorem 4.3.}

\begin{theorem}
Suppose that $\xi_n$ is a supermartingale with respect to a filtration $\mathscr{F}_n$, and that the sequence $\xi_n$ is uniformly
integrable. Then there is some $\xi \in L^1(\Omega,\mathscr{A},P)$ such that
$\xi_n \to \xi$ in $L^1$.
\label{uniform}
\end{theorem}
\begin{proof}
Because the sequence $\xi_n$ is uniformly integrable, there is some $M$ such that for each $n \geq 1$,
\[
E(|\xi_n|) \leq M+1.
\]
Thus, because $\xi_n$ is a supermartingale, Doob's martingale convergence theorem tells us that there is some
$\xi \in L^1(\Omega,\mathscr{A},P)$ such that for almost all $\omega \in \Omega$,
\[
\lim_{n \to \infty} \xi_n(\omega) = \xi(\omega).
\]
Because $\xi_n$ is uniformly integrable and converges almost surely to $\xi$, the Vitali convergence theorem\footnote{
V. I. Bogachev, {\em Measure Theory}, volume I, p.~268, Theorem 4.5.4;
\url{http://individual.utoronto.ca/jordanbell/notes/L0.pdf},
p.~8, Theorem 9.} tells us that $\xi_n \to \xi$ in $L^1$.
\end{proof}


The above theorem shows in particular that a uniformly integrable martingale converges to some limit in $L^1$. The following theorem shows
that the terms of the sequence are equal to the conditional expectations of this limit with respect to the natural filtration.\footnote{Zdzis\l{}aw Brze\'zniak and Tomasz Zastawniak, {\em Basic Stochastic Processes},
p.~77, Theorem 4.4.}

\begin{theorem}
Suppose that a sequence of random variables $\xi_n$ is uniformly integrable and is a martingale with respect
to its natural filtration
\[
\mathscr{F}_n=\sigma(\xi_1,\ldots,\xi_n).
\]
Then there is some $\xi \in L^1(\Omega,\mathscr{A},P)$ such that $\xi_n \to \xi$ in $L^1$ and such that for each $n \geq 1$,
for almost all $\omega \in \Omega$,
\[
\xi_n(\omega) = E(\xi|\mathscr{F}_n)(\omega).
\]
\label{universal}
\end{theorem}
\begin{proof}
By Theorem \ref{uniform}, there is some $\xi \in L^1(\Omega,\mathscr{A},P)$ such that
$\xi_n \to \xi$ in $L^1$. 
The hypothesis that the sequence $\xi_n$ is a martingale with respect to $\mathscr{F}_n$ 
tells us that for 
that for $n \geq 1$ and for
any $m \geq n$,
\[
E(\xi_m|\mathscr{F}_n) = \xi_n,
\]
and so for $A \in \mathscr{F}_n$,
\[
\int_A \xi_m dP = \int_A  E(\xi_m|\mathscr{F}_n) dP = 
\int_A \xi_n dP.
\]
Thus
\begin{align*}
\left| \int_A (\xi_n - \xi) dP \right|&=\left| \int_A (\xi_m - \xi) dP \right|\\
&\leq \int_A |\xi_m-\xi| dP\\
&\leq E(|\xi_m-\xi|).
\end{align*}
But $E(|\xi_m-\xi|) \to 0$ as $m \to \infty$. Since $m$ does not appear in the left-hand side, we have
\[
\left| \int_A (\xi_n - \xi) dP \right| = 0,
\]
and thus
\[
\int_A \xi_n dP = \int_A \xi dP.
\]
But $E(f|\mathscr{F}_n)$ is the unique element of $L^1(\Omega,\mathscr{F}_n,P)$ such that for
each $A \in \mathscr{F}_n$,
\[
\int_A E(f|\mathscr{F}_n) dP = \int_A f dP,
\]
and because $\xi_n$ satisfies this, we get that $\xi_n = E(f|\mathscr{F}_n)$ in $L^1$, i.e., 
for almost all $\omega \in \Omega$,
\[
\xi_n(\omega) = E(f|\mathscr{F}_n)(\omega),
\]
proving the claim.
\end{proof}




\section{L\'evy's continuity theorem}
For a metrizable topological space $X$, we denote by $\mathscr{P}(X)$ the set of Borel
probability measures on $X$. The \textbf{narrow topology on $\mathscr{P}(X)$} is the coarsest topology
such that for each $f \in C_b(X)$, the map
\[
\mu \mapsto \int_X f d\mu
\]
is continuous $\mathscr{P}(X) \to \mathbb{C}$. 

A subset $\mathscr{H}$ of $\mathscr{P}(X)$ is called \textbf{tight} if for each $\epsilon>0$ there is a compact
subset $K_\epsilon$ of $X$ such that if $\mu \in \mathscr{H}$ then
$\mu(X \setminus K_\epsilon) < \epsilon$, i.e. $\mu(K_\epsilon)>1-\epsilon$. (An element $\mu$ of $\mathscr{P}(X)$ is called tight
when $\{\mu\}$ is a tight subset of $\mathscr{P}(X)$.)

For a Borel probability measure $\mu$ on $\mathbb{R}^d$, we define its \textbf{characteristic function} 
$\tilde{\mu}:\mathbb{R}^d \to \mathbb{C}$ by
\[
\tilde{\mu}(u) = \int_{\mathbb{R}^d} e^{ix\cdot u} d\mu(x), \qquad u \in \mathbb{R}^d.
\]
$\tilde{\mu}$ is  bounded by $1$ and is uniformly  continuous.
Because $\mu(\mathbb{R}^d)=1$,
\[
\tilde{\mu}(0) = 1.
\]


\begin{lemma}
Let $\mu \in \mathscr{P}(\mathbb{R})$. For  $\delta>0$, 
\[
\mu\left(\left\{x \in \mathbb{R}: |x| \geq \frac{2}{\delta} \right\} \right)
\leq \frac{1}{\delta} \int_{-\delta}^\delta (1-\tilde{\mu}(u)) du;
\]
in particular, the right-hand side of this inequality is real.
\label{twodelta}
\end{lemma}
\begin{proof}
Using Fubini's theorem and the fact that
all real $t$, $1-\frac{\sin t}{t} \geq 0$, 
\begin{align*}
\int_{-\delta}^\delta (1-\tilde{\mu}(u)) du&=\int_{-\delta}^\delta \left( \int_{\mathbb{R}} (1-e^{ixu}) d\mu(x)\right) du\\
&=\int_{\mathbb{R}} \left(\int_{-\delta}^\delta 1-e^{iu x} du\right) d\mu(x)\\
&=\int_{\mathbb{R}} \left(u - \frac{e^{iux}}{ix}\right)_{-\delta}^\delta d\mu(x)\\
&=\int_{\mathbb{R}} \left(2\delta - \frac{e^{i\delta x}}{ix}+\frac{e^{-i\delta x}}{ix}\right) d\mu(x)\\
&=2\delta \int_{\mathbb{R}} \left(1 -  \frac{\sin(\delta x)}{\delta x} \right) d\mu(x)\\
&\geq 2\delta \int_{|\delta x| \geq 2}  \left(1 -  \frac{\sin(\delta x)}{\delta x} \right) d\mu(x)\\
&\geq 2\delta \int_{|\delta x| \geq 2}  \left(1 -  \frac{1}{|\delta x|} \right) d\mu(x)\\
&\geq 2\delta \int_{|\delta x| \geq 2} \frac{1}{2} d\mu(x)\\
&=\delta \mu(\{x \in \mathbb{R}: |\delta x| \geq 2\}).
\end{align*}
\end{proof}


The following lemma gives a condition on the characteristic functions of a sequence of Borel probability
measures on $\mathbb{R}$ under which the sequence is tight.\footnote{Krishna B. Athreya and Soumendra N. Lahiri,
{\em Measure Theory and Probability Theory}, p.~329, Lemma 10.3.3.}

\begin{lemma}
Suppose that $\mu_n \in \mathscr{P}(\mathbb{R})$ and that 
$\tilde{\mu}_n$ converges pointwise to a function $\phi:\mathbb{R} \to \mathbb{C}$ 
that is continuous at $0$. Then the sequence $\mu_n$ is tight.
\label{1dtight}
\end{lemma}
\begin{proof}
Write $\phi_n=\tilde{\mu}_n$.
Because $|\phi_n| \leq 1$, for each $\delta>0$, by the dominated convergence theorem we have
\[
\frac{1}{\delta} \int_{-\delta}^\delta (1-\phi_n(t)) dt \to
\frac{1}{\delta} \int_{-\delta}^\delta (1-\phi(t)) dt.
\]
On the other hand, that $\phi$ is continuous at $0$ implies that for any $\epsilon>0$ there is some
$\eta>0$ such that when $|t|<\eta$, $|\phi(t)-1|<\epsilon$, and hence for $\delta<\eta$,
\[
\frac{1}{\delta} \int_{-\delta}^\delta (1-\phi(t)) dt
\leq 2\sup_{|t| \leq \delta} |1-\phi(t)|
\leq 2\epsilon,
\]
thus
\[
\frac{1}{\delta} \int_{-\delta}^\delta (1-\phi(t)) dt \to 0, \qquad \delta \to 0.
\]

Let $\epsilon>0$. There is some 
$\delta>0$ for which
\[
\left|\frac{1}{\delta} \int_{-\delta}^\delta (1-\phi(t)) dt \right| < \epsilon.
\]
Then there is some $n_\delta$ such that when $n \geq n_\delta$,
\[
\left| \frac{1}{\delta}\int_{-\delta}^\delta (1-\phi_n(t)) dt - \frac{1}{\delta} \int_{-\delta}^\delta (1-\phi(t)) dt\right|
<\epsilon,
\]
whence
\[
\left| \frac{1}{\delta}\int_{-\delta}^\delta (1-\phi_n(t)) dt \right| < 2\epsilon.
\]
Lemma \ref{twodelta} then says
\[
\mu_n\left( \left\{ x \in \mathbb{R}: |x| \geq \frac{2}{\delta} \right\} \right)
\leq \frac{1}{\delta} \int_{-\delta}^\delta (1-\phi_n(t)) dt
<2\epsilon. 
\]
Furthermore, any Borel probability measure on a Polish space is tight (\textbf{Ulam's theorem}).\footnote{Alexander S. Kechris,
{\em Classical Descriptive Set Theory}, p.~107, Theorem 17.11.}
Thus, for each $1 \leq n < n_\delta$, there is a compact set $K_n$ for which 
$\mu_n(\mathbb{R} \setminus K_n)<\epsilon$. 
Let 
\[
K_\epsilon = K_1 \cup \cdots \cup K_{n_\delta-1} \cup \left\{x \in \mathbb{R}: |x| \leq \frac{2}{\delta}\right\},
\]
which is a compact set, and for any $n \geq 1$,
\[
\mu_n(\mathbb{R} \setminus K_\epsilon)<2\epsilon,
\]
showing that the sequence $\mu_n$ is tight.
\end{proof}



For  metrizable spaces $X_1,\ldots,X_d$, let
$X = \prod_{i=1}^d X_i$ and let $\pi_i:X \to X_i$ be the projection map. 
We  establish that if $\mathscr{H}$
is a subset of $\mathscr{P}(X)$ such that for each $1 \leq i \leq d$ the family
of $i$th marginals of $\mathscr{H}$ is tight, then $\mathscr{H}$ itself is tight.\footnote{Luigi Ambrosio, Nicola Gigli, and Giuseppe Savare,
{\em Gradient Flows: In Metric Spaces and in the Space of Probability Measures},
p.~119, Lemma 5.2.2;
V. I. Bogachev, {\em Measure Theory}, volume II, p.~94, Lemma 7.6.6.}

\begin{lemma}
Let $X_1,\ldots,X_d$ be metrizable topological spaces, let
$X=\prod_{i=1}^d X_i$,  and let $\mathscr{H} \subset \mathscr{P}(X)$.
Suppose that for each $1 \leq i \leq d$,
\[
\mathscr{H}_i = \{{\pi_i}_*\mu: \mu \in \mathscr{H}\}
\]
is a tight set in $\mathscr{P}(X_i)$. Then $\mathscr{H}$ is a tight set in $\mathscr{P}(X)$.
\label{producttight}
\end{lemma}
\begin{proof}
For $\mu \in \mathscr{H}$, write $\mu_i = {\pi_i}_* \mu$.
Let $\epsilon>0$ and take $1 \leq i \leq d$. 
Because $\mathscr{H}_i$ is tight,
there is a compact subset $K_i$ of $X_i$ such that for all $\mu_i \in \mathscr{H}_i$,
\[
\mu_i(X_i \setminus K_i) < \frac{\epsilon}{d}.
\]
Let
\[
K=\prod_{i=1}^d K_i =  \bigcap_{i=1}^d \pi_i^{-1}(K_i).
\]
Then for any $\mu \in \mathscr{H}$,
\begin{align*}
\mu(X \setminus K)&=\mu\left(X \setminus \bigcap_{i=1}^d \pi_i^{-1}(K_i) \right)\\
&=\mu\left( \bigcup_{i=1}^d \pi_i^{-1}(K_i)^c \right)\\
&=\mu\left( \bigcup_{i=1}^d \pi_i^{-1}(X_i \setminus K_i) \right)\\
&\leq \sum_{i=1}^d \mu(\pi_i^{-1}(X_i \setminus K_i))\\
&=\sum_{i=1}^d \mu_i(X_i \setminus K_i)\\
&< \sum_{i=1}^d \frac{\epsilon}{d}\\
&=\epsilon,
\end{align*}
which shows that $\mathscr{H}$ is tight.
\end{proof}




We now prove \textbf{L\'evy's continuity theorem}, which we shall use to prove the martingale central limit theorem.\footnote{cf. Jean Jacod and Philip Protter, {\em Probability Essentials}, second ed., p.~167, Theorem 19.1.}


\begin{theorem}[L\'evy's continuity theorem]
Suppose that $\mu_n \in \mathscr{P}(\mathbb{R}^d)$, $n \geq 1$. 
\begin{enumerate}
\item If $\mu \in \mathscr{P}(\mathbb{R}^d)$ and $\mu_n \to \mu$ narrowly,
then for any $u \in \mathbb{R}^d$,
\[
\tilde{\mu}_n(u) \to \tilde{\mu}(u), \qquad n \to \infty.
\]
\item If there is some $\phi:\mathbb{R}^d \to \mathbb{C}$ to which $\tilde{\mu}_n$ converges pointwise and
$\phi$ is continuous at $0$, then there is some $\mu \in \mathscr{P}(\mathbb{R}^d)$ such that
$\phi = \tilde{\mu}$ and such that $\mu_n \to \mu$ narrowly.
\end{enumerate}
\end{theorem}
\begin{proof}
Suppose that $\mu_n \to \mu$ narrowly. For each $u \in \mathbb{R}^d$, the function
$x \mapsto  e^{i x \cdot u}$ is continuous $\mathbb{R}^d \to \mathbb{C}$ and is bounded, so
\[
\tilde{\mu}_n(u) = \int_{\mathbb{R}^d} e^{ix\cdot u} d\mu_n(x) \to \int_{\mathbb{R}^d} e^{ix\cdot u} d\mu(x)
=\tilde{\mu}(u).
\]

Suppose that $\tilde{\mu}_n$ converges pointwise to $\phi$ and that $\phi$ is continuous at $0$.
For $1 \leq i \leq d$, let $\pi_i:\mathbb{R}^d \to \mathbb{R}$ be the projection map and define
$\iota_i:\mathbb{R} \to \mathbb{R}^d$  by taking
the $i$th entry of $\iota_i(t)$ to be $t$ and the other entries to be $0$.
Fix $1 \leq i \leq d$ and write $\nu_n = {\pi_i}_* \mu_n \in \mathscr{P}(\mathbb{R})$, and for $t \in \mathbb{R}$
we calculate
\begin{align*}
\tilde{\nu}_n(t)&=
\int_{\mathbb{R}} e^{ist} d\nu_n(s)\\
&=\int_{\mathbb{R}^d} e^{i\pi_i(x) t} d\mu_n(x)\\
&=\int_{\mathbb{R}^d} e^{i x\cdot \iota_i(t)} d\mu_n(x)\\
&=\tilde{\mu}_n(\iota_i(t)),
\end{align*}
so $\tilde{\nu}_n = \tilde{\mu}_n \circ \iota_i$. 
By hypothesis, $\tilde{\nu}_n$ converges pointwise to $\phi \circ \iota_i$. 
Because $\phi$ is continuous at $0 \in \mathbb{R}^d$,
the function $\phi \circ \iota_i$ is continuous at $0 \in \mathbb{R}$. 
Then Lemma \ref{1dtight} tells us that the sequence $\nu_n$ is tight.
That is, for each $1 \leq i \leq d$, the set 
\[
\{{\pi_i}_* \mu_n: n \geq 1\}
\]
is tight in $\mathscr{P}(\mathbb{R})$. Thus Lemma \ref{producttight} tells us that
the set
\[
\{\mu_n: n \geq 1\}
\]
is tight in $\mathscr{P}(\mathbb{R}^d)$. 

\textbf{Prokhorov's theorem}\footnote{V. I. Bogachev,
{\em Measure Theory}, volume II, p.~202, Theorem 8.6.2.}
 states that if $X$ is a Polish space, then a subset $\mathscr{H}$
of $\mathscr{P}(X)$ is tight if and only if each sequence of elements of $\mathscr{H}$ has a subsequence that converges
narrowly to some element of $\mathscr{P}(X)$. 
Thus, there is a subsequence $\mu_{a(n)}$ of $\mu_n$ and some $\mu \in \mathscr{P}(\mathbb{R}^d)$ such that
$\mu_{a(n)}$ converges narrowly to $\mu$. 
By the first part of the theorem, we get that $\tilde{\mu}_{a(n)}$ converges pointwise to
$\tilde{\mu}$. 
But by hypothesis $\tilde{\mu}_n$ converges pointwise to $\phi$, so
$\phi=\tilde{\mu}$.

Finally we prove that $\mu_n \to \mu$ narrowly. Let $\mu_{b(n)}$ be a subsequence of $\mu_n$. Because $\{\mu_n: n \geq 1\}$ is tight,
Prokhorov's theorem tells us that there is a subsequence $\mu_{c(n)}$ of $\mu_{b(n)}$ that converges narrowly to some
$\lambda \in \mathscr{P}(\mathbb{R}^d)$. By the first part of the theorem, $\tilde{\mu}_{c(n)}$ converges pointwise to
$\tilde{\lambda}$. By hypothesis $\tilde{\mu}_{c(n)}$ converges pointwise to $\phi$, so
$\tilde{\lambda}=\phi=\tilde{\mu}$. Then
$\lambda = \mu$. That is, any subsequence of $\mu_n$ itself has a subsequence that converges narrowly to
$\mu$, which implies that the sequence
$\mu_n$ converges narrowly to $\mu$. (For a sequence $x_n$ in a topological space $X$ and $x \in X$, 
$x_n \to x$ if and only if each subsequence of $x_n$ has a subsequence that converges to $x$.)
\end{proof}


\section{Martingale central limit theorem}
Let $\gamma_d$ be the \textbf{standard Gaussian measure on $\mathbb{R}^d$}: $\gamma_d$ has density
\[
\frac{1}{\sqrt{(2\pi)^d}} e^{-\frac{1}{2}|x|^2}
\]
with respect to Lebesgue measure on $\mathbb{R}^d$. 

We now prove the \textbf{martingale central limit theorem}.\footnote{Jean Jacod and Philip Protter, {\em Probability Essentials}, second ed., p.~235, Theorem 27.7.}


\begin{theorem}[Martingale central limit theorem]
Suppose $X_j$ is a sequence in $L^3(\Omega,\mathscr{A},P)$ satisfying the following, with
$\mathscr{F}_k=\sigma(X_1,\ldots,X_k)$:
\begin{enumerate}
\item $E(X_{j}|\mathscr{F}_{j-1}) = 0$.\\
\item $E(X_{j}^2| \mathscr{F}_{j-1}) = 1$.\\
\item There is some $K$ for which $E(|X_{j}|^3 | \mathscr{F}_{j-1}) \leq K$.
\end{enumerate}
Then $\frac{S_n}{\sqrt{n}}$ converges in distribution to some random variable $Z:\Omega \to \mathbb{R}$ with
$Z_*P=\gamma_1$, where
\[
S_n = \sum_{j=1}^n X_j.
\]
\end{theorem}
\begin{proof}
For positive integers $n$ and $j$, define
\[
\phi_{n,j}(u) = E(e^{iu\frac{1}{\sqrt{n}}X_j} | \mathscr{F}_{j-1}).
\]
For each $\omega \in \Omega$, by Taylor's theorem there is some $\xi_{n,j}(\omega)$  between $0$ and $X_j(\omega)$ such
that
\[
e^{iu\frac{1}{\sqrt{n}}X_j(\omega)}  = 1 + iu\frac{1}{\sqrt{n}}X_j(\omega) - \frac{u^2}{2n} X_j(\omega)^2
-\frac{iu^3}{6n^{3/2}} \xi_{n,j}(\omega)^3.
\]
Because $f \mapsto E(f|\mathscr{F}_{j-1})$ is a positive operator and
$|\xi_{n,j}|^3 \leq |X_j|^3$, we have, by the last hypothesis of the theorem,
\begin{equation}
E(|\xi_{n,j}|^3 | \mathscr{F}_{j-1}) \leq E(|X_j|^3 | \mathscr{F}_{j-1}) \leq K
\label{absinequality}
\end{equation}
we use this inequality later in the proof.
Now, using that $E(X_j|\mathscr{F}_{j-1})=0$ and $E(X_j^2|\mathscr{F}_{j-1})=1$,
\begin{align*}
\phi_{n,j}(u)&=1+iu\frac{1}{\sqrt{n}} E(X_j|\mathscr{F}_{j-1})
-\frac{u^2}{2n} E(X_j^2|\mathscr{F}_{j-1})
-\frac{iu^3}{6n^{3/2}} E(\xi_{n,j}^3|\mathscr{F}_{j-1})\\
&=1-\frac{u^2}{2n} -\frac{iu^3}{6n^{3/2}} E(\xi_{n,j}^3|\mathscr{F}_{j-1}).
\end{align*}
For $p \geq 1$, 
\begin{align*}
E(e^{iu\frac{1}{\sqrt{n}} S_p})&=
E(e^{iu\frac{1}{\sqrt{n}} S_{p-1}} e^{iu\frac{1}{\sqrt{n}} X_p})\\
&=E(E(e^{iu\frac{1}{\sqrt{n}} S_{p-1}} e^{iu\frac{1}{\sqrt{n}} X_p}|\mathscr{F}_{p-1}))\\
&=E(e^{iu\frac{1}{\sqrt{n}} S_{p-1}} E( e^{iu\frac{1}{\sqrt{n}} X_p}|\mathscr{F}_{p-1}))\\
&=E(e^{iu\frac{1}{\sqrt{n}} S_{p-1}} \phi_{n,p}(u))\\
&=E\left(e^{iu\frac{1}{\sqrt{n}} S_{p-1}}
\left( 1-\frac{u^2}{2n} -\frac{iu^3}{6n^{3/2}} E\left(\xi_{n,p}^3|\mathscr{F}_{p-1}\right) \right) \right),
\end{align*}
which we write as
\[
E\left(e^{i \frac{u}{\sqrt{n}} S_p} - \left(1-\frac{u^2}{2n}\right) e^{i \frac{u}{\sqrt{n}} S_{p-1}} \right)
=-E\left(e^{i\frac{u}{\sqrt{n}} S_{p-1}} \frac{iu^3}{6n^{3/2}}E\left(\xi_{n,p}^3|\mathscr{F}_{p-1}\right) \right).
\]
Now using \eqref{absinequality} we get
\[
\begin{split}
&\left|E\left(e^{i \frac{u}{\sqrt{n}} S_p} - \left(1-\frac{u^2}{2n}\right) e^{i \frac{u}{\sqrt{n}} S_{p-1}} \right)\right|\\
\leq&E\left( \left| e^{i\frac{u}{\sqrt{n}} S_{p-1}} \frac{iu^3}{6n^{3/2}}E\left(\xi_{n,p}^3|\mathscr{F}_{p-1}\right)  \right| \right)\\
=&E\left( \frac{|u|^3}{6n^{3/2}} \left|E\left(\xi_{n,p}^3|\mathscr{F}_{p-1}\right)\right| \right)\\
&\leq \frac{|u|^3}{6n^{3/2}} \cdot K.
\end{split}
\]

Let $u \in \mathbb{R}$ and let $n=n(u)$ be large enough so that
$0 \leq 1-\frac{u^2}{2n} \leq 1$. For $1 \leq p \leq n$, multiplying the above inequality by $\left(1-\frac{u^2}{2n}\right)^{n-p}$
yields
\begin{equation}
\left| \left(1-\frac{u^2}{2n} \right)^{n-p} E(e^{iu\frac{1}{\sqrt{n}}S_p})
-\left(1-\frac{u^2}{2n}\right)^{n-p+1} E(e^{iu\frac{1}{\sqrt{n}}S_{p-1}}\right|
\leq K \frac{|u|^3}{6n^{3/2}}.
\label{np}
\end{equation}
Now, because $\sum_{p=1}^n (a_p-a_{p-1})=a_n-a_0$,
\[
\begin{split}
&\sum_{p=1}^n \left( \left(1-\frac{u^2}{2n} \right)^{n-p} E(e^{iu\frac{1}{\sqrt{n}}S_p})
-\left(1-\frac{u^2}{2n}\right)^{n-(p-1)}
E(e^{iu\frac{1}{\sqrt{n}}S_{p-1}})\right)\\
=&E(e^{iu\frac{1}{\sqrt{n}}S_n})-\left(1-\frac{u^2}{2n}\right)^n E(e^{iu\frac{1}{\sqrt{n}}S_0})\\
=&E(e^{iu\frac{1}{\sqrt{n}}S_n}) - \left(1-\frac{u^2}{2n}\right)^n.
\end{split}
\]
Using this with \eqref{np} gives
\[
\left| E(e^{iu\frac{1}{\sqrt{n}}S_n}) - \left(1-\frac{u^2}{2n}\right)^n \right| \leq n\cdot K\frac{|u|^3}{6n^{3/2}}
=K\frac{|u|^3}{6n^{1/2}}. 
\]
But if $|a_n-b_n| \leq c_n$, $c_n \to 0$, and $b_n \to b$, then $a_n \to b$. As
\[
\lim_{n \to \infty} \left(1-\frac{u^2}{2n}\right)^n = e^{-\frac{u^2}{2}}
\]
and $K\frac{|u|^3}{6n^{1/2}} \to 0$, we therefore get that
\[
E(e^{iu\frac{1}{\sqrt{n}}S_n}) \to e^{-\frac{u^2}{2}}
\]
 as $n \to \infty$. 

Let $\mu_n=\left(\frac{S_n}{\sqrt{n}}\right)_*P$ and let $\phi(u)=e^{-\frac{u^2}{2}}$. We have just established that
$\tilde{\mu}_n \to \phi$ pointwise. The function $\phi$ is continuous at $0$, so 
L\'evy's continuity theorem tells us that there is a Borel probability measure $\mu$ on
$\mathbb{R}$ such that $\phi = \tilde{\mu}$ and such that $\mu_n$ converges narrowly to $\mu$. 
But $\phi(u)=e^{-\frac{u^2}{2}}$ is the characteristic function of $\gamma_1$, so we have that
$\mu_n$ converges narrowly to $\gamma_1$. 


\end{proof}


\end{document}