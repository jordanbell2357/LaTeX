\documentclass{article}
\usepackage{amsmath,amssymb,graphicx,subfig,mathrsfs,amsthm,siunitx}
%\usepackage{tikz-cd}
%\usepackage{hyperref}
\newcommand{\inner}[2]{\left\langle #1, #2 \right\rangle}
\newcommand{\tr}{\ensuremath\mathrm{tr}\,} 
\newcommand{\Span}{\ensuremath\mathrm{span}} 
\def\Re{\ensuremath{\mathrm{Re}}\,}
\def\Im{\ensuremath{\mathrm{Im}}\,}
\newcommand{\id}{\ensuremath\mathrm{id}} 
\newcommand{\rank}{\ensuremath\mathrm{rank\,}} 
\newcommand{\co}{\ensuremath\mathrm{co}\,} 
\newcommand{\cco}{\ensuremath\overline{\mathrm{co}}\,}
\newcommand{\supp}{\ensuremath\mathrm{supp}}
\newcommand{\epi}{\ensuremath\mathrm{epi}\,}
\newcommand{\Tr}{\ensuremath\mathrm{Tr}\,}
\newcommand{\lsc}{\ensuremath\mathrm{lsc}\,}
\newcommand{\ext}{\ensuremath\mathrm{ext}\,}
\newcommand{\cl}{\ensuremath\mathrm{cl}\,}
\newcommand{\dom}{\ensuremath\mathrm{dom}\,}
\newcommand{\LSC}{\ensuremath\mathrm{LSC}}
\renewcommand{\div}{\ensuremath\mathrm{div}\,}
\newcommand{\curl}{\ensuremath\mathrm{curl}\,}
\newcommand{\USC}{\ensuremath\mathrm{USC}}
\newcommand{\upto}{\nearrow}
\newcommand{\downto}{\searrow}
\newcommand{\norm}[1]{\left\Vert #1 \right\Vert}
\newtheorem{theorem}{Theorem}
\newtheorem{lemma}[theorem]{Lemma}
\newtheorem{proposition}[theorem]{Proposition}
\newtheorem{corollary}[theorem]{Corollary}
\theoremstyle{definition}
\newtheorem{definition}[theorem]{Definition}
\newtheorem{example}[theorem]{Example}
\begin{document}
\title{The Hamilton-Jacobi equation}
\author{Jordan Bell\\ \texttt{jordan.bell@gmail.com}\\Department of Mathematics, University of Toronto}
\date{\today}

\maketitle

\section{Example of free particle in one dimension}
Define $L(q,v)=\frac{m}{2}v^2$, where $m$ is a nonzero constant. Fixing two times $t_0<t_1$, we define the action for a path $\gamma$ in $\mathbb{R}$ by
\begin{eqnarray*}
S(\gamma) &=& \int_{t_0}^{t_1} L(\gamma(t),\dot{\gamma}(t)) dt\\
&=&\frac{m}{2} \int_{t_0}^{t_1} (\dot{\gamma}(t))^2 dt. 
\end{eqnarray*}
Suppose that $\gamma$ is satisfies the Euler-Lagrange equation for the Lagrangian $L$. That is,
\[
\frac{d}{dt} \left(\frac{\partial L}{\partial v}(\gamma(t),\dot{\gamma}(t)) \right)  - \frac{\partial L}{\partial q}(\gamma(t),\dot{\gamma}(t)) = 0,
\]
and here $\frac{\partial L}{\partial v}=mv$ and $\frac{\partial L}{\partial q}=0$, so
\[
\frac{d}{dt} (m \dot{\gamma}(t))=0,
\]
i.e.
\[
m\ddot{\gamma}(t)=0,
\]
so $\ddot{\gamma}(t)=0$, and hence $\gamma(t)=at+b$ for some constants $a,b$. If we are given the conditions $\gamma(t_0)=q_0$
and $\gamma(t_1)=q_1$, then a solution of the Euler-Lagrange equation that satisfies these conditions must be
\[
\gamma(t) = q_0 + \frac{q_1-q_0}{t_1-t_0} ( t-t_0).
\]
The action of this path is
\[
S(\gamma)=\frac{m}{2} \int_{t_0}^{t_1} \left(\frac{q_1-q_0}{t_1-t_0} \right)^2 dt=
\frac{m}{2} \frac{(q_1-q_0)^2}{t_1-t_0}.
\]
The dimensions of the right hand side are 
\[
\si{kg. m^2 . s^{-1}}=\si{kg . m . s^{-2} . m. s}=\si{N . m . s} = \si{J.s},
\]
 which are indeed the dimensions
that action ought to have.
If instead of talking about action that is a function of paths we talk about action that is a function of the end point of a motion and the time
at which the motion ends, taking the time and location at which the motion starts as fixed, then
\[
S(q,t) = \frac{m}{2} \frac{(q-q_0)^2}{t-t_0},
\]
for which
\[
\frac{\partial S}{\partial q} = m \frac{q-q_0}{t-t_0}
\]
and
\[
\frac{\partial S}{\partial t} = - \frac{m}{2} \frac{(q-q_0)^2}{(t-t_0)^2}.
\]
These satisfy
\[
\frac{\partial S}{\partial t}  + \frac{1}{2m} \left(\frac{\partial S}{\partial q}\right)^2=0.
\]
If we write
\[
H(q,p) = \frac{p}{m} \frac{\partial L}{\partial v}\left(q,\frac{p}{m} \right)  - L\left(q,\frac{p}{m}\right) = \frac{p}{m} \cdot m \cdot \frac{p}{m} - \frac{m}{2} \left( \frac{p}{m}\right)^2 = 
\frac{p^2}{2m}.
\]
Then,
\[
\frac{\partial S}{\partial t}(q,t) + H\left(q,\frac{\partial S}{\partial q}(q,t) \right)=0.
\]


\section{Motivation for the Hamilton-Jacobi equation}
Suppose that $\gamma$ is a path that satisfies the Euler-Lagrange equation for some Lagrangian $L$. If we perturb the path to start at the same position at time $t_0$ but to end  at
$q+\delta q$ instead of at $q$, then the perturbed path is $s \mapsto \gamma(s)+(\delta \gamma)(s)$, where $(\delta \gamma)(0)=0$ and $(\delta \gamma)(t)=\delta q$.  
Then, first doing a Taylor approximation in which we drop all powers of $\delta \gamma$ or $\delta \dot{\gamma}$ higher than the first and then using the Euler-Lagrange
equation, and
using Einstein summation notation, 
\begin{eqnarray*}
S(\gamma+\delta \gamma)-S(\gamma)&=&\int_{t_0}^t L(\gamma+\delta \gamma,\dot{\gamma}+\delta \dot{\gamma}) - L(\gamma,\dot{\gamma}) ds\\
&=&\int_{t_0}^t \frac{\partial L}{\partial q_i}(\gamma,\dot{\gamma}) \delta \gamma_i + \frac{\partial L}{\partial v_i}(\gamma,\dot{\gamma}) \delta \dot{\gamma_i} ds\\
&=&\int_{t_0}^t \left( \frac{d}{dt} \frac{\partial L}{\partial v_i}(\gamma,\dot{\gamma}) \right) \delta \gamma_i + \frac{\partial L}{\partial v_i}(\gamma,\dot{\gamma}) \delta \dot{\gamma_i} ds\\
&=&\int_{t_0}^t \frac{d}{dt} \left(  \frac{\partial L}{\partial v_i}(\gamma,\dot{\gamma}) \delta \gamma_i \right) ds\\
&=&\frac{\partial L}{\partial v_i}(\gamma(t),\dot{\gamma}(t)) (\delta \gamma_i)(t) - \frac{\partial L}{\partial v_i}(\gamma(t_0),\dot{\gamma}(t_0)) (\delta \gamma_i)(t_0)\\
&=&\frac{\partial L}{\partial v_i}(\gamma(t),\dot{\gamma}(t)) \delta q.
\end{eqnarray*}
We have not been precise about what we mean by perturbing a path, but what we have obtained suggests that if
we think of $S$ as a function of the endpoint of a path and the time at which the path ends rather than as a function of a path
itself, we have
\[
\frac{\partial S}{\partial q_i} = \frac{\partial L}{\partial v_i}(\gamma(t),\dot{\gamma}(t)) = p_i(t).
\]

If on the other hand we fix the point $q$ at which a path ends and change the time at which it arrives at this point 
from $t$ to $t+\delta t$, then, doing a Taylor expansion and dropping all
powers of $\delta t$ higher than the first,
\[
\gamma(t+\delta t)-\gamma(t) = \dot{\gamma}(t)\delta t.
\]
But $\gamma(t+\delta t)=q$, so
\[
\gamma(t) = q-  \dot{\gamma}(t)\delta t.
\]
Then
\[
\delta S = L \delta t -\frac{\partial L}{\partial v_i} v_i \delta t.
\]
Defining $H=\frac{\partial L}{\partial v_i} v_i - L$, what we have done suggests that 
\[
\frac{\partial S}{\partial t} = -H.
\]

Then, using $\frac{\partial S}{\partial q_i}=p_i$,  with which $H(q,p)=H\left(q,\frac{\partial S}{\partial q} \right)$, and using $\frac{\partial S}{\partial t}=-H(q,p)$,
we have
\[
\frac{\partial S}{\partial t}+H\left(q,\frac{\partial S}{\partial q}\right)=0.
\]
We call this equation the {\em Hamilton-Jacobi equation}.

To precisely sort out where the Hamilton-Jacobi equation comes from and what it means, the only place
I can imagine that does an adequate job is Abraham and Marsden.\footnote{Abraham and Marsden, {\em Foundations of Mechanics},
second ed.} Certainly there are other sources that
present this more precisely than I have presented it, but it is almost universal to thoughtlessly confound 
the variables on which $H$ or $S$ depends with paths; that is, to write things like $\frac{\partial H}{\partial q}$ and
also to think of $q$ not as a point but rather as a path which for each time goes through a particular point, in which case
one has no certain way of knowing whether $\frac{dq}{dt}=0$, as is the case for the derivative of any fixed point, or to say
that $q$ is a path and that $\frac{dq}{dt}$ is a tangent vector at the point $q(t)$ on the path.
If one plainly states that what one has said is only suggestive of how symbols work together then one does not
need to apologize for the absence of precision, but there is a foul area between suggestive symbol manipulation and
actual precision in which
one tricks oneself into believing that one has given a precise presentation, and this is the path followed by
 some presentations
of the Hamilton-Jacobi equation.


\section{Harmonic oscillator}
Let
\[
H=\frac{p^2}{2m}+\frac{1}{2}m\omega^2 q^2.
\]
The Hamilton-Jacobi equation for this Hamiltonian is
\[
\frac{\partial S}{\partial t}+\frac{1}{2m} \left( \frac{\partial S}{\partial q} \right)^2 + \frac{1}{2}m\omega^2 q^2 =0.
\]
We set $S(q,t)=S_0(q)-Et$; for this to make sense presumes that there is in fact a constant $E$ and a function $S_0$ so that
 $S(q,t)-S_0(q)$ depends just on $t$. With this,
$\frac{\partial S}{\partial q} = \frac{\partial S_0}{\partial q}$
and $\frac{\partial S}{\partial t}=-E$,
and so the Hamilton-Jacobi equation becomes
\[
-E+ \frac{1}{2m}  \left( \frac{\partial S_0}{\partial q} \right)^2 + \frac{1}{2}m\omega^2 q^2 = 0 ,
\]
or
\[
\frac{1}{2m}  \left( \frac{\partial S_0}{\partial q} \right)^2 + \frac{1}{2}m\omega^2 q^2=E.
\]
Supposing that $S_0$ is nonnegative we get 
\[
\frac{\partial S_0}{\partial q}(q) = \sqrt{2mE-m^2 \omega^2 q^2},
\]
a primitive of which is
\begin{eqnarray*}
S_0 &=& \int \sqrt{2mE-m^2 \omega^2 q^2} dq\\
 &=&\frac{E}{\omega} \left( \arcsin \frac{m\omega q}{\sqrt{2mE}} + \frac{m\omega q}{\sqrt{2mE}} \sqrt{1-\left(\frac{m\omega q}{\sqrt{2mE}}\right)^2}\right)
 \end{eqnarray*}
for which
\[
\frac{\partial S_0}{\partial E} = \frac{1}{\omega} \arcsin \frac{m \omega q}{\sqrt{2mE}}.
\]
But $\frac{\partial S_0}{\partial E}=t$ (using the expression involving $S$ and $Et$), so $\omega t =  \arcsin \frac{m \omega q}{\sqrt{2mE}}$,
hence
\[
\sin(\omega t) = \frac{\sqrt{m}\omega q}{\sqrt{2E}},
\]
and therefore
\[
q=\sqrt{\frac{2E}{m\omega^2}} \sin(\omega t).
\] 
As well,
\[
p = \frac{\partial S}{\partial q} = \frac{\partial S_0}{\partial q} = \sqrt{2mE-m^2 \omega^2 q^2} ,
\]
and using the above expression for $q$ this becomes
\[
p= \sqrt{2mE - m^2 \omega^2 \frac{2E}{m\omega^2} \sin^2(\omega t)} = \sqrt{2mE - 2mE\sin^2(\omega t)}.
\]
We have thus written $q$ and $p$ as functions of $E$ and $t$. Since for a particular trajectory the energy is fixed, on a particular
trajectory the position and momentum have thus been expressed as functions of $t$. 


\section{Schr\"odinger equation}
Write 
\[
i\hbar \frac{\partial \psi}{\partial t}=H\psi,
\]
called the {\em Schr\"odinger equation}. 
If $H(q,p)=\frac{p^2}{2m}+V(q)$, and $p=-i\hbar \nabla$, then $p^2=- \hbar^2 \Delta$  and the Schr\"odinger equation is
\[
i\hbar \frac{\partial \psi}{\partial t} = -\frac{\hbar^2}{2m} \Delta \psi + V(q)\psi.
\]
Supposing that there is a solution $\psi$ of the form $\psi=e^{i\frac{S}{\hbar}}$, we get
\begin{eqnarray*}
i\hbar e^{i\frac{S}{\hbar}}  \frac{i}{\hbar} \frac{\partial S}{\partial t}&=&-\frac{\hbar^2}{2m} \frac{\partial}{\partial q} \left(e^{i\frac{S}{\hbar}} \frac{i}{\hbar}
\frac{\partial S}{\partial q} \right)+V(q)e^{i\frac{S}{\hbar}}\\
&=&-\frac{\hbar^2}{2m}  \left( e^{i\frac{S}{\hbar}} \left(\frac{i}{\hbar}
\frac{\partial S}{\partial q}  \right)^2 + e^{i\frac{S}{\hbar}} \frac{i}{\hbar} \frac{\partial^2 S}{\partial q^2} \right)+V(q)e^{i\frac{S}{\hbar}}.
\end{eqnarray*}
Diving both sides by $e^{i\frac{S}{\hbar}}$ gives
\[
- \frac{\partial S}{\partial t} = \frac{1}{2m} \left( \frac{\partial S}{\partial q} \right)^2 -\frac{i\hbar}{2m} \frac{\partial^2 S}{\partial q^2}
+V(q).
\]
Taking $\hbar \to 0$ yields the equation
\[
-\frac{\partial S}{\partial t} = \frac{1}{2m}\left( \frac{\partial S}{\partial q}\right)^2 + V(q) = H\left(q,\frac{\partial S}{\partial q}\right),
\]
which is the Hamilton-Jacobi equation.

The above derivation of the Hamilton-Jacobi equation from the Schr\"odinger equation is suggestive symbol manipulation. Rather than stating that we assume  $\psi=e^{i\frac{S}{\hbar}}$,
I could have written that we ``make an Ansatz''; ``ein Ansatz'' means ``an approach'', and is used to mean a guess which may work out.
Of course, if there is some problem for which one knows there is a unique solution and we find an explicit solution starting from some unjustified assumption,
then we don't need to have justified the assumption because we can explicitly check that what we have found is a solution. But this is the only situation where there is precision
to ``making an Ansatz''. Otherwise, to talk about an Ansatz is a sophisticated sounding way of saying  ``we make an assumption'', and after making this assumption
we have no guarantee that anything we end up with need make any sense. This does not mean that it is useless to make unjustified assumptions; but it is
deceitful to smuggle Ans\"atze into the realm of proved things, and confuses those who later would rely on one's work. 


\end{document}