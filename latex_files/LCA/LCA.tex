\documentclass{article}
\usepackage{amsmath,amssymb,graphicx,subfig,mathrsfs,amsthm}
%\usepackage{tikz-cd}
\usepackage{hyperref}
\newcommand{\innerL}[2]{\langle #1, #2 \rangle_{L^2}}
\newcommand{\inner}[2]{\langle #1, #2 \rangle}
\def\Re{\ensuremath{\mathrm{Re}}\,}
\def\Im{\ensuremath{\mathrm{Im}}\,}
\newcommand{\HSnorm}[1]{\Vert #1 \Vert_{\ensuremath\mathrm{HS}}}
\newcommand{\HSinner}[2]{\left\langle #1, #2 \right\rangle_{\ensuremath\mathrm{HS}}}
\newcommand{\tr}{\textrm{tr}} 
\newcommand{\supp}{\mathrm{supp}\,} 
\newcommand{\Span}{\textrm{span}} 
\newcommand{\id}{\textrm{id}} 
\newcommand{\Hom}{\textrm{Hom}}
\newcommand{\HS}{B_{\ensuremath\mathrm{HS}}} 
\newcommand{\norm}[1]{\Vert #1 \Vert}
\renewcommand{\div}{\mathrm{div}}
\newtheorem{theorem}{Theorem}
\newtheorem{lemma}[theorem]{Lemma}
\newtheorem{proposition}[theorem]{Proposition}
\newtheorem{corollary}[theorem]{Corollary}
\newtheorem{definition}[theorem]{Definition}
\begin{document}
\title{Locally compact abelian groups}
\author{Jordan Bell\\ \texttt{jordan.bell@gmail.com}\\Department of Mathematics, University of Toronto}
\date{\today}
\maketitle

\section{Introduction}
These notes are a gloss on the first chapter of Walter Rudin's {\em Fourier Analysis on Groups}, and 
may be helpful to someone reading Rudin. The results I do prove are proved in more detail than they are in Rudin.
I caution that before reading the first chapter of that book 
it is know about the Gelfand transform on commutative Banach algebras
 because results from that are used without even stating them by Rudin. I at least state them.

\section{Locally compact abelian groups}
Let $G$ be a locally compact abelian (LCA) group. There is a Haar measure $m$ on $G$. 
It is a fact that if $m$ and $m'$ are Haar measures on $G$, then there is a positive constant $\lambda$ such that 
$m'=\lambda m$.\footnote{Walter Rudin, {\em Fourier Analysis on Groups}, p.~2, \S 1.1.3.}
If $G$ is compact
then there is a unique Haar measure with $m(G)=1$, and if $G$ is discrete then there is a unique Haar measure
with $m(\{x\})=1$ for each $x \in G$.




Because Haar measures on $G$ are positive multiples of each other,
the elements of $L^p(G)$ do not depend on the Haar measure we use, but the value of
$\norm{f}_p$ will, where   $f:G \to \mathbb{C}$ is a  Borel function.

If $X$ is a normed vector space and $f:G \to X$ is a function, we say that $f$ is \textbf{uniformly continuous}
if for every $\epsilon>0$ there is a neighborhood $V$ of $0$ in $G$ such that $x-y \in V$ implies that
$\norm{f(x)-f(y)}_X < \epsilon$.

If $f:G \to \mathbb{C}$ is a function and $x \in G$, we define $f_x:G \to \mathbb{C}$ by 
\[
f_x(y)=f(y-x), \qquad y \in G.
\]

\begin{theorem}
If $1 \leq p < \infty$ and $f \in L^p(G)$, then
\[
x \mapsto f_x
\]
is uniformly continuous $G \to L^p(G)$.
\end{theorem}
\begin{proof}
Let $\epsilon>0$. Because $C_c(G)$ is dense in $L^p(G)$, there is some $g \in C_c(G)$ such that 
$\norm{g-f}_p < \epsilon$. Let $K = \supp g$, and because $K$ is compact, $m(K)<\infty$. $g \in C_c(G)$ implies that $g$ is uniformly continuous on $G$, so there
is a neighborhood $V$ of $0$ in $G$ such that if $x-y \in V$ then $|g(x)-g(y)|<\epsilon (2m(K))^{-1/p}$. Then, for all $x \in V$ and $y \in G$, as $y-(y-x) \in V$ we have
$|g(y-x)-g(y)|<\epsilon (2m(K))^{-1/p}$. That is, for all $x \in V$ we have
\[
\norm{g_x-g}_\infty < \epsilon (2m(K))^{-1/p}.
\]
For $x \in V$, $\supp (g_x-g) \subset (K+x) \cup K$ and $m(K+x)=m(K)$, so for all $x \in V$,
\begin{eqnarray*}
\norm{g_x-g}_p &=& \left( \int_{\supp(g_x-g)} |g_x-g|^p dm \right)^{1/p}\\
& \leq& \left( \int_{\supp(g_x-g)}  \epsilon^p (2m(K))^{-1} dm\right)^{1/p}\\
&\leq&\epsilon.
\end{eqnarray*}
Because $\norm{f_x-g_x}_p=\norm{f-g}_p$, for all $x \in V$ we have
\[
\norm{f_x-f}_p \leq \norm{f_x-g_x}_p+\norm{g_x-g}_p+\norm{g-f}_p < 3\epsilon.
\]
Then, let $x,y \in G$ with $y-x \in V$. The above inequality tells us
\[
\norm{f_{y-x}-f}_p < 3\epsilon.
\]
But $f_x-f_y = (f-f_{y-x})_x$ and $\norm{h_x}_p=\norm{h}_p$, so
\[
\norm{f_x-f_y}_p =\norm{f-f_{y-x}}_p < 3\epsilon,
\]
showing that $x \mapsto f_x$ is uniformly continuous.
\end{proof}

If $f$ and $g$ are  Borel functions on $G$, for $x \in G$ such that the integral exists we define
\[
(f*g)(x) = \int_G f(x-y) g(y) dm(y).
\]
The operation of taking the convolution of functions is particularly suitable for functions that belong
to $L^1(G)$, because if $f,g \in L^1(G)$, then $(f*g)(x)$ is defined for almost all $x \in G$, and satisfies
\[
\norm{f*g}_1 \leq \norm{f}_1 \norm{g}_1;
\]
this is proved in Rudin, together with other properties of the convolution.\footnote{Walter
Rudin, {\em Fourier Analysis on Groups}, p.~4, \S 1.1.6.} From the above inequality, it follows that $L^1(G)$ with convolution as multiplication
is a commutative Banach algebra. The map $^*:L^1(G) \to L^1(G)$ defined by $f^*(x)=\overline{f(-x)}$ for $f \in L^1(G)$, $x \in G$,
is an isometric involution.

If $G$ is discrete, define $e$ on $G$ by $e(0)=1$ and $e(x)=0$ for $x \neq 0$. Then $\norm{e}_1 = \int_G e(x) dm(x) = \sum_{x \in G} e(x)=1$, so
$e \in L^1(G)$. For $f \in L^1(G)$ and $x \in G$,
\[
(f*e)(x) = \sum_{y \in G} f(x-y)e(y) = f(x)e(0)=f(x),
\]
showing that $f*e=f$. Hence $e$ is unity in $L^1(G)$.  It turns out that if $G$ is not discrete then $L^1(G)$ does not have a unity.\footnote{Walter
Rudin, {\em Fourier Analysis on Groups}, p.~30, \S 1.7.3.}


\section{Dual groups}
Let $\mathbb{T}=\{z \in \mathbb{C}: |z| =1\}$ with the subspace topology inherited from $\mathbb{C}$. If $G$ is a locally compact abelian group, we denote by
$\Gamma$ the set of continuous homomorphisms $G \to \mathbb{T}$. For $\gamma \in \Gamma$ and $x \in G$, we write
\[
\inner{x}{\gamma} = \gamma(x).
\]
We call a homomorphism $G \to \mathbb{T}$ a \textbf{character} of $G$, so elements of $\Gamma$ are the continuous characters of $G$.

If $A$ is a Banach algebra and $\Delta$ is the set of algebra homomorphisms $h:A \to \mathbb{C}$ that are not identically zero,
the \textbf{Gelfand transform} of $x \in A$ is the map $\hat{x}:\Delta \to \mathbb{C}$ defined by
\[
\hat{x}(h)=h(x), \qquad h \in \Delta.
\]
$\Delta$ is called the \textbf{maximal ideal space} of $A$. 


If $f \in L^1(G)$, we define $\hat{f}:\Gamma \to \mathbb{C}$ by
\[
\hat{f}(\gamma) = \int_G f(x) \inner{-x}{\gamma} dm(x), \qquad \gamma \in \Gamma,
\]
and call $\hat{f}$ the \textbf{Fourier transform} of $f$. 

The following theorem 
is from Rudin.\footnote{Walter Rudin, {\em Fourier Analysis on Groups}, p.~7, \S 1.2.2.} It establishes a bijection between the dual
group $\Gamma$  of $G$ and the maximal ideal space $\Delta$
of $L^1(G)$, and shows, if $\Gamma$ and $\Delta$ are identified, that the Fourier transform is the same as the Gelfand transform. 

\begin{theorem}
For each $\gamma \in \Gamma$, the map 
$h:L^1(G) \to \mathbb{C}$ defined by
$h(f) =  \hat{f}(\gamma)$ belongs to $\Delta$. If $h \in \Delta$, then there is 
a unique $\gamma \in \Gamma$ such that $h(f)=\hat{f}(\gamma)$ for all $f \in L^1(G)$.
\label{maxideal}
\end{theorem}
\begin{proof}
For $f,g \in L^1(G)$ and $k=f*g$, then for any $\gamma \in \Gamma$, using $\inner{-x}{\gamma}=
\inner{-x+y}{\gamma} \inner{-y}{\gamma}$,
\begin{eqnarray*}
\hat{k}(\gamma)&=&\int_G (f*g)(x) \inner{-x}{\gamma} dm(x)\\
&=&\int_G \inner{-x}{\gamma} \left( \int_G f(x-y) g(y) dm(y) \right) dm(x)\\
&=&\int_G g(y) \inner{-y}{\gamma} \left(\int_G f(x-y) \inner{-x+y}{\gamma} dm(x) \right) dm(y)\\
&=&\int_G g(y) \inner{-y}{\gamma} \hat{f}(x) dm(y)\\
&=&\hat{f}(\gamma) \hat{g}(\gamma),
\end{eqnarray*}
showing that $f \mapsto \hat{f}(\gamma)$ is an algebra homomorphism. By Urysohn's lemma, for every  neighborhood
$V$ of $0 \in G$, there is some $f \in C_c(G)$ with $f(0)=1$, $0 \leq f \leq 1$, and $\supp f \subset V$. 
As $\inner{0}{\gamma}=1$ and because $x \mapsto \inner{-x}{\gamma}$ is continuous, it follows that there is some neighborhood $V$ of $0$ and some $f$ as above such that
$\hat{f}(\gamma) = \int_G f(x) \inner{-x}{\gamma} dm(x) \neq 0$, showing that $f \mapsto \hat{f}(\gamma)$ is nonzero. 

Now let $h:L^1(G) \to \mathbb{C}$ be a nonzero algebra homomorphism. 
A homomorphism from a Banach algebra to $\mathbb{C}$ is linear functional with norm $1$.
It is a fact that for every bounded linear functional $\Lambda$ on $L^1(G)$ there is some $\phi \in
L^\infty(G)$ such that $\Lambda(f)=\int_G f\phi dm$ for all $f \in L^1(G)$, and $\norm{\phi}_\infty= \norm{\Lambda}$.\footnote{Walter Rudin, {\em Fourier Analysis
on Groups}, p.~268, E10.} Therefore, there is some $\phi \in L^\infty(G)$, $\norm{\phi}_\infty=1$, such that
\[
h(f) = \int_G f \phi dm, \qquad f \in L^1(G).
\]
Then for $f,g \in L^1(G)$ we have
\begin{eqnarray*}
h(f) \int_G g \phi dm&=&h(f) h(g)\\
&=&h(f*g)\\
&=&\int_G (f*g) \phi dm\\
&=&\int_G \left( \int_G f(x-y) g(y) dm(y) \right) \phi(x) dm(x)\\
&=&\int_G g(y) \left(\int_G f(x-y) \phi(x) dm(x) \right) dm(y)\\
&=&\int_G g(y) h(f_y) dm(y),
\end{eqnarray*}
i.e.
\[
\int_G (h(f)\phi(y)-h(f_y))g(y) dm(y) = 0.
\]
Because this is true for all $g \in L^1(G)$, it follows that for all $f \in L^1(G)$ and for almost all $y \in G$, 
\[
h(f)\phi(y)=h(f_y).
\] 
Because $h \neq 0$, there is some $g \in L^1(G)$ with
$h(g) \neq 0$.
Then for almost all $(x,y) \in G \times G$,
\[
h(g)\phi(x+y) = h(g_{x+y}) = h((g_x)_y) = h(g_{x})\phi(y) = h(g)\phi(x) \phi(y),
\]
and as $h(g) \neq 0$, for almost all $(x,y) \in G \times G$,
\[
\phi(x+y)=\phi(x)\phi(y).
\]
We define 
$\phi_1:G \to \mathbb{C}$ by $\phi_1(y)=\frac{h(g_{y})}{h(g)}$, which is continuous because
$y \mapsto g_{y}$ is continuous and $h$ is continuous.
The function $\phi_1$ satisfies $\phi(y)=\phi_1(y)$  for almost all $y \in G$, so for almost all $(x,y) \in G \times G$,
\[
\phi_1(x+y)=\phi_1(x)\phi_1(y).
\]
$(x,y) \mapsto \phi_1(x+y)$ and $(x,y) \mapsto \phi_1(x)\phi_1(y)$ are continuous and the above equality
holds for almost all $(x,y) \in G \times G$, so the above equality in fact holds for all $(x,y) \in G \times G$.
Hence $\phi_1 \in \Gamma$.
Define $\gamma:G \to \mathbb{C}$ by $\gamma(x)=\phi_1(-x)$. Then $\gamma \in \Gamma$, and $\gamma(x)=\phi(-x)$ for almost all $x \in G$, so
for all $f \in L^1(G)$,
\begin{eqnarray*}
\hat{f}(\gamma) &=& \int_G f(x)\inner{-x}{\gamma} dm(x) \\
&=& \int_G f(x) \gamma(-x) dm(x)\\
 &=& \int_G f(x) \phi(x) dm(x)\\
 &=&h(f).
\end{eqnarray*}

Suppose that $\gamma_1,\gamma_2 \in \Gamma$ satisfy $\hat{f}(\gamma_1)=\hat{f}(\gamma_2)$ for all $f \in L^1(G)$. Then
$\inner{-x}{\gamma_1}=\inner{-x}{\gamma_2}$ for almost all $x \in G$, and because these are continuous they are in fact equal for all $x \in G$, i.e.
$\gamma_1=\gamma_2$.
\end{proof}



Let $A(\Gamma)=\{\hat{f}:f \in L^1(G)\}$. Elements of $A(\Gamma)$ are functions $\Gamma \to \mathbb{C}$. So far
$\Gamma$ has not been given  a topology. We shall be interested in the initial topology on $\Gamma$ with respect to
the family of functions $A(\Gamma)$. That is, the topology on $\Gamma$ is the coarsest topology so that each element of
$A(\Gamma)$ is continuous. 
Furthermore, it is a fact that the topology on $\Gamma$ is equal to the subspace topology on $\Gamma$ inherited
from $L^1(G)^*$ with the weak-* topology. Finally,
let the maximal ideal space $\Delta$ of $L^1(G)$ have the subspace topology inherited from
$L^1(G)^*$ with the weak-* topology. In Theorem \ref{maxideal} we presented a bijection 
$\Gamma \to \Delta$, and can be proved that this bijection is a homeomorphism.

It is proved in Rudin\footnote{Walter
Rudin, {\em Fourier Analysis on Groups}, p.~10, \S 1.2.6.} that with this topology, $\Gamma$ is a locally compact abelian group. 
The proof constructs a basis for the topology of $\Gamma$, and because we will use this basis later it will be useful to write out the proof.

\begin{theorem}
\begin{enumerate}
\item $(x,\gamma) \mapsto \inner{x}{\gamma}$ is continuous $G \times \Gamma \to \mathbb{C}$.
\item For $r>0$ let $U_r = \{z \in \mathbb{C}: |1-z|<r\}$. If $K$ is a compact subset of $G$ then
\[
W(K,r) = \{\gamma \in \Gamma: \textrm{$\inner{x}{\gamma} \in U_r$ for all $x \in K$}\}
\]
is an open subset of $\Gamma$, and if $C$ is a compact subset of $\Gamma$ then
\[
V(C,r) = \{x \in G: \textrm{$\inner{x}{\gamma} \in U_r$ for all $\gamma \in C$}\}
\]
is an open subset of $G$.
\item $\{\gamma_0+W(K,r): \textrm{$\gamma_0 \in \Gamma$, $K$ is a compact subset of $\Gamma$, $r>0$}\}$ is a basis for the topology of $\Gamma$.
\item $\Gamma$ is a locally compact abelian group.
\end{enumerate}
\label{topologybasis}
\end{theorem}
\begin{proof}
Let $f \in L^1(G)$. 
For $x_0\in G$, $\gamma_0 \in \Gamma$ and $\epsilon>0$, because $x \mapsto f_x$ is  continuous there is a neighborhood
$V$ of $x_0$ such that $\norm{f_x-f_{x_0}}_1<\epsilon$ for all $x \in V$, and because $\hat{f}_{x_0}$ is continuous there is 
a neighborhood $W$ of $\gamma_0$ such that $|\hat{f}_{x_0}(\gamma)-\hat{f}_{x_0}(\gamma_0)|<\epsilon$ for all $\gamma \in W$. If
$(x,\gamma) \in V \times W$, then
\begin{eqnarray*}
|\hat{f}_x(\gamma)-\hat{f}_{x_0}(\gamma_0)|& \leq& |\hat{f}_x(\gamma)-\hat{f}_{x_0}(\gamma)| + |\hat{f}_{x_0}(\gamma)-\hat{f}_{x_0}(\gamma_0)|\\
&<&\norm{f_x-f_{x_0}}_1 + \epsilon\\
&<&2\epsilon,
\end{eqnarray*}
showing that $(x,\gamma) \mapsto \hat{f}_x(\gamma)$ is continuous. But
\[
\inner{x}{\gamma} \hat{f}(\gamma) = \hat{f}_x(\gamma), \qquad x \in G, \gamma \in \Gamma,
\]
and it follows that $(x,\gamma) \mapsto \inner{x}{\gamma}$ is continuous.

Let $K$ be a compact subset of $G$, $r>0$, and $\gamma_0 \in W(K,r)$. For $x_0 \in K$, 
$|\inner{x_0}{\gamma_0}-1|=\delta<r$. 
Because $(x,\gamma) \mapsto
\inner{x}{\gamma}$ is continuous there is a neighborhood $V_{x_0}$ of $x_0$ and a neighborhood $W_{x_0}$ of $\gamma_0$ such that
$|\inner{x}{\gamma}-\inner{x_0}{\gamma_0}|<r-\delta$ for $(x,\gamma) \in V_{x_0} \times W_{x_0}$, and then
$\inner{x}{\gamma} \in U_r$ for $(x,\gamma) \in V_{x_0} \times W_{x_0}$. As $K \subset \bigcup_{x_0 \in K} V_{x_0}$ and $K$ is compact, there are 
$x_1,\ldots,x_n \in K$ such that $K \subset \bigcup V_{x_i}$. Let $W = \bigcap W_{x_i}$, which is a finite intersection of neighborhoods of 
$\gamma_0$ and hence itself a neighborhood of $\gamma_0$. It is apparent that $W \subset W(K,r)$, showing
that $W(K,r)$ is open.

Let $C$ be a compact subset of $G$, $r>0$, and $x_0 \in V(C,r)$. For $\gamma_0 \in C$, $|\inner{x_0}{\gamma_0}-1|=\delta<r$,
so there is a neighborhood $V_{\gamma_0}$ of $x_0$ and a neighborhood $W_{\gamma_0}$ of
$\gamma_0$ such that $|\inner{x}{\gamma}-\inner{x_0}{\gamma_0}|<r-\delta$ for
$(x,\gamma) \in V_{\gamma_0} \times W_{\gamma_0}$, therefore $\inner{x}{\gamma} \in U_r$ for
$(x,\gamma) \in V_{\gamma_0} \times W_{\gamma_0}$. There are $\gamma_1,\ldots,\gamma_n \in C$ such that
$C \subset \bigcup W_{\gamma_i}$, and  $V = \bigcap V_{\gamma_i}$ is a neighborhood of $x_0$ that is contained in $V(C,r)$, showing that $V(C,r)$ is open.

Let $\gamma_0 \in \Gamma$ and let
 $W$ be a neighborhood of $\gamma_0$. Because $\Gamma$ has the initial topology for $A(\Gamma)$, 
a local subbasis at $\gamma_0$ is given by sets of the form $\{\gamma \in \Gamma: |\hat{f}(\gamma)-\hat{f}(\gamma_0)|<\epsilon\}$, $f \in L^1(G)$ and $\epsilon>0$.
Therefore, there are $f_1,\ldots,f_n \in L^1(G)$ and $\epsilon_1,\ldots,\epsilon_n>0$ such that
\begin{equation}
\bigcap \{\gamma \in \Gamma: |\hat{f}_i(\gamma)-\hat{f}_i(\gamma_0)|<\epsilon_i\} \subset W.
\label{localbasis}
\end{equation}
Let $\epsilon$ be the minimum of the $\epsilon_i$,  let $g_i \in C_0(G)$ with $\norm{f_i-g_i}_1 < \frac{\epsilon}{3}$,  let
$K$ be the union of the supports of $g_i$, and let $M$ be the maximum of $\norm{g_i}_1$. With
$r<\frac{\epsilon}{3M}$, for $\gamma \in \gamma_0+W(K,r)$ we have 
\begin{eqnarray*}
|\hat{f}_i(\gamma)-\hat{f}_i(\gamma_0)|&\leq&|\hat{f}_i(\gamma)-\hat{g}_i(\gamma)| + |\hat{g}_i(\gamma)-\hat{g}_i(\gamma_0)|
+|\hat{g}_i(\gamma_0)-\hat{f}_i(\gamma_0)|\\
&\leq&\norm{f_i-g_i}_1 +  \left| \int_K g_i(x)(\inner{-x}{\gamma}-\inner{-x}{\gamma_0}) dm(x) \right| + \norm{f_i-g_i}_1\\
&<&\frac{2}{3}\epsilon+\int_K |g_i(x)| |1-\inner{-x}{\gamma-\gamma_0}| dm(x)\\
&\leq&\frac{2}{3}\epsilon+r\norm{g_i}_1\\
&<&\epsilon.
\end{eqnarray*}
Thus, if $\gamma \in \gamma_0 + W(K,r)$ then $\gamma$ belongs to the intersection \eqref{localbasis}, and this shows
that $\gamma_0+W(K,r) \subset W$. This establishes that the collection of those sets of the form $\gamma_0+W(K,r)$, for
$\gamma_0 \in \Gamma$, $K$ a compact subset of $\Gamma$, and $r>0$, is a basis for the topology of $\Gamma$. 

Let $\gamma_1,\gamma_2 \in \Gamma$, let $K$ be a compact subset of $\Gamma$, and let $r>0$. Because
\[
(\gamma_1+W(K,r/2))-(\gamma_2+W(K,r/2)) \subset \gamma_1-\gamma_2 +W(K,r),
\]
for $(\gamma',\gamma'') \in (\gamma_1+W(K,r/2)) \times (\gamma_2 + W(K,r/2))$ we have
$\gamma'-\gamma'' \in \gamma_1-\gamma_2 + W(K,r)$, and this shows that $(\gamma',\gamma'') \mapsto
\gamma'-\gamma''$ is continuous, which shows that $\Gamma$ is a topological group. 
$\Gamma$ is locally compact because it is homeomorphic to the maximal ideal space $\Delta$ of $L^1(G)$, and it is a fact that
the maximal ideal space of any commutative Banach algebra is a locally compact Hausdorff space. This completes the proof.
\end{proof}


A fact that Rudin uses but merely asserts is the following (he refers to it being true for the Gelfand transform, and 
then merely asserts its truth there).

\begin{theorem}
$A(\Gamma) \subset C_0(\Gamma)$.
\end{theorem}
\begin{proof}
Let $\Gamma'$ be the weak-* closure of $\Gamma$ in $L^1(G)^*$. For $\Lambda \in \Gamma'$, there is a net
$\gamma_i \in \Gamma$ that weak-* converges to $\Lambda$. For $x,y \in G$,
\[
\Lambda(ab)=\lim_i \gamma_i(ab) = \lim_i \gamma_i(a)\gamma_i(b) = \left( \lim_i \gamma_i(a) \right) \left( \lim_i \gamma_i(b) \right)=
\Lambda(a)\Lambda(b).
\]
Similarly, $\Lambda$ is linear, so $\Lambda$ is an algebra homomorphism $G \to \mathbb{C}$. Hence
taking the closure of $\Gamma$ in $L^1(G)^*$ has added precisely the map that is identically $0$:
\[
\Gamma' = \Gamma \cup \{0\}.
\]
With $K = \{\Lambda \in L^1(G)^*: \norm{\Lambda} \leq 1\}$,  the Banach-Alaoglu theorem tells us that
$K$ is a weak-* compact subset of $L^1(G)^*$. 
It is apparent that $\Gamma' \subset K$, so $\Gamma'$ is a weak-* compact subset of $L^1(G)^*$. 


If $f \in L^1(G)$ and $\epsilon>0$, then because $\hat{f}:\Gamma \to \mathbb{C}$ is continuous,
\[
K_0=\{\gamma \in \Gamma: |\hat{f}(\gamma)| \geq \epsilon\}
\]
is a closed subset of $\Gamma$. The only way $K_0$ would fail to be a weak-* closed subset of $\Gamma'$
is if $0$ belonged to its weak-* closure, and it is straightforward to see that this is not the case. So $K_0$ is in fact
a weak-* closed subset of the weak-* compact set $\Gamma'$, and hence is itself weak-* compact. It follows that $K_0$
is a compact subset of $\Gamma$, and this shows that $\hat{f} \in C_0(\Gamma)$.
\end{proof}

Now that we know that $A(\Gamma) \subset C_0(\Gamma)$, it does not take long to verify
that the conditions of the Stone-Weierstrass theorem are satisfied (for distinct $\gamma_1,\gamma_2 \in \Gamma$
there is some $f \in L^1(G)$ with $\hat{f}(\gamma_1) \neq \hat{f}(\gamma_2)$; $A(\Gamma)$ is self-adjoint;
for each $\gamma \in \Gamma$ there is some $f \in L^1(G)$ with $\hat{f}(\gamma) \neq 0$), and hence
that $A(\Gamma)$ is dense in $C_0(\Gamma)$. 


\begin{theorem}
If $G$ is discrete then $\Gamma$ is compact, and if $G$ is compact then $\Gamma$ is discrete.
\end{theorem}
\begin{proof}
We remarked earlier that the bijection $\Gamma \to \Delta$ is a homeomorphism, where $\Delta$
is the maximal ideal space of $L^1(G)$ and has the subspace topology inherited from $L^1(G)^*$ with the weak-*
topology. If $G$ is discrete, then $L^1(G)$ is unital, and it is a fact that the maximal ideal space of a unital commutative
Banach algebra is compact, so $\Gamma$ is compact.

Suppose that $G$ is compact, with Haar measure $m$ satisfing $m(G)=1$. If $\gamma \in \Gamma$ and there
is some $x_0 \in G$ with $\gamma(x_0) \neq 1$ (i.e. $\gamma$ is not the $0$ homomorphism), then
\[
\int_G \inner{x}{\gamma} dm(x) = \inner{x_0}{\gamma} \int_G \inner{x-x_0}{\gamma} dm(x) = 
\inner{x_0}{\gamma} \int_G \inner{x}{\gamma} dm(x).
\] 
As $\inner{x_0}{\gamma} \neq 1$, this means that $\int_G \inner{x}{\gamma} dm(x) =0$. Therefore,
\[
\int_G \inner{x}{\gamma} dm(x) =\begin{cases}
1&\gamma =0,\\
0&\gamma \neq 0.
\end{cases}
\]
As $G$ is compact, $\chi_G \in L^1(G)$, and 
\[
\hat{\chi}_G(\gamma)=
\int_G  \chi_G(x) \inner{-x}{\gamma} dm(x)
=\int_G \inner{-x}{\gamma} dm(x)
=\begin{cases}
1&\gamma =0,\\
0&\gamma \neq 0.
\end{cases}
\]
$\hat{\chi}_G$ is continuous, so $\{\gamma \in \Gamma: \hat{\chi}_G(\gamma) =0\}$
is a closed subset of $\Gamma$, hence its complement $\{\gamma \in \Gamma: \hat{\chi}_G(\gamma) \neq  0\}$ is an open subset
of $\Gamma$. But by the above this complement is $\{0\}$, and $\Gamma$ is a topological group  so  this implies that
each singleton is open, meaning
that $\Gamma$ is discrete.
\end{proof}



\section{Regular complex Borel measures on $G$}
Let $M(G)$ be the set of regular complex Borel measures on $G$. If $E$ is a Borel set in $G$, define $E_n = \{(x_1,\ldots,x_n): x_1+\cdots+x_n \in E\}$,
which is a Borel set in $G^n$. For $\mu_1,\ldots,\mu_n \in M(G)$, we define
\[
(\mu_1 * \cdots * \mu_n) (E)  = (\mu_1 \times \cdots \times \mu_n)(E_n).
\]
It is proved in Rudin\footnote{Walter Rudin, {\em Fourier Analysis on Groups}, p.~13, \S 1.3.1.} that $\mu_1 * \cdots * \mu_n \in
M(G)$, and that with convolution as multiplication
and norm $\norm{\mu}=|\mu|(G)$ (the
\textbf{total variation norm}),
 $M(G)$ is a commutative Banach algebra with unity $\delta_0$. 

The \textbf{Fourier transform} of $\mu \in M(G)$ is the function $\hat{\mu}:\Gamma \to \mathbb{C}$ defined by
\[
\hat{\mu}(\gamma) = \int_G \inner{-x}{\gamma} d\mu(x), \qquad \gamma \in \Gamma.
\]
We write $B(\Gamma)=\{\hat{\mu}:\mu \in M(G)\}$.
For $f \in L^1(G)$ we have proved that $\hat{f} \in C_0(G)$. We prove now that for $\mu \in M(G)$, $\hat{\mu}$ is bounded and uniformly continuous
on $\Gamma$. However, $\delta_0 \in M(G)$, and for $\gamma \in \Gamma$,
\[
\hat{\delta}_0(\gamma) = \int_G \inner{-x}{\gamma} d\delta_0(x) = \inner{0}{\gamma} = \gamma(0) = 1,
\] 
so $\hat{\delta}_0 \not \in C_0(\Gamma)$. The proof is from Rudin.\footnote{Walter Rudin, {\em Fourier Analysis on Groups},
p.~15, \S 1.3.3.}

\begin{theorem}
If $\mu \in M(G)$, then $\hat{\mu}:\Gamma \to \mathbb{C}$ is bounded and uniformly continuous.
\end{theorem}
\begin{proof}
For any $\gamma \in \Gamma$, 
\[
|\hat{\mu}(\gamma)| \leq \int_G |\inner{-x}{\gamma}| d|\mu|(x) = \int_G d|\mu|(x) = |\mu|(G)<\infty,
\]
where $|\mu|$ is the variation of $\mu$. Hence $\hat{\mu}$ is bounded.

$|\mu|$ is regular, so for any $\delta>0$ there is some compact set $K$ such that $|\mu|(K')<\delta$, where $K'=G \setminus K$.
For $\gamma_1,\gamma_2 \in \Gamma$,  $\inner{x}{\gamma_1-\gamma_2}=\inner{x}{\gamma_1} \inner{x}{\gamma_2}^{-1}$ and hence
\[
|\inner{x}{\gamma_1}-\inner{x}{\gamma_2}| = |1- \inner{x}{\gamma_1-\gamma_2}|,
\]
with which we get
\[
|\hat{\mu}(\gamma_1)-\hat{\mu}(\gamma_2)| \leq \int_G |1- \inner{x}{\gamma_1-\gamma_2}| d|\mu|(x).
\]
We know that $W(K,\delta)$ defined in Theorem \ref{topologybasis} is an open neighborhood of $0$. If $\gamma_1-\gamma_2 \in
W(K,\delta)$, then by the definition of $W(K,\delta)$, for all $x \in K$ we have $|1-\inner{x}{\gamma_1-\gamma_2}|<\delta$, giving
\[
\int_G |1- \inner{x}{\gamma_1-\gamma_2}| d|\mu|(x) \leq \int_K \delta d|\mu|(x) + \int_{K'} 2 d|\mu|(x)<\delta \norm{\mu} + 2\delta,
\]
showing that $\hat{\mu}$ is uniformly continuous.


\end{proof}

If $f \in L^1(G)$, we define $\mu_f \in M(G)$ by
\[
\mu_f(E) = \int_E f(x) dm(x),
\]
for Borel subsets $E$ of $G$, where $m$ is Haar measure on $G$. Thus, $\mu_f$ is absolutely continuous with respect
to $m$ and has density $f$. Then, for $\gamma \in \Gamma$,
\[
\hat{\mu}_f(\gamma) = \int_G \inner{-x}{\gamma} d\mu_f(x) = \int_G \inner{-x}{\gamma} f(x) dm(x) = \hat{f}(\gamma),
\]
showing that $A(\Gamma) \subset B(\Gamma)$. Furthermore, $\norm{f}_1 = \norm{\mu_f}$, hence it makes sense to
identify $L^1(G)$ with its image in $M(G)$ under the map $f \mapsto \mu_f$. 
To talk about a function that belongs to $M(G)$ is to speak about $\mu_f$ for some $f \in L^1(G)$.  It is proved in Rudin
that $L^1(G)$ is a closed ideal in the Banach algebra $M(G)$.\footnote{Walter Rudin, {\em Fourier Analysis on Groups},
p.~16, \S 1.3.4.}

Rudin calls the following theorem the \textbf{uniqueness theorem}.\footnote{Walter
Rudin, {\em Fourier Analysis on Groups}, p.~17, \S 1.3.6.}

\begin{theorem}
If $\mu \in M(\Gamma)$ and 
\[
\int_\Gamma \inner{x}{\gamma} d\mu(\gamma)=0
\]
for all $x \in G$, then $\mu=0$.
\label{uniqueness}
\end{theorem}
\begin{proof}
Let $f \in L^1(G)$.
\begin{eqnarray*}
\int_\Gamma \hat{f}(\gamma) d\mu(\gamma)&=&\int_\Gamma \int_G f(x) \inner{-x}{\gamma} dm(x) d\mu(\gamma)\\
&=&\int_G f(x) \int_\Gamma \inner{-x}{\gamma} d\mu(\gamma) dm(x)\\
&=&0.
\end{eqnarray*}
Because $A(\Gamma)$ is dense in $C_0(\Gamma)$, it follows that for all $\phi \in C_0(\Gamma)$, 
\[
\int_\Gamma \phi d\mu=0,
\]
and this implies that $\mu=0$. 
\end{proof}


\section{Positive-definite functions}
A function $\phi:G \to \mathbb{C}$ is called \textbf{positive-definite} if for every $N$ and every $x_1,\ldots,x_N \in G$, $c_1,\ldots,c_N \in \mathbb{C}$, we have
\begin{equation}
\sum_{n,m=1}^N c_n \overline{c_m} \phi(x_n-x_m) \geq 0.
\label{posdef}
\end{equation}
In particular, the left-hand side of the above inequality is real. 

Let $\phi$ be a character of $G$, which we do not assume to be continuous. Then,
\[
\sum_{n,m=1}^N c_n \overline{c_m} \phi(x_n-x_m) = \sum_{n,m=1}^N c_n \overline{c_m} \phi(x_n) \overline{\phi(x_m)} 
=\left| \sum_{n=1}^N c_n \phi(x_n)  \right|^2 \geq 0,
\]
so any character of $G$, whether or not it is continuous, is positive-definite.

\begin{lemma}
If $\phi:G \to \mathbb{C}$ is positive-definite, then
\[
\phi(0) \geq 0,
\]
\[
\phi(-x)=\overline{\phi(x)}, \quad |\phi(x)| \leq \phi(0), \qquad x \in G,
\]
and
\[
|\phi(x)-\phi(y)|^2 \leq 2\phi(0) \Re(\phi(0)-\phi(x-y)), \qquad x,y \in G.
\]
\label{posdeflemma}
\end{lemma}
\begin{proof}
Take $N=1$, $c_1=1$, $x_1=0$. Then \eqref{posdef} is
\[
\phi(0) \geq 0.
\]

Take $N=2$, $x_1=0, x_2=x, c_1=1,c_2=c$. Then \eqref{posdef} is
\begin{equation}
\phi(0)+\overline{c}\phi(-x)+c\phi(x)+|c|^2 \phi(0) \geq 0.
\label{eq5}
\end{equation}
Because $\phi(0)$ is real, this means that $c\phi(x)+\overline{c}\phi(-x)$ is real, hence is equal to its own complex conjugate.
Writing $\phi(x)=A+iB$ and $\phi(-x)=C+iD$, for $c=1$ this implies that
\[
A+iB + C+iD = A-iB + C - iD
\]
and for $c=i$ this is
\[
iA-B -iC + D = -iA -B+iC +D.
\]
The first equation tells us $B+D=0$, and the second equation tells us $A-C=0$. Thus
\[
\phi(-x) = C+iD = A -iB = \overline{\phi(x)}.
\]

Use \eqref{eq5} with $c$ chosen so that $c\phi(x)=-|\phi(x)|$.  $|c|=1$, and using $\phi(-x)=\overline{\phi(x)}$,
\[
2\phi(0) + 2|\phi(x)| \geq 0.
\]

Take $N=3$, $x_1=0, x_2=x, x_3=y$, $c_1=1$, $\lambda \in \mathbb{R}$,
\[
c_2 = \frac{\lambda |\phi(x)-\phi(y)|}{\phi(x)-\phi(y)},
\]
$c_3=-c_2$; since $\phi(0) \geq 0$, the claim is obviously true for the case $x=y$, which we discard. Then \eqref{posdef} is
\[
(1+2|c_2|^2)\phi(0) +\overline{c_2}(\phi(-x)-\phi(-y))+c_2(\phi(x)-\phi(y))-|c_2|^2( \phi(x-y)+ \phi(y-x)) \geq 0.
\]
Using the definition of $c_2$ and the fact that $\overline{\phi(z)}=\phi(-z)$,
\[
(1+2\lambda^2) \phi(0) +2\lambda  |\phi(x)-\phi(y)| -\lambda^2 ( \phi(x-y)+ \overline{\phi(x-y)}) \geq 0,
\]
or
\[
\lambda^2(2\phi(0)- 2\Re \phi(x-y))  +2\lambda |\phi(x)-\phi(y)| +\phi(0) \geq 0.
\]
The fact that this quadratic polynomial does not take negative values implies that it has either 0 or 1 real roots, and hence that its discriminant
is $\leq 0$:
\[
4|\phi(x)-\phi(y)|^2  - 4(2\phi(0)- 2\Re \phi(x-y)) \phi(0) \leq 0,
\]
which is the claim.
\end{proof}

We remind ourselves that $f^*(x)=\overline{f(-x)}$.

\begin{theorem}
If $f \in L^2(G)$, then $\phi = f* f^*$ is positive-definite and belongs to $C_0(G)$.
\label{selfconv}
\end{theorem}
\begin{proof}
\begin{eqnarray*}
\sum_{n,m=1}^N c_n \overline{c_m} \phi(x_n-x_m)&=&\sum_{n,m=1}^Nc_n\overline{c_m} \int_G f(x_n-x_m - y) \overline{f(-y)} dy\\
&=&\sum_{n,m=1}^Nc_n\overline{c_m} \int_G f(x_n - y) \overline{f(x_m-y)} dy\\
&=&\int_G \left| \sum_{n=1}^N c_n f(x_n-y)\right|^2 dy\\
&\geq&0.
\end{eqnarray*}

It is a fact that if $1<p<\infty$, $\frac{1}{p}+\frac{1}{q}=1$ and $f \in L^p(G)$, $g \in L^q(G)$, then $f*g \in C_0(G)$.\footnote{Walter
Rudin, {\em Fourier Analysis on Groups}, p.~4, \S 1.1.6.}
\end{proof}


The following theorem is from Rudin.\footnote{Walter Rudin, {\em Fourier Analysis on Groups}, p.~19, \S 1.4.2.}

\begin{theorem}
If $\mu \in M(\Gamma)$, $\mu \geq 0$ (i.e. $\mu=|\mu|$), and 
\[
\phi(x)=\int_\Gamma \inner{x}{\gamma} d\mu(\gamma), \qquad x \in G,
\]
then $\phi$ is uniformly continuous and positive-definite.
\end{theorem}
\begin{proof}
\begin{eqnarray*}
\sum_{n,m=1}^N c_n \overline{c_m} \phi(x_n-x_m)&=&\int_\Gamma \sum_{n,m=1}^N c_n \overline{c_m} \inner{x_n-x_m}{\gamma} d\mu(\gamma)\\
&=&\int_\Gamma \sum_{n,m=1}^N c_n \overline{c_m} \inner{x_n}{\gamma} \overline{\inner{x_m}{\gamma}} d\mu(\gamma)\\
&=&\int_\Gamma \left| \sum_{n=1}^N c_n \inner{x_n}{\gamma} \right|^2 d\mu(\gamma)\\
&\geq&0,
\end{eqnarray*}
showing that $\phi$ is positive-definite.

$\mu$ is regular, so for any $\delta>0$ there is a compact set $C$ such that $\mu(C')<\delta$, where $C'=\Gamma \setminus C$. 
We know $V(C,\delta)$ defined in Theorem \ref{topologybasis} is an open neighborhood of $0$. If $x_1-x_2 \in V(C,\delta)$, then 
by the definition of $V(C,\delta)$, for all $\gamma \in C$ we have $|1-\inner{x_1-x_2}{\gamma}|<\delta$, and so
\begin{eqnarray*}
|\phi(x_1)-\phi(x_2)|&\leq&\int_\Gamma |1-\inner{x_1-x_2}{\gamma}| d\mu(\gamma)\\
&=&\int_C |1-\inner{x_1-x_2}{\gamma}| d\mu(\gamma) + \int_{C'} |1-\inner{x_1-x_2}{\gamma}| d\mu(\gamma)\\
&\leq&\int_C \delta d\mu(\gamma) + \int_{C'} 2 d\mu(\gamma)\\
&<& \delta \norm{\mu} + 2\delta.
\end{eqnarray*}
\end{proof}


The above theorem shows that the inverse Fourier transform of a nonnegative measure on $\Gamma$ is a uniformly continuous positive-definite function
on $G$. The following theorem shows that any continuous positive-definite function on $G$ has this form. We remind ourselves
that if a positive-definite function is continuous then it is uniformly continuous, by
Lemma \ref{posdeflemma}. 
We are following the proof given in Rudin.\footnote{Walter Rudin, {\em Fourier Analysis on Groups}, p.~19, \S 1.4.3.}

\begin{theorem}[Bochner's theorem]
If $\phi:G \to \mathbb{C}$ is uniformly continuous and positive-definite, then there is some nonnegative measure $\mu \in M(\Gamma)$ such that
\[
\phi(x) = \int_\Gamma \inner{x}{\gamma} d\mu(\gamma), \qquad x \in G.
\]
\end{theorem}
\begin{proof}
As $\phi$ is positive-definite, $\phi(0)$ is a nonnegative real number, and $|\phi(x)| \leq \phi(0)$ for all $x \in G$. If $\phi(0)=0$ then
use $\mu=0$. Otherwise, it makes sense to divide $\phi$ by $\phi(0)$ and the resulting function is also continuous and positive-definite. 
Thus without loss of generality we suppose that $\phi(0)=1$. 

Using the fact that $\phi$ is positive-definite one shows that for any $f \in C_c(G)$,
$\int_G \int_G f(x) \overline{f(y)} \phi(x-y) dm(x) dm(y)$ is nonnegative
by approximating this integral with finite sums. Then as $C_c(G)$ is dense in $L^1(G)$, the previous integral is nonnegative for any
$f\in L^1(G)$. We define $T_\phi:L^1(G) \to \mathbb{C}$ by
\[
T_\phi(f) = \int_G f \phi dm, \qquad f \in L^1(G),
\]
and define, for $f,g \in L^1(G)$,
\begin{eqnarray*}
[f,g] &=& T_\phi(f* g^*)\\
& =& \int_G (f*g^*)(x) \phi(x) dm(x)\\
&=&\int_G \int_G f(x-y) \overline{g(-y)} \phi(x) dm(y) dm(x)\\
&=&\int_G \int_G f(x) \overline{g(y)} \phi(x-y) dm(x) dm(y).
\end{eqnarray*}
Therefore, $[f,f] \geq 0$, and thus  $[\cdot,\cdot]$ is an inner product on $L^1(G)$ and hence satisfies the Cauchy-Schwarz inequality:
\[
|[f,g]|^2 \leq [f,f] [g,g], \qquad f,g \in L^1(G).
\]

Suppose that $V$ is a symmetric neighborhood of $0$ in $G$ and define $g = \frac{\chi_V}{m(V)}$. 
\begin{eqnarray*}
[f,g]-T_\phi(f)&=&\int_G \int_G f(x) \overline{g(y)} \phi(x-y) dm(x)dm(y) - \int_G f(x) \phi(x) dm(x)\\
&=&\int_G \frac{1}{m(V)} \int_V f(x) \phi(x-y) dm(y)dm(x) \\
&&- \int_G \frac{1}{m(V)} \int_V f(x) \phi(x) dm(y) dm(x)\\
&=&\int_G f(x) \frac{1}{m(V)} \int_V (\phi(x-y)-\phi(x)) dm(y) dm(x)
\end{eqnarray*}
and
\[
[g,g]-1 = \frac{1}{m(V)^2} \int_V \int_V (\phi(x-y)-1) dm(x)dm(y).
\]
Because $\phi$ is uniformly continuous, for any $\delta>0$ there is some $V$ such that both these integrals have absolute value $<\delta$,
and then using the Cauchy-Schwarz inequality we get
\begin{equation}
|T_\phi(f)|^2 \leq [f,f], \qquad f \in L^1(G).
\label{Tphi}
\end{equation}

Let $f \in L^1(G)$ 
and define $h=f*f^*$ and $h^n=h^{n-1}*h$ for $n \geq 2$. $|\phi(x)| \leq \phi(0)=1$ tells us $\norm{\phi}_\infty=1$ and so
$\norm{T_\phi} \leq 1$. In fact, one checks that $\norm{T_\phi}=1$. Applying \eqref{Tphi} to $h, h^2, h^4, \ldots$ we obtain
\[
|T_\phi(f)|^2 \leq T_\phi(f*f^*) = T_\phi(h) \leq (T_\phi (h*h^*))^{1/2} = (T_\phi(h^2))^{1/2} \leq \ldots,
\]
so for any $n \geq 1$, because $\norm{T_\phi} = 1$,
\[
|T_\phi(f)|^2 \leq (T_\phi(h^{2^n}))^{2^{-n}} \leq \norm{h^{2^n}}_1^{2^{-n}}. 
\]
The spectral radius formula tells us that 
\[
\lim_{n \to \infty} \norm{h^{2^n}}_1^{2^{-n}} = \norm{\hat{h}}_\infty.
\]
But $\hat{h} = \hat{f} \widehat{f^*}=|\hat{f}|^2$, so 
$|T_\phi(f)|^2 \leq \norm{ \hat{f}}_\infty^2$, i.e.
\[
|T_\phi(f)| \leq \norm{\hat{f}}_\infty, \qquad f \in L^1(G).
\]
We define $S_\phi$ on $A(\Gamma)$ by $S_\phi(\hat{f}) = T_\phi(f)$; this makes sense because if $\hat{f_1}=\hat{f_2}$ then
$|T_\phi(f_1-f_2)| \leq \norm{\hat{f_1}-\hat{f_2}}_\infty = 0$, so $T_\phi(f_1)=T_\phi(f_2)$. 
The above inequality means that
\[
|S_\phi(\hat{f})| \leq \norm{\hat{f}}_\infty, \qquad \hat{f} \in A(\Gamma).
\]
Therefore $S_\phi$ is a bounded linear functional on $A(\Gamma)$, and because $A(\Gamma)$ is dense in $C_0(\Gamma)$, 
$S_\phi$ can be extended to a bounded linear functional on $C_0(\Gamma)$ with norm $\norm{T_\phi} = 1$. 
But $\Gamma$ is a locally compact Hausdorff space, so by the Riesz representation theorem there is a unique measure
$\mu \in M(\Gamma)$ such that
\[
S_\phi(g) = \int_\Gamma g(-\gamma) d\mu(\gamma), \qquad g \in C_0(\Gamma),
\]
and $\norm{\mu}=\norm{S_\phi} = 1$; we state the above with $g(-\gamma)$ rather than $g(\gamma)$ for later convenience.
For $f \in L^1(G)$, 
\begin{eqnarray*}
T_\phi(f) &=& S_\phi(\hat{f})\\
&=& \int_\Gamma \hat{f}(-\gamma) d\mu(\gamma) \\
&=& \int_\Gamma \int_G f(x) \inner{-x}{-\gamma} dm(x) d\mu(\gamma)\\
&=&\int_G f(x) \left(\int_\Gamma \inner{x}{\gamma} d\mu(\gamma) \right) dm(x).
\end{eqnarray*}
But the definition of $T_\phi$ states
\[
T_\phi(f) = \int_G f(x) \phi(x) dm(x).
\]
Since these two expressions for $T_\phi(f)$ are equal for all $f \in L^1(G)$, we get that
\[
\int_\Gamma \inner{x}{\gamma} d\mu(\gamma)  = \phi(x)
\]
for almost all $x \in G$. Since both sides of the above equality are continuous, they are equal for all $x \in G$. 
For $x=0$, 
\[
1=\phi(0) = \int_\Gamma \inner{0}{\gamma} d\mu(\gamma) = \int_\Gamma d\mu(\gamma) \leq \int_\Gamma d|\mu|(\gamma)
\leq \norm{\mu}=1.
\]
Hence $\int_\Gamma d\mu(\gamma) = \int_\Gamma d|\mu|(\gamma)$, from which it follows that $\mu=|\mu|$, and therefore
$\mu$ is a nonnegative measure.
\end{proof}



\section{The inversion theorem}
Define $B(G)$ to be the set of those $f:G \to \mathbb{C}$ for which there is some $\mu_f \in M(\Gamma)$ such that
\[
f(x) = \int_\Gamma \inner{x}{\gamma} d\mu_f(\gamma), \qquad x \in G.
\]
It is apparent from Theorem \ref{uniqueness} that there is at most one $\mu_f \in M(\Gamma)$ such that the above holds.

The following proof is from Rudin.\footnote{Walter Rudin, {\em Fourier Analysis on Groups}, p.~22, \S 1.5.1.}

\begin{theorem}[Inversion theorem]
If $f \in L^1(G) \cap B(G)$, then $\hat{f} \in L^1(\Gamma)$.

If the Haar measure $m$ on $G$ is fixed, then there is a Haar measure $m_\Gamma$ on $\Gamma$ such that for all $f \in
L^1(G) \cap B(G)$,
\[
f(x) = \int_\Gamma \hat{f}(\gamma) \inner{x}{\gamma} dm_\Gamma(\gamma), \qquad x \in G.
\]
\end{theorem}
\begin{proof}
Write $B^1 = L^1(G) \cap B(G)$. For $f \in B^1$ and $h \in L^1(G)$,
\begin{eqnarray*}
(h*f)(0)&=&\int_G h(x) f(-x) dm(x)\\
&=&\int_G h(x) \int_\Gamma \inner{-x}{\gamma} d\mu_f(\gamma) dm(x)\\
&=&\int_\Gamma \hat{h}(\gamma) d\mu_f(\gamma).
\end{eqnarray*}
For $g \in B^1$, we have $h*g \in L^1(G)$ and $h*f \in L^1(G)$, and so using the above equality,
\[
\int_\Gamma \widehat{h*g} d\mu_f = ((h*g)*f)(0) = ((h*f)*g)(0) = \int_\Gamma \widehat{h*f} d\mu_g,
\]
hence 
\[
\int_\Gamma \hat{h} \hat{g} d\mu_f = \int_\Gamma \hat{h} \hat{f} d\mu_g, \qquad f,g \in B^1, h \in L^1(G).
\]
Because $A(\Gamma)$ is dense in $C_0(\Gamma)$ and the above holds for all $h \in L^1(G)$, it follows that
\begin{equation}
\hat{g} d\mu_f = \hat{f} d\mu_g, \qquad f,g \in B^1.
\label{swappy}
\end{equation}

We define $T:C_c(\Gamma) \to \mathbb{C}$ as follows. Let $\psi \in C_c(\Gamma)$, $K=\supp \psi$. 
For $\gamma_0 \in K$, there is some $\phi \in C_0(\Gamma)$ such that $\phi(\gamma_0) \neq 0$, and because $A(\Gamma)$
is dense in $C_0(\Gamma)$ there is therefore some $f \in L^1(G)$ such that $\hat{f}(\gamma_0) \neq 0$. With
$\delta=|\hat{f}(\gamma_0)|$, there is some $u \in C_c(G)$ with $\norm{u-f}_1<\delta$, and 
\[
|\hat{u}(\gamma_0)-\hat{f}(\gamma_0)| \leq \norm{u-f}_1 < \delta,
\]
which shows that $\hat{u}(\gamma_0) \neq 0$. $\widehat{u*u^*}=|\hat{u}|^2 \geq 0$, so there is some open neighborhood
$U$ of $\gamma_0$ on which $\widehat{u*u^*}$ is positive, as it is a continuous function. Since $K$ is compact, it is covered
by finitely many of these open neighborhoods. Call the corresponding functions $u_1,\ldots,u_n \in C_c(G)$, and write
\[
g=u_1*u_1^* + \cdots + u_n*u_n^*.
\]
$g$ satisfies $\hat{g}(\gamma)>0$ for all $\gamma \in K$. Because each $u_n$ belongs to $C_c(G)$, 
$u_n*u_n^*$ belongs to $C_c(G)$ and so $g \in C_c(G)$. Moreover, by Theorem \ref{selfconv}, each $u_n*u_n^*$ is positive-definite,
and one checks that as $g$ is a linear combination of positive-definite functions with nonnegative coefficients it is itself
positive-definite. Because $g$ is positive-definite, by Bochner's theorem it belongs to $B(G)$, and because $g \in C_c(G)$, $g$
belongs to $L^1(G)$. Hence $g \in B^1$. 
We have now proved that there is at least one element of $B^1$ whose Fourier transform does not vanish on $K$. Suppose that
$f$ is any such function. Then using \eqref{swappy},
\[
\int_\Gamma \frac{\psi}{\hat{f}} d\mu_f = \int_\Gamma \frac{\psi}{\hat{f}\hat{g}} \hat{g} d\mu_f = 
\int_\Gamma \frac{\psi}{\hat{f} \hat{g}} \hat{f} d\mu_g = \int_G \frac{\psi}{\hat{g}} d \mu_g.
\]
Thus, it makes sense to define
\begin{equation}
T\psi = \int_\Gamma \frac{\psi}{\hat{g}} d\mu_g.
\label{Tpsi}
\end{equation}
One checks that $T$ is linear.
Because $g$ is positive-definite, the measure $\mu_g$ supplied by Bochner's theorem is nonnegative, and hence if
$\psi \geq 0$ then $T\psi \geq 0$, namely, $T$ is positive. There are $f \in B^1$ and $\psi \in C_c(G)$ such that
$\int_\Gamma \psi d\mu_f \neq 0$, and $\psi \hat{f} \in C_c(G)$, so there is some $g \in B^1$ satisfying 
\[
T(\psi \hat{f}) = \int_\Gamma \frac{\psi \hat{f}}{\hat{g}} d\mu_g = \int_\Gamma \psi d\mu_f \neq 0,
\]
showing that $T \neq 0$. 

Let $\psi \in C_c(\Gamma)$ and $\gamma_0 \in \Gamma$. There is some $g \in B^1$ such that $\hat{g}$ is positive
on both $K$ and $K+\gamma_0$. For $f(x)=\inner{-x}{\gamma_0} g(x)$, $x \in G$, we have
$\hat{f}(\gamma)=\hat{g}(\gamma+\gamma_0)$, $\gamma \in \Gamma$, and $\mu_f(E) = \mu_g(E-\gamma_0)$. 
For $\psi_0 \in C_c(\Gamma)$ defined by $\psi_0(\gamma)=\psi(\gamma-\gamma_0)$,
\[
T\psi_0 = \int_\Gamma \frac{\psi(\gamma-\gamma_0)}{\hat{g}(\gamma)} d\mu_g(\gamma) = 
\int_\Gamma \frac{\psi(\gamma)}{\hat{f}(\gamma)} d\mu_f(\gamma) =T\psi,
\]
showing that $T$ is translation invariant. Then by the Riesz representation theorem, there is some
nonnegative regular  measure $m_\Gamma$ on $\Gamma$ satisfying
\[
T\psi = \int_\Gamma \psi dm_\Gamma, \qquad \psi \in C_c(\Gamma).
\]
This measure $m_\Gamma$ is translation invariant and not the zero measure because $T$ has these properties,
and this means 
that it is a Haar measure on $\Gamma$. 

For $f \in B^1$,
\[
\int_\Gamma \psi d\mu_f = T(\psi \hat{f}) = \int_\Gamma \psi \hat{f} dm_\Gamma,  \qquad \psi \in C_c(\Gamma),
\]
which implies that
\[
d\mu_f = \hat{f} dm_\Gamma, \qquad f \in B^1.
\]
Because $\norm{\mu_f}<\infty$ (as $\mu_f \in M(\Gamma)$), the above equality implies that $\hat{f} \in L^1(\Gamma)$. 
Moreover,  by the definition of $\mu_f$, for any $x \in G$ we have
\[
f(x) = \int_\Gamma \inner{x}{\gamma} d\mu_f(\gamma) = \int_\Gamma \inner{x}{\gamma} \hat{f} dm_\Gamma(\gamma).
\]

\end{proof}


Using the inversion theorem, we prove the following lemma.\footnote{Walter Rudin, {\em Fourier Analysis on Groups},
p.~23, \S 1.5.2.}

\begin{lemma}
$\{x_0+V(C,r): \textrm{$x_0 \in G$, $C$ is a compact subset of $G$, $r>0$}\}$ is a basis for the topology of $G$.

$\Gamma$ separates points in $G$.
\label{VCr}
\end{lemma}
\begin{proof}
Let $m_\Gamma$ be the Haar measure on $\Gamma$ specified in the inversion theorem.
Suppose that $V$ is a neighborhood of $0$ in $G$. Let $W$ be a compact neighborhood of $0$ in $G$ satisfying $W-W \subset V$. (One proves
that there are such $W$.) Define $f=\frac{\chi_W}{\sqrt{m(W)}}$ and $g=f*f^*$. $g$ is continuous and positive-definite,
and $\supp g \subset  W - W$. Because $g$ is continuous and positive-definite, by Bochner's theorem it belongs to $B(G)$,
and because $\supp g \subset W-W$ it belongs to $L^1(G)$, so we can apply the inversion theorem to get $\hat{g} \in L^1(\Gamma)$ and
\[
\int_\Gamma \hat{g} dm_\Gamma(\gamma) = g(0) = \int_G f(-y) \overline{f(-y)} dm(y) = \int_G |f(y)|^2 dm(y) = 1.
\]
Because $\hat{g}=|\hat{f}|^2 \geq 0$, there is a compact set $C$ in $\Gamma$ such that
\[
\int_C \hat{g}(\gamma) dm_\Gamma(\gamma) > \frac{2}{3}.
\]
To say that $x \in V(C,1/3)$ means $|1-\inner{x}{\gamma}|<\frac{1}{3}$ for all $\gamma \in  C$ and hence $\Re \inner{x}{\gamma}>\frac{2}{3}$,
and satisfies
\[
g(x) = \Re \int_C \hat{g}(\gamma) \inner{x}{\gamma} dm_\Gamma(\gamma) + \Re \int_{C'} \hat{g}(\gamma) \inner{x}{\gamma} dm_\Gamma(\gamma).
\]
The first term is $>\frac{4}{9}$ and the second term has absolute value $<\frac{1}{3}$, so $g(x)>\frac{1}{9}$ for $x \in V(C,1/3)$. 
But $g(x)>\frac{1}{9}$ means that $x \in \supp g = W-W \subset V$, so
\[
V(C,1/3) \subset V,
\]
from which it follows that $V(C,r)$, $C$ compact and $r>0$, is a local basis at $0$.

For $x_0 \in G$, $x_0 \neq 0$, let $V$ be a neighborhood of $0$ that does not include $x_0$,
and the above gives $x \not \in V(C,1/3)$, i.e., there is some $\gamma \in \Gamma$ such that $|1-\inner{x}{\gamma}| \geq \frac{1}{3}$,
and hence $\inner{x}{\gamma} \neq 1$. Therefore, if $x_1 \neq x_2$, there is some $\gamma \in \Gamma$ such that $\inner{x_1-x_2}{\gamma} \neq 1$,
and hence $\inner{x_1}{\gamma} \neq \inner{x_2}{\gamma}$, which is what it means to say that $\Gamma$ separates points in $G$.
\end{proof}


\section{Pontryagin duality theorem}
The dual group $\Gamma$ of $G$ is itself a locally compact abelian group and so has a dual group $\widehat{\Gamma}$.
We proved in Theorem \ref{topologybasis} that $(x,\gamma) \mapsto \inner{x}{\gamma}$ is continuous,
and therefore for any $x \in G$, the function $\alpha(x):\Gamma \to \mathbb{C}$ defined by
\[
\inner{\gamma}{\alpha(x)} = \inner{x}{\gamma}, \qquad \gamma \in \Gamma, 
\]
belongs to $\widehat{\Gamma}$. For $x,y \in G$, $\alpha(xy) \in \widehat{\Gamma}$ satisfies
\[
\inner{\gamma}{\alpha(xy)} = \inner{xy}{\gamma} = \inner{x}{\gamma} \inner{y}{\gamma} = \inner{\gamma}{\alpha(x)}
\inner{\gamma}{\alpha(y)},
\]
showing that $\alpha:G \to \widehat{\Gamma}$ is a homomorphism. The following theorem, proved in Rudin,\footnote{Walter Rudin, {\em Fourier Analysis on Groups},
p.~28, \S 1.7.2.} shows that $\alpha$ is an isomorphism
of topological groups. That is, it states that a locally compact abelian group is isomorphic as a topological group to
its double dual. Let $\textbf{LCA}$ denote the category of locally compact abelian groups, where morphisms are continuous group
homomorphisms. Taking the double dual of an element of \textbf{LCA} is a functor, and it can be proved that there is a natural
isomorphism between the identity functor in \textbf{LCA} and the double dual functor.\footnote{For more, see the nLab page:
\url{http://ncatlab.org/nlab/show/Pontrjagin+dual}}

\begin{theorem}[Pontryagin duality theorem]
$\alpha:G \to \widehat{\Gamma}$ defined by
\[
\inner{\gamma}{\alpha(x)} = \inner{x}{\gamma}, \qquad \gamma \in \Gamma, 
\]
is an isomorphism of topological groups.
\end{theorem}


We proved earlier that if $G$ is discrete then $\Gamma$ is compact and that if $G$ is compact then $\Gamma$ is discrete, but had not
established that if $\Gamma$ is compact then $G$ is discrete or if $\Gamma$ is discrete then $G$ is compact, but we obtain these conclusions
 from
the Pontryagin 
duality theorem: if $\Gamma$ is compact then $\widehat{\Gamma}$ is discrete, and $G$ is isomorphic as a topological group
to $\widehat{\Gamma}$ so $G$ is discrete, and likewise if $\Gamma$ is discrete then $G$ is compact.







\section{Further reading}
Keith Conrad, \url{http://www.math.uconn.edu/~kconrad/blurbs/gradnumthy/characterQ.pdf} works out explicitly the form of all characters of $\mathbb{Q}$, and
shows that the group of all characters of $\mathbb{Q}$ (namely, the dual group of $\mathbb{Q}$ when $\mathbb{Q}$ has the discrete topology), is isomorphic
as a group to the quotient
group
$\mathbb{A}_\mathbb{Q}/\mathbb{Q}$. This gives a satisfying reason for caring about the $p$-adic numbers $\mathbb{Q}_p$ and the adeles $\mathbb{A}_\mathbb{Q}$.

\end{document}