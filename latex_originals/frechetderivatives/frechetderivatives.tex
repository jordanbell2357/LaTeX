\documentclass{article}
\usepackage{amsmath,amssymb,graphicx,subfig,mathrsfs,amsthm}
%\usepackage{tikz-cd}
%\usepackage{hyperref}
\newcommand{\innerL}[2]{\langle #1, #2 \rangle_{L^2}}
\newcommand{\inner}[2]{\left\langle #1, #2 \right\rangle}
\newcommand{\HSinner}[2]{\left\langle #1, #2 \right\rangle_{\ensuremath\mathrm{HS}}}
\newcommand{\tr}{\ensuremath\mathrm{tr}\,} 
\newcommand{\Span}{\ensuremath\mathrm{span}} 
\def\Re{\ensuremath{\mathrm{Re}}\,}
\def\Im{\ensuremath{\mathrm{Im}}\,}
\newcommand{\id}{\ensuremath\mathrm{id}} 
\newcommand{\rank}{\ensuremath\mathrm{rank\,}} 
\newcommand{\point}{\ensuremath\sigma_{\mathrm{point}}} 
\newcommand{\Hom}{\ensuremath\mathrm{Hom}}
\newcommand{\norm}[1]{\left\Vert #1 \right\Vert}
\newtheorem{theorem}{Theorem}
\newtheorem{lemma}[theorem]{Lemma}
\newtheorem{proposition}[theorem]{Proposition}
\newtheorem{corollary}[theorem]{Corollary}
\theoremstyle{definition}
\newtheorem{definition}[theorem]{Definition}
\begin{document}
\title{Fr\'echet derivatives and G\^ateaux derivatives}
\author{Jordan Bell\\ \texttt{jordan.bell@gmail.com}\\Department of Mathematics, University of Toronto}
\date{\today}

\maketitle

\section{Introduction}
In this note all vector spaces are real.
 If $X$ and $Y$ are normed spaces,
we denote by $\mathscr{B}(X,Y)$ the set of bounded linear maps $X \to Y$, and write $\mathscr{B}(X)=\mathscr{B}(X,X)$.
$\mathscr{B}(X,Y)$ is a normed space with the operator norm.


\section{Remainders}
If $X$ and $Y$ are normed spaces, let $o(X,Y)$ be the set of all maps $r:X \to Y$ 
for which there is some map $\alpha:X \to Y$ satisfying:
\begin{itemize}
\item $r(x)=\norm{x}\alpha(x)$ for all $x \in X$,
\item $\alpha(0)=0$,
\item $\alpha$ is continuous at $0$.
\end{itemize}
Following
Penot,\footnote{Jean-Paul Penot, {\em Calculus Without Derivatives}, p.~133, \S 2.4.} we call elements of $o(X,Y)$ {\em remainders}.
It is immediate that $o(X,Y)$ is a vector space.

If $X$ and $Y$ are normed spaces, if $f:X \to Y$ is a function, and if $x_0 \in X$, we say that $f$ is {\em stable at $x_0$} if there is some
$\epsilon>0$ and some $c>0$ such that $\norm{x-x_0} \leq \epsilon$ implies that $\norm{f(x-x_0)} \leq c \norm{x-x_0}$.
If $T:X \to Y$ is a bounded linear map, then $\norm{Tx} \leq \norm{T} \norm{x}$ for all $x \in X$, and thus a bounded linear map
is stable at $0$. The following lemma shows that the composition of  a remainder with a function that is stable at $0$ is a remainder.\footnote{Jean-Paul
Penot, {\em Calculus Without Derivatives}, p.~134, Lemma 2.41.}

\begin{lemma}
Let $X,Y$ be normed spaces and let $r \in o(X,Y)$. 
If $W$ is a normed space and $f:W \to X$ is stable at 0, then $r \circ f \in o(W,Y)$. If $Z$ is a normed space
and $g:Y \to Z$ is stable at 0, then $g \circ r \in o(X,Z)$.
 \label{stable}
\end{lemma}
\begin{proof}
$r \in o(X,Y)$ means that there is some $\alpha:X \to Y$ satisfying  $r(x)=\norm{x}\alpha(x)$ for all
$x \in X$, that takes the value $0$ at $0$, and that is continuous at $0$.
As $f$ is stable at $0$, there is some $\epsilon>0$ and some $c>0$ for which $\norm{w} \leq \epsilon$ implies that
$\norm{f(w)} \leq c \norm{w}$. 
Define $\beta:W \to Y$ by
\[
\beta(w)=\begin{cases}
\frac{\norm{f(w)}}{\norm{w}} \alpha(f(w))&w \neq 0\\
0&w=0,
\end{cases}
\]
for which we have
\[
(r \circ f)(w)=\norm{w}\beta(w), \qquad w \in W.
\]
If 
$\norm{w} \leq \epsilon$, then
$\norm{\beta(w)} \leq c \norm{\alpha(f(w))}$.
But $f(w) \to 0$ as $w \to 0$, and
 because $\alpha$ is continuous at $0$ we get that $\alpha(f(w)) \to \alpha(0)=0$ as $w \to 0$.
So the above inequality gives us $\beta(w) \to 0$ as $w \to 0$. As  $\beta(0)=0$, the function $\beta:W \to Y$ is continuous at $0$, and therefore
 $r \circ f$ is remainder.


As $g$ is stable at $0$, there is some $\epsilon>0$ and some $c>0$ for which $\norm{y} \leq \epsilon$ implies that
$\norm{g(y)} \leq c \norm{y}$. 
Define $\gamma:X \to Z$ by
\[
\gamma(x)
=\begin{cases}
\frac{g(\norm{x}\alpha(x))}{\norm{x}}&x \neq 0\\
0&x=0.
\end{cases}
\]
For all $x \in X$,
\[
(g \circ r)(x) = g(\norm{x}\alpha(x)) = \norm{x} \gamma(x).
\]
Since $\alpha(0) = 0$ and $\alpha$ is continuous at $0$, there is some $\delta>0$ such that
$\norm{x} \leq \delta$ implies that $\norm{\alpha(x)} \leq \epsilon$. Therefore, if
$\norm{x} \leq \delta \wedge 1$ then
\[
\norm{g(\norm{x}\alpha(x))} \leq c \norm{x} \norm{\alpha(x)} \leq c \norm{x} \epsilon,
\]
and hence if $\norm{x} \leq \delta \wedge 1$ then $\norm{\gamma(x)} \leq c\epsilon$. This shows that
$\gamma(x) \to 0$ as $x \to 0$, and since $\gamma(0)=0$  the function $\gamma:X \to Z$ is continuous at $0$, showing that $g \circ r$ is a remainder.
\end{proof}


If $Y_1,\ldots,Y_n$ are normed spaces where $Y_k$ has norm $\norm{\cdot}_k$, then 
$\norm{(y_1,\ldots,y_n)} = \max_{1 \leq k \leq n} \norm{y_k}_k$ is a norm
on $\prod_{k=1}^n Y_k$, and one can prove that the topology induced by this norm is the product
topology.

\begin{lemma}
If $X$ and $Y_1,\ldots,Y_n$ are normed spaces, then a function
$r:X \to \prod_{k=1}^n Y_k$ is a remainder
 if and only if 
each of $r_k:X \to Y_k$ are remainders, $1 \leq k \leq n$, where
 $r(x)=(r_1(x),\ldots,r_n(x))$ for all $x \in X$.
\end{lemma}
\begin{proof}
Suppose that there is some function $\alpha:X \to \prod_{k=1}^n Y_k$ such that
$r(x)=\norm{x} \alpha(x)$ for all $x \in X$. With $\alpha(x)=(\alpha_1(x),\ldots,\alpha_n(x))$, we have
\[
r_k(x)=\norm{x} \alpha_k(x), \qquad x \in X.
\]
Because $\alpha(x) \to 0$ as $x \to 0$, for each $k$ we have $\alpha_k(x) \to 0$ as $x \to 0$, which shows that
$r_k$ is a remainder.

Suppose that each $r_k$ is a remainder. Thus, for each $k$ there is a function $\alpha_k:X \to Y_k$
satisfying $r_k(x)=\norm{x} \alpha_k(x)$ for all $x \in X$ and $\alpha_k(x) \to 0$ as $x \to 0$. Then the function
$\alpha:X \to \prod_{k=1}^n Y_k$ defined by $\alpha(x)=(\alpha_1(x),\ldots,\alpha_n(x))$ satisfies
$r(x)=\norm{x} \alpha(x)$. Because $\alpha_k(x) \to 0$ as $x \to 0$ for each of the finitely many $k$, $1 \leq k \leq n$, 
we have $\alpha(x) \to 0$ as $x \to 0$.
\end{proof}



\section{Definition and uniqueness of Fr\'echet derivative}
Suppose that $X$ and $Y$ are normed spaces, that  $U$ is an open subset of $X$, and  that $x_0 \in U$. A function  $f:U \to Y$
 is said to be {\em Fr\'echet differentiable at $x_0$} if there is some $L \in \mathscr{B}(X,Y)$
and some $r \in o(X,Y)$ such that
\begin{equation}
f(x) = f(x_0) + L(x-x_0)+r(x-x_0), \qquad x \in U.
\label{frechet}
\end{equation}
Suppose there are bounded linear maps $L_1,L_2$ and remainders $r_1,r_2$ that satisfy the above. 
Writing $r_1(x)=\norm{x}\alpha_1(x)$ and $r_2(x)=\norm{x}\alpha_2(x)$ for all $x \in X$, we have
\[
L_1(x-x_0)+\norm{x-x_0}\alpha_1(x-x_0)=L_2(x-x_0)+\norm{x-x_0}\alpha_2(x-x_0), \qquad x \in U,
\]
i.e.,
\[
L_1(x-x_0)-L_2(x-x_0) = \norm{x-x_0}(\alpha_2(x-x_0)-\alpha_1(x-x_0)), \qquad x \in U.
\]
For $x \in X$, there is some $h>0$ such that for all $|t| \leq h$ we have  $x_0+tx \in U$, and then
\[
L_1(tx)-L_2(tx) = \norm{tx}(\alpha_2(tx)-\alpha_1(tx)),
\]
hence, for $0<|t| \leq h$,
\[
L_1(x)-L_2(x) = \norm{x}(\alpha_2(tx)-\alpha_1(tx)).
\]
But $\alpha_2(tx)-\alpha_1(tx) \to 0$ as $t \to 0$, which implies that $L_1(x)-L_2(x)=0$. As this is true for all $x \in X$,
we have $L_1=L_2$ and then $r_1=r_2$. 
If $f$ is Fr\'echet differentiable at $x_0$, 
the bounded linear map $L$ in \eqref{frechet} is called {\em the  Fr\'echet derivative of $f$ at $x_0$}, and we define
$Df(x_0)=L$. Thus, 
\[
f(x)=f(x_0)+Df(x_0)(x-x_0)+r(x-x_0), \qquad x \in U.
\]
If $U_0$ is the set of those points in $U$ at which $f$ is Fr\'echet differentiable, then
$Df:U_0 \to \mathscr{B}(X,Y)$.


Suppose that $X$ and $Y$ are normed spaces and that $U$ is an open subset of $X$.
We denote by $C^1(U,Y)$ the set of functions $f:U \to Y$ that are Fr\'echet differentiable at each point in $U$ and for
which the function $Df:U \to \mathscr{B}(X,Y)$ is continuous. We say that an element of $C^1(U,Y)$ is {\em continuously
differentiable}. We denote by $C^2(U,Y)$ those elements $f$ of $C^1(U,Y)$ such that
\[
Df \in C^1(U,\mathscr{B}(X,Y));
\]
that is, $C^2(U,Y)$ are those $f \in C^1(U,Y)$ such that the function $Df:U \to \mathscr{B}(X,Y)$ is Fr\'echet differentiable at each point in $U$ and such that 
the function
\[
D(Df):U \to \mathscr{B}(X,\mathscr{B}(X,Y))
\]
is continuous.\footnote{See Henri Cartan, {\em Differential Calculus}, p.~58, \S 5.1, and Jean Dieudonn\'e, {\em Foundations of Modern Analysis},
enlarged and corrected printing, p.~179, Chapter VIII, \S 12.}


The following theorem characterizes continuously differentiable functions $\mathbb{R}^n \to \mathbb{R}^m$.\footnote{Henri
Cartan, {\em Differential Calculus}, p.~36, \S 2.7.}

\begin{theorem}
Suppose that $f:\mathbb{R}^n \to \mathbb{R}^m$ is Fr\'echet differentiable at each point in $\mathbb{R}^n$, and write
\[
f=(f_1,\ldots,f_m).
\]
$f \in C^1(\mathbb{R}^n,\mathbb{R}^m)$ if and only if for each  $1 \leq i \leq m$ and
$1 \leq j \leq n$ the function
\[
\frac{\partial f_i}{\partial x_j}:\mathbb{R}^n \to \mathbb{R} 
\]
is continuous.
\end{theorem}


\section{Properties of the Fr\'echet derivative}
\label{properties}
If $f:X \to Y$ is Fr\'echet differentiable at $x_0$, then because a bounded linear map is continuous and in particular
continuous at $0$, and because a remainder is continuous at $0$, we get that $f$ is continuous
at $x_0$.

We now prove that Fr\'echet differentiation at a point is linear.

\begin{lemma}[Linearity]
Let $X$ and $Y$ be normed spaces, let $U$ be an open subset of $X$ and let $x_0 \in U$.  If $f_1,f_2:U \to Y$  are both
Fr\'echet differentiable at $x_0$ and if $\alpha \in \mathbb{R}$,
then $\alpha f_1+f_2$ is Fr\'echet differentiable at $x_0$ and
\[
D(\alpha f_1+f_2)(x_0)=\alpha Df_1(x_0)+Df_2(x_0).
\]
\end{lemma}
\begin{proof}
There are remainders $r_1,r_2 \in o(X,Y)$ such that 
\[
f_1(x)=f_1(x_0)+Df_1(x_0)(x-x_0)+r_1(x-x_0), \qquad x \in U,
\]
and
\[
f_2(x)=f_2(x_0)+Df_2(x_0)(x-x_0)+r_2(x-x_0), \qquad x \in U.
\]
Then for all $x \in U$,
\begin{eqnarray*}
(\alpha f_1+f_2)(x)-(\alpha f_1+f_2)(x_0)&=&\alpha f_1(x)-\alpha f_1(x_0)+f_2(x)-f_2(x_0)\\
&=&\alpha Df_1(x_0)(x-x_0)+ \alpha r_1(x-x_0)\\
&&+Df_2(x_0)(x-x_0)+r_2(x-x_0)\\
&=&(\alpha Df_1(x_0)+Df_2(x_0))(x-x_0)\\
&&+(\alpha r_1+r_2)(x-x_0),
\end{eqnarray*}
and $\alpha r_1+r_2 \in o(X,Y)$.
\end{proof}


The following lemma gives an alternate characterization of a function being Fr\'echet differentiable
at a point.\footnote{Jean-Paul Penot, {\em Calculus Without Derivatives}, p.~136, Lemma 2.46.}

\begin{lemma}
Suppose that $X$ and $Y$ are normed space, that $U$ is an open subset of $X$, and that $x_0 \in U$. A function
$f:U \to Y$ is Fr\'echet differentiable at $x_0$ if and only if there is some function
$F:U \to \mathscr{B}(X,Y)$ that is continuous at $x_0$ and for which
\[
f(x)-f(x_0)=F(x)(x-x_0), \qquad x \in U.
\]
\end{lemma}
\begin{proof}
Suppose that there is a function $F:U \to \mathscr{B}(X,Y)$ that is continuous at $x_0$ and that satisfies
$f(x)-f(x_0)=F(x)(x-x_0)$ for all $x \in U$. Then, for $x \in U$,
\begin{eqnarray*}
f(x)-f(x_0)&=&F(x)(x-x_0)-F(x_0)(x-x_0)+F(x_0)(x-x_0)\\
&=&F(x_0)(x-x_0)+r(x-x_0),
\end{eqnarray*}
where   $r:X \to Y$ is defined by
\[
r(x)=\begin{cases}
(F(x+x_0)-F(x_0))(x)&x+x_0 \in U\\
0&x+x_0 \not \in U.
\end{cases}
\]
We further define
\[
\alpha(x) = \begin{cases}
\frac{(F(x+x_0)-F(x_0))(x)}{\norm{x}}&x+x_0 \in U, x \neq 0\\
0&x+x_0 \not \in U\\
0&x=0,
\end{cases}
\]
with which $r(x)=\norm{x}\alpha(x)$ for all $x \in X$. To prove that $r$ is a remainder it suffices to prove that
$\alpha(x) \to 0$ as $x \to 0$. Let $\epsilon>0$.
That $F:U \to \mathscr{B}(X,Y)$ is continuous at $x_0$ tells us that there is some $\delta>0$ for which
$\norm{x}<\delta$ implies that $\norm{F(x+x_0)-F(x_0)}<\epsilon$ and hence 
\[
\norm{(F(x+x_0)-F(x_0))(x)} \leq \norm{F(x+x_0)-F(x_0)} \norm{x} < \epsilon \norm{x}.
\]
Therefore, if $\norm{x}<\delta$ then
$\norm{\alpha(x)} < \epsilon$, which establishes that $r$ is a remainder and therefore that
$f$ is Fr\'echet differentiable at $x_0$, with Fr\'echet derivative $Df(x_0)=F(x_0)$.

Suppose that $f$ is Fr\'echet differentiable at $x_0$: there is some $r \in o(X,Y)$ such that
\[
f(x)=f(x_0)+Df(x_0)(x-x_0)+r(x-x_0), \qquad x \in U,
\]
where $Df(x_0) \in \mathscr{B}(X,Y)$. As $r$ is a remainder, there is some
$\alpha:X \to Y$ satisfying $r(x)=\norm{x}\alpha(x)$ for all $x \in X$, and such that $\alpha(0)=0$ and $\alpha(x) \to 0$ 
as $x \to 0$.
For each $x \in X$, by the Hahn-Banach extension
theorem\footnote{Walter Rudin, {\em Functional Analysis}, second ed., p.~59, Corollary to Theorem 3.3.}
there is some $\lambda_x \in X^*$ such that $\lambda_x x = \norm{x}$ and $|\lambda_x v| \leq \norm{v}$ for all $v \in X$.
Thus,
\[
r(x)=(\lambda_x x)\alpha(x), \qquad x \in X.
\]
Define $F:U \to \mathscr{B}(X,Y)$ by
\[
F(x)=Df(x_0)+(\lambda_{x-x_0})\alpha(x-x_0),
\]
i.e. for $x \in U$ and $v \in X$,
\[
F(x)(v)=Df(x_0)(v)+(\lambda_{x-x_0} v) \alpha(x-x_0) \in Y.
\]
Then for $x \in U$,
\[
r(x-x_0)=(\lambda_{x-x_0}(x-x_0))\alpha(x-x_0)=F(x)(x-x_0)-Df(x_0)(x-x_0),
\]
and hence
\[
f(x)=f(x_0)+F(x)(x-x_0), \qquad x \in U.
\]
To complete the proof  it suffices to prove that $F$ is continuous at $x_0$. But both $\lambda_0=0$ and $\alpha(0)=0$ so
$F(x_0)=Df(x_0)$, and for  $x \in U$ and $v \in X$,
\begin{eqnarray*}
\norm{(F(x)-F(x_0))(v)}&=&\norm{(\lambda_{x-x_0}v)\alpha(x-x_0)}\\
&=&|\lambda_{x-x_0}v| \norm{\alpha(x-x_0)}\\
&\leq& \norm{v} \norm{\alpha(x-x_0)},
\end{eqnarray*}
so $\norm{F(x)-F(x_0)} \leq \norm{\alpha(x-x_0)}$. From this and the fact that $\alpha(0)=0$ and $\alpha(x) \to 0$ as $x \to 0$
we get that $F$ is continuous at $x_0$, completing the proof.
\end{proof}

We now prove the chain rule for Fr\'echet derivatives.\footnote{Jean-Paul Penot,
{\em Calculus Without Derivatives}, p.~136, Theorem 2.47.}

\begin{theorem}[Chain rule]
Suppose that $X,Y,Z$ are normed spaces and that $U$ and $V$ are open subsets of $X$ and $Y$ respectively.
If $f:U \to Y$ satisfies $f(U) \subseteq V$ and is Fr\'echet differentiable at $x_0$ and if  $g:V \to Z$ is Fr\'echet differentiable at $f(x_0)$, then
$g \circ f:U \to Z$ is Fr\'echet differentiable at $x_0$, and its Fr\'echet derivative at $x_0$ is
\[
D(g \circ f)(x_0) = Dg(f(x_0)) \circ Df(x_0).
\]
\label{chainrule}
\end{theorem}
\begin{proof}
Write $y_0=f(x_0)$,  $L_1=Df(x_0)$, and $L_2=Dg(y_0)$. Because $f$ is
Fr\'echet differentiable at $x_0$, there is some $r_1 \in o(X,Y)$
such that
\[
f(x)=f(x_0)+L_1(x-x_0) +r_1(x-x_0), \qquad x\in U,
\]
and because $g$ is Fr\'echet differentiable at $y_0$ there is some $r_2 \in o(Y,Z)$ such that
\[
g(y)=g(y_0)+L_2(y-y_0)+r_2(y-y_0), \qquad y \in V.
\]
For all $x \in U$ we have $f(x) \in V$, and using the above formulas,
\begin{align*}
g(f(x)) &= g(y_0) + L_2(f(x)-y_0)+r_2(f(x)-y_0)\\
&=g(y_0)+L_2\Big(L_1(x-x_0)+r_1(x-x_0)\Big)+r_2\Big(L_1(x-x_0)+r_1(x-x_0)\Big)\\
&=g(y_0)+L_2(L_1(x-x_0)) + L_2(r_1(x-x_0))  +r_2\Big(L_1(x-x_0)+r_1(x-x_0)\Big).
\end{align*}
Define $r_3:X \to Z$ by $r_3(x)=r_2(L_1 x + r_1(x))$, and fix any $c>\norm{L_1}$.
Writing $r_1(x)=\norm{x}\alpha_1(x)$, the fact that $\alpha(0)=0$ and that
$\alpha$ is continuous at $0$ gives us that
there is some $\delta>0$ such that if $\norm{x} < \delta$ then $\norm{\alpha(x)} < c-\norm{L_1}$, and hence if
$\norm{x} < \delta$ then $\norm{r_1(x)} \leq (c-\norm{L_1})\norm{x}$. Then, $\norm{x}<\delta$ implies that
\[
\norm{L_1x + r_1(x)} \leq \norm{L_1x}+\norm{r_1(x)} \leq \norm{L_1}\norm{x}+(c-\norm{L_1})\norm{x}=c\norm{x}.
\]
This shows that $x \mapsto L_1x + r_1(x)$ is stable at $0$ and so by Lemma \ref{stable} that $r_3 \in o(X,Z)$. 
Then, $r:X \to Z$ defined by $r=L_1 \circ r_1+r_3$ is a sum of two remainders and so is itself a remainder, and we have
\[
g \circ f(x) = g \circ f(x_0)+L_2 \circ L_1 (x-x_0)+r(x-x_0), \qquad x \in U.
\]
But $L_1 \in \mathscr{B}(X,Y)$ and $L_2 \in \mathscr{B}(Y,Z)$, so $L_2 \circ L_1 \in \mathscr{B}(X,Z)$. This shows
that
$g \circ f$ is Fr\'echet differentiable at $x_0$ and that its Fr\'echet derivative at $x_0$ is
\[
L_2 \circ L_1=Dg(y_0) \circ Df(x_0) = Dg(f(x_0)) \circ Df(x_0).
\]
\end{proof}


The following is the product rule for Fr\'echet derivatives. By $f_1 \cdot f_2$ we mean the function $x \mapsto f_1(x)f_2(x)$.

\begin{theorem}[Product rule]
Suppose that $X$ is a normed space, that $U$ is an open subset of $X$, that
$f_1,f_2:U \to \mathbb{R}$ are functions, and that $x_0 \in U$.
If $f_1$ and $f_2$ are both Fr\'echet differentiable at $x_0$, then $f_1\cdot f_2$ is Fr\'echet differentiable at $x_0$,
and its Fr\'echet derivative at $x_0$ is
\[
D(f_1\cdot f_2)(x_0)=f_2(x_0)Df_1(x_0)+f_1(x_0)Df_2(x_0).
\] 
\end{theorem}
\begin{proof}
There are $r_1,r_2 \in o(X,\mathbb{R})$ with which
\[
f_1(x)=f_1(x_0)+Df_1(x_0)(x-x_0)+r_1(x-x_0), \qquad x \in U
\]
and
\[
f_2(x)=f_2(x_0)+Df_2(x_0)(x-x_0)+r_2(x-x_0), \qquad x \in U.
\]
Multiplying the above two formulas, 
\begin{eqnarray*}
f_1(x)f_2(x)&=&f_1(x_0)f_2(x_0)+f_2(x_0)Df_1(x_0)(x-x_0)+f_1(x_0)Df_2(x_0)(x-x_0)\\
&&+Df_1(x_0)(x-x_0) Df_2(x_0)(x-x_0)+r_1(x-x_0)r_2(x-x_0)\\
&&+f_1(x_0)r_2(x-x_0)+r_2(x-x_0) Df_1(x_0)(x-x_0)\\
&&+f_2(x_0)r_1(x-x_0)+r_1(x-x_0)Df_2(x_0)(x-x_0).
\end{eqnarray*}
Define $r:X \to \mathbb{R}$ by
\begin{eqnarray*}
r(x)&=&Df_1(x_0)x Df_2(x_0)x + r_1(x)r_2(x)+f_1(x_0)r_2(x)+r_2(x)Df_1(x_0)x\\
&&+f_2(x_0)r_1(x)+r_1(x)Df_2(x_0)x,
\end{eqnarray*}
for which we have, for $x \in U$,
\[
f_1(x)f_2(x) = f_1(x_0)f_2(x_0)+f_2(x_0)Df_1(x_0)(x-x_0)+f_1(x_0)Df_2(x_0)(x-x_0) + r(x-x_0).
\]
Therefore, to prove the claim it suffices to prove that $r \in o(X,\mathbb{R})$.
Define $\alpha:X \to \mathbb{R}$ by $\alpha(0)=0$ and $\alpha(x)=\frac{Df_1(x_0)x Df_2(x_0)x}{\norm{x}}$ for $x \neq 0$.
For $x \neq 0$,
\begin{eqnarray*}
|\alpha(x)| &=& \frac{|Df_1(x_0)x| |Df_2(x_0)x|}{\norm{x}}\\
& \leq& \frac{\norm{Df_1(x_0)} \norm{x} \norm{Df_2(x_0)} \norm{x}}{\norm{x}}\\
&=&\norm{Df_1(x_0)} \norm{Df_2(x_0)} \norm{x}.
\end{eqnarray*}
Thus $\alpha(x) \to 0$ as $x \to 0$, showing that the first term in the expression for $r$ belongs to $o(X,\mathbb{R})$. Likewise,
each of the other five terms in the expression for $r$ belongs to $o(X,\mathbb{R})$, and hence $r \in o(X,\mathbb{R})$, completing
the proof.
\end{proof}





\section{Dual spaces}
If $X$ is a normed space, we denote by $X^*$  the set of bounded linear maps $X \to \mathbb{R}$, i.e. $X^*=\mathscr{B}(X,\mathbb{R})$. $X^*$ is itself a normed
space with the operator norm.
 If $X$ is a normed space, the {\em dual pairing} $\inner{\cdot}{\cdot}:X \times X^* \to \mathbb{R}$ is 
 \[
 \inner{x}{\psi}=\psi(x), \qquad x \in X, \psi \in X^*.
 \]


If $U$ is an open subset of $X$ and if a function $f:U \to \mathbb{R}$ is Fr\'echet differentiable at $x_0 \in U$, then
$Df(x_0)$ is a bounded linear map $X \to \mathbb{R}$, and so belongs to $X^*$. 
If $U_0$ are those points in $U$ at which $f:U \to \mathbb{R}$ is Fr\'echet differentiable, then
\[
Df:U_0 \to X^*.
\]

In the case that $X$ is a Hilbert space with inner product $\inner{\cdot}{\cdot}$,
 the Riesz representation theorem shows that
  $R:X \to X^*$ defined by  $R(x)(y)=\inner{y}{x}$ is an isometric isomorphism.
If $f:U \to \mathbb{R}$ is Fr\'echet differentiable at $x_0 \in U$, then we define
\[
\nabla f(x_0) = R^{-1}(Df(x_0)),
\]
and call $\nabla f(x_0) \in X$ the {\em gradient of $f$ at $x_0$}. 
With $U_0$ denoting the set of those points in $U$ at which $f$ is Fr\'echet differentiable,
\[
\nabla f:U_0 \to X.
\]
(To define the gradient we merely used that $R$ is a bijection,
but to prove properties of the gradient one uses that $R$ is an isometric isomorphism.)


\textbf{Example.} Let $X$ be a  Hilbert space, $A \in \mathscr{B}(X)$, $v \in X$, and define 
\[
f(x)=\inner{Ax}{x}-\inner{x}{v}, \qquad x \in X.
\]
For all $x_0,x \in X$ we have, because the inner product of a real Hilbert space is symmetric,
\begin{eqnarray*}
f(x)-f(x_0)&=&\inner{Ax}{x}-\inner{x}{v}-\inner{Ax_0}{x_0}+\inner{x_0}{v}\\
&=&\inner{Ax}{x}-\inner{Ax_0}{x}+\inner{Ax_0}{x} -\inner{Ax_0}{x_0}-\inner{x-x_0}{v}\\
&=&\inner{A(x-x_0)}{x}+\inner{Ax_0}{x-x_0} - \inner{x-x_0}{v}\\
&=&\inner{x-x_0}{A^*x}+\inner{x-x_0}{Ax_0} -\inner{x-x_0}{v}\\
&=&\inner{x-x_0}{A^*x+Ax_0-v}\\
&=&\inner{x-x_0}{A^*x-A^*x_0+A^*x_0+Ax_0-v}\\
&=&\inner{x-x_0}{(A^*+A)x_0-v}+ \inner{x-x_0}{A^*(x-x_0)}.
\end{eqnarray*}
With $Df(x_0)(x-x_0)=\inner{x-x_0}{(A^*+A)x_0-v}$, or $Df(x_0)(x)=\inner{x}{(A^*+A)x_0-v}$,  
we have that $f$ is Fr\'echet differentiable at each $x_0 \in X$.
Furthermore, its gradient at $x_0$ is
\[
\nabla f(x_0)=(A^*+A)x_0-v.
\]

For each $x_0 \in X$, the function $f:X \to \mathbb{R}$ is Fr\'echet differentiable at $x_0$, and thus
\[
Df:X \to X^*,
\]
and we can ask at what points $Df$ has a Fr\'echet derivative. 
For $x_0,x,y \in X$,
\begin{align*}
(Df(x)-Df(x_0))(y)=&\inner{y}{(A^*+A)x-v}-\inner{y}{(A^*+A)x_0-v}\\
=&\inner{y}{(A^*+A)(x-x_0)}.
\end{align*}
For $D(Df)(x_0)(x-x_0)(y)=\inner{y}{(A^*+A)(x-x_0)}$, in other words with
\[
D^2f(x_0)(x)(y)=D(Df)(x_0)(x)(y)=\inner{y}{(A^*+A)x},
\] 
we have that $Df$ is Fr\'echet differentiable at each $x_0 \in X$.
Thus
\[
D^2 f:X \to \mathscr{B}(X,X^*).
\]
Because $D^2 f(x_0)$ does not depend on $x_0$,  it is  Fr\'echet differentiable at each point in $X$, with
$D^3f(x_0)=0$ for all $x_0 \in X$. Here $D^3f:X \to \mathscr{B}(X,\mathscr{B}(X,X^*))$.



\section{G\^ateaux derivatives}
Let $X$ and $Y$ be normed spaces, let $U$ be an open subset of $X$, let $f:U \to Y$ be a function, and let $x_0 \in U$.
If there is some $T  \in \mathscr{B}(X,Y)$ such that for all $v \in X$ we have
\begin{equation}
\lim_{t \to 0} \frac{f(x_0+tv)-f(x_0)}{t}=T v,
\label{gateaux}
\end{equation}
then we say that $f$ is {\em G\^ateaux differentiable at $x_0$} and call $T$ the {\em G\^ateaux derivative of
$f$ at $x_0$}.\footnote{Our definition of the G\^ateaux derivative follows Jean-Paul Penot,
{\em Calculus Without Derivatives}, p.~127, Definition 2.23.} It is apparent that there is at most one
$T \in \mathscr{B}(X,Y)$ that satisfies \eqref{gateaux} for all $v \in X$. We write $f'(x_0)=T$. Thus, $f'$ is a map from the set
of points in $U$ at which $f$ is G\^ateaux differentiable to $\mathscr{B}(X,Y)$. If $V \subseteq U$ and $f$ is G\^ateaux differentiable
at each element of $V$, we say that $f$ is {\em G\^ateaux differentiable on $V$}.


\textbf{Example.} Define $f:\mathbb{R}^2 \to \mathbb{R}$ by
$f(x_1,x_2)=\frac{x_1^4x_2}{x_1^6+x_2^3}$ for $(x_1,x_2) \neq (0,0)$ and
$f(0,0)=0$. For $v=(v_1,v_2) \in \mathbb{R}^2$ and $t \neq 0$,
\[
\frac{f(0+tv)-f(0)}{t}=\frac{f(tv_1,tv_2)}{t}=
\begin{cases}
\frac{1}{t} \cdot \frac{t^5v_1^4v_2}{t^6v_1^6+t^3v_2^3}&v \neq (0,0)\\
0&v=(0,0)
\end{cases}
=\begin{cases}
\frac{tv_1^4v_2}{t^3v_1^6+v_2^3}&v \neq (0,0)\\
0&v=(0,0).
\end{cases}
\]
Hence, for any $v \in \mathbb{R}^2$, we have $\frac{f(0+tv)-f(0)}{t} \to 0$ as $t \to 0$. Therefore, 
$f$ is G\^ateaux differentiable at $(0,0)$ and $f'(0,0)v = 0 \in \mathbb{R}$ for all $v \in \mathbb{R}^2$, i.e. $f'(0,0)=0$.
However, for $(x_1,x_2) \neq (0,0)$,
\[
f(x_1,x_1^2)=\frac{x_1^6}{x_1^6+x_1^6}=\frac{1}{2},
\]
from which it follows that $f$ is not continuous at $(0,0)$. We stated in \S \ref{properties} that if a function is Fr\'echet differentiable
at a point then it is continuous at that point, and so $f$ is not Fr\'echet differentiable at $(0,0)$. Thus, a function that is 
G\^ateaux differentiable at a point need not be Fr\'echet differentiable at that point.

We prove that being Fr\'echet differentiable at a point implies being G\^ateaux differentiable at the point, and that in this case the
G\^ateaux derivative is equal to the Fr\'echet derivative.

\begin{theorem}
Suppose that $X$ and $Y$ are normed spaces, that $U$ is an open subset of $X$, that $f \in Y^U$, and that
$x_0 \in U$. If $f$ is Fr\'echet differentiable at $x_0$, then $f$ is G\^ateaux differentiable at $x_0$ and $f'(x_0)=Df(x_0)$.
\end{theorem}
\begin{proof}
Because $f$ is Fr\'echet differentiable at $x_0$, there is some $r \in o(X,Y)$ for which 
\[
f(x)=f(x_0)+Df(x_0)(x-x_0)+r(x-x_0), \qquad x \in U.
\]
For $v \in X$ and nonzero $t$ small enough that $x_0+tv \in U$,
\[
\frac{f(x_0+tv)-f(x_0)}{t} = \frac{Df(x_0)(x_0+tv-x_0)+r(x_0+tv-x_0)}{t} = \frac{tDf(x_0)v+r(tv)}{t}.
\]
Writing $r(x)=\norm{x}\alpha(x)$,
\[
\frac{f(x_0+tv)-f(x_0)}{t}  = \frac{tDf(x_0)+\norm{tv}\alpha(tv)}{t}= Df(x_0)v + \norm{v}\alpha(tv).
\]
Hence, 
\[
\lim_{t \to 0} \frac{f(x_0+tv)-f(x_0)}{t} = Df(x_0)v.
\]
This holds for all $v \in X$, and as $Df(x_0) \in \mathscr{B}(X,Y)$ we get that $f$ is G\^ateaux differentiable at $x_0$ and that
$f'(x_0)=Df(x_0)$.
\end{proof}



If $X$ is a vector space and $u,v \in X$, let
\[
[u,v] = \{(1-t)u+tv: 0 \leq t \leq 1\},
\]
namely, the line segment joining $u$ and $v$. 
The following is a mean value theorem for G\^ateaux derivatives.\footnote{Antonio Ambrosetti and
Giovanni Prodi, {\em A Primer of Nonlinear Analysis}, p.~13, Theorem 1.8.}

\begin{theorem}[Mean value theorem]
Let $X$ and $Y$  be  normed spaces, let $U$ be an open subset of $X$, and let $f:U \to Y$ be G\^ateaux differentiable on
$U$. If $u,v \in U$ and $[u,v] \subset U$, then
\[
\norm{f(u)-f(v)} \leq \sup_{w \in [u,v]} \norm{f'(w)} \cdot \norm{u-v}.
\]
\label{MVT}
\end{theorem}
\begin{proof}
If $f(u)=f(v)$ then immediately the claim is true. Otherwise, $f(v)-f(u) \neq 0$, and so by the Hahn-Banach extension theorem\footnote{Walter
Rudin, {\em Functional Analysis}, second ed., p.~59, Corollary.} there
is some $\psi \in Y^*$ satisfying $\psi(f(v)-f(u)) = \norm{f(v)-f(u)}$ and $\norm{\psi}=1$. 
Define $h:[0,1] \to \mathbb{R}$ by
\[
h(t)=\inner{f((1-t)u+tv)}{\psi}.
\]
For $0<t<1$ and $\tau \neq 0$ satisfying $t+\tau \in [0,1]$, we have
\begin{eqnarray*}
\frac{h(t+\tau)-h(t)}{\tau}&=&\frac{1}{\tau}\inner{f((1-t-\tau)u+(t+\tau)v)}{\psi}-\frac{1}{\tau}\inner{f((1-t)u+tv)}{\psi}\\
&=&\inner{\frac{f((1-t)u+tv+(v-u)\tau)-f((1-t)u+tv)}{\tau}}{\psi}.
\end{eqnarray*}
Because $f$ is G\^ateaux differentiable at $(1-t)u+tv$, 
\[
\lim_{\tau \to 0} \frac{f((1-t)u+tv+(v-u)\tau)-f((1-t)u+tv)}{\tau} = f'((1-t)u+tv)(v-u),
\]
so because $\psi$ is continuous,
\[
\lim_{\tau \to 0} \frac{h(t+\tau)-h(t)}{\tau} = \inner{f'((1-t)u+tv)(v-u)}{\psi},
\]
which shows that $h$ is differentiable at $t$ and that
\[
h'(t)=\inner{f'((1-t)u+tv)(v-u)}{\psi}.
\]
$h:[0,1] \to \mathbb{R}$ is a composition of continuous functions so it is continuous. Applying the mean value theorem, there is some
$\theta$, $0<\theta<1$, for which
\[
h'(\theta)=h(1)-h(0).
\]
On the one hand,
\[
h'(\theta)=\inner{f'((1-\theta)u+\theta v)(v-u)}{\psi}.
\]
On the other hand,
\[
h(1)-h(0)=\inner{f(v)}{\psi}-\inner{f(u)}{\psi}=\inner{f(v)-f(u)}{\psi}=\norm{f(v)-f(u)}.
\]
Therefore
\begin{eqnarray*}
\norm{f(v)-f(u)} &=& |\inner{f'((1-\theta)u+\theta v)(v-u)}{\psi}|\\
&\leq& \norm{\psi} \norm{f'((1-\theta)u+\theta v)(v-u)}\\
&=&\norm{f'((1-\theta)u+\theta v)(v-u)}\\
&\leq&\norm{f'((1-\theta)u+\theta v)}\norm{v-u}\\
&\leq&\sup_{w \in [u,v]} \norm{f'(w)} \norm{v-u}.
\end{eqnarray*}
\end{proof}



\section{Antiderivatives}
Suppose that $X$ is a Banach space and that $f:[a,b] \to X$ be continuous. Define $F:[a,b] \to X$ by
\[
F(x)=\int_a^x f.
\]
Let $x_0 \in (a,b)$. For $x \in (a,b)$, we have
\[
F(x)-F(x_0)=\int_a^x f - \int_a^{x_0} f = \int_{x_0}^x f = f(x_0)(x-x_0)+\int_{x_0}^x (f-f(x_0)),
\]
from which it follows that $F$ is Fr\'echet differentiable at $x_0$, and that
\[
DF(x_0)(x-x_0)=f(x_0)(x-x_0).
\]
If we identify $f(x_0) \in X$ with the map $x \mapsto f(x_0)x$, namely if we say that $X=\mathscr{B}(\mathbb{R},X)$, then $DF(x_0)=f(x_0)$. 

Let $X$ be a normed space, let $Y$ be a Banach space,  let $U$ be an open subset of $X$, and let $f \in C^1(U,Y)$.
Suppose that $u,v \in U$ satisfy $[u,v] \subset U$. Write $I=(0,1)$ 
and define $\gamma:I \to U$ by $\gamma(t)=(1-t)u+tv$.
We have
\[
D\gamma(t)=v-u, \qquad t \in I,
\]
and thus by Theorem \ref{chainrule}, 
\[
D(f \circ \gamma)(t) = Df(\gamma(t)) \circ D\gamma(t), \qquad t \in I,
\]
that is,
\[
D(f \circ \gamma)(t) = Df(\gamma(t)) \circ (v-u), \qquad t \in I,
\]
i.e.
\[
D(f\circ \gamma)(t) = Df(\gamma(t))(v-u), \qquad t \in I.
\]
If $t \in I$ and $t+h \in I$, then
\begin{eqnarray*}
D(f\circ \gamma)(t+h)
-D(f\circ \gamma)(t) &=& Df(\gamma(t+h))(v-u)-
Df(\gamma(t))(v-u)\\
&=&\left(Df(\gamma(t+h))-Df(\gamma(t))\right)(v-u),
\end{eqnarray*}
and hence
\[
\norm{D(f\circ \gamma)(t+h)
-D(f\circ \gamma)(t)} \leq \norm{Df(\gamma(t+h))-Df(\gamma(t))} \norm{v-u}.
\]
Because $Df:U \to \mathscr{B}(X,Y)$ is continuous, it follows that 
\[
\norm{D(f\circ \gamma)(t+h)
-D(f\circ \gamma)(t)} \to 0
\]
as $h \to 0$, i.e. that $D(f \circ \gamma)$ is continuous at $t$, and thus that
\[
D(f \circ \gamma):I \to \mathscr{B}(\mathbb{R},Y)
\]
is continuous. If we identify $\mathscr{B}(\mathbb{R},Y)$ with $Y$, then
\[
D(f \circ \gamma):I \to Y.
\]
On the one hand,
\[
\int_0^1 D(f \circ \gamma) = (f \circ \gamma)(1)-(f \circ \gamma)(0)=f(v)-f(u).
\]
On the other hand,
\[
\int_0^1 D(f\circ \gamma) = \int_0^1 Df(\gamma(t))(v-u) dt = \left( \int_0^1 Df((1-t)u+tv) dt \right) (v-u);
\]
here,
\[
\int_0^1 Df((1-t)u+tv) dt  \in \mathscr{B}(X,Y).
\]
Therefore
\[
f(v)-f(u)=  \left( \int_0^1 Df((1-t)u+tv) dt \right) (v-u).
\]

\end{document}