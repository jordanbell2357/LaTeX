\documentclass[10pt]{article}
\usepackage{amsmath,amssymb,mathrsfs,graphicx,amsthm}
%\usepackage{times}
%\textwidth 4.5in  \textheight 7.125in
\newtheorem{theorem}{Theorem}
\newtheorem{remark}[theorem]{Remark}
\newtheorem{lemma}[theorem]{Lemma}
\newtheorem*{definition}{Definition}
\newtheorem{proposition}[theorem]{Proposition}
\newtheorem{corollary}[theorem]{Corollary}
\newcommand{\comment}[1]{}
\begin{document}
%\title{A note on  Costas sequences and cyclotomic permutations}
\title{What is a Costas array?}
\author{Jordan Bell\\ \texttt{jordan.bell@gmail.com}\\Department of Mathematics, University of Toronto}

\maketitle

To find the speed and distance
of a moving object relative to an observer,
a signal with a known frequency is sent from the observer towards the object.
It reflects off the object, and the reflected signal has a new frequency,
due to the Doppler effect. Knowing the frequencies of the emitted and
reflected signals, the speed
of the moving object relative to the observer can be determined.
Knowing the time between emitting the signal and receiving the
reflected signal, the distance of the moving object to the object
can be determined. See \cite[p. 74]{levanon}.

Let's say we have a stationary observer, and an object is moving radially
away from the observer with a constant speed. One wants to find the
speed $v$ of the moving object and its distance $d$
from the observer.   

Suppose that at times $t_1,\ldots,t_n$ signals are sent from the observer
to the moving object and that the $i$th signal takes time $\delta_i$
to hit the object.
At times $t_1,\ldots,t_n$ let the corresponding distances of the
object be $d_1,\ldots,d_n$. We note that
$d_i=d_1+(t_i-t_1)v$.
Suppose that the signal speed is $c$.
The first signal hits the object when $\delta_1 c-d_1=\delta_1 v$, hence when
\[
\delta_1=\frac{d_1}{c-v}=\frac{d_1}{c}\frac{1}{1-\frac{v}{c}}=\frac{d_1}{c}
+O(c^{-2}).
\]
The $i$th signal hits the object when $\delta_ic-d_i=\delta_iv$, hence when
\[
\delta_i=\frac{d_i}{c-v}=\frac{d_i}{c}\frac{1}{1-\frac{c}{v}}
=\frac{d_i}{c}+O(c^{-2})
=\frac{d_1}{c}+\frac{(t_i-t_1)v}{c}+O(c^{-2}).
\]
Supposing that $(t_i-t_1)v$ is negligible compared to $d_1$,
we can say that approximately $\delta_1=\cdots=\delta_n$;
write $\delta=\delta_1=\cdots=\delta_n$.

If the emitted signals have frequencies $e_1,\ldots,e_n$ and
the reflected signals have frequencies $r_1,\ldots,r_n$, 
by the Doppler shift we have $r_i=e_i\cdot \frac{1-v/c}{1+v/c}$
\cite[p. 4]{levanon}. Thus
the frequency shift for each signal is
\[
r_i-e_i=-2v\frac{e_i}{c+v}=-\frac{2ve_i}{c}+O(c^{-2}).
\]
We assume that the set of frequencies is narrowband, so that approximately
$e_1=\cdots=e_n$. Hence the frequency shifts are approximately equal;
write
$\xi=-\frac{2ve_1}{c}=\cdots=-\frac{2ve_n}{c}$.

Then the frequency of the signal received at time $t_i+\delta$ 
will be equal to the sum of $\xi$ and
the frequency of the signal emitted at time $t_i$.
Let $e(t)$ be the frequency of the signal emitted at time $t$, and 
let $r(t)$ be the frequency of the reflected signal received at time $t$.
So $r(t_i+\delta)=e(t_i)+\xi$ for each $i=1,\ldots,n$.

Let's use units such that the times $t_1,\ldots,t_n$ are integers.
We define the {\em cross-correlation} of the signals $e$ and $r$
by
\[
C_{e,r}(v,h)=\sum_{i=-\infty}^\infty \delta_{e(i),r(i+v)-h},
\]
where $\delta_{i,j}$ here is the Kronecker delta.
For $t$ other than $t_1,\ldots,t_n$ and $t$ other than
$t_1+\delta,\ldots,t_n+\delta$,
it is convenient to take respectively $e(t)$ and $r(t)$ to be undefined
rather than equal to 0. When either $e(t)$ or $r(t)$ is undefined
the Kronecker delta will then be 0.
Then 
\begin{eqnarray*}
C_{e,r}(v,h)&=&\sum_{i=-\infty}^\infty \delta_{e(i),r(i+v)-h}\\
&=&\sum_{i=-\infty}^\infty \delta_{e(i+\delta-v),r(i+\delta)-h}\\
&=&\sum_{i=-\infty}^\infty \delta_{e(i+\delta-v),e(i)+\xi-h}\\
&=&\sum_{i=-\infty}^\infty \delta_{e(i),e(i-\delta+v)+\xi-h}\\
&=&C_{e,e}(v-\delta,h-\xi).
\end{eqnarray*}

Certainly $C_{e,e}(v,h) \leq C_{e,e}(0,0)=n$ for all $v,h$. 
Thus
$C_{e,r}(v,h) \leq C_{e,r}(\delta,\xi)=n$ for all $v,h$.

We wish to find
\[
\max_{(v,h) \neq (0,0)} C_{e,e}(v,h).
\]
For any $1 \leq j<k \leq n$, let $v=t_k-t_j$ and $h=e(t_k)-e(t_j)$.
\begin{eqnarray*}
C_{e,e}(v,h)&=&C_{e,e}(t_k-t_j,e(t_k)-e(t_j))\\
&=&\sum_{i=-\infty}^\infty \delta_{e(i),e(i+t_k-t_j)-e(t_k)+e(t_j)}\\
&\geq&\delta_{e(t_j),e(t_j+t_k-t_j)-e(t_k)+e(t_j)}\\
&=&1,
\end{eqnarray*}
so $\max_{(v,h) \neq (0,0)} C_{e,e}(v,h) \geq 1$.

We want to figure out $\delta$ and $\xi$ by knowing $C_{e,r}(v,h)$ for
all $v,h$.
If we have that $\max_{(v,h) \neq (0,0)} C_{e,e}(v,h)=1$, then either
$(v,h)=(\delta,\xi)$ and $C_{e,r}(v,h)=n$,
or $(v,h) \neq (\delta,\xi)$ and 
$C_{e,r}(v,h) \leq 1$, that is, the autocorrelation at the pair $(\delta,
\xi)$ that we wish to find is as sharply distinguished as possible
from the autocorrelation
at the other time-frequency pairs $(v,h)$.

\section{Mathematics}
Define a new sequence $f$ by
\[
f(i)=|\{1 \leq j \leq n:e(t_j) \leq e(t_i)\}|, \qquad
1 \leq i \leq n.
\]
The sequence $f$ is a normalization of the sequence $e$. One quickly checks
that
$\max_{(v,h) \neq (0,0)} C_{e,e}(v,h)=1$ is equivalent to 
$\max_{(v,h) \neq (0,0)} C_{f,f}(v,h)=1$.

Let $T=\{(i,f(i)):i=1,\ldots,n\} \subset \mathbb{Z}^2$.
The condition $\max_{(v,h) \neq (0,0)} C_{f,f}(v,h)=1$
holds if and only if $|T \cap (T+(v,h))| \leq 1$ for
all $(v,h) \neq (0,0)$.

\begin{definition}
A {\em Costas sequence of length $n$} is a sequence $f(1),\ldots,f(n)$
of length $n$ such that if $T=\{(i,f(i)):i=1,\ldots,n\}$, then
$|T \cap (T+(v,h))| \leq 1$ for all $(v,h) \neq (0,0)$. We call the set
$T$ a {\em Costas array}.
\end{definition}

This can also be said in the following way.
For $1 \leq i_1 \leq i_2 \leq n$
and $1 \leq i_3 \leq i_4 \leq n$,
if $(i_1-i_2,f(i_1)-f(i_2))=
(i_3-i_4,f(i_3)-f(i_4))$ then $i_1=i_3$ and $i_2=i_4$.

Drakakis \cite{review} 
gives a survey of Costas arrays.

\section{Generalizations}
Now that we have detached Costas sequences from the time-frequency setting
in which they originate, 
we can investigate them as mathematical objects, and also investigate
modifications of Costas sequences.

Let $[n]=\{1,\ldots,n\} \subset \mathbb{Z}$.
A Costas sequence $f$ of length $n$ is a bijection $f:[n] \to [n]$ that satisfies
the condition on overlapping translates.

Moran \cite{moran}


\bibliographystyle{amsplain}
\bibliography{costas}

\end{document}
