\documentclass{article}
\usepackage{amsmath,amssymb,mathrsfs,amsthm}
%\usepackage{tikz-cd}
%\usepackage{hyperref}
\newcommand{\inner}[2]{\left\langle #1, #2 \right\rangle}
\newcommand{\tr}{\ensuremath\mathrm{tr}\,} 
\newcommand{\Span}{\ensuremath\mathrm{span}} 
\def\Re{\ensuremath{\mathrm{Re}}\,}
\def\Im{\ensuremath{\mathrm{Im}}\,}
\newcommand{\id}{\ensuremath\mathrm{id}} 
\newcommand{\HS}{\ensuremath\mathrm{HS}} 
\newcommand{\var}{\ensuremath\mathrm{var}} 
\newcommand{\Lip}{\ensuremath\mathrm{Lip}} 
\newcommand{\GL}{\ensuremath\mathrm{GL}}
\newcommand{\diam}{\ensuremath\mathrm{diam}} 
\newcommand{\sgn}{\ensuremath\mathrm{sgn}\,} 
\newcommand{\lcm}{\ensuremath\mathrm{lcm}} 
\newcommand{\supp}{\ensuremath\mathrm{supp}\,}
\newcommand{\dom}{\ensuremath\mathrm{dom}\,}
\newcommand{\upto}{\nearrow}
\newcommand{\downto}{\searrow}
\newcommand{\norm}[1]{\left\Vert #1 \right\Vert}
\newtheorem{theorem}{Theorem}
\newtheorem{lemma}[theorem]{Lemma}
\newtheorem{proposition}[theorem]{Proposition}
\newtheorem{corollary}[theorem]{Corollary}
\theoremstyle{definition}
\newtheorem{definition}[theorem]{Definition}
\newtheorem{example}[theorem]{Example}
\begin{document}
\title{The Hilbert transform on $\mathbb{R}$}
\author{Jordan Bell\\ \texttt{jordan.bell@gmail.com}\\Department of Mathematics, University of Toronto}
\date{\today}

\maketitle


\section{The principal value integral}
Let $A_\epsilon=\{x \in \mathbb{R}: |x| \geq \epsilon\}$. For $f \in \bigcap_{\epsilon>0} L^1(A_\epsilon)$, 
if $\int_{A_\epsilon}  f(x) dx$ has a limit as $\epsilon \to 0$, we denote it by
\[
PV \int_{\mathbb{R}} f(x) dx = \lim_{\epsilon \to 0} \int_{|x| \geq \epsilon} f(x) dx.
\]

Let $\mathscr{S}$ be the collection of Schwartz functions $\mathbb{R} \to \mathbb{C}$ and let
$\mathscr{S}'$ be its dual space, whose elements are called tempered distributions.
For $\phi \in \mathscr{S}$, for $k>j$,
\begin{align*}
\left| \int_{A_{1/j}}  \frac{\phi(x)}{x} dx - \int_{A_{1/k}} \frac{\phi(x)}{x} dx\right|&=\left| \int_{1/k \leq |x| < 1/j} \frac{\phi(x)}{x} dx \right|\\
&=\left| \int_{1/k \leq |x| < 1/j} \frac{\phi(x)-\phi(0)}{x} dx \right|\\
&\leq \int_{1/k \leq |x| < 1/j} \norm{\phi'}_b\\
&=\norm{\phi'}_b\ \cdot \left( \frac{1}{j}-\frac{1}{k} \right)\\
&=\norm{\phi'}_b\ \cdot \frac{k-j}{kj}\\
&\leq \frac{\norm{\phi'}_b}{j}.
\end{align*}
Therefore $\int_{A_{1/j}} \frac{\phi(x)}{x} dx$ is a Cauchy sequence in $\mathbb{C}$ and hence
converges.
Then the following limit exists:
\[
\inner{\phi}{W}=PV \int_{\mathbb{R}} \frac{\phi(x)}{x} dx = \lim_{\epsilon \to 0} \int_{|x| \geq \epsilon} \frac{\phi(x)}{x} dx.
\]
It is apparent that $W:\mathscr{S} \to \mathbb{C}$ is linear, and
one proves that
$W \in \mathscr{S}'$.


For $\phi \in \mathscr{S}$,
by Hadamard's lemma, there is a $C^\infty$ function $\psi:\mathbb{R} \to \mathbb{C}$ such that
$\phi(x)=\phi(0)+x \psi(x)$ for all $x$. For $\epsilon>0$,
\begin{align*}
\int_\epsilon^1 \phi'(x) \log x dx&=x \psi(x) \log x \bigg|_\epsilon^1 - \int_\epsilon^1 (\phi(x)-\phi(0)) \frac{1}{x} dx\\
&=-\epsilon \psi(\epsilon) \log \epsilon - \int_\epsilon^1 \frac{\phi(x)}{x} dx - \phi(0) \log \epsilon
\end{align*}
and
\begin{align*}
\int_{-1}^{-\epsilon} \phi'(x) \log|x| dx&=x \psi(x) \log(-x) \bigg|_{-1}^{-\epsilon} - \int_{-1}^{-\epsilon} (\phi(x)-\phi(0)) \frac{1}{x} dx\\
&=-\epsilon \psi(-\epsilon) \log \epsilon - \int_{-1}^{-\epsilon} \frac{\phi(x)}{x} dx 
+\phi(0) \log \epsilon.
\end{align*}
Hence
\[
\int_{\epsilon \leq |x| \leq 1} \phi'(x) \log |x| dx = -\epsilon \cdot (\psi(\epsilon)+\psi(-\epsilon)) \cdot \log \epsilon - 
\int_{\epsilon \leq |x| \leq 1} \frac{\phi(x)}{x} dx.
\]
On the other hand,
\[
\int_{|x| \geq 1} \phi'(x) \log |x| dx = - \int_{|x| \geq 1} \frac{\phi(x)}{x} dx.
\]
Therefore
\[
PV  \int_{\mathbb{R}} \phi'(x) \log |x| dx = - PV \int_{\mathbb{R}} \frac{\phi(x)}{x} dx.
\]



Let $\mu \phi(x)=\phi(-x)$ and $\tau_y \phi(x)=\phi(x-y)$.
Then
\[
\tau_y \mu \phi(x) = \mu \phi(x-y) = \phi(y-x) = \tau_x \phi(y).
\]
Write
\begin{align*}
\phi * \psi(x) &= \int_{\mathbb{R}} \phi(y) \psi(x-y) dy\\
&=\int_{\mathbb{R}} \phi(y) (\tau_y \psi)(x) dy\\
&=\int_{\mathbb{R}} \phi(y) (\tau_x \mu \psi)(y) dy.
\end{align*}

For $u \in \mathscr{S}'$
and for $\phi \in \mathscr{S}$, define
$\phi * u:\mathbb{R} \to \mathbb{C}$ by
\[
(\phi * u)(x) = \inner{\tau_x \mu \phi}{u}.
\]
One  proves that $\phi*u$ is a tempered distribution, and satisfies
\[
\inner{\psi}{\phi*u} = \inner{(\mu \phi)*\psi}{u}.
\]


Define $\tau_x u \in \mathscr{S}'$ by
\[
\inner{\phi}{\tau_x u} = \inner{\tau_{-x} \phi}{u}.
\]




\section{The Hilbert transform}
For $\epsilon>0$, for $\phi \in \mathscr{S}$ let
\begin{align*}
H_\epsilon \phi(x) &=\frac{1}{\pi} \int_{|y| \geq \epsilon} \frac{\phi(x-y)}{y} dy\\
&=\frac{1}{\pi} \int_{|y| \geq \epsilon} \frac{\tau_y \phi(x)}{y} dy\\
&=\frac{1}{\pi} \int_{|y| \geq \epsilon} \frac{\tau_x \mu \phi(y)}{y} dy.
\end{align*}
Define
\begin{align*}
H\phi(x)&=\lim_{\epsilon \to 0} H_\epsilon \phi(x)\\
&=\frac{1}{\pi} \cdot PV \int_{\mathbb{R}} \frac{\tau_x \mu \phi(y)}{y} dy\\
&=\frac{1}{\pi} \cdot PV \int_{\mathbb{R}} \frac{\phi(x-y)}{y} dy\\
&=\frac{1}{\pi} \cdot PV \int_{\mathbb{R}} \frac{\phi(y)}{x-y} dy.
\end{align*}

For $x \in \mathbb{R}$,
\begin{align*}
\phi*W(x)&=\inner{\tau_x \mu \phi}{W}\\
&=PV \int_{\mathbb{R}} \frac{\tau_x \mu \phi(y)}{y} dy\\
&=PV \int_{\mathbb{R}} \frac{\phi(x-y)}{y} dy\\
&=PV \int_{\mathbb{R}} \frac{\phi(y)}{x-y} dy\\
&=\lim_{\epsilon \to 0} \int_{|x| \geq \epsilon} \frac{\phi(y)}{x-y} dy.
\end{align*}
Thus
\[
H \phi = \frac{1}{\pi} \phi*W = \phi*\left( \frac{W}{\pi}\right).
\]

We calculate
\begin{align*}
\inner{\phi}{\widehat{W}}&=\inner{\widehat{\phi}}{W}\\
&= \lim_{\epsilon \to 0} \int_{\epsilon \leq |\xi| \leq 1/\epsilon} \frac{\widehat{\phi}(\xi)}{\xi} d\xi\\
&=\lim_{\epsilon \to 0} \int_{|\xi| \geq \epsilon} \left( \int_{\mathbb{R}} \phi(x) e^{-2\pi ix\xi} dx \right) \frac{1}{\xi} d\xi\\
&=\lim_{\epsilon \to 0}\int_{\mathbb{R}} \phi(x) \left( \int_{\epsilon \leq |\xi| \leq 1/\epsilon} \frac{e^{-2\pi ix\xi}}{\xi} d\xi\right) dx\\
&=\lim_{\epsilon \to 0}\int_{\mathbb{R}} \phi(x) \left( \int_{\epsilon \leq |\xi| \leq 1/\epsilon} -i \frac{\sin 2\pi x\xi}{\xi} d\xi\right) dx.
\end{align*}
Check that
\[
\lim_{\epsilon \to 0} \int_{\epsilon \leq |\xi| \leq 1/\epsilon}  \frac{\sin 2\pi x\xi}{\xi} d\xi
=\pi \cdot \sgn x.
\]
Then, using the dominated convergence theorem,
\[
\inner{\phi}{\widehat{W}} = \int_{\mathbb{R}} \phi(x) \cdot -i\pi \cdot \sgn x dx.
\]
Thus, $\widehat{W} = -\pi i \cdot\sgn$. 

Now,
\[
\widehat{\phi*u} = \widehat{\phi} \cdot \widehat{u}.
\]
Then
\[
\widehat{H\phi}(\xi) = \frac{1}{\pi} \widehat{\phi*W}(\xi) = \frac{1}{\pi} \widehat{\phi}(\xi) \cdot \widehat{W}(\xi)
=\widehat{\phi}(\xi) \cdot -i \cdot \sgn(\xi).
\]
Let
\[
m_H(\xi) = - i \cdot \sgn(\xi),
\]
with which
\[
\widehat{H\phi} = m_H \cdot \widehat{\phi}.
\]
Writing $F\phi = \widehat{\phi}$,
\[
FH \phi = m_H \cdot F\phi.
\]
So
\[
H \phi = F^{-1}(m_H \cdot F\phi),
\]
and hence
\[
H^2 \phi = F^{-1}(m_H \cdot F H\phi)
=F^{-1}(m_H \cdot m_H F\phi).
\]
For $\xi \neq 0$, $m_H(\xi)^2 = -1$, which yields
\[
H^2 \phi = F^{-1}(-F\phi) = -\phi.
\]
Thus $H^2 = -\id$.
Therefore $\norm{H\phi}_{L^2}=\norm{\phi}_{L^2}$.

Thus it makes sense to define $H:L^2(\mathbb{R}) \to L^2(\mathbb{R})$. 
For $f,g \in L^2(\mathbb{R})$, by Plancherel's theorem, and as $\overline{m_H}=-m_H$,
\begin{align*}
\inner{Hf}{g}&=\inner{\widehat{Hf}}{\widehat{g}}\\
&=\int_{\mathbb{R}} \widehat{Hf}(\xi) \cdot \overline{\widehat{g}(\xi)} d\xi\\
&=\int_{\mathbb{R}} m_H(\xi) \cdot \widehat{f}(\xi) \overline{\widehat{g}(\xi)} d\xi\\
&=-\int_{\mathbb{R}} \widehat{f}(\xi) \cdot \overline{m_H(\xi) \cdot \widehat{g}(\xi)} d\xi\\
&=-\int_{\mathbb{R}} \widehat{f}(\xi) \cdot \overline{\widehat{Hg}(\xi)} d\xi\\
&=-\inner{f}{Hg}.
\end{align*}
But $\inner{Hf}{g}=\inner{f}{H^*g}$, so
\[
\inner{f}{H^*g} = \inner{f}{-Hg},
\]
which implies that $H^*g=-Hg$ and thus $H^*=-H$. Furthermore,
\[
\widehat{H^*g}(\xi) = -\widehat{Hg}(\xi) = -m_H(\xi) \cdot \widehat{g}(\xi) = i \cdot \sgn(\xi) \cdot \widehat{g}(\xi).
\]





\section{The Poisson kernel}
For $y>0$,
calculate
\begin{align*}
\int_{\mathbb{R}}e^{2\pi i\xi x} e^{-2\pi y|\xi|} d\xi&=\frac{1}{2\pi ix+2\pi y} - \frac{1}{2\pi ix-2\pi y}\\
&=\frac{2\pi ix-2\pi y-2\pi ix-2\pi y}{-4\pi^2 x^2 - 4\pi^2 y^2}\\
&=\frac{y}{\pi(x^2+y^2)}
\end{align*}
and
\begin{align*}
-i\int_{\mathbb{R}} e^{2\pi i\xi x} \sgn(\xi) e^{-2\pi y|\xi|} d\xi&=\frac{i}{2\pi ix+2\pi y}
+\frac{i}{2\pi ix-2\pi y}\\
&=\frac{i(2\pi ix-2\pi y+2\pi ix+2\pi y)}{-4\pi^2 x^2 -4\pi^2 y^2}\\
&=\frac{-4\pi x}{-4\pi^2 x^2-4\pi^2 y^2}\\
&=\frac{x}{\pi(x^2+y^2)}.
\end{align*}



For $y>0$ let
\[
P_y(x) = \frac{1}{\pi} \frac{y}{x^2+y^2} 
\]
and
\[
Q_y(x) = \frac{1}{\pi} \frac{x}{x^2+y^2}.
\]
Then
\[
\widehat{P}_y(\xi) = e^{-2\pi y|\xi|}
\]
and
\[
\widehat{Q}_y(\xi) = -i \cdot \sgn(\xi) e^{-2\pi y|\xi|}.
\]
Also,
\[
P_y(x)+iQ_y(x) = \frac{1}{\pi} \frac{y+ix}{x^2+y^2} = \frac{1}{\pi} \frac{1}{y-ix}.
\]


For a Borel measurable function $f:\mathbb{R} \to \mathbb{C}$ for which the integral exists,
\begin{align*}
(P_y*f)(x) &= \int_{\mathbb{R}} P_y(x-t) f(t) dt\\
&= \int_{\mathbb{R}}f(t) \frac{y}{\pi((x-t)^2+y^2)}  dt\\
&=\int_{\mathbb{R}} f(x-t) \frac{y}{\pi(t^2+y^2)} dt
\end{align*}
and 
\begin{align*}
(Q_y*f)(x) &= \int_{\mathbb{R}} Q_y(x-t) f(t) dt\\
&= \int_{\mathbb{R}}f(t)  \frac{x-t}{\pi((x-t)^2+y^2)}  dt\\
&=\int_{\mathbb{R}} f(x-t) \frac{t}{\pi(t^2+y^2)} dt.
\end{align*}
Then
\begin{align*}
P_y*f(x)+iQ_y*f(x)&=\int_{\mathbb{R}} f(x-t)  \frac{1}{\pi} \frac{1}{y-it} dt\\
&=\int_{\mathbb{R}} f(t) \frac{1}{\pi} \frac{1}{y-ix+it} dt\\
&=\frac{i}{\pi} \int_{\mathbb{R}} \frac{f(t)}{x+iy-t} dt.
\end{align*}



For $y_1,y_2>0$,
\begin{align*}
\widehat{P_{y_1}*P_{y_2}}(\xi)&=\widehat{P_{y_1}}(\xi) \cdot \widehat{P_{y_2}}(\xi)\\
&= e^{-2\pi y_1 |\xi|} \cdot  e^{-2\pi y_2|\xi|}\\
&=e^{-2\pi(y_1+y_2) |\xi|}\\
&=\widehat{P_{y_1+y_2}}(\xi).
\end{align*}
Therefore $(P_y)_{y>0}$ is a semigroup using convolution: for $y_1,y_2>0$,
\[
P_{y_1} * P_{y_2} = P_{y_1+y_2}.
\]


Let $\mathbb{H} = \{z \in \mathbb{C} : \Im z>0\}$ and for $\phi \in \mathscr{S}$ let
\[
F_\phi(z) = \frac{i}{\pi} \int_{\mathbb{R}} \frac{\phi(t)}{z-t} dt,\qquad z \in \mathbb{H},
\]
which is a complex analytic function. For
$z=x+iy \in \mathbb{H}$,
\[
P_y *\phi (x) + iQ_y * \phi(x) = \frac{i}{\pi} \int_{\mathbb{R}} \frac{\phi(t)}{x+iy-t} dt
=F_\phi(z).
\]

It is proved that for $1 \leq p < \infty$ and $f \in L^p(\mathbb{R})$,
$Q_\epsilon*f - H_\epsilon f \to 0$ in $L^p$ as $\epsilon \to 0$, and that for almost all $x \in \mathbb{R}$,
$Q_\epsilon*f(x) - H_\epsilon f(x) \to 0$ as $\epsilon \to 0$.\footnote{Loukas Grafakos,
{\em Classical Fourier Analysis}, second ed.,
p.~254, Theorem 4.1.5.}

For $1<p<\infty$, it can be proved that there is some $C_p$ such that 
\[
\norm{H\phi}_{L^p} \leq C_p \norm{\phi}_{L^p}
\]
for all $\phi \in \mathscr{S}$, with $C_p \leq 2p$ for $2 \leq p < \infty$
and $C_p \leq \frac{2p}{p-1}$ for $1<p \leq 2$.\footnote{Loukas Grafakos,
{\em Classical Fourier Analysis}, second ed.,
p.~255, Theorem 4.1.7.}


\end{document}