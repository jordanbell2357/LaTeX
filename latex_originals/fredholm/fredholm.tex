\documentclass{article}
\usepackage{amsmath,amssymb,graphicx,subfig,mathrsfs,amsthm,siunitx}
%\usepackage{tikz-cd}
\usepackage{hyperref}
\newcommand{\inner}[2]{\left\langle #1, #2 \right\rangle}
\newcommand{\tr}{\ensuremath\mathrm{tr}\,} 
\newcommand{\Span}{\ensuremath\mathrm{span}} 
\def\Re{\ensuremath{\mathrm{Re}}\,}
\def\Im{\ensuremath{\mathrm{Im}}\,}
\newcommand{\id}{\ensuremath\mathrm{id}} 
\newcommand{\sgn}{\ensuremath\mathrm{sgn}\,} 
\newcommand{\rank}{\ensuremath\mathrm{rank\,}} 
\newcommand{\co}{\ensuremath\mathrm{co}\,} 
\newcommand{\cco}{\ensuremath\overline{\mathrm{co}}\,}
\newcommand{\supp}{\ensuremath\mathrm{supp}\,}
\newcommand{\epi}{\ensuremath\mathrm{epi}\,}
\newcommand{\lsc}{\ensuremath\mathrm{lsc}\,}
\newcommand{\ext}{\ensuremath\mathrm{ext}\,}
\newcommand{\cl}{\ensuremath\mathrm{cl}\,}
\newcommand{\dom}{\ensuremath\mathrm{dom}\,}
\newcommand{\LSC}{\ensuremath\mathrm{LSC}}
\newcommand{\USC}{\ensuremath\mathrm{USC}}
\newcommand{\Cyl}{\ensuremath\mathrm{Cyl}\,}
\newcommand{\extreals}{\overline{\mathbb{R}}}
\newcommand{\upto}{\nearrow}
\newcommand{\downto}{\searrow}
\newcommand{\norm}[1]{\left\Vert #1 \right\Vert}
\newtheorem{theorem}{Theorem}
\newtheorem{lemma}[theorem]{Lemma}
\newtheorem{proposition}[theorem]{Proposition}
\newtheorem{corollary}[theorem]{Corollary}
\theoremstyle{definition}
\newtheorem{definition}[theorem]{Definition}
\newtheorem{example}[theorem]{Example}
\begin{document}
\title{The Fredholm determinant}
\author{Jordan Bell\\ \texttt{jordan.bell@gmail.com}\\Department of Mathematics, University of Toronto}
\date{\today}

\maketitle

\section{Introduction}
By $\mathbb{N}$ we mean the set of positive integers. In this note we write inner products as conjugate linear in the first variable, following the notation of Reed and Simon.
The purpose of this note is to make sense of $\det(I+A)$ for bounded trace class operators on a Hilbert space. This definition is consistent with the definition of the determinant for
a finite dimensional Hilbert space.


\section{Singular value decomposition}
Let $H$ be a Hilbert space with an inner product that is conjugate linear in the first variable. We do not presume unless we say
so that $H$ is separable.\footnote{Often one assumes separability not because statements are false
for nonseparable Hilbert spaces but because it is notationally easier to talk about separable Hilbert spaces, and this
indolence hides where separability matters. Moreover, whether or not $H$ is separable, it does not take long to prove that the image
of a compact linear operator is separable.}

We denote by $\mathscr{B}(H)$ the set of bounded linear operators $H \to H$.
For any $A \in \mathscr{B}(H)$, $A^*A$ is positive and one proves that it has a unique positive square root $|A| \in \mathscr{B}(H)$. 
We call $|A|$ the {\em absolute value of $A$}.

We say that $U \in \mathscr{B}(H)$ is a {\em partial isometry} if there is a closed subspace $X$ of $H$ such that the restriction
of $U$ to $X$ is an isometry $X \to U(X)$ and $\ker U = X^\perp$. One proves that for any $A \in \mathscr{B}(H)$, there is a unique partial
isometry $U$ satisfying both $\ker U=\ker A$ and $A=U|A|$, and $A=U|A|$ is called the {\em polar decomposition of $A$}. Some useful identities that
the polar decomposition satisfies are
\[
U^*U|A|=|A|, \quad U^*A=|A|, \quad UU^*A=A.
\]


If $A \in \mathscr{B}(H)$ is compact and self-adjoint, the spectral theorem tells us that there is an orthonormal set $\{e_n: n \in \mathbb{N}\}$
in $H$ and $\lambda_n \in \mathbb{R}$, $|\lambda_1| \geq |\lambda_2| \geq \cdots$, such that 
\[
Ax = \sum_{n \in \mathbb{N}} \lambda_n \inner{e_n}{x}e_n, \qquad x \in H.
\]
If $A \in \mathscr{B}(H)$ is compact, then $|A|$ is compact (the compact operators are an ideal and $|A|=U^*A$), so the spectral theorem tells us that there is an orthonormal set
$\{e_n: n \in \mathbb{N}\}$ in $H$ and $\lambda_n \in \mathbb{R}$, $\lambda_1 \geq \lambda_2 \geq \cdots \geq 0$, such that
\[
|A|x = \sum_{n \in \mathbb{N}} \lambda_n \inner{e_n}{x}e_n, \qquad x\in H.
\]
Write $\sigma(A)=\lambda_n(|A|)$. $\sigma_n(A)$ are called the {\em singular values of $A$}.

Using the spectral theorem and the polar decomposition, one proves that for any compact $A \in \mathscr{B}(H)$, there are orthonormal
sets $\{e_n: n \in \mathbb{N}\}$ and $\{f_n: n \in \mathbb{N}\}$ in $H$ such that
\[
Ax = \sum_{n \in \mathbb{N}} \sigma_n(A) \inner{f_n}{x}e_n, \qquad x \in H.
\]
This is called the {\em singular value decomposition of $A$}. 


\section{Trace class operators}
We denote by $\mathscr{B}_1(H)$ the set of those  $A \in \mathscr{B}(H)$ that are compact and such that
\[
\norm{A}_1 = \sum_{n \in \mathbb{N}} \sigma_n(A) < \infty.
\]
Elements of $\mathscr{B}_1(H)$ are called {\em trace class operators}.
It can be proved that 
$\mathscr{B}_1(H)$ with the norm $\norm{\cdot}_1$ is a Banach space. 

Let $\mathscr{E}$ be an orthonormal basis for $H$.
We define $\tr: \mathscr{B}_1(H) \to \mathbb{C}$ by
\[
\tr A = \sum_{e \in \mathscr{E}} \inner{e}{Ae}, \qquad A \in \mathscr{B}_1(H).
\]
One proves that the value of this sum is the same for any orthonormal basis of $H$, and that $\tr$ is a bounded linear operator.

It is a fact that if $A \in \mathscr{B}_1(H)$ then $A^* \in \mathscr{B}_1(H)$, $\tr A^* = \overline{\tr A}$, and $\norm{A^*}_1 =\norm{A}_1$. Another fact is that
if $A \in \mathscr{B}_1(H)$ and $B \in \mathscr{B}(H)$, then $AB,BA \in \mathscr{B}_1(H)$, $\tr(AB)=\tr(BA)$,  $|\tr(BA)| \leq 
\norm{B}\norm{A}_1$, and $\norm{AB}_1 \leq \norm{A}_1 \norm{B}$, $\norm{BA}_1 \leq \norm{B}\norm{A}_1$.  Finally,
if $A \in \mathscr{B}_1(H)$ then $\norm{A} \leq \norm{A}_1$. 



\section{Logarithms}
We say that $A \in \mathscr{B}(H)$ is {\em invertible} if $A^{-1} \in \mathscr{B}(H)$.  
If $A \in \mathscr{B}(H)$ and $\norm{A}<1$, one checks that  $I-A$ is invertible: $(I-A)^{-1}= \sum_{n=0}^\infty A^n$. Furthermore, suppose that $A \in \mathscr{B}_1(H)$ and $\norm{A}<1$. Then,
\[
\log(I-A)=\sum_{n=1}^\infty -\frac{A^n}{n},
\]
hence
\[
\tr \log(I-A) = \sum_{n=1}^\infty -\frac{\tr(A^n)}{n},
\]
and
\begin{equation}
\exp \tr \log(I-A) = \prod_{n=1}^\infty \exp\left(-\frac{\tr(A^n)}{n}\right).
\label{explicitdet}
\end{equation}
Because $\det \exp B=\exp \tr B$ for any $B \in \mathscr{B}(\mathbb{R}^N)$, if $A \in \mathscr{B}(\mathbb{R}^N)$ satisfies $\norm{A}<1$ then
$\det (I-A)= \exp \tr \log (I-A)$.
The above expression makes
sense for any $A \in \mathscr{B}_1(H)$ with $\norm{A}<1$, so it makes sense to define
\[
\det (I-A)=\exp \tr \log (I-A)
\]
 in this case, for which we have the explicit formula \eqref{explicitdet}.
We have not shown that
$\det$ has the properties we might expect it to have, but at least in the case $H=\mathbb{R}^N$ it is equal to the ordinary
determinant, which is
the least we could demand of a function which we denote by $\det$.



\section{Tensor products} 
Suppose that $H$ is a Hilbert space over $K$ with inner product
$\inner{\cdot}{\cdot}$  that is conjugate linear in the first variable.
For each $n \in \mathbb{N}$ and $x_1,\ldots,x_n \in H$, let $x_1 \otimes \cdots \otimes x_n$  be the  multilinear function
on $H^n$ defined by 
\[
(x_1 \otimes \cdots \otimes x_n)(y_1,\ldots,y_n) = \prod_{k=1}^n \inner{x_k}{y_k}, \qquad (y_1,\ldots,y_n) \in H^n.
\]
The set of all finite linear combinations of such  multilinear functions is a vector  space for which there is a unique inner product that satisfies
\[
\inner{x_1 \otimes \cdots \otimes x_n}{y_1 \otimes \cdots \otimes y_n} = \prod_{k=1}^n \inner{x_k}{y_k}, \qquad
x_1,\ldots,x_n,y_1,\ldots,y_n \in H.
\]
We denote by $\bigotimes^n H$ the completion of this vector space using this inner product.\footnote{See
Michael Reed and Barry Simon, {\em Methods of Modern Mathematical Physics, volume I: Functional Analysis}, revised and enlarged
ed., p. 50.}
Thus,
$\bigotimes^n H$ is a Hilbert space.
We call this a {\em tensor product}, but it does not have the universal
property of tensor products, so is not a categorical tensor product.\footnote{Paul Garrett, \url{http://www.math.umn.edu/~garrett/m/v/nonexistence_tensors.pdf}} 

If $A \in \mathscr{B}(H)$, one proves that there is a unique $T \in \mathscr{B}(\bigotimes^n H)$ that satisfies
\[
T(x_1 \otimes \cdots \otimes x_n)=Ax_1 \otimes \cdots \otimes Ax_n, \qquad
x_1\otimes \cdots \otimes x_n \in \textstyle{\bigotimes^n H},
\]
and we write $T=\bigotimes^n A$. For $A,B \in \mathscr{B}(H)$
and $x_1 \otimes \cdots \otimes x_n \in \bigotimes^n H$,
it is straightforward to check that 
\[
\textstyle{(\bigotimes^n (AB))(x_1 \otimes \cdots \otimes x_n) = (\bigotimes^n A)(\bigotimes^n B)(x_1 \otimes \cdots \otimes x_n)},
\]
from which it follows that $\bigotimes^n (AB) = \bigotimes^n A \bigotimes^n B$.

\section{Exterior powers}
Let $S_n$ denote the group of permutations on $n$ symbols and let $\sgn(\pi)$ denote the sign of the permutation $\pi$. 
For $x_1,\ldots,x_n \in H$, we define $x_1 \wedge \cdots \wedge x_n \in \bigotimes^n H$ by
\[
x_1 \wedge \cdots \wedge x_n = (n!)^{-1/2} \sum_{\pi \in S_n} \sgn(\pi) x_{\pi(1)} \otimes \cdots \otimes x_{\pi(n)}.
\]
We define $\bigwedge^n H$ to be the closure in $\bigotimes^n H$ of the set
of all finite linear combinations of elements of $\bigotimes^n H$ of the form $x_1 \wedge \cdots \wedge x_n$.\footnote{cf. Paul Garrett, \url{http://www.math.umn.edu/~garrett/m/algebra/notes/28.pdf}}
Thus, $\bigwedge^n H$ is a Hilbert space.

For $x_1 \wedge \cdots \wedge x_n, y_1 \wedge \cdots \wedge y_n \in \bigwedge^n H$, one proves that\footnote{Michael
Reed and Barry Simon, {\em Methods of Modern Mathematical Physics, volume IV: Analysis of Operators}, p.~321.}
\[
\inner{x_1 \wedge \cdots \wedge x_n}{y_1 \wedge \cdots \wedge y_n}= \det(\inner{x_i}{y_j}),
\]
where
\[
\det(a_{i,j}) = \sum_{\pi \in S_n} \sgn(\pi) a_{1,\pi(1)} \cdots a_{n,\pi(n)}.
\]
In particular,
if $\{x_1,\ldots,x_n\}$ is an orthonormal set in $H$, then $\norm{x_1 \wedge \cdots \wedge x_n}=1$.


For $A \in \mathscr{B}(H)$ and $x_1 \wedge \cdots \wedge x_n \in \bigwedge^n H$, it is apparent that
\[
\textstyle{(\bigotimes^n A)x_1 \wedge \cdots \wedge x_n = Ax_1 \wedge \cdots \wedge Ax_n.}
\]
 It follows that $\bigotimes^n A$ sends
an element of $\bigwedge^n H$ to an element $\bigwedge^n H$, and hence that the restriction of $\bigotimes^n A$ to
$\bigwedge^n H$ belongs to $\mathscr{B}(\bigwedge^n H)$. We denote this restriction by $\bigwedge^n A$. 
Because $\bigotimes^n (AB) = \bigotimes^n A \bigotimes^n B$, we also have
$\bigwedge^n (AB) = \bigwedge^n A \bigwedge^n B$. 

\section{Finite dimensional Hilbert spaces}
Suppose that $H$ is an $n$-dimensional Hilbert space and that $A \in \mathscr{B}(H)$.
If $H$ has dimension $n$ and $\{e_1,\ldots,e_n\}$ is an orthonormal basis for $H$, one proves that
\[
\{e_{i_1} \wedge \cdots \wedge e_{i_k}: 1 \leq i_1 < \cdots <i_k \leq n\}
\]
is an orthonormal basis for $\bigwedge^k H$, and hence that 
$\bigwedge^k H$ has dimension $\binom{n}{k}$. So $\bigwedge^n H$ has dimension 1, and
as $\bigwedge^n A \in \mathscr{B}(\bigwedge^n H)$, there is some scalar $\alpha$ such that
$(\bigwedge^n A) v = \alpha v$ for all $v \in \bigwedge^n H$. (A linear map from a one dimensional vector space to itself
is multiplication by a scalar.)  On the one hand,
\begin{eqnarray*}
\inner{e_1 \wedge \cdots \wedge e_n}{\textstyle{(\bigwedge^n A)}(e_1 \wedge \cdots \wedge e_n)}&=&\inner{e_1 \wedge \cdots \wedge e_n}{\alpha e_1 \wedge \cdots \wedge e_n}\\
&=&\alpha \inner{e_1 \wedge \cdots \wedge e_n}{e_1 \wedge \cdots \wedge e_n}\\
&=&\alpha.
\end{eqnarray*}
On the other hand,
\begin{eqnarray*}
\inner{e_1 \wedge \cdots \wedge e_n}{\textstyle{(\bigwedge^n A)}(e_1 \wedge \cdots \wedge e_n)}&=&\inner{e_1 \wedge \cdots \wedge e_n}{Ae_1 \wedge \cdots \wedge Ae_n}\\
&=&\det A.
\end{eqnarray*}
Thus, $\bigwedge^n A$ is the map $v \mapsto \det(A)v$.\footnote{Michael
Reed and Barry Simon, {\em Methods of Modern Mathematical Physics, volume IV: Analysis of Operators}, p.~321, Lemma 2.}

Still taking $H$ to be $n$-dimensional, take $A \in \mathscr{B}(H)$, and let
$e_1,\ldots,e_n$ be an orthonormal basis for $H$ such that $V_k=\Span\{e_1,\ldots,e_k\}$ and such that
$A(V_k)=V_k$ for $k=1,\ldots,n$; such a basis is obtained using the {\em Schur decomposition} of $A$.


The equality
\begin{eqnarray*}
\tr(\textstyle{\bigwedge^k A})&=&\sum_{1 \leq i_1 < \cdots < i_k \leq n} \inner{e_{i_1} \wedge \cdots \wedge e_{i_k}}{(\textstyle{\bigwedge^k A})(e_{i_1} \wedge \cdots \wedge e_{i_k})}\\
&=&\sum_{1 \leq i_1 < \cdots i_k \leq n} \inner{e_{i_1} \wedge \cdots \wedge e_{i_k}}{Ae_{i_1} \wedge \cdots \wedge Ae_{i_k}}\\
&=&\sum_{1 \leq i_1 < \cdots < i_k \leq n} \det(\inner{e_{i_i}}{Ae_{i_j}})\\
&=&\sum_{1 \leq i_1 < \cdots < i_k \leq n} \lambda_{i_1} \cdots \lambda_{i_k}
\end{eqnarray*}
and the equality
\begin{eqnarray*}
\det(I+A)&=&\inner{e_1\wedge \cdots \wedge e_n}{(I+A)e_1 \wedge \cdots \wedge (I+A)e_n}\\
&=&\det(\inner{e_i}{(I+A)e_j})\\
&=&\prod_{j=1}^n (1+\lambda_j)
\end{eqnarray*}
together give
\[
\det(I+A)=\sum_{j=0}^n \tr(\textstyle{\bigwedge^j}A).
\]

If $H$ is infinite dimensional we define $\det(I+A)$ for $A \in \mathscr{B}_1(H)$ following the above formula.
For this definition to make sense we use the following lemma.\footnote{Michael
Reed and Barry Simon, {\em Methods of Modern Mathematical Physics, volume IV: Analysis of Operators}, p.~323, Lemma 3.} It is stated in Reed and Simon for separable
Hilbert spaces. In reading the proof I don't see how separability is essential to proving the result, but without carefully working out the proof and refreshing myself about the
singular value decomposition, it would be dishonest to assert that the result  is true for nonseparable Hilbert spaces.

\begin{lemma}
Let $H$ be a separable Hilbert space and let $A \in \mathscr{B}_1(H)$. For any $k \in \mathbb{N}$ we have
$\bigwedge^k A \in \mathscr{B}_1(\bigwedge^k H)$, and
\[
\norm{\textstyle{\bigwedge^k A}}_1 = \sum_{i_1<\cdots < i_k} \mu_{i_1} \cdots \mu_{i_k},
\]
and
\[
\norm{\textstyle{\bigwedge^k A}}_1 \leq \frac{\norm{A}_1^k}{k!}.
\]
\end{lemma}



\begin{definition}
If $H$ is a separable Hilbert space and
$A \in \mathscr{B}_1(H)$, we define
\[
\det(I+A) = \sum_{k=0}^\infty \tr(\textstyle{\bigwedge^k A}).
\]
We call  $\det$ the {\em Fredholm determinant on $H$}.
\end{definition}

For a separable Hilbert space,
Reed and Simon also prove that\footnote{Michael
Reed and Barry Simon, {\em Methods of Modern Mathematical Physics, volume IV: Analysis of Operators}, p.~323, Lemma 4.}
\[
|\det(I+A)| \leq \exp(\norm{A}_1), \qquad A \in \mathscr{B}_1(H),
\]
that
\[
\det(I+A)\det(I+B) = \det(I+A+B+AB), \qquad A,B \in \mathscr{B}_1(H),
\]
and that for $A \in \mathscr{B}_1(H)$,
$I+A$ is invertible if and only if $\det (I+A) \neq 0$.\footnote{Michael
Reed and Barry Simon, {\em Methods of Modern Mathematical Physics, volume IV: Analysis of Operators}, p.~325, Theorem XIII.105.}


\end{document}