\documentclass{article}
\usepackage{amsmath,amssymb,mathrsfs,amsthm}
%\usepackage{tikz-cd}
\usepackage{hyperref}
\newcommand{\inner}[2]{\left\langle #1, #2 \right\rangle}
\newcommand{\tr}{\ensuremath\mathrm{tr}\,} 
\newcommand{\Span}{\ensuremath\mathrm{span}} 
\def\Re{\ensuremath{\mathrm{Re}}\,}
\def\Im{\ensuremath{\mathrm{Im}}\,}
\newcommand{\id}{\ensuremath\mathrm{id}} 
\newcommand{\var}{\ensuremath\mathrm{var}} 
\newcommand{\Lip}{\ensuremath\mathrm{Lip}} 
\newcommand{\GL}{\ensuremath\mathrm{GL}} 
\newcommand{\Fix}{\ensuremath\mathrm{Fix}\,} 
\newcommand{\diam}{\ensuremath\mathrm{diam}} 
\newcommand{\sgn}{\ensuremath\mathrm{sgn}\,} 
\newcommand{\lcm}{\ensuremath\mathrm{lcm}} 
\newcommand{\supp}{\ensuremath\mathrm{supp}\,}
\newcommand{\dom}{\ensuremath\mathrm{dom}\,}
\newcommand{\upto}{\nearrow}
\newcommand{\downto}{\searrow}
\newcommand{\norm}[1]{\left\Vert #1 \right\Vert}
\newtheorem{theorem}{Theorem}
\newtheorem{lemma}[theorem]{Lemma}
\newtheorem{proposition}[theorem]{Proposition}
\newtheorem{corollary}[theorem]{Corollary}
\theoremstyle{definition}
\newtheorem{definition}[theorem]{Definition}
\newtheorem{example}[theorem]{Example}
\begin{document}
\title{The left shift map and expanding endomorphisms of the circle}
\author{Jordan Bell\\ \texttt{jordan.bell@gmail.com}\\Department of Mathematics, University of Toronto}
\date{\today}

\maketitle


\section{Sequences}
Let $m \geq 2$ and
let $N_m = \{0,\ldots,m-1\}$, which is a discrete topological space. Let
$I = \mathbb{Z}_{\geq 1}$,
 for $i \in I$ write $A_i = N_m$, and let 
$\nu_i$ be the probability measure on $A_i$ defined by $\nu_i (\{a\})=\frac{1}{m}$ for $a \in A_i$. 
Let
\[
\Sigma_m = \prod_{i \in I} A_i.
\]
Define $\pi_i:\Sigma_m \to A_i$ by $\pi_i(x) = x(i)$. A cylinder set is a subset of $\Sigma_m$ of the form
\[
\prod_{i \in  I} B_i,
\]
where $B_i \subset A_i$ and $\{i \in I: B_i \neq A_i\}$ is finite. 
In other words, a cylinder set is an intersection of finitely many sets of the form $\pi_i^{-1}(B_i)$ with
$B_i \subset A_i$. Let $\mathscr{C}$ be the collection of cylinder sets. The product $\sigma$-algebra
is the $\sigma$-algebra generated by $\mathscr{C}$. 

Assign $\Sigma_m$ the product topology, the initial
topology for $\{\pi_i: i \in I\}$. Because $A_i$ is finite, with the discrete topology it is compact and so
$\Sigma_m$ is compact. 
The discrete topology on $A_i$ is induced by the metric $d_i(a,b) = |a-b|$. For $x,y \in \Sigma_m$ let
\[
d(x,y) = \sum_{i \in \mathbb{Z}_{\geq 1}} \frac{d_i(x(i),y(i))}{m^i} = \sum_{i \in \mathbb{Z}_{\geq 1}} \frac{|x(i)-y(i)|}{m^i}.
\]
It is a fact that $d$ is a metric on $\Sigma_m$ that induces the product topology.\footnote{cf. \url{http://individual.utoronto.ca/jordanbell/notes/uniformmetric.pdf}}


It is a fact that the Borel $\sigma$-algebra of $\Sigma_m$ is 
equal to the product $\sigma$-algebra. (This is true for any countable product of second-countable topological
spaces.\footnote{\url{http://individual.utoronto.ca/jordanbell/notes/kolmogorov.pdf}})
Let $\mu_m = \bigotimes_{i \in I} \nu_i$, the product measure:\footnote{\url{http://individual.utoronto.ca/jordanbell/notes/productmeasure.pdf}}
for
$\prod_{i \in I} B_i \in \mathscr{C}$,
\[
\mu_m \left(\prod_{i \in I} B_i\right) = \prod_{i \in I} \nu_i(B_i).
\]


\section{The left shift}
Define
$\sigma:\Sigma_m \to \Sigma_m$ by
\[
(\sigma x)(i) = x(i+1),\qquad i \in I.
\]
For $\prod_{i \in I} B_i \in \mathscr{C}$, let $C_1=N_m$ and otherwise let $C_i=B_{i-1}$. Then
\[
\sigma^{-1} \left( \prod_{i \in I} B_i \right) = \prod_{i \in I} C_i.
\]
This shows that $\sigma$ is continuous. 
Moreover, this shows that for $C \in \mathscr{C}$,
\[
(\sigma_* \mu_m)(C) = \mu_m(\sigma^{-1}(C)) = \mu_m(C).
\]
It follows that
\begin{equation}
\sigma_* \mu_m = \mu_m.
\label{sigmamum}
\end{equation}
That is,
$\sigma$ is measure-preserving.

For $k \geq 1$,
$\Fix(\sigma^k)$ is the set of those $x \in \Sigma_m$ such that 
for each $i \in I$, $x(i+k)=x(i)$. Check that
$|\Fix(\sigma^k)| = m^k$. 





\section{The circle}
Let $\mathbb{T} = \mathbb{R} / \mathbb{Z}$, which is a compact abelian group using addition, and let
$\mu$ be the Haar measure with $\mu(\mathbb{T})=1$. 
For $m \in \mathbb{Z}_{\geq 1}$ let $E_m:\mathbb{T} \to \mathbb{T}$ be
\[
E_m t =mt,
\]
which is an endomorphism of the topological group $\mathbb{T}$: $E_m$ is continuous, and for
$s,t \in \mathbb{T}$, $E_m(s+t) = E_m s+E_m t$. 


Define $\phi:\Sigma_m \to \mathbb{T}$ by
\[
\phi(x) = \sum_{i \geq 1} \frac{x(i)}{m^i} + \mathbb{Z}.
\]
$\phi$ is continuous and surjective.
For $x \in \Sigma_m$,
\begin{align*}
(E_m \circ \phi)(x) &= \sum_{i \geq 1} m\cdot \frac{x(i)}{m^i} + \mathbb{Z}\\
&= \sum_{i \geq 2}  \frac{x(i)}{m^{i-1}} + \mathbb{Z}\\
&=\sum_{i \geq 1}  \frac{x(i+1)}{m^i} + \mathbb{Z}\\
&=(\phi \circ \sigma)(x),
\end{align*}
which means that
\begin{equation}
E_m \circ \phi = \phi \circ \sigma.
\label{semiconjugate}
\end{equation}
Thus $E_m:\mathbb{T} \to \mathbb{T}$ and 
$\sigma:\Sigma_m \to \Sigma_m$ are 
\textbf{topologically semiconjugate}.

Check that 
\begin{equation}
\phi_* \mu_m = \mu.
\label{phimum}
\end{equation}
Using \eqref{sigmamum},  \eqref{semiconjugate}, and \eqref{phimum},
\begin{align*}
{E_m}_*\mu &={E_m}_*(\phi_* \mu_m)\\
&=(E_m \circ \phi)_* \mu_m\\
&=(\phi \circ \sigma)_* \mu_m\\
&=\phi_* (\sigma_* \mu_m)\\
&=\phi_* \mu_m\\
&=\mu.
\end{align*}
This means that $E_m:\mathbb{T} \to \mathbb{T}$ is measure-preserving.


For $k \geq 1$,
\[
\phi \circ \sigma^k = (E_m \circ \phi) \circ \sigma^{k-1} = \cdots = E_m^k \circ \phi. 
\]
If $x \in \Fix(\sigma^k)$, then
\[
\phi(x) = (\phi \circ \sigma^k)(x) = (E_m^k \circ \phi)(x),
\]
hence $\phi(x) \in \Fix(E_m^k)$. 
Now, let $z_0(i)=0$ for all $i$ and let $z_1(i)=m-1$ for all $i$; $z_0,z_1 \in \Fix(\sigma^k)$. $\phi(z_0)=0 + \mathbb{Z}$ and 
$\phi(z_1) = \sum_{i \geq 1} \frac{m-1}{m^i} + \mathbb{Z} = 1 + \mathbb{Z} = 0+\mathbb{Z}$, so $\phi(z_0)=\phi(z_1)$. 
Check that if $x,y \in \Fix(\sigma^k)$ are distinct and $\{x,y\} \neq \{z_0,z_1\}$ then $\phi(x) \neq \phi(y)$. 
It follows that\footnote{Michael Brin and Garrett Stuck, {\em Introduction to Dynamical Systems}, p.~6, \S 1.3.}
\[
|\Fix(E_m^k)| = |\Fix(\sigma^k)| - 1 = m^k-1.
\]
 



\end{document}