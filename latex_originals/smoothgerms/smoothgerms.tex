\documentclass{article}
\usepackage{amsmath,amssymb,mathrsfs,amsthm}
%\usepackage{tikz-cd}
\usepackage{hyperref}
\newcommand{\inner}[2]{\left\langle #1, #2 \right\rangle}
\newcommand{\tr}{\ensuremath\mathrm{tr}\,} 
\newcommand{\Span}{\ensuremath\mathrm{span}} 
\def\Re{\ensuremath{\mathrm{Re}}\,}
\def\Im{\ensuremath{\mathrm{Im}}\,}
\newcommand{\id}{\ensuremath\mathrm{id}} 
\newcommand{\var}{\ensuremath\mathrm{var}} 
\newcommand{\ev}{\ensuremath\mathrm{ev}} 
\newcommand{\Lip}{\ensuremath\mathrm{Lip}} 
\newcommand{\GL}{\ensuremath\mathrm{GL}} 
\newcommand{\Hess}{\ensuremath\mathrm{Hess}\,} 
\newcommand{\diam}{\ensuremath\mathrm{diam}} 
\newcommand{\sgn}{\ensuremath\mathrm{sgn}\,} 
\newcommand{\lcm}{\ensuremath\mathrm{lcm}} 
\newcommand{\supp}{\ensuremath\mathrm{supp}\,}
\newcommand{\dom}{\ensuremath\mathrm{dom}\,}
\newcommand{\upto}{\nearrow}
\newcommand{\downto}{\searrow}
\newcommand{\norm}[1]{\left\Vert #1 \right\Vert}
\newtheorem{theorem}{Theorem}
\newtheorem{lemma}[theorem]{Lemma}
\newtheorem{proposition}[theorem]{Proposition}
\newtheorem{corollary}[theorem]{Corollary}
\theoremstyle{definition}
\newtheorem{definition}[theorem]{Definition}
\newtheorem{example}[theorem]{Example}
\begin{document}
\title{Germs of smooth functions}
\author{Jordan Bell\\ \texttt{jordan.bell@gmail.com}\\Department of Mathematics, University of Toronto}
\date{\today}

\maketitle

\section{Sheafs}
Let $M=\mathbb{R}^m$. 
For an open set $U$ in $M$, write $\mathcal{F}(U) = C^\infty(U)$, which is a commutative ring with unity $1_M(x)=1$. 
For open sets $V \subset U$ in $M$, define $r_{U,V}:\mathcal{F}(U) \to \mathcal{F}(V)$ by
$r_{U,V} f = f|_V$, which is a homomorphism of rings. 
$\mathcal{F}$ is a \textbf{presheaf}, a contravariant functor from the category of
open  sets in $M$ to the category of commutative unital rings. For $\mathcal{F}$ to be a \textbf{sheaf} means
the following:
\begin{enumerate}
\item If $U_i$, $i \in I$, is an open cover of an open set $U$ and if $f,g \in \mathcal{F}(U)$ satisfy
$r_{U,U_i} f = r_{U,U_i} g$ for all $i \in I$, then $f=g$.
\item If $U_i$, $i \in I$, is an open cover of an open set $U$ and for each $i \in I$ there is some $f_i \in \mathcal{F}(U_i)$ such that
for all $i,j \in I$, $r_{U_i, U_i \cap U_j} f_i = r_{U_j, U_i \cap U_j} f_j$, then there is some $f \in \mathcal{F}(U)$ such that 
$r_{U,U_i} f = f_i$ for each $i \in I$.
\end{enumerate}
For the first condition, let $p \in U$. As $U_i$ is an open cover of $U$, there is some $i$ for which $p \in U_i$. 
As $f|_{U_i} = g|_{U_i}$, $f(p)=g(p)$. Therefore $f=g$. For the second condition, 
let $p \in U$. If $p \in U_i$ and $p \in U_j$, then $f_i(p) = f_j(p)$. This shows that it makes
sense to define $f:U \to \mathbb{R}$ by $f(p)=f_i(p)$, for any $i$ such that $p \in U_i$. 
Then $f|_{U_i} = f_i$, which implies that $f \in \mathcal{F}(U)$: for each $p \in U$, there is some
open neighborhood $U_i$ of $p$ on which $f$ is smooth. Therefore
$\mathcal{F}$ is a sheaf.


\section{Stalks and germs}
For $p \in M$, let $\mathcal{U}_p$ be the set of open neighborhoods of $p$. For $U,V \in \mathcal{U}_p$, say $U \leq V$
when $V \subset U$. For $U \leq V \leq W$ and $f \in \mathcal{F}(U)$,
\[
(r_{V,W} \circ r_{U,V}) (f) = r_{V,W} f|_V = f_W = r_{U,W} f.
\]
For $f \in \mathcal{F}(U)$ and $g \in \mathcal{F}(V)$, say $f \sim_p g$ if
there is some $W \in \mathcal{U}_p$, $W \geq U$, $W \geq V$, such that  
$r_{U,W} f = r_{V,W} g$. 
Let 
\[
\mathcal{R}_p =   \bigsqcup_{U \in \mathcal{U}_p} \mathcal{F}(U),
\]
and
let $\mathcal{F}_p$ be the direct limit of the direct system $\mathcal{F}(U)$, $r_{U,V}$ of commutative unital rings: 
\[
\mathcal{F}_p = \mathcal{R}_p / \sim_p.
\]
We call $\mathcal{F}_p$ the \textbf{stalk of $\mathcal{F}$ at $p$}.
An element of $\mathcal{F}_p$ is called a \textbf{germ of $\mathcal{F}$ at $p$}.
In other words, 
for $f \in \mathcal{R}_p$, let
$[f]_p$ be the set of those $g \in \mathcal{R}_p$ such that 
$f \sim_p g$, equivalently, 
$f|_{U_f \cap U_g} = g|_{U_f \cap U_g}$. A germ of $\mathcal{F}$ at $p$ is such an equivalence class $[f]_p$, and
\[
\mathcal{F}_p = \left\{ [f]_p: f \in \mathcal{R}_p \right\}.
\]


\section{Maximal ideals}
For $p \in M$, and $f,g \in \mathcal{R}_p$ with $f \sim_p g$, $f(p)=g(p)$. Thus it makes sense to define
$\ev_p:\mathcal{F}_p \to \mathbb{R}$ by $\ev_p [f]_p = f(p)$. Now, for $[f]_p,[g]_p \in \mathcal{F}_p$, 
\[
\ev_p ([f]_p + [g]_p) = \ev_p([f+g]_p) = (f+g)(p) = f(p)+g(p) = \ev_p [f]_p + \ev_p [g]_p,
\]
\[
\ev_p ([f]_p [g]_p) = \ev_p ([fg]_p) = (fg)(p) = f(p)g(p) = \ev_p [f]_p \cdot \ev_p [g]_p,
\]
$\ev_p [1_M]_p = 1$. This means that $\ev_p:\mathcal{F}_p \to \mathbb{R}$ is a homomorphism of unital rings. It is straightforward
that $\ev_p$ is surjective. Write
$\mathfrak{m}_p = \ker \ev_p$. By the first isomorphism theorem, there is an isomorphism of unital rings
$\mathcal{F}_p / \mathfrak{m}_p \to \mathbb{R}$. Therefore
$\mathfrak{m}_p$ is a maximal ideal in $\mathcal{F}_p$. 
Now, if $[f]_p \in \mathcal{F}_p \setminus \mathfrak{m}_p$ then 
$\ev_p [f]_p \neq 0$, hence $f(p) \neq 0$. Then there is some $U \in \mathcal{U}_p$ such that
$f(x) \neq 0$ for $x \in U$, and $(1/f)(x) = \frac{1}{f(x)}$ belongs to $\mathcal{F}(U)$. Then
$[1/f]_p \in \mathcal{F}_p$ and $[f]_p \cdot [1/f]_p = [f \cdot 1/f]_p = [1_M]_p$, which shows that if
$[f]_p \in \mathcal{F}_p \setminus \mathfrak{m}_p$ then $[f]_p$ has an inverse $[1/f]_p$ in $\mathcal{F}_p$. 
This means $\mathfrak{m}_p$ is the set of noninvertible elements of $\mathcal{F}_p$, which means that
$\mathcal{F}_p$ is a \textbf{local ring}. 

For $1 \leq i \leq m$ define the coordinate function $x^i:M \to \mathbb{R}$ by $x^i(p) = p_i$, which belongs to
$\mathcal{F}(M)$. Because $\ev_0 x^i = 0$, $[x^i]_0 \in \mathfrak{m}_0$. We prove 
\textbf{Hadamard's lemma},
that the ring $\mathfrak{m}_0$ is generated
by the germs of the coordinate functions at $0$.\footnote{Liviu Nicolaescu, {\em An Invitation to Morse Theory}, second ed.,
p.~14, Lemma 1.13.}

\begin{lemma}[Hadamard's lemma]
The ideal $\mathfrak{m}_0$ is generated by the set $\{[x^i]_0: 1 \leq i \leq m\}$.
\label{hadamard}
\end{lemma}
\begin{proof}
Let $[f]_0 \in \mathfrak{m}_0$ with $f \in \mathcal{F}(B_r)$ for some $r>0$. For
$y \in B_r$, using the fundamental theorem of calculus and using the chain rule,
\[
f(y) = f(y) - f(0) = \int_0^1 \frac{d}{ds} f(sy) ds = \int_0^1 \sum_{i=1}^m x^i(y) (\partial_i f)(sy) ds
=\sum_{i=1}^m x^i(y) u_i(y),
\]
and $u_i \in \mathcal{F}(B_r)$. This means that
$[f]_0 = \sum_{i=1}^m [x^i]_0 [u_i]_0$, which shows that $[f]_0$ belongs to the ideal generated by 
the set  $\{[x^i]_0: 1 \leq i \leq m\}$. 
\end{proof}



For a  multi-index $\alpha \in \mathbb{Z}_{\geq 0}^m$, write
\[
|\alpha| = \sum_{i=1}^m \alpha_i,\qquad \alpha! = \alpha_1! \cdots \alpha_m!
\]
and
\[
\partial^\alpha = \partial_1^{\alpha_1} \cdots \partial_m^{\alpha_m},\qquad x^\alpha = (x^1)^{\alpha_1} \cdots (x^m)^{\alpha_m},
\]
and say $\alpha \leq \beta$ if $\alpha_i \leq \beta_i$ for each $i$.
We shall use the fact that
\[
\partial^\alpha x^\beta = \begin{cases}
\frac{\beta!}{(\beta-\alpha)!} x^{\beta-\alpha}&\alpha \leq \beta\\
0&\textrm{otherwise}.
\end{cases}
\]



\begin{lemma}
For $f \in \mathcal{R}_0$, if $(\partial^\alpha f)(0) = 0$ for all $|\alpha|<k$, then $[f]_0 \in \mathfrak{m}_0^k$. 
\end{lemma}
\begin{proof}
For $k=1$, if $(\partial^\alpha f)(0)=0$ for $\alpha=(0,\ldots,0)$ then $\ev_0 f = f(0)=0$, hence
$[f]_0 \in \mathfrak{m}_0$. 
Suppose the claim is true for some
$k \geq 1$, and suppose that $f \in \mathcal{R}_0$ and that $(\partial^\alpha f)(0)=0$ for all $|\alpha|<k+1$.
A fortiori, $(\partial^\alpha f)(0)=0$ for all $|\alpha|<k$ and then by the induction hypothesis we get
$[f]_0 \in \mathfrak{m}_0^k$. Now, 
Lemma \ref{hadamard} tells us that the ideal $\mathfrak{m}_0$ is generated by the set
$\{[x^i]_0: 1 \leq i \leq m\}$, and then
the product ideal $\mathfrak{m}_0^k$ is generated by the
set 
\begin{align*}
\{[x^{i_1}]_0 \cdots [x^{i_k}]_0 : 1 \leq i_1,\ldots,i_k \leq m\}&=\{[x^{i_1} \cdots x^{i_k}]_0: 1 \leq i_1, \ldots, i_k \leq m\}\\
&=\{[x^\alpha]_0: |\alpha|=k\},
\end{align*}
for $x^\alpha = (x^1)^{\alpha_1} \cdots (x^m)^{\alpha_m}$. 
As $[f]_0 \in \mathfrak{m}^k$, there are $[u_\alpha]_0 \in \mathcal{F}_0$, $|\alpha|=k$, such that
\[
[f]_0 = \sum_{|\alpha|=k} [u_\alpha]_0 [x^\alpha]_0.
\]
For $|\alpha|=k$, on some set in $\mathcal{U}_0$, using the Leibniz rule,
\[
\partial^\alpha f = \sum_{|\beta|=k} \partial^\alpha (u_\beta x^\beta)
=\sum_{|\beta|=k} \sum_{\gamma \leq \alpha} \binom{\alpha}{\gamma} (\partial^{\alpha-\gamma} u_\beta) (\partial^{\gamma} x^\beta).
\]
And for $\gamma \neq \beta$, $(\partial^\gamma x^\beta)(0) = 0$, so 
\[
\partial^\alpha f \in u_\alpha \partial^\alpha x^\alpha + h,\qquad [h]_0 \in \mathfrak{m}_0.
\]
But $(\partial^\alpha f)(0)=0$, so $u_\alpha(0)=0$, which means that $u_\alpha \in \mathfrak{m}_0$. And
\[
[x^\alpha]_0 = [x^1]_0^{\alpha_1} \cdots [x^m]_0^{\alpha_m} \in \mathfrak{m}_0^{|\alpha|} = \mathfrak{m}_0^k,
\]
so $[u_\alpha]_0 [x^\alpha]_0 \in \mathfrak{m}_0^{k+1}$, showing that $[f]_0 \in \mathfrak{m}_0^{k+1}$. This completes the proof
by induction.
\end{proof}




\section{Hessians}
For an open set $U$ in $\mathbb{R}^m$ and $\phi \in \mathcal{F}(U)$, 
$\phi':U \to \mathscr{L}(\mathbb{R}^m,\mathbb{R})$, and $\nabla \phi:U \to \mathbb{R}^m$ satisfies
\[
\inner{\nabla \phi(x)}{v} = \phi'(x)(v),\qquad x \in U,\quad v \in \mathbb{R}^m.
\]
$x \in U$ is a \textbf{critical point of $\phi$} if
$\phi'(x)=0$, equivalently
$\nabla \phi(x)=0$.
Define $\Hess \phi:U \to \mathscr{L}(\mathbb{R}^m,\mathbb{R}^m)$ by
\[
\Hess \phi = (\nabla \phi)'.
\]
This satisfies\footnote{\url{http://individual.utoronto.ca/jordanbell/notes/gradienthilbert.pdf}}
\[
\phi''(x)(u)(v) = \inner{v}{\Hess \phi(x)(u)},\qquad x \in U,\qquad u,v, \in \mathbb{R}^m.
\]
A critical point $x$ of $\phi$ is called \textbf{nondegenerate} if $\Hess \phi(x)$ is invertible
in $\mathscr{L}(\mathbb{R}^m,\mathbb{R}^m)$. 



For $\phi \in \mathcal{R}_p$, let $J_\phi$ be the ideal in the ring $\mathcal{F}_p$ generated by the set
\[
\{[\partial_i \phi]_p: 1 \leq i \leq m\}.
\]
We call $J_\phi$ the \textbf{Jacobian ideal of $\phi$ at $p$}.
If $p$ is a critical point of $\phi$, then
$(\partial_i \phi)(p)=0$ for each $i$, hence
$[\partial_i \phi]_p \in \mathfrak{m}_p$ for each $i$.

If $0$ is a nondegenerate critical point of $\phi$, we 
prove that $\mathfrak{m}_0 \subset J_\phi$.\footnote{Liviu Nicolaescu, {\em An Invitation to Morse Theory}, second ed.,
p.~15, Lemma 1.15.}

\begin{theorem}
Let $U$ be an open set in $\mathbb{R}^m$ containing $0$ and let $\phi \in \mathcal{F}(U)$.
If $0$ is a nondegenerate critical point of $\phi$, then
$J_\phi = \mathfrak{m}_0$.
\end{theorem}
\begin{proof}
Let $f=\nabla \phi$, which is a smooth function $U \to \mathbb{R}^m$. 
Because $0$ is a nondegenerate critical point of $\phi$,
$f'(0)$ is invertible in $\mathscr{L}(\mathbb{R}^m,\mathbb{R}^m)$ and 
hence by the \textbf{inverse function theorem},\footnote{Serge Lang,
{\em Real and Functional Analysis}, third ed., p.~361, chapter XIV, Theorem 1.2.}
$f$ is a local $C^\infty$ isomorphism at $x$: 
there is some open set $V$, $x \in V$ and $V \subset U$, such that $W=f(V)$ is open in $\mathbb{R}^m$,
and there is a smooth function $g:W \to V$ such that
$g \circ f = \id_V$ and $f \circ g = \id_W$. 
\end{proof}







\end{document}