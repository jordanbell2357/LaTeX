\documentclass{article}
\usepackage{amsmath,amssymb,graphicx,subfig,mathrsfs,amsthm}
%\usepackage{tikz-cd}
%\usepackage{hyperref}
\newcommand{\inner}[2]{\left\langle #1, #2 \right\rangle}
\newcommand{\tr}{\ensuremath\mathrm{tr}\,} 
\newcommand{\Span}{\ensuremath\mathrm{span}} 
\def\Re{\ensuremath{\mathrm{Re}}\,}
\def\Im{\ensuremath{\mathrm{Im}}\,}
\newcommand{\id}{\ensuremath\mathrm{id}} 
\newcommand{\rank}{\ensuremath\mathrm{rank\,}} 
\newcommand{\co}{\ensuremath\mathrm{co}\,} 
\newcommand{\cco}{\ensuremath\overline{\mathrm{co}}\,}
\newcommand{\supp}{\ensuremath\mathrm{supp}}
\newcommand{\epi}{\ensuremath\mathrm{epi}\,}
\newcommand{\lsc}{\ensuremath\mathrm{lsc}\,}
\newcommand{\ext}{\ensuremath\mathrm{ext}\,}
\newcommand{\cl}{\ensuremath\mathrm{cl}\,}
\newcommand{\dom}{\ensuremath\mathrm{dom}\,}
\newcommand{\LSC}{\ensuremath\mathrm{LSC}}
\newcommand{\USC}{\ensuremath\mathrm{USC}}
\newcommand{\upto}{\nearrow}
\newcommand{\downto}{\searrow}
\newcommand{\norm}[1]{\left\Vert #1 \right\Vert}
\newtheorem{theorem}{Theorem}
\newtheorem{lemma}[theorem]{Lemma}
\newtheorem{proposition}[theorem]{Proposition}
\newtheorem{corollary}[theorem]{Corollary}
\theoremstyle{definition}
\newtheorem{definition}[theorem]{Definition}
\newtheorem{example}[theorem]{Example}
\begin{document}
\title{Semicontinuous functions and convexity}
\author{Jordan Bell\\ \texttt{jordan.bell@gmail.com}\\Department of Mathematics, University of Toronto}
\date{\today}

\maketitle

\section{Lattices}
If $(A,\leq)$ is a partially ordered set and $S$ is a subset of $A$, a {\em supremum of $S$} is an upper bound that is $\leq$ any upper
bound of $S$, and an {\em infimum of $S$} is a lower bound that is $\geq$ any lower bound of $S$. 
Because a partial order is antisymmetric, if a supremum exists it is unique, and we denote it by $\bigvee S$, and if an infimum exists it is unique, and
we denote it by $\bigwedge S$.
 If $(A,\leq)$ is a lattice, then one proves by induction that both the supremum and the infimum exist for every finite nonempty subset of $A$. 
Vacuously, every element of a partially ordered set is an upper bound for $\emptyset$ and a lower bound for $\emptyset$. Thus, if $\emptyset$ has a supremum $x$
then $x \leq y$ for all $y \in A$, and if $\emptyset$ has an infimum $x$ then $x \geq y$ for all $y \in A$. That is,
\[
\bigvee \emptyset = \bigwedge A, \qquad \bigwedge \emptyset =  \bigvee A.
\]


If $X$ is a set and $R \subseteq [-\infty,\infty]$, the set $R^X$ is partially ordered where $f \leq g$ if
$f(x) \leq g(x)$ for all $x \in X$. Moreover,  $R^X$  is a lattice: 
\[
(f \vee g)(x) = \max \{f(x),g(x)\}, \qquad (f \wedge g)(x) = \min\{ f(x),g(x)\}.
\]


\section{Urysohn's lemma}
A Hausdorff topological space $(X,\tau)$ is said to be  {\em normal} if for every pair of disjoint closed sets $E,F$ there are disjoint open sets $U,V$ with
$E \subset U$ and $F \subset V$. Every metrizable space is a normal topological space, but there are normal topological spaces that are not metrizable. 
A useful fact about normal topological spaces is {\em Urysohn's lemma}:\footnote{Gert K. Pedersen,
{\em Analysis Now}, revised printing, p.~24, Theorem 1.5.6.} For each pair of disjoint nonempty closed sets $E,F$ there is a continuous function
$f:X \to [0,1]$ such that $f(E)=0$ and $f(F)=1$.

A locally compact Hausdorff space need not be normal. For example, the real numbers with the {\em rational sequence topology} is Hausdorff and locally compact but is not normal. We say that a topological space
is {\em $\sigma$-compact} if it is the union of countably many compact subsets. The following lemma states that if a locally compact Hausdorff space is $\sigma$-compact
then it is normal.\footnote{Gert K. Pedersen, {\em Analysis Now}, revised printing, p.~39, Proposition 1.7.8.} 

\begin{lemma}
If a locally compact Hausdorff space is $\sigma$-compact, then it is normal.
\end{lemma}

{\em Urysohn's metrization theorem} states that if a topological space is normal and is second-countable (its topology has a countable basis) then it is metrizable. However, a metric space need not be second-countable: a metric space is second-countable precisely when it is separable, and hence
the converse of Urysohn's metrization theorem is false. The following lemma shows that a second-countable locally compact Hausdorff space is $\sigma$-compact,
and hence metrizable by Urysohn's metrization theorem.

\begin{lemma}
If a locally compact Hausdorff space is second-countable, then it is $\sigma$-compact.
\end{lemma}
\begin{proof}
Let $(X,\tau)$ be a second-countable locally compact Hausdorff space. Because it is second-countable, there is a countable subset $\mathscr{B}$ of $\tau$ that is a basis for
$\tau$. 
If $x \in X$, then because $X$ is locally compact there is an open set $U$ containing $x$ which is itself contained in a compact set $K$, and then there is some $V \in \mathscr{B}$
such that $x \in V \subseteq U$. The closure $\overline{V}$ of $V$ is contained in $K$ and hence $\overline{V}$ is compact.
Defining $\mathscr{B}'$ to be those $V \in \mathscr{B}$ such that $\overline{V}$ is compact,
it follows that $\mathscr{B}'$ is a basis for $\tau$. The closures of the elements of $\mathscr{B}'$ are countably many compact sets whose union is equal to $X$, showing that
$X$ is $\sigma$-compact.
\end{proof}


\section{Lower semicontinuous functions}
If $(X,\tau)$ is a topological space, then $f:X \to [-\infty,\infty]$ is said to be {\em lower semicontinuous} if $t \in \mathbb{R}$ implies that
$f^{-1}(t,\infty] \in \tau$. We say that $f$ is {\em finite} if $-\infty<f(x) < \infty$ for all $x \in X$. 

If $A \subseteq X$ and $t \in \mathbb{R}$, then
\[
\chi_A^{-1}(t,\infty]=\begin{cases}
X&t < 0\\
A&0 \leq t< 1\\
\emptyset&t\geq1
\end{cases}
\]
We see that the characteristic function of a set is lower semicontinuous if and only if the set is open.

The following theorem characterizes lower semicontinuous functions in terms of nets.\footnote{Gert K. Pedersen,
{\em Analysis Now}, revised printing, p.~26, Proposition 1.5.11.}

\begin{theorem}
If $(X,\tau)$ is a topological space and $f:X \to [-\infty,\infty]$ is a function, then $f$ is lower semicontinuous if and only if 
 $(x_\alpha)_{\alpha \in I}$ being a convergent net in $X$ implies that 
\[
f(\lim x_\alpha) \leq \liminf f(x_\alpha).
\]
\label{liminf}
\end{theorem}
\begin{proof}
Suppose that  $f$ is lower semicontinuous and $x_\alpha \to x$. Say  $t<f(x)$.  Because $f$ is lower semicontinuous,
$f^{-1}(t,\infty] \in \tau$. As $x \in f^{-1}(t,\infty]$ and $x_\alpha \to x$, there is some $\alpha_t$ such that 
$\alpha \geq \alpha_t$ implies $x_\alpha \in  f^{-1}(t,\infty]$. That is, if $\alpha \geq \alpha_t$ then
$f(x_\alpha)>t$. This implies that $\liminf f(x_\alpha) \geq t$. But this is true for all $t<f(x)$, hence
$\liminf f(x_\alpha) \geq f(x)$.

Suppose that $x_\alpha \to x$ implies that $f(x) \leq \liminf f(x_\alpha)$. Let $t \in \mathbb{R}$ and let $F=f^{-1}(-\infty,t]$. 
If $x \in \overline{F}$, then there is a net $x_\alpha \in F$ with $x_\alpha \to x$. 
As $x_\alpha \to x$, by hypothesis
$f(x) \leq \liminf f(x_\alpha)$.
By definition of $F$ we have $f(x_\alpha) \leq t$ for each $\alpha$, and hence $f(x) \leq t$.
This means that
$x \in F$, showing that $F$ is closed and so the complement of $F$ is open. But the complement of $F$ is 
$f^{-1}(t,\infty]$, showing that $f$ is lower semicontinuous.
\end{proof}

Let $\LSC(X)$ be the set of all lower semicontinuous functions $X \to [-\infty,\infty]$. $\LSC(X)$ is a partially ordered set:
$f \leq g$ means that $f(x) \leq g(x)$ for all $x \in X$. The following theorem shows that $\LSC(X)$ is a  lattice that contains the supremum of each of its subsets.\footnote{Gert K. Pedersen, {\em Analysis Now}, revised printing, p.~27, Proposition 1.5.12.} 


\begin{theorem}
If $(X,\tau)$ is a topological space, then $\LSC(X)$ is a  lattice, and if $\mathscr{F} \subseteq \LSC(X)$ then $g:X \to [-\infty,\infty]$ defined by
\[
g(x)=\sup_{f \in \mathscr{F}} f(x), \qquad x \in X
\]
belongs to $\LSC(X)$.
\label{LSClattice}
\end{theorem}
\begin{proof}
If $\mathscr{F}=\emptyset$, then $g$ is the constant function $x \mapsto -\infty$, which is lower
semicontinuous as for any $t \in \mathbb{R}$, the inverse image of $(t,\infty]$ is $\emptyset$, which belongs to $\tau$. 
Otherwise, let
 $t \in \mathbb{R}$. Saying  $x \in g^{-1}(t,\infty]$
 means that $g(x)>t$, and with $\epsilon=g(x)-t$ there is some some $f \in \mathscr{F}$ satisfying $f(x)>g(x)-\epsilon=t$. Therefore, if $x \in g^{-1}(t,\infty]$ then
 $x \in \bigcup_{f \in \mathscr{F}} f^{-1}(t,\infty]$. On the other hand, if $x \in \bigcup_{f \in \mathscr{F}} f^{-1}(t,\infty]$, then there is some $f \in \mathscr{F}$ with $f(x)>t$, and hence
 $g(x)>t$.
 Therefore
 \[
  g^{-1}(t,\infty] = \bigcup_{f \in \mathscr{F}} f^{-1}(t,\infty].
 \]
 But each $f \in \mathscr{F}$ is lower semicontinuous so the right-hand side is a union of elements of $\tau$, and hence
 $g^{-1}(t,\infty] \in \tau$, showing that 
$g$ is lower semicontinuous.
Therefore every subset of $\LSC(X)$ has a supremum.

Suppose that $f_1,f_2 \in \LSC(X)$, and define $f:X \to [-\infty,\infty]$ by 
\[
f(x)=\min\{f_1(x),f_2(x)\}, \qquad x \in X.
\]
For $t \in \mathbb{R}$, it is apparent that
\[
f^{-1}(t,\infty] = f_1^{-1}(t,\infty] \cap f_2^{-1}(t,\infty].
\]
As $f_1$ and $f_2$ are each lower semicontinuous, these two inverse images are each open sets, and so their intersection is an open set. Therefore $f$ is lower semicontinuous,
showing that $\LSC(X)$ is a lattice.
\end{proof}

One is sometimes interested in lower semicontinuous functions that do not take the value $-\infty$. As the following theorem
shows, the sum of two lower semicontinuous functions that do not take the value $-\infty$ is also a lower semicontinuous function.

\begin{theorem}
If $X$ is a topological space, if $f,g \in \LSC(X)$, and if $f,g > -\infty$, then $f+g \in \LSC(X)$, and if $r>0$ and $f \in \LSC(X)$ then $rf \in \LSC(X)$.
\end{theorem}
\begin{proof}
If $(x_\alpha)_{\alpha \in I}$ is a net that converges to $x \in X$, then, by Theorem \ref{liminf},
\[
(f+g)(x) \leq \liminf f(x_\alpha) + \liminf g(x_\alpha) \leq \liminf f(x_\alpha)+g(x_\alpha) = \liminf (f+g)(x_\alpha),
\]
and by Theorem \ref{liminf} this tells us that $f+g$ is lower semicontinuous. As well,
\[
(rf)(x)=rf(x) \leq r \liminf f(x_\alpha) = \liminf rf(x_\alpha) = \liminf (rf)(x_\alpha),
\]
showing that $rf \in \LSC(X)$.
\end{proof}



The following theorem shows that if $f_n \in \LSC(X)$ are each finite and converge uniformly on $X$ to $f \in \mathbb{R}^X$, then $f \in \LSC(X)$.\footnote{Gert K. Pedersen, {\em Analysis Now}, revised printing, p.~27, Proposition 1.5.12.}


\begin{theorem}
If $(X,\tau)$ is a topological space, if $f_n \in \LSC(X)$ are finite, and if $f_n$ converge uniformly in $X$ to $f \in \mathbb{R}^X$, then $f \in \LSC(X)$.
\end{theorem}
\begin{proof}
If $\epsilon>0$, then there is some $N$ such that $n \geq N$ and $x \in X$ imply that 
$|f_n(x)-f(x)| < \epsilon$. Define
\[
\delta_n = \sup \{|f_n(x)-f(x)|:x \in X\}.
\]
Thus, for all $\epsilon>0$ there is some $N$ such that $n \geq N$ implies that $\delta_n \leq \epsilon$. If $(x_\alpha)_{\alpha \in I}$ is a convergent net in $X$, then, for all $n \geq N$,
\begin{eqnarray*}
f(\lim x_\alpha)&\leq&\delta_n + f_n(\lim x_\alpha)\\
&\leq&\delta_n+\liminf f_n(x_\alpha)\\
&\leq&2\delta_n + \liminf f(x_\alpha)\\
&\leq&2\epsilon+\liminf f(x_\alpha).
\end{eqnarray*}
This is true for all $\epsilon$, so we get 
\[
f(\lim x_\alpha) \leq \liminf f(x_\alpha),
\]
and therefore $f$ is lower semicontinuous.
\end{proof}


The following theorem shows in particular that on a normal topological space $X$, any finite nonnegative lower semicontinuous function is the the supremum of the set of all continuous
functions
$X \to \mathbb{R}$ that are dominated by it.\footnote{Gert K. Pedersen, {\em Analysis Now}, revised printing, p.~27, Proposition 1.5.13.}
To say that continuous functions $X \to [0,1]$ {\em separate points and closed sets} means that if $x \in X$ and $F$ is a disjoint closed set, then there is a continuous function
$g:X \to [0,1]$ such that $g(x)=1$ and $g(F)=0$. Urysohn's lemma states that in a normal topological space continuous functions separate closed sets, so in particular they separate
points and closed sets.


\begin{theorem}
If $(X,\tau)$ is a  topological space such that continuous functions $X \to [0,1]$ separate points and closed sets, and $f \geq 0$ is a finite  lower semicontinuous function on $X$, then $f$ is the supremum of the set of continuous functions $g: X \to \mathbb{R}$  such that $g \leq f$.
\label{positiveLSC}
\end{theorem}
\begin{proof}
Let $\mathscr{M}(f)$ be the set of all continuous functions $g:X \to \mathbb{R}$ with $g \leq f$. As $f \geq 0$ we get $0 \in \mathscr{M}(f)$, and so $\mathscr{M}(f)$ is nonempty. 
If $x \in X$ and $\epsilon>0$, let 
\[
F=f^{-1}(-\infty, f(x)-\epsilon].
\]
$X\setminus F = f^{-1}(f(x)-\epsilon,\infty) \in \tau$, so $F$ is closed. It is apparent that $x \not \in F$. Therefore there is a continuous function $h:X \to [0,1]$ such that
$h(x)=1$ and $h(F)=0$.  If $f(x)-\epsilon \leq 0$ then, as $h \geq 0$  and $f \geq 0$, we have $(f(x)-\epsilon)h \leq f$.
If $f(x)-\epsilon>0$ and $y \in X$, then
\[
(f(x)-\epsilon)h(y) 
=
\begin{cases}
0&y \in F\\
(f(x)-\epsilon)h(y)&y \not \in F
\end{cases}
\leq
\begin{cases}
0&y \in F\\
f(x)-\epsilon&y \not \in F
\end{cases}
\leq 
f(y).
\]
Therefore $(f(x)-\epsilon)h \leq f$, and because 
the function $(f(x)-\epsilon)h$ is continuous we have
$(f(x)-\epsilon)g \in \mathscr{M}(f)$
Because this is an element of $\mathscr{M}(f)$ we get
\[
\Big( \bigvee \mathscr{M}(f) \Big)(x) \geq (f(x)-\epsilon)g(x) = f(x)-\epsilon.
\]
As $\epsilon$ was arbitrary it follows that $\Big( \bigvee \mathscr{M}(f) \Big)(x) \geq f(x)$, and as $x$ was arbitrary we have $\bigvee \mathscr{M}(f) \geq f$.
But $f$ is an upper bound for $\mathscr{M}(f)$, so $\bigvee \mathscr{M}(f) \leq f$. Therefore $f=\bigvee \mathscr{M}(f)$. 
\end{proof}


The following is a formulation of the extreme value theorem for lower semicontinuous functions on a compact topological space. 

\begin{theorem}[Extreme value theorem]
If $X$ is a compact topological space and if $f$ is a  lower semicontinuous function on $X$, then 
\[
K=\left\{x \in X: f(x) = \inf_{y \in X} f(y)\right\}
\]
is a nonempty closed subset of $X$.
\label{evt}
\end{theorem}
\begin{proof}
Let $C=f(X) \subseteq [-\infty,\infty]$, and for $c \in C$ let $F_c=\{x \in X: f(x) \leq c\}$. Because $F_c=X \setminus f^{-1}(c,\infty]$ and $f$ is lower
semicontinuous, $F_c$ is a closed set. Suppose that $c_1,\ldots,c_n \in C$. Taking $c=\min \{c_k: 1 \leq k \leq n\} \in C$,
we have
\[
\bigcap_{k=1}^n F_{c_k}  = F_c \neq \emptyset.
\]
This shows that the collection $\{F_c : c \in  C\}$ has the finite intersection property (the intersection of finitely many members of it is nonempty). But a topological space is compact if and only if for every collection
of closed subsets with the finite intersection property the intersection of all the members of the collection is nonempty.\footnote{James Munkres,
{\em Topology}, second ed., p.~169, Theorem 26.9.} Applying this theorem, we get that 
\[
\bigcap_{c \in C} F_c \neq \emptyset,
\]
and this intersection is closed because each member is closed.

Let $x \in \bigcap_{c \in C} F_c$. Then for all $c \in C$ we have
$f(x) \leq c$, and because $C=f(X)$ this means that for all $y \in X$ we have $f(x) \leq f(y)$. Therefore $x \in K$. Let
$x \in K$. Then for all $y \in X$ we have $f(x) \leq f(y)$, hence for all $c \in C$ we have $f(x) \leq c$, hence
$x \in \bigcap_{c \in C} F_c$. Therefore $K= \bigcap_{c \in C} F_c$, which we have shown is nonempty and closed, proving the claim.
\end{proof}


\section{Upper semicontinuous functions}
If $(X,\tau)$ is a topological space, then $f:X \to [-\infty,\infty]$ is said to be {\em upper semicontinuous} if $t \in \mathbb{R}$ implies that $f^{-1}[-\infty,t) \in 
\tau$. We denote by $\USC(X)$ the set of upper semicontinuous functions $X \to [-\infty,\infty]$. We say that $f$ is {\em finite} if $-\infty<f(x)<\infty$ for all $x \in X$. 


If $A \subseteq X$ and $t \in \mathbb{R}$, then
\[
\chi_A^{-1}[t,\infty)=\begin{cases}
X&t \leq 0\\
A&0<t \leq 1\\
\emptyset&t > 1
\end{cases}
\]
But $\chi_A^{-1}[t,\infty)$ is closed if and only if $\chi_A^{-1}[-\infty,t)$ is open, hence $\chi_A$ is upper semicontinuous if and only if 
$A$ is closed.

It is apparent that $f:X \to [-\infty,\infty]$ is upper semicontinuous if and only if $-f:X \to [-\infty,\infty]$ is lower semicontinuous. 
Because the set of all open intervals are a basis for the topology of $\mathbb{R}$, a function $X \to \mathbb{R}$ is continuous if and only if it is both
lower semicontinuous and upper semicontinuous. 
That is,
\[
C(X) =  \mathbb{R}^X \cap \LSC(X) \cap \USC(X).
\]


\section{Approximating integrable functions}
If $X$ is a Hausdorff space, if $\mathfrak{M}$ is a $\sigma$-algebra on $X$ that contains the Borel $\sigma$-algebra of $X$ (equivalently, if every open set belongs
to $\mathfrak{M}$), and if $\mu$ is a measure on
$\mathfrak{M}$, we say that $\mu$ is {\em outer regular on $E \in \mathfrak{M}$}   if
\[
\mu(E) = \inf\{ \mu(V): \textrm{$E \subseteq V$ and $V$ is open}\}, 
\]
and 
we say that $\mu$ is {\em inner regular on $E$} if 
\[
\mu(E) = \sup\{ \mu(K) : \textrm{$K \subseteq E$ and $K$ is compact}\}.
\]

We state the following to motivate the conditions in Theorem \ref{vitali}.
If $X$ is a locally compact Hausdorff space and $\lambda$ is a positive linear functional on $C_c(X)$ ($f \geq 0$ implies that $\lambda f \geq 0$), then the {\em Riesz-Markov theorem}\footnote{Walter
Rudin, {\em Real and Complex Analysis}, third ed., p.~40, Theorem 2.14.} states that there is a $\sigma$-algebra $\mathfrak{M}$ on $X$ that contains the Borel 
$\sigma$-algebra of $X$ and there is a unique complete measure $\mu$ on $\mathfrak{M}$ that satisfies:
\begin{enumerate}
\item If $f \in C_c(X)$ then $\lambda f = \int_X fd\mu$.
\item If $K$ is compact then $\mu(K)<\infty$.
\item $\mu$ is outer regular on all $E \in \mathfrak{M}$
\item $\mu$ is inner regular on all open sets and  on all sets with finite measure.
\end{enumerate}


The following theorem gives conditions under which we can bound an integrable function above and below by semicontinuous functions that can be chosen as close
as we please in $L^1$ norm.\footnote{Walter Rudin, {\em Real and Complex Analysis}, third ed., p.~56, Theorem 2.25.}



\begin{theorem}[Vitali-Carath\'eodory theorem]
Let $X$ be a locally compact Hausdorff space, let $\mathfrak{M}$ be a $\sigma$-algebra containing the Borel $\sigma$-algebra of $X$, and let $\mu$ be a complete
measure on $\mathfrak{M}$ that satisfies $\mu(K) < \infty$ for compact $K$, that is outer regular on all measurable sets, and that is inner regular on open sets and on sets with
finite measure. If $f \in L^1(\mu)$ is real valued and if $\epsilon>0$, then there is some upper semicontinuous function $u$ that is bounded above and some lower semicontinuous
function $v$ that
is bounded below such that
$u \leq f \leq v$ and such that
\[
\int_X (v-u) d\mu <\epsilon.
\]
\label{vitali}
\end{theorem}
\begin{proof}
Let $g \in L^1(\mu)$ be $\geq 0$ and let $\epsilon>0$.
There is a nondecreasing sequence of measurable simple functions $s_n$ such 
that for  all $x \in X$ we have $g(x)=\lim_{n \to \infty} s_n(x)$.\footnote{Walter
Rudin, {\em Real and Complex Analysis}, third ed., p.~15, Theorem 1.17. A {\em simple function} is a finite linear combination of characteristic functions, either over
$\mathbb{R}$ or $\mathbb{C}$.}
Writing $s_0=0$ and $t_n=s_n-s_{n-1}$, each $t_n$ is a measurable simple function and is $\geq 0$.
Then, there are
some $c_i \geq 0$ and measurable sets $E_i$ such that
for  all $x \in X$ we have
\[
g(x) = \lim_{n \to \infty} \sum_{i=1}^n t_i(x) = 
\sum_{i=1}^\infty t_i(x)=\sum_{i=1}^\infty c_i \chi_{E_i}(x).
\]
Integrating this we get\footnote{Walter Rudin, {\em Real and Complex Analysis}, third ed., p.~22, Theorem 1.27.}
\[
\int_X g d\mu = \sum_{i=1}^\infty \int_X c_i \chi_{E_i} d\mu=\sum_{i=1}^\infty c_i \mu(E_i).
\]
As $g \in L^1(\mu)$ the left-hand side is finite and so the right-hand side is too, hence  there is then some $N$ such that 
$\sum_{i=N+1}^\infty c_i \mu(E_i)<\frac{\epsilon}{2}$.

For each $i$, because $\mu$ is outer regular on $E_i$ there is an open set $V_i$ containing $E_i$ such that
$c_i \mu(V_i \setminus E_i) < 2^{-i-2}\epsilon$. Each $E_i$ has finite measure so $\mu$ is inner regular on  $E_i$, hence there
is a compact set $K_i$ contained in $E_i$ such that $c_i \mu(E_i \setminus K_i) < 2^{-i-2}\epsilon$.
Define 
\[
v = \sum_{i=1}^\infty c_i \chi_{V_i}, \qquad u = \sum_{i=1}^N c_i \chi_{K_i}.
\]
Each $V_i$ is open so the characteristic function $\chi_{V_i}$ is lower semicontinuous, and $c_i \geq 0$ so
each of the functions $\sum_{i=1}^n c_i \chi_{V_i}$ is a sum of finitely many lower semicontinuous functions and hence is lower semicontinuous.
But $v$ is the supremum of the  functions $\sum_{i=1}^n c_i \chi_{V_i}$, so  $v$ is lower semicontinuous. 
As each $K_i$ is closed the characteristic function $\chi_{K_i}$ is upper semicontinuous, and $c_i \geq 0$ so $u$ is a sum of finitely many upper semicontinuous functions
and hence is upper semicontinuous. $u$ is a finite sum so is bounded  above, and $v$ is a sum of nonnegative terms so is bounded below by $0$.

Because $K_i \subseteq E_i \subseteq V_i$ we have $u \leq g \leq v$, and
\begin{align*}
v-u &= \sum_{i=1}^\infty c_i \chi_{V_i} - \sum_{i=1}^N c_i \chi_{K_i} \\
&= \sum_{i=1}^N c_i(\chi_{V_i}-\chi_{K_i}) + \sum_{i=N+1}^\infty c_i \chi_{V_i}\\
&\leq \sum_{i=1}^\infty c_i(\chi_{V_i}-\chi_{K_i}) + \sum_{i=N+1}^\infty c_i \chi_{E_i};
\end{align*}
the inequality is because $\chi_{V_i}-\chi_{K_i}+\chi_{E_i} \geq \chi_{V_i}$.
Integrating,
\begin{align*}
\int_X (v-u) d\mu& \leq \sum_{i=1}^\infty c_i \mu(V_i \setminus K_i) + \sum_{i=N+1}^\infty c_i \mu(E_i)\\
&= \sum_{i=1}^\infty (c_i \mu(V_i \setminus E_i)  + c_i \mu(E_i \setminus V_i)) + \sum_{i=N+1}^\infty c_i \mu(E_i)\\
&\leq \sum_{i=1}^\infty (2^{-i-2} \epsilon + 2^{-i-2}\epsilon) + \frac{\epsilon}{2}\\
&= \epsilon.
\end{align*}

Let $f=f^+ -f^-$, for $f^+,f^- \in L^1(\mu)$ with $f^+, f^- \geq 0$, and let $\epsilon>0$. From what we have established above, there is an upper semicontinuous function $u_1$ that is bounded
above and a lower semicontinuous function $v_1$ that is bounded below satisfying $u_1 \leq f^+ \leq v_1$ and
\[
\int_X (v_1-u_1) d\mu < \frac{\epsilon}{2},
\]
and similarly $u_2,v_2$ with $u_2 \leq f^- \leq v_2$ and
\[
\int_X (v_2-u_2) d\mu < \frac{\epsilon}{2}.
\]
We have
\[
u_1 - v_2 \leq f ^+ - f^- =f=f^+-f^- \leq v_1-u_2.
\]
That $v_2$ is lower semicontinuous and bounded below means that $-v_2$ is upper semicontinuous and bounded above, hence $u_1-v_2$ is upper semicontinuous
and bounded above. That $u_2$ is upper semicontinuous and bounded above means that $-u_2$ is lower semicontinuous and bounded below, so
$v_1-u_2$ is lower semicontinuous and bounded below. Taking $u=u_1-v_2$ and $v=v_1-u_2$,
we have $u \leq f \leq v$, and 
\[
\int_X (v-u) d\mu = \int_X (v_1-u_2-u_1+v_2) d\mu = \int_X (v_1-u_1) d\mu + \int_X (v_2-u_2) d\mu
<\epsilon.
\]
\end{proof}





\section{Convex functions}
If $X$ is a set and $f:X \to [-\infty,\infty]$ is a function, its {\em epigraph} is the set
\[
\epi f=\{(x,\alpha) \in X \times \mathbb{R} : \alpha \geq f(x)\}.
\]
When $X$ is a vector space,
we say that $f$ is {\em convex} if $\epi f$ is a convex subset of the vector space $X \times \mathbb{R}$. 
The {\em effective domain} of a convex function $f$ is the set
\[
\dom f = \{x \in X: f(x)< \infty\}.
\]
To say that $x \in \dom f$ is to say that there is some $\alpha \in \mathbb{R}$ such that $(x,\alpha) \in \epi f$, from which
it follows that if $f:X \to [-\infty,\infty]$ is a convex function  then $\dom f$ is a convex subset of $X$.
A convex function $f$ is said to be {\em proper} if $\dom f \neq \emptyset$ and $f(x)>-\infty$ for all $x \in X$,
i.e. if $f$ does not only take the value $\infty$ and never takes the value $-\infty$.


If $C$ is a nonempty convex subset of $X$ and $f:C \to \mathbb{R}$ is a function, we extend $f$ to $X$ by defining $f(x)=\infty$
for $x \not \in C$.  One checks that this extension is a convex function if and only if 
\[
f((1-t)x+ty) \leq (1-t)f(x)+tf(y), \qquad x,y \in C, \quad 0<t<1,
\]
and we call $f:C \to \mathbb{R}$ convex if $f:X \to (-\infty,\infty]$ is convex.
If this extension  is convex, then it has effective domain $C$ and is proper. 



The following lemma is straightforward to prove.\footnote{Charalambos D. Aliprantis and Kim C. Border,
{\em Infinite Dimensional Analysis: A Hitchhiker's Guide}, third ed., p.~187, Lemma 5.41.}

\begin{lemma}
If $X$ is a vector space, if $C$ is a convex subset of $X$, if $f:C \to \mathbb{R}$ is convex, if $x,x+z,x-z \in C$,
and if $0 \leq \delta \leq 1$, then
\[
|f(x+\delta z)-f(x)| \leq \delta \max \{f(x+z)-f(x),f(x-z)-f(x)\}.
\]
\label{convexinequality}
\end{lemma}


The following lemma asserts that a convex function that is bounded above on some neighborhood of  an interior point of a convex subset of a topological
vector space is continuous at that
point.\footnote{Charalambos D. Aliprantis and Kim C. Border,
{\em Infinite Dimensional Analysis: A Hitchhiker's Guide}, third ed., p.~188, Theorem 5.42.}

\begin{lemma}
If $X$ is a topological vector space, if $C$ is a convex subset of $X$, if $f:C \to \mathbb{R}$ is convex, if $x$ is in the interior
of $C$, and if $f$ is bounded above on some neighborhood of $x$, then $f$ is continuous at $x$.
\label{localcontinuity}
\end{lemma}
\begin{proof}
There is some neighborhood of $x$ contained in $C$ on which $f$ is bounded above. Thus, there is some open neighborhood $U$ of the origin
such that $x+U \subseteq C$ and such that $f$ is bounded above on $x+U$. Being bounded above means that there is some $M$ such that $y \in x+U$ implies that $f(y) \leq M$. Any open neighborhood of $0$ contains a balanced open neighborhood of $0$,\footnote{Walter
Rudin, {\em Functional Analysis}, second ed., p.~12, Theorem 1.14. For a set $V$ to be {\em balanced} means that $|\alpha| \leq 1$ implies
that $\alpha V \subseteq V$.} so let $V$ be a balanced open neighborhood of $0$ contained in $U$. 

Let $\epsilon>0$ and take
$\delta>0$ small enough that $\delta (M-f(x))<\epsilon$. 
For $y \in x+\delta V$, there is some $z \in V$ with
$y=x+\delta z$, and because $V$ is balanced we have $x+z,x-z \in x+V$.  Then we can apply
Lemma \ref{convexinequality} to get
\[
|f(y)-f(x)| \leq \delta \max \{f(x+z)-f(x),f(x-z)-f(x)\} \leq \delta (M-f(x))<\epsilon.
\]
But $x+\delta V$ is an open neighborhood of $x$ (because scalar multiplication is continuous), so we have shown that if $\epsilon>0$ then there is some open neighborhood of $x$ such that
$y$ being in this neighborhood implies that $|f(y)-f(x)| < \epsilon$. This means that $f$ is continuous at $x$.
\end{proof}

The following theorem shows that properties that by themselves are weaker than continuity on a set are equivalent to it for a convex function on an open convex set.\footnote{Charalambos D. Aliprantis and Kim C. Border,
{\em Infinite Dimensional Analysis: A Hitchhiker's Guide}, third ed., p.~188, Theorem 5.43.} We prove three of the five implications because two are immediate.

\begin{theorem}
If $X$ is a topological vector space, if $C$ is an open convex subset of $X$, and if $f:C \to \mathbb{R}$ is convex, then the following are equivalent:
\begin{enumerate}
\item $f$ is continuous on $C$.
\item $f$ is upper semicontinuous on $C$.
\item For each $x \in C$ there is some neighborhood of $x$ on which $f$ is bounded above.
\item There is some $x \in C$ and some neighborhood of $x$ on which $f$ is bounded above.
\item There is some $x \in C$ at which $f$ is continuous.
\end{enumerate}
\end{theorem}
\begin{proof}
Suppose that $f$ is upper semicontinuous on $C$, and say $x \in C$. Because $f$ is upper semicontinuous and $C$ is open, the set
\[
U=\{y \in C: f(y)<f(x)+1\} = C \cap f^{-1}(-\infty,f(x)+1)
\]
is open. $x \in U$ because $f(x)<f(x)+1$, so $U$ is a neighborhood of $x$, and $f$ is bounded on $U$.

Suppose that $x \in C$ and $U$ is a neighborhood of $x$ on which $f$ is bounded above. 
Lemma \ref{localcontinuity} states that $f$ is continuous at $x$, showing that $f$ is continuous at some point in $C$.

Suppose that $f$ is continuous at some $x \in C$, and let $y$ be another point in $C$. The function $t \mapsto x+t(y-x)$ is continuous
$\mathbb{R} \to X$, and because it sends $1$ to $y \in C$ and $C$ is open,
there is some $t>1$ such that $x+t(y-x) \in C$. (That is, the line segment from $x$ to $y$ remains in $C$ for some length past $y$.)
Set $z=x+t(y-x)$,
i.e. $z=(1-t)x+ty$, 
i.e.
\[
y=\left(1-\frac{1}{t} \right)x+\frac{1}{t}z,
\]
or
\[
y= \lambda x + (1-\lambda)z,
\]
where $\lambda=1-\frac{1}{t}$ satisfies $0<\lambda<1$.
$f$ being continuous at $x$ means that for every $\epsilon>0$ there is some open neighborhood $V$ of $0$ such that $y \in x+V$
implies that $|f(y)-f(x)|<\epsilon$. Take $x+V \subseteq C$, which we can do because $C$ is open. In particular, 
there is some $M$ such that  $f(w) \leq M$ for $w \in x+V$.
If $v \in V$, then 
\[
y+\lambda v = \lambda x +(1-\lambda)z + \lambda v = \lambda (x+v) + (1-\lambda)z.
\]
$x+v \in C$ and $z \in C$, and because $C$ is convex this tells us $y+\lambda v \in C$. Therefore
$y + \lambda V \subseteq C$. 
Because $f$ is convex we have
\[
f(y+\lambda v) = f( \lambda (x+v) + (1-\lambda)z ) \leq \lambda f(x+v) + (1-\lambda) f(z) \leq
\lambda M + (1-\lambda) f(z).
\]
This holds for every $v \in V$, so $f$ is bounded above  by $\lambda M + (1-\lambda)f(z)$ on $y+\lambda V$.
$y+\lambda V$ is an open neighborhood of $y$ on which $f$ is bounded above, so we can apply 
 Lemma \ref{localcontinuity}, which tells us that $f$ is continuous at $y$. Since $f$ is continuous at each point in $C$, it is continuous on $C$.
\end{proof}



\section{Convex hulls}
If $X$ is a topological vector space over $\mathbb{R}$, let  $X^*$ denote the set of  continuous linear maps
$X \to \mathbb{R}$. $X^*$ is  called the {\em dual space of $X$} and  is  a vector space. 
We call the bilinear map $\inner{\cdot}{\cdot}:X \times X^* \to \mathbb{R}$ defined by 
\[
\inner{x}{\lambda}=\lambda x, \qquad x \in X, \lambda \in X^*
\]
the {\em dual pairing} of $X$ and $X^*$.
If $X$ is locally convex, it follows from the Hahn-Banach separation theorem\footnote{Walter Rudin, {\em Functional Analysis}, second ed., p.~59, Theorem 3.4.} that
for distinct $x,y \in X$ there is some $\lambda \in X^*$ such that $\lambda x \neq \lambda y$.
The {\em weak topology} on $X$ is the initial topology for the set of functions $X^*$, and we denote 
the vector space $X$ with the weak topology by $X_w$. $X_w$ is a locally convex space whose dual space is $X^*$.\footnote{Walter Rudin, {\em Functional Analysis}, second ed., p.~64, Theorem 3.10.}

If $X$ is a vector space and $E$ is a subset of $X$, the {\em convex hull} of $E$ is the set of all convex combinations of finitely many points in $E$ and is denoted
by $\co E$. The convex hull $\co E$ is a convex set, and  it is straightforward to prove that $\co E$ is equal to the intersection of all convex sets containing $E$. 
If $X$ is a topological vector  space, the {\em closed convex hull} of $E$ is the closure of the convex hull $\co E$ and is denoted by $\cco E$. One proves that the closed convex hull $\cco E$ is equal to the intersection of all closed convex sets containing $E$.

A {\em closed half-space} in a locally convex space $X$  over $\mathbb{R}$ is a set of the form $\{ x \in X: \inner{x}{\lambda} \leq \beta\}$, for $\beta \in \mathbb{R}$
and  $\lambda \in X^*$ with $\lambda \neq 0$. (If $X$ is merely a topological vector space then it may be the case that $X^*=\{0\}$, for example $L^p[0,1]$ for $0<p<1$.)
If $\inner{x}{\lambda} \leq \beta$, $\inner{y}{\lambda} \leq \beta$, and $0 \leq t \leq 1$, then 
\[
\inner{(1-t)x+ty}{\lambda} = (1-t)\inner{x}{\lambda}+t\inner{y}{\lambda} \leq (1-t)\beta + t \beta = \beta,
\]
so a closed half-space is convex. 

\begin{lemma}
If $X$ is a locally convex space over $\mathbb{R}$ and $E$ is a subset of $X$, then $\cco E$ is the intersection of all closed half-spaces containing $E$.
\end{lemma}
\begin{proof}
If $E = \emptyset$ then $\cco E = \emptyset$. But every closed half-space contains $\emptyset$ and the intersection of all of these is also $\emptyset$, so the claim is true in this
case.
If $\cco E=X$ then there are no closed half-spaces that contain $E$, and as an intersection over an empty index set is
equal to the universe which is $X$ here, so the claim is true in this case also. Otherwise,
$\cco E \neq \emptyset, X$, and
let $a \not \in \cco E$.  Because $\{a\}$
is a compact convex set and $\cco E$ is a  disjoint nonempty closed convex set, we can apply the Hahn-Banach separation theorem,\footnote{Walter
Rudin, {\em Functional Analysis}, second ed., p.~59, Theorem 3.4.} which tells us that there is some $\lambda_a \in X^*$ and some $\gamma_a \in \mathbb{R}$ such that
\[
 \lambda_a x < \gamma_a < \lambda_a a, \qquad x \in \cco E. 
\]
 $\cco E$ is contained in the closed half-space $\{x \in X: \inner{x}{\lambda_a} \leq \gamma_a\}$ and $a$ is not. Hence
\[
\cco E = \bigcap_{a \not \in \cco E} \{x \in X: \inner{x}{\lambda_a} \leq \gamma_a\}.
\]
This shows that $\cco E$ is equal to an intersection of closed half-spaces containing $E$. Since a closed half-space is closed and convex and
$\cco E$ is the intersection of all closed convex sets containing $E$, it follows that $\cco E$ is equal to the intersection of all closed half-spaces containing $E$.
\end{proof}

If $X$ is a vector convex space and $f:X \to [-\infty,\infty]$ is a function, the {\em convex hull of $f$} is the
function $\co f:X \to [-\infty,\infty]$ defined by
\[
\co f = \bigvee \{g \in [-\infty,\infty]^X: \textrm{$g$ is convex and $g \leq f$}\}.
\]
We have $\co f \leq f$. The following lemma shows
that the supremum of a set of convex functions is itself a convex function (because an intersection of convex sets is itself a convex set and for a function to be convex
means that its epigraph is a convex set), and hence
that the convex hull of a function is itself a convex function.

\begin{lemma}
If $X$ is a set and $\mathscr{F} \subseteq [-\infty,\infty]^X$, then $F=\bigvee \mathscr{F}$ satisfies
\[
\epi F = \bigcap_{f \in \mathscr{F}} \epi f.
\]
\label{epigraphintersection}
\end{lemma}
\begin{proof}
If $\mathscr{F} = \emptyset$, then $\bigvee \mathscr{F}$ is the function $x \mapsto -\infty$, whose epigraph is $X \times \mathbb{R}$.
The intersection over the empty set of subsets of $X \times \mathbb{R}$ is equal to the universe $X \times \mathbb{R}$, so the claim holds in this case.
 Otherwise, let
$F = \bigvee \mathscr{F}$, which satisfies
\[
F(x) = \sup\{ f(x): f \in \mathscr{F}\}, \qquad x \in X.
\]
Suppose that $(x,\alpha) \in \epi F$. This means that $F(x) \leq \alpha$, so $f(x) \leq \alpha$ for all $f \in \mathscr{F}$, hence
$(x,\alpha) \in \epi f$ for all $f \in \mathscr{F}$, and therefore
\[
(x,\alpha) \in \bigcap_{f \in \mathscr{F}} \epi f.
\]
Suppose that $x \in X$ and $\alpha \in \mathbb{R}$ and that $(x,\alpha) \in \epi f$ for all $f \in \mathscr{F}$. This means
that $f(x) \leq \alpha$ for all $f \in \mathscr{F}$, hence $F(x) \leq \alpha$ and so $(x,\alpha) \in \epi F$.
\end{proof}

Lower semicontinuity of a function can also be expressed using the notion of epigraphs.

\begin{lemma}
If $X$ is a topological space and $f:X \to [-\infty,\infty]$ is a function, then $f$ is lower semicontinuous if and only if
$\epi f$ is a closed subset of $X \times \mathbb{R}$. 
\label{epilsc}
\end{lemma}
\begin{proof}
Suppose that $f$ is lower semicontinuous and let $(x_i,\alpha_i) \in \epi f$ be a net that converges to $(x,\alpha) \in X \times \mathbb{R}$.
Then $x_i \to x$ and $\alpha_i \to \alpha$, and 
Theorem \ref{liminf} gives us
\[
f(x) \leq \liminf f(x_i) \leq \liminf \alpha_i = \lim \alpha_i = \alpha.
\]
Hence $f(x) \leq \alpha$, which means that $(x,\alpha) \in \epi f$. Therefore $\epi f$ is closed.

Suppose that $\epi f$ is closed and let $t \in \mathbb{R}$. The set
\[
\epi f \cap (X \times \{t\}) = \{(x,t): x \in X, t \geq f(x) \}
\]
 is a closed subset of $X \times \mathbb{R}$. This implies that $f^{-1}[-\infty,t]$ is a closed subset of $X$, which is equivalent to $f^{-1}(t,\infty]$ being an open subset of
 $X$. This is true for all $t \in \mathbb{R}$, so $f$ is lower semicontinuous.
\end{proof}

The following lemma shows that if a convex  lower semicontinuous function takes the value $-\infty$ then it is nowhere
finite. This means that if there is some point at which a convex  lower semicontinuous function takes a finite value then it 
does not take the value $-\infty$, namely, if a convex  lower semicontinuous function takes a finite value at some point then it
is proper.

\begin{lemma}
If $X$ is a topological vector space, if  $f:X \to [-\infty,\infty]$ is convex and lower semicontinuous, and if there is some $x_0 \in X$ such that
$f(x_0)=-\infty$, then $f(x) \in \{-\infty,\infty\}$ for all $x \in X$.
\end{lemma}
\begin{proof}
Suppose by contradiction that there is some $x \in X$ such that $-\infty<f(x)<\infty$. Because
$f$ is convex, for every $\lambda \in (0,1]$ we have
\[
f((1-\lambda)x+\lambda x_0) \leq (1-\lambda) f(x)+\lambda f(x_0) =\textrm{finite} -\infty=-\infty,
\]
hence $f((1-\lambda)x+\lambda x_0)=-\infty$ for all $\lambda \in (0,1]$.
Because $f$ is lower semicontinuous, 
\[
f\Big( \lim_{\lambda \to 0} ( (1-\lambda)x+\lambda x_0 ) \Big) \leq \liminf_{\lambda \to 0} f((1-\lambda)x+\lambda x_0),
\] 
and hence
\[
f(x) \leq -\infty,
\]
a contradiction. Therefore there is no $x \in X$ such that $-\infty<f(x)<\infty$.
\end{proof}

If $X$ is a topological space and $f:X \to [-\infty,\infty]$ is a function, the {\em lower semicontinuous hull of $f$} is the function
$\lsc f:X \to [-\infty,\infty]$ defined by
\[
\lsc f = \bigvee \{g \in \LSC(X): g \leq f\}.
\]
By Theorem \ref{LSClattice},  $\lsc f \in \LSC(X)$. It is apparent that a function is lower semicontinuous if and only
if it is equal to its lower semicontinuous hull. The following lemma shows that the epigraph of the lower semicontinuous
hull of a function is equal to the closure of its epigraph.\footnote{Jean-Paul Penot, {\em Calculus Without Derivatives},
p.~18, Proposition 1.21.}

\begin{lemma}
If $X$ is a topological space and $f:X \to [-\infty,\infty]$ is a function, then
\[
\epi \lsc f  = \overline{\epi f} 
\]
\label{epiclosure}
\end{lemma}
\begin{proof}
Check that $\overline{\epi f}=\epi g$ for $g:X \to [-\infty,\infty]$ defined by\footnote{Let $\mathscr{N}(x)$ be the neighborhood filter at $x$. $\liminf_{y \to x} f(y)$ is defined to be
\[
 \sup_{N \in \mathscr{N}(x)} \inf_{y \in N \setminus \{x\}} f(y).
\]}
\[
g(x) = \liminf_{y \to x} f(y), \qquad  x \in X,
\]
and that $\lsc f = g$.\end{proof}


The notion of being a convex function applies to functions on a vector space, and the notion of being lower semicontinuous applies to functions
on a topological space. 
The following theorem shows  that the lower semicontinuous hull of a convex function on a topological vector space is convex.\footnote{R. Tyrrell Rockafellar, {\em Conjugate Duality and Optimization}, p.~15, Theorem 4.}

\begin{theorem}
If $X$ is a  topological vector space and  $f:X \to [-\infty,\infty]$ is convex, then $\lsc f$ is convex. 
\end{theorem}
\begin{proof}
For $\lsc f$ to be a convex function means that $\epi \lsc f$ is a convex set. But  Lemma \ref{epiclosure} tells us that
 $\epi \lsc f = \overline{\lsc f}$. As $f$ is convex, the epigraph $\epi f$ is convex, and the closure of a convex set is convex.\footnote{Walter Rudin,
 {\em Functional Analysis}, second ed., p.~11, Theorem 1.13.} Hence $\epi \lsc f$ is convex, and so $\lsc f$ is a convex function.
\end{proof}



\section{Extreme points}
If $X$ is a vector space and $C$ is a subset of $X$, a nonempty subset of $S$ of $C$ is called an {\em extreme set of $C$} if $x,y \in C, 0<t<1$ and
$(1-t)x+ty \in S$ together imply that $x,y \in S$. An {\em extreme point of $C$} is an element $x$ of $C$ such that the singleton $\{x\}$ is an extreme set of $C$.
The set of extreme points of $C$ is denoted by $\ext C$. If $C$ is a convex set and $S$ is an extreme set of $C$ that is itself convex,
then $S$ is called a {\em face of $C$}.

\begin{lemma}
If $X$ is a vector space, if $C$ is a convex subset of $X$, and if $a \in C$, then $a$ is an extreme point of $C$ if and only if $C \setminus \{a\}$
is a convex set.
\end{lemma}
\begin{proof}
Suppose that $a$ is an extreme point of $C$ and let $x,y \in C \setminus \{a\}$ and $0 < t  < 1$. 
Because $C$ is convex, $(1-t)x+ty \in C$. If $(1-t)x+ty=a$, then because $a$ is an extreme  point of $C$ we would have
$x=a$ and $y=a$, contradicting $x,y \in C \setminus \{a\}$. Therefore $(1-t)x+ty \in C \setminus \{a\}$. 

Suppose that $C \setminus \{a\}$ is a convex set. Suppose that $x,y \in C$, $0<t<1$, and   $(1-t)x+ty = a$.
Assume by contradiction that $x \neq a$. If $y=a$ then we get $(1-t)x+ta=a$, or $(1-t)x=(1-t)a$, hence $x=a$, a contradiction. If $y \neq a$, then using
that $C \setminus \{a\}$ is convex, we have $(1-t)x+ty \in C \setminus \{a\}$, contradicting that $(1-t)x+ty =a$. Therefore $x=a$. We similarly show that
$y =a$. Therefore $a$ an extreme point of $C$.
\end{proof}


The following lemma is about the set of maximizers of a convex function, and does not involve a topology on the vector space.\footnote{Charalambos D. Aliprantis and Kim C. Border,
{\em Infinite Dimensional Analysis: A Hitchhiker's Guide}, third ed., p.~296, Lemma 7.64.}

\begin{lemma}
If $X$ is a vector space, if $C$ is a convex susbset of $X$, and if $f:C \to \mathbb{R}$ is  convex, then
\[
F=\left\{x \in C: f(x) = \sup_{y \in C} f(y) \right\}
\]
is either an extreme set of $C$ or is empty.
\label{extremeset}
\end{lemma}
\begin{proof}
Suppose that there is some $x_0$ at which $f$ is maximized, i.e. that $F$ is nonempty, and let $M=f(x_0)$.
Suppose that $x,y \in C$, $0<t<1$, and $(1-t)x+ty \in F$. If at least one of $x,y$ do not belong to $F$, then, as $(1-t)x+ty \in F$ and as $f$ is convex,
\[
M=f((1-t)x+ty) \leq (1-t)f(x)+tf(y) < (1-t) M + tM = M,
\]
a contradiction. Therefore $x,y \in F$, showing that $F$ is an extreme set of $C$.
\end{proof}



Elements of an extreme set need not be extreme points, but the following lemma shows that in a locally convex space if an extreme set
is compact then it contains an extreme point.\footnote{Charalambos D. Aliprantis and Kim C. Border,
{\em Infinite Dimensional Analysis: A Hitchhiker's Guide}, third ed., p.~296, Lemma 7.65.}

\begin{lemma}
If $X$ is a real locally convex space, if $C$ is a subset of $X$, and if $F$ is a compact extreme set of $C$, then
\[
F \cap \ext C \neq \emptyset.
\]
\label{compactextreme}
\end{lemma}
\begin{proof}
Define $\mathscr{F} = \{G \subseteq F: \textrm{$G$ is a compact extreme set of $C$}\}$. $F \in \mathscr{F}$ so $\mathscr{F}$ is nonempty. 
$\mathscr{F}$ is a partially ordered set ordered by set inclusion. If $T$ is a chain in $\mathscr{F}$, then the intersection of finitely many elements
of $T$ is equal to the minimum of these elements which is an extreme set and hence is nonempty. Therefore
the chain $T$ has the finite intersection property, and because the elements of $T$ are closed subsets of $F$ and $F$ is compact, the intersection
of all the elements of $T$ is nonempty.\footnote{James Munkres,
{\em Topology}, second ed., p.~169, Theorem 26.9.}
One checks that this intersection belongs to $\mathscr{F}$ (one must verify that it is an extreme set of $C$) and is a lower bound for $T$. We have shown that every chain in $\mathscr{F}$ has a lower bound in 
$\mathscr{F}$, and applying Zorn's lemma, there is a minimal element $G$ in $\mathscr{F}$ (if an element of $\mathscr{F}$ is contained in $G$ then it is equal to
$G$).

Assume by contradiction that there are $a,b \in G$ with $a \neq b$. Then there is some
$\lambda \in X^*$ such that $\lambda a > \lambda b$. $G$ is compact and $\lambda$ is continuous, so
\[
G_0=\left\{ c \in G: \lambda c = \sup_{y \in G} \lambda y \right\}
\]
 is nonempty and closed. We shall show that $G_0$ is an extreme set of $G$. Suppose that $x,y \in G$, $0<t<1$, and $(1-t)x+ty \in G_0$.
 Let $M = \sup_{y \in G} \lambda y$. If  at least one of $x ,y$ do not belong to $G_0$, then
 \[
 M=\lambda ((1-t)x+ty) = (1-t)\lambda x + t \lambda y < (1-t)M+tM=M,
 \]
 a contradiction. Hence $x,y \in G_0$, showing that $G_0$ is an extreme set of $G$. Check that $G_0$ being an extreme set of $G$ implies that $G_0$ is an extreme
 set of $C$. Then $G_0 \in \mathscr{F}$, but as $\lambda a>\lambda b$ we have $b \not \in G_0$, so that $G_0$ is strictly contained in $G$, contradicting that
 $G$ is a minimal element of $\mathscr{G}$.  Therefore $G$ has a single element, as $G$ being an extreme set means that it is nonempty.
 But $G$ is an extreme set of $C$, and since $G$ is a singleton this means that the single point it contains is an extreme point of $C$.
 \end{proof}


The following theorem gives conditions under which  a function on a set has a maximizer that  is  an extreme point of the set.\footnote{Charalambos D. Aliprantis and Kim C. Border,
{\em Infinite Dimensional Analysis: A Hitchhiker's Guide}, third ed., p.~298, Theorem 7.69.}


\begin{theorem}[Bauer maximum principle]
If $X$ is a real locally convex  space, if $C$ is a compact convex subset of $X$, and if
$f:C \to \mathbb{R}$ is upper semicontinuous, then there is a maximizer of $f$ that belongs to $\ext C$.
\end{theorem}
\begin{proof}
Because $f$ is upper semicontinuous and $C$ is compact, it follows from Theorem \ref{evt} that 
\[
F=\left\{x \in C: f(x) = \sup_{y \in C} f(y) \right\}
\]
is a nonempty closed subset of $C$.  Since $C$ is convex and $f$ is a convex function, by Lemma \ref{extremeset}, $F$ is an extreme
set of $C$. $F$ is a closed subset of the compact set $C$, so $F$ is compact. Hence $F$ is a compact extreme set of $C$, and
by Lemma \ref{compactextreme} there is an extreme point in $F$, which was the claim.
\end{proof}


\section{Duality}
\begin{lemma}
A  convex function on a real locally convex space is lower semicontinuous if and only if it is weakly  lower semicontinuous.
\label{weaklylsc}
\end{lemma}
\begin{proof}
Let $X$ be a locally convex space and let
$X_w$ denote this vector space with the weak topology, with which it is a locally convex space.
As $\mathbb{R}$ is a locally convex space, the product $X \times \mathbb{R}$ is a locally convex space, and one checks that
 $X \times \mathbb{R}$ with the weak topology is $X_w \times \mathbb{R}$. 
Thus, to say that a subset of $X \times \mathbb{R}$ is weakly closed is equivalent to saying that it is closed
in $X_w \times \mathbb{R}$.  
Furthermore, the closure of a convex set in a locally convex space is equal to its weak closure.\footnote{Walter Rudin, {\em Functional Analysis}, second ed., p.~66, Theorem 3.12.}
In particular, a convex subset of a locally convex space is closed if and only if it is weakly closed.
Therefore, a convex subset of $X \times \mathbb{R}$ is closed if and only if it is  closed in $X_w \times \mathbb{R}$. 


By Lemma \ref{epilsc},
 a function $f:X \to [-\infty,\infty]$ is
lower semicontinuous if and only if $\epi f$ is a closed subset of $X \times \mathbb{R}$, and is weakly
lower semicontinuous if and only if $\epi f$ is a closed subset of $X_w \times \mathbb{R}$. 
\end{proof}


A topological vector space is said to have the {\em Heine-Borel property} if every closed and bounded subset  of it is compact. 
The following theorem gives conditions under which a function is minimized on a set that is  not necessarily compact but which is convex, closed, and bounded.



\begin{theorem}
If $X$ is a real  locally convex space such that $X_w$ has the Heine-Borel property, if $f:X \to [-\infty,\infty]$ is a  lower semicontinuous   convex function, 
and if $C$ is a convex closed bounded  subset of $X$, then 
\[
K=\left\{ x \in C: f(x) = \inf_{y \in C} f(y) \right\}
\]
is a nonempty closed subset of $X$.
\end{theorem}
\begin{proof}
Because $X$ is a locally convex space and $C$ is convex, the fact that $C$ is closed implies that it is weakly closed.\footnote{Walter Rudin, {\em Functional Analysis}, second ed., p.~66, Theorem 3.12.}
It is straightforward to prove that if a subset of a topological vector space is bounded then it is weakly bounded.\footnote{The converse is true in a locally convex space. Walter
Rudin, {\em Functional Analysis}, second ed., p.~70, Theorem 3.18.} 
Thus, $C$ is weakly closed and weakly bounded, and because $X_w$ has the Heine-Borel property we get that $C$ is weakly compact.
In other words, $C_w$ is compact, where $C_w$ is the set $C$ with the subspace topology inherited from $X_w$.
By Lemma \ref{weaklylsc}, $f$ is weakly lower semicontinuous, i.e. $f:X_w \to [-\infty,\infty]$ is lower semicontinuous.
Thus, the restriction of $f$ to $C_w$  is lower semicontinuous.
We have established that $C_w$ is compact and that the restriction of $f$ to $C_w$ is lower semicontinuous, so we can 
apply
Theorem \ref{evt} (the extreme value theorem) to obtain that
$K$ is a nonempty closed subset of $C_w$. Finally,  $K$ being a closed subset of $C_w$ implies that $K$ is a closed subset of $X$.
\end{proof}



\section{Convex conjugation}
 If $X$ is a locally convex space and $X^*$ is its dual space, the {\em strong dual topology} on $X^*$ is the seminorm topology induced by
the seminorms $\lambda \mapsto \sup_{x \in E} |\lambda x|$, where $E$ are the bounded subsets of $X$. Because these seminorms are a separating family, $X^*$ with the strong dual topology
is a locally convex space.
(If $X$ is a normed space then the strong dual topology on $X^*$ is equal to the operator norm topology on $X^*$.\footnote{K\^osaku
Yosida, {\em Functional Analysis}, sixth ed., p.~111, Theorem 1.})
A locally convex space is said to be {\em reflexive} if the strong dual of its strong dual is isomorphic as a locally convex space to the original space.

 

If $X$ is a real locally convex space, the {\em convex conjugate}\footnote{Also called the {\em Fenchel transform}.} of a function $f:X \to [-\infty,\infty]$ is the function
$f^*:X^* \to [-\infty,\infty]$ defined by
\[
f^*(\lambda)=\sup \{ \inner{x}{\lambda} - f(x): x \in X\}=\sup \{\inner{x}{\lambda}-f(x): x \in \dom f\}.
\]
The {\em convex biconjugate} of $f$ is the function $f^{**}:X \to [-\infty,\infty]$ defined by
\[
f^{**}(x) = \sup\{ \inner{x}{\lambda}-f^*(\lambda): \lambda \in X^*\}=\sup\{\inner{x}{\lambda}-f^*(\lambda): \lambda \in \dom f^*\}.
\]
The convex biconjugate of a function on  a real reflexive locally convex space is the convex conjugate of its convex conjugate.
From the definition of $f^*$ it is apparent that for all $x \in X$ and $\lambda \in X^*$,
\begin{equation}
\inner{x}{\lambda} \leq f(x)+f^*(\lambda),
\label{young}
\end{equation}
called {\em Young's inequality}. 


The following theorem establishes some properties of the convex conjugates and convex biconjugates of
any function from a real locally convex space to $[-\infty,\infty]$.\footnote{Viorel Barbu and Teodor Precupanu, {\em Convexity and Optimization in Banach Spaces}, fourth ed., p.~77, Proposition 2.19.}

\begin{theorem}
If $X$ is a real locally convex space and $f:X \to [-\infty,\infty]$ is a function, then
\begin{itemize}
\item $f^*$ is convex and weak-* lower semicontinuous,
\item $f^{**}$ is convex and weakly lower semicontinuous,
\item $f^{**} \leq f$.
\end{itemize}
If $f_1,f_2:X \to [-\infty,\infty]$ are functions satisfying $f_1 \leq f_2$,
then $f_1^* \geq f_2^*$. 
\label{conjugateproperties}
\end{theorem}
\begin{proof}
For each $x \in X$, it is apparent that the function $\lambda \mapsto \inner{x}{\lambda}$ is convex and weak-* continuous, and a fortiori
 is weak-* lower semicontinuous.
Whether $f(x)$ is finite or infinite, the function $\lambda \mapsto \inner{x}{\lambda}$ is weak-* lower semicontinuous.
By Lemma \ref{epigraphintersection}, the supremum of a collection of convex functions is a convex function, and by Theorem \ref{LSClattice}
 the supremum of a collection of lower semicontinuous functions is a lower semicontinuous function.
$f^*$ is the supremum of this set of functions, and therefore is convex and weak-* lower semicontinuous.

For each $\lambda \in X^*$,
the function $x \mapsto \inner{x}{\lambda}-f^*(\lambda)$ is convex and is weakly  semicontinuous, and a fortiori is weakly
lower semicontinuous. As $f^{**}$ is the supremum of this set of functions, $f^{**}$ is convex and weakly lower semicontinuous.

For every $x \in X$ and  $\lambda \in X^*$ we have by Young's inequality \eqref{young},
\[
\inner{x}{\lambda}-f^*(\lambda) \leq f(x),
\]
and hence for every $x \in X$,
\[
f^{**}(x) = \sup_{\lambda \in X^*} (\inner{x}{\lambda}-f^*(\lambda)) \leq f(x),
\]
and this means that $f^{**} \leq f$.

For $\lambda \in X^*$, because $f_1 \leq f_2$,
\[
f_2^*(\lambda) = \sup_{x \in X} ( \inner{x}{\lambda}-f_2(x)) \leq \sup_{x \in X} (\inner{x}{\lambda}-f_1(x)) = f_1^*(\lambda),
\]
which means that $f_1^* \geq f_2^*$.
\end{proof}



We remind ourselves that to say that a convex function  is {\em proper} means that at some point it takes a value other than $\infty$,
and that it nowhere  takes the value $-\infty$. The following lemma shows that any lower semicontinuous proper convex function
on a real locally convex space  is bounded below by a continuous affine functional.


\begin{lemma}
If $X$ is a real locally convex space and $f:X \to (-\infty,\infty]$ is a lower semicontinuous proper convex function, then there is some
$\mu \in X^*$ and some $c \in \mathbb{R}$ such that $f \geq \mu+c$.
\label{affine}
\end{lemma}
\begin{proof}
The fact that $f$ is a convex function tells us that $\epi f$ is a convex subset of $X \times \mathbb{R}$, and as $f$ is proper,
$\dom f \neq \emptyset$ and $x_0 \in \dom f$ satisfies $f(x_0)>-\infty$.
The fact that $f$ is lower semicontinuous tells us that $\epi f$ is a closed subset of $X \times \mathbb{R}$.
Let $x_0 \in \dom f$. We have $(x_0,f(x_0)-1) \not \in \epi f$. The singleton $\{(x_0,f(x_0)-1)\}$ is a compact convex
set and $\epi f$ is a disjoint closed convex set, so we can apply the Hahn-Banach separation theorem to obtain that there is some $\Lambda \in
(X \times \mathbb{R})^*$
and some $\gamma \in \mathbb{R}$ satisfying
\[
\Lambda (x,\alpha) < \gamma < \Lambda (x_0,f(x_0)-1), \qquad (x,\alpha) \in \epi f.
\]
There is some $\lambda \in X^*$ and some $\beta \in \mathbb{R}^*=\mathbb{R}$ such that $\Lambda(x,\alpha ) =\lambda x + \beta \alpha$
for all $(x,\alpha) \in X \times \mathbb{R}$.
So we have
\[
 \lambda x + \beta \alpha < \gamma<\lambda x_0 + \beta(f(x_0)-1), \qquad (x,\alpha) \in \epi f.
\]
And $(x_0,f(x_0)) \in \epi f$, so
\[
\lambda x_0 + \beta f(x_0)<\lambda x_0 + \beta ( f(x_0)-1),
\]
hence $\beta<0$. If $x \in \dom f$ then $(x,f(x)) \in \epi f$ and
\[
\lambda x + \beta f(x) < \lambda x_0 + \beta (f(x_0)-1).
\]
Rearranging, and as $\beta<0$,
\[
f(x) > -\frac{1}{\beta}\lambda x +\frac{1}{\beta}\lambda x_0+ f(x_0)-1, \qquad x \in \dom f.
\]
If $x \not \in \dom f$ then  $f(x) = \infty$, for which the above inequality also holds.
\end{proof}



We proved in Theorem \ref{conjugateproperties} that the conjugate of any
function is  convex and   weak-* lower semicontinuous, which a fortiori gives that it is lower semicontinuous.
In the following lemma we show that  a convex lower semicontinuous
function is proper if and only if its convex conjugate is proper.\footnote{Viorel Barbu and Teodor Precupanu, {\em Convexity and Optimization in Banach Spaces}, fourth ed., p.~78, Corollary 2.21.}

\begin{lemma}
If $X$ is a locally convex space and $f:X \to [-\infty,\infty]$ is a lower semicontinuous convex function, then $f$ is proper if and only if $f^*$ is proper.
\label{properiff}
\end{lemma}
\begin{proof}
Suppose that $f$ is proper.
By Lemma \ref{affine} there is some $\mu \in X^*$ and some $c \in \mathbb{R}$ such that 
$f(x) \geq \mu x + c$ for all $x \in X$. For
any $\lambda  \in X^*$ we have
\[
f^*(\lambda) = \sup_{x \in X} (\lambda x-f(x)) \leq \sup_{x \in X} (\lambda x -\mu x-c),
\]
thus $f^*(\mu)=-c < \infty$, so $\dom f^* \neq \emptyset$. And there is some $x_0 \in X$ such that
$f(x_0) \neq \infty$, giving $\sup_{x \in X} (\lambda x -f(x)) \geq \lambda x_0 - f(x_0)>-\infty$, showing
that $f^*(\lambda)>-\infty$ for all $\lambda \in X^*$. Therefore $f^*$ is proper.

Suppose that $f^*$ is proper. If $f$ took only the value $\infty$ then $f^*$ would take only  the value $-\infty$, and $f^*$ being proper means that
it in fact never takes the value $-\infty$. Let $x \in X$. As $f^*$ is proper there is some $\lambda \in X^*$ for which
$f^*(\lambda) < \infty$, and using  Young's inequality \eqref{young} we get
\[
f(x) \geq - f^*(\lambda) +\inner{x}{\lambda}>-\infty.
\]
Thus, for every $x \in X$ we have $f(x)>-\infty$, so we have verified that $f$ is proper.
\end{proof}


The following theorem is called the {\em Fenchel-Moreau theorem}, and gives necessary and sufficient conditions for 
 a function to equal  its convex biconjugate.\footnote{Viorel Barbu and Teodor Precupanu, {\em Convexity and Optimization in Banach Spaces}, fourth ed., p.~79, Theorem 2.22.}

\begin{theorem}[Fenchel-Moreau theorem]
If $X$ is a real locally convex space and $f:X \to [-\infty,\infty]$ is a function, then $f=f^{**}$ if and only if one of the following three conditions holds:
\begin{enumerate}
\item $f$ is a proper convex lower semicontinuous function
\item $f$ is the constant function $\infty$
\item $f$ is the constant function $-\infty$
\end{enumerate}
\end{theorem}
\begin{proof}
Suppose that $f$ is a proper convex lower semicontinuous function. From Lemma  \ref{properiff}, its convex conjugate $f^*$ is a proper convex
function. 
As $f^*$ does not take only the value $\infty$  we get from the definition of $f^{**}$  that $f^{**}(x) > -\infty$ for every $x \in X$.
Theorem \ref{conjugateproperties} tells us that $f^{**} \leq f$, and suppose by contradiction that there is some
$x_0 \in X$ for which $-\infty<f^{**}(x_0) < f(x_0)$.
For this $x_0$ we have $(x_0,f^{**}(x_0)) \not \in \epi f$, so we can apply the Hahn-Banach separation theorem to the sets
$\{(x_0,f^{**}(x_0))\}$ and $\epi f$ to get that there is some $\Lambda \in (X \times \mathbb{R})^*$ and some $\gamma \in \mathbb{R}$
for which
\[
 \Lambda (x,\alpha) < \gamma <  \Lambda (x_0,f^{**}(x_0)) , \qquad (x,\alpha) \in \epi f.
\]
But $\Lambda (x,\alpha)$ can be written as $\lambda x + \beta \alpha$ for some $\lambda \in X^*$ and some $\beta \in \mathbb{R}^*=\mathbb{R}$, so 
\begin{equation}
\lambda x + \beta \alpha < \gamma < \lambda x_0 + \beta f^{**}(x_0), \qquad (x,\alpha) \in \epi f.
\label{lambdainequality}
\end{equation}
If $\beta$ were $>0$ then the left-hand side of \eqref{lambdainequality} would be $\infty$, because for $x \in \dom f$ there are arbitrarily large $\alpha$ such that
$(x,\alpha) \in \epi f$.  But the left-hand side is upper bounded by
the constant right-hand side, so  $\beta \leq 0$. 
Either $f(x_0)<\infty$ or $f(x_0)=\infty$. In the first case, $(x_0,f(x_0)) \in \epi f$, and applying \eqref{lambdainequality}
gives $\beta f(x_0) < \beta f^{**}(x_0)$. The assumption that $f^{**}(x_0)<f(x_0)$ then implies that $\beta<0$.
In the case that $f(x_0)=\infty$,  assume by contradiction that $\beta=0$. 
Then \eqref{lambdainequality} becomes
\begin{equation}
\lambda x <\gamma< \lambda x_0, \qquad x \in \dom f.
\label{alphais0}
\end{equation}
Let $\mu \in \dom f^*$; $f^*$ is proper so $\dom f^* \neq \emptyset$.
For all $h>0$ we have
\begin{align*}
f^*(\mu+h\lambda)&=\sup\{ \inner{x}{\mu+h\lambda} - f(x): x\in X\}\\
&=\sup\{ \inner{x}{\mu+h\lambda} - f(x): x\in \dom f\}\\
&\leq \sup\{ \inner{x}{\mu}-f(x): x \in \dom f\} + h \sup\{\inner{x}{\lambda}:x \in \dom f\}\\
&=f^*(\mu)+ h  \sup\{\inner{x}{\lambda}:x \in \dom f\}.
\end{align*}
But using the definition of $f^{**}$ we have
$f^{**}(x_0)\geq \inner{x_0}{\mu+h\lambda} - f^*(\mu+h\lambda)$, so
\begin{align*}
f^{**}(x_0)&\geq \inner{x_0}{\mu}+h \inner{x_0}{\lambda}- f^*(\mu+h\lambda)\\
&\geq \inner{x_0}{\mu}+h \inner{x_0}{\lambda}  -f^*(\mu)-h  \sup\{\inner{x}{\lambda}:x \in \dom f\}\\
&=\inner{x_0}{\mu} -f^*(\mu) +h( \inner{x_0}{\lambda} -  \sup\{\inner{x}{\lambda}:x \in \dom f\}).
\end{align*}
But \eqref{alphais0} tells us that $\inner{x_0}{\lambda}-\sup\{\inner{x}{\lambda}:x \in \dom f\} >0$, and since the above inequality
holds for arbitrarily large $h$ we get $f^{**}(x_0)=\infty$, contradicting that $f^{**}(x_0)<f(x_0)$. Therefore, $\beta<0$, and then
we can divide \eqref{lambdainequality} by $\beta$ to obtain
\[
\frac{1}{\beta}\lambda x  + \alpha > \frac{\gamma}{\beta} > \frac{1}{\beta} \lambda x_0 + f^{**}(x_0), \qquad (x,\alpha) \in \epi f,
\]
hence
\begin{align*}
\lambda \left( -\frac{x_0}{\beta} \right) - f^{**}(x_0)&>\sup\left\{ -\frac{1}{\beta} \lambda x  -\alpha: (x,\alpha) \in \epi f \right\}\\
&=\sup\left\{ -\frac{1}{\beta} \lambda x - f(x): x \in \dom f\right\}\\
&=f^*\left( -\frac{1}{\beta} \lambda \right).
\end{align*}
From the definition of $f^{**}$ we have
\[
f^{**}(x_0 )\geq \inner{x_0}{-\frac{1}{\beta} \lambda} - f^*\left(-\frac{1}{\beta} \lambda\right),
\]
and this and the above give
\[
\lambda \left( -\frac{x_0}{\beta} \right) - f^*\left( -\frac{1}{\beta} \lambda \right)
>\inner{x_0}{-\frac{1}{\beta} \lambda} - f^*\left(-\frac{1}{\beta} \lambda\right),
\]
but the two sides are equal, a contradiction. Therefore, $f^{**}(x_0) = f(x_0)$. 

Suppose that $f$ is the constant function $\infty$. Then $f^*$ is the constant function $-\infty$, and this means that $f^{**}$
is the constant function $\infty$, giving $f=f^{**}$.

Suppose that $f$ is the constant function $-\infty$. This implies that $f^*$ is the constant function $\infty$, and so
$f^{**}$ is the constant function $-\infty$, giving $f=f^{**}$.

Suppose that $f=f^{**}$. By Theorem \ref{conjugateproperties}, $f^{**}$ is convex and weakly lower semicontinuous, so a fortiori it is lower semicontinuous, 
hence $f$ is convex and lower semicontinuous.
Suppose that $f$ is neither the constant function
$\infty$ nor the constant function $-\infty$. 
If $f$ took the value $-\infty$ then $f^*$ would take
only the value $\infty$, and then $f^{**}$ would take only the value $-\infty$, contradicting that $f=f^{**}$ is not the constant function
$-\infty$. 
Therefore, if $f=f^{**}$ then either $f$ is the constant function $\infty$, or $f$ is the constant function $-\infty$, or $f$ is a proper convex 
 lower
semicontinuous function.
\end{proof}



If $X$ is a topological space and $f:X \to [-\infty,\infty]$ is a function, the {\em closure of $f$} is the function $\cl f:X \to [-\infty,\infty]$ that is defined to be
$\lsc f$ if $(\lsc f)(x)>-\infty$ for all $x \in X$, and  defined to be the constant function $-\infty$ if there is some $x \in X$ such that $(\lsc f)(x)=-\infty$.
We say that a function is {\em closed} if it is equal to its closure, and thus
 to say that a function is closed is to say that it is lower semicontinuous, and 
 either does not take the value $-\infty$ or only takes the value $-\infty$. One checks that $(\cl f)^*=f^*$, and combined with the Fenchel-Moreau theorem one
 can obtain the following.


\begin{corollary}
If $X$ is a real locally convex space and $f:X \to [-\infty,\infty]$ is a convex function, then $f^{**}=\cl f$.
\end{corollary}



\end{document}