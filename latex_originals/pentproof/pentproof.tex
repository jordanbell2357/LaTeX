\documentclass{article}
\usepackage{amsmath,amssymb,graphicx,subfig,amsthm}
\newcommand{\norm}[1]{\Vert #1 \Vert}
\newtheorem{theorem}{Theorem}
\newtheorem{remark}[theorem]{Remark}
\begin{document}
\title{Proof of the pentagonal number theorem}
\author{Jordan Bell\\ \texttt{jordan.bell@gmail.com}\\Department of Mathematics, University of Toronto}
\date{\today}
\maketitle

Let $A_0=\prod_{k=1}^\infty (1-z^k)$. We will use the identity
\[
\prod_{k=1}^N (1-a_k) = 1- a_1 - \sum_{k=2}^N a_k(1-a_1)\cdots(1-a_{k-1}),
\]
which 
is straightforward to prove by induction. We apply the identity with $a_k=z^k$ and $N=\infty$, which gives
\begin{eqnarray*}
A_0&=&1-z-\sum_{k=2}^\infty z^k(1-z)\cdots(1-z^{k-1})\\
&=&1-z-\sum_{k=0}^\infty z^{k+2}(1-z)\cdots(1-z^{k+1}).
\end{eqnarray*}
For $n \geq 1$ let
$A_n=\sum_{k=0}^\infty z^{nk}(1-z^n)\cdots (1-z^{n+k})$. We have
$A_0=1-z-z^2A_1$, and for $n \geq 1$ we have
\begin{eqnarray*}
A_n&=&1-z^n+\sum_{k=1}^\infty z^{nk}(1-z^n)\cdots(1-z^{n+k})\\
&=&1-z^n+\sum_{k=1}^\infty z^{nk}(1-z^{n+1})\cdots(1-z^{n+k})\\
&&-\sum_{k=1}^\infty z^{n(k+1)}(1-z^{n+1})\cdots(1-z^{n+k})\\
&=&1-z^n+z^n(1-z^{n+1})+\sum_{k=2}^\infty z^{nk}(1-z^{n+1})\cdots(1-z^{n+k})\\
&&-\sum_{k=1}^\infty z^{n(k+1)}(1-z^{n+1})\cdots(1-z^{n+k})\\
&=&1-z^{2n+1}+\sum_{k=0}^\infty z^{n(k+2)}(1-z^{n+1})\cdots(1-z^{n+k+2})\\
&&-\sum_{k=0}^\infty z^{n(k+2)}(1-z^{n+1})\cdots(1-z^{n+k+1})\\
&=&1-z^{2n+1}-\sum_{k=0}^\infty z^{n(k+2)+n+k+2} (1-z^{n+1})\cdots (1-z^{n+k+1})\\
&=&1-z^{2n+1}-z^{3n+2}\sum_{k=0}^\infty z^{(n+1)k}(1-z^{n+1})\cdots(1-z^{n+k+1})\\
&=&1-z^{2n+1}-z^{3n+2}A_{n+1}.
\end{eqnarray*}
Therefore $A_n=1-z^{2n+1}-z^{3n+2}A_{n+1}$ for all $n \geq 0$.

We then check by induction that for all $M$
\begin{eqnarray*}
A_0&=&1-z+\sum_{n=1}^M (-1)^n \Big( z^{n(3n+1)/2}-z^{(n+1)(3n+2)/2} \Big)\\
&&+(-1)^{M+1} z^{(M+1)(3M+2)/2}A_{M+1},
\end{eqnarray*}
and taking $M=\infty$ gives the pentagonal number theorem.

\end{document}