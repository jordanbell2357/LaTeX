\documentclass{article}
\usepackage{amsmath,amssymb,graphicx,subfig,mathrsfs,amsthm,siunitx}
%\usepackage{tikz-cd}
\usepackage{hyperref}
\newcommand{\inner}[2]{\left\langle #1, #2 \right\rangle}
\newcommand{\tr}{\ensuremath\mathrm{tr}\,} 
\newcommand{\Span}{\ensuremath\mathrm{span}} 
\def\Re{\ensuremath{\mathrm{Re}}\,}
\def\Im{\ensuremath{\mathrm{Im}}\,}
\newcommand{\id}{\ensuremath\mathrm{id}} 
\newcommand{\sgn}{\ensuremath\mathrm{sgn}\,} 
\newcommand{\rank}{\ensuremath\mathrm{rank\,}} 
\newcommand{\co}{\ensuremath\mathrm{co}\,} 
\newcommand{\cco}{\ensuremath\overline{\mathrm{co}}\,}
\newcommand{\supp}{\ensuremath\mathrm{supp}\,}
\newcommand{\epi}{\ensuremath\mathrm{epi}\,}
\newcommand{\lsc}{\ensuremath\mathrm{lsc}\,}
\newcommand{\ext}{\ensuremath\mathrm{ext}\,}
\newcommand{\cl}{\ensuremath\mathrm{cl}\,}
\newcommand{\dom}{\ensuremath\mathrm{dom}\,}
\newcommand{\LSC}{\ensuremath\mathrm{LSC}}
\newcommand{\USC}{\ensuremath\mathrm{USC}}
\newcommand{\Cyl}{\ensuremath\mathrm{Cyl}\,}
\newcommand{\extreals}{\overline{\mathbb{R}}}
\newcommand{\upto}{\nearrow}
\newcommand{\downto}{\searrow}
\newcommand{\norm}[1]{\left\Vert #1 \right\Vert}
\newtheorem{theorem}{Theorem}
\newtheorem{lemma}[theorem]{Lemma}
\newtheorem{proposition}[theorem]{Proposition}
\newtheorem{corollary}[theorem]{Corollary}
\theoremstyle{definition}
\newtheorem{definition}[theorem]{Definition}
\newtheorem{example}[theorem]{Example}
\begin{document}
\title{Test functions, distributions, and Sobolev's lemma}
\author{Jordan Bell\\ \texttt{jordan.bell@gmail.com}\\Department of Mathematics, University of Toronto}
\date{\today}

\maketitle

\section{Introduction}
If $X$ is a topological vector space, we denote by $X^*$ the set of continuous linear functionals on $X$. With the weak-* topology, $X^*$ is a locally convex space, whether or not
$X$ is a locally convex space. (But in this note, we only talk about locally convex spaces.)

The purpose of this note is to collect the material given in Walter Rudin, {\em Functional Analysis}, second ed., chapters 6 and 7,
 involved in stating and proving Sobolev's lemma.
 


\section{Test functions}
Suppose that $\Omega$ is an open subset of $\mathbb{R}^n$. We denote by $\mathscr{D}(\Omega)$ the set of all $\phi \in C^\infty(\Omega)$ such that
$\supp \phi$ is a compact subset of $\Omega$. Elements of $\mathscr{D}(\Omega)$ are called {\em test functions}. For $N=0,1,\ldots$ and $\phi \in \mathscr{D}(\Omega)$, write
\[
\norm{\phi}_N = \sup\{|(D^\alpha \phi)(x)|: x\in \Omega, |\alpha| \leq N\},
\]
where
\[
D^\alpha = D_1^{\alpha_1} \cdots D_n^{\alpha_n}, \qquad |\alpha| = \alpha_1 + \cdots + \alpha_n.
\]
For each compact
subset $K$ of $\Omega$, we define
\[
\mathscr{D}_K = \{\phi \in \mathscr{D}(\Omega): \supp \phi \subseteq K\},
\] 
and define   $\tau_K$ to be the locally convex topology on $\mathscr{D}_K$ determined by the  family of seminorms $\{\norm{\cdot}_N: N \geq 0\}$. One proves that 
$\mathscr{D}_K$  with the topology $\tau_K$ is a Fr\'echet space. 
As sets,
\[
\mathscr{D}(\Omega)= \bigcup_K \mathscr{D}_K.
\]
Define $\beta$ to be the collection of all convex balanced subsets $W$ of $\mathscr{D}(\Omega)$ such that for every compact subset $K$ of $\Omega$ we have
$W\cap \mathscr{D}_K \in \tau_K$; to say that $W$ is {\em balanced} means that if $c$ is a complex number with $|c| \leq 1$ then
$cW \subseteq W$. One proves that
 $\{\phi+W: \phi \in \mathscr{D}(\Omega), W \in \beta\}$ is a basis for a topology $\tau$ on $\mathscr{D}(\Omega)$,  that  
 $\beta$ is a local basis at $0$ for this topology,
and that with the topology $\tau$,  $\mathscr{D}(\Omega)$ 
is a locally convex space.\footnote{Walter Rudin, {\em Functional Analysis}, second ed., p.~152, Theorem 6.4; cf. Helmut H. Schaefer,
{\em Topological Vector Spaces}, p.~57.} For each compact subset $K$ of $\Omega$,
 one proves that the topology $\tau_K$ is equal to the subspace topology on $\mathscr{D}_K$ inherited from
 $\mathscr{D}(\Omega)$.\footnote{Walter
 Rudin, {\em Functional Analysis}, second ed., p.~153, Theorem 6.5.} 

We write $\mathscr{D}'(\Omega)= (\mathscr{D}(\Omega))^*$, and elements of $\mathscr{D}'(\Omega)$ are called {\em distributions}. 
With the weak-* topology, $\mathscr{D}'(\Omega)$ is a locally convex space.

It is a fact that a linear functional  $\Lambda$ on $\mathscr{D}(\Omega)$ is continuous if and only if 
for every compact subset $K$ of $\Omega$ there is a nonnegative integer $N$ and a constant
$C$ such that $|\Lambda \phi| \leq C \norm{\phi}_N$ for all $\phi \in \mathscr{D}_K$.\footnote{Walter
Rudin, {\em Functional Analysis}, second ed., p.~156, Theorem 6.8.}


For $\Lambda \in \mathscr{D}'(\Omega)$ and $\alpha$ a multi-index,
 we define
\[
(D^\alpha \Lambda)(\phi) = (-1)^{|\alpha|} \Lambda(D^\alpha \phi), \qquad \phi \in \mathscr{D}(\Omega).
\]
Let $K$ be a compact subset of $\Omega$. As $\Lambda$ is continuous, there is a nonnegative integer $N$ and a constant
$C$ such that $|\Lambda \phi| \leq C \norm{\phi}_N$ for all $\phi \in \mathscr{D}_K$. Then
\[
|(D^\alpha \Lambda)(\phi)| = |\Lambda(D^\alpha \phi)| \leq C \norm{D^\alpha \phi}_N \leq C \norm{\phi}_{N+|\alpha|},
\]
which shows that $D^\alpha \Lambda \in \mathscr{D}'(\Omega)$. 


 The {\em Leibniz formula} is the statement that 
for all $f,g \in C^\infty(\mathbb{R}^n)$,  
\[
D^\alpha(fg) = \sum_{\beta \leq \alpha} \binom{\alpha}{\beta} (D^{\alpha-\beta}f)(D^\beta g),
\]
where $\binom{\alpha}{\beta}$ are multinomial coefficients.

For $\Lambda \in \mathscr{D}'(\Omega)$ and $f \in C^\infty(\Omega)$, we define 
\[
(f\Lambda)(\phi) = \Lambda(f\phi), \qquad \phi \in \mathscr{D}(\Omega);
\]
this makes sense because $f\phi \in \mathscr{D}(\Omega)$ when $\phi \in \mathscr{D}(\Omega)$. It is apparent that $f\Lambda$
is linear, and in the following lemma we prove that $f\Lambda$ is continuous.\footnote{Walter
Rudin, {\em Functional Analysis}, second ed., p.~159, \S 6.15.}

\begin{lemma}
If $\Lambda \in \mathscr{D}'(\Omega)$ and $f \in C^\infty(\Omega)$, then $f \Lambda \in \mathscr{D}'(\Omega)$.
\label{module}
\end{lemma}
\begin{proof}
Suppose that $K$ is a compact subset of $\Omega$. Because $\Lambda$ is continuous, 
there is some nonnegative integer $N$ and some constant $C$ such that
\[
|\Lambda \phi| \leq C \norm{\phi}_N, \qquad \phi \in \mathscr{D}_K.
\]

For $|\alpha| \leq N$, by the Leibniz formula, for all
$\phi \in \mathscr{D}_K$,
\[
D^\alpha(f\phi) = \sum_{\beta \leq \alpha} \binom{\alpha}{\beta} (D^{\alpha-\beta}f)(D^\beta \phi).
\]
Because $f \in C^\infty(\Omega)$, there is some $C_{\alpha}$ such that $|(D^{\alpha-\beta}f)(x)| \leq C_{\alpha}$ for $\beta \leq \alpha$ and for $x \in K$. 
Using $\phi(x)=0$ for $x \not \in K$,  the above statement of the Leibniz formula, and  the inequality
just obtained,
it follows that there is some $C_{\alpha}'$ such that $|(D^\alpha(f\phi))(x)| \leq C_{\alpha}' \norm{\phi}_N$ for all $x \in \Omega$.
This gives
\[
\norm{f\phi}_N=\sup_{|\alpha| \leq N} \sup_{x \in \Omega} |(D^\alpha (f\phi))(x)|
\leq
\sup_{|\alpha| \leq N} C_{\alpha}' \norm{\phi}_N=C'\norm{\phi}_N;
\]
the last equality is how we define $C'$, which is a maximum of finitely many $C_\alpha'$ and so finite.
Then,
\[
|(f\Lambda)(\phi)| = |\Lambda(f\phi)| \leq C \norm{f\phi}_N \leq CC'\norm{\phi}_N, \qquad \phi \in \mathscr{D}_K.
\]
This bound shows that $f\Lambda$ is continuous.
\end{proof}

The above lemma shows that $f \Lambda \in \mathscr{D}'(\Omega)$ when $f \in C^\infty(\Omega)$ and $\Lambda \in \mathscr{D}'(\Omega)$.
Therefore $D^\alpha(f\Lambda) \in \mathscr{D}(\Omega)$, and the following lemma, proved in Rudin, states that the Leibniz formula can be used with $f\Lambda$.\footnote{Walter Rudin, {\em Functional Analysis}, second ed.,
p.~160, \S 6.15.}

\begin{lemma}
If $f \in C^\infty(\Omega)$ and $\Lambda \in \mathscr{D}'(\Omega)$, then 
\[
D^\alpha(f\Lambda) = \sum_{\beta \leq \alpha} \binom{\alpha}{\beta} (D^{\alpha-\beta}f)(D^\beta \Lambda).
\]
\label{leibniz}
\end{lemma}


If $f:\Omega \to \mathbb{C}$ is locally integrable,  define
\[
\Lambda \phi =\int_{\Omega} \phi(x) f(x) dx, \qquad \phi \in \mathscr{D}(\Omega).
\]
For $\phi \in \mathscr{D}_K$,
\[
|\Lambda \phi | \leq \norm{\phi}_0 \int_K |f| dx,
\]
from which it follows that $\Lambda$ is continuous. 
If $\mu$ is a complex Borel measure on $\mathbb{R}^n$ or a positive Borel measure on $\mathbb{R}^n$ that assigns finite measure to compact
sets,  define
\[
\Lambda \phi = \int_{\Omega} \phi d\mu, \qquad \phi \in \mathscr{D}(\Omega).
\]
For $\phi \in \mathscr{D}_K$,
\[
|\Lambda \phi| \leq \norm{\phi}_0 |\mu|(K),
\]
from which it follows that $\Lambda$ is continuous.
Thus, we can encode certain functions and measures as distributions. I will dare to say that we can encode most functions and measures
that we  care about as distributions.


If $\Lambda_1,\Lambda_2 \in \mathscr{D}'(\Omega)$ and $\omega$ is an open subset of $\Omega$, we say that
{\em $\Lambda_1 = \Lambda_2$ in $\omega$} if $\Lambda_1 \phi = \Lambda_2 \phi$ for all $\phi \in \mathscr{D}(\omega)$. 

Let $\Lambda \in \mathscr{D}'(\Omega)$ and let  $\omega$ be an open subset of $\Omega$. We say that $\Lambda$
{\em vanishes on $\omega$} if $\Lambda \phi = 0$ for all $\phi \in \mathscr{D}(\omega)$.
Taking $W$ to be the union of all open subsets $\omega$ of $\Omega$ on which $\Lambda$ vanishes, we define the {\em support of $\Lambda$} to be the set 
$\Omega \setminus W$. 




\section{The Fourier transform}
Let $C_0(\mathbb{R}^n)$ be the set of those continuous functions $f:\mathbb{R}^n \to \mathbb{C}$ such that for every $\epsilon>0$, there is some compact
set $K$ such that $|f(x)|<\epsilon$ for $x \not \in K$. With the supremum norm $\norm{\cdot}_\infty$, $C_0(\mathbb{R}^n)$ is a Banach space.

Let $m_n$ be {\em normalized Lebesgue measure on $\mathbb{R}^n$}: 
\[
dm_n(x) = (2\pi)^{-n/2} dx.
\]
Using $m_n$, we define
\[
\norm{f}_{L^p} = \left( \int_{\mathbb{R}^n} |f|^p dm_n\right)^{1/p}, \qquad 1 \leq p < \infty
\]
and 
\[
(f*g)(x) = \int_{\mathbb{R}^n} f(x-y) g(y) dm_n(y).
\]


For $t \in \mathbb{R}^n$, define $e_t:\mathbb{R}^n \to \mathbb{C}$ by
\[
e_t(x) = \exp(it\cdot x), \qquad x \in \mathbb{R}^n.
\]
The {\em Fourier transform} of $f \in L^1(\mathbb{R}^n)$ is the function $\hat{f}:\mathbb{R}^n \to \mathbb{C}$ defined by
\[
(\mathscr{F}f)(t) = \hat{f}(t) = \int_{\mathbb{R}^n} f e_{-t} dm_n, \qquad t \in \mathbb{R}^n.
\]
Using the dominated convergence theorem, one shows that $\hat{f}$ is continuous.






For $f \in C^\infty(\mathbb{R}^n)$ and $N$ a nonnegative integer, write
\[
p_N(f) = \sup_{|\alpha| \leq N} \sup_{x \in \mathbb{R}^n} (1+|x|^2)^N |(D^\alpha f)(x)|,
\]
and let $\mathscr{S}_n$ be the set of those $f \in C^\infty(\mathbb{R}^n)$ such that for every nonnegative integer $N$, $p_N(f)<\infty$.
$\mathscr{S}_n$ is a vector space, and with the locally convex topology determined by the family of seminorms
$\{p_N: N \geq 0\}$ it is a Fr\'echet space.\footnote{Walter Rudin, {\em Functional Analysis}, second ed., p.~184, Theorem 7.4.}
Further, one proves that $\mathscr{F}:\mathscr{S}_n \to \mathscr{S}_n$ is a continuous linear map.\footnote{Walter
Rudin, {\em Functional Analysis}, second ed., p.~184, Theorem 7.4.}

The {\em Riemann-Lebesgue lemma} is the statement that if $f \in L^1(\mathbb{R}^n)$, then $\hat{f} \in C_0(\mathbb{R}^n)$.\footnote{Walter
Rudin, {\em Functional Analysis}, second ed., p.~185, Theorem 7.5.}


The {\em  inversion theorem}\footnote{Walter Rudin, {\em Functional Analysis},
second ed., p.~186, Theorem 7.7.} is the statement that if
 $g \in \mathscr{S}_n$ then
\[
g(x) = \int_{\mathbb{R}^n} \hat{g}e_x dm_n, \qquad x \in \mathbb{R}^n,
\]
and that if
$f \in L^1(\mathbb{R}^n)$ and $\hat{f} \in L^1(\mathbb{R}^n)$,  and we define $f_0 \in C_0(\mathbb{R}^n)$  by 
\[
f_0(x) = \int_{\mathbb{R}^n} \hat{f} e_x dm_n, \qquad x \in \mathbb{R}^n,
\]
then $f(x)=f_0(x)$ for almost all $x \in \mathbb{R}^n$. For $g \in \mathscr{S}_n$,
as $\hat{g} \in \mathscr{S}_n$, the function
$f(t)=\hat{g}(-t)$ belongs to $\mathscr{S}_n$. The inversion theorem tells us that for all $x \in \mathbb{R}^n$,
\[
g(x) = \int_{\mathbb{R}^n} \hat{g}(t) e_x(t) dm_n(t) = \int_{\mathbb{R}^n} \hat{g}(-t) e_x(-t) dm_n(t) 
=\int_{\mathbb{R}^n} f(t) e_{-x}(t) dm_n(t),
\]
and hence that $g=\hat{f}$. 
This shows that $\mathscr{F}:\mathscr{S}_n \to \mathscr{S}_n$ is onto.
Using the inversion theorem, one checks that 
\[
\int_{\mathbb{R}^n} f\overline{g} dm_n = \int_{\mathbb{R}^n} \hat{f} \overline{\hat{g}} dm_n,
\qquad f,g \in \mathscr{S}_n,
\]
and so $\norm{f}_{L^2} = \norm{\mathscr{F}f}_{L^2}$ for $f \in \mathscr{S}_n$. It is a fact that
$\mathscr{S}_n$ is a dense subset of the Hilbert space $L^2(\mathbb{R}^n)$, and it follows that there is a unique bounded
linear operator $L^2(\mathbb{R}^n) \to L^2(\mathbb{R}^n)$, that is equal to $\mathscr{F}$ on $\mathscr{S}_n$, and that is unitary.
We denote this  $\mathscr{F}:L^2(\mathbb{R}^n) \to L^2(\mathbb{R}^n)$. 

It is a fact that $\mathscr{D}(\mathbb{R}^n)$ is a dense subset of $\mathscr{S}_n$ and that
the identity map $i:\mathscr{D}(\mathbb{R}^n) \to \mathscr{S}_n$ is continuous.\footnote{Walter Rudin, {\em Functional Analysis},
second ed., p.~189, Theorem 7.10.} If $L_1,L_2 \in (\mathscr{S}_n)^*$  are distinct, then there is some $f \in \mathscr{S}_n$ such that
$L_1 f \neq L_2 f$, and as $\mathscr{D}(\mathbb{R}^n)$ is dense in $\mathscr{S}_n$, there is a sequence $f_j \in \mathscr{D}(\mathbb{R}^n)$
with $f_j \to f$ in $\mathscr{S}_n$. As
\[
(L_1 \circ i)(f_j) - (L_2 \circ i)(f_j) =  L_1 f_j - L_2 f_j \to L_1 f_j -L_2 f_j \neq 0,
\]
there is some $f_j$ with $(L_1 \circ i)(f_j) \neq (L_2 \circ i)(f_j)$, and hence $L_1 \circ i \neq L_2 \circ i$. This shows that
$L \mapsto L \circ i$ is a one-to-one linear map $(\mathscr{S}_n)^* \to \mathscr{D}'(\mathbb{R}^n)$. 
Elements of $\mathscr{D}'(\mathbb{R}^n)$ of the form $L \circ i$ for $L \in (\mathscr{S}_n)^*$ are called
{\em tempered distributions}, and we denote the set of tempered distributions by
$\mathscr{S}_n'$.
It is a fact that every distribution with compact support is tempered.\footnote{Walter Rudin, {\em Functional Analysis}, second ed., p.~190, Example 7.12 (a).}


\section{Sobolev's lemma}
Suppose that $\Omega$ is an open subset of $\mathbb{R}^n$. We say that a measurable function $f:\Omega \to \mathbb{C}$
 is {\em locally $L^2$}
if $\int_K |f|^2 dm_n<\infty$ for every compact subset $K$ of $\Omega$.
We say that $\Lambda \in \mathscr{D}'(\Omega)$ is {\em locally $L^2$} if there is a function $g$ that is locally $L^2$ in $\Omega$
such that $\Lambda \phi=\int_{\Omega} \phi g dm_n$ for every $\phi \in \mathscr{D}(\Omega)$.

 The following proof of Sobolev's lemma follows Rudin.\footnote{Walter Rudin, {\em Functional Analysis}, second ed., p.~202, Theorem 7.25.}

\begin{theorem}[Sobolev's lemma]
Suppose that $n,p,r$ are integers, $n>0$, $p \geq 0$, and 
\[
r>p+\frac{n}{2}.
\]
Suppose that $\Omega$ is an open subset of $\mathbb{R}^n$, that $f:\Omega \to \mathbb{C}$ is locally $L^2$, and that the distribution derivatives
$D_j^k f$ are locally $L^2$ for $1 \leq j \leq n$, $1 \leq k \leq r$. Then there is some $f_0 \in C^p(\Omega)$ such that $f_0(x)=f(x)$ for almost all $x \in \Omega$.
\end{theorem}
\begin{proof}
To say that the distribution derivative $D_j^k f$ is locally $L^2$ means that there is some $g_{j,k}:\Omega \to \mathbb{C}$ that is locally $L^2$ such that 
\[
D_j^k \Lambda_f = \Lambda_{g_{j,k}}.
\]

Suppose that  $\omega$ is an open subset of $\Omega$ whose closure $K$ is a compact subset of $\Omega$. There is some $\psi \in \mathscr{D}(\Omega)$
with $\psi(x)=1$ for $x \in K$, and we define $F:\mathbb{R}^n \to \mathbb{C}$ by
\[
F(x) = \begin{cases}
\psi(x)f(x)&x \in \Omega,\\
0&x \not \in \Omega;
\end{cases}
\]
in particular, for $x \in K$ we have $F(x)=f(x)$, and for $x \not \in \supp \psi$ we have $F(x)=0$.
Because $\supp \psi \subset \Omega$ is compact and $f$ is locally $L^2$,
\[
\norm{F}_{L^2} = \left( \int_{\supp \psi} |\psi f|^2 dm_n\right)^{1/2} \leq \norm{\psi}_0 \left(\int_{\supp \psi} |f|^2 dm_n \right)^{1/2} < \infty,
\]
and using the Cauchy-Schwarz inequality, $\norm{F}_{L^1} \leq \norm{F}_{L^2} m_n(\supp \psi)^{1/2}<\infty$, so
\[
F \in L^2(\mathbb{R}^n) \cap L^1(\mathbb{R}^n).
\]
Then,
\begin{equation}
\int_{\mathbb{R}^n} |\widehat{F}|^2 dm_n < \infty.
\label{Fhat}
\end{equation}

Because $\Lambda_F = \psi \Lambda_f$ in $\Omega$,
 the Leibniz formula tells us that in $\Omega$,
\[
D_j^r \Lambda_F =D_j^r(\psi \Lambda_f)
=\sum_{s=0}^r \binom{r}{s} (D_j^{r-s} \psi)(D_j^s \Lambda_f)
=\sum_{s=0}^r \binom{r}{s} (D_j^{r-s} \psi)(\Lambda_{g_{j,s}}),
\]
hence, defining $H_j:\mathbb{R}^n \to \mathbb{C}$ by
\[
H_j(x) = \begin{cases}
\sum_{s=0}^r \binom{r}{s} (D_j^{r-s} \psi)(x)g_{j,s}(x)&x \in \Omega,\\
0&x \not \in \Omega,
\end{cases}
\]
we have $D_j^r \Lambda_F = \Lambda_{H_j}$
in $\Omega$.  It is apparent that $H_j \in L^2(\mathbb{R}^n) \cap L^1(\mathbb{R}^n)$.

Let $\phi \in \mathscr{D}(\mathbb{R}^n)$. There are $\phi_1,\phi_2 \in \mathscr{D}(\mathbb{R}^n)$ with $\phi=\phi_1+\phi_2$ and
$\supp \phi_1 \subset \Omega$, $\supp \phi_2 \subset \mathbb{R}^n \setminus \supp \psi$.\footnote{$\phi_1$ and $\phi_2$ are constructed
using a partition of unity. See
Walter Rudin, {\em Functional Analysis}, second ed., p.~162, Theorem 6.20.} We have just established that
$(D_j^r \Lambda_F)\phi_1=\Lambda_{H_j} \phi_1$. For $\phi_2$, it is apparent that
\[
(D_j^r \Lambda_F)\phi_2 = \Lambda_F(D_j^r \phi_2) = \int_{\mathbb{R}^n} (D_j^r \phi_2)(x) F(x) dm_n(x)=0
\]
and
\[
\Lambda_{H_j} \phi_2 = \int_{\mathbb{R}^n} \phi_2(x) H_j(x) dm_n(x) = 0.
\]
Hence $(D_j^r \Lambda_F)(\phi) = \Lambda_{H_j}\phi$. It is apparent that $\Lambda_{H_j}$ has compact support, 
so $D_j^r \Lambda_F = \Lambda_{H_j}$ are tempered distributions. Let $\xi \in \mathscr{S}_n$, and take $\phi \in \mathscr{S}_n$ with $\xi=\hat{\phi}$. Then,
\begin{eqnarray*}
(D_j^r \Lambda_F)  \phi&=&\Lambda_F D_j^r \phi\\
& =& \int_{\mathbb{R}^n} (D_j^r \phi)(x) F(x) dm_n(x)\\
&=&\int_{\mathbb{R}^n} \mathscr{F}(D_j^r \phi)(y) \widehat{F}(y) dm_n(y)\\
&=&\int_{\mathbb{R}^n} (iy_j)^r \xi(y) \widehat{F}(y) dm_n(y),
\end{eqnarray*}
and
\[
\Lambda_{H_j} \phi = \int_{\mathbb{R}^n} \phi(x) H_j(x) dm_n(x) = \int_{\mathbb{R}^n} \xi(y) \widehat{H_j}(y) dm_n(y).
\]
It follows that  $(iy_j)^r\widehat{F}(y)= \widehat{H_j}(y)$ for all $y \in \mathbb{R}^n$. But $\widehat{H_j} \in L^2(\mathbb{R}^n)$, so
\begin{equation}
\int_{\mathbb{R}^n} y_i^{2r} |\widehat{F}(y)|^2 dm_n(y)< \infty, \qquad 1 \leq i \leq n.
\label{Fhati}
\end{equation}

Using \eqref{Fhat}, \eqref{Fhati}, and the inequality
\[
(1+|y|)^{2r} < (2n+2)^r(1+y_1^{2r} + \cdots + y_n^{2r}), \qquad y \in \mathbb{R}^n,
\]
we get
\[
J = \int_{\mathbb{R}^n} (1+|y|)^{2r} |\widehat{F}(y)|^2 dm_n(y)<\infty.
\]
Let $\sigma_{n-1}$ be surface measure on $S^{n-1}$, with  $\sigma_{n-1}(S^{n-1})=\frac{2\pi^{n/2}}{\Gamma(n/2)}$. 
Using the Cauchy-Schwarz inequality and the change of variable $y=tu$, $u \in S^{n-1}$, $t \geq 0$,
\begin{eqnarray*}
\left( \int_{\mathbb{R}^n} (1+|y|)^p |\widehat{F}(y)| dm_n(y)\right)^2 & =&\left( \int_{\mathbb{R}^n} (1+|y|)^r |\widehat{F}(y)| (1+|y|)^{p-r} dm_n(y)\right)^2\\
&\leq&J  \int_{\mathbb{R}^n} (1+|y|)^{2p-2r} dm_n(y)\\
&=&J(2\pi)^{-n/2} \int_{0}^\infty \int_{S^{n-1}} (1+t)^{2p-2r} t^{n-1} d\sigma_{n-1}(u) dt\\
&=&\frac{2J}{\Gamma(n/2)} \int_{0}^\infty (1+t)^{2p-2r} t^{n-1} dt.
\end{eqnarray*}
This integral is finite if and only if $2p-2r+n-1<-1$, and we have assumed that $r>p+\frac{n}{2}$.
Therefore,
\[
 \int_{\mathbb{R}^n} (1+|y|)^p |\widehat{F}(y)| dm_n(y)< \infty,
\]
from which we get that $y^\alpha \widehat{F}(y)$ is in $L^1(\mathbb{R}^n)$ for $|\alpha| \leq p$.

Define 
\[
F_\omega(x) = \int_{\mathbb{R}^n} \widehat{F} e_x dm_n, \qquad x \in \mathbb{R}^n.
\]
(Note that $F$ depends on $\omega$.)
$F,\widehat{F} \in L^1(\mathbb{R}^n)$ so by the inversion theorem we have $F(x)=F_\omega(x)$ for almost all $x \in \mathbb{R}^n$. 
$F_\omega \in C_0(\mathbb{R}^n)$. If $p \geq 1$, then we shall show that $F_\omega \in C^p(\Omega)$.
Take $e_k$ to be the standard basis for $\mathbb{R}^n$. For $1 \leq k_1 \leq n$ and $\epsilon \neq 0$,
\begin{eqnarray*}
\frac{F_\omega(x+\epsilon e_{k_1})-F_\omega(x)}{\epsilon}&=&\frac{1}{\epsilon} \int_{\mathbb{R}^n} \widehat{F}(y) \left(\exp(i \epsilon e_{k_1} \cdot y)-1\right)
\exp(ix\cdot y) dm_n(y)\\
&=&\int_{\mathbb{R}^n} iy_{k_1} \widehat{F}(y) \frac{e^{i\epsilon y_{k_1}}-1}{i\epsilon y_k} e_x(y) dm_n(y).
\end{eqnarray*}
But $\left| iy_{k_1} \widehat{F}(y) \frac{e^{i\epsilon y_{k_1}}-1}{i\epsilon y_{k_1}} e_x(y)\right| \leq |y_{k_1} \widehat{F}(y)|$ and $y_{k_1} \widehat{F}(y)$ belongs to $L^1(\mathbb{R}^n)$ (supposing
 $p \geq 1$)
so we can apply the dominated convergence theorem, which gives us
\[
(D_{k_1} F_\omega)(x)  = \lim_{\epsilon \to 0} \frac{F_\omega(x+\epsilon e_{k_1})-F_\omega(x)}{\epsilon} = \int_{\mathbb{R}^n} iy_{k_1} \widehat{F}(y) e_x(y) dm_n(y).
\]
From the above expression, it is apparent that  $D_{k_1}F_\omega$ is continuous. This is true for all $1 \leq k_1 \leq n$, so 
$F_\omega \in C^1(\mathbb{R}^n)$. If $p \geq 2$, then $y_{k_1}y_{k_2} \widehat{F}(y)$ is in $L^1(\mathbb{R}^n)$ for any $1 \leq k_2 \leq n$, and repeating the above
argument we get $F_\omega \in C^2(\mathbb{R}^n)$. In this way, $F_\omega \in C^p(\mathbb{R}^n)$. 

For all $x \in \omega$, $f(x)=F(x)$, so $f(x)=F_\omega(x)$ for almost all $x \in \omega$.
If $\omega'$ is an open subset of $\Omega$ whose closure is a compact
subset of $\Omega$ and $\omega \cap \omega' \neq \emptyset$, then $F_\omega,F_{\omega'} \in C^p(\mathbb{R}^n)$ satisfy
$f(x)=F_\omega(x)$ for almost all $x \in \omega$ and $f(x) = F_{\omega'}(x)$ for almost
all $x \in \omega'$, so $F_\omega(x)=F_{\omega'}(x)$ for almost all $x \in \omega \cap
\omega'$. Since $F_\omega,F_{\omega'}$ are continuous, this implies that $F_\omega(x)=
F_{\omega'}(x)$ for all $x \in \omega \cap \omega'$. Thus, it makes sense to define $f_0(x)=F_\omega(x)$ for $x \in \omega$. 
Because every point in $\Omega$ has an open neighborhood of the kind $\omega$ and the restriction of $f_0$ to each $\omega$ belongs to
$C^p(\omega)$, it follows that $f_0 \in C^p(\Omega)$. 
\end{proof}










\end{document}
