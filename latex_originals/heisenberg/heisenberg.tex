\documentclass{article}
\usepackage{amsmath,amssymb,mathrsfs,amsthm}
%\usepackage{tikz-cd}
\usepackage{hyperref}
\newcommand{\inner}[2]{\left\langle #1, #2 \right\rangle}
\newcommand{\tr}{\ensuremath\mathrm{tr}\,} 
\newcommand{\Span}{\ensuremath\mathrm{span}} 
\def\Re{\ensuremath{\mathrm{Re}}\,}
\def\Im{\ensuremath{\mathrm{Im}}\,}
\newcommand{\id}{\ensuremath\mathrm{id}} 
\newcommand{\var}{\ensuremath\mathrm{var}} 
\newcommand{\Lip}{\ensuremath\mathrm{Lip}} 
\newcommand{\GL}{\ensuremath\mathrm{GL}} 
\newcommand{\diam}{\ensuremath\mathrm{diam}} 
\newcommand{\sgn}{\ensuremath\mathrm{sgn}\,} 
\newcommand{\lcm}{\ensuremath\mathrm{lcm}} 
\newcommand{\supp}{\ensuremath\mathrm{supp}\,}
\newcommand{\dom}{\ensuremath\mathrm{dom}\,}
\newcommand{\upto}{\nearrow}
\newcommand{\downto}{\searrow}
\newcommand{\norm}[1]{\left\Vert #1 \right\Vert}
\newcommand{\HS}[1]{\left\Vert #1 \right\Vert_{\mathrm{HS}}}
\newtheorem{theorem}{Theorem}
\newtheorem{lemma}[theorem]{Lemma}
\newtheorem{proposition}[theorem]{Proposition}
\newtheorem{corollary}[theorem]{Corollary}
\theoremstyle{definition}
\newtheorem{definition}[theorem]{Definition}
\newtheorem{example}[theorem]{Example}
\begin{document}
\title{The Heisenberg group and Hermite functions}
\author{Jordan Bell\\ \texttt{jordan.bell@gmail.com}\\Department of Mathematics, University of Toronto}
\date{\today}

\maketitle


\section{The Heisenberg group}
For $(z,t),(w,s) \in \mathbb{C}^n \times \mathbb{R}$, define the operation
\[
(z,t)(w,s) = \left(z+w,t+s+\frac{1}{2} \Im(z \cdot \overline{w})\right),
\]
which satisfies
\[
(z,t)(0,0) = (z,t),
\]
and because $\Im(z \cdot \overline{z})=0$,
\[
(z,t)^{-1}=(-z,-t).
\]
We denote $\mathbb{C}^n \times\mathbb{R}$ with this operation by $H^n$. This is a Lie group of dimension $2n+1$, 
called the \textbf{Heisenberg group}.

Writing $z=x+iy$ define
\[
X_j = \frac{\partial}{\partial x_j} - \frac{1}{2}y_j \frac{\partial}{\partial t},\qquad 1 \leq j \leq n,
\]
and
\[
Y_j = \frac{\partial}{\partial y_j} + \frac{1}{2}x_j \frac{\partial}{\partial t},\qquad 1 \leq j \leq n,
\]
and
\[
T = \frac{\partial}{\partial t}.
\]

We  calculate the Lie brackets of these vector fields. For $X_j$ and $X_k$,
\begin{align*}
X_j X_k &= \left(\frac{\partial}{\partial x_j} - \frac{1}{2}y_j \frac{\partial}{\partial t}\right)
\left( \frac{\partial}{\partial x_k} - \frac{1}{2}y_k \frac{\partial}{\partial t}\right)\\
&=\frac{\partial^2}{\partial x_j \partial x_k} - \frac{1}{2} y_k \frac{\partial}{\partial x_j} \frac{\partial}{\partial t}
-\frac{1}{2}y_j \frac{\partial}{\partial t} \frac{\partial}{\partial x_k} + \frac{1}{4} y_j y_k \frac{\partial^2}{\partial t^2},
\end{align*}
yielding
\[
[X_j,X_k] = X_jX_k-X_kX_j = 0.
\]
For $Y_j$ and $Y_k$,
\begin{align*}
Y_j Y_k&=\left( \frac{\partial}{\partial y_j} + \frac{1}{2}x_j \frac{\partial}{\partial t}\right)
\left( \frac{\partial}{\partial y_k} + \frac{1}{2}x_k \frac{\partial}{\partial t}\right)\\
&=\frac{\partial^2}{\partial y_j \partial y_k} +\frac{1}{2} x_k \frac{\partial}{\partial y_j} \frac{\partial}{\partial t}
+\frac{1}{2}x_j \frac{\partial}{\partial t} \frac{\partial}{\partial y_k}
+\frac{1}{4} x_jx_k \frac{\partial^2}{\partial t^2},
\end{align*}
yielding
\[
[Y_j,Y_k] = Y_jY_k-Y_kY_j=0.
\]
For $X_j$ and $Y_j$,
\begin{align*}
X_j Y_j&=\left( \frac{\partial}{\partial x_j} - \frac{1}{2}y_j \frac{\partial}{\partial t}\right)
\left(\frac{\partial}{\partial y_j} + \frac{1}{2}x_j \frac{\partial}{\partial t}\right)\\
&=\frac{\partial^2}{\partial x_j \partial y_j} +\frac{1}{2}\frac{\partial}{\partial t}
+\frac{1}{2} x_j \frac{\partial}{\partial x_j} \frac{\partial}{\partial t}
-\frac{1}{2}y_j \frac{\partial}{\partial t} \frac{\partial}{\partial y_j}
-\frac{1}{4}y_j x_j \frac{\partial^2}{\partial t^2},
\end{align*}
and
\begin{align*}
Y_jX_j&=\left(\frac{\partial}{\partial y_j} + \frac{1}{2}x_j \frac{\partial}{\partial t}\right)\left( \frac{\partial}{\partial x_j} - \frac{1}{2}y_j \frac{\partial}{\partial t}\right)\\
&=\frac{\partial^2}{\partial y_j \partial x_j} - \frac{1}{2}\frac{\partial}{\partial t}
-\frac{1}{2}y_j \frac{\partial}{\partial y_j} \frac{\partial}{\partial t}
+\frac{1}{2} x_j \frac{\partial}{\partial t} \frac{\partial}{\partial x_j}
-\frac{1}{4}x_jy_j \frac{\partial^2}{\partial t^2},
\end{align*}
yielding
\[
[X_j,Y_j]=X_jY_j-Y_jX_j = \frac{\partial}{\partial t} =T.
\]
For $X_j$ and $Y_k$ with $j \neq k$,
\begin{align*}
X_jY_k&=\left( \frac{\partial}{\partial x_j} - \frac{1}{2}y_j \frac{\partial}{\partial t}\right)
\left( \frac{\partial}{\partial y_k} + \frac{1}{2}x_k \frac{\partial}{\partial t}\right)\\
&=\frac{\partial^2}{\partial x_j \partial y_k} +\frac{1}{2}x_k \frac{\partial}{\partial x_j} \frac{\partial}{\partial t}
-\frac{1}{2}y_j \frac{\partial}{\partial t}  \frac{\partial}{\partial y_k} 
-\frac{1}{4} y_j x_k \frac{\partial^2}{\partial t^2}
\end{align*}
and
\begin{align*}
Y_k X_j&=\left( \frac{\partial}{\partial y_k} + \frac{1}{2}x_k \frac{\partial}{\partial t}\right)\left( \frac{\partial}{\partial x_j} - \frac{1}{2}y_j \frac{\partial}{\partial t}\right)\\
&=\frac{\partial^2}{\partial y_k \partial x_j} - \frac{1}{2} y_j \frac{\partial}{\partial y_k} \frac{\partial}{\partial t}
+\frac{1}{2}x_k \frac{\partial}{\partial t} \frac{\partial}{\partial x_j}
-\frac{1}{4} x_k y_j \frac{\partial^2}{\partial t^2},
\end{align*}
yielding
\[
[X_j,Y_k]=0.
\]
For $X_j$ and $T$,
\[
X_j T = \left(\frac{\partial}{\partial x_j} - \frac{1}{2}y_j \frac{\partial}{\partial t}\right)
\frac{\partial}{\partial t}
=\frac{\partial}{\partial x_j} \frac{\partial}{\partial t} - \frac{1}{2} y_j \frac{\partial^2}{\partial t^2}
=TX_j,
\]
yielding
\[
[X_j,T] = 0.
\]
For $Y_j$ and $T$, 
\[
Y_j T = \left(\frac{\partial}{\partial y_j} + \frac{1}{2}x_j \frac{\partial}{\partial t}\right) \frac{\partial}{\partial t}
=\frac{\partial}{\partial y_j} \frac{\partial}{\partial t} + \frac{1}{2} x_j \frac{\partial^2}{\partial t^2}=TY_j,
\]
yielding
\[
[Y_j,T]=0.
\]
We summarize the above calculations in the following theorem.

\begin{theorem}
The Lie brackets of the vector fields $X_j,Y_j$, $1 \leq j \leq n$, and $T$ are:
\begin{itemize}
\item $[X_j,X_k]=0$
\item $[Y_j,Y_k]=0$
\item $[X_j,Y_j]=T$
\item $[X_j,Y_k]=0$ for $j \neq k$
\item $[X_j,T]=0$
\item $[Y_j,T]=0$
\end{itemize}
\end{theorem}

The Lie algebra of the $H^n$ is called the \textbf{Heisenberg Lie algebra} and is denoted $\mathfrak{h}^n$.
The above vector fields are left-invariant and are a basis for $\mathfrak{h}^n$.\footnote{Sundaram Thangavelu,
{\em An Introduction to the Uncertainty Principle: Hardy's Theorem on Lie Groups}, p.~47, \S 2.1.}


\section{Representation theory}
For a Hilbert space $H$, we denote by $\mathscr{B}(H)$ the set of 
bounded linear operators $H \to H$, which  is a Banach algebra with the operator norm. We denote by $\mathscr{B}_0(H)$ the set
of
compact operators $H \to H$, which is a closed ideal of the Banach algebra $\mathscr{B}(H)$. 
We denote by $\mathscr{B}_{\mathrm{HS}}(H)$ the collection of Hilbert-Schmidt operators $H \to H$: if $\{e_i: i \in I\}$ is an orthonormal basis
of $H$, a linear map $A:H \to H$ is called a  \textbf{Hilbert-Schmidt operator} if
\[
\HS{A}^2 = \sum_{i \in I} \norm{Ae_i}^2 < \infty.
\]
This satisfies $\norm{A} \leq \HS{A}$.\footnote{\url{http://individual.utoronto.ca/jordanbell/notes/traceclass.pdf}, Theorem 16.}
A Hilbert-Schmidt operator is a compact operator.
A linear map $U:H \to H$ is called a \textbf{unitary operator} if it is a bijection and satisfies
\[
\inner{Ux}{Uy} = {x}{y},\qquad x,y \in H.
\]
We denote the
set of unitary operators $H \to H$ by $\mathscr{U}(H)$. 




For $\lambda \in \mathbb{R}, \lambda \neq 0$, for $(x+iy,t) \in H^n$, and for $f \in L^2(\mathbb{R}^n)$, define
\[
\pi_\lambda(x+iy,t)f(\xi) = e^{i\lambda t} e^{i\lambda\left(x\cdot \xi+\frac{1}{2}x\cdot y\right)} f(\xi+y),
\qquad \xi \in \mathbb{R}^n.
\]
It is apparent that $\pi_\lambda(z,t)$ is a linear map $L^2(\mathbb{R}^n) \to L^2(\mathbb{R}^n)$. 


For $(x+iy,t),(u+iv,s) \in H^n$ we calculate
\begin{align*}
\pi_\lambda (x+iy,t) \pi_\lambda(u+iv,s) f(\xi)&=\pi_\lambda(x+iy,t) e^{i\lambda s} e^{i\lambda\left(u\cdot \xi+\frac{1}{2}u \cdot v\right)}
f(\xi+v)\\
&=e^{i\lambda t} e^{i\lambda\left(x\cdot \xi+\frac{1}{2}x\cdot y\right)}
e^{i\lambda s} e^{i\lambda\left(u\cdot (\xi+y)+\frac{1}{2}u \cdot v\right)} f(\xi+y+v)\\
&=e^{i\lambda(t+s)} e^{i\lambda\left((x+u)\cdot \xi + \frac{1}{2}x\cdot y+u\cdot y+\frac{1}{2}u\cdot v\right)}
f(\xi+y+v).
\end{align*}
On the other hand, with $z=x+iy$ and $w=u+iv$,
\begin{align*}
(z,t)(w,s)&=\left(z+w,t+s+\frac{1}{2} \Im(z \cdot \overline{w})\right)\\
&=\left(x+iy+u+iv,t+s+\frac{1}{2}\Im((x+iy)\cdot (u-iv))\right)\\
&=\left(x+u+i(y+v),t+s+\frac{1}{2}\Im(x\cdot u - ix\cdot v + iy \cdot u + y \cdot v)\right)\\
&=\left(x+u+i(y+v),t+s-\frac{1}{2}x\cdot v + \frac{1}{2} y\cdot u \right),
\end{align*}
for which
\begin{align*}
\pi_\lambda((z,t)(w,s))f(\xi)&=e^{i\lambda\left(t+s-\frac{1}{2}x\cdot v + \frac{1}{2} y\cdot u\right)}
e^{i\lambda\left((x+u)\cdot \xi+\frac{1}{2}(x+u)\cdot (y+v) \right)} f(\xi+y+v)\\
&=e^{i\lambda(t+s)} e^{i\lambda\left((x+u)\cdot \xi+\frac{1}{2}x\cdot y+y \cdot u
+\frac{1}{2}u\cdot v\right)} f(\xi+y+v),
\end{align*}
and therefore
\[
\pi_\lambda (x+iy,t) \pi_\lambda(u+iv,s)  = \pi_\lambda((z,t)(w,s)).
\]

We calculate
\[
\pi_\lambda(0,0) f(\xi) = f(\xi)
\]
and
\[
 \pi_\lambda(x+iy,t) \pi_\lambda((x+iy,t)^{-1}) f = 
\pi_\lambda(0,0)f = f.
\]
For $f,g \in L^2(\mathbb{R}^n)$,
\[
\begin{split}
&\inner{\pi_\lambda(x+iy,t)f}{\pi_\lambda(x+iy,t)g}\\
&\int_{\mathbb{R}^n} \pi_\lambda(x+iy,t)f(\xi) \overline{\pi_\lambda(x+iy,t)g(\xi)} d\xi\\
=&\int_{\mathbb{R}^n} 
e^{i\lambda t} e^{i\lambda\left(x\cdot \xi+\frac{1}{2}x\cdot y\right)} f(\xi+y)
e^{-i\lambda t} e^{-i\lambda\left(x\cdot \xi+\frac{1}{2}x\cdot y\right)} \overline{g(\xi+y)}
d\xi\\
=&\int_{\mathbb{R}^n} f(\xi+y) \overline{g(\xi+y)} d\xi\\
=&\inner{f}{g}.
\end{split}
\]
Therefore $\pi_\lambda(z,t)$ is a unitary operator $L^2(\mathbb{R}^n) \to L^2(\mathbb{R}^n)$, and
\[
\pi_\lambda:H^n \to \mathscr{U}(L^2(\mathbb{R}^n))
\]
 is a group homomorphism, namely, $\pi_\lambda$ is a unitary
representation of $H^n$ on $L^2(\mathbb{R}^n)$.\footnote{cf. \url{https://www.math.ubc.ca/~cass/research/pdf/Unitary.pdf}}
Furthermore, using that $y \mapsto f(\cdot+y)$ is continuous $\mathbb{R}^n \to L^2(\mathbb{R}^n)$, 
\begin{align*}
\norm{\pi_\lambda(x+iy,t) f -f}^2
&=\int_{\mathbb{R}^n} | e^{i\lambda t} e^{i\lambda\left(x\cdot \xi+\frac{1}{2}x\cdot y\right)} f(\xi+y) - f(\xi)|^2 d\xi \to 0
\end{align*}
as $(z,t) \to 0$, showing that $\pi_\lambda:H^n \to \mathscr{U}(L^2(\mathbb{R}^n))$ is \textbf{strongly continuous}. (That is, it is continuous when
$\mathscr{U}(L^2(\mathbb{R}^n))$ is assigned the strong
operator topology.)

\begin{theorem}
For $\lambda \in \mathbb{R}$, $\lambda \neq 0$, the map $\pi_\lambda$ defined by
\[
\pi_\lambda(x+iy,t)f(\xi) = e^{i\lambda t} e^{i\lambda\left(x\cdot \xi+\frac{1}{2}x\cdot y\right)} f(\xi+y),
\]
for $(x+iy,t) \in H^n$, $f \in L^2(\mathbb{R}^n)$, and $\xi \in \mathbb{R}^n$, is  a strongly continuous unitary representation of $H^n$
on $L^2(\mathbb{R}^n)$. 
\end{theorem}


We call $\pi_1$ the \textbf{Schr\"odinger representation}. Its kernel is
 \[
 \Gamma=\{(0,2\pi k): k \in \mathbb{Z}\}.
 \]
 For $f \in L^1(H^n/\Gamma)$ we define
 \[
 \pi_1(f) = \int_{H^n/\Gamma} f(z,t) \pi_1(z,t) dz dt.
 \]
For $f,g \in L^1(H^n/\Gamma)$,
\[
(f*g)(z,t) = \int_{H^n/\Gamma} f((z,t) \cdot (w,s)^{-1}) g(w,s) dw ds, \qquad (z,t) \in H^n/\Gamma.
\] 
It is a fact that Lebesgue measure on $\mathbb{C}^n \times \mathbb{R}$ is a 
bi-invariant Haar measure on $H^n$, and using this we calculate
\[
\begin{split}
&\pi_1(f*g)\\
=&\int_{H^n/\Gamma} \left(  \int_{H^n/\Gamma} f((z,t) \cdot (w,s)^{-1}) g(w,s) dw ds\right) \pi_1(z,t) dz dt\\
=&\int_{H^n/\Gamma} g(w,s) \left( \int_{H^n/\Gamma}  f((z,t) \cdot (w,s)^{-1}) \pi_1((z,t)\cdot (w,s)^{-1})
  dz dt \right) \pi_1(w,s) dw ds\\
=&\int_{H^n/\Gamma} g(w,s) \pi_1(f) dw ds\\
=&\pi_1(f) \pi_1(g).
\end{split}
\]

\begin{lemma}
For $f,g \in L^1(H^n/\Gamma)$,
\[
\pi_1(f*g) = \pi_1(f) \pi_1(g).
\]
\end{lemma}
 
 We define
 \[
 W(z) = \pi_1(z,0),
 \]
with which
\[
\pi_1(z,t) = e^{it} W(z).
\]
Define
\[
f_1(z) = (2\pi)^{-1/2} \int_0^{2\pi} f(z,t) e^{it} dt.
\]
Then
\begin{align*}
\pi_1(f) &= \int_{H^n/\Gamma} f(z,t) e^{it} W(z) dz dt\\
&=
\int_{\mathbb{C}^n} W(z) \left( \int_0^{2\pi} f(z,t) e^{it} dt \right) dz\\
&=(2\pi)^{1/2} \int_{\mathbb{C}^n}  f_1(z) W(z) dz.
\end{align*}


For $f \in L^1(\mathbb{C}^n)$, define
\[
f^{\#}(z,t) = (2\pi)^{-1} e^{-it} f(z).
\]
$f^{\#} \in  L^1(H^n/\Gamma)$, and
\[
f^{\#}_1(z) = (2\pi)^{-1/2} \int_0^{2\pi} f^{\#}(z,t) e^{it} dt = 
(2\pi)^{-1/2} f(z),
\]
thus
\[
\pi_1(f^{\#}) =(2\pi)^{1/2} \int_{\mathbb{C}^n} f^{\#}(z) W(z) dz
=\int_{\mathbb{C}^n} f(z) W(z) dz.
\]
We define $W:L^1(\mathbb{C}^n) \to \mathscr{U}(L^2(\mathbb{R}^n))$ by
\[
W(f) = \pi_1(f^{\#}),
\]
called the \textbf{Weyl transform}.



For $f,g \in L^1(\mathbb{C}^n)$ and for $(z,t) \in H^n/\Gamma$,
\[
\begin{split}
&(f^{\#} * g^{\#})(z,t)\\
=&\int_{H^n/\Gamma} f^{\#}((z,t)\cdot (w,s)^{-1}) g^{\#}(w,s) dw ds\\
=&\int_{H^n/\Gamma} f^{\#}((z,t)\cdot (-w,-s ) ) g^{\#}(w,s) dw ds\\
=&\int_{H^n/\Gamma} f^{\#}\left(z-w,t-s-\frac{1}{2}\Im(z\cdot \overline{w}) \right)
g^{\#}(w,s) dw ds\\
=&\int_{H^n/\Gamma} (2\pi)^{-2} e^{-i\left(t-s-\frac{1}{2}\Im(z\cdot \overline{w})\right)} 
f(z-w) e^{-is} g(w) dw ds\\
=&(2\pi)^{-1} e^{-it} \int_{\mathbb{C}^n} f(z-w) g(w) e^{\frac{i}{2} \Im(z \cdot \overline{w})}
dw\\
=&(f \times g)^{\#}(z,t),
\end{split}
\]
for
\[
(f \times g)(z) =  \int_{\mathbb{C}^n} f(z-w) g(w) e^{\frac{i}{2} \Im(z \cdot \overline{w})}
dw,
\]
called  the \textbf{twisted convolution}. 
Using what we have established so far gives the following.

\begin{lemma}
For $f,g \in L^1(\mathbb{C}^n)$,
\[
W(f \times g)=\pi_1((f \times g)^{\#})=
\pi_1(f^{\#} * g^{\#})=
\pi_1(f^{\#}) \pi_1(g^{\#})
=W(f) W(g)
\]
\end{lemma}


For $\phi \in L^1(\mathbb{C}^n)$, we define
\[
K_\phi(\xi,\eta) = \int_{\mathbb{R}^n} \phi(x+i(\eta-\xi)) e^{\frac{i}{2}(\xi+\eta)\cdot x} dx,\qquad
(\xi,\eta) \in \mathbb{R}^n \times \mathbb{R}^n,
\]
which satisfies, for $f \in L^2(\mathbb{R}^n)$ and $\xi \in \mathbb{R}^n$,
\begin{align*}
W(\phi) f(\xi)&=\int_{\mathbb{C}^n} \phi(z) W(z) f(\xi) dz\\
&=\int_{\mathbb{R}^n} \int_{\mathbb{R}^n} \phi(x+iy) e^{i\left(x\cdot \xi + \frac{1}{2}x\cdot y\right)}  f(\xi+y) dy dx\\
&=\int_{\mathbb{R}^n} \int_{\mathbb{R}^n} \phi(x+i(y-\xi)) e^{\frac{i}{2}(x\cdot \xi +x\cdot y)} f(y) dy dx\\
&=\int_{\mathbb{R}^n} \left(\int_{\mathbb{R}^n} \phi(x+i(y-\xi)) e^{\frac{i}{2}(\xi+y)\cdot x)} dx \right) f(y) dy\\
&=\int_{\mathbb{R}^n} K_\phi(\xi,y)f(y) dy.
\end{align*}
Thus $K_\phi$ is an integral kernel for  the operator $W(\phi)$. 


We show in the following theorem that the Weyl transform sends elements of $L^1(\mathbb{C}^n)$ to  compact operators on
$L^2(\mathbb{R}^n)$, and that 
it sends square integrable functions to Hilbert-Schmidt operators.\footnote{Sundaram Thangavelu, {\em Lectures on Hermite and Laguerre Expansions},
p.~13, Theorem 1.2.1.}

\begin{theorem}
$W:L^1(\mathbb{C}^n) \to \mathscr{B}_0(L^2(\mathbb{R}^n))$, and for $\phi \in L^1(\mathbb{C}^n) \cap L^2(\mathbb{C}^n)$ we have
$W(\phi) \in \mathscr{B}_{\mathrm{HS}}(L^2(\mathbb{R}^n))$ and
\[
\norm{\phi}_{L^2(\mathbb{R}^n)} = (2\pi)^{-n/2} \norm{W(\phi)}_{\mathrm{HS}}.
\]
\end{theorem}
\begin{proof}
First take $\phi \in L^1(\mathbb{C}^n) \cap L^2(\mathbb{C}^n)$. It follows from this that $K_\phi \in L^2(\mathbb{R}^n \times
\mathbb{R}^n)$, and because $K_\phi$ is the integral kernel of $W(\phi)$ this implies\footnote{Michael Reed and Barry Simon, {\em Methods
of Modern Mathematical Physics, volume I: Functional Analysis}, revised and enlarged edition,
p.~210, Theorem VI.23.} that $W(\phi) \in \mathscr{B}_{\mathrm{HS}}(L^2(\mathbb{R}^n))$ and
\[
\HS{W(\phi)}^2 = \int_{\mathbb{R}^n \times \mathbb{R}^n} |K(\xi,\eta)|^2 d\xi d\eta.
\]
\end{proof}






\section{Hermite functions}
For $\phi \in \mathscr{S}(\mathbb{R}^n)$, define
\[
\widehat{\phi}(\xi) =(\mathscr{F} \phi)(\xi) =   (2\pi)^{-n/2} \int_{\mathbb{R}^n} \phi(x) e^{-ix\cdot \xi} dx,\qquad \xi \in \mathbb{R}^n.
\]
$\mathscr{S}(\mathbb{R}^n)$ is a dense linear subspace of $L^2(\mathbb{R}^n)$, and the Fourier transform
 extends  to a unique Hilbert space isomorphism $L^2(\mathbb{R}^n) \to L^2(\mathbb{R}^n)$.
 For $f,g \in L^2(\mathbb{R})$,
 \[
 \inner{f}{g} = \int_{\mathbb{R}^n} f(x) \overline{g(x)} dx.
 \]

For $\phi \in \mathscr{S}(\mathbb{R})$, let
\[
(D\phi)(x) = \phi'(x),\qquad (M\phi)(x)=x\phi(x),\qquad x \in \mathbb{R},
\]
and let
\[
A = -D+M,\qquad B=D+M.
\]
Let
\[
H  = \sum_{j=1}^n (-D_j^2+M_j^2) = \frac{1}{2} \sum_{j=1}^n (A_j B_j + B_j A_j),
\]
which satisfies
\[
(H \phi)(x) = -(\Delta \phi)(x) +|x|^2 \phi(x),
\]
called the \textbf{Hermite operator}.


For $k \geq 0$, define
\[
H_k(x)=(-1)^k e^{x^2} D^k e^{-x^2}
\]
and
\[
h_k(x) = (2^k k! \sqrt{\pi})^{-1/2} e^{-x^2/2}  H_k(x).
\]
The Hermite functions are an orthonormal basis for $L^2(\mathbb{R})$.
Let $\mathbb{N}$ be the nonnegative integers, and for $\alpha \in \mathbb{N}^n$ let
\[
\Phi_\alpha = h_{\alpha_1} \otimes \cdots \otimes h_{\alpha_n},
\]
which are an orthonormal basis for $L^2(\mathbb{R}^n)$. 
It is a fact that
\[
A_j \Phi_\alpha = (2\alpha_j + 2)^{1/2} \Phi_{\alpha+e_j},\qquad B_j \Phi_\alpha
=(2\alpha_j)^{1/2} \Phi_{\alpha-e_j}
\]
and
\[
H\Phi_\alpha = (2|\alpha|+n)\Phi_\alpha.
\]


It is a fact that
\[
\widehat{h}_k = (-i)^k h_k,
\]
whence
\[
\widehat{\Phi}_\alpha = (-i)^{|\alpha|} \Phi_\alpha.
\]
Because $\{\Phi_\alpha: \alpha \in \mathbb{N}^n\}$ is an orthonormal basis for 
$L^2(\mathbb{R}^n)$, for $f \in L^2(\mathbb{R}^n)$, 
\[
f = \sum_{\alpha} \inner{f}{\Phi_\alpha} \Phi_\alpha.
\]
and 
then
\[
\widehat{f} = \sum_{\alpha} \inner{f}{\Phi_\alpha} (-i)^{|\alpha|} \Phi_\alpha.
\]


Let $E_k$ be the linear span of $\{\Phi_\alpha: |\alpha| = k\}$, which has dimension
$\binom{k+n-1}{k}$. 
For $f \in E_k$, $Hf = (2k+n)f$. 
Let $P_k:L^2(\mathbb{R}^n) \to E_k$ be the projection:
\[
P_k f = \sum_{|\alpha|=k} \inner{f}{\Phi_\alpha} \Phi_\alpha,
\qquad f \in L^2(\mathbb{R}^n).
\]
Let
\[
\Phi_k(x,y) = \sum_{|\alpha|=k} \Phi_\alpha(x) \Phi_\alpha(y),
\qquad x,y \in \mathbb{R}^n.
\]
For $x \in \mathbb{R}^n$ we calculate
\begin{align*}
\int_{\mathbb{R}^n} \Phi_k(x,y) f(y) dy&=
\sum_{|\alpha|=k}\Phi_\alpha(y) \int_{\mathbb{R}^n} f(y) \Phi_\alpha(y) dy\\
&=\sum_{|\alpha|=k} \Phi_\alpha(y) \inner{f}{\Phi_\alpha}\\
&=(P_k f)(y),
\end{align*}
thus $\Phi_k$ is a kernel for the projection operator $P_k$. 

Using the $1$-dimensional Mehler's formula  we obtain the $n$-dimensional
Mehler's formula:\footnote{\url{http://individual.utoronto.ca/jordanbell/notes/hermitefunctions.pdf}}
\[
\sum_{\alpha} r^{|\alpha|} \Phi_\alpha(x) \Phi_\alpha(y) = \pi^{-\frac{n}{2}} (1-r^2)^{-\frac{n}{2}} 
\exp\left(-\frac{1}{2} \frac{1+r^2}{1-r^2} (|x|^2+|y|^2)+\frac{2r}{1-r^2} x\cdot y\right).
\]



\section{Special Hermite functions}
We first define the \textbf{Fourier-Wigner transform}. For $f,g \in L^2(\mathbb{R}^n)$ and $z=x+iy \in \mathbb{C}^n$,
\[
V(f,g)(z) = (2\pi)^{-n/2} \int_{\mathbb{R}^n} e^{ix\cdot \xi} f\left(\xi+\frac{1}{2}y\right) \overline{g\left(\xi-\frac{1}{2}y\right)} d\xi.
\]
The following theorem relates the inner product on $L^2(\mathbb{R}^n)$ and the inner product on $L^2(\mathbb{C}^n)$.\footnote{Sundaram Thangavelu, {\em Lectures on Hermite and Laguerre Expansions},
p.~14, Proposition 1.3.1.}

\begin{theorem}
For $f,g,\phi,\psi \in L^2(\mathbb{R}^n)$,
\[
\int_{\mathbb{C}^n} V(f,g)(z) \overline{V(\phi,\psi)(z)} dz = \inner{f}{\phi} \inner{\psi}{g}.
\]
\end{theorem}

We now define the \textbf{special Hermite functions} on $\mathbb{C}^n$. For $\alpha,\beta \in \mathbb{N}^n$, let
\[
\Phi_{\alpha \beta}(z) = V(\Phi_\alpha,\Phi_\beta)(z).
\]
We calculate
\begin{align*}
\inner{W(z)\Phi_\alpha}{\Phi_\beta}&=\int_{\mathbb{R}^n} W(z)\Phi_\alpha(\xi) \Phi_\beta(\xi) d\xi\\
&=\int_{\mathbb{R}^n} e^{i\left(x\cdot \xi+\frac{1}{2}x\cdot y\right)} \Phi_\alpha(\xi+y) \Phi_\beta(\xi) d\xi\\
&=\int_{\mathbb{R}^n} e^{ix\cdot \xi} \Phi_\alpha\left(\xi+\frac{1}{2}y\right) \Phi_\beta\left(\xi-\frac{1}{2}y\right) d\xi\\
&=(2\pi)^{n/2} V(\Phi_\alpha,\Phi_\beta).
\end{align*}

\begin{lemma}
For $\alpha,\beta \in \mathbb{N}^n$ and $z \in \mathbb{C}^n$,
\[
\Phi_{\alpha\beta}(z) = (2\pi)^{-n/2} \inner{W(z)\Phi_\alpha}{\Phi_\beta}.
\]
\end{lemma}

Using that the Hermite functions $\Phi_\alpha$ are an orthonormal basis for $L^2(\mathbb{R}^n)$, it is proved that the special Hermite functions
$\Phi_{\alpha\beta}$ are an orthonormal basis for $L^2(\mathbb{C}^n)$.\footnote{Sundaram Thangavelu, {\em Lectures on Hermite and Laguerre Expansions},
p.~16, Theorem 1.3.2.}

\end{document}