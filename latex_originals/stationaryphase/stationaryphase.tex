\documentclass{article}
\usepackage{amsmath,amssymb,mathrsfs,amsthm}
%\usepackage{tikz-cd}
\usepackage{hyperref}
\newcommand{\inner}[2]{\left\langle #1, #2 \right\rangle}
\newcommand{\tr}{\ensuremath\mathrm{tr}\,} 
\newcommand{\Span}{\ensuremath\mathrm{span}} 
\def\Re{\ensuremath{\mathrm{Re}}\,}
\def\Im{\ensuremath{\mathrm{Im}}\,}
\newcommand{\id}{\ensuremath\mathrm{id}} 
\newcommand{\ca}{\ensuremath\mathrm{ca}} 
\newcommand{\var}{\ensuremath\mathrm{var}} 
\newcommand{\ind}{\ensuremath\mathrm{ind}} 
\newcommand{\Lip}{\ensuremath\mathrm{Lip}} 
\newcommand{\GL}{\ensuremath\mathrm{GL}} 
\newcommand{\diam}{\ensuremath\mathrm{diam}} 
\newcommand{\diag}{\ensuremath\mathrm{diag}} 
\newcommand{\grad}{\ensuremath\mathrm{grad}\,}
\newcommand{\Hess}{\ensuremath\mathrm{Hess}\,}
\newcommand{\sgn}{\ensuremath\mathrm{sgn}\,} 
\newcommand{\lcm}{\ensuremath\mathrm{lcm}} 
\newcommand{\supp}{\ensuremath\mathrm{supp}\,}
\newcommand{\dom}{\ensuremath\mathrm{dom}\,}
\newcommand{\upto}{\nearrow}
\newcommand{\downto}{\searrow}
\newcommand{\norm}[1]{\left\Vert #1 \right\Vert}
\newcommand{\HS}[1]{\left\Vert #1 \right\Vert_{\mathrm{HS}}}
\newtheorem{theorem}{Theorem}
\newtheorem{lemma}[theorem]{Lemma}
\newtheorem{proposition}[theorem]{Proposition}
\newtheorem{corollary}[theorem]{Corollary}
\theoremstyle{definition}
\newtheorem{definition}[theorem]{Definition}
\newtheorem{example}[theorem]{Example}
\begin{document}
\title{Stationary phase,  Laplace's method, and the Fourier transform for Gaussian integrals}
\author{Jordan Bell\\ \texttt{jordan.bell@gmail.com}\\Department of Mathematics, University of Toronto}
\date{\today}

\maketitle

\section{Critical points}
Let $U$ be a nonempty open subset of $\mathbb{R}^n$ and let
 $\phi:U \to \mathbb{R}$ be smooth. Then $\phi':U \to \mathscr{L}(\mathbb{R}^n;\mathbb{R})
=(\mathbb{R}^n)^*$. 
For each $x \in U$, $\grad \phi(x)$ is the unique element of $\mathbb{R}^n$ satisfying\footnote{\url{http://individual.utoronto.ca/jordanbell/notes/gradienthilbert.pdf}}
\[
\inner{\grad \phi(x)}{y} = \phi'(x)(y),\qquad y\in \mathbb{R}^n,
\]
and $\grad \phi:U \to \mathbb{R}^n$ is itself smooth. 
$\Hess \phi:U \to \mathscr{L}(\mathbb{R}^n;\mathbb{R}^n)$ is the derivative of $\grad \phi$. One checks that
\[
\phi''(x)(u)(v) = \inner{\Hess \phi(x)(u)}{v},\qquad x \in U,\quad u,v \in \mathbb{R}^n,
\]
and $(\Hess \phi (x))^*=\Hess \phi (x)$. 

We call $p \in U$ a \textbf{critical point of $\phi$} when $\grad \phi(p)=0$, and we denote the
set of critical points of $\phi$ by $C_\phi$. 
For $p \in C_\phi$ and $\lambda \in \mathbb{R}$ let 
$v(p,\lambda)$ denote the dimension of the kernel of
$\Hess \phi(p) - \lambda$, and we then define the \textbf{Morse index of $p$} to be
\[
m_\phi(p) = \sum_{\lambda<0} v(p,\lambda).
\]
In other words, $m_\phi(p)$ is the number of negative eigenvalues of $\Hess \phi(p)$ counted according
to geometric multiplicity. 
We say that $p \in C_\phi$ is
nondegenerate when
$\Hess \phi(p) \in \mathscr{L}(\mathbb{R}^n;\mathbb{R}^n)$ is invertible.

For $A \in \mathscr{L}(\mathbb{R}^n;\mathbb{R}^n)$  self-adjoint and for $\lambda \in \mathbb{R}$,
let
$v(\lambda)$ be the dimension of
the kernel of $A-\lambda$. Let $\nu_+=\sum_{\lambda>0} v(\lambda)$,  let
$\nu_-=\sum_{\lambda<0} v(\lambda)$, and  let $\nu_0=v(0)$. 
Because $A$ is self-adjoint,
$\nu_+ + \nu_- + \nu_0=n$. 
We define
the \textbf{signature of $A$} as  $\sgn(A)=\nu_+-\nu_-$. In other words, 
$\sgn(A)$ is the number of positive eigenvalues of $A$ counted according to geometric
 multiplicity minus the number of negative
eigenvalues of $A$ counted according to geometric multiplicity.\footnote{cf. Sylvester's law of inertia, \url{http://individual.utoronto.ca/jordanbell/notes/principalaxis.pdf}}

We can  connect the notions of Morse index and signature. For
$p \in C_\phi$, write $A=\Hess \phi(p)$. For $p$ to be a nondegenerate critical point means that $A$ is invertible and
 because $\mathbb{R}^n$ is finite-dimensional this is equivalent to
$\nu_0=0$. Then $\nu_+=n-\nu_-$ which yields $\sgn(A) = n-2\nu_-=n-2m_\phi(p)$. 


The \textbf{Morse lemma}\footnote{Serge Lang, {\em Differential and Riemannian Manifolds}, p.~182,
chapter VII, Theorem 5.1.} states that if $0$ is a nondegenerate critical point of $\phi$
then there is an open subset $V$ of $U$ with $0 \in V$ and a 
$C^\infty$-diffeomorphism $\Phi:V \to V$, $\Phi(0)=0$, such that 
\[
\phi(x) = \phi(0)+\frac{1}{2}\inner{\Hess \phi(0)(\Phi(x))}{\Phi(x)},\qquad x \in V.
\]



\section{Stationary phase}
Let $U$ be a nonempty connected open subset of $\mathbb{R}^n$, and
let $a,\phi:U \to \mathbb{R}$ be smooth functions such that $a$ has compact support. 
Suppose
that each $p \in C_\phi \cap \supp a$ is nondegenerate.\footnote{In particular, $\phi$ is called a Morse function if it has
no degenerate critical points, and in this case of course each $p \in C_\phi \cap \supp a$ is nondegenerate.} 
The \textbf{stationary phase approximation} states that
\begin{align*}
\int_U a(x) e^{it\phi(x)} dx &= \sum_{p \in C_\phi \cap \supp a} 
\left( \frac{2\pi}{t}\right)^{n/2} \frac{e^{\frac{i\pi \sgn(\Hess \phi(p))}{4}}}{|\det \Hess \phi(p)|^{1/2}}
e^{it\phi(p)} a(p)\\
&+ O(t^{-\frac{n}{2}-1})
\end{align*}
as $t \to \infty$.\footnote{Liviu Nicolaescu, {\em An Invitation to Morse Theory}, second ed.,
p.~183, Proposition 3.88.}


Let $A \in \mathscr{L}(\mathbb{R}^n;\mathbb{R}^n)$ be self-adjoint and invertible and define
\[
\phi(x) = \frac{1}{2}\inner{Ax}{x},\qquad x \in U.
\]
We calculate $\grad \phi(x) = Ax$,
so $C_\phi = \{0\}$. 
The Hessian of $\phi$ is
 $\Hess \phi(x) = A$, and because $A$ is invertible, $0$ is indeed a nondegenerate critical point of $\phi$. 
Thus we have the following.

\begin{theorem}
For a nonempty connected open subset of $\mathbb{R}^n$ and for smooth functions $a,\phi:U \to \mathbb{R}$
such that $a$ has compact support and such that each $p \in C_\phi$ is nondegenerate,
\[
\int_U a(x) e^{\frac{1}{2}\inner{Ax}{x}} dx = \left( \frac{2\pi}{t}\right)^{n/2} 
\frac{e^{\frac{i\pi\sgn(A)}{4}}}{|\det A|^{1/2}}
e^{\frac{1}{2}it\inner{Ap}{p}} a(p) + O(t^{-\frac{n}{2}-1})
\]
as $t \to \infty$. 
\end{theorem}


\section{The Fourier transform}
For $A \in \mathscr{L}(\mathbb{R}^n;\mathbb{R}^n)$ self-adjoint, 
the spectral theorem tells us that
are $\lambda_1,\ldots,\lambda_n \in \mathbb{R}$ and
an orthonormal basis $\{v_1,\ldots,v_n\}$ for $\mathbb{R}^n$ such that
$A v_j = \lambda_j v_j$.  


We call $A \in \mathscr{L}(\mathbb{R}^n;\mathbb{R}^n)$ \textbf{positive}
when it is self-adjoint and satisfies
 $\inner{Ax}{x} \geq 0$ for all $x \in \mathbb{R}^n$. 
In this case, the eigenvalues of $A$ are nonnegative, thus the signature of
$A$ is $\sigma(A)=n$. 
Suppose furthermore that $A$ is invertible, and
let $P=(v_1,\ldots,v_n)$ and $\Lambda=\diag(\lambda_1,\ldots,\lambda_n)$. 
Then
\[
P^T A P = \Lambda,\qquad \Lambda^{1/2} = \diag(\lambda_1^{1/2},\ldots,\lambda_n^{1/2}),
\qquad A^{1/2} = P\Lambda^{1/2}P^T.
\]
For $\xi \in \mathbb{R}^n$ and $t>0$, using the change of variables formula with the fact that 
$|\det P|=1$ and then using Fubini's theorem,
\[
\begin{split}
&\int_{\mathbb{R}^n} \exp\left(-\frac{1}{2}t \inner{Ax}{x}-i\inner{P\xi}{x}\right) dx\\
=&\int_{\mathbb{R}^n} \exp\left(-\frac{1}{2}t \inner{\Lambda^{1/2}P^Tx}{\Lambda^{1/2}P^Tx} -i \inner{P\xi}{x}\right) dx\\
=&\int_{\mathbb{R}^n} \exp\left(-\frac{1}{2}t \inner{\Lambda^{1/2}P^T Py}{\Lambda^{1/2}P^T Py}-i\inner{P\xi}{Py}\right)  |\det P| dy\\
=&\int_{\mathbb{R}^n} \exp\left(-\frac{1}{2}t
\norm{\Lambda^{1/2} y}^2 -i \inner{\xi}{y} \right) dy\\
=&\prod_{j=1}^n \int_{\mathbb{R}} \exp\left(-\frac{1}{2} t \lambda_j y_j^2 -i \xi_j y_j \right) dy_j.
\end{split}
\]
Using\footnote{\url{http://individual.utoronto.ca/jordanbell/notes/bochnertheorem.pdf}} 
\[
\int_{\mathbb{R}} e^{-ax^2+bx+c} dx = \sqrt{\frac{\pi}{a}} \exp\left(\frac{b^2}{4a}+c\right),
\qquad \Re a>0, b,c \in \mathbb{C},
\]
gives
\[
\int_{\mathbb{R}} \exp\left(-\frac{1}{2} t \lambda_j y_j^2 -i \xi_j y_j \right) dy_j
=\frac{1}{\lambda_j^{1/2}} \left( \frac{2\pi}{t}\right)^{1/2} \exp\left(-\frac{\xi_j^2}{2t\lambda_j}\right),
\]
and using $\det A = \prod_{j=1}^n \lambda_j$ 
we have
\[
\begin{split}
&\int_{\mathbb{R}^n} \exp\left(-\frac{1}{2}t \inner{Ax}{x}-i\inner{P\xi}{x}\right) dx\\
=&\prod_{j=1}^n\frac{1}{\lambda_j^{1/2}} \left( \frac{2\pi}{t}\right)^{1/2} \exp\left(-\frac{\xi_j^2}{2t\lambda_j}\right)\\
=&(\det A)^{-1/2} \left(\frac{2\pi}{t}\right)^{n/2} \exp\left(-\frac{1}{2t}\sum_{j=1}^n \frac{\xi_j^2}{\lambda_j} \right),
\end{split}
\]
and because
\[
\Lambda^{-1} \xi = \sum_{j=1}^n \frac{\xi_j}{\lambda_j} e_j,\qquad
\inner{\Lambda^{-1} \xi}{\xi}=\sum_{j=1}^n \frac{\xi_j^2}{\lambda_j}
\]
this becomes
\[
\begin{split}
&\int_{\mathbb{R}^n} \exp\left(-\frac{1}{2}t \inner{Ax}{x}-i\inner{P\xi}{x}\right) dx\\
=&(\det A)^{-1/2} \left(\frac{2\pi}{t}\right)^{n/2} \exp\left(-\frac{1}{2t} \inner{\Lambda^{-1}\xi}{\xi} \right)\\
=&(\det A)^{-1/2} \left(\frac{2\pi}{t}\right)^{n/2} \exp\left(-\frac{1}{2t}\inner{A^{-1}P\xi}{P\xi}\right),
\end{split}
\]
and so, as $P$ is invertible we get the following.

\begin{theorem}
When $A \in \mathscr{L}(\mathbb{R}^n;\mathbb{R}^n)$ is positive and invertible,
for $t>0$ and $\xi \in \mathbb{R}^n$   we have
\[
\begin{split}
&\int_{\mathbb{R}^n} \exp\left(-\frac{1}{2}t \inner{Ax}{x}-i\inner{\xi}{x}\right) dx\\
=&(\det A)^{-1/2} \left(2\pi t^{-1}\right)^{n/2} \exp\left(-\frac{1}{2t}\inner{A^{-1}\xi}{\xi}\right).
\end{split}
\]
\end{theorem}


\section{Gaussian integrals}
Let $A \in \mathscr{L}(\mathbb{R}^n;\mathbb{R}^n)$ be positive and invertible and let
$b \in \mathbb{R}^n$. 
As above,
\begin{align*}
\int_{\mathbb{R}^n} \exp\left(-\frac{1}{2}\inner{Ax}{x}+\inner{P b}{x} \right)dx&=\int_{\mathbb{R}^n} \exp\left(-\frac{1}{2}\norm{\Lambda^{1/2}}^2
+\inner{b}{y}\right) dy\\
&=\prod_{j=1}^n \int_{\mathbb{R}} \exp\left(-\frac{1}{2}\lambda_j y_j^2 + b_j y_j\right)
dy_j\\
&=\prod_{j=1}^n \frac{(2\pi)^{1/2}}{\lambda_j^{1/2}} \exp\left(\frac{b_j^2}{2\lambda_j}\right)\\
&=(\det A)^{-1/2} (2\pi)^{n/2} \exp\left(\frac{1}{2} \sum_{j=1}^n \frac{b_j^2}{\lambda_j} \right)\\
&=(\det A)^{-1/2} (2\pi)^{n/2} \exp\left(\frac{1}{2} \inner{A^{-1}Pb}{Pb}\right),
\end{align*}
which gives the following.\footnote{cf. Gaussian measures on $\mathbb{R}^n$:
\url{http://individual.utoronto.ca/jordanbell/notes/gaussian.pdf}}

\begin{theorem}
If $A \in \mathscr{L}(\mathbb{R}^n;\mathbb{R}^n)$ is positive and invertible, then for $b \in \mathbb{R}^n$,
\[
\int_{\mathbb{R}^n} \exp\left(-\frac{1}{2}\inner{Ax}{x}+\inner{b}{x} \right)dx
=(\det A)^{-1/2} (2\pi)^{n/2} \exp\left(\frac{1}{2} \inner{A^{-1}b}{b}\right).
\]
\end{theorem}


\section{Laplace's method}
Let $D$ be the open ball in $\mathbb{R}^n$ with center $0$ and radius $1$ and let
$S:D \to \mathbb{R}$ be smooth, attain its minimum value only at $0$, and satisfy 
$\det \Hess S(x) >0$ for all $x \in D$.
Let $g:D \to \mathbb{R}$ be smooth and for $t>0$ let
\[
J(t) = \int_D e^{-t S(x)} g(x) dx.
\]
\textbf{Laplace's method}\footnote{Peter D. Miller, {\em Applied Asymptotic Analysis},
p.~92, Exercise 3.16 and 
R. Wong, {\em Asymptotic Approximations of Integrals}, p.~495, Theorem 3.} tells us
\[
J(t) = (2\pi t^{-1})^{n/2} (\det \Hess S(0))^{-1/2} e^{-tS(0)} g(0) (1+O(t^{-1}))
\]
as $t \to \infty$. 

Let $A \in \mathscr{L}(\mathbb{R}^n;\mathbb{R}^n)$ be positive and invertible. 
Define $S:D \to \mathbb{R}$ by
\[
S(x) = \frac{1}{2} \inner{Ax}{x}.
\]
Then as above $P^T A P =\Lambda$, with which
$S(x) = \frac{1}{2}\inner{P\Lambda P^Tx}{x}
=\frac{1}{2} \norm{\Lambda^{1/2} P^T x}^2$.  We get the following from  according  Laplace's method.

\begin{theorem}
Let $A \in \mathscr{L}(\mathbb{R}^n;\mathbb{R}^n)$ be positive and invertible and let $g:D \to \mathbb{R}$ be smooth.
Then
\[
J(t) = (2\pi t^{-1})^{n/2} (\det A)^{-1/2} g(0)(1+O(t^{-1})),
\]
as $t \to \infty$. 
\end{theorem}


\end{document}