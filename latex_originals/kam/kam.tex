\documentclass{article}
\usepackage{amssymb,mathrsfs,amsthm,amsmath}
\newtheorem{theorem}{Theorem}
\newtheorem{corollary}[theorem]{Corollary}
\newcommand{\norm}[1]{\Vert #1 \Vert}
\newcommand{\rank}{\operatorname{rank}}
\begin{document}
\title{Notes on the KAM theorem}
\author{Jordan Bell\\ \texttt{jordan.bell@gmail.com}\\Department of Mathematics, University of Toronto}
\date{\today}              

\maketitle

\section{Introduction}
I hope eventually to expand these notes into a standalone presentation of KAM that presents a precise formulation of the theorem and gives detailed proofs of everything. There are few presentations of KAM in the literature that
give a precise formulation of the theorem, and even those that give precise
formulations such as \cite{MR2026200} and \cite{poeschel} glide over
some details. Gallavotti \cite{gallavotti} explains the history of quasi-periodic phenomena in celestial mechanics.

Let $\mathbb{T}^n=\mathbb{R}^n/\mathbb{Z}^n$.

For $x,y \in \mathbb{R}^n$, let $\langle x,y \rangle = \sum_{j=1}^n x_j y_j$.
Let $\norm{x}=\sum_{j=1}^n x_j^2$ and let $\norm{x}_\infty=\max_{1 \leq j \leq n} |x_j|$. For $x,y \in \mathbb{R}^n$, we have $|\langle x,y \rangle|
\leq n \norm{x}_\infty \norm{y}_\infty$.

If $(M,\omega)$ is a symplectic manifold and $H \in C^\infty(M)$, then the
{\em Hamiltonian vector field} with energy function $H$ is the vector
field $X_H$ on $M$ uniquely determined by the condition $\omega_x (X_H(x),v)=(dH)(x)(v)$ for all points $x \in M$ and tangent vectors $v \in T_x M$.

We say that $(q^1,\ldots,q^n,p_1,\ldots,p_n)$ are {\em canonical coordinates}
for $(M,\omega)$ if $\omega=\sum_{j=1}^n dq^j \wedge dp_j$. If
$(q^1,\ldots,q^n,p_1,\ldots,p_n)$ are canonical coordinates for $(M,\omega)$
and $H \in C^\infty(M)$ then
\[
X_H(x)=((\partial_p H)(x),(-\partial_q H)(x))
\]
for all $x \in M$, where
\[
\partial_q H=\Big( \frac{\partial H}{\partial q^1},\ldots,\frac{\partial H}{\partial q^n} \Big), \quad \partial_p H=\Big( \frac{\partial H}{\partial p_1},\ldots,\frac{\partial H}{\partial p_n} \Big).
\]
Let $\phi$ be the flow of $X_H$ on $M$. Then
\[
\frac{d(q^j(\phi_t(x)))}{dt}=\frac{\partial H}{\partial p_j}(\phi_t(x)),
\quad \frac{d(p_j(\phi_t(x)))}{dt}=-\frac{\partial H}{\partial q^j}(\phi_t(x)),
\]
called {\em Hamilton's equations}.

\section{Action-angle coordinates}
Let $(M,\omega)$ be a $2n$-dimensional symplectic manifold. Let $f_1,\ldots,f_n \in C^\infty(M)$. If $\{f_i,f_j\}=0$ for all
$1 \leq i,j \leq n$ (namely the functions are in involution) and if at each point in $M$
the differentials of the functions are linearly
independent in the cotangent space at that point, then we say that
the set of functions is {\em completely integrable}.

We define the {\em momentum map} $F:M \to \mathbb{R}^n$ by
$F=f_1 \times \cdots \times f_n$.

We say that $F$ is {\em locally trivial} at a value $y_0$ in its range if
there is a neighborhood $U$ of $y_0$ such that for all $y\in U$ there is a smooth map $h_y:F^{-1}(U) \to F^{-1}(y_0)$ such that $F \times h_y$ is a diffeomorphism from $F^{-1}(U)$ to $U \times F^{-1}(y_0)$. The
{\em bifurcation set} of $F$ is the set $\Sigma_F$ of $y_0 \in \mathbb{R}^n$ at which
$F$ fails to be locally trivial.

The following theorem is proved in \cite[Theorem 5.2.21]{foundations}.

\begin{theorem}
Let $U \subseteq \mathbb{R}^n$ be open. If $F|F^{-1}(U):F^{-1}(U) \to U$
is a proper map then each of the vector fields $X_{f_i}|F^{-1}(U)$ is complete, $U \subseteq \mathbb{R}^n \setminus \Sigma_F$, and the fibers
of the locally trivial fibration $F|F^{-1}(U)$ are disjoint unions
of manifolds each diffeomorphic with $\mathbb{T}^n$.
\end{theorem}  

Let $\nu \in \mathbb{R}^n$, and define the linear flow $F$ on $\mathbb{R}^n$
by
$F_t(v)=v+t\nu$. Let $\pi:\mathbb{R}^n \to \mathbb{T}^n$ be the projection
map and let $\phi_t:\mathbb{T}^n \to \mathbb{T}^n$ be such that
$\pi \circ F_t = \phi_t \circ \pi$; if $\pi(v_1)=\pi(v_2)$ then
$\pi \circ F_t(v_1)=\pi \circ F_t(v_2)$, so such a map exists,  and is clearly unique.
A flow $\phi$ on $\mathbb{T}^n$ induced by a linear flow on $\mathbb{R}^n$ is called a {\em quasi-periodic} flow.

Say that $\nu \neq \mu$, and let $\phi$ be the flow induced by $\nu$
and $\psi$ be the flow induced by $\mu$. Then for some $i$, $\nu_i \neq
\mu_i$ and for any $t$ such that $t(\nu_i-\mu_i) \not \in \mathbb{Z}$,
$\phi_t(\theta) \neq \psi_t(\theta)$ for any $\theta \in \mathbb{T}^n$. Hence
$\phi \neq \psi$. Thus  
a quasi-periodic flow is induced by
a unique vector $\nu \in \mathbb{R}^n$. We call $\nu$ the {\em frequency vector} of the flow $\phi$.

We say that
$\nu \in \mathbb{R}^n$ is {\em resonant} if there is some $0 \neq k \in \mathbb{Z}^n$ such that $\langle k,\nu \rangle=0$, and we say that it is {\em nonresonant} otherwise.

Let $\phi$ be the quasi-periodic flow on $\mathbb{T}^n$ with frequency vector $\nu \in \mathbb{R}^n$.
It can be shown that each orbit of $\phi$ is dense in $\mathbb{T}^n$ if and only
if $\nu$ is nonresonant. This is proved in \cite[pp. 818--820]{foundations}; that each orbit of $\phi$ is dense in $\mathbb{T}^n$ if $\nu$ is nonresonant is proved in \cite[Theorem 444]{MR2445243}.

Let $H=f_1$; we call this distinguished function the Hamiltonian, and we are
concerned with the flow of the Hamiltonian vector field $X_H$.

The following theorem is proved in \cite[Theorem 5.2.24]{foundations}. 

\begin{theorem}
Let $c$ be in the range of $F$, let $I_c^0$ denote a connected component
of $F^{-1}(c)$, and let $\phi$ be the flow of $X_H$. Then there is a quasiperiodic flow $\psi$ on $\mathbb{T}^n$ and a diffeomorphism $g:\mathbb{T}^n \to I_c^0$ such that $g \circ \psi_t=\phi_t|I_c^0 \circ g$.
\end{theorem}

Let $\mathbb{R}^{2n}=\{q^1,\ldots,q^n,p_1,\ldots,p_n\}$ and let
$\omega=\sum_{j=1}^n dq^j \wedge dp_j$.
Let $J=\begin{bmatrix}0&I\\-I&0\end{bmatrix}$, where $I$ is the $n \times n$
identity matrix. For $u,v \in \mathbb{R}^{2n}$ we have that $\omega(u,v)=\langle u,Jv \rangle$. 

Let $B^n$ be an open ball in $\mathbb{R}^n$. $B^n \times \mathbb{T}^n$
is a symplectic submanifold of $\mathbb{R}^{2n}$. We define coordinates
$I^j=q^j$ and $\theta_j=p_j+\mathbb{Z}$, $j=1,\ldots,n$.
If $H \in C^\infty(B^n \times \mathbb{T}^n)$ does not depend on $\theta_1,\ldots,\theta_n$ then we say that it has {\em action-angle coordinates}
in $B^n \times \mathbb{T}^n$.

If $H \in C^\infty(B^n \times \mathbb{T}^n)$ admits action-angle coordinates
$(I,\theta)$ then for all $x \in B^n \times \mathbb{T}^n$ we have
\[
\frac{d(I^j(\phi_t(x)))}{dt}=\frac{\partial H}{\partial \theta_j}(\phi_t(x))=0,
\]
i.e. $I^j(\phi_t(x))=I^j(x)$ for all $t$, and as $H$ depends only
on $I$ this gives
\[
\frac{d(\theta_j(\phi_t(x)))}{dt}=-\frac{\partial H}{\partial I^j}(\phi_t(x))=
-\frac{\partial H}{\partial I^j}(x)=\nu_j,
\]
where $\nu=\nu(I(x))$.
We integrate this equation from $0$ to $t$ and get
\[
\theta_j(\phi_t(x))-\theta_j(x)=t\nu_j.
\]
Thus for $x \in B^n \times \mathbb{T}^n$, given
$I(x)$ the trajectory $\phi_t(x)$ of $x$ under the Hamiltonian flow of $H$ 
can be explicitly seen if we know $\nu(I(x))$.
We say that a value of $I$ determines an {\em invariant torus} for the Hamiltonian flow of $H$.

If $(M,\omega)$ is a symplectic manifold and $H \in C^\infty(M)$, we say
that $H$ {\em admits action-angle coordinates} $(I,\theta)$ on an open set
$U \subset M$ if there exists a symplectic diffeomorphism $\psi:U \to
B^n \times \mathbb{T}^n$ such that $H \circ \psi^{-1}$ has action-angle
coordinates $(I,\theta)$ in $B^n \times \mathbb{T}^n$. If $H$ admits action-angle coordinates, then one can check that the push-forward $\psi_* X_H$ is
the Hamiltonian vector field $X_{H \circ \psi^{-1}}$, so that
\[
\psi_* X_H = - \sum_{j=1}^n \frac{\partial(H\circ \psi^{-1})}{\partial I_j}
\frac{\partial}{\partial \theta_j}.
\]

Let $f_1,\ldots,f_n \in C^\infty(\mathbb{R}^{2n})$. If the set
$\{f_1,\ldots,f_n\}$ is completely integrable, with $H=f_1$,
then for any open set $U \subseteq \mathbb{R}^{2n} \setminus \Sigma_F$
for which $F^{-1}(c)=\mathbb{T}^n$ for all $c \in U$, 
Abraham and Marsden \cite[pp. 398--400]{foundations} find action-angle coordinates in $U$.
Here $F=f_1 \times \cdots f_n$, the momentum map.
This construction is also explained by
Arnold \cite[pp. 282--284]{arnold}.

Suppose that $H \in C^\infty(B^n \times \mathbb{T}^n)$ has action-angle
coordinates $(I,\theta)$, and assume that for all $I \in B^n$,
\[
\det (\partial^2_I H (I)) \neq 0.
\]
Then by the inverse function theorem, for every $I \in B^n$ there
is a neighborhood $U$ of $I$ and a neighborhood $V$ of
$\nu=\partial_I H(I)$ such that $\partial_I H:U \to V$ is a diffeomorphism.
In $U \times \mathbb{T}^n$ we can use $\nu$ and $\theta$ as coordinates.

For $\nu \in \mathbb{R}^n$, let $g_\nu=\{k \in \mathbb{Z}^n:\langle \nu,k \rangle=0\}$, and let $\rank(g_\nu)$ be the rank of the $\mathbb{Z}$-module $g_\nu$, 
i.e. the maximal number of elements of $g_\nu$ that are linearly
independent over $\mathbb{Z}$. The proof of the following theorem follows
\cite[Proposition 2.1]{treschev}.

\begin{theorem}
Let $\nu \in \Omega$ and let $r=\rank(g_\nu)$. In the torus with frequency $\nu$, each trajectory is dense in some
$(n-r)$-dimensional subtorus and the $n$-dimensional torus is foliated
by these $(n-r)$-dimensional tori.
\end{theorem}
\begin{proof}
There exists a basis $k_1,\ldots,k_r$ of $g_\nu$ and vectors
$k_1^*,\ldots,k_{n-r}^* \in \mathbb{Z}^n$ such that the $n \times n$ matrix $K_0$ with rows
$k_1^*,\ldots,k_{n-r}^*,k_1,\ldots,k_r,$ has determinant $1$. (I should
show why such a basis exists.)
Let $K_0=\begin{bmatrix}K^*\\K \end{bmatrix}$. $K^*$ is an $(n-r) \times n$ matrix
and $K$ is an $r \times n$ matrix.

Let $q=K_0 \theta$. Since $\det(K_0)=1$, $K_0$ is invertible over
$\mathbb{Z}$. The coordinate $\theta$ is only determined up to
$\mathbb{Z}^n$, and for $q_1-q_2 \in \mathbb{Z}^n$ then also
$\theta_1-\theta_2 \in \mathbb{Z}^n$. Thus $q=K_0 \theta$ are coordinates
on $\mathbb{T}^n$.
The equation $\dot{\theta}=\nu$ can be written using the $q$ coordinates
as $\dot{q}=K_0 \nu$. Then
\[
K_0 \nu=\begin{bmatrix}K^*\\K\end{bmatrix}\nu=
\begin{bmatrix}K^* \nu\\K \nu \end{bmatrix}=
\begin{bmatrix}K^* \nu\\0 \end{bmatrix}.
\]
Let $\nu^*=K^* \nu$.

We see that $\{l \in \mathbb{Z}^n:l_1=\cdots=l_{n-r}=0\} \subseteq g_{K_0 \nu}$; since they both have rank $r$, they are equal. It follows that $\nu^* \in \mathbb{R}^{n-r}$ is
nonresonant. Hence any trajectory on the $n$-dimensional torus with frequency
$\nu$ is dense in the $r$-dimensional torus $\{q \in \mathbb{T}^n:
q_{n-r+1}=\cdots=q_n=\text{constant}\}$.
\end{proof}


\section{Diophantine frequency vectors}
For $c>0$ and $\gamma \geq 0$ we define 
\[
D_n(c,\gamma)=\{\nu \in \mathbb{R}^n: |\langle k,\nu \rangle| \geq
\frac{1}{c\norm{k}_\infty^\gamma} \ \ \textrm{for all} \ k \in \mathbb{Z}^n\}.
\]

We further define $D_n(\gamma)=\bigcup_{c>0} D_n(c,\gamma)$.

\begin{theorem}
\label{thm:smallgamma}
For any $\nu \in \mathbb{R}^n$ and for any positive integer $K$, there is some $0 \neq k \in \mathbb{Z}^n$ with $\norm{k}_\infty \leq 2K$
such that 
\[
|\langle k,\nu \rangle | \leq \frac{n\norm{\nu}_\infty}{(2K)^{n-1}}.
\]
\end{theorem}
\begin{proof}
Let $B_K=\{k \in \mathbb{Z}^n: 0<\norm{k}_\infty \leq K\}$. The set $B_K$ has $(2K+1)^n-1$ elements. For
$k \in B_K$ we have
\[
|\langle k,\nu \rangle | \leq n \norm{k}_\infty \norm{\nu}_\infty \leq nK \norm{\nu}_\infty.
\]
Let $A=nK \norm{\nu}_\infty$.

Let $M=(2K+1)^n-2$.  In the set $\{|\langle k,\nu \rangle |: k \in B_K\}$, there are two elements that
are in  same interval $[\frac{(j-1)A}{M},\frac{jA}{M}]$, $j=1,\ldots,M$, since $B_K$ has $M+1$ elements and there
are $M$ such intervals. That is, there are $k',k'' \in B_K$ such that 
$|\langle k',\nu \rangle|, |\langle k'',\nu \rangle| \in [\frac{(j-1)A}{M},\frac{jA}{M}]$ for some $j$. Hence
$|\langle k',\nu \rangle - \langle k'',\nu \rangle| \leq \frac{A}{M}=\frac{nK \norm{\nu}_\infty}{(2K+1)^n-2}$.

One can show by induction that for all $n \geq 1$, $\frac{K}{(2K+1)^n-2} \leq \frac{1}{(2K)^{n-1}}$. Therefore for $k=k'-k''$ we have
\[
|\langle k,\nu \rangle | \leq \frac{n \norm{\nu}_\infty}{(2K)^{n-1}},
\]
Finally, $\norm{k}_\infty \leq \norm{k'}_\infty+\norm{k''}_\infty \leq 2K$.
\end{proof}

\begin{corollary}
If $\gamma<n-1$ then $D_n(\gamma)=\emptyset$.
\end{corollary}
\begin{proof}
Let $c>0$. Suppose that there is some $\nu \in D_n(c,\gamma)$. Let $K$
be the least integer such that $(2K)^{n-1-\gamma}$ is greater than
$2cn \norm{\nu}_\infty$; since $n-1-\gamma>0$ such a $K$ exists.

By Theorem \ref{thm:smallgamma}, there is some $0 \neq k \in \mathbb{Z}^n$ with 
\[
|\langle k,\nu \rangle | \leq \frac{n\norm{\nu}_\infty}{(2K)^{n-1}}.
\]
Then
\begin{eqnarray*}
|\langle k,\nu \rangle |& \leq& \frac{n\norm{\nu}_\infty (2K)^{-\gamma}}{(2K)^{n-1-\gamma}}\\
&\leq&\frac{n\norm{\nu}_\infty (2K)^{-\gamma}}{2cn\norm{\nu}_\infty}\\
&=&\frac{1}{2c(2K)^\gamma}\\
&\leq&\frac{1}{2c(4\norm{k}_\infty)^\gamma}\\
&<&\frac{1}{c\norm{k}_\infty^\gamma},
\end{eqnarray*}
contradicting that $\nu \in D_n(c,\gamma)$. Therefore for all $c>0$,
$D_n(c,\gamma)=\emptyset$.

\end{proof}

Treschev and Zubelevich give a construction for points in $D_n(c,n-1)$ for sufficiently
large $c$ \cite[Theorem 9.2]{treschev}. Thus there is some $C(n)$ such that
for all $c \geq C(n)$, $D_n(c,n-1) \neq \emptyset$. It is clear that for $\gamma' \geq \gamma$ we have the inclusion
$D_n(c,\gamma) \subseteq D_n(c,\gamma')$. Hence this construction
also shows that $D_n(c,\gamma) \neq \emptyset$
for all $\gamma \geq n-1$ and $c \geq C(n)$. However this construction does not show that
$m(D_n(c,n-1))>0$ for $c \geq C(n)$. Indeed, one can show that $m(D_n(n-1))=0$, but also that the set $D_n(n-1)$ has Hausdorff dimension $n$ \cite[p. 5]{poeschel}. 

Our proof of the following theorem expands on \cite[Theorem 9.3]{treschev}.  Let
$Q_n(L)=\{\nu \in \mathbb{R}^n: \norm{\nu}_\infty \leq \frac{L}{2}\}$, the cube in
$\mathbb{R}^n$ of edge length $L$. Let $m$ be $n$-dimensional Lebesgue measure. We will use the fact that the maximal $n-1$ dimensional area of the intersection of $Q_n(L)$ and a hyperplane is $\sqrt{2}L^{n-1}$ \cite{MR840631}.

\begin{theorem}
\label{thm:biggamma}
Let $L>0$.
For $\gamma>n-1$ and $c>0$,
\[
m(Q_n(L) \setminus D_n(c,\gamma)) \leq \frac{4\sqrt{2}n (3L)^{n-1}}{c} \Big(1-\frac{1}{\gamma-n+1} \Big).
\]
\end{theorem}
\begin{proof}
Let $Q_n=Q_n(L)$.
Let $\Pi_k=\{\nu \in \mathbb{R}^n:|\langle \nu,k\rangle | <
\frac{1}{c\norm{k}_\infty^\gamma}\}$. 
Let $\nu \in Q_n \setminus D_n(c,\gamma)$. Then there is some $k \neq 0$ such that
$|\langle k,\nu \rangle| < \frac{1}{c \norm{k}_\infty^\gamma}$, and so $\nu \in \Pi_k$. Thus
\[
Q_n \setminus D_n(c,\gamma) \subseteq \bigcup_{k \neq 0}
(Q_n \cap \Pi_k),
\]
so
\[
m(Q_n \setminus D_n(c,\gamma)) \leq \sum_{k \neq 0} m(Q_n \cap \Pi_k).
\]

Let $k \neq 0$.
$\Pi_k$ is the region bounded by the two hyperplanes
$\pi_1 = \{\nu \in \mathbb{R}^n: \langle \nu,k \rangle = \frac{1}{c \norm{k}_\infty^\gamma}\}$ and $\pi_2=\{\nu \in \mathbb{R}^n: \langle \nu,k \rangle = -\frac{1}{c \norm{k}_\infty^\gamma}\}$.
Let $p_1=\frac{k}{c \norm{k}_\infty^\gamma \norm{k}} \in \pi_1$
and $p_2=-\frac{k}{c \norm{k}_\infty^\gamma \norm{k}} \pi_2$.
For any two points $\nu_1,\nu_2 \in \pi_1$ we can check that
$\langle p_1-p_2,\nu_1-\nu_2 \rangle =0$, and for any two points $\nu_1,\nu_2 \in \pi_2$
we can check that $\langle p_1-p_2,\nu_1-\nu_2 \rangle =0$. Thus the vector
$p_1-p_2$ is orthogonal to each of the hyperplanes $\pi_1$ and $\pi_2$. It follows
that the distance between the hyperplanes $\pi_1$ and $\pi_2$ is the distance between the points
$p_1$ and $p_2$, which is $2\cdot \frac{\norm{k}}{c\norm{k}_\infty^\gamma \norm{k}^2}$. Since $\norm{k} \geq
\norm{k}_\infty$, this is $\leq \frac{2}{c\norm{k}_\infty^{\gamma+1}}$.
Therefore
\[
m(Q_n \cap \Pi_k) \leq \frac{2}{c\norm{k}_\infty^{\gamma+1}} \cdot \sqrt{2}L^{n-1},
\]
where we use the fact that the maximal $n-1$ dimensional area of the intersection of $Q_n=Q_n(L)$ and a hyperplane is $\sqrt{2}L^{n-1}$ \cite{MR840631}.

For each positive integer $l$, the hypercube $\{k \in \mathbb{Z}^n: 
\norm{k}_\infty =l\}$ has $2n$ faces, on each of which there are $(2l+1)^{n-1}$
points with integer coordinates. Hence for each integer positive integer $l$, we have
$\#\{k \in \mathbb{Z}^n: \norm{k}_\infty = l \} \leq 2n (2l+1)^{n-1}$. 

Therefore
\begin{eqnarray*}
m(Q_n \setminus D_n(c,\gamma))&\leq&\sum_{k \neq 0} m(Q_n \cap \Pi_k)\\
&\leq&\sum_{k \neq 0} \frac{2\sqrt{2}L^{n-1}}{c\norm{k}_\infty^{\gamma+1}}\\
&=&\sum_{l=1}^\infty \sum_{\norm{k}_\infty =l} \frac{2\sqrt{2}L^{n-1}}{cl^{\gamma+1}} \\\
&\leq&\sum_{l=1}^\infty 2n (2l+1)^{n-1} \frac{2\sqrt{2}L^{n-1}}{cl^{\gamma+1}}\\
&\leq&\sum_{l=1}^\infty 2n (3l)^{n-1} \frac{2\sqrt{2}L^{n-1}}{cl^{\gamma+1}}\\
&=&\frac{4\sqrt{2}n (3L)^{n-1}}{c} \sum_{l=1}^\infty \frac{1}{l^{\gamma-n+2}}.
\end{eqnarray*}

Since the terms in the sum are positive and decreasing, we can estimate the sum using an integral:
\[
\sum_{l=1}^\infty \frac{1}{l^{\gamma-n+2}} \leq 1+ \int_1^\infty
\frac{dx}{x^{\gamma-n+2}} = 1+ \frac{1}{\gamma-n+1},
\]
finishing the proof.
\end{proof}

\begin{corollary}
If $\gamma>n-1$ then $m(\mathbb{R}^n \setminus D_n(\gamma))=0.$ 
\end{corollary}
\begin{proof}
Let $L>0$. For every $c>0$, $m(Q_n(L) \setminus D_n(\gamma))
\leq m(Q_n(L) \setminus D_n(c,\gamma))$. By Theorem \ref{thm:biggamma},
$m(Q_n(L) \setminus D_n(c,\gamma)) \to 0$ as $c \to \infty$. Hence
$m(Q_n(L) \setminus D_n(\gamma))=0$. But then
\[
m(\mathbb{R}^n \setminus
D_n(\gamma)) = \lim_{L \to \infty} m(Q_n(L) \setminus D_n(\gamma))
=\lim_{L \to \infty} 0 =0.
\]
\end{proof}

Fix $\gamma>n-1$.
Let $\alpha=\frac{1}{c}$. Let $A_\alpha$ be an $\alpha$-neighborhood of the boundary of $\Omega$. We will make whatever assumption
about $\partial \Omega$ we need in order to get $m(A_\alpha)=O(\alpha)$. 

Suppose that $L$ is sufficiently large so that $\Omega \subseteq Q_n(L)$. Then Theorem \ref{thm:biggamma} gives us that $m(\Omega \setminus D_n(c,\gamma))=
O(\alpha)$.

Let $\Omega_\alpha=D_n(c,\gamma) \cap (\Omega \setminus A_\alpha)$.
Since $\Omega \setminus \Omega_\alpha = (\Omega \setminus D_n(c,\gamma)) \cup
(\Omega \cap A_\alpha)$, we have
$m(\Omega \setminus \Omega_\alpha)=O(\alpha)$.



\section{Statement of KAM}
If we have a Hamiltonian system which admits action-angle coordinates
in $B^n \times \mathbb{T}^n$, then the trajectories of points in phase
space are constrained to lie on invariant tori. Moreover, on these tori
the dynamics of the system are quasi-periodic; a priori we don't have a reason to expect that the dynamics should be so nice just because the trajectories lie on tori. But a generic Hamiltonian on the same phase space (I would like to make
this notion precise) does not admit action-angle coordinates. The KAM
theorem is a statement about the dynamics induced by making a sufficiently small change to a Hamiltonian. If we perturb a Hamiltonian which admits
action-angle coordinates to one which probably does not, if the perturbation
is sufficiently small, then most of the trajectories of points under the flow of the new Hamiltonian will also lie on tori. In some sense which I want to clarify, the invariant tori of the new Hamiltonian are close to the invariant
tori of the Hamiltonian that admits action-angle coordinates. It is not
clear to me how an invariant torus of the old Hamiltonian transforms into
an invariant torus of the new Hamiltonian; in what sense does an invariant torus for the old Hamiltonian become an invariant torus for the new Hamiltonian?

In particular, a consequence of the KAM theorem is that if we make a small
perturbation of a Hamiltonian system that admits action-angle coordinates then
the trajectories of most points will not be dense on a hypersurface in phase space, since they
are constrained to lie on $n$-dimensional tori. In other words, the
new Hamiltonian system is not ergodic, since the invariant tori have lower
dimension than $n-1$, and so have $n-1$-dimensional measure 0.

Let's explain the KAM theorem in another way. Suppose that we have a 
symplectic manifold $M$ and a Lagrangian foliation $\mathscr{F}_0$ whose
leaves are tori, and suppose that the leaves of $\mathscr{F}_0$ are invariant
tori for a Hamiltonian $H_0$. That is,
the Hamiltonian vector field $X_{H_0}$ is tangent to all the leaves
in $\mathscr{F}_0$.
Now let $H=H_0+\epsilon H_1$. The leaves
of the foliation $\mathscr{F}_0$ will not be invariant under the flow 
of $H$. We would like to obtain a symplectomorphism $\Phi:M \to M$
such that the Hamiltonian vector field $X_H$ is tangent to most
leaves in the foliation $\mathscr{F}=\Phi(\mathscr{F}_0)$.
Here we mean most in a measure theoretic sense that depends on the magnitude
$\epsilon$ of the perturbation away from the Hamiltonian that admits action-angle
coordinates.

How do we construct a diffeomorphism? Often the best way is to demand
that it be the time $1$ flow of a vector field, so $\Phi=\Phi_1$ for
some $\Phi_t$, and to see if such a vector
field exists. Suppose that $f$ is a function such that if $\Phi_t$ is the flow of $X_f$ then $\Phi_1=\Phi$.
 

\section{Normal forms}
Normal forms of vector fields,
homological equation \cite{wiggins}.

\bibliographystyle{amsplain}
\bibliography{kam}

\end{document}
