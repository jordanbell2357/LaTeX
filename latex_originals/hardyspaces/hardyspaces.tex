\documentclass{article}
\usepackage{amsmath,amssymb,mathrsfs,amsthm}
%\usepackage{tikz-cd}
%\usepackage{hyperref}
\newcommand{\inner}[2]{\left\langle #1, #2 \right\rangle}
\newcommand{\tr}{\ensuremath\mathrm{tr}\,} 
\newcommand{\Span}{\ensuremath\mathrm{span}} 
\def\Re{\ensuremath{\mathrm{Re}}\,}
\def\Im{\ensuremath{\mathrm{Im}}\,}
\newcommand{\id}{\ensuremath\mathrm{id}} 
\newcommand{\var}{\ensuremath\mathrm{var}} 
\newcommand{\Lip}{\ensuremath\mathrm{Lip}} 
\newcommand{\GL}{\ensuremath\mathrm{GL}}
\newcommand{\diam}{\ensuremath\mathrm{diam}} 
\newcommand{\sgn}{\ensuremath\mathrm{sgn}\,} 
\newcommand{\lcm}{\ensuremath\mathrm{lcm}} 
\newcommand{\supp}{\ensuremath\mathrm{supp}\,}
\newcommand{\dom}{\ensuremath\mathrm{dom}\,}
\newcommand{\upto}{\nearrow}
\newcommand{\downto}{\searrow}
\newcommand{\norm}[1]{\left\Vert #1 \right\Vert}
\newtheorem{theorem}{Theorem}
\newtheorem{lemma}[theorem]{Lemma}
\newtheorem{proposition}[theorem]{Proposition}
\newtheorem{corollary}[theorem]{Corollary}
\theoremstyle{definition}
\newtheorem{definition}[theorem]{Definition}
\newtheorem{example}[theorem]{Example}
\begin{document}
\title{Hardy spaces}
\author{Jordan Bell\\ \texttt{jordan.bell@gmail.com}\\Department of Mathematics, University of Toronto}
\date{\today}

\maketitle

\section{Hardy spaces}
Let $D_r=\{z:|z|<r\}$. For a continuous function $f:D_1 \to \mathbb{C}$,
let
\[
M_p(r,f) = \left( \frac{1}{2\pi} \int_0^{2\pi} |f(re^{i\theta})|^p d\theta \right)^{1/p},
\qquad 0<p<\infty,
\]
and
\[
M_\infty(r,f) = \sup_{0 \leq \theta \leq 2\pi} |f(r^{i\theta})|.
\]
Let $H^p$ be the collection of analytic functions $f:D_1 \to \mathbb{C}$ 
such that $\norm{f}_{H^p}<\infty$, where
\[
\norm{f}_{H^p} = \sup_{0<r<1} M_p(r,f).
\]
Let $h^p$ be the collection of harmonic functions $u:D_1 \to \mathbb{R}$ such that $\norm{u}_{H^p} <\infty$.

\begin{lemma}
For $0<p \leq \infty$,
$H^p$ and $h^p$ are linear spaces.
If $p<q$ then $H^q \subset H^p$ and $h^q \subset h^p$. 
For an analytic function $f:D_1 \to \mathbb{C}$, $f \in H^p$ if and only if $\Re f, \Im f \in h^p$. 
\end{lemma}
\begin{proof}
For $a \geq 0$, $b \geq 0$,
\[
(a+b)^p \leq \begin{cases}
a^p+b^p&0<p<1\\
2^{p-1}(a^p+b^p)&p>1.
\end{cases}
\]
\begin{align*}
\norm{f+g}_{H^p}^p &= \sup_{0<r<1}  \frac{1}{2\pi} \int_0^{2\pi} |(f+g)(re^{i\theta})|^p d\theta\\
&\leq 2^p (\norm{f}_{H^p}^p+\norm{g}_{H^p}^p)\\
&\leq 2^p \cdot 2 (\norm{f}_{H^p}+\norm{g}_{H^p})^p,
\end{align*}
hence $\norm{f+g}_{H^p} \leq 2^{1+1/p} (\norm{f}_{H^p}+\norm{g}_{H^p})$. 
It follows that $H^p$ and $h^p$ are linear spaces.
\end{proof}


\section{Subharmonic functions}
Let $D$ be a domain,  namely, a nonempty connected open set in $\mathbb{C}$. 
A function $g:D \to \mathbb{R}$ is called \textbf{subharmonic} if it is continuous, 
and for any domain $B$ with $\overline{B} \subset D$ and continuous function
$U:\overline{B} \to \mathbb{R}$ such that $U|B$ is harmonic and such that
$g(z) \leq U(z)$ for all $z \in \partial B$, 
it follows that $g(z) \leq U(z)$ for all $z \in B$.\footnote{Peter L. Duren,
{\em Theory of $H^p$ Spaces}, p.~7, Theorem 1.4.}

\begin{theorem}
Let $g:D \to \mathbb{R}$ be continuous. $g$ is subharmonic if and only if
for any $a \in D$ there is some $r_a>0$ such that $D_{r_a}(a) \subset D$ and for each
$0<r<r_a$,
\[
g(a) \leq \frac{1}{2\pi} \int_0^{2\pi} g(a+ r e^{i\theta}) d\theta.
\]
\end{theorem}


\begin{lemma}
If $f:D \to \mathbb{C}$ is analytic and $0<p<\infty$ then $g(z)=|f(z)|^p$ is subharmonic.
\end{lemma}


\begin{theorem}
Let $g:D_1 \to \mathbb{R}$ be subharmonic and define
\[
m(r) = \frac{1}{2\pi} \int_0^{2\pi} g(re^{i\theta}) d\theta,\qquad 0 \leq r < 1.
\]
$m$ is increasing and $r \mapsto m(e^r)$ is convex.
\end{theorem}


\footnote{Peter L. Duren,
{\em Theory of $H^p$ Spaces}, p.~46, Theorem 3.13.}

\begin{theorem}[Fej\'er-Riesz inequality]
If $f \in H^p$, then
\[
\int_{-1}^1 |f(x)|^p dx \leq \frac{1}{2} \int_0^{2\pi} |f(e^{i\theta})|^p d\theta.
\]
\end{theorem}


\footnote{Peter L. Duren,
{\em Theory of $H^p$ Spaces}, p.~54, Theorem 4.1.}

\begin{theorem}
If $1<p<\infty$ there is some $A_p$ such that
if $u \in h^p$ and $v$ is the harmonic conjugate of $u$, $v(0)=0$, then 
\[
M_p(r,v) \leq A_p M_p(r,u),\qquad 0 \leq r <1.
\]
\end{theorem}


\end{document}