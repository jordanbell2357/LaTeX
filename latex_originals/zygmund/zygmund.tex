\documentclass{article}
\usepackage{amsmath,amssymb,mathrsfs,amsthm}
%\usepackage{tikz-cd}
%\usepackage{hyperref}
\newcommand{\inner}[2]{\left\langle #1, #2 \right\rangle}
\newcommand{\tr}{\ensuremath\mathrm{tr}\,} 
\newcommand{\Span}{\ensuremath\mathrm{span}} 
\def\Re{\ensuremath{\mathrm{Re}}\,}
\def\Im{\ensuremath{\mathrm{Im}}\,}
\newcommand{\id}{\ensuremath\mathrm{id}} 
\newcommand{\var}{\ensuremath\mathrm{var}} 
\newcommand{\Lip}{\ensuremath\mathrm{Lip}} 
\newcommand{\GL}{\ensuremath\mathrm{GL}} 
\newcommand{\diam}{\ensuremath\mathrm{diam}} 
\newcommand{\sgn}{\ensuremath\mathrm{sgn}\,} 
\newcommand{\lcm}{\ensuremath\mathrm{lcm}} 
\newcommand{\supp}{\ensuremath\mathrm{supp}\,}
\newcommand{\dom}{\ensuremath\mathrm{dom}\,}
\newcommand{\upto}{\nearrow}
\newcommand{\downto}{\searrow}
\newcommand{\norm}[1]{\left\Vert #1 \right\Vert}
\newtheorem{theorem}{Theorem}
\newtheorem{lemma}[theorem]{Lemma}
\newtheorem{proposition}[theorem]{Proposition}
\newtheorem{corollary}[theorem]{Corollary}
\theoremstyle{definition}
\newtheorem{definition}[theorem]{Definition}
\newtheorem{example}[theorem]{Example}
\begin{document}
\title{Zygmund's Fourier restriction theorem and Bernstein's inequality}
\author{Jordan Bell\\ \texttt{jordan.bell@gmail.com}\\Department of Mathematics, University of Toronto}
\date{\today}

\maketitle

\section{Zygmund's restriction theorem}
Write $\mathbb{T}^d = \mathbb{R}^d / \mathbb{Z}^d$.
Write $\lambda_d$ for  the Haar measure on $\mathbb{T}^d$ for which $\lambda_d(\mathbb{T}^d)=1$. 
For $\xi \in \mathbb{Z}^d$, we define $e_\xi:\mathbb{T}^d \to S^1$ by
\[
e_\xi(x) = e^{2\pi i\xi\cdot x}, \qquad x \in \mathbb{T}^d.
\]
For
$f \in L^1(\mathbb{T}^d)$, we define its \textbf{Fourier transform} $\hat{f}:\mathbb{Z}^d \to \mathbb{C}$
by
\[
\hat{f}(\xi) =\int_{\mathbb{T}^d} f \overline{e_\xi} d\lambda_d=  \int_{\mathbb{T}^d} f(x) e^{-2\pi i\xi\cdot x} dx, \qquad \xi\in\mathbb{Z}^d.
\]
For $x \in \mathbb{R}^d$, we write $|x|=|x|_2=\sqrt{x_1^2+\cdots+x_d^2}$,
$|x|_1 = |x_1|+\cdots+|x_d|$, and
$|x|_\infty=\max \{|x_j|: 1 \leq j \leq d\}$.



For $1 \leq p < \infty$, we write
\[
\norm{f}_p  = \left( \int_{\mathbb{T}^d} |f(x)|^p dx \right)^{1/p}.
\]
For $1 \leq p \leq q \leq \infty$, $\norm{f}_p \leq \norm{f}_q$.

Parseval's identity tells us that for $f \in L^2(\mathbb{T}^d)$,
\[
\norm{\hat{f}}_{\ell^2} = \left( \sum_{\xi \in \mathbb{Z}^d} |\hat{f}(\xi)|^2 \right)^{1/2} = \norm{f}_2,
\]
and the Hausdorff-Young inequality tells us that for $1 \leq p \leq 2$ and $f \in L^p(\mathbb{T}^d)$, 
\[
\norm{\hat{f}}_{\ell^q} = \left( \sum_{\xi \in \mathbb{Z}^d} |\hat{f}(\xi)|^q \right)^{1/q} \leq 
\norm{f}_p,
\]
where $\frac{1}{p}+\frac{1}{q}=1$; $\norm{\hat{f}}_{\ell^\infty}=\sup_{\xi \in \mathbb{Z}^d} |\hat{f}(\xi)|$.


\textbf{Zygmund's theorem} is the following.\footnote{Mark A. Pinsky, {\em Introduction to Fourier Analysis and Wavelets}, p.~236, Theorem 4.3.11.}

\begin{theorem}[Zygmund's theorem]
For $f \in L^{4/3}(\mathbb{T}^2)$ and $r>0$,
\begin{equation}
\left( \sum_{|\xi|=r} |\hat{f}(\xi)|^2 \right)^{1/2} \leq 5^{1/4} \norm{f}_{4/3}.
\label{zygmund}
\end{equation}
\end{theorem}
\begin{proof}
Suppose that 
\[
S=\left( \sum_{|\xi|=r} |\hat{f}(\xi)|^2 \right)^{1/2} > 0.
\]
For $\xi \in \mathbb{Z}^2$, we define 
\[
c_\xi = \frac{\overline{\hat{f}(\xi)}}{S} \chi_{|\zeta|=r}.
\]
Then
\begin{equation}
\sum_{|\xi|=r} |c_\xi|^2 = \sum_{|\xi|=r}
\frac{|\hat{f}(\xi)|^2}{|S|^2}
=1.
\label{rsum}
\end{equation}
We have
\begin{align*}
S^2&=\sum_{|\xi|=r} |\hat{f}(\xi)|^2\\
&=\sum_{|\xi|=r} \hat{f}(\xi) \overline{\hat{f}(\xi)}\\
&=\left(\sum_{|\xi|=r} \hat{f}(\xi) c_\xi\right) S,
\end{align*}
hence, defining $c:\mathbb{T}^2 \to \mathbb{C}$ by
\[
c(x)=\sum_{\xi \in \mathbb{Z}^d} c_\xi e^{2\pi i\xi\cdot x}=
\sum_{|\xi|=r} c_\xi e^{2\pi i\xi\cdot x}, \qquad x \in \mathbb{T}^2,
\]
we have, applying Parseval's identity,
\[
S=\sum_{|\xi|=r} \hat{f}(\xi) c_\xi
=\int_{\mathbb{T}^2} f(x) \overline{c(x)} dx.
\]
For $p=\frac{4}{3}$, let $\frac{1}{p}+\frac{1}{q}=1$, i.e.
$q=4$. 
H\"older's inequality
tells us
\[
\int_{\mathbb{T}^2} |f(x) \overline{c(x)}| dx 
\leq \norm{f}_{4/3} \norm{c}_4.
\]

For $\rho \in \mathbb{Z}^2$, we define
\[
\gamma_\rho = \sum_{\mu-\nu=\rho} c_\mu \overline{c_\nu}.
\]
Then  define 
$\Gamma(x) = |c(x)|^2$,
which satisfies
\[
\Gamma(x) = c(x) \overline{c(x)}=
\sum_{\xi \in \mathbb{Z}^2} \sum_{\zeta \in \mathbb{Z}^2} c_\xi \overline{c_\zeta} e^{2\pi i(\xi-\zeta)\cdot x}
=\sum_{\rho \in \mathbb{Z}^2} \gamma_\rho e^{2\pi i\rho \cdot x}.
\]
Parseval's identity tells us
\[
\norm{c}_4^4 = \norm{\Gamma}_2^2 = \sum_{\rho \in \mathbb{Z}^2} |\gamma_\rho|^2.
\]
First,
\[
\gamma_0 = \sum_{\mu \in \mathbb{Z}^2} c_\mu \overline{c_\mu}
=\sum_{\mu \in \mathbb{Z}^2} |c_\mu|^2
=1.
\]
Second, suppose that $\rho \in \mathbb{Z}^2, |\rho|=2r$. If $\rho/2 \in \mathbb{Z}^2$, then $\gamma_\rho = c_{\rho/2} \overline{c_{-\rho/2}}$, 
and if $\rho/2 \not \in \mathbb{Z}^2$ then $\gamma_\rho=0$. It follows that
\begin{equation}
\sum_{|\rho|=2r} |\gamma_\rho|^2 = \sum_{|\mu|=r} |\gamma_{2\mu}|^2 = 
\sum_{|\mu|=r} |c_\mu|^2 |c_{-\mu}|^2.
\label{2r}
\end{equation}
Third, suppose that $\rho \in \mathbb{Z}^2, 0<|\rho|<2r$. Then, for
\[
C_\rho=\{\mu \in \mathbb{Z}^2: |\mu|=r, |\mu-\rho|=|\rho|\},
\] 
we have $|C_\rho| \leq 2$. If $|C_\rho|=0$ then $\gamma_\rho=0$. If $|C_\rho|=1$ and $C_\rho=\{\mu\}$, then
$\gamma_\rho = c_\mu \overline{c_{\mu-\rho}}$ and so
$|\gamma_\rho|^2 = |c_\mu|^2 |c_{\mu-\rho}|^2$.
If $|C_\rho|=2$ and $C_\rho=\{\mu,m\}$, then
$\gamma_\rho=c_\mu \overline{c_{\mu-\rho}}+c_m \overline{c_{m-\rho}}$ and so
\[
|\gamma_\rho|^2 \leq 2 |c_\mu|^2 |c_{\mu-\rho}|^2+2|c_m|^2 |c_{m-\rho}|^2.
\]
It follows that
\[
\sum_{0<|\rho|<2r} |\gamma_\rho|^2 \leq 4\sum_{|\mu|=r, |\nu|=r, 0<|\mu-\nu|<2r} |c_\mu|^2 |c_\nu|^2.
\]
Using \eqref{2r} and then \eqref{rsum},
\begin{align*}
\sum_{0<|\rho| \leq 2r} |\gamma_\rho|^2&  \leq 4\sum_{|\mu|=r, |\nu|=r, 0<|\mu-\nu|<2r} |c_\mu|^2 |c_\nu|^2+
\sum_{|\mu|=r} |c_\mu|^2 |c_{-\mu}|^2\\
&\leq 4\sum_{|\mu|=r, |\nu|=r, 0<|\mu-\nu|<2r} |c_\mu|^2 |c_\nu|^2+4\sum_{|\mu|=r} |c_\mu|^2 |c_{-\mu}|^2\\
&\leq 4\sum_{|\mu|=r, |\nu|=r}|c_\mu|^2 |c_\nu|^2\\
&=4\left( \sum_{|\mu|=r} |c_\mu|^2 \right)^2\\
&=4.
\end{align*}
Fourth, if $\rho \in \mathbb{Z}^2, |\rho|>2r$ then $\gamma_\rho=0$. Putting the above together,
we have
\[
\sum_{\rho \in \mathbb{Z}^2} |\gamma_\rho|^2 \leq 1+4=5.
\]
Hence $\norm{c}_4^4\leq 5$, and therefore 
\[
|S|=\left| \int_{\mathbb{T}^2} f(x) \overline{c(x)} dx \right| 
\leq \int_{\mathbb{T}^2} |f(x) \overline{c(x)}| dx
\leq \norm{f}_{4/3} \norm{c}_4
\leq \norm{f}_{4/3} 5^{1/4},
\]
proving the claim.
\end{proof}


\section{Tensor products of functions}
For $f_1:X_1 \to \mathbb{C}$ and $f_2:X_2 \to \mathbb{C}$, we define
$f_1 \otimes f_2:X_1 \times X_2 \to \mathbb{C}$ by
\[
f_1 \otimes f_2(x_1,x_2) = f_1(x_1)f_2(x_2), \qquad (x_1,x_2) \in X_1 \times X_2.
\]
For $f_1 \in L^1(\mathbb{T}^{d_1})$ and $f_2 \in L^1(\mathbb{T}^{d_2})$, it follows from Fubini's theorem that
$f_1 \otimes f_2 \in L^1(\mathbb{T}^{d_1+d_2})$. 

For $\xi_1 \in \mathbb{Z}^{d_1}$ and $\xi_2 \in \mathbb{Z}^{d_2}$, Fubini's theorem gives us
\begin{align*}
\widehat{f_1 \otimes f_2}(\xi_1,\xi_2)&=\int_{\mathbb{T}^{d_1+d_2}} f_1 \otimes f_2 (x_1,x_2) e^{-2\pi i(\xi_1,\xi_2)\cdot (x_1,x_2)} d\lambda_{d_1+d_2}(x_1,x_2)\\
&=\int_{\mathbb{T}^{d_1}} \left( \int_{\mathbb{T}^{d_2}}  f_1 \otimes f_2 (x_1,x_2) e^{-2\pi i(\xi_1,\xi_2)\cdot (x_1,x_2)} d\lambda_{d_2}(x_2) \right) d\lambda_{d_1}(x_1)\\
&=\int_{\mathbb{T}^{d_1}} f_1(x_1) e^{-2\pi i\xi_1 \cdot x_1} \left( \int_{\mathbb{T}^{d_2}} f_2(x_2) e^{-2\pi i\xi_2 \cdot x_2} d\lambda_{d_2}(x_2) \right)
d\lambda_{d_1}(x_1)\\
&=\hat{f_1}(\xi_1) \hat{f_2}(\xi_2)\\
&=\hat{f_1} \otimes \hat{f_2}(\xi_1,\xi_2),
\end{align*}
showing that the Fourier transform of a tensor product is the tensor product of the Fourier transforms.




\section{Approximate identities and Bernstein's inequality  for $\mathbb{T}$}
An \textbf{approximate identity} is a sequence $k_N$ in $L^\infty(\mathbb{T}^d)$ such that
(i) $\sup_N \norm{k_N}_1 < \infty$,  (ii) for each $N$,
\[
\int_{\mathbb{T}^d} k_N(x) d\lambda_d(x)=1,
\]
and 
(iii) for each $0<\delta<\frac{1}{2}$,
\[
\lim_{n \to \infty} \int_{\delta\leq x \leq 1-\delta} |k_N(x)| d\lambda_d(x)=0.
\]
Suppose that $k_N$ is an approximate identity.
It is a fact that if $f \in C(\mathbb{T}^d)$ then $k_N * f \to f$ in $C(\mathbb{T}^d)$,
if $1 \leq p < \infty$ and $f \in L^p(\mathbb{T}^d)$ then
$k_N*f \to f$ in $L^p(\mathbb{T}^d)$, and if $\mu$ is a complex Borel measure on $\mathbb{T}^d$ then
$k_N * \mu$ weak-* converges to $\mu$.\footnote{Camil Muscalu and Wilhelm Schlag, {\em Classical and Multilinear Harmonic
Analysis}, volume I, p.~10, Proposition 1.5.} (The Riesz representation theorem tells us that the 
Banach space $\mathcal{M}(\mathbb{T}^d)=rca(\mathbb{T}^d)$ of complex Borel measures on $\mathbb{T}^d$, with the total variation norm, is the dual space
of the Banach space $C(\mathbb{T}^d)$.)

A \textbf{trigonometric polynomial} is a function $P:\mathbb{T}^d \to \mathbb{C}$ of the form
\[
P(x) = \sum_{\xi \in \mathbb{Z}^d} a_\xi e^{2\pi i\xi \cdot x}, \qquad x \in \mathbb{T}^d
\]
for which there is some $N \geq 0$ such that $a_\xi=0$ whenever $|\xi|_\infty>N$. We say that $P$ has \textbf{degree} $N$; thus,
if $P$ is a trigonometric polynomial of degree $N$ then $P$ is a trigonometric polynomial of degree $M$ for each $M \geq N$.

For $f \in L^1(\mathbb{T})$, we define $S_N f \in C(\mathbb{T})$ by
\[
(S_N f)(x) = \sum_{|j| \leq N} \hat{f}(j) e^{2\pi ijx}, \qquad x \in \mathbb{T}.
\]
We define the \textbf{Dirichlet kernel} $D_N:\mathbb{T} \to \mathbb{C}$ by
\[
D_N(x) = \sum_{|j| \leq N}  e^{2\pi ijx}, \qquad x \in \mathbb{T},
\]
which satisfies, for $f \in L^1(\mathbb{T})$,
\[
D_N * f = S_N f.
\]
We define the \textbf{Fej\'er kernel} $F_N \in C(\mathbb{T})$ by
\[
F_N = \frac{1}{N+1} \sum_{n=0}^N D_n,
\]
We can write the Fej\'er kernel as 
\[
F_N (x) = \sum_{|j| \leq N} \left(1-\frac{|j|}{N+1}\right) e^{2\pi ijx}
=\sum_{j \in \mathbb{Z}} \chi_{[-N,N]}(j) \left(1-\frac{|j|}{N+1}\right) e^{2\pi ijx},
\]
where $\chi_A$ is the indicator function of the set $A$.
It is straightforward to prove that $F_N$ is an approximate identity.

 We define the \textbf{$d$-dimensional Fej\'er kernel} $F_{N,d} \in C(\mathbb{T}^d)$ by
 \[
 F_{N,d} = \underbrace{F_N \otimes \cdots \otimes F_N}_{d}.
 \]
 We can write $F_{N,d}$ as
 \[
 F_{N,d}(x) = \sum_{|\xi|_\infty \leq N} \left(1-\frac{|\xi_1|}{N+1}\right)\cdots
 \left(1-\frac{|\xi_d|}{N+1}\right) e^{2\pi i\xi \cdot x},\qquad x \in \mathbb{T}^d.
 \]
 Using the fact that $F_N$ is an approximate identity on $\mathbb{T}$, one proves that
 $F_{N,d}$ is an approximate identity on $\mathbb{T}^d$. 
  


The following is \textbf{Bernstein's inequality for
$\mathbb{T}$}.

\begin{theorem}[Bernstein's inequality]
If $P$ is a trigonometric polynomial of degree $N$, then
\[
\norm{P'}_\infty \leq 4\pi N\norm{P}_\infty.
\]
\end{theorem}
\begin{proof}
Define
\[
Q=((e_{-N}P)*F_{N-1})e_N-((e_NP)*F_{N-1})e_{-N}.
\]
The Fourier transform of the first term on the right-hand side is, for $j \in \mathbb{Z}$,
\begin{align*}
(\widehat{e_{-N}P * F_{N-1}})*\widehat{e_N}(j)&=\sum_{k \in \mathbb{Z}} \widehat{e_{-N}P}(j-k) \widehat{F_{N-1}}(j-k)\widehat{e_N}(k)\\
&=\widehat{e_{-N}P}(j-N) \widehat{F_{N-1}}(j-N)\\
&=\widehat{P}(j) \widehat{F_{N-1}}(j-N),
\end{align*}
and the Fourier transform of the second term is
\[
\widehat{P}(j) \widehat{F_{N-1}}(j+N).
\]
Therefore, for $j \in \mathbb{Z}$, using $\widehat{P}= \chi_{[-N,N]} \widehat{P}$,
\begin{align*}
\widehat{Q}(j)&=\widehat{P}(j)\left(\widehat{F_{N-1}}(j-N)-\widehat{F_{N-1}}(j+N)\right)\\
&=\widehat{P}(j)\left( \chi_{[-N+1,N-1]}(j-N) \left(1-\frac{|j-N|}{N}\right)
-\chi_{[-N+1,N-1]} \left(1-\frac{|j+N|}{N}\right) \right)\\
&=\widehat{P}(j)\bigg(\chi_{[1,N]}(j)\left(1+\frac{j-N}{N}\right)
+\chi_{[N,2N-1]}(j)\left(1-\frac{j-N}{N}\right)\\
&-\chi_{[-2N+1,-N]}(j)\left(1+\frac{j+N}{N}\right)
-\chi_{[-N,-1]}(j)\left(1-\frac{j+N}{N}\right) \bigg)\\
&=\widehat{P}(j)\left(\chi_{[1,N]}(j)\left(1+\frac{j-N}{N}\right)-
\chi_{[-N,-1]}(j)\left(1-\frac{j+N}{N}\right)\right)\\
&=\widehat{P}(j)\left(\frac{j}{N} \chi_{[1,N]}(j)+
\frac{j}{N} \chi_{[-N,-1]}(j)\right)\\
&=\frac{j}{N} \widehat{P}(j).
\end{align*}
On the other hand,
\[
\widehat{P'}(j) = 2\pi i j \widehat{P}(j),
\]
so that
\[
P'=2\pi iNQ,
\]
i.e.
\[
P'=2\pi iN( ((e_{-N}P)*F_{N-1})e_N-((e_NP)*F_{N-1})e_{-N}).
\]
Then, by Young's inequality,
\begin{align*}
\norm{P'}_\infty&=2\pi N \norm{((e_{-N}P)*F_{N-1})e_N-((e_NP)*F_{N-1})e_{-N}}_\infty\\
&\leq 2\pi N \norm{((e_{-N}P)*F_{N-1})e_N}_\infty
+2\pi N\norm{((e_NP)*F_{N-1})e_{-N}}_\infty\\
&= 2\pi N \norm{(e_{-N}P)*F_{N-1}}_\infty
+2\pi N \norm{(e_NP)*F_{N-1}}_\infty\\
&\leq 2\pi N \norm{e_{-N} P}_\infty \norm{F_{N-1}}_1 + 2\pi N \norm{e_NP}_\infty \norm{F_{N-1}}_1\\
&=4\pi N \norm{P}_\infty.
\end{align*}
\end{proof}






 
 
\end{document}