\documentclass{article}
\usepackage{amsmath,amssymb,graphicx,subfig,mathrsfs,amsthm}
\newcommand{\norm}[1]{\Vert #1 \Vert}
\newcommand{\supp}{\mathop{\mathrm{supp}}}
\theoremstyle{definition}
\newtheorem{theorem}{Theorem}
\newtheorem{lemma}[theorem]{Lemma}
\newtheorem{corollary}[theorem]{Corollary}
\begin{document}
\title{$L^p$ norms of trigonometric polynomials}
\author{Jordan Bell}
\date{\today}

\maketitle

\section{Introduction}
A {\em trigonometric polynomial of degree $n$} is an expression  of the form
\[
\sum_{k=-n}^n c_k e^{ikt}, \qquad c_k \in \mathbb{C}.
\]
Using the identity $e^{it}=\cos t+i\sin t$, we can write a trigonometric polynomial of degree $n$ in the form
\[
a_0+\sum_{k=1}^n a_k \cos kt + \sum_{k=1}^n b_k \sin kt, \qquad a_k, b_k \in \mathbb{C}.
\]

For $1 \leq p < \infty$ and for a $2\pi$-periodic function $f$, we define the $L^p$ norm of $f$ by
\[
\norm{f}_p=\left(\frac{1}{2\pi} \int_0^{2\pi} |f(t)|^p dt \right)^{1/p}.
\]
For a continuous $2\pi$-periodic function $f$, we define the $L^\infty$ norm of $f$ by
\[
\norm{f}_\infty=\max_{0 \leq t \leq 2\pi} |f(t)|.
\] 


If $f$ is a continuous $2\pi$-periodic function, then there is a sequence of trigonometric polynomials $f_n$ such that
$\norm{f-f_n}_\infty \to 0$ as $n \to \infty$
\cite[p.~54, Corollary~5.4]{steinI}. 

If $1 \leq p < \infty$ and $f$ is a continuous $2\pi$-periodic function, then 
\[
\norm{f}_p = \left( \frac{1}{2\pi} \int_0^{2\pi} |f(t)|^p dt \right)^{1/p} \leq \left( \frac{1}{2\pi} \int_0^{2\pi} \norm{f}_\infty^p dt \right)^{1/p}= \norm{f}_\infty.
\]
Jensen's inequality \cite[p.~44, Theorem 2.2]{lieb} (cf. \cite[p.~113, Problem 7.5]{master}) tells us that if
$\phi:[0,\infty) \to \mathbb{R}$ is convex, then for any function $h:[0,2\pi] \to [0,\infty)$ we have
\[
\phi \left( \frac{1}{2\pi} \int_0^{2\pi} h(t) dt \right) \leq \frac{1}{2\pi} \int_0^{2\pi} \phi(h(t)) dt.
\]
If $1 \leq p < q < \infty$, then $\phi:[0,\infty) \to \mathbb{R}$ defined by $\phi(x)=x^{q/p}$ is convex. Hence, if $1 \leq p < q < \infty$ then for any $2\pi$-periodic function $f$,
\begin{eqnarray*}
\norm{f}_p&=&(\phi(\norm{f}_p^p))^{1/q}\\
&=&\left( \phi\left( \frac{1}{2\pi}\int_0^{2\pi} |f(t)|^p dt \right) \right)^{1/q}\\
&\leq&\left( \frac{1}{2\pi} \int_0^{2\pi} \phi(|f(t)|^p) dt \right)^{1/q}\\
&=&\left( \frac{1}{2\pi} \int_0^{2\pi} |f(t)|^q dt \right)^{1/q}\\
&=&\norm{f}_q.
\end{eqnarray*}


The {\em Dirichlet kernel} $D_n$ is defined by
\[
D_n(t)=\sum_{k=-n}^n e^{ikt}=1+2\sum_{k=1}^n \cos kt.
\]
One can show \cite[p.~71, Exercise 1.1]{katznelson} that
\[
\norm{D_n}_1 = \frac{4}{\pi^2}\cdot \log n+O(1).
\]
(On the other hand, it can quickly be seen that $\norm{D_n}_\infty=2n+1$, and it follows from Parseval's identity that
$\norm{D_n}_2=\sqrt{2n+1}$.)

P\'olya and Szeg{\H o} \cite[Part VI]{polya} present various problems about trigonometric polynomials together  with solutions to them.
A result on $L^\infty$ norms of trigonometric polynomials that P\'olya and Szeg{\H o} present is for the sum $A_n(t)=\sum_{k=1}^n \frac{\sin kt}{k}$. The local maxima and local minima of $A_n$ can be
explicitly determined \cite[p.~74, no.~23]{polya}, and 
it can be shown that \cite[p.~74, no.~25]{polya}
\[
\norm{A_n}_\infty \sim
\int_0^\pi \frac{\sin t}{t} dt.
\]

\section{$L^p$ norms}
If $1 \leq p < q < \infty$, then  \cite[p.~123, Exercise~1.8]{katznelson} (cf. \cite[p.~102, Theorem 2.6]{devore}) there is some $C(p,q)$ such that for any trigonometric polynomial $f$ of degree $n$, we
have
\[
\norm{f}_q \leq C(p,q) n^{\frac{1}{p}-\frac{1}{q}} \norm{f}_p.
\]
This inequality is sharp  \cite[p.~230]{timan}: for $1 \leq p < q < \infty$ there is some $C'(p,q)$ such that
if $F_n(t)=\frac{1}{n}\sum_{k=0}^{n-1}D_k(t)$ ($F_n$ is called the {\em Fej\'er kernel})  then
\[
\norm{F_n}_q > C'(p,q) n^{\frac{1}{p}-\frac{1}{q}} \norm{F_n}_p.
\]

\begin{theorem}
Let $1 \leq p \leq q \leq \infty$.
If $\hat{f}(j)=0$ for $|j|>n+1$ then
\[
\norm{f}_q \leq 5(n+1)^{\frac{1}{p}-\frac{1}{q}} \norm{f}_p.
\]
\end{theorem}
\begin{proof}
Let $K_n(t)=\sum_{j=-n}^n \Big(1-\frac{|j|}{n+1}\Big)e^{ijt}$, the Fej\'er kernel.
From this expression we get $|K_n(t)| \leq K_n(0)= n+1$.
 It's straightforward to show
that $K_n(t)=\frac{1}{n+1}\Big(\frac{\sin \frac{n+1}{2}t}{\sin \frac{1}{2}t} \Big)^2$.
Since $\sin \frac{t}{2}>\frac{t}{\pi}$ for $0 < t < \pi$, we get $|K_n(t)| \leq   \frac{\pi^2}{(n+1)t^2}$, and
thus we obtain
\[
|K_n(t)| \leq \min\Big(n+1, \frac{\pi^2}{(n+1)t^2} \Big).
\]

Then, for any $r \geq 1$,
\begin{eqnarray*}
\norm{K_n}_r^r&=&\frac{1}{2\pi} \int_0^{2\pi} |K_n(t)|^r dt\\
&\leq&\frac{1}{2\pi} \int_0^{\frac{\pi}{n+1}} (n+1)^r dt
+\frac{1}{2\pi} \int_{\frac{\pi}{n+1}}^{2\pi} \Big(\frac{\pi^2}{(n+1)t^2}\Big)^r dt\\
&=&\frac{(n+1)^{r-1}}{2} 
+\frac{1}{2}\frac{1}{(n+1)^r}\frac{1}{2r-1}\Big( (n+1)^{2r-1}-\frac{1}{2^{2r-1}}\Big)\\
&\leq&\frac{(n+1)^{r-1}}{2} 
+\frac{1}{2}\frac{1}{(n+1)^r}\frac{1}{2r-1}(n+1)^{2r-1}\\
&\leq&(n+1)^{r-1}.
\end{eqnarray*}
Hence $\norm{K_n}_r \leq (n+1)^{1-\frac{1}{r}}$.

Let $V_n(t)=2K_{2n+1}(t)-K_n(t)$, the \textbf{de la Vall\'ee Poussin kernel}.
Then
\[
\norm{V_n}_r \leq 2\norm{K_{2n+1}}_r+\norm{K_n}_r \leq 2(2n+2)^{1-\frac{1}{r}}+(n+1)^{1-\frac{1}{r}}
\leq 5(n+1)^{1-\frac{1}{r}}.
\]

For $|j| \leq n+1$ we have $\widehat{V_n}(j)=1$, and one thus checks that $V_n * f=f$. Take $\frac{1}{q}+1=\frac{1}{p}+\frac{1}{r}$.
By
Young's inequality we have 
\[
\norm{f}_q=\norm{V_n * f}_q \leq \norm{V_n}_r \norm{f}_p
\leq 5(n+1)^{\frac{1}{p}-\frac{1}{q}} \norm{f}_p.
\]
\end{proof}

Let $X_n=\{a_0+\sum_{k=1}^n a_k \cos kt + b_k \sin kt: a_k, b_k \in \mathbb{R}\}$, the real vector space of real valued trigonometric polynomials of degree $n$, have
norm
\[
\norm{f}_{X_n}=\max \{|a_0|, |a_1|,\ldots,|a_n|, |b_1|,\ldots,|b_n|\}.
\]
Let $Y_{n,p}$ be the same vector space with the $L^p$ norm. 
Ash and Ganzburg \cite{MR1458861} give upper and lower bounds on the operator norm of the map $i:X_n \to Y_{n,p}$ defined by $i(f)=f$.
  
  
Bernstein's inequality \cite[p.~50, Exercise~7.16]{katznelson} states that for $1 \leq p \leq \infty$, if $f$
is a trigonometric polynomial of degree $n$, then
\[
\norm{f'}_p \leq n \norm{f}_p.
\]
In the other direction, if $f \in C^1$ then
\[
\begin{split}
&\frac{1}{2\pi} \int_0^{2\pi} f(s) ds +\frac{1}{2\pi} \int_0^t s f'(s) ds + \frac{1}{2\pi} \int_t^{2\pi} (s-2\pi) f'(s)ds\\
=&\frac{1}{2\pi} \int_0^{2\pi} f(s) ds+\frac{1}{2\pi} \int_0^{2\pi} s f'(s) ds - \int_t^{2\pi} f'(s) ds\\
=&\frac{1}{2\pi} \int_0^{2\pi} f(s) ds + \frac{1}{2\pi}sf(s)\Big|_0^{2\pi}-\frac{1}{2\pi} \int_0^{2\pi} f(s) ds-f(s)\Big|_t^{2\pi}\\
=&f(t).
\end{split}
\]
Hence
\begin{eqnarray*}
|f(t)|&\leq&\frac{1}{2\pi}\int_0^{2\pi} |f(s)|ds+\frac{1}{2\pi}\int_0^t s|f'(s)|ds+\frac{1}{2\pi} \int_t^{2\pi}(2\pi-s)|f'(s)|ds\\
&\leq&\frac{1}{2\pi}\int_0^{2\pi} |f(s)|ds+\int_0^t |f'(s)|ds + \int_t^{2\pi} |f'(s)| ds\\
&=&\norm{f}_1+2\pi\norm{f'}_1,
\end{eqnarray*}
so
\[
\norm{f}_\infty \leq \norm{f}_1+2\pi\norm{f'}_1.
\]
This is an instance of the Sobolev inequality \cite{oh}.

It turns out that for a trigonometric polynomial the mass cannot
be too concentrated. More precisely, the number of nonzero terms of a trigonometric polynomial restricts how concentrated its mass can be.  
Let $d\mu=\frac{dt}{2\pi}$. Thus $\mu([0,2\pi])=1$. 
A result of Tur\'an \cite[p.~89, Lemma 1]{montgomery} states that if $\lambda_1,\ldots,\lambda_N \in \mathbb{Z}$ and
$T(t)=\sum_{n=1}^N b_n e^{i\lambda_n t}$, $b_n \in \mathbb{C}$, then for any closed arc $I \subset [0,2\pi]$,
\[
\norm{T}_\infty \leq \left( \frac{2 e}{\mu(I)} \right)^{N-1} \max_{t \in I} |T(t)|.
\]
Nazarov \cite[p.~452]{havin} shows  that there is some constant $A$ such that if $E$ is a closed subset of $[0,2\pi]$ (not necessarily an arc),
then
\[
\norm{\hat{T}}_1 \leq \left( \frac{A}{\mu(E)} \right)^N \max_{t \in E} |f(T)|.
\]
Nazarov \cite{MR1771766} proves that there exists some constant $C$ such that if $0 \leq q \leq 2$ and $\mu(E)\geq \frac{1}{3}$, then 
\[
\norm{T}_q \leq e^{C(N-1)\left(1-\frac{\mu(E)}{2\pi}\right)} \left( \frac{1}{2\pi} \int_E |T(t)|^q dt \right)^{1/q}.
\]
These results of Turan and Nazarov are examples of the {\em uncertainty principle} \cite{MR1448337}, which is the general principle that a constrain on the support of the Fourier
transform of a function constrains the support of the function itself. 

In \cite{MR0028445}, Hardy and Littlewood present inequalities for norms of $2\pi$-periodic functions in terms of certain series
formed from their Fourier coefficients. Let $c_k \in \mathbb{C}$, $k \in \mathbb{Z}$, be such that $c_k \to 0$ as $k \to \pm \infty$, and
define
  $c_0^*,c_1^*,c_{-1}^*,c_2^*,c_{-2}^*,\ldots$
to be the absolute values of the $c_k$ ordered in decreasing magnitude. For real $r>1$, define
\[
S_r^*(c)=\left( \sum_{k=-\infty}^\infty {c_k^*}^r (|k|+1)^{r-2} \right)^{1/r}.
\]
For instance, if $c_k=1$ for $-N \leq k \leq N$ and $c_k=0$ for $|k|>N$, then $S_r^*(c)=\left(1+2\sum_{k=2}^{N+1} k^{r-2} \right)^{1/r}$. 
Hardy and Littlewood state the result \cite[p.~164, Theorem 2]{MR0028445} that if $1<p \leq 2$ then there is some constant $A(p)$ such that for any sequence $c$, with $c_k \to 0$ as $k \to \pm \infty$,
if $f(t)=\sum_{k=-\infty}^\infty c_k e^{ikt}$ and $\norm{f}_p < \infty$ then
\[
S_p^*(c) \leq A(p)\norm{f}_p.
\]
A proof of this is given in Zygmund \cite[vol.~II, p.~128, chap.~XII, Theorem~6.3]{zygmund}.
Asking if this inequality holds for $p=1$ suggests the following question that Hardy and Littlewood pose at the end of their paper \cite[p.~168]{MR0028445}: Is there a constant $A$ such that
for all distinct positive integers $m_k, k=1,\ldots,N$, we have
\[
\norm{\sum_{k=1}^N \cos m_k t}_1 > A \log N?
\]
McGehee, Pigno and Smith \cite{MR621019} prove that there is some $K$ such that for all $N$, if
$n_1,\ldots,n_N$ are distinct integers and $c_1,\ldots,c_N \in \mathbb{C}$ satisfy $|c_k| \geq 1$, then
\[
\norm{\sum_{k=1}^N c_k e^{in_k t}}_1 > K \log N.
\]
Thus
\[
\norm{\sum_{k=1}^N \cos m_k t}_1 =\frac{1}{2}\cdot \norm{\sum_{k=1}^N e^{im_kt}+e^{-im_kt}}_1
\geq \frac{1}{2}\cdot K\log(2N).
\]





For $k \geq 2$, define $T_N(t)=\sum_{n=1}^N e^{in^k t}$. Since $\norm{T_N}_\infty=N$, for each $p \geq 1$ we have $\norm{T_N}_p \leq N$.  Hua's lemma \cite[p.~116, Theorem~4.6]{nathanson} states that if $\epsilon>0$, then 
\[
\norm{T_N}_{2^k}=O\left( N^{1-\frac{k}{2^k}+\epsilon} \right).
\]
Hua's lemma is used in additive number theory. The number of sets of integer solutions of the equation
\[
f(x_1,\ldots,x_n)=N, \qquad a_r \leq x_r \leq b_r
\]
is equal to (cf. \cite[p.~151]{hua})
\[
\sum_{a_1 \leq x_1 \leq b_1} \cdots \sum_{a_n \leq x_n \leq b_n} \int_0^1 e^{2\pi i(f(x_1,\ldots,x_n)-N)t} dt.
\]







Borwein and Lockhart \cite{MR1814174}: what is the expected $L^p$ norm of a trigonometric polynomial of order $n$?
Kahane \cite[Chapter~6]{kahane} also presents material on random trigonometric polynomials. 







Nursultanov and Tikhonov \cite{MR3078275}: the sup on a subset of $\mathbb{T}$ of a trigonometric polynomial $f$ of degree $n$  being lower
bounded in terms of $\norm{f}_\infty$, $n$,  and the measure of the subset. 

\section{$\ell^p$ norms}
For a $2\pi$-periodic function $f$, we define $\hat{f}:\mathbb{Z} \to \mathbb{C}$ by
\[
\hat{f}(k)=\frac{1}{2\pi} \int_0^{2\pi} e^{-ikt} f(t) dt.
\]
For $1 \leq p < \infty$, we define the $\ell^p$ norm of $\hat{f}$ by 
\[
\norm{\hat{f}}_p=\left( \sum_{k=-\infty}^\infty |\hat{f}(k)|^p \right)^{1/p},
\]
and we define the $\ell^\infty$ norm of $\hat{f}$ by 
\[
\norm{\hat{f}}_\infty=\max_{k \in \mathbb{Z}} |\hat{f}(k)|.
\]

Parseval's identity \cite[p.~80, Theorem~1.3]{steinI} states that $\norm{f}_2=\norm{\hat{f}}_2$.




If $1 \leq p < \infty$, then
\[
\norm{\hat{f}}_\infty \leq \left( \cdots+\norm{\hat{f}}_\infty^p + \cdots \right)^{1/p}= \norm{\hat{f}}_p.
\]
If $1 \leq p < q < \infty$, then, since for each $k$, $\frac{|\hat{f}(k)|}{\norm{\hat{f}}_q} \leq 1$,
\[
1=\left( \sum_{k=-\infty}^\infty \left( \frac{|\hat{f}(k)|}{\norm{\hat{f}}_q} \right)^q \right)^{1/q} \leq
\left( \sum_{k=-\infty}^\infty \left( \frac{|\hat{f}(k)|}{\norm{\hat{f}}_q} \right)^p \right)^{1/q} 
=\frac{\norm{\hat{f}}_p^{p/q}}{\norm{\hat{f}}_q^{p/q}}.
\]
Hence for $1 \leq p < p \leq \infty$,
\[
\norm{\hat{f}}_q \leq \norm{\hat{f}}_p.
\]

For $1 \leq p < \infty$, if $f$ is a trigonometric polynomial of degree $n$ then 
\[
\norm{\hat{f}}_p 
=\left( \sum_{k=-n}^n |\hat{f}(k)|^p \right)^{1/p} \leq \left( \sum_{k=-n}^n \norm{\hat{f}}_\infty^p \right)^{1/p}
=(2n+1)^{1/p} \norm{\hat{f}}_\infty.
\]

For $1 \leq p < q < \infty$, we have \cite[p.~123, Problem~8.3]{master} (this is  Jensen's inequality for sums)
\[
\left( \sum_{k=-n}^n \frac{1}{2n+1} |\hat{f}(k)|^p \right)^{1/p} \leq \left( \sum_{k=-n}^n \frac{1}{2n+1} |\hat{f}(k)|^q \right)^{1/q},
\]
i.e.
\[
(2n+1)^{-1/p}  \norm{\hat{f}}_p \leq (2n+1)^{-1/q} \norm{\hat{f}}_q.
\]
Hence for $1 < p < q < \infty$,
\[
\norm{\hat{f}}_p \leq (2n+1)^{\frac{1}{p}-\frac{1}{q}} \norm{\hat{f}}_q.
\]


For any $t$,
\[
|f(t)|=\left| \sum_{k=-\infty}^\infty \hat{f}(k)e^{ikt} \right| \leq \sum_{k=-\infty}^\infty |\hat{f}(k)e^{ikt}|=\sum_{k=-\infty}^\infty |\hat{f}(k)|=\norm{\hat{f}}_1.
\]
Hence
\[
\norm{f}_\infty \leq \norm{\hat{f}}_1.
\]








For any $k \in \mathbb{Z}$,
\[
|\hat{f}(k)| = \left| \frac{1}{2\pi} \int_0^{2\pi} e^{-ikt} f(t) dt \right|
\leq \frac{1}{2\pi} \int_0^{2\pi} |f(t)| dt = \norm{f}_1.
\]
Hence
\[
\norm{\hat{f}}_\infty \leq \norm{f}_1.
\]
The Hausdorff-Young inequality \cite[p.~57, Corollary~2.4]{steinIV} states that for $1 \leq p \leq 2$ and $\frac{1}{p}+\frac{1}{q}=1$, if $f \in L^p$ then
\[
\norm{\hat{f}}_q \leq \norm{f}_p.
\]
The dual Hausdorff-Young inequality \cite[p.~58, Corollary~2.5]{steinIV} states that for $1 \leq p \leq 2$ and $\frac{1}{p}+\frac{1}{q}=1$, if $f \in L^q$ then
\[
\norm{f}_q \leq \norm{\hat{f}}_q.
\]
A survey on the Hausdorff-Young inequality is given in \cite{MR1348739}) 





For $M+1 \leq k \leq M+N$, let $a_k \in \mathbb{C}$ and let
$S(t)=\sum_{k=M+1}^{N+1} a_k e^{ikt}$. Let $t_1,\ldots,t_R \in \mathbb{R}$, and let $\delta$ be such that
if $r \neq s$ then
\[
\norm{t_r-t_s} \geq \delta,
\]
where $\norm{t}=\min_k | t-k|$ is the distance from $t$ to a nearest integer. 
{\em The large sieve} \cite{sieve} is an inequality of the form
\[
\sum_{r=1}^R |S(2\pi t_r)|^2 \leq \Delta(N,\delta) \sum_{k=M+1}^{M+N} |a_k|^2.
\]
A result of Selberg \cite[p.~559, Theorem~3]{sieve} shows that the large sieve is valid for $\Delta=N-1+\delta^{-1}$.

Kristiansen \cite{MR0340916} 

Boas \cite{MR0023366}






For $F:\mathbb{Z}/n \to \mathbb{C}$, its Fourier transform $\hat{F}:\mathbb{Z}/n \to \mathbb{C}$ (called the 
{\em discrete Fourier transform}) is defined by
\[
\hat{F}(k)=\frac{1}{n} \sum_{j=0}^{n-1} F(j) e^{-2\pi ijk/n}, \qquad 0 \leq k \leq n-1,
\]
and one can prove \cite[p.~223, Theorem~1.2]{steinI} that 
\[
F(j)=\sum_{k=0}^{n-1} \hat{F}(k) e^{2\pi ikj/N}, \qquad 0 \leq j \leq n-1.
\]
One can also prove Parseval's identity for the Fourier transform on $\mathbb{Z}/n$ \cite[p.~223, Theorem ~1.2]{steinI}. It states 
\[
\sum_{k=0}^{n-1} |\hat{F}(k)|^2=\frac{1}{n}\sum_{j=0}^{n-1} |F(j)|^2.
\]

Let $P(t)=\sum_{k=0}^{n-1} a_k e^{ikt}$. Define $F:\mathbb{Z}/n \to \mathbb{C}$ by
\[
F(j)=\sum_{k=0}^{n-1} a_k e^{2\pi ikj/n}, \qquad 0 \leq j \leq n-1.
\]
(That is, $\hat{F}(k)=a_k$.)
We then have 
\[
\sum_{k=0}^{n-1} |a_k|^2=\frac{1}{n}\sum_{j=0}^{n-1} |F(j)|^2 = \frac{1}{n} \sum_{j=0}^{n-1} \big|P\Big(\frac{2\pi j}{n}\Big)\big|^2.
\]
Thus
\[
\norm{P}_2=\left( \frac{1}{n}  \sum_{j=0}^{n-1} \big|P\Big(\frac{2\pi j}{n}\Big)\big|^2 \right)^{1/2}.
\]
The Marcinkiewicz-Zygmund inequalities \cite[vol.~II, p.~28, chap.~X, Theorem~7.5]{zygmund} state that there is a constant $A$ such that 
for $1 \leq p \leq \infty$, if $f$ is a trigonometric polynomial of degree $n$ then
\[
\left( \frac{1}{2n+1} \sum_{k=0}^{2n} \big|f\Big(\frac{2\pi k}{2n+1}\Big)\big|^p \right)^{1/p} \leq A(2\pi)^{1/p} \norm{f}_p,
\]
and for each $1 < p < \infty$ there exists some $A_p$ such that if $f$ is a trigonometric polynomial of degree $n$ then
\[
\norm{f}_p \leq A_p \left( \frac{1}{2n+1} \sum_{k=0}^{2n} \big|f\Big(\frac{2\pi k}{2n+1}\Big)\big|^p \right)^{1/p}.
\]

M{\'a}t{\'e} and Nevai \cite[p.~148, Theorem~6]{MR558399} prove that for $p>0$, if $S_n$ is a trigonometric polynomial of degree $n$ then
\[
\norm{S_n}_\infty \leq \left( \frac{(1+np)e}{2} \right)^{1/p} \norm{S_n}_p.
\]
M{\'a}t{\'e} and Nevai \cite{MR558399} prove a version of Bernstein's inequality for $0<p<1$, and their result
can be sharpened to the following \cite{MR1016168}: For $0<p<1$, if $T_n$ is a trigonometric polynomial
of order $n$ then
\[
\norm{T_n'}_p \leq n \norm{T_n}_p.
\]

Let $\supp \hat{f}=\{k \in \mathbb{Z}: \hat{f}(k) \neq 0\}$. A subset $\Lambda$ of $\mathbb{Z}$ is called a {\em Sidon set}  \cite[p.~121, \S 5.7.2]{groups} if there exists a constant $B$
such that for every trigonometric polynomial $f$ with $\supp \hat{f} \subseteq \Lambda$ we have
\[
\norm{\hat{f}}_1 \leq B \norm{f}_\infty.
\]
Let $B(\Lambda)$ be the least such $B$.
A sequence of positive integers $\lambda_k$ is said to be {\em lacunary} if there is a constant $\rho$ such that $\lambda_{k+1} > \rho \lambda_k$ for all
$k$. If $\lambda_k$ is a lacunary sequence, then $\{\lambda_k\}$ is a Sidon set \cite[p.~154, Corollary~6.17]{schlag}. 
If $\Lambda \subset \mathbb{Z}$ is a Sidon set, then \cite[p.~128, Theorem~5.7.7]{groups} (cf. \cite[p.~157, Corollary~6.19]{schlag}) for any $2 < p < \infty$,
for every trigonometric polynomial $f$ with $\supp \hat{f} \subseteq \Lambda$ we have 
\[
\norm{f}_p \leq B(\Lambda) \sqrt{p} \norm{f}_2,
\]
and
\[
\norm{f}_2 \leq 2B(\Lambda) \norm{f}_1.
\]

Let $0< p <\infty$. A subset $E$ of $\mathbb{Z}$ is called a {\em $\Lambda(p)$-set} if 
for every $0<r<p$ there is some $A(E,p)$ such that for all trigonometric polynomials $f$ with $\supp \hat{f} \subset E$ we have
\[
\norm{f}_p \leq A(E,p) \norm{f}_2.
\]
$\Lambda(p)$ sets were introduced by Rudin, and he discusses them in his autobiography \cite[Chapter~28]{MR1413303}. 
A modern survey of $\Lambda(p)$-sets is given by Bourgain \cite{bourgain}.

Bochkarev \cite{MR2301609} proves various lower bounds on the $L^1$ norms of certain trigonometric polynomials.
Let $c_k \in \mathbb{C}$, $k \geq 1$. 
If there are constants $A$ and $B$ such that
\[
A\frac{(\log k)^s}{\sqrt{k}} \leq |c_k| \leq B\frac{(\log k)^s}{\sqrt{k}}, \qquad k \geq 1,
\]
then \cite[p.~58, Theorem~19]{MR2301609}
\[
\norm{\sum_{k=1}^n c_k e^{ik^2 t}}_1 \gg 
\begin{cases}
(\log n)^{s-\frac{1}{2}},&s>\frac{1}{2},\\
\log \log n,&s=\frac{1}{2}.
\end{cases}
\]

If $P(t)=\sum_{k=0}^n a_k e^{ikt}$ with $a_k \in \{-1,1\}$, then by the Cauchy-Schwarz inequality and Parseval's identity we have
\[
\norm{P}_1=\frac{1}{2\pi}\int_0^{2\pi} 1 \cdot |P(t)| dt \leq \norm{1}_2\cdot \norm{P}_2 = 1\cdot \norm{\hat{P}}_2 = \sqrt{n+1}.
\]
Newman \cite{newman} shows that in fact we can do better than what we get using the Cauchy-Schwarz inequality and Parseval's identity:
\[
\norm{P}_1 < \sqrt{n+0.97}.
\]

A {\em Fekete polynomial} is a polynomial of the form $\sum_{k=1}^{l-1} \left( \frac{k}{l} \right) z^k$, $l$ prime, where
$\left( \frac{k}{l} \right)$ is the Legendre symbol. Let $P_l(t)=\sum_{k=1}^{l-1} \left( \frac{k}{l} \right) e^{ikt}$. Erd{\'e}lyi \cite{MR2899817}
proves upper and lower bounds on $\left( \frac{1}{|I|} \int_I |P_l(t)|^q dt \right)^{1/q}$, $q>0$, where $I$ is an arc in $[0,2\pi]$.

\bibliographystyle{plain}
\bibliography{Lp}

\end{document}
