\documentclass{article}
\usepackage{amsmath,amssymb,mathrsfs,amsthm}
%\usepackage{tikz-cd}
\usepackage{hyperref}
\newcommand{\inner}[2]{\left\langle #1, #2 \right\rangle}
\newcommand{\tr}{\ensuremath\mathrm{tr}\,} 
\newcommand{\Span}{\ensuremath\mathrm{span}} 
\def\Re{\ensuremath{\mathrm{Re}}\,}
\def\Im{\ensuremath{\mathrm{Im}}\,}
\newcommand{\id}{\ensuremath\mathrm{id}} 
\newcommand{\diam}{\ensuremath\mathrm{diam}} 
\newcommand{\lcm}{\ensuremath\mathrm{lcm}} 
\newcommand{\supp}{\ensuremath\mathrm{supp}\,}
\newcommand{\dom}{\ensuremath\mathrm{dom}\,}
\newcommand{\upto}{\nearrow}
\newcommand{\downto}{\searrow}
\newcommand{\norm}[1]{\left\Vert #1 \right\Vert}
\newtheorem{theorem}{Theorem}
\newtheorem{lemma}[theorem]{Lemma}
\newtheorem{proposition}[theorem]{Proposition}
\newtheorem{corollary}[theorem]{Corollary}
\theoremstyle{definition}
\newtheorem{definition}[theorem]{Definition}
\newtheorem{example}[theorem]{Example}
\begin{document}
\title{The Kolmogorov continuity theorem, H\"older continuity, and the Kolmogorov-Chentsov theorem}
\author{Jordan Bell\\ \texttt{jordan.bell@gmail.com}\\Department of Mathematics, University of Toronto}
\date{\today}

\maketitle

\section{Modifications}
Let $(\Omega,\mathscr{F},P)$ be a probability space, let $I$ be a nonempty set, and let
$(E,\mathscr{E})$ be a measurable space. A \textbf{stochastic process with index set $I$ and state space $E$} is a family 
$(X_t)_{t \in I}$ of random variables $X_t:(\Omega,\mathscr{F}) \to (E,\mathscr{E})$.
If $X$ and $Y$ are stochastic processes, we say that \textbf{$X$ is a modification of $Y$} if 
for each $t \in I$, 
\[
P\{\omega \in \Omega: X_t(\omega)=Y_t(\omega)\}=1.
\]

\begin{lemma}
If $X$ is a modification of $Y$, then $X$ and $Y$ have the same finite-dimensional distributions.
\end{lemma}
\begin{proof}
For $t_1,\ldots,t_n \in  I$, let $A_i \in \mathscr{E}$ for each $1 \leq i \leq n$, and let
\[
A = \bigcap_{i=1}^n X_{t_i}^{-1}(A_i) \in \mathscr{F}, \qquad B = \bigcap_{i=1}^n Y_{t_i}^{-1}(A_i) \in \mathscr{F}.
\]
If $\omega \in A \setminus B$ then there is some $i$ for which $\omega \not \in Y_{t_i}^{-1}(A_i)$, and
$\omega \in X_{t_i}^{-1}(A_i)$ so $X_{t_i}(\omega) \neq Y_{t_i}(\omega)$. Therefore
\[
A \triangle B \subset \bigcup_{i=1}^n \{\omega \in \Omega: X_{t_i}(\omega) \neq Y_{t_i}(\omega)\}.
\]
Because $X$ is a modification of $Y$, the right-hand side is a union of finitely many $P$-null sets,
hence is itself a $P$-null set.  $A$ and $B$ each belong to $\mathscr{F}$, so
$P(A \triangle B)=0$.\footnote{We have not assumed that $(\Omega,\mathscr{F},P)$ is a complete measure space, so we must verify
that a set is measurable before speaking about its measure.}
Because $P(A \triangle B)=0$, $P(A)=P(B)$, i.e.
\[
P(X_{t_1} \in A_1, \ldots, X_{t_n} \in A_n) = P(Y_{t_1} \in A_1, \ldots, Y_{t_n} \in A_n).
\]
This implies that\footnote{\url{http://individual.utoronto.ca/jordanbell/notes/finitedimdistributions.pdf}}
\[
P_*(X_{t_1} \otimes \cdots \otimes X_{t_n}) = P_*(Y_{t_1} \otimes \cdots \otimes Y_{t_n}),
\]
namely,  $X$ and $Y$ have the same finite-dimensional distributions.
\end{proof}


\section{Continuous modifications}
Let $E$ be a Polish space with Borel $\sigma$-algebra $\mathscr{E}$. 
A stochastic process $(X_t)_{t \in \mathbb{R}_{\geq 0}}$ is called \textbf{continuous} if for each $\omega \in \Omega$, the \textbf{path}
$t \mapsto X_t(\omega)$ is continuous $\mathbb{R}_{\geq 0} \to E$. 

A \textbf{dyadic rational} is an element of
\[
D=\bigcup_{i=0}^\infty 2^{-i} \mathbb{Z}.
\]


The \textbf{Kolmogorov continuity theorem} gives conditions under which a stochastic process whose state space is a Polish space 
has a continuous modification.\footnote{Heinz
Bauer, {\em Probability Theory}, p.~335, Theorem 39.3. It was only after working through the proof given by Bauer that I realized that the statement is true when the state space is a Polish space rather than merely $\mathbb{R}^d$. In the proof I do not use that $|\cdot|$ is a norm on $\mathbb{R}^d$, and only use that $d(x,y)=|x-y|$ is a metric on $\mathbb{R}^d$,
so it is straightforward to rewrite the proof.}
This is like the \textbf{Sobolev lemma},\footnote{Walter Rudin, {\em Functional Analysis}, second ed.,
p.~202, Theorem 7.25.} which states that if
$f \in H^s(\mathbb{R}^d)$ and
$s>k+\frac{d}{2}$, then there is some $\phi \in C^k(\mathbb{R}^d)$
such that $f=\phi$ almost everywhere. It does not make sense to say that an element of a Sobolev space is itself $C^k$, because elements of
Sobolev
spaces are equivalence classes of functions, but it does make sense to say that there is a $C^k$ version of this element.

\begin{theorem}[Kolmogorov continuity theorem]
Suppose that $(\Omega,\mathscr{F},P,(X_t)_{t \in \mathbb{R}_{\geq 0}})$ is a stochastic process with state space
$\mathbb{R}^d$. If there are $\alpha,\beta,c>0$ such that 
\begin{equation}
E(|X_t-X_s|^\alpha) \leq c | t-s|^{1+\beta}, \qquad s,t \in \mathbb{R}_{\geq 0},
\label{kolmogorov}
\end{equation}
then the stochastic process has a continuous modification that itself satisfies \eqref{kolmogorov}.
\end{theorem}
\begin{proof}
Let $0<\gamma<\frac{\beta}{\alpha}$ and let
\[
\delta = \beta - \alpha \gamma > 0.
\]
For $m \geq 1$, let $S_m$ be the set of all pairs $(s,t)$ with
\[
s,t \in 
\{j2^{-m}: 0 \leq j \leq 2^m\},
\]
and $|s-t|=2^{-m}$. There are $2\cdot 2^m$ such pairs, i.e. $|S_m|=2\cdot 2^m$. 
Let
\[
A_m = \bigcup_{(s,t) \in S_m} \{|X_s-X_t| \geq 2^{-\gamma m}\} \in \mathscr{F}.
\]
For $(s,t) \in S_m$,
using Chebyshev's inequality and \eqref{kolmogorov} we get
\begin{align*}
P(|X_t-X_s| \geq 2^{-\gamma m})&\leq (2^{\gamma m})^\alpha E(|X_t-X_s|^\alpha)\\
&\leq 2^{\alpha \gamma m} \cdot c|t-s|^{1+\beta}\\
&= c2^{\alpha \gamma m} 2^{-m(1+\beta)}\\
&< c2^{-m-\delta m}.
\end{align*}
Hence
\[
P(A_m) \leq \sum_{(s,t) \in S_m} P \{|X_s-X_t| \geq 2^{-\gamma m}\}
<\sum_{(s,t) \in S_m} c 2^{-m-\delta m}
=2c\cdot 2^{-\delta m}.
\]
Because $\sum_m P(A_m)<\infty$, the Borel-Cantelli lemma tells us that
\[
P\left(\bigcap_{n=1}^\infty \bigcup_{m=n}^\infty A_m \right)=P(N_0)=0,
\]
where for  each $\omega \in \Omega \setminus N_0$ there is some $m_0(\omega)$ such that
$\omega \not \in A_m$ when $m \geq m_0(\omega)$. That is,
for $\omega \in \Omega \setminus N_0$ 
there is some $m_0(\omega)$ such that  
\begin{equation}
|X_t(\omega)-X_s(\omega)| < 2^{-\gamma m},
\qquad m \geq m_0(\omega), \quad (s,t) \in S_m.
\label{393}
\end{equation}

Now let $\omega \in \Omega \setminus N_0$ and let $s,t \in [0,1]$ be dyadic rationals satisfying
\[
0<|s-t| \leq 2^{-m_0(\omega)}.
\]
Let $m=m(s,t)$ be the greatest integer  such that $|s-t| \leq 2^{-m}$:  
\begin{equation}
2^{-m-1} < |s-t| \leq 2^{-m},
\label{maximalm}
\end{equation}
which implies that $m \geq m_0(\omega)$.
There are some $i_0, j_0 \in \{0,1,2,3,\ldots,2^m\}$ such that 
\[
s_0 = i_0 2^{-m} \leq s < (i_0+1)2^{-m}, \qquad t_0= j_0 2^{-m} \leq t < (j_0+1)2^{-m}.
\]
As $0 \leq s-s_0 < 2^{-m}$ and $0 \leq t-t_0 < 2^{-m}$, there are 
sequences $\sigma_j, \tau_j \in \{0,1\}$, $j>m$, each of which have cofinitely many zero entries, such that
\[
s=s_0+\sum_{j > m} \sigma_j 2^{-j}, \qquad t=t_0+ \sum_{j > m} \tau_j 2^{-j}.
\]
Because $0\leq s-s_0<2^{-m}$ and $\leq t-t_0<2^{-m}$,
\[
2^{-m} > |(s-s_0)-(t-t_0)|
=|(s-t)-(s_0-t_0)|
\geq |s_0-t_0|-|s-t|,
\]
and with \eqref{maximalm},
\[
|s_0-t_0| < 2^{-m}+|s-t| \leq 2^{-m}+2^{-m} =2^{-m+1}.
\] 
Thus $|i_0-j_0| < 2$, so
$|i_0-j_0| \in \{0,1\}$ and so either
$s_0=t_0$ or $(s_0,t_0) \in S_m$. 
In the first case, $|X_{t_0}(\omega)-X_{s_0}(\omega)|=0$.
 In the second case, since $m \geq m_0(\omega)$,
by \eqref{393} we have
\begin{equation}
|X_{t_0}(\omega)-X_{s_0}(\omega)| < 2^{-\gamma m} .
\label{396}
\end{equation}
Define by induction
\[
s_n = s_{n-1}+\sigma_{m+n}2^{-(m+n)}, \qquad n \geq 1,
\]
i.e.
\[
s_n = s_0 + \sum_{m<j \leq m+n} \sigma_j 2^{-j}.
\]
For each $n \geq 1$,  $s_n-s_{n-1} \in \{0,2^{-(m+n)}\}$, so either
$s_n=s_{n-1}$ or $(s_{n-1},s_n) \in S_{m+n}$, and because
 $m+n \geq m+1>m_0(\omega)$, applying \eqref{393} yields
\[
|X_{s_n}(\omega)-X_{s_{n-1}}(\omega)| < 2^{-\gamma(m+n)}.
\]
Because the sequence $\sigma_j$ is eventually equal to $0$, the sequence $s_n$ is eventually equal to $s$.  Thus
\[
\sum_{n=1}^\infty (X_{s_n}(\omega)-X_{s_{n-1}}(\omega))
=X_s(\omega)-X_{s_0}(\omega),
\]
whence
\[
|X_s(\omega)-X_{s_0}(\omega)| \leq \sum_{n=1}^\infty |X_{s_n}(\omega)-X_{s_{n-1}}(\omega)|
<\sum_{n=1}^\infty 2^{-\gamma(m+n)}
=\frac{2^{-\gamma (m+1)}}{1-2^{-\gamma}}.
\]
By the same reasoning we get 
\[
|X_t(\omega)-X_{t_0}(\omega)| < \frac{2^{-\gamma (m+1)}}{1-2^{-\gamma}}.
\]
Using these and \eqref{396} yields
\begin{align*}
|X_t(\omega)-X_s(\omega)|&\leq |X_t(\omega)-X_{t_0}(\omega)|
+|X_{t_0}(\omega)-X_{s_0}(\omega)|+|X_s(\omega)-X_{s_0}(\omega)|\\
&<\frac{2^{-\gamma (m+1)}}{1-2^{-\gamma}}+2^{-\gamma m}+\frac{2^{-\gamma (m+1)}}{1-2^{-\gamma}}\\
&=C\cdot 2^{-\gamma(m+1)},
\end{align*}
for $C=2^\gamma+\frac{2}{1-2^{-\gamma}}$. 
By \eqref{maximalm}, $2^{-(m+1)}<|t-s|$,
hence
\begin{equation}
|X_t(\omega)-X_s(\omega)| \leq C |t-s|^\gamma.
\label{398}
\end{equation}
This is true for all dyadic rationals $s,t \in [0,1]$ with $|s-t| \leq 2^{-m_0(\omega)}$; when $|s-t|=0$ it is immediate.

For $k \geq 1$, let $X^k_t=X_{k+t}$, which satisfies \eqref{kolmogorov}. 
By what we have worked out, 
there is a $P$-null set $N_1' \in \mathscr{F}$ such that for each
$\omega \in \Omega \setminus N_1'$ there is some $m_1'(\omega)$ such that
$m \geq m_1'(\omega)$ and $(s,t) \in S_m$ imply that $|X^1_t(\omega)-X^1_s(\omega)|
<2^{-\gamma m}$. 
Let $N_1=N_0 \cup N_1'$, which is $P$-null, and for
$\omega \in \Omega \setminus N_1$ let $m_1(\omega)=\max\{m_0(\omega),m_1'(\omega)\}$. 
For $s,t \in D \cap [0,1]$ with $|s-t| \leq 2^{-m_1(\omega)}$, what we have worked out yields
\[
|X_t(\omega)-X_s(\omega)| \leq C|t-s|^\gamma, \qquad 
|X^1_t(\omega)-X^1_s(\omega)| \leq C|t-s|^\gamma.
\]
By induction, we get that for each $k \geq 1$ there are $P$-null sets $N_0 \subset N_1 \subset \cdots \subset N_k$ and for
each $\omega \in \Omega \setminus N_k$ there is some $m_k(\omega)$ such that
for $s,t \in D \cap [0,1]$ with $|s-t| \leq 2^{-m_k(\omega)}$,
\begin{align*}
|X_t(\omega)-X_s(\omega)| &\leq C|t-s|^\gamma\\
|X^1_t(\omega)-X^1_s(\omega)|&\leq C|t-s|^\gamma\\
&\ldots\\
|X^k_t(\omega)-X^k_s(\omega)|&\leq C|t-s|^\gamma.
\end{align*}
Let
\[
N_\gamma=\bigcup_{k \geq 1} N_k,
\]
 which is an increasing sequence of sets whose union is $P$-null. For $\omega \in \Omega \setminus N_\gamma$,
there is a nondecreasing sequence $m_k(\omega)$ such that when
$0 \leq j \leq k$ and  $s,t \in D \cap [j,j+1]$ with
$|s-t| \leq 2^{-m_k(\omega)}$, it is the case that $|X_t(\omega)-X_s(\omega)| \leq C|t-s|^\gamma$. 
For $s,t \in D \cap [0,k+1]$ with $|s-t| \leq 2^{-m_k(\omega)}$, 
because $|s-t| \leq \frac{1}{2}$, either there is some $0 \leq j \leq k$ for which 
$s,t \in [j,j+1]$ or there is some $1 \leq j \leq k$ for which, say, $s<j<t$. In the first case,
$|X_t(\omega)-X_s(\omega)| \leq C|t-s|^\gamma$.  In the second case, 
because $|j-s|<|t-s| \leq 2^{-m_k(\omega)}$ and $|t-j| < |t-s| \leq 2^{-m_k(\omega)}$, we get, because $s,j \in D \cap [j-1,j]$
and
$j,t \in D \cap [j,j+1]$,
\begin{align*}
|X_t(\omega)-X_s(\omega)|&\leq |X_t(\omega)-X_j(\omega)| + |X_j(\omega)-X_s(\omega)|\\
&\leq C|t-j|^\gamma+C|j-s|^\gamma\\
&< 2C|t-s|^\gamma. 
\end{align*}
Thus for
\[
C_\gamma=2C =  2^{\gamma+1}+\frac{4}{1-2^{-\gamma}},
\]
we have established that for $\omega \in \Omega \setminus N_\gamma$, $k \geq 1$, and $s,t \in D \cap [0,k+1]$
satisfying $|t-s| \leq 2^{-m_k(\omega)}$, it is the case that
\begin{equation}
|X_t(\omega)-X_s(\omega)| \leq C_\gamma |t-s|^\gamma.
\label{399}
\end{equation}
This implies that for each $\omega \in \Omega \setminus N_\gamma$ and for $k \geq 1$,  the mapping
$t \mapsto X_t(\omega)$ is uniformly continuous on $D \cap [0,k+1]$. 
For $t \in \mathbb{R}_{\geq 0}$ and $\omega \in \Omega\setminus N_\gamma$, define
\begin{equation}
Y_t(\omega) = \lim_{\stackrel{s \to t}{s \in D}} X_s(\omega).
\label{3910}
\end{equation}
For each $k \geq 0$,
because $t \mapsto X_t(\omega)$ is uniformly continuous $D \cap  [0,k+1] \to \mathbb{R}^d$, 
where
$D \cap [0,k+1]$  is dense in $[0,k+1]$ and $\mathbb{R}^d$ is a complete metric space,
the map $t \mapsto Y_t(\omega)$ is uniformly continuous $[0,k+1] \to \mathbb{R}^d$.\footnote{Charalambos D. Aliprantis
and Kim C. Border, {\em Infinite Dimensional Analysis: A Hitchhiker's Guide}, third ed., p.~77, Lemma 3.11.}
Then $t \mapsto Y_t(\omega)$ is continuous $\mathbb{R}_{\geq 0} \to \mathbb{R}^d$.
For $\omega \in N_\gamma$, we define
\[
Y_t(\omega) = 0,\qquad t \in \mathbb{R}_{\geq 0}.
\]
Then for each $\omega \in \Omega$, $t \mapsto Y_t(\omega)$ is continuous $\mathbb{R}_{\geq 0} \to \mathbb{R}^d$.
For $t \in \mathbb{R}_{\geq 0}$, $\omega \mapsto Y_t(\omega)$ is the pointwise limit of the sequence of mappings
$\omega \mapsto X_s(\omega)$ as $s \to t$, $s \in D$.  For each $s \in D$, $\omega \mapsto X_s(\omega)$ is measurable
$\mathscr{F} \to \mathscr{B}_{\mathbb{R}^d}$, which implies that $\omega \mapsto Y_t(\omega)$ is itself measurable 
$\mathscr{F} \to \mathscr{B}_{\mathbb{R}^d}$.\footnote{Charalambos D. Aliprantis
and Kim C. Border, {\em Infinite Dimensional Analysis: A Hitchhiker's Guide}, third ed., p.~142, Lemma 4.29.}
 Namely, $(Y_t)_{t \in \mathbb{R}_{\geq 0}}$ is a continuous stochastic process.

We must show that $Y$ is a modification of $X$. For $s \in D$, for all $\omega \in \Omega \setminus N_\gamma$ we have
$Y_s(\omega) = X_s(\omega)$. For $t \in \mathbb{R}_{\geq 0}$, there is a sequence $s_n \in D$ tending to $t$, and then
for all $\omega \in \Omega \setminus N_\gamma$  by \eqref{3910} we have $X_{s_n}(\omega) \to Y_t(\omega)$. 
$P(N_\gamma)=0$, namely, $X_{s_n}$ converges to $Y_t$ almost surely. Because $X_{s_n}$ converges to $Y_t$
almost surely and $P$ is a probability measure, $X_{s_n}$ converges in measure to $Y_t$.\footnote{Charalambos D. Aliprantis
and Kim C. Border, {\em Infinite Dimensional Analysis: A Hitchhiker's Guide}, third ed., p.~479, Theorem 13.37.}
On the other hand, 
for
$\eta>0$, by  Chebyshev's inequality and \eqref{kolmogorov},
\[
P\{|X_{s_n}-X_t| \geq \eta\} \leq \eta^{-\alpha} E(|X_{s_n}-X_t|^\alpha) \leq \eta^{-\alpha} \cdot c |s_n-t|^{1+\beta},
\]
and because this is true for each $\eta>0$,
this shows that $X_{s_n}$ converges in measure to $X_t$. Hence, 
the limits $Y_t$ and $X_t$ are equal as  equivalence classes of
measurable functions $\Omega \to \mathbb{R}^d$.\footnote{\url{http://individual.utoronto.ca/jordanbell/notes/L0.pdf}}
That is, $P\{Y_t = X_t\}=1$.  This is true for each $t \in \mathbb{R}_{\geq 0}$, showing that $Y$ is a modification of $X$, completing the proof.
\end{proof}

\section{H\"older continuity}
Let $(X,d)$ and $(Y,\rho)$ be metric spaces, let $0<\gamma<1$, and let
$\phi:X \to Y$ be a function. For 
$x_0 \in X$, we say that $\phi$ is \textbf{$\gamma$-H\"older continuous at $x_0$}
if there is some $0<\epsilon_{x_0}<1$ and some $C_{x_0}$ such that when
$d(x,x_0)< \epsilon_{x_0}$,
\[
\rho(\phi(x),\phi(x_0)) \leq C_{x_0} d(x,x_0)^\gamma.
\]
We say that $\phi$ is \textbf{locally $\gamma$-H\"older continuous} if for each
$x_0 \in X$ there is some $0<\epsilon_{x_0}<1$ and some
$C_{x_0}$ such that when $d(x,x_0)< \epsilon_{x_0}$ and $d(y,x_0)< \epsilon_{x_0}$,
\[
\rho(\phi(x),\phi(y)) \leq C_{x_0} d(x,y)^\gamma.
\]
We say that $\phi$ is \textbf{uniformly $\gamma$-H\"older continuous} if there is some
$C$ such that for all $x,y \in X$,
\[
\rho(\phi(x),\phi(y)) \leq C d(x,y)^\gamma.
\]

We establish properties of H\"older continuous functions in the following.\footnote{Achim Klenke, {\em Probability Theory: A Comprehensive Course}, p.~448, Lemma 21.3.}

\begin{lemma}
Let $V$ be a nonempty subset of $\mathbb{R}_{\geq 0}$, let $0<\gamma<1$, and let
$f:V \to \mathbb{R}^d$ be locally $\gamma$-H\"older continuous. 
\begin{enumerate}
\item If $0<\gamma'<\gamma$ then $f$ is locally $\gamma'$-H\"older continuous.
\item If $V$ is compact, then $f$ is uniformly $\gamma$-H\"older continuous.
\item If $V$ is an interval of length $T>0$ and there is some $\epsilon>0$ and some $C$ such that for all
$s,t \in V$ with $|t-s| \leq \epsilon$ we have
\begin{equation}
|f(t)-f(s)| \leq C |t-s|^\gamma,
\label{epsilon}
\end{equation}
then
\[
|f(t)-f(s)| \leq C \left\lceil \frac{T}{\epsilon} \right\rceil^{1-\gamma}  |t-s|^\gamma, \qquad s,t \in V.
\]
\end{enumerate}
\end{lemma}
\begin{proof}
For $t_0 \in \mathbb{R}_{\geq 0}$, there is some $0<\epsilon_{t_0}<1$ and some $C_{t_0}$ such that when
$|t-t_0|< \epsilon_{t_0}$,
\[
|f(t)-f(t_0)| \leq C_{t_0} |t-t_0|^\gamma \leq C_{t_0} |t-t_0|^{\gamma'},
\]
showing that $f$ is locally $\gamma'$-H\"older continuous.


With the metric inherited from $\mathbb{R}_{\geq 0}$, $V$ is a compact  metric space. 
For $t \in V$ and $\epsilon>0$, write
\[
B_\epsilon(t) = \{v \in V: |v-t| < \epsilon\},
\]
which is an open subset of $V$.
Because $f$ is locally $\gamma$-H\"older continuous, for each $t \in V$ there is some $0<\epsilon_t<1$ and some
$C_t$ such that for all $u,v \in B_{\epsilon_t}(t)$,
\begin{equation}
|f(u)-f(v)| \leq C_t |u-v|^\gamma.
\label{Bepsilon}
\end{equation}
Write $U_t=B_{\epsilon_t}(t)$. Because $t \in U_t$, 
$\{U_t: t \in V\}$ is an open cover of $V$, and because $V$ is compact there are $t_1,\ldots,t_n \in V$ such that
$\mathfrak{U}=\{U_{t_1},\ldots,U_{t_n}\}$ is an open cover of $V$.
Because $V$ is a compact metric space, there is a \textbf{Lebesgue number} $\delta>0$ of the open cover $\mathfrak{U}$:\footnote{Charalambos D. Aliprantis
and Kim C. Border, {\em Infinite Dimensional Analysis: A Hitchhiker's Guide}, third ed., p.~85, Lemma 3.27.}
for each $t \in V$, there is some $1 \leq i \leq n$ such that $B_\delta(t) \subset U_{t_i}$. 
Let 
\[
C = \max\{C_{t_1},\ldots,C_{t_n},2\norm{f}_u \delta^{-\gamma}\},
\]
 For $s,t \in V$ with $|t-s| < \delta$, i.e.
  $s \in B_\delta(t)$, there is some $1 \leq i \leq n$ with
 $s,t \in U_{t_i}$. By \eqref{Bepsilon},
 \[
 |f(s)-f(t)| \leq C_{t_i} |s-t|^\gamma \leq C|s-t|^\gamma.
 \]
 On the other hand, for $s,t \in V$ with $|t-s| \geq \delta$, 
 \[
 |f(s)-f(t)| \leq 2 \norm{f}_u  \leq 2 \norm{f}_u \left(\frac{|s-t|}{\delta} \right)^\gamma 
 =2\norm{f}_u \delta^{-\gamma} |s-t|^\gamma
 \leq C|s-t|^\gamma.
 \]
 Thus, for all $s,t \in V$, 
 \[
 |f(s)-f(t)| \leq C |s-t|^\gamma,
 \]
 showing that $f$ is uniformly $\gamma$-H\"older continuous.
 
 Let $n = \left\lceil \frac{T}{\epsilon} \right\rceil$. For $s,t \in V$, because $V$ is an interval of length $T$, 
 $|s-t| \leq T \leq \epsilon n$, and then applying \eqref{epsilon}, because $\frac{|t-s|}{n} \leq \epsilon$,
\begin{align*}
|f(t)-f(s)|&= \left| \sum_{k=1}^n f\left(s+(t-s)\frac{k}{n}\right)-f\left(s+(t-s)\frac{k-1}{n}\right)\right|\\
&\leq \sum_{k=1}^n \left| f\left(s+(t-s)\frac{k}{n}\right)-f\left(s+(t-s)\frac{k-1}{n}\right)\right|\\
&\leq \sum_{k=1}^n C\left|\frac{t-s}{n}\right|^\gamma\\
&=C n^{1-\gamma} |t-s|^\gamma.
\end{align*}
\end{proof}


The following theorem does not speak about a version of a stochastic process. Rather, it shows what can be said about a stochastic
process that satisfies \eqref{kolmogorov} when almost all of its sample paths are continuous.\footnote{Heinz Bauer, {\em Probability Theory}, p.~338, Theorem 39.4.}

\begin{theorem}
If a stochastic process $(X_t)_{t \in \mathbb{R}_{\geq 0}}$ with state space $\mathbb{R}^d$ satisfies \eqref{kolmogorov} and for almost every
$\omega \in \Omega$ the map $t \mapsto X_t(\omega)$ is continuous $\mathbb{R}_{\geq 0} \to \mathbb{R}^d$, then
for almost every $\omega \in \Omega$, 
for  every $0<\gamma<\frac{\beta}{\alpha}$,
the map $t \mapsto X_t(\omega)$ is  locally 
$\gamma$-H\"older continuous.
\label{holder}
\end{theorem}
\begin{proof}
There is a $P$-null set $N \in \mathscr{F}$ such that for $\omega \in \Omega \setminus N$, the map
$t \mapsto X_t(\omega)$ is continuous $\mathbb{R}_{\geq 0} \to \mathbb{R}^d$. 
For each $0<\gamma<\frac{\beta}{\alpha}$, we have established in \eqref{399} that there is a $P$-null set
$N_\gamma \in \mathscr{F}$ such that for $k \geq 1$ 
there is some $m_k(\omega)$ such that when 
 $s,t \in D \cap [0,k+1]$ and $|t-s| \leq 2^{-m_k(\omega)}$,
 \begin{equation}
 |X_t(\omega)-X_s(\omega)| \leq C_\gamma |t-s|^\gamma,
 \label{3913}
 \end{equation}
 where $C_\gamma = 2^{\gamma+1}+\frac{4}{1-2^{-\gamma}}$. 
 Write  $\delta(k,\omega) = 2^{-m_k(\omega)}$, and
 let $M_\gamma = N_\gamma \cup N$.
 For $\omega \in \Omega \setminus M_\gamma$, the map $t \mapsto 
 X_t(\omega)$ is continuous $\mathbb{R}_{\geq 0} \to \mathbb{R}^d$. 
 For $k \geq 1$ and for $s,t \in [0,k+1]$ satisfying $|s-t| \leq \delta(k,\omega)$, say with $s \leq t$,
let $m=\frac{t-s}{2}$ and let
$s\leq s_n \leq t$ be a sequence of dyadic rationals decreasing to $s$ and let
$s \leq t_n \leq t$ be a sequence of dyadic rationals inceasing to $t$. 
Then $s_n,t_n \in D \cap [0,k+1]$
and $|s_n-t_n| \leq |s-t| \leq \delta(k,\omega)$,  
 so  by \eqref{3913}, 
 \[
 |X_{t_n}(\omega)-X_{s_n}(\omega)| \leq C_\gamma |t_n-s_n|^\gamma.
 \]
Because $\omega \in \Omega \setminus N$, $X_{t_n}(\omega) \to X_t(\omega)$ and
$X_{s_n}(\omega) \to X_s(\omega)$, so
\begin{align*}
|X_t(\omega)-X_s(\omega)|& \leq |X_t(\omega)-X_{t_n}(\omega)|
+|X_{t_n}(\omega)-X_{s_n}(\omega)|+|X_s(\omega)-X_{s_n}(\omega)|\\
&\leq |X_t(\omega)-X_{t_n}(\omega)|+C_\gamma |t_n-s_n|^\gamma+|X_s(\omega)-X_{s_n}(\omega)|\\
&\downarrow C_\gamma |t-s|^\gamma,
\end{align*}
thus
\[
|X_t(\omega)-X_s(\omega)| \leq C_\gamma |t-s|^\gamma,
\]
showing that for  $0<\gamma<\frac{\beta}{\alpha}$ and 
$\omega \in \Omega \setminus M_\gamma$, the map $t \mapsto X_t(\omega)$ is locally
$\gamma$-H\"older continuous. 

Let $0<\gamma_n<\frac{\beta}{\alpha}$ be a sequence increasing to $\frac{\beta}{\alpha}$ and let 
\[
M = \bigcup_{n \geq 1} M_{\gamma_n},
\]
which is a $P$-null set. 
Let $0<\gamma<\frac{\beta}{\alpha}$ and let $n$ be such that $\gamma_n \geq \gamma$.
For $\omega \in \Omega \setminus M$, the map
$t \mapsto X_t(\omega)$ is locally $\gamma_n$-H\"older continuous, and because
$\gamma \leq \gamma_n$ this implies that the map is locally $\gamma$-H\"older continuous, completing the proof.
\end{proof}




Bauer attributes the following theorem to Kolgmorov and Chentsov.\footnote{Nikolai Nikolaevich Chentsov,
1930--1993,
 obituary in Russian Math. Surveys \textbf{48} (1993), no. 2, 161--166.}
 It does not merely state that for any
$0<\gamma<\frac{\beta}{\alpha}$ there is a modification that is locally $\gamma$-H\"older continuous, but that
there is a modification that for all $0<\gamma<\frac{\beta}{\alpha}$ is locally
$\gamma$-H\"older continuous.\footnote{Heinz Bauer, {\em Probability Theory},
p.~339, Corollary 39.5.}

\begin{theorem}[Kolmogorov-Chentsov theorem]
If a stochastic process $(X_t)_{t \in \mathbb{R}_{\geq 0}}$ with state space $\mathbb{R}^d$ satisfies \eqref{kolmogorov},
then $X$ has a modification $Y$ such that for all $\omega \in \Omega$ and $0<\gamma<\frac{\beta}{\alpha}$,
the path $t \mapsto Y_t(\omega)$ is locally $\gamma$-H\"older continuous.
\end{theorem}
\begin{proof}
Applying the Kolmogorov continuity theorem, there is a continuous modification $Z$ of $X$ that 
also satisfies \eqref{kolmogorov}. 
By Theorem \ref{holder}, there is a $P$-null set $M$ such that for 
$\omega \in \Omega \setminus M$ and $0<\gamma<\frac{\beta}{\alpha}$,
the map
$t \mapsto Z_t(\omega)$ is locally $\gamma$-H\"older continuous.
For $t \in \mathbb{R}_{\geq 0}$, define
\[
Y_t(\omega) = \begin{cases}
Z_t(\omega)&\omega \in \Omega \setminus M\\
0&\omega \in M,
\end{cases}
\]
i.e. $Y_t = 1_{\Omega \setminus M} Z_t$, which is measurable $\mathscr{F} \to \mathscr{B}_{\mathbb{R}^d}$, and so
$(Y_t)_{t \in \mathbb{R}_{\geq 0}}$ is a stochastic process. For every $\omega \in \Omega$ and $0<\gamma<\frac{\beta}{\alpha}$, the map
$t \mapsto Y_t(\omega)$ is locally $\gamma$-H\"older continuous. 
For $t \in \mathbb{R}_{\geq 0}$,
\[
\{X_t \neq Y_t\} = \{X_t \neq Y_t, X_t = Z_t\} \cup \{X_t \neq Y_t, X_t \neq Z_t\}
\subset \{Y_t \neq Z_t\} \cup \{X_t \neq Z_t\}.
\]
Because $P(Y_t \neq Z_t)=P(M)=0$ and $P(X_t \neq Z_t)=0$, since $Z$ is a modification of $X$, we get
$P(X_t \neq Y_t)=0$, namely, $Y$ is a modification of $X$.
\end{proof}

\end{document}