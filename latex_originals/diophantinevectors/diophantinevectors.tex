\documentclass{article}
\usepackage{amsmath,amssymb,mathrsfs,amsthm}
%\usepackage{tikz-cd}
%\usepackage{hyperref}
\newcommand{\inner}[2]{\left\langle #1, #2 \right\rangle}
\newcommand{\tr}{\ensuremath\mathrm{tr}\,} 
\newcommand{\Span}{\ensuremath\mathrm{span}} 
\def\Re{\ensuremath{\mathrm{Re}}\,}
\def\Im{\ensuremath{\mathrm{Im}}\,}
\newcommand{\id}{\ensuremath\mathrm{id}} 
\newcommand{\var}{\ensuremath\mathrm{var}} 
\newcommand{\Lip}{\ensuremath\mathrm{Lip}} 
\newcommand{\GL}{\ensuremath\mathrm{GL}} 
\newcommand{\diam}{\ensuremath\mathrm{diam}} 
\newcommand{\sgn}{\ensuremath\mathrm{sgn}\,} 
\newcommand{\lcm}{\ensuremath\mathrm{lcm}} 
\newcommand{\supp}{\ensuremath\mathrm{supp}\,}
\newcommand{\dom}{\ensuremath\mathrm{dom}\,}
\newcommand{\upto}{\nearrow}
\newcommand{\downto}{\searrow}
\newcommand{\norm}[1]{\left\Vert #1 \right\Vert}
\newtheorem{theorem}{Theorem}
\newtheorem{lemma}[theorem]{Lemma}
\newtheorem{proposition}[theorem]{Proposition}
\newtheorem{corollary}[theorem]{Corollary}
\theoremstyle{definition}
\newtheorem{definition}[theorem]{Definition}
\newtheorem{example}[theorem]{Example}
\begin{document}
\title{Diophantine vectors}
\author{Jordan Bell\\ \texttt{jordan.bell@gmail.com}\\Department of Mathematics, University of Toronto}
\date{\today}

\maketitle


\section{Dirichlet's approximation theorem}
Let $m \geq 1$, and
for $v \in \mathbb{R}^m$ write
\[
|v|_\infty = \max \{|v_j|:1 \leq j \leq m\}.
\]
For a positive integer $r$, let
\[
V_r = \{k \in \mathbb{Z}^m: 0 < |k|_\infty \leq r\},
\] 
which has $N_r=(2r+1)^m-1$ elements. 
For any $k \in V_r$,
\[
|\inner{v}{k}| \leq m |k|_\infty |v|_\infty \leq m r |v|_\infty,
\]
Let $I_1,\ldots,I_{N_r-1}$ be consecutive closed intervals with
\[
[0,mr|v|_\infty] = \bigcup_{j=1}^{N_r-1} I_j.
\]
Then there is some $j$ and some $k',k'' \in V_r$, $k' \neq k''$, with $|\inner{v}{k'}|,|\inner{v}{k''}| \in I_j$.
If $\inner{v}{k'}, \inner{v}{k''}$ have the same sign, then $k=k'-k''$ satisfies
$|\inner{v}{k}| \leq |I_j|$, and if
$\inner{v}{k'},\inner{v}{k''}$ have different signs then
$k=k'+k''$ satisfies $|\inner{v}{k}| \leq |I_j|$.
In either case, $k \in V_{2r}$, and $k$ satisfies
\[
|\inner{v}{k}| \leq |I_j| =  \frac{mr|v|_\infty}{N_r-1} = \frac{mr|v|_\infty}{(2r+1)^m-2}.
\]




\section{Diophantine vectors}
For real $\tau,\gamma>0$,
let
$D(\tau,\gamma)$ be the set of those $v \in \mathbb{R}^m$ such that for any nonzero $k \in \mathbb{Z}^m$,
\[
|\inner{v}{k}| \geq \gamma |k|_\infty^{-\tau}.
\]
In other words, 
\[
D(\tau,\gamma) =  \bigcap_{k \in \mathbb{Z}^m \setminus \{0\}} \{v \in \mathbb{R}^m: |\inner{v}{k}| \geq \gamma |k|_\infty^{-\tau}\}
= \bigcap_{k \in \mathbb{Z}^m \setminus \{0\}} D(\tau,\gamma,k).
\]
Each $D(\tau,\gamma,k)$ is closed, so $D(\tau,\gamma)$ is closed. 
Let 
\[
D(\tau) = \bigcup_{\gamma>0} D(\tau,\gamma).
\]
If $\gamma_1 \geq \gamma_2$ and $v \in D(\tau,\gamma_1)$, let $k \in \mathbb{Z}^m \setminus \{0\}$. Then
$|\inner{v}{k}| \geq \gamma_1 |k|_\infty^{-\tau} \geq \gamma_2 |k|_\infty^{-\tau}$, so
$v \in D(\tau,\gamma_2)$, i.e. 
\[
D(\tau,\gamma_1) \subset D(\tau,\gamma_2),\qquad \gamma_1 \geq \gamma_2.
\] 
Therefore
\[
D(\tau,N_1^{-1}) \subset D(\tau,N_2^{-1})\qquad N_1 \leq N_2,
\]
and
\[
D(\tau) = \bigcup_{N \geq 1} D(\tau,N^{-1}),
\]
showing that $D(\tau)$ is an $F_\sigma$ set.




If $0 \leq \tau<m-1$ and $\gamma>0$,
suppose by contradiction that there is some $v \in D(\tau,\gamma)$.
Now, by Dirichlet's theorem, for each positive integer $r$ there is some
$k_r \in V_{2r}$ satisfying $|\inner{v}{k_r}| \leq m |v|_\infty 2^{-m} r^{-m+1}$. 
Then, as $|k_r|_\infty \leq 2r$,
\[
m |v|_\infty 2^{-m} r^{-m+1} \geq |\inner{v}{k_r}| \geq  \gamma |k_r|_\infty^{-\tau}
\geq\gamma (2r)^{-\tau}
=\gamma (2r)^{\tau-m+1} (2r)^{m-1},
\]
hence
\[
(2r)^{-\tau+m-1} \geq \frac{2}{cm|v|_\infty}.
\]
As $\tau<m-1$, taking $r \to \infty$ yields a contradiction. Therefore
\[
D(\tau) = \emptyset, \qquad 0 \leq \tau<m-1.
\]



\section{Measures of sets}
Denote by $\mu$ Lebesgue measure on $\mathbb{R}^m$.
Let $e_1,\ldots,e_m$ be the standard basis for $\mathbb{R}^m$, so
\[
|v|_1 = \sum_{j=1}^m |v_j| = \sum_{j=1}^m |\inner{v}{e_j}|.
\]
Let $C=\{v \in \mathbb{R}^m: |v|_\infty \leq 1\}$. 
Let $A_m$ be the supremum of the  $(m-1)$-dimensional Hausdorff measure of
the intersection of an $(n-1)$-dimensional affine subspace of $\mathbb{R}^m$ and $C$.

We calculate the following.\footnote{Dmitry Treschev and Oleg Zubelevich, {\em Introduction
to the Perturbation Theory of Hamiltonian Systems},
p.~166, Theorem 9.3.}

\begin{theorem}
For $\tau>m-1$ and $\gamma>0$,
\[
\mu(C \setminus D(\tau,\gamma)) \leq 4\gamma m A_m  3^{m-1} \zeta(\tau+2-m).
\]
\label{Dtaugamma}
\end{theorem}
\begin{proof}
Let $k \in \mathbb{Z}^m \setminus \{0\}$, and
for $t \in \mathbb{R}$, let 
\[
P_{k,t} = \{x \in \mathbb{R}^m: \inner{x}{k} = t\},
\]
and let
\[
U_k = \left\{ x \in \mathbb{R}^m : |\inner{x}{k}| < \gamma |k|_\infty^{-\tau}  \right\}.
\]
$U$ is the set of points between the hyperplanes $P_{k,- \gamma |k|_\infty^{-\tau} }$ and $P_{k,\gamma |k|_\infty^{-\tau} }$.
The distance between the hyperplanes $P_{k,s}$ and $P_{k,s}$ is $\frac{|s-t|}{|k|_2}$, so
the distance between the hyperplanes $P_{k,- \gamma |k|_\infty^{-\tau} }$ and $P_{ k,\gamma |k|_\infty^{-\tau} }$ is
$d_k=\frac{2\gamma |k|_\infty^{-\tau}}{|k|_2}$. And $|x|_2 \geq |x|_\infty$, so
$d_k \leq 2\gamma |k|_\infty^{-\tau-1}$. 
But $\mu(C \cap U_k) \leq d A_m$, so
\[
\mu(C \cap U) \leq 2\gamma |k|_\infty^{-\tau-1} A_m.
\]
Now, $U_k = \mathbb{R}^m \setminus D(\tau,\gamma,k)$, so 
\[
C \setminus D(\tau,\gamma) = C \setminus \bigcap_{k \in \mathbb{Z}^m \setminus \{0\}} D(\tau,\gamma,k)
=\bigcup_{k \in \mathbb{Z}^m \setminus \{0\}} (C \cap U_k).
\]
We remind ourselves that for $r$ a positive integer,
the set $V_r = \{k \in \mathbb{Z}^m: 0 < |k|_\infty \leq r\}$ has 
$N_r=(2r+1)^m-1$ elements. 
Therefore 
\[
\{k \in \mathbb{Z}^m \setminus \{0\}: |k|_\infty = r\} = V_r \setminus V_{r-1}
\]
has
\[
N_r-N_{r-1}=(2r+1)^m - (2r-1)^m \leq 2m (2r+3)^{m-1}
\]
elements,
using $a^m-b^m=(a-b)(a^{m-1}+a^{m-2}b + \cdots + ab^{m-2}+b^{m-1})$. Therefore
\begin{align*}
\mu(C \setminus D(\tau,\gamma))&\leq \sum_{k \in  \mathbb{Z}^m \setminus \{0\}} \mu(C \cap U_k)\\
&\leq \sum_{k \in \mathbb{Z}^m \setminus \{0\}} 2\gamma |k|_\infty^{-\tau-1} A_m\\
&=2\gamma A_m \sum_{r=1}^\infty \sum_{\{k \in \mathbb{Z}^m \setminus \{0\}: |k|_\infty = r\}}  r^{-\tau-1}\\
&\leq 2\gamma A_m \sum_{r=1}^\infty 2m (2r+1)^{m-1} \cdot  r^{-\tau-1}\\
&=4\gamma m A_m \sum_{r=1}^\infty (2r+1)^{m-1} r^{-\tau-1}.
\end{align*}
We estimate
\[
\sum_{r=1}^\infty (2r+1)^{m-1} r^{-\tau-1} \leq \sum_{r=1}^\infty (3r)^{m-1} r^{-\tau-1}
=3^{m-1} \sum_{r=1}^\infty r^{m-\tau-2},
\]
and therefore
\[
\mu(C \setminus D(\tau,\gamma)) \leq 4\gamma m A_m \cdot 3^{m-1} \zeta(\tau+2-m).
\]
\end{proof}

But
\[
C \setminus D(\tau) =  \bigcap_{N \geq 1} (C \setminus D(\tau,N^{-1})),
\]
and by Theorem \ref{Dtaugamma}, if $\tau>m-1$ then $\mu(C \setminus D(\tau,N^{-1})) \to 0$ as $N \to \infty$. 
Therefore 
\[
\mu(C \cap D(\tau)) = 0,\qquad \tau>m-1.
\]



\section{Cohomological equation}
Let $\mathbb{T}^m = \{z \in \mathbb{C}^m: |z_1|=1,\ldots,|z_m|=1\}$
and write $\nu$ for the Haar measure on $\mathbb{T}^m$ for which $\nu(\mathbb{T}^m)=1$.
For $k \in \mathbb{Z}^m$ let $\chi_k(z) = \prod_{j=1}^m z_j^{k_j}$. 
Let
$\Delta(\tau,\gamma)$ be the set of those
$z \in \mathbb{T}^m$ such that
\[
|\chi_k(z)-1| \geq \gamma |k|_1^{-\tau},\qquad k \in \mathbb{Z}^m \setminus \{0\}.
\]
Let 
\[
\Delta(\tau) = \bigcup_{\gamma>0} \Delta(\tau,\gamma),
\]
and then
\[
\Delta = \bigcup_{\tau>0} \Delta(\tau).
\]
For $\lambda \in \mathbb{T}^m$ define $R_\lambda:\mathbb{T}^m \to \mathbb{T}^m$ by
\[
R_\lambda(z) = \lambda \cdot z= (\lambda_1 z_1,\ldots,\lambda_m z_m).
\]


In the following theorem, \eqref{cohomology} is called a \textbf{cohomological equation}.\footnote{Anatole Katok, {\em Combinatorial Constructions in Ergodic Theory and Dynamics}, p.~71, Theorem 11.5.}


\begin{theorem}
For $\lambda \in \mathbb{T}^m$,
$\lambda \in \Delta$ if and only if for any
$h \in C^\infty(\mathbb{T}^m)$ there is some $\psi \in C^\infty(\mathbb{T}^m)$ such that
\begin{equation}
h(z) - \int_{\mathbb{T}^m} h d\nu = \psi(R_\lambda z) - \psi(z),\qquad z \in \mathbb{T}^m.
\label{cohomology}
\end{equation}
\end{theorem}
\begin{proof}
It is a fact that $\chi_k \in \widehat{\mathbb{T}}^m$ and that
$k \mapsto \chi_k$ is an isomorphism of topological groups $\mathbb{Z}^d \to \widehat{\mathbb{T}}^m$. 
For $f \in L^1(\nu)$,
$\widehat{f}:\mathbb{Z}^m \to \mathbb{C}$ is defined by
\[
\widehat{f}(k) = \int_{\mathbb{T}^m} f(z) \overline{\chi_k(z)} d\nu(z)
= \int_{\mathbb{T}^m} f(z) \chi_k(z)^{-1} d\nu(z).
\] 
If the Fourier series of $f$ converges pointwise,
\[
f(z) = \sum_{k \in \mathbb{Z}^m} \widehat{f}(k) \chi_k(z),\qquad z \in \mathbb{T}^m.
\]
It is a fact that $f \in C^\infty(\mathbb{T}^m)$ if and only if
for any $R>0$ there is some $C_R$ such that 
\[
|\widehat{f}(k)| \leq C_R |k|_1^{-R},\qquad k \in \mathbb{Z}^m \setminus \{0\}.
\]
For $\psi \in L^1(\mathbb{T}^m)$ and $k \in \mathbb{T}^m$,
because $\nu$ is invariant under multiplication in $\mathbb{T}^m$,
\begin{align*}
\widehat{\psi \circ R_\lambda}(k)
&=\int_{\mathbb{T}^m} \psi(\lambda z)  \overline{\chi_k(z)} d\nu(z)\\
&=\int_{\mathbb{T}^m} \psi(z) \overline{\chi_k(\lambda^{-1}x)} d\nu(z)\\
&=\chi_k(\lambda) \widehat{\psi}(k).
\end{align*}


Suppose that for every $h$ is $C^\infty$ that there is some $\psi \in C^\infty(\mathbb{T}^m)$ satisfying \eqref{cohomology}.
Taking the Fourier transform of \eqref{cohomology},
\[
\widehat{h}(k) - \delta_0(k) \cdot \int_{\mathbb{T}^m} h d\nu = \chi_k(\lambda) \widehat{\psi}(k) - \widehat{\psi}(k),\qquad
k \in \mathbb{Z}^m,
\]
then, if $\chi_k(\lambda) \neq 1$,
\[
\widehat{\psi}(k) = \frac{\widehat{h}(k)}{\chi_k(\lambda)-1}.
\]
Now suppose by contradiction that
$\lambda \not \in \Delta$. 
This means that there are $\tau_N \to \infty$ such that for each $N$, there is some
$\gamma_N>0$ and some $k_N \in \mathbb{Z}^m \setminus \{0\}$ such that
$|\chi_{k_N}(\lambda)-1| < \gamma_N |k_N|_1^{-\tau_N}$. 
Define 
\[
\widehat{h}(k) = |\chi_k(\lambda)-1|^{1/2} \cdot 1_{\{k_N\}}(k).
\]
For $R>0$ let $\tau_N \geq 2R$. Then for $k \in \mathbb{Z}^m$, 
 either $\widehat{h}(k)=0$ or if $k=k_N$ then
\[
|\widehat{h}(k)| = |\chi_{k_N}(\lambda)-1|^{1/2} <\gamma_N^{1/2} |k_N|_1^{-\tau_N/2}
\leq \gamma_N^{1/2} |k_N|_1^{-R}
=\gamma_N^{1/2} |k|_1^{-R},
\]
which shows that $h$ is $C^\infty$. 
There is some $\psi \in C^\infty(\mathbb{T}^m)$ satisfying \eqref{cohomology}, according
to which, for $\chi_k(\lambda) \neq 1$,
\[
\widehat{\psi}(k) = \frac{\widehat{h}(k)}{\chi_k(\lambda)-1}.
\]
But  $|\widehat{\psi}(k)|$ is either $0$ or if $k=k_N$ then 
$|\chi_{k_N}(\lambda)-1|^{-1/2} > \gamma_N^{-1/2} |k_N|_1^{\tau_N/2}$. 
Thus the Fourier coefficients of $\psi$ are unbounded, which contradicts that
$\psi$ is $C^\infty$. Therefore $\lambda \in \Delta$. 

Now suppose that $\lambda \in D$ and let $h \in C^\infty(\mathbb{T}^m)$. 
Define $\psi$ by
\[
\widehat{\psi}(k) =\begin{cases}
\frac{\widehat{h}(k)}{\chi_k(\lambda)-1}&\chi_k(\lambda) \neq 1\\
0&\chi_k(\lambda)=1.
\end{cases}
\]
The facts that $\lambda \in D$ and that $h$ is $C^\infty$ yield that $\psi$ is $C^\infty$. It is straightforward
from the definition of $\widehat{\psi}(k)$ that $\psi$ satisfies \eqref{cohomology}.
\end{proof}




\end{document}